\begin{figure}
\centering
\begin{tikzpicture}[scale=1.5]
\def \tic {0.05}
\def \zre{0} %real of zero
\def \zim{0} %imag of zero
\def \pre{0.8} %real of pole
\def \pim{0} %imag of pole
\def \pabs{\pre} % largest radius for poles to determine roc
\def \rocmax{1.6} % sketch of outer roc domain
%
\begin{scope}
% basic diagram features:
\filldraw[even odd rule,C2!50] (0,0) circle(\pabs) decorate
[decoration={snake, segment length=15pt, amplitude=1pt}]
{(0,-3pt) circle(\rocmax)}; % sketch the roc domain
%
\draw[help lines, C7!50, step=0.25cm] (-\rocmax,-\rocmax) grid (\rocmax,\rocmax);
%
\draw[C3, thick] (0,0) circle(1);  % unit circle, i.e. DTFT domain
%\draw (-2*\tic,1+2*\tic) node{$1$};
\draw (1+2*\tic,-2*\tic) node{$1$}; % indicate that this is the unit circle
\draw[->] (-1.75,0)--(1.75,0) node[right]{$\Re\{z\}$}; % axis label
\draw[->] (0,-1.75)--(0,1.75) node[above]{$\Im\{z\}$}; % axis label
%
\draw (1.2*0.86602540378*\rocmax,1.2*0.5*\rocmax) node[C2!75]{KB}; % indicate the roc
%
% the z-transfer function specific stuff:
\draw (1,1) node[black]{$g=+1$}; % indicate gain factor
%
% draw the poles / zeros and if desired ticks:
\draw[C0, ultra thick] (\pre,+\pim) node{\Huge $\times$};
\draw[C0, ultra thick] (\pre,-\pim) node{\Huge $\times$};
\draw[C0, ultra thick] (\zre,+\zim) node{\Huge $\circ$};
\draw[C0, ultra thick] (\zre,-\zim) node{\Huge $\circ$};
%
\draw (-\tic,\zim) -- (\tic,\zim) node[right]{$\zim$};
\draw (-\tic,\pim) -- (\tic,\pim) node[right]{$\pim$};
\draw (\zre,\tic) -- (\zre,-\tic) node[below]{$\zre$};
\draw (\pre,\tic) -- (\pre,-\tic) node[below]{$\pre$};
\end{scope}
%
\begin{scope}[scale=0.5, xshift=8cm]
\def\tic{0.1};
\def\Om{360/8}
\draw[help lines, C7!25, step=0.5cm] (-1,-1) grid (8,1);
\draw[->] (-1.5,0) -- (8.5,0) node[right]{$k$};
\draw[->] (0,-1) -- (0,2) node[above]{$x[k]=\pabs^k \epsilon[k] \,\ztransf\, X(z)=\frac{z}{z-\pre}$ für $|z|>\pabs$};
\foreach \x in {-1,0,...,8}{\draw (\x,+\tic) -- (\x,-\tic)  node[left]{$\x$};};
\foreach \y in {-1,1}{\draw (\tic,\y) -- (-\tic,\y)  node[left]{$\y$};};
\foreach \k in {-2,-1}
{
  \draw[stem] plot coordinates
  {
    (\k,{0*\k})
  };
};
\foreach \k in {0,1,...,8}{\draw[stem] plot coordinates{(\k,{\pabs^\k})};};
\end{scope}
%
\end{tikzpicture}
%
%
%
\begin{tikzpicture}[scale=1.5]
\def \tic {0.05}
\def \zre{0} %real of zero
\def \zim{0} %imag of zero
\def \pre{1} %real of pole
\def \pim{0} %imag of pole
\def \pabs{\pre} % largest radius for poles to determine roc
\def \rocmax{1.6} % sketch of outer roc domain
%
\begin{scope}
% basic diagram features:
\filldraw[even odd rule,C2!50] (0,0) circle(\pabs) decorate
[decoration={snake, segment length=15pt, amplitude=1pt}]
{(0,-3pt) circle(\rocmax)}; % sketch the roc domain
%
\draw[help lines, C7!50, step=0.25cm] (-\rocmax,-\rocmax) grid (\rocmax,\rocmax);
%
\draw[C3, thick] (0,0) circle(1);  % unit circle, i.e. DTFT domain
%\draw (-2*\tic,1+2*\tic) node{$1$};
\draw (1+2*\tic,-2*\tic) node{$1$}; % indicate that this is the unit circle
\draw[->] (-1.75,0)--(1.75,0) node[right]{$\Re\{z\}$}; % axis label
\draw[->] (0,-1.75)--(0,1.75) node[above]{$\Im\{z\}$}; % axis label
%
\draw (1.2*0.86602540378*\rocmax,1.2*0.5*\rocmax) node[C2!75]{KB}; % indicate the roc
%
% the z-transfer function specific stuff:
\draw (1,1) node[black]{$g=+1$}; % indicate gain factor
%
% draw the poles / zeros and if desired ticks:
\draw[C0, ultra thick] (\pre,+\pim) node{\Huge $\times$};
\draw[C0, ultra thick] (\pre,-\pim) node{\Huge $\times$};
\draw[C0, ultra thick] (\zre,+\zim) node{\Huge $\circ$};
\draw[C0, ultra thick] (\zre,-\zim) node{\Huge $\circ$};
%
\draw (-\tic,\zim) -- (\tic,\zim) node[right]{$\zim$};
\draw (-\tic,\pim) -- (\tic,\pim) node[right]{$\pim$};
\draw (\zre,\tic) -- (\zre,-\tic) node[below]{$\zre$};
\draw (\pre,\tic) -- (\pre,-\tic) node[below]{$\pre$};
\end{scope}
%
\begin{scope}[scale=0.5, xshift=8cm]
\def\tic{0.1};
\def\Om{360/8}
\draw[help lines, C7!25, step=0.5cm] (-1,-1) grid (8,1);
\draw[->] (-1.5,0) -- (8.5,0) node[right]{$k$};
\draw[->] (0,-1) -- (0,2) node[above]{$x[k]=\pabs^k \epsilon[k] \,\ztransf\, X(z)=\frac{z}{z-\pre}$ für $|z|>\pabs$};
\foreach \x in {-1,0,...,8}{\draw (\x,+\tic) -- (\x,-\tic)  node[left]{$\x$};};
\foreach \y in {-1,1}{\draw (\tic,\y) -- (-\tic,\y)  node[left]{$\y$};};
\foreach \k in {-2,-1}
{
  \draw[stem] plot coordinates
  {
    (\k,{0*\k})
  };
};
\foreach \k in {0,1,...,8}{\draw[stem] plot coordinates{(\k,{\pabs^\k})};};
\end{scope}
%
\end{tikzpicture}
%
%
%
\begin{tikzpicture}[scale=1.5]
\def \tic {0.05}
\def \zre{0} %real of zero
\def \zim{0} %imag of zero
\def \pre{1.2} %real of pole
\def \pim{0} %imag of pole
\def \pabs{\pre} % largest radius for poles to determine roc
\def \rocmax{1.6} % sketch of outer roc domain
%
\begin{scope}
% basic diagram features:
\filldraw[even odd rule,C2!50] (0,0) circle(\pabs) decorate
[decoration={snake, segment length=15pt, amplitude=1pt}]
{(0,-3pt) circle(\rocmax)}; % sketch the roc domain
%
\draw[help lines, C7!50, step=0.25cm] (-\rocmax,-\rocmax) grid (\rocmax,\rocmax);
%
\draw[C3, thick] (0,0) circle(1);  % unit circle, i.e. DTFT domain
%\draw (-2*\tic,1+2*\tic) node{$1$};
\draw (1+2*\tic,-2*\tic) node{$1$}; % indicate that this is the unit circle
\draw[->] (-1.75,0)--(1.75,0) node[right]{$\Re\{z\}$}; % axis label
\draw[->] (0,-1.75)--(0,1.75) node[above]{$\Im\{z\}$}; % axis label
%
\draw (1.2*0.86602540378*\rocmax,1.2*0.5*\rocmax) node[C2!75]{KB}; % indicate the roc
%
% the z-transfer function specific stuff:
\draw (1,1) node[black]{$g=+1$}; % indicate gain factor
%
% draw the poles / zeros and if desired ticks:
\draw[C0, ultra thick] (\pre,+\pim) node{\Huge $\times$};
\draw[C0, ultra thick] (\pre,-\pim) node{\Huge $\times$};
\draw[C0, ultra thick] (\zre,+\zim) node{\Huge $\circ$};
\draw[C0, ultra thick] (\zre,-\zim) node{\Huge $\circ$};
%
\draw (-\tic,\zim) -- (\tic,\zim) node[right]{$\zim$};
\draw (-\tic,\pim) -- (\tic,\pim) node[right]{$\pim$};
\draw (\zre,\tic) -- (\zre,-\tic) node[below]{$\zre$};
\draw (\pre,\tic) -- (\pre,-\tic) node[below]{$\pre$};
\end{scope}
%
\begin{scope}[scale=0.5, xshift=8cm]
\def\tic{0.1};
\def\Om{360/8}
\draw[help lines, C7!25, step=0.5cm] (-1,-1) grid (8,1);
\draw[->] (-1.5,0) -- (8.5,0) node[right]{$k$};
\draw[->] (0,-1) -- (0,2) node[above]{$x[k]=\pabs^k \epsilon[k] \,\ztransf\, X(z)=\frac{z}{z-\pre}$ für $|z|>\pabs$};
\foreach \x in {-1,0,...,8}{\draw (\x,+\tic) -- (\x,-\tic)  node[left]{$\x$};};
\foreach \y in {-1,1}{\draw (\tic,\y) -- (-\tic,\y)  node[left]{$\y$};};
\foreach \k in {-2,-1}
{
  \draw[stem] plot coordinates
  {
    (\k,{0*\k})
  };
};
\foreach \k in {0,1,...,8}{\draw[stem] plot coordinates{(\k,{\pabs^\k})};};
\end{scope}
%
\end{tikzpicture}
%
\caption{\textbf{Kausales} 1-Pol Signal mit fester Nullstelle in $z=0$.
Links $z$-Ebene, rechts: zugehörige \textbf{rechtseitige} Folge
$z_0^k \cdot \epsilon[k]$ für $z_0 = \e^{\Sigma_0+\im\Omega_0}$ mit $\Omega_0=0$
und Variation $\Sigma_0$. Die Folge ganz unten ist nicht beschränkt.
Vgl. \fig{fig:0B03A693AD_rightsided} (3.2) für Laplace Bereich.}
\label{fig:single_pole_rightsided_tikz}
\end{figure}

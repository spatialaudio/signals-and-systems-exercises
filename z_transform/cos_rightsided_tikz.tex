\begin{figure}
\centering
\begin{tikzpicture}[scale=1.5]
\def \tic {0.05}
\def \zre{0} %real of zero
\def \zim{0} %imag of zero
\def \pre{0.7071} %real of pole
\def \pim{0.7071} %imag of pole
\def \pabs{1} % largest radius for poles to determine roc
\def \rocmax{1.6} % sketch of outer roc domain
%
\begin{scope}
% basic diagram features:
\filldraw[even odd rule,C2!50] (0,0) circle(\pabs) decorate
[decoration={snake, segment length=15pt, amplitude=1pt}]
{(0,-3pt) circle(\rocmax)}; % sketch the roc domain
%
\draw[help lines, C7!50, step=0.25cm] (-\rocmax,-\rocmax) grid (\rocmax,\rocmax);
%
\draw[C3, thick] (0,0) circle(1);  % unit circle, i.e. DTFT domain
%\draw (-2*\tic,1+2*\tic) node{$1$};
\draw (1+2*\tic,-2*\tic) node{$1$}; % indicate that this is the unit circle
\draw[->] (-1.75,0)--(1.75,0) node[right]{$\Re\{z\}$}; % axis label
\draw[->] (0,-1.75)--(0,1.75) node[above]{$\Im\{z\}$}; % axis label
%
\draw (1.2*0.86602540378*\rocmax,1.2*0.5*\rocmax) node[C2!75]{KB}; % indicate the roc
%
% the z-transfer function specific stuff:
\draw (1,1) node[black]{$g=+1$}; % indicate gain factor
%
% draw the poles / zeros and if desired ticks:
\draw[C0, ultra thick] (\pre,+\pim) node{\Huge $\times$};
%\draw[C0, ultra thick] (\pre,-\pim) node{\Huge $\times$};
\draw[C0, ultra thick] (\zre,+\zim) node{\Huge $\circ$};
\draw[C0, ultra thick] (\zre,-\zim) node{\Huge $\circ$};
%
%\draw (-\tic,\zim) -- (\tic,\zim) node[right]{$\zim$};
%\draw (-\tic,\pim) -- (\tic,\pim) node[right]{$\pim$};
%\draw (\zre,\tic) -- (\zre,-\tic) node[below]{$\zre$};
%\draw (\pre,\tic) -- (\pre,-\tic) node[below]{$\pre$};
\draw[dashed, C7] (0,0) --(1,+1);
\draw[dashed, C7] (0,0) --(1,-1);
\end{scope}
%
\begin{scope}[scale=0.5, xshift=9cm]
\def\tic{0.1};
\def\Om{360/8}
\draw[help lines, C7!25, step=0.5cm] (-1,-1) grid (8,1);
\draw[->] (-1.5,0) -- (8.5,0) node[right]{$k$};
\draw[->] (0,-1) -- (0,2.5) node[above]{$x_+[k]=a_+^k \epsilon[k] \,\ztransf\, X_+(z)=\frac{z}{z-a_+}$ für $|z|>1$, $a_+=\e^{+\im\frac{\pi}{4}}$};
\foreach \x in {-1,0,...,8}{\draw (\x,+\tic) -- (\x,-\tic)  node[left]{$\x$};};
\foreach \y in {-1,1}{\draw (\tic,\y) -- (-\tic,\y)  node[left]{$\y$};};
\foreach \k in {-2,-1}
{
  \draw[stem] plot coordinates
  {
    (\k,{0*\k})
  };
};
\foreach \k in {0,1,...,8}{\draw[stem] plot coordinates{(\k,{cos(360/8*\k)})};};
\foreach \k in {0,1,...,8}{\draw[stem, C1, ultra thin] plot coordinates{(\k,{sin(360/8*\k)})};};
\draw[C0] (0.75,1.5)node{$\Re\{x[k]\}$};
\draw[C1] (2.75,1.5)node{$\Im\{x[k]\}$};
\end{scope}
%
\end{tikzpicture}
%
%
%
\begin{tikzpicture}[scale=1.5]
\def \tic {0.05}
\def \zre{0} %real of zero
\def \zim{0} %imag of zero
\def \pre{+0.7071} %real of pole
\def \pim{-0.7071} %imag of pole
\def \pabs{1} % largest radius for poles to determine roc
\def \rocmax{1.6} % sketch of outer roc domain
%
\begin{scope}
% basic diagram features:
\filldraw[even odd rule,C2!50] (0,0) circle(\pabs) decorate
[decoration={snake, segment length=15pt, amplitude=1pt}]
{(0,-3pt) circle(\rocmax)}; % sketch the roc domain
%
\draw[help lines, C7!50, step=0.25cm] (-\rocmax,-\rocmax) grid (\rocmax,\rocmax);
%
\draw[C3, thick] (0,0) circle(1);  % unit circle, i.e. DTFT domain
%\draw (-2*\tic,1+2*\tic) node{$1$};
\draw (1+2*\tic,-2*\tic) node{$1$}; % indicate that this is the unit circle
\draw[->] (-1.75,0)--(1.75,0) node[right]{$\Re\{z\}$}; % axis label
\draw[->] (0,-1.75)--(0,1.75) node[above]{$\Im\{z\}$}; % axis label
%
\draw (1.2*0.86602540378*\rocmax,1.2*0.5*\rocmax) node[C2!75]{KB}; % indicate the roc
%
% the z-transfer function specific stuff:
\draw (1,1) node[black]{$g=+1$}; % indicate gain factor
%
% draw the poles / zeros and if desired ticks:
\draw[C0, ultra thick] (\pre,+\pim) node{\Huge $\times$};
%\draw[C0, ultra thick] (\pre,-\pim) node{\Huge $\times$};
\draw[C0, ultra thick] (\zre,+\zim) node{\Huge $\circ$};
\draw[C0, ultra thick] (\zre,-\zim) node{\Huge $\circ$};
%
%\draw (-\tic,\zim) -- (\tic,\zim) node[right]{$\zim$};
%\draw (-\tic,\pim) -- (\tic,\pim) node[right]{$\pim$};
%\draw (\zre,\tic) -- (\zre,-\tic) node[below]{$\zre$};
%\draw (\pre,\tic) -- (\pre,-\tic) node[below]{$\pre$};
\draw[dashed, C7] (0,0) --(1,+1);
\draw[dashed, C7] (0,0) --(1,-1);
\end{scope}
%
\begin{scope}[scale=0.5, xshift=9cm]
\def\tic{0.1};
\def\Om{360/8}
\draw[help lines, C7!25, step=0.5cm] (-1,-1) grid (8,1);
\draw[->] (-1.5,0) -- (8.5,0) node[right]{$k$};
\draw[->] (0,-1) -- (0,2.5) node[above]{$x_-[k]=a_-^k \epsilon[k] \,\ztransf\, X_-(z)=\frac{z}{z-a_-}$ für $|z|>1$, $a_-=\e^{-\im\frac{\pi}{4}}$};
\foreach \x in {-1,0,...,8}{\draw (\x,+\tic) -- (\x,-\tic)  node[left]{$\x$};};
\foreach \y in {-1,1}{\draw (\tic,\y) -- (-\tic,\y)  node[left]{$\y$};};
\foreach \k in {-2,-1}
{
  \draw[stem] plot coordinates
  {
    (\k,{0*\k})
  };
};
\foreach \k in {0,1,...,8}{\draw[stem] plot coordinates{(\k,{cos(360/8*\k)})};};
\foreach \k in {0,1,...,8}{\draw[stem, C1, ultra thin] plot coordinates{(\k,{sin(-360/8*\k)})};};
\draw[C0] (0.75,1.5)node{$\Re\{x[k]\}$};
\draw[C1] (2.75,1.5)node{$\Im\{x[k]\}$};
\end{scope}
%
\end{tikzpicture}
%
%
%
\begin{tikzpicture}[scale=1.5]
\def \tic {0.05}
\def \pre{0.7071} %real of pole
\def \pim{0.7071} %imag of pole
\def \pabs{1} % largest radius for poles to determine roc
\def \rocmax{1.6} % sketch of outer roc domain
%
\begin{scope}
% basic diagram features:
\filldraw[even odd rule,C2!50] (0,0) circle(\pabs) decorate
[decoration={snake, segment length=15pt, amplitude=1pt}]
{(0,-3pt) circle(\rocmax)}; % sketch the roc domain
%
\draw[help lines, C7!50, step=0.25cm] (-\rocmax,-\rocmax) grid (\rocmax,\rocmax);
%
\draw[C3, thick] (0,0) circle(1);  % unit circle, i.e. DTFT domain
%\draw (-2*\tic,1+2*\tic) node{$1$};
\draw (1+2*\tic,-2*\tic) node{$1$}; % indicate that this is the unit circle
\draw[->] (-1.75,0)--(1.75,0) node[right]{$\Re\{z\}$}; % axis label
\draw[->] (0,-1.75)--(0,1.75) node[above]{$\Im\{z\}$}; % axis label
%
\draw (1.2*0.86602540378*\rocmax,1.2*0.5*\rocmax) node[C2!75]{KB}; % indicate the roc
%
% the z-transfer function specific stuff:
\draw (1,1) node[black]{$g=+2$}; % indicate gain factor
%
% draw the poles / zeros and if desired ticks:
\draw[C0, ultra thick] (\pre,+\pim) node{\Huge $\times$};
\draw[C0, ultra thick] (\pre,-\pim) node{\Huge $\times$};
\draw[C0, ultra thick] (0,0) node{\Huge $\circ$};
\draw[C0, ultra thick] (\pre,0) node{\Huge $\circ$};
%
%\draw (-\tic,\zim) -- (\tic,\zim) node[right]{$\zim$};
%\draw (-\tic,\pim) -- (\tic,\pim) node[right]{$\pim$};
%\draw (\zre,\tic) -- (\zre,-\tic) node[below]{$\zre$};
%\draw (\pre,\tic) -- (\pre,-\tic) node[below]{$\pre$};
\draw[dashed, C7] (0,0) --(1,+1);
\draw[dashed, C7] (0,0) --(1,-1);
\draw[dashed, C7] (0.7071,0.7071) -- (0.7071,-0.7071);
\end{scope}
%
\begin{scope}[scale=0.5, xshift=9cm]
\def\tic{0.1};
\def\Om{360/8}
\draw[help lines, C7!25, step=0.5cm] (-2,-2) grid (8,2);
\draw[->] (-1.5,0) -- (8.5,0) node[right]{$k$};
\draw[->] (0,-2) -- (0,2.5) node[above]{$x[k]=x_+[k]+x_-[k]=2 \cdot \cos(\frac{2\pi}{8})\epsilon[k] \,\ztransf\,
X(z)=2 \cdot \frac{z^2-\frac{1}{\sqrt{2}} z}{z^2-\sqrt{2} z +1}$ für $|z|>1$};
\foreach \x in {-1,0,...,8}{\draw (\x,+\tic) -- (\x,-\tic)  node[left]{$\x$};};
\foreach \y in {-1,1}{\draw (\tic,\y) -- (-\tic,\y)  node[left]{$\y$};};
\foreach \k in {-2,-1}
{
  \draw[stem] plot coordinates
  {
    (\k,{0*\k})
  };
};
\foreach \k in {0,1,...,8}{\draw[stem] plot coordinates{(\k,{2*cos(360/8*\k)})};};
\end{scope}
%
\end{tikzpicture}
%
\caption{Aufgabe \ref{sec:542FA69517}. Oben \& Mitte: \textbf{Kausale, komplexe}
1-Pol Signale $x_+[k]$, $x_-[k]$ mit fester Nullstelle in $z=0$.
Unten: 2-Pol/2-Nullstellen \textbf{kausales, reelles} Signal resultierend aus
der Addition $x_+[k]+x_-[k]$.
Durch Pole auf dem Einheitskreis sind es \textbf{harmonisch schwingende}
Signale bzw.
\textbf{grenzstabile} Systeme (hier sogar periodisch mit $N=8$).
Addition ist eine Parallelschaltung von Systemen, dies
erfordert Umformung in eine Reihenschaltung zur Darstellung des unteren
PN-Diagramms.}
\label{fig:542FA69517}
\end{figure}

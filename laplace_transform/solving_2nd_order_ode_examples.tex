\documentclass[a4paper,11pt,oneside]{scrartcl}
\usepackage[utf8]{inputenc}
\usepackage[T1]{fontenc}
\usepackage{lmodern}
\KOMAoptions{DIV=14}
\usepackage[english]{babel}
%
\usepackage{graphicx}
\usepackage{caption}
\usepackage{subcaption}
%
\usepackage{chngcntr}
\counterwithout{figure}{section}
\counterwithout{table}{section}
%
\usepackage{xcolor}
\usepackage{url}
%
\usepackage{amsmath,amssymb}
\usepackage{trfsigns}
\usepackage{nicefrac}
\usepackage{comment}
\usepackage{cancel}
%
\usepackage{tikz}
\usetikzlibrary{shapes.misc}
\tikzset{cross/.style={cross out, draw,
         minimum size=2*(#1-\pgflinewidth),
         inner sep=0pt, outer sep=0pt}}
\usepackage{circuitikz}
%
\newcommand{\eq}[1]{Eq. (\ref{#1})}
%
\newcommand{\fscom}[2][red]{\textcolor{#1}{#2}}
\definecolor{CalcColor}{rgb}{0.0,0.0,0.5}
\newcommand{\ExCalcCol}[2][CalcColor]{\textcolor{#1}{#2}}
%\excludecomment{calc}
\includecomment{calc}


\newcommand\uc{u_\mathrm{C}}
\newcommand\duc{\dot{u}_\mathrm{C}}
\newcommand\dduc{\ddot{u}_\mathrm{C}}

\newcommand\ul{u_\mathrm{L}}
\newcommand\dul{\dot{u}_\mathrm{L}}
\newcommand\ddul{\ddot{u}_\mathrm{L}}

\newcommand\uri{u_\mathrm{R1}}
\newcommand\duri{\dot{u}_\mathrm{R1}}
\newcommand\dduri{\ddot{u}_\mathrm{R1}}

\newcommand\urii{u_\mathrm{R2}}
\newcommand\durii{\dot{u}_\mathrm{R2}}
\newcommand\ddurii{\ddot{u}_\mathrm{R2}}

\newcommand\dx{\dot{x}}
\newcommand\ddx{\ddot{x}}
\newcommand\dy{\dot{y}}
\newcommand\ddy{\ddot{y}}

%\renewcommand\c{\imath}
%\newcommand\dc{\dot{\imath}}
%\newcommand\ddc{\ddot{\imath}}
\renewcommand\c{I}
\newcommand\dc{\dot{I}}
\newcommand\ddc{\ddot{I}}


\newcommand\fsd{\mathrm{d}}

%
\title{Electric RLC circuit to 2nd order ordinary differential equation (ODE)
with constant coefficients}
\author{Frank Schultz, fs446}
%
\begin{document}
%
\noindent Uni Rostock, IEF, INT, Signal- \& Systemtheorie, \#24015, SS2019,
\today, CC BY 4.0

\noindent Prof. Sascha Spors, Frank Schultz, Till Rettberg

\noindent \textbf{Electric RLC circuit to 2nd order ordinary differential equation (ODE)
with constant coefficients}

%##############################################################################
%##############################################################################
\section{Problem Statement}

\begin{minipage}{0.5\textwidth}
Find differential equation $f(\dx,\ddx,y,\dy,\ddy)=0$ and from that
Laplace system function $H(s)=\frac{Y(s)}{X(s)}$  of the
electric RLC circuit shown right for input voltage $x(t)$ and output voltage $y(t)$.
%
$R_1$, $R_2$ denote resistors, $C$ a capacitor and $L$ an inductor.
%
\end{minipage}
\begin{minipage}{0.5\textwidth}
\begin{center}
\begin{footnotesize}
\begin{circuitikz}[european, scale=0.55, /tikz/circuitikz/bipoles/length=1cm]
\draw (1,0) to [R,l=$R_1$,o-] (3,0);
\draw (3,0) to [C,l=$C$] (5,0);
\draw (5.5,0) to [L,l=$L$,*-] (5.5,-2.5);
\draw (5.5,-2.5) to [R,l=$R_2$,-*] (5.5,-5);
\draw (5,0) to [short,-o,] (7,0);
\draw (1,-5) to [short,o-o] (7,-5);
\draw[-latex] (1,-0.5)--(1,-4.5) node[midway,left]{$x(t)$};
\draw[-latex] (7,-0.5)--(7,-4.5) node[midway,right]{$y(t)$};
\end{circuitikz}
\end{footnotesize}
\end{center}
\end{minipage}

\section{Solution}
%
We use notation $\dot{u} = \frac{\fsd u}{\fsd t}$, $\ddot{u} = \frac{\fsd^2 u}{\fsd t^2}$
for temporal derivatives. We use large $I$ to indicate the current $i(t)$ for
derivative notation, since $\dot{i}$ and $\ddot{i}$ are hard to read.
%
Except the constants $R_1$, $R_2$, $C$ and $L$, all variables are functions
with respect to time.

The voltage / current relations for $R$, $L$ and $C$ are
\begin{align*}
&\uri = R_1 \c, \quad \urii = R_2 \c, \quad \ul = L \dc, \quad \uc = \frac{1}{C}\int \c \fsd t\\
&\c = \frac{\uri}{R_1}, \quad \c = \frac{\urii}{R_2}, \quad \c = \frac{1}{L} \int \ul \fsd t, \quad \c = C \duc
\end{align*}

\subsection{Method I: Complex Voltage Divider}
%
If we are not at all interested in (i) initial conditions of voltage and
current and (ii) in the ODE, we can simply apply the complex voltage
divider between $Y(\omega)$ and $X(\omega)$
%
\begin{align}
H(\omega) = \frac{Y(\omega)}{X(\omega)} =
\frac{\im \omega L + R_2}{R_1 + \frac{1}{\im \omega C} + \im \omega L + R_2}
\end{align}
%
and substitute $\im \omega \rightarrow s = \sigma + \im\omega$
%
\begin{align}
&H(s) = \frac{Y(s)}{X(s)} = \frac{s L + R_2}{R_1 + \frac{1}{s C} + s L + R_2}=
\frac{s^2 + \frac{R_2}{L} s}{s^2 + \frac{R_1 + R_2}{L} s + \frac{1}{L C}}
\end{align}
%
to obtain the Laplace system function.
%
\subsection{Method II: ODE with the Current as Intermediate Variable}
%
Here, we derive ODEs, but we are not interested in initial conditions.
With above voltage / current relations, we can state ODEs following from mesh
equations for input $x$ and output $y$ as
%
\begin{align}
x& = R_1 \c + \frac{1}{C} \int \c \fsd t + L \dc + R_2 \c,\\
y& = L \dc + R_2 \c.
\end{align}
%
The Laplace transform of $x$, $y$ and $\c$ in both equations yields
%
\begin{align}
X(s)& = R_1 \c(s) + \frac{1}{s C} \c(s) + s L \c(s) + R_2 \c(s)\\
Y(s)& = s L \c(s) + R_2 \c(s),
\end{align}
%
considering vanishing initial conditions,
%
from which the transfer function
%
\begin{align}
H(s)& = \frac{Y(s)}{X(s)} = \frac{s L \c(s) + R_2 \c(s)}{R_1 \c(s) + \frac{1}{sC} \c(s) + s L \c(s) + R_2 \c(s)},\\
%H(s)& = \frac{Y(s)}{X(s)} = \frac{s L + R_2}{R_1+R_2 + \frac{1}{s C} + s L}\\
%H(s)& = \frac{Y(s)}{X(s)} = \frac{s + \frac{R_2}{L}}{\frac{R_1+R_2}{L} + \frac{1}{s L C} + s}\\
H(s)& = \frac{Y(s)}{X(s)} = \frac{s^2 + \frac{R_2}{L} s}{s^2 + \frac{R_1+R_2}{L} s + \frac{1}{L C}}
\end{align}
%
results by canceling the current $I(s)$.

\subsection{Method III: ODE}
%
The most tedious derivation aims at finding the full ODE $f(\dx,\ddx,y,\dy,\ddy)=0$
without any intermediate variable, such as the current involved.
%
The circuit under discussion requires more careful inspection than other RLC
circuits , since there is only one current involved and the unknown voltage drop
$y$ is along two electric elements.

There is certainly a more formalised way to solve this problem, however a suitable
approach would be as follows:
%
Let us start with above stated mesh equations
%
\begin{align}
x& = R_1 \c + \frac{1}{C} \int \c \fsd t + L \dc + R_2 \c,\\
y& = L \dc + R_2 \c
\end{align}
%
and rearrange to equate them
$x - R_1 \c - \frac{1}{C} \int \c \fsd t - R_2 \c = L \dc = y - R_2 \c$
to
%
\begin{align}
%x - R_1 \c - \frac{1}{C} \int \c \fsd t - R_2 \c = y - R_2 \c\\
x - (R_1+R_2) \c - \frac{1}{C} \int \c \fsd t = y - R_2 \c.
\end{align}
%
We aim to find relations in order to substitute the current. Since we deal with a
second order ODE, we expect second order derivatives, at least for the output $y$.
%
Thus, it appears reasonable to apply a temporal derivation
%
\begin{align}
\dx - (R_1+R_2) \dc - \frac{1}{C} \c  = \dy - R_2 \dc.
\end{align}
%
Recall that $\c = \frac{1}{L} \int \ul \fsd t \rightarrow \dc = \frac{1}{L} \ul$.
%
Inserting this
%
\begin{align}
\dx - \frac{R_1+R_2}{L} \ul - \frac{1}{C} \c  = \dy - R_2 \frac{1}{L} \ul.
\end{align}
%
This simplified things, so let us perform these two steps again.
%
Second temporal derivation
%
\begin{align}
\ddx - \frac{R_1+R_2}{L} \dul - \frac{1}{C} \dc  = \ddy - \frac{R_2}{L} \dul
\end{align}
%
and again inserting $\dc = \frac{1}{L} \ul$ yields
%
\begin{align}
\ddx - \frac{R_1+R_2}{L} \dul - \frac{1}{L C} \ul  = \ddy - \frac{R_2}{L} \dul.
\end{align}
%
Now, we should rearrange this equation
%
\begin{align}
\label{eq:ODE_with_uL}
\ddx + \frac{R_2}{L} \dul = \ddy + \frac{R_1+R_2}{L} \dul + \frac{1}{L C} \ul
\end{align}
%
due to our expectation that coefficients for $x, \dx, \ddx, y, \dy, \ddy$
will appear as positive constants.
%
Next, we need to get rid of $\ul$ and $\dul$. For that we can use the mesh
equations
%
\begin{align}
\label{eq:ODE_helper_x}
&x = \uri + \uc + \ul + \urii \rightarrow& \ul = x - \uri - \uc - \urii,\\
\label{eq:ODE_helper_y}
&y = \ul + \urii \rightarrow& \ul = y - \urii.
\end{align}
%
Applying \eq{eq:ODE_helper_x} into the left, $x$-related  side of \eq{eq:ODE_with_uL}
and applaying \eq{eq:ODE_helper_y} into the right, $y$-related side of \eq{eq:ODE_with_uL}
results in
%
\begin{align}
\label{eq:ODE_tmp1}
\ddx + \frac{R_2}{L} (\dx - \duri - \duc - \durii) = \ddy + \frac{R_1+R_2}{L} (\dy - \durii) + \frac{1}{L C} (y - \urii).
\end{align}
%
It appears meaningful to sort terms
\begin{align}
\ddx + \frac{R_2}{L} \dx - \frac{R_2}{L} (\duri + \duc + \durii)
=
\ddy + \frac{R_1+R_2}{L} \dy + \frac{1}{L C} y
- \frac{R_1+R_2}{L} \durii - \frac{1}{L C} \urii.
\end{align}
Since we have no degrees of freedom left
(we have used all relations between voltage, current and Kirchhoff's laws)
we are in hope that the ODE reduces to
\begin{align}
\label{eq:ODE}
\boxed{
\ddx + \frac{R_2}{L} \dx
=
\ddy + \frac{R_1+R_2}{L} \dy + \frac{1}{L C} y}
\end{align}
under the assumption that the remaining terms
\begin{align}
- \frac{R_2}{L} (\duri + \duc + \durii)
\stackrel{?}{=}
- \frac{R_1+R_2}{L} \durii - \frac{1}{L C} \urii
\end{align}
and thus cancel out.
%
Rearranging yields
\begin{align}
&R_2 C (\duri + \duc + \durii)
\stackrel{?}{=}
(R_1+R_2) C \durii + \urii,\\
&R_2 C \duri+
R_2 C \duc+
\cancel{R_2 C \durii}
\stackrel{?}{=}
R_1 C \durii + \cancel{R_2 C \durii} + \urii,\\
&R_2 C \duri+
R_2 C \duc
\stackrel{?}{=}
R_1 C \durii+
\urii.
\end{align}
%
Inserting the current (recall that $\c = C \duc$)
\begin{align}
R_2 C R_1 \dc+
R_2 C \frac{1}{C} \c+
=
R_1 C R_2 \dc+
R_2 \c
\end{align}
proves identity as required.

Thus, the derived ODE in \eq{eq:ODE} can be transformed to Laplace domain
\begin{align}
s^2 X(s) + \frac{R_2}{L} s X(s)
=
s^2 Y(s) + \frac{R_1+R_2}{L} s Y(s) + \frac{1}{L C} Y(s)
\end{align}
%
for \underline{vanishing initial conditions} and we can confirm the already known result
%
\begin{align}
\boxed{
H(s) = \frac{Y(s)}{X(s)} =
\frac{s^2 + \frac{R_2}{L} s}
{s^2 + \frac{R_1+R_2}{L} s + \frac{1}{L C}}
}.
\end{align}
%
\bibliographystyle{IEEEtranSA.bst}
\bibliography{literature}
\end{document}

\documentclass[11pt,a4paper,DIV=12]{scrartcl}
\usepackage{scrlayer-scrpage}
\usepackage[utf8]{inputenc}
\usepackage{fouriernc}
\usepackage[T1]{fontenc}
\usepackage[german]{babel}
\usepackage[hidelinks]{hyperref}
\usepackage{natbib}
\usepackage{url}
\usepackage{amsmath}
\usepackage{amsfonts}
\usepackage{amssymb}
\usepackage{trfsigns}
\usepackage{marvosym}
\usepackage{nicefrac}
\usepackage{graphicx}
\usepackage{subcaption}
\usepackage{xcolor}
\usepackage{comment}
\usepackage{mdframed}
\usepackage{tikz}
\usepackage{circuitikz}
\usepackage{pgfplots}
\usepackage{bm}
\usepackage{cancel}
\bibliographystyle{dinat}

\usepackage{../sig_sys_macros}

%------------------------------------------------------------------------------
\ohead{Signal- und Systemtheorie Übung}
\cfoot{\pagemark}
\ofoot{\tiny\url{https://github.com/spatialaudio/signals-and-systems-exercises}}

\begin{document}
%
\noindent Signal- und Systemtheorie Übung\footnote{This tutorial is provided as
Open Educational Resource (OER), to be found at
\url{https://github.com/spatialaudio/signals-and-systems-exercises}
accompanying the OER lecture
\url{https://github.com/spatialaudio/signals-and-systems-lecture}.
%
Both are licensed under a) the Creative Commons Attribution 4.0 International
License for text and graphics and b) the MIT License for source code.
%
Please attribute material from the tutorial as \textit{Frank Schultz,
Continuous- and Discrete-Time Signals and Systems - A Tutorial Featuring
Computational Examples, University of Rostock} with
\texttt{github URL, commit number and/or version tag, year, (file name and/or
content)}.}---Frank Schultz, Sascha Spors,
Institut für Nachrichtentechnik (INT),
Fakultät für Informatik und Elektrotechnik (IEF),
Universität Rostock \&
Robert Hauser, Universität Rostock---Sommersemester 2022, Version: \today
%

\section*{Aufgabe: XXX}
\vspace{-0.4cm}{\tiny XXXXXXXXXX} %insert unique 10digit hex hash from random_hex.txt (and delete used hex their)
%
%
%
\subsection*{a)}
\subsection*{b)}
\subsection*{c)}
%
%
%
\section*{Lösung}
%
%
%
\subsection*{a)}
\subsection*{b)}
\subsection*{c)}
%
%
%
% test if all equation/graphics stuff is properly working:
\begin{align}
\label{eq:im}
\im^2=-1 \myDFT\quad\ztransf\quad\DTFT
\end{align}
%
%
%
\begin{center}
\begin{tikzpicture}[scale=0.8]
\def \xmin {-4}
\def \xmax {4.2}
\def \ymin {-1}
\def \ymax {1.5}
\draw[->] (\xmin-0.1 ,0) -- (\xmax+0.1,0) node[below]{$t$};
\draw[->] (0,\ymin-0.1) -- (0,\ymax+0.1) node[left]{$x(t)$};
\foreach \x in {\xmin,...,3}{\draw (\x,0.05) -- (\x,-0.05);}
\draw (0.05,-1) -- (-0.05,-1);
\draw (0.05,1) -- (-0.05,1) node[left]{\small $1$};
\draw(2,0) node[below]{\small $2$};
\begin{scope}
\draw[C0, ultra thick, domain=-2:2,variable=\t,samples=100,smooth] plot(\t,{ cos (4*pi*\t r)});
\draw[C0, ultra thick] (-4,0) -- (-2,0);
\draw[C0, ultra thick] (2,0) -- (4,0);
\draw[C0, dashed, thin] (2,0) -- (2,1);
\draw[C0, dashed, thin] (-2,0) -- (-2,1);
\end{scope}
\end{tikzpicture}
\end{center}
%
%
%
\begin{center}
\begin{tikzpicture}[scale=1.5]
\def \tic {0.05}
\def \pabs{0} % largest radius for poles to determine roc-> here a FIR
\def \rocmax{1.6} % sketch of outer roc domain
%
% basic diagram features:
\filldraw[even odd rule,C2!50] (0,0) circle(\pabs) decorate
[decoration={snake, segment length=15pt, amplitude=1pt}]
{(0,-3pt) circle(\rocmax)}; % sketch the roc domain
%
\draw[help lines, C7!50, step=0.25cm] (-\rocmax,-\rocmax) grid (\rocmax,\rocmax);
%
\draw[C3, thick] (0,0) circle(1);  % unit circle, i.e. DTFT domain
\draw[C7, thin] (0,0) circle(1.4142);  % unit circle, i.e. DTFT domain
\draw[C7, thin] (0,0) circle(0.7071);  % unit circle, i.e. DTFT domain
%
\draw (1+2*\tic,-2*\tic) node{$1$}; % indicate that this is the unit circle
\draw (0.5,-2*\tic) node{$\nicefrac{1}{2}$};
\draw[->] (-1.75,0)--(1.75,0) node[right]{$\Re\{z\}$}; % axis label
\draw[->] (0,-1.75)--(0,1.75) node[above]{$\Im\{z\}$}; % axis label
%
\draw (1.2*0.86602540378*\rocmax,1.2*0.5*\rocmax) node[C2!75]{KB}; % indicate the roc
%
% the z-transfer function specific stuff:
\draw (0.75,1.25) node[black]{$g=1$}; % indicate gain factor
%
% draw the poles / zeros and if desired ticks:
\draw[C0, ultra thick] (0,0) node{\Huge $\times$};
\draw[C0] (-4*\tic,4*\tic) node{10};
\draw[C0, ultra thick] (0,+1) node{\Huge $\circ$};
\draw[C0, ultra thick] (0,-1) node{\Huge $\circ$};
\draw[C0, ultra thick] (+1,+1) node{\Huge $\circ$};
\draw[C0, ultra thick] (-1,-1) node{\Huge $\circ$};
\draw[C0, ultra thick] (-1,+1) node{\Huge $\circ$};
\draw[C0, ultra thick] (+1,-1) node{\Huge $\circ$};
\draw[C0, ultra thick] (+1/2,+1/2) node{\Huge $\circ$};
\draw[C0, ultra thick] (-1/2,-1/2) node{\Huge $\circ$};
\draw[C0, ultra thick] (-1/2,+1/2) node{\Huge $\circ$};
\draw[C0, ultra thick] (+1/2,-1/2) node{\Huge $\circ$};
\end{tikzpicture}
\end{center}
%
%
%
\begin{figure*}
\centering
\begin{subfigure}{\textwidth}
\centering
\begin{tikzpicture}[scale=1]
\def\tic{0.1};
\def\Om{360/8}
\draw[help lines, C7!25, step=1cm] (0,-1) grid (7,1);
\draw[->] (-1.5,0) -- (9,0) node[right]{$k$};
\draw[->] (0,-1) -- (0,2) node[left]{$x[k]=\cos(\frac{2\pi}{8}k)$};
\foreach \x in {-1,0,...,8}{\draw (\x,+\tic) -- (\x,-\tic)  node[left]{$\x$};};
\foreach \y in {-1,1}{\draw (\tic,\y) -- (-\tic,\y)  node[left]{$\y$};};
\foreach \k in {-1,0,...,8}{\draw[stem2] plot coordinates{(\k,{cos(\Om*\k)})};};
\foreach \k in {-1,8}{\draw[stem2, C3] plot coordinates{(\k,{cos(\Om*\k)})};};
\end{tikzpicture}
\caption{$N=8\in\mathbb{Z}$, daher periodisch in 8 Signalwerten / 8 Folgengliedern,
Periode in blau.}
\label{fig:cos8}
\end{subfigure}
%
\begin{subfigure}{\textwidth}
\centering
\begin{tikzpicture}[scale=1]
\def\tic{0.1};
\def\Om{360/8.5}
\draw[->] (-1.5,0) -- (9,0) node[right]{$k$};
\draw[->] (0,-1) -- (0,2) node[left]{$x[k]=\cos(\frac{2\pi}{8.5}k)$};
\foreach \x in {-1,0,...,8}{\draw (\x,+\tic) -- (\x,-\tic)  node[left]{$\x$};};
\foreach \y in {-1,1}{\draw (\tic,\y) -- (-\tic,\y)  node[left]{$\y$};};
\foreach \k in {-1,0,...,8}{\draw[stem2,C1] plot coordinates{(\k,{cos(\Om*\k)})};};
\end{tikzpicture}
\caption{$N=8.5\notin\mathbb{Z}$, daher \textbf{nicht} periodisch in $N$.
Aber periodisch in $2 N = 17$.}
\label{fig:cos8.5}
\end{subfigure}
%
%
%
\begin{subfigure}{\textwidth}
\centering
\begin{tikzpicture}[scale=1]
\def\tic{0.1};
\def\Om{360/9}
\draw[help lines, C7!25, step=1cm] (0,-1) grid (8,1);
\draw[->] (-1.5,0) -- (9,0) node[right]{$k$};
\draw[->] (0,-1) -- (0,2) node[left]{$x[k]=\cos(\frac{2\pi}{9}k)$};
\foreach \x in {-1,0,...,9}{\draw (\x,+\tic) -- (\x,-\tic)  node[left]{$\x$};};
\foreach \y in {-1,1}{\draw (\tic,\y) -- (-\tic,\y)  node[left]{$\y$};};
\foreach \k in {-1,0,...,9}{\draw[stem2] plot coordinates{(\k,{cos(\Om*\k)})};};
\foreach \k in {-1,9}{\draw[stem2, C3] plot coordinates{(\k,{cos(\Om*\k)})};};
\end{tikzpicture}
\caption{$N=9\in\mathbb{Z}$, daher periodisch in 9 Werten, Periode in blau.}
\label{fig:cos9}
\end{subfigure}
\caption{Cosinusfolgen mit $\Omega_0=\frac{2\pi}{N}$ für $N=8,\,8.5,\,9$ über
\textit{Zeit}-Index $k$ als
sogenannter Stem-Plot (\textit{stem} [englisch] - Stiel, Halm, Stängel [deutsch]).}
\label{fig:Cosinusfolgen}
\end{figure*}
%
%
%
\eq{eq:im}, \fig{fig:Cosinusfolgen} vs. \fig{fig:cos9}



\renewcommand{\refname}{Buchzitate}
\cite{Strang2007,Strang2010}
\clearpage
\bibliography{../tutorial_latex_deu/literatur}
\end{document}

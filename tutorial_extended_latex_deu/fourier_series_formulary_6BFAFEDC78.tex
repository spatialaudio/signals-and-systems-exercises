\documentclass[11pt,a4paper,DIV=12]{scrartcl}
\usepackage{scrlayer-scrpage}
\usepackage[utf8]{inputenc}
\usepackage{fouriernc}
\usepackage[T1]{fontenc}
\usepackage[german]{babel}
\usepackage[hidelinks]{hyperref}
\usepackage{natbib}
\usepackage{url}
\usepackage{amsmath}
\usepackage{amsfonts}
\usepackage{amssymb}
\usepackage{trfsigns}
\usepackage{marvosym}
\usepackage{nicefrac}
\usepackage{graphicx}
\usepackage{subcaption}
\usepackage{xcolor}
\usepackage{comment}
\usepackage{mdframed}
\usepackage{tikz}
\usepackage{circuitikz}
\usepackage{pgfplots}
\usepackage{bm}
\usepackage{cancel}
\bibliographystyle{dinat}

\usepackage{../sig_sys_macros}

%------------------------------------------------------------------------------
\ohead{Signal- und Systemtheorie Übung}
\cfoot{\pagemark}
\ofoot{\tiny\url{https://github.com/spatialaudio/signals-and-systems-exercises}}

\begin{document}
%
\noindent Signal- und Systemtheorie Übung\footnote{This tutorial is provided as
Open Educational Resource (OER), to be found at
\url{https://github.com/spatialaudio/signals-and-systems-exercises}
accompanying the OER lecture
\url{https://github.com/spatialaudio/signals-and-systems-lecture}.
%
Both are licensed under a) the Creative Commons Attribution 4.0 International
License for text and graphics and b) the MIT License for source code.
%
Please attribute material from the tutorial as \textit{Frank Schultz,
Continuous- and Discrete-Time Signals and Systems - A Tutorial Featuring
Computational Examples, University of Rostock} with
\texttt{github URL, commit number and/or version tag, year, (file name and/or
content)}.}---Frank Schultz, Sascha Spors,
Institut für Nachrichtentechnik (INT),
Fakultät für Informatik und Elektrotechnik (IEF),
Universität Rostock \&
Robert Hauser, Universität Rostock---Sommersemester 2023, Version: \today\\
\verb|fourier_series_formulary_6BFAFEDC78|
%

\noindent Main author: https://github.com/robhau, check: https://github.com/fs446

\section*{Formelsammlung Komplexe Fourierreihe}

\noindent Komplexe Schreibweise der Fourierreihe und hinfort notwendige Annahme,
dass $x(t) = x(t+T)$, also periodisch mit Periodendauer $T$ und
Grundschwingungs-Kreisfrequenz $\omega=\frac{2\pi}{T}$.
Notation für $k$-ten Fourierkoeffizienten $X_k \in\mathbb{C}$
(bzw. auch verwendet $c_k, C_k, F_k, G_k, H_k$).
Zeitsignal darf auch komplexwertig sein, i.e. $x(t) \in\mathbb{C}$.

\textbf{Zeitsignal-Synthese}
\begin{align}
	x(t)=\frac{1}{T}\sum_{k=-\infty}^{+\infty}X_k\e^{+\im k \omega t}
\end{align}

\textbf{Zeitsignal-Analyse}, i.e. Berechnung der Fourierkoeffizienten
\begin{gather}
	X_k=\int_{t_0}^{t_0+T}x(t)\e^{-\im k \omega t}\mathrm{d}t
\end{gather}
%
%
%
\textbf{Differentiation} (falls $x(t)$ differenzierbar)
%
%
\begin{gather}
	\frac{\mathrm{d}}{\mathrm{d}t}x(t)
	=\frac{\mathrm{d}}{\mathrm{d}t}\frac{1}{T}\sum_{k=-\infty}^{+\infty}X_k\e^{+\im k \omega t}
	=\frac{1}{T}\sum_{k=-\infty}^{+\infty}X_k\frac{\mathrm{d}}{\mathrm{d}t}\e^{+\im k \omega t}
	=\frac{1}{T}\sum_{k=-\infty}^{+\infty}\im\omega kX_k\e^{+\im k \omega t}
\end{gather}
%
%
\textbf{Integration} (unbestimmt)
%
%
\begin{align}
	\int x(t)\mathrm{d}t&=\int\frac{1}{T}\sum_{k=-\infty}^{+\infty}X_k\e^{+\im k \omega t}\mathrm{d}t \\
	&= \frac{1}{T}\sum_{k=-\infty}^{-1}\bigg [X_k\int\e^{+\im k \omega t}\mathrm{d}t\bigg ]+\frac{1}{T}\sum_{k=1}^{+\infty}\bigg [X_k\int\e^{+\im k \omega t}\mathrm{d}t\bigg ]+\int X_0\mathrm{d}t\\
	&=\frac{1}{T}\sum_{k=-\infty}^{-1}\frac{1}{\im\omega k}X_k\e^{+\im k \omega t}+\frac{1}{T}\sum_{k=1}^{+\infty}\frac{1}{\im\omega k}X_k\e^{+\im k \omega t}+\int X_0\mathrm{d}t
\end{align}
\textbf{Modulation}
\begin{gather}
	y(t)=\e^{\im n\omega t}x(t) \quad n\in\mathbb{Z}\\
	Y_k=\int_0^Tx(t)\e^{-\im(k-n)\omega t}\mathrm{d}t\quad\Bigg | \quad m = k-n \\
	Y_{m+n}=\int_0^Tx(t)\e^{-\im  m \omega t}\mathrm{d}t=X_m \\
	Y_{k}=X_{k-n}
\end{gather}
\textbf{Verschiebung}
\begin{gather}
	y(t)=x(t-\tau)\quad\tau\in\mathbb{R}\\
	Y_k=\int_0^{T}y(t)\e^{-\im\omega k t}\mathrm{d}t
	=\int_0^{T}x(t-\tau)\e^{-\im k \omega t}\mathrm{d}t \quad \Bigg | \quad l = t-\tau\\
	Y_k=\int_{-\tau}^{T-\tau}x(l)\e^{-\im\omega k (l+\tau)}\mathrm{d}l=\e^{-\im \omega k \tau}\int_{0}^{T}x(l)\e^{-\im\omega k l}\mathrm{d}l \\
	Y_k=\e^{-\im\omega k \tau}X_k
\end{gather}
\textbf{Zeitumkehr}
\begin{gather}
	y(t)=x(-t) \\
	Y_k=\int_0^{T}y(t)\e^{-\im\omega k t}\mathrm{d}t=\int_0^{T}x(-t)\e^{-\im\omega k t}\mathrm{d}t\quad\Bigg | \quad \tau = -t \\
	Y_k=-\int_0^{-T}x(\tau)\e^{\im\omega k t}\mathrm{d}\tau=\int_{-T}^{0}x(\tau)\e^{\im\omega k t}\mathrm{d}\tau \quad \Bigg |\quad m = -k \\
	Y_{-m}=\int_{-T}^{0}x(\tau)\e^{-\im\omega m t}\mathrm{d}\tau \\
	Y_{-m}=X_m \\
	Y_{k}=X_{-k}
\end{gather}
\textbf{Komplex konjugiert}
\begin{align}
	x^*(t)&=\left [\frac{1}{T}\sum_{k=-\infty}^{+\infty}c_k\e^{+\im k \omega t}\right ]^*=\left [\frac{1}{T}\sum_{k=-\infty}^{+\infty}(\Re\{c_k\}+\im\Im\{c_k\})(\cos(k\omega t)+\im\sin(k\omega t))\right ]^*\nonumber \\
	&=\left [\frac{1}{T}\sum_{k=-\infty}^{+\infty}\Re\{c_k\}\cos(k\omega t)-\Im\{c_k\}\sin(k\omega t)+\im (\Re\{c_k\}\sin(k\omega t)+\Im\{c_k\}\cos(k\omega t)) \right ]^*\nonumber \\
	&=\left [\frac{1}{T}\sum_{k=-\infty}^{+\infty}\left (\Re\{c_k\}\cos(k\omega t)-\Im\{c_k\}\sin(k\omega t)\right )+\im\frac{1}{T}\sum_{k=-\infty}^{+\infty}\left (\Re\{c_k\}\sin(k\omega t)+\Im\{c_k\}\cos(k\omega t)\right )\right ]^*\nonumber \\
	&=\frac{1}{T}\sum_{k=-\infty}^{+\infty}\left (\Re\{c_k\}\cos(k\omega t)-\Im\{c_k\}\sin(k\omega t)\right )-\im\sum_{k=-\infty}^{+\infty}\left (\Re\{c_k\}\sin(k\omega t)+\Im\{c_k\}\cos(k\omega t)\right )\nonumber \\
	&=\frac{1}{T}\sum_{k=-\infty}^{+\infty}\left (\Re\{c_k\}\cos(k\omega t)-\Im\{c_k\}\sin(k\omega t-\im(\Re\{c_k\}\sin(k\omega t)+\Im\{c_k\}\cos(k\omega t))\right )\nonumber \\
	&=\frac{1}{T}\sum_{k=-\infty}^{+\infty}\left(\Re\{c_k\}-\im\Im\{c_k\}\right)\left(\cos(\omega k t)-\im\sin(k\omega t) \right)=\frac{1}{T}\sum_{k=-\infty}^{+\infty}c_k^*\e^{-\im k \omega t}
	= \frac{1}{T}\sum_{k=-\infty}^{+\infty}c_{-k}^*\e^{+\im k \omega t}
\end{align}
\textbf{Faltung} (zyklisch in $T$)
\begin{align}
	y(t)&=x(t)\circledast h(t)
	=\int_{-\frac{T}{2}}^{\frac{T}{2}}x(\tau)h(t-\tau)\diff \tau
	=\int_{-\frac{T}{2}}^{\frac{T}{2}}\left (\frac{1}{T}\sum_{k=-\infty}^{\infty}X_k\e^{+\im k \omega \tau}\right )\cdot \left (\frac{1}{T}\sum_{i=-\infty}^{+\infty}H_i\e^{+\im i \omega (t-\tau)}\right )\diff \tau\nonumber \\
	&=\frac{1}{T^2}\int_{-\frac{T}{2}}^{\frac{T}{2}}\sum_{k=-\infty}^{+\infty}\sum_{i=-\infty}^{+\infty}\left (X_kH_i\e^{+\im k \omega \tau}\e^{+\im i \omega (t-\tau)}\right )\diff \tau
	=\frac{1}{T^2}\sum_{k=-\infty}^{+\infty}\sum_{i=-\infty}^{+\infty}\left (\int_{-\frac{T}{2}}^{\frac{T}{2}}X_kH_i\e^{+\im (k-i) \omega \tau}\e^{+\im i \omega t}\diff \tau\right )\nonumber \\
	&=\frac{1}{T^2}\sum_{k=-\infty}^{+\infty}\sum_{i=-\infty}^{+\infty}\left (X_kH_i\e^{-\im i \omega t}\int_{-\frac{T}{2}}^{+\frac{T}{2}}\e^{+\im (k-i)\omega t}\diff \tau \right )
\end{align}
Für $k=i$ ergibt das Integral $T$.
%
%
\begin{align}
	\int_{-\frac{T}{2}}^{+\frac{T}{2}}\e^{+\im (k-i)\omega \tau}\diff \tau \underset{k=i}{=}\int_{-\frac{T}{2}}^{+\frac{T}{2}}1\diff \tau =\tau \Bigg |_{\tau=-\frac{T}{2}}^{\tau=+\frac{T}{2}}=T
\end{align}
%
%
Für $k\neq i$ ergibt das Integral $0$ (i.e. die Nullstellen einer gewichteten Spaltfunktion)
%
%
\begin{align}
	\int_{-\frac{T}{2}}^{+\frac{T}{2}}\e^{+\im (k-i)\omega \tau}\diff \tau &\underset{k\neq i}{=}\frac{\e^{+\im (k-i)\omega \tau}}{+\im (k-i)\omega}\Bigg |_{\tau=-\frac{T}{2}}^{\tau=+\frac{T}{2}}
	=\frac{1}{\im (k-i)\omega }\left (\e^{+\im (k-i)\frac{2\pi}{T}\frac{T}{2}}-\e^{-\im (k-i)\frac{2\pi}{T}\frac{T}{2}}\right )\nonumber \\
	&=\frac{2}{ (k-i)\omega}\sin((k-i)\pi)=0
\end{align}
%
Da sich also nur die Koeffizientenpaare addieren, bei denen der Index gleich ist, können wir die Doppelsumme auch als eine einfache Summe schreiben.
%
%
\begin{align}
	\frac{1}{T^2}\sum_{k=-\infty}^{+\infty}\sum_{i=-\infty}^{+\infty}\left (X_kH_i\e^{-\im i \omega t}\int_{-\frac{T}{2}}^{+\frac{T}{2}}\e^{+\im (k-i)\omega t}\diff \tau \right )
	=\frac{1}{T^2}\sum_{k=i=-\infty}^{+\infty}TX_kH_i\e^{+\im i\omega t}
	=\frac{1}{T}\sum_{k=-\infty}^{+\infty}X_kH_k\e^{+\im k \omega t}
\end{align}
%
Daher zusammenfassend
\begin{align}
y(t) = x(t)\circledast h(t) = \frac{1}{T}\sum_{k=-\infty}^{+\infty} Y_k \e^{+\im k \omega t} = \frac{1}{T}\sum_{k=-\infty}^{+\infty}X_kH_k\e^{+\im k \omega t}
\end{align}


\textbf{Multiplikation}
%
\begin{align}
	y(t)&=x(t)\cdot h(t)
	=\left (\frac{1}{T}\sum_{k=-\infty}^{+\infty}X_k\e^{+\im k \omega t}\right )\cdot\left ( \frac{1}{T}\sum_{\kappa=-\infty}^{+\infty}H_{\kappa}\e^{+\im \kappa \omega t}\right )
	=\frac{1}{T^2}\sum_{k=-\infty}^{+\infty}\sum_{\kappa=-\infty}^{+\infty}X_kH_{\kappa}\e^{\im (k+\kappa)\omega t}\nonumber \\
	&=\cdots +\frac{1}{T^2}\sum_{\kappa=-\infty}^{+\infty}X_{-1}H_{\kappa}\e^{+\im(\kappa-1)\omega t}+\frac{1}{T^2}\sum_{\kappa=-\infty}^{+\infty}X_0H_{\kappa}\e^{+\im \kappa \omega t}+\frac{1}{T^2}\sum_{\kappa=-\infty}^{+\infty}X_1H_{\kappa}\e^{+\im(i+1)\omega t}+\cdots\nonumber \\
	&=\frac{1}{T}\Bigg [\cdots+\frac{1}{T}X_{-1}H_{-1}\e^{-2\im \omega }+\frac{1}{T}X_{-1}H_0\e^{-\im \omega t}+\frac{1}{T}X_{-1}H_1+\cdots+\frac{1}{T}X_0H_{-1}\e^{-\im \omega t}+\frac{1}{T}X_0H_0\nonumber\\
	&+\frac{1}{T}X_0H_1\e^{+\im  \omega t}
	+\cdots+\frac{1}{T}X_1H_{-1}+\frac{1}{T}X_1H_0\e^{+\im \omega t}
	+\frac{1}{T}X_1H_1\e^{+2\im \omega t}+\cdots\Bigg]
\end{align}
%
%
Noch einmal die allgemeine Form der Fourierreihe:
%
\begin{align}
	x(t)=\frac{1}{T}\sum_{n=-\infty}^{+\infty}c_n\e^{+\im n \omega t}.
\end{align}
%
Für $n=0$ beträgt $c_n$
%
\begin{align}
	c_n=\cdots + \frac{1}{T}X_{-1}H_{1}+\frac{1}{T}X_0H_0+\frac{1}{T}X_1H_{-1}.
\end{align}
%
Dabei durchläuft der Index von $X$ $\mathbb{Z}$ und der Index von $H$ ist eine
Lösung der Gleichung
%
\begin{align}
	\kappa = n-k.
\end{align}
%
Also
%
\begin{align}
	y(t)=x(t)\cdot h(t)=\frac{1}{T}\sum_{n=-\infty}^{+\infty}\left(\sum_{k=-\infty}^{+\infty}\frac{1}{T}X_kH_{n-k}\right)\e^{\im n \omega t}.
\end{align}

Der Ausdruck in der großen Klammer ist eine diskrete Faltungssumme, hier werden
also zwei Koeffizientenfolgen miteinander gefaltet.

%
\textbf{Skalierung} (vgl. \cite[Kap. 1, S. 22]{Butz2012})\\
$y(t) = x(a\cdot t)$	\\
$a>0$: $C_k$ ändert sich nicht, $\omega_y = \omega_x \cdot a$.\\
$a<0$: $C_{k_y} = C_{-k_x}$, $\omega_y = \omega_x \cdot |a|$.\\
\textbf{Parseval`sches Theorem}:
\begin{align}
	z\cdot z^*=(a+\im b)\cdot (a-\im b)=a^2+b^2=|z|^2.
\end{align}
%
%
\begin{align}
	\int_{t_0}^{t_0+T}|x_b(t)|^2\diff t
	&=\int_{t_0}^{t_0+T}x_b(t)\cdot x_b^*(t)\diff t
	=\int_{t_0}^{t_0+T}\left (\frac{1}{T}\sum_{k=-\infty}^{+\infty}c_k\e^{+\im k \omega t}\right )\cdot \left (\frac{1}{T}\sum_{i=-\infty}^{+\infty}c_i^*\e^{-\im i \omega t}\right )\diff t\nonumber \\
	&=\frac{1}{T^2}\int_{t_0}^{t_0+T}\sum_{k=-\infty}^{+\infty}\sum_{i=-\infty}^{+\infty}c_kc_i^*\e^{\im (k-i)\omega t}\diff t =\frac{1}{T^2}\sum_{k=-\infty}^{+\infty}\sum_{i=-\infty}^{+\infty}\int_{t_0}^{t_0+T}c_kc_i^*\e^{\im (k-i)\omega t}\diff t
\end{align}
%
Es müssen 2 Fälle unterschieden werden: $k=i$ und $k\neq i$.
%
\begin{align}
	\int_{t_0}^{t_0+T}c_kc_i^*\e^{+\im (k-i)\omega t}\diff t \underset{k=i}{=}c_kc_k^*t\Bigg |_{t=t_0}^{t=t_0+T}=Tc_kc_k^*=T|c_k|^2
\end{align}
%
\begin{align}
	\int_{t_0}^{t_0+T}c_kc_i^*\e^{+\im (k-i)\omega t}\diff t
	&\underset{k\neq i}{=}\frac{c_kc_i^*}{\im (k-i) \omega }\e^{+\im (k-i)\omega t}\Bigg |_{t=t_0}^{t=t_0+T}
	=\frac{c_kc_i^*}{\im (k-i) \omega }\left (\e^{+\im (k-i)\omega (t_0+T)}-\e^{+\im (k-i)\omega t_0}\right )\nonumber \\
	&=\frac{c_kc_i^*}{\im (k-i) \omega }\left (\e^{+\im ((k-i)\omega t_0+(k-i)2\pi)}-\e^{+\im (k-i)\omega t_0}\right )=0
\end{align}
Es werden also nur die Terme addiert, bei denen $k=i$ gilt.
%

Wir können die Doppelsumme vereinfachen und folgende Korrespondenz aufstellen:
%
\begin{align}
	\int_{t_0}^{t_0+T}|x(t)|^2\diff t
	\quad\fourier\quad\frac{1}{T}\sum_{k=-\infty}^{+\infty}|c_k|^2.
\end{align}
\newpage
Alle Zeit-Funktionen in folgender Tabelle sind $T$-periodisch, mit zugehöriger
Grundschwingungs-Kreisfrequenz $\omega = \frac{2\pi}{T}$.\\
\begin{tabular}{|lcc|}
	\hline
	&&\\
	&\textbf{Komplexe Fourierreihe, Fourier Series (FS)}& \\
	\hline
	&&\\
	& $x(t)=\frac{1}{T}\sum\limits_{k=-\infty}^{+\infty}X_k\e^{+\im\omega k t} $
	& $X_k=\frac{1}{T}\int\limits_0^Tx(t)\e^{-\im\omega k t}\mathrm{d}t $ \\&&\\
	\hline
	&&\\
	\textbf{Eigenschaften} & & \\
	Linearität & $Af(t)+Bg(t)$ & $AF_k+BG_k$\\&& \\
	Symmetrien & $x(-t)$ & $X_{-k}$ \\ &&\\
	& $x^*(t)$ & $X^*_{-k}$ \\ &&\\
	\hline
	\textbf{Sätze} & & \\&& \\
	Faltung & $x(t)\circledast h(t)$ & $X_k\cdot H_k$ \\&& \\
	Multiplikation & $x(t)\cdot h(t)$ & $\frac{1}{T}X_k \ast H_k $ \\&& \\
	Zeitskalierung $(a\in\mathbb{R}\setminus\{0\})$& $x(a\cdot t)$ & $\begin{cases}
		X_k,a\cdot\omega, &a>0\\
		X_{-k},|a|\cdot\omega, &a<0
	\end{cases}$\\
	Verschiebung $(\tau \in \mathbb{R})$ & $x(t-\tau)$& $\e^{-\im\omega k \tau}X_k$\\&& \\
	Modulation $(n \in \mathbb{Z})$& $\e^{\im n \omega t}x(t)$ & $X_{k-n}$ \\&& \\
	Differentiation & $\frac{\mathrm{d}}{\mathrm{d}t}x(t)$ & $\im\omega k X_k$\\&& \\
	unbestimmtes Integral &$\int x(t)\mathrm{d}t $ & $\frac{1}{\im\omega k}X_k$\\&& \\
	Parselval`sches Theorem & $\int\limits_0^T|x(t)|^2\mathrm{d}t$ & $\frac{1}{T}\sum\limits_{k=-\infty}^{+\infty}|X_k|^2$\\&& \\
	\hline
\end{tabular}







%\renewcommand{\refname}{Buchzitate}
%\cite{*}
\clearpage
\bibliography{../tutorial_latex_deu/literatur}
\end{document}

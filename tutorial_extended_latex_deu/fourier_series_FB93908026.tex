\documentclass[11pt,a4paper,DIV=12]{scrartcl}
\usepackage{scrlayer-scrpage}
\usepackage[utf8]{inputenc}
\usepackage{fouriernc}
\usepackage[T1]{fontenc}
\usepackage[german]{babel}
\usepackage[hidelinks]{hyperref}
\usepackage{natbib}
\usepackage{url}
\usepackage{amsmath}
\usepackage{amsfonts}
\usepackage{amssymb}
\usepackage{trfsigns}
\usepackage{marvosym}
\usepackage{nicefrac}
\usepackage{graphicx}
\usepackage{subcaption}
\usepackage{xcolor}
\usepackage{comment}
\usepackage{mdframed}
\usepackage{tikz}
\usepackage{circuitikz}
\usepackage{pgfplots}
\usepackage{bm}
\usepackage{cancel}
\bibliographystyle{dinat}

\usepackage{../sig_sys_macros}
\newcommand{\sinc}{\mathrm{sinc}}

%------------------------------------------------------------------------------
\ohead{Signal- und Systemtheorie Übung}
\cfoot{\pagemark}
\ofoot{\tiny\url{https://github.com/spatialaudio/signals-and-systems-exercises}}

\begin{document}
%
\noindent Signal- und Systemtheorie Übung\footnote{This tutorial is provided as
Open Educational Resource (OER), to be found at
\url{https://github.com/spatialaudio/signals-and-systems-exercises}
accompanying the OER lecture
\url{https://github.com/spatialaudio/signals-and-systems-lecture}.
%
Both are licensed under a) the Creative Commons Attribution 4.0 International
License for text and graphics and b) the MIT License for source code.
%
Please attribute material from the tutorial as \textit{Frank Schultz,
Continuous- and Discrete-Time Signals and Systems - A Tutorial Featuring
Computational Examples, University of Rostock} with
\texttt{github URL, commit number and/or version tag, year, (file name and/or
content)}.}---Frank Schultz, Sascha Spors,
Institut für Nachrichtentechnik (INT),
Fakultät für Informatik und Elektrotechnik (IEF),
Universität Rostock \&
Robert Hauser, Universität Rostock---Sommersemester 2023, Version: \today\\
\verb|fourier_series_FB93908026|
%

\noindent Main author: https://github.com/robhau, check: TBD

\tableofcontents

\section{Übung: Fourier-Reihe}
In dieser Lerneinheit wollen wir uns tiefer mit Fourier-Reihen beschäftigen.
Zunächst werden wir mit Hilfe der reellen Fourier-Reihe einige periodische Signale
darstellen.
%
Danach schauen wir uns die komplexe Fourier-Reihe an und werden neben der
Darstellung periodischer Signale als komplexe Fourier-Reihe auch beobachten,
wie sich die Koeffizienten bspw. bei einer Zeitverschiebung verhalten.

Hinweis: Die Musterlösungen sind noch nicht vollständig gegengeprüft, vielleicht
sind noch ein paar kleine Typos drin, Feedback sehr willkommen.

\subsection{Aufgaben}
%
Berechnen Sie die Koeffizienten für die reelle sowie für die komplexe Fourierreihe für
%
%
\begin{itemize}
	%
	\item[a) ] $x_{a}(t)=
				\begin{cases}
					A &,|t|\leq\frac{T_h}{2}\\
					0 &,\text{sonst}
				\end{cases}$\\
				$0<\frac{T_h}{T}\leq 1,$
	%
	\item[b) ] $x_{b}(t)=
				\begin{cases}
					-A &,-\frac{T_h}{2}\leq t < 0\\
					A &,0 \leq t \leq +\frac{T_h}{2}\\
					0 &,\text{sonst}
				\end{cases}$\\
				$0 < \frac{T_h}{T} \leq 1$
	%
	\item[c) ] $x_c(t)=
				\begin{cases}
					\frac{2A}{T_h}t+A&,-\frac{T_h}{2}\leq t <0\\
					-\frac{2A}{T_h}t+A&,0\leq t \leq +\frac{T_h}{2}\\
					0&,\text{sonst}
				\end{cases}$\\
				$0<\frac{T_h}{T}\leq 1$
	%
\end{itemize}
%
%
Berechnen Sie die Koeffizienten für die komplexe Fourierreihe für
%
\begin{itemize}
	%
	\item[d) ] $x_d(t) = ax_{a}(t)+bx_{b}(t)$\\
				$a,b\in\mathbb{C}$
	%
	\item[e) ] $x_{e}(t)=
				\begin{cases}
					A, &0\leq t\leq T_h\\
					0, &\text{sonst}
				\end{cases}\\
				0< \frac{T}{T_h}\leq 1$
	%
	\item[f) ] $x_{f}(t)=\e^{+\im\omega_0 t}x_{a}(t)\\
			  	\omega_0 = n\cdot \omega \quad \quad n\in\mathbb{Z}$
	%
	\item[g) ] $x_{g}(t) =
				\begin{cases}
					A &, -\frac{T_h}{2}\leq t \leq 0 \\
					-A &, 0<t\leq \frac{T_h}{2}\\
					0 &,\text{ sonst}
				\end{cases}\\
				0<\frac{T_h}{T}\leq 1$
	%
	\item[h) ] $x_{h}(t) = x_{a}(t)\circledast x_{b}(t)$
	%
	\item[i) ]$x_{i}(t)=\frac{\diff x_{c}(t)}{\diff t}$
	%
	\item[j) ] $x_{j}(t)=x(at)\\
				a\in\mathbb{R}\setminus\{0\}$
	%
	\item[k) ]$x_{k}(t)=1$
	%
	\item[l) ] $x_{k}(t)=x_{b}(t)\cdot x_{c}(t)$
\end{itemize}
%
Skizzieren Sie für $T=4$, $T_h=2$, $A=1$, $B=\im$, $n= 2$, $a=\frac{1}{2}$ und $C=-\frac{A}{2}\frac{T_h}{2}$ (Integrationskonstante):
%
\begin{itemize}
	\item die Koeffizienten der reellen Fourierreihen über $0\leq k \leq 5$!
	\item Betrag und Phase der Koeffizienten der komplexen Fourierreihe über
	$-5\leq k \leq 5$!
\end{itemize}
%
\begin{itemize}
	\item[m) ] Untersuchen Sie den Zusammenhang der Fourierkoeffizienten einer
	integrierten Funktion!
	\item[n) ] Untersuchen Sie den Zusammenhang der Fourierkoeffizienten einer
	komplex konjugierten Funktion!
	\item[o) ] Berechnen Sie $\int_{t_0}^{t_0+T}|x_b(t)|^2\diff t$!
\end{itemize}
%
Die Definition einer Zeit-Funktion gilt immer für eine Periode $T$ und resultierende
Grundschwingung $\omega = \frac{2\pi}{T}$.


\subsection{Einleitung Reelle Fourierreihe}
%
Wir definieren eine periodische Funktion mit der Periode
%
$T\in\mathbb{R}^+\setminus\{0\}$
%
wie folgt:
%
\begin{align}
	x(t)=x(t+T).
\end{align}
%
Die reelle Fourierreihe der Funktion $x(t)$ mit der Periode $T$ und der
Grundschwingungs-Kreisfrequenz $\omega=\frac{2\pi}{T}$ hat die Form (bspw. ausgeschrieben in
\cite[Kap. 7, S. 474]{Bronstein2015}).
%
\begin{align}
	\label{eq:RealFourierSeries}
	x(t)=\frac{a_0}{2T}+\frac{1}{T}\sum_{k=1}^{+\infty}\big[ a_k\cos(k\omega t)+b_k\sin(k\omega t)\big]
\end{align}
%
Die Koeffizienten berechnen sich dabei durch folgende Formeln
(vgl. \cite[Kap. 7, S. 474]{Bronstein2015}).
%
\begin{gather}
	\label{eq:Koeff}
	a_k=2\int_{t_0}^{t_0+T}x(t)\cos(k\omega t)\diff t \\
	b_k=2\int_{t_0}^{t_0+T}x(t)\sin(k\omega t)\diff t
\end{gather}
%
Die Mittelungs-Gewichtung der Koeffizienten mit $\frac{1}{T}$ findet bei
unser hier benutzten Konvention in der
Synthese der Fourierreihe in \eqref{eq:RealFourierSeries} statt.
%
Dadurch lassen sich die Ergebnisse leichter mit anderen Werkzeugen
der Fourieranalyse vergleichen, bzw. sind konsistenter mit der
später kennenzulernenden Fourier Transformation (FT), Fouriertransformation für
zeitdiskrete Signale (DTFT), sowie Diskrete Fourier Transformation (DFT).
%

Der Koeffizient $a_0$ nimmt bei der Berechnung eine Sonderstellung ein, denn für
$k=0$ verschwindet die Cosinus-Schwingung aus dem Integranden und wir berechnen
den (doppelten) Gleichanteil der Funktion:
%
\begin{align}
	\label{eq:A0}
	a_0=2\int_{t_0}^{t_0+T}x(t)\diff t.
\end{align}
%
Fall es sich um ungerade oder gerade Funktionen handeln sollte, vereinfacht sich
die Berechnung der Koeffizienten (vgl. \cite[Kap.7, S. 476]{Bronstein2015}): (Achtung obere Integralgrenze)
%
\begin{itemize}
	\item Die Funktion ist gerade, d.h. $x(t)=x(-t)$:
\end{itemize}
%
\begin{gather}
	\label{eq:KoeffGerade}
	a_k=4\int_0^{\frac{T}{2}}x(t)\cos(k \omega t)\diff t,\\
	b_k=0.
\end{gather}
%
\begin{itemize}
	\item Die Funktion ist ungerade, d.h. $x(t)=-x(-t)$:
\end{itemize}
%
\begin{gather}
	\label{eq:KoeffUngerade}
	a_k = 0,\\
	b_k=4\int_0^{\frac{T}{2}}x(t)\sin(k \omega t)\diff t.
\end{gather}
%

\subsection{Einleitung komplexe Fourierreihe}
%
Wenn wir uns die Mathematik Vorlesungen in Erinnerung rufen, dann erinnern wir
uns bestimmt an die Eulersche Identität:
%
\begin{gather}
	\label{eq:EulerIdentity}
	\e^{\im z}=\cos(z)+\im\sin(z)\\
	\label{eq:ConjugatedEulerIdenity}
	\e^{-\im z}=\cos(z)-\im\sin(z)
\end{gather}
%
Mit dessen Hilfe lassen sich $\cos(\omega t)$ und $\sin(\omega t)$ auch in komplexer Form darstellen:
%
\begin{gather}
	\label{eq:CosComplex}
	\cos(\omega t)=\frac{1}{2}\left (\e^{+\im\omega t}+\e^{-\im\omega t}\right)\\
	\label{eq:SinComplex}
	\sin(\omega t)=\frac{1}{2\im}\left (\e^{+\im\omega t}-\e^{-\im\omega t}\right )
\end{gather}
%
Wenn wir nun die Gleichungen \eqref{eq:CosComplex} und \eqref{eq:SinComplex} in
\eqref{eq:RealFourierSeries} einsetzen, lässt sich die Fourierreihe auch anders
darstellen:
%
\begin{align}
	x(t)
	&=\frac{a_0}{2T}+\frac{1}{T}\sum_{k=1}^{+\infty}\left [a_k\cdot\frac{1}{2}\left (\e^{+\im k \omega t}+\e^{-\im k \omega t}\right)+b_k\cdot\frac{1}{2\im}\left (\e^{+\im k \omega t}-\e^{-\im k \omega t}\right )\right ]\\
	&=\frac{a_0}{2T}+\frac{1}{T}\sum_{k=1}^{+\infty}\left [\frac{1}{2}\left(a_k-\im b_k\right)\e^{+\im k \omega t}+\frac{1}{2}\left(a_k+\im b_k\right)\e^{-\im k \omega t}\right ].
\end{align}
%
Wenn wir nun die Terme vor den komplexen Zeigern zusammenfassen, ergibt sich folgende Darstellung:
%
\begin{align}
	x(t)=\frac{1}{T}\sum_{k=-\infty}^{+\infty}c_k\e^{+\im k \omega t}
\end{align}
%
mit
%
\begin{align}
	\label{eq:ComplexRealCoeffConnection}
	c_k=
		\begin{cases}
			\frac{a_0}{2} &,k=0 ,\\
			\frac{1}{2}(a_k-\im b_k) &,k>0,\\
			\frac{1}{2}(a_{|k|}+\im b_{|k|}) &,k<0.
		\end{cases}
\end{align}
%
Im Gegensatz zur reellen Schreibweise können die Koeffizienten nun auch komplex
sein, also $c_k\in\mathbb{C}$.
%
Wir nehmen zur Kenntnis, dass auch unser Zeitsignal durchaus auch
komplex-wertig dfiniert sein kann $x(t)\in\mathbb{C}$.
%
Komplex-wertiges Signal und komplex-wertige Fourerreihenkoeffizienten bilden
ein Transformationspaar.

%
Es stellt sich nun die Frage, ob man $c_k$ direkt berechnen kann oder den (evtl.
umständlichen) Weg über $a_k$ und $b_k$ gehen muss.

$c_0$ ergibt sich offensichtlich aus Gleichung \eqref{eq:A0}:
%
\begin{align}
	c_0=\int_{t_0}^{t_0+T}x(t)\diff t.
\end{align}
%
Nun schauen wir, ob sich auch für $k\neq0$ die Koeffizientenbestimmung
vereinfacht.
%
Betrachten wir nun den Fall $k>0$:
%
\begin{align}
	c_k&\underset{k>0}{=}\frac{1}{2}(a_k-\im b_k)
	=\int_{t_0}^{t_0+T}x(t)\cos(k \omega t)\diff t-\im\int_{t_0}^{t_0+T}x(t)\sin(k \omega t)\diff t\nonumber\\
	&=\int_{t_0}^{t_0+T}x(t)\left (\cos(k\omega t)-\im\sin(k\omega t) \right )\diff t\nonumber\\
	&\underset{\eqref{eq:ConjugatedEulerIdenity}}{=}\int_{t_0}^{t_0+T}x(t)\e^{-\im k \omega t}\diff t.
\end{align}
%
Zum Schluss berechnen wir noch die Koeffizienten $c_k$ für $k<0$:
%
\begin{align}
	\underset{k<0}{=}\frac{1}{2}(a_{|k|}+\im b_{|k|})
	=\int_{t_0}^{t_0+T}x(t)\cos(|k|\omega)\diff t+\im\int_{t_0}^{t_0+T}x(t)\sin(|k| \omega t)\diff t.
\end{align}
%
$\cos(|k| \omega t)$ ist eine gerade Funktion, also können wir die Betragsstriche
weglassen.
%
$\sin(|k| \omega t)$ ist eine ungerade Funktion. Wir können die Betragsstriche
weglassen, wenn wir die Funktion mit $-1$ multiplizieren:
%
\begin{align}
	c_k&\underset{k<0}{=}\int_{t_0}^{t_0+T}x(t)\cos(|k|\omega)\diff t+\im\int_{t_0}^{t_0+T}x(t)\sin(|k| \omega t)\diff t\nonumber\\
	&=\int_{t_0}^{t_0+T}x(t)\cos(k\omega)\diff t-\im\int_{t_0}^{t_0+T}x(t)\sin(k \omega t)\diff t\nonumber\\
	&=\int_{t_0}^{t_0+T}x(t)\left (\cos(k\omega t)-\im\sin(k\omega t)\right )\diff t\nonumber\\
	&=\int_{t_0}^{t_0+T}x(t)\e^{-\im k \omega t}\diff t.
\end{align}
%
Für $k<0$ und $k>0$ gleichen sich die Berechnungsvorschriften. Für $k=0$ würde
$\e^{-\im k \omega t}=1$ ergeben, weshalb wir allgemein folgende Gleichung
aufstellen können:
%
\begin{align}
	c_k=\int_{t_0}^{t_0+T}x(t)\e^{-\im k \omega t}\diff t.
\end{align}
%



\newpage
\subsection{Lösungen Reelle Fourierreihe}

\subsubsection{Aufgabe a) mit reeller Fourierreihe}
\label{sec:TaskAReal}
%
Wir wollen zunächst die Koeffizienten der reellen Fourierreihe der
$T$-periodischen Funktion $x_a(t)$ berechnen.
\begin{gather}
	x_{a}(t)=
		\begin{cases}
			A &,|t| \leq \frac{T_h}{2} \\
			0 &,\text{ sonst}
		\end{cases}\\
	t\in[-\frac{T}{2},+\frac{T}{2}]\nonumber
\end{gather}
%
\begin{figure}[h!]
	\centering
	\begin{tikzpicture}[domain=0:2]
		\def\T{0.4}
		\draw[->] (-3,0) -- (3,0) node[below right] {$t$};
		\draw[->] (0,0) -- (0,1.5) node[above] {$x_{a}(t)$};
		\foreach \pos in {-2,...,2} {
			\draw[-, C0, ultra thick] (\pos-\T,0) -- (\pos-\T,1) -- (\pos+\T,1)
			-- (\pos+\T,0) -- (\pos+1-\T,0);
		};
		\draw[-, C0, ultra thick] (-\T,0) node[below] {$\frac{-T_h}{2}$};
		\draw[-, C0, ultra thick] (+\T,0) node[below] {$\frac{+T_h}{2}$};
		\draw[-, C0, ultra thick] (1,0) node[below] {$T$};
		\draw[-, C0, ultra thick] (2,0) node[below] {$2 T$};
		\draw[-, C0, ultra thick] (-2-\T,1) node[left] {$A$};
	\end{tikzpicture}
\end{figure}
%
Wir sehen, dass die Funktion gerade ist, also dass $x_{a}(t)=x_{a}(-t)$.
Wir können uns also die Berechnung von $b_k$ sparen und brauchen nur $a_k$ betrachten.

Zunächst berechnen wir den Gleichanteil:
\begin{align}
	\frac{a_0} {2} = \int_{-\frac{T}{2}}^{+\frac{T}{2}}x_{a}(t)\diff t =
	\int_{-\frac{T_h}{2}}^{+\frac{T_h}{2}}A\diff t
	=A t \Bigg |_{t=-\frac{T_h}{2}}^{t=\frac{T_h}{2}}
	=AT_h
\end{align}
%
Danach folgt die Berechnung der übrigen Koeffizienten $a_k$.
%
Wir nutzen die vereinfachte Berechnungsmethode aus \ref{eq:KoeffGerade}:
%
\begin{gather}
	a_k=4\int_0^{\frac{T}{2}}x_{a}(t)\cos(k\omega t)\diff t
	=4\int_0^{\frac{T_h}{2}}A\cos\left(k\frac{2\pi}{T} t\right)\diff t
	=4A\frac{T}{2\pi k}\sin\left(k\frac{2\pi}{T} t\right)\Bigg |_{t=0}^{t=\frac{T_h}{2}}\\\nonumber
	=2A\frac{T}{\pi k}\frac{T_h}{T_h}\sin\left(k\pi \frac{T_h}{T}\right)
	=2AT_h\sinc\left(k\pi\frac{T_h}{T} \right).
\end{gather}
%
$x_{a}(t)$ als reelle Fourierreihe geschrieben lautet also:
%
\begin{align}
	x_{a}(t)=A\cdot\frac{T_h}{T}+\frac{1}{T}\sum_{k=1}^{+\infty}2AT_h\sinc\left(k\pi\frac{T_h}{T}\right)\cos(k\omega t).
\end{align}
%
\begin{figure}
	\centering
	\begin{subfigure}{\textwidth}
		\includegraphics[width=\textwidth,keepaspectratio]{../fs/FB93908026_timedomain_0}
		\caption{Zeitbereich}
	\end{subfigure}
	\begin{subfigure}{\textwidth}
		\includegraphics[width=\textwidth,keepaspectratio]{../fs/FB93908026_frequence_0}
		\caption{Fourierkoeffizienten}
	\end{subfigure}
	\caption{Synthese der Fourierreihe von $x_a(t)$ für $A=1$, $T=4$, $T_h=2$ und
		$0\leq k\leq5$.}
\end{figure}
\newpage
\subsubsection{Aufgabe b) mit reeller Fourierreihe}
%
Nun wollen wir die Koeffizienten der Fourierreihe folgender für
eine Periode $T$ definierter Funktion berechnen:
%
%
\begin{gather}
	x_{b}(t)=
		\begin{cases}
			-A &,-\frac{T_h}{2}\leq t < 0\\
			A &,0\leq t \leq +\frac{T_h}{2}\\
			0 &,\text{ sonst}
		\end{cases}\\
	t \in [-\frac{T}{2},+\frac{T}{2}] \nonumber
\end{gather}
%
%
%
\begin{figure}[h!]
	\centering
	\begin{tikzpicture}[domain=0:2]
		\def\T{0.4}
		\draw[->] (-3,0) -- (3,0) node[below right] {$t$};
		\draw[->] (0,0) -- (0,1.5) node[above] {$x_{b}(t)$};
		\foreach \pos in {-2,...,2} {
			\draw[-, red, ultra thick] (\pos-\T,0) -- (\pos-\T,-1)--(\pos,-1)--(\pos,1) -- (\pos+\T,1) -- (\pos+\T,0) -- (\pos+1-\T,0);
		};
		\draw[-, black, ultra thick] (-\T,0) node[below] {$\frac{-T_h}{2}$};
		\draw[-, black, ultra thick] (+\T,0) node[below] {$\frac{+T_h}{2}$};
		\draw[-, black, ultra thick] (1,0) node[below] {$T$};
		\draw[-, black, ultra thick] (2,0) node[below] {$2 T$};
		\draw[-, black, ultra thick] (-2-\T,1) node[left] {$A$};
	\end{tikzpicture}
\end{figure}
%
Wir sehen, dass die Funktion ungerade ist, also $x_{b}(t)=-x_{b}(-t)$.
%
Daher können wir die Regel aus \ref{eq:KoeffUngerade} verwenden:
%
\begin{align}
	b_k&=4\int_0^{\frac{T}{2}}x_{b}(t)\sin\left(k\frac{2\pi}{T} t\right)\diff t
	=4\int_0^{\frac{T_h}{2}}A\sin\left(k\frac{2\pi}{T} t\right)\diff t\\
	&=-4A\frac{T}{2\pi k}\cos\left(k\frac{2\pi}{T} t\right)\Bigg |_0^{\frac{T_h}{2}}
	=2A\frac{T}{\pi k}\left(1-\cos\left(\pi k \frac{T_h}{T}\right)\right).
\end{align}
%
Es gilt (vgl. \cite[Kap. 2, S. 82]{Bronstein2015}):
%
\begin{gather}
	\label{eq:sinsquaredcosinus}
	\sin^2\left(\frac{\alpha}{2}\right)=\frac{1}{2}(1-\cos(\alpha)).
\end{gather}
%
Somit kann $b_k$ noch zusammengefasst werden.
\begin{gather}
	2A\frac{T}{\pi k}\frac{2}{2}\left(1-\cos\left(\pi k\frac{T_h}{T}\right)\right)
	=4A\frac{T^2}{\pi^2 k^2}\frac{\pi k}{T}\frac{T_h^2}{T_h^2}\sin^2\left(\pi k \frac{T_h}{2T}\right)
	=\frac{A\pi kT_h^2}{T}\sinc^2\left(\pi k\frac{T_h}{2T}\right)
\end{gather}
%
$x_{b}(t)$ als reelle Fourierreihe dargestellt sieht also folgendermaßen aus:
\begin{align}
	x_{b}(t)=\frac{1}{T}\sum_{k=1}^{+\infty}\frac{A\pi kT_h^2}{T}\sinc^2\left(\pi k\frac{T_h}{2T}\right)\sin(k\omega t).
\end{align}
\begin{figure}
	\centering
	\begin{subfigure}{\textwidth}
		\includegraphics[width=\textwidth,keepaspectratio]{../fs/FB93908026_timedomain_1}
		\caption{Zeitbereich}
	\end{subfigure}
	\begin{subfigure}{\textwidth}
		\includegraphics[width=\textwidth,keepaspectratio]{../fs/FB93908026_frequence_1}
		\caption{Fourierkoeffizienten}
	\end{subfigure}
	\caption{Synthese der Fourierreihe von $x_b(t)$ für $A=1$, $T=4$, $T_h=2$ und $0\leq k\leq 5$.}
\end{figure}
\newpage
\subsubsection{Aufgabe c) mit reeller Fourierreihe}
%
Als letzte Aufgabe im Bereich der reellen Fourierkoeffizienten wollen wir uns
die Dreieckschwingung mit Spalt angucken.
%
\begin{figure}[h!]
	\centering
	\begin{tikzpicture}[domain=0:2]
		\def\T{0.4}
		\draw[->] (-3,0) -- (3,0) node[below right] {$t$};
		\draw[->] (0,0) -- (0,1.5) node[above] {$x_{c}(t)$};
		\foreach \pos in {-2,...,2} {
			\draw[-, C0, ultra thick] (\pos-\T,0) -- (\pos,1) -- (\pos+\T,0) -- (\pos+1-\T,0);
		};
		\draw[-, C0, ultra thick] (-\T,0) node[below] {$\frac{-T_h}{2}$};
		\draw[-, C0, ultra thick] (+\T,0) node[below] {$\frac{+T_h}{2}$};
		\draw[-, C0, ultra thick] (1,0) node[below] {$T$};
		\draw[-, C0, ultra thick] (2,0) node[below] {$2 T$};
		\draw[-, C0, ultra thick] (-2-\T,1) node[left] {$A$};
	\end{tikzpicture}
\end{figure}
%
Es handelt sich um eine gerade Funktion, also $x_{c}(t)=x_{c}(-t)$.
%
Wir können also die Berechnung mit Hilfe von \eqref{eq:KoeffGerade} vereinfachen.
%
%
%
\begin{align}
	\frac{a_0}{2}
	&=\int_{t_0}^{t_0+T}x_{c}(t)\diff t
	=2\int_0^{+\frac{T_h}{2}}\left(-\frac{2A}{T_h}t+A\right)\diff t
	=2\left [-\frac{At^2}{T_h}\Bigg |_{t=0}^{t=+\frac{T_h}{2}}+At\Bigg |_{t=0}^{t=+\frac{T_h}{2}}\right ]\nonumber \\
	&=2\left (-\frac{AT_h^2}{4T_h}+\frac{AT_h}{2}\right )
	=2\left (\frac{AT_h}{4}+\frac{AT_h}{2}\right )
	=2\frac{AT_h}{4}
	=\frac{AT_h}{2}
\end{align}
%
%
%
\begin{align}
	a_k
	&=2\int_{t_0}^{t_0+T}x_{c}(t)\cos(k\omega t)\diff t
	=4\int_0^{+\frac{T_h}{2}}\bigg (-\frac{2A}{T_h}t+A\bigg )\cos(k\omega t)\diff t\nonumber \\
	&=4\left [\frac{ \left( -\frac{2A}{T_h}t+A\right) }{k\omega}\sin(k\omega t)\Bigg |_{t=0}^{t=+\frac{T_h}{2}}-\int_0^{+\frac{T_h}{2}}-\frac{2A}{T_hk\omega}\sin(k\omega t)\diff t\right ]\nonumber \\
	&=4\left [\frac{\left( -\frac{2A}{T_h}\frac{T_h}{2}+A\right) }{k\omega}\sin\left( k\omega \frac{T_h}{2}\right) -0-\frac{2A}{T_hk^2\omega^2}\cos(k\omega t)\diff t\Bigg |_{t=0}^{t=+\frac{T_h}{2}}\right ]\nonumber \\
	&=4\left [\frac{2AT^2}{T_hk^24\pi^2}\left (1-\cos\left(k\pi\frac{T_h}{T}\right )\right)\right ]
	=\frac{2AT^2}{T_hk^2\pi^2}\left (1-\cos\left(k\pi\frac{T_h}{T}\right)\right )
\end{align}
%
%
%
Es gilt (vgl. \cite[Kap. 2, S. 82]{Bronstein2015}):
%
\begin{gather}
	\sin^2\left(\frac{\alpha}{2}\right)=\frac{1}{2}(1-\cos(\alpha)).
\end{gather}
%
Somit kann $a_k$ vereinfacht werden.
\begin{gather}
	\frac{2AT^2}{T_hk^2\pi^2}\frac{2}{2}\left (1-\cos\left(k\pi\frac{T_h}{T}\right)\right )
	=\frac{4AT^2}{T_hk^2\pi^2}\frac{T_h}{T_h}\sin^2\left(k\pi\frac{T_h}{2T}\right)
	=AT_h\sinc^2\left(k\pi\frac{T_h}{2T}\right)
\end{gather}
%
Zum Schluss stellen wir $x_{c}(t)$ noch als Fourierreihe dar:
%
\begin{align}
	x_{c}(t)
	=A\frac{T_h}{2T}+\frac{1}{T}\sum_{k=1}^{+\infty}AT_h\sinc^2\left(k\pi\frac{T_h}{2T}\right).
\end{align}
%
\begin{figure}
	\centering
	\begin{subfigure}{\textwidth}
		\includegraphics[width=\textwidth,keepaspectratio]{../fs/FB93908026_timedomain_2}
		\caption{Zeitbereich}
	\end{subfigure}
	\begin{subfigure}{\textwidth}
		\includegraphics[width=\textwidth,keepaspectratio]{../fs/FB93908026_frequence_2}
		\caption{Fourierkoeffizienten}
	\end{subfigure}
	\caption{Synthese der Fourierreihe von $x_c(t)$ für $A=1$, $T=4$, $T_h=2$ und
		$0\leq k\leq5$.}
\end{figure}

\newpage
\subsection{Lösungen Komplexe Fourierreihe}

\subsubsection{Aufgabe a) mit komplexer Fourierreihe}
Nun wollen wir die Koeffizienten der komplexen Fourierreihe zu Aufgabe a)
berechnen. Den Gleichanteil $c_0$ kennen wir schon aus \ref{sec:TaskAReal}.
%
Wir müssen also nur noch die Koeffizienten $c_k$ für $k\neq 0$ bestimmen.
%
\begin{align}
	c_k&=\int_{t_0}^{t_0+T}x_{a}(t)\e^{-\im k \omega t}\diff t
	=\int_{-\frac{T}{2}}^{+\frac{T}{2}}x_a(t)\e^{-\im k \frac{2\pi}{T}t}\diff t\nonumber\\
	&=\int_{-\frac{T_h}{2}}^{+\frac{T_h}{2}}A\e^{-\im k \frac{2\pi}{T}t}\diff t
	=A\frac{T}{-\im 2\pi k}\e^{-\im k \frac{2\pi}{T}t}\Bigg |_{t=-\frac{T_h}{2}}^{t=+\frac{T_h}{2}}\nonumber\\
	&=A\frac{T}{\pi k}\cdot\frac{1}{2\im}\left (\e^{\im k \pi \frac{T_h}{T}}-\e^{-\im k \pi \frac{T_h}{T}}\right )
	=A\frac{T}{\pi k}\frac{T_h}{T_h}\sin\left(k\pi\frac{T_h}{T}\right)\nonumber\\
	&=AT_h\sinc\left(k\pi\frac{T_h}{T}\right)
\end{align}
%
$\sinc(k)$ konvergiert für $k\rightarrow0$ gegen 1.
%
Daher können $c_0$ und $c_{k\neq0}$ zusammengefasst werden.
%
Zum Schluss schreiben wir $x_{a}(t)$ noch einmal als komplexe Fourierreihe.
%
\begin{align}
	x_{a}(t)=\frac{1}{T}\sum_{k=-\infty}^{+\infty}AT_h\sinc\left(k\pi\frac{T_h}{T}\right)\e^{+\im k \omega t}
\end{align}
%

%
\begin{figure}
	\centering
	\begin{subfigure}{\textwidth}
		\includegraphics[width=\textwidth,keepaspectratio]{../fs/FB93908026_timedomain_3}
		\caption{Zeitbereich}
	\end{subfigure}
	\begin{subfigure}{\textwidth}
		\includegraphics[width=\textwidth,keepaspectratio]{../fs/FB93908026_frequence_3}
		\caption{Betrag und Phase der Fourierkoeffizienten}
	\end{subfigure}
	\caption{Synthese der Fourierreihe von $x_a(t)$ für $A=1$, $T=4$, $T_h=2$ und
		$-5\leq k\leq5$.}
\end{figure}
\newpage
\subsubsection{Aufgabe b) mit komplexer Fourierreihe}
Als nächstes stehen die Koeffizienten der komplexen Fourierreihe zu Aufgabe b) auf dem Plan.
%
Der Gleichanteil ist uns schon aus der reellen Fourierreihe bekannt, nämlich $0$. %
Wir können das aber auch aus der Graphik ableiten, denn die Flächen oberhalb und
unterhalb der Zeitachse innerhalb einer Periode heben sich auf.
%
Wir müssen also nur noch die Koeffizienten $c_k$ für $k\neq 0$ bestimmen:
%
%
%
\begin{align}
	c_k&=\int_{t_0}^{t_0+T}x_{b}(t)\e^{-\im k \omega t}\diff t
	=\int_{-\frac{T}{2}}^{+\frac{T}{2}}x_{b}(t)\e^{-\im k \omega t}\diff t \nonumber \\
	%
	&= \int_{-\frac{T_h}{2}}^{0}(-A)\e^{-\im k \omega t}\diff t+\int_{0}^{+\frac{T_h}{2}}A\e^{-\im k \omega t}\diff t  \nonumber \\
	%
	&= \frac{-A}{-\im k \omega}\e^{-\im k \omega t}\Bigg |_{t=-\frac{T_h}{2}}^{t=0}+\frac{A}{-\im k \omega}\e^{-\im k \omega t}\Bigg |_{t=0}^{t=+\frac{T_h}{2}} \nonumber \\
	%
	&= \frac{AT}{-\im k 2\pi }\left (\e^{\im k \frac{2\pi}{T}\cdot \frac{T_h}{2}}-1\right )+\frac{AT}{-\im k 2\pi}\left (\e^{-\im\frac{2\pi}{T}\cdot\frac{T_h}{2}}-1\right )\nonumber \\
	%
	&=\frac{AT}{\im k 2 \pi}\left (2 -\e^{\im\pi\frac{T_h}{T}}-\e^{-\im\pi\frac{T_h}{T}} \right )\nonumber \\
	%
	\label{eq:CKTaskBComplex}
	&=\frac{AT}{\im k \pi}\left ( 1-\cos(k\pi\frac{T_h}{T})\right ).
\end{align}
%
%
%
Mit Gleichung \ref{eq:sinsquaredcosinus} lässt sich der Term noch
vereinfachen.
%
\begin{gather}
	\frac{AT}{\im k \pi}\frac{2}{2}\left ( 1-\cos(k\pi\frac{T_h}{T})\right )
	=2\frac{AT}{\im k\pi}\frac{T_h^2}{T_h^2}\frac{T}{T}\frac{2k\pi}{2k\pi}\sin^2\left(k\pi\frac{T_h}{2T}\right)
	= Ak\pi\frac{T_h^2}{2T} \sinc^2\left(k\pi\frac{T_h}{2T}\right)\e^{-\im\frac{\pi}{2}}
\end{gather}
%
Für $k=0$ ist oberer Term gleich $0$, somit stellt er auch eine Beschreibung von
$c_k$ für $k=0$ da.
%
Die komplexe Fourierreihe hat also folgende Form:
\begin{align}
	x_{b}(t)=\frac{1}{T}\sum_{k=-\infty}^{+\infty}Ak\pi\frac{T_h^2}{2T} \sinc^2\left(k\pi\frac{T_h}{2T}\right)\e^{-\im\frac{\pi}{2}}\e^{\im k \omega t}
\end{align}
%
%
\begin{figure}
	\centering
	\begin{subfigure}{\textwidth}
		\includegraphics[width=\textwidth,keepaspectratio]{../fs/FB93908026_timedomain_4}
		\caption{Zeitbereich}
	\end{subfigure}
	\begin{subfigure}{\textwidth}
		\includegraphics[width=\textwidth,keepaspectratio]{../fs/FB93908026_frequence_4}
		\caption{Betrag und Phase der Fourierkoeffizienten}
	\end{subfigure}
	\caption{Synthese der Fourierreihe von $x_b(t)$ für $A=1$, $T=4$, $T_h=2$ und
		$-5\leq k\leq5$.}
\end{figure}
%
%
\newpage
\subsubsection{Aufgabe c) mit komplexer Fourierreihe}
%
Auch hier haben wir den Gleichanteil ($c_0=\frac{a_0}{2}$) schon berechnet,
es fehlen nur noch die Koeffizienten $c_k$ für $k\neq 0$:
%
%
%
\begin{align}
	c_k&=\int_{t_0}^{t_0+T}x_{c}(t)\e^{-\im k \omega t}\diff t
	=\int_{-\frac{T_h}{2}}^0\left (\frac{2A}{T_h}t+A\right )\e^{-\im k \omega t}\diff t+\int_0^{+\frac{T_h}{2}}\left (-\frac{2A}{T_h}t+A\right )\e^{-\im k \omega t}\diff t\nonumber \\
	&= \frac{\frac{2A}{T_h}t+A}{-\im k \omega}\e^{-\im k \omega}\Bigg|_{t=-\frac{T_h}{2}}^{t=0}+\int_{-\frac{T_h}{2}}^0\frac{2A}{T_h\im k\omega}\e^{-\im k \omega t}\diff t+\frac{-\frac{2A}{T_h}t+A}{-\im k \omega}\e^{-\im k \omega t}\Bigg |_{t=0}^{t=+\frac{T_h}{2}}-\int_0^{+\frac{T_h}{2}}\frac{2A}{T_h\im k\omega}\e^{-\im k \omega t}\diff t\nonumber \\
	&=-\frac{A}{\im k \omega}+0+\frac{2A}{T_h k^2\omega^2}-\frac{2A}{T_hk^2\omega^2}\e^{+\im k \omega \frac{T_h}{2}}+\frac{A}{\im k \omega}-\frac{2A}{T_hk^2\omega^2}\e^{-\im k \omega \frac{T_h}{2}}+\frac{A}{k^2\omega^2}\nonumber \\
	&=\frac{4AT^2}{T_hk^24\pi^2}-\frac{2AT^2}{T_hk^24\pi^2}\left (\e^{+\im k \frac{2\pi}{T}\frac{T_h}{2}}+\e^{-\im k\frac{2\pi}{T}\frac{T_h}{T}}\right )
	=A\frac{T^2}{T_hk^2\pi^2}\left (1-\cos(k\pi\frac{T_h}{T})\right ).
\end{align}
%
%
%
Wie fast schon gewohnt können wir den Koeffizienten noch vereinfachen
mit Hilfe von Gleichung \ref{eq:sinsquaredcosinus}.
%
%
\begin{gather}
	c_k=A\frac{T^2}{T_hk^2\pi^2}\frac{2}{2}\left (1-\cos(k\pi\frac{T_h}{T})\right )
	=2A\frac{T^2}{T_hk^2\pi^2}\frac{2T_h}{2T_h}\sin\left(k\pi\frac{T_h}{2T}\right)
	=\frac{AT_h}{2}\sinc^2\left(k\pi\frac{T_h}{2T}\right)
\end{gather}
%
%
Für $k\rightarrow0$ konvergiert die $\sinc$-Funktion gegen $1$.
%
Somit kann $c_0$ auch mit dem oberen Term beschrieben werden.
%
Die komplexe Fourierreihe von $x_{c}(t)$ hat folgende Form:
%
%
\begin{align}
	x_{c}(t)=\frac{1}{T}\sum_{k=-\infty}^{+\infty}\frac{AT_h}{2}\sinc^2\left(k\pi\frac{T_h}{2T}\right)\e^{+\im k \omega t}
\end{align}
%
%
\begin{figure}
	\centering
	\begin{subfigure}{\textwidth}
		\includegraphics[width=\textwidth,keepaspectratio]{../fs/FB93908026_timedomain_5}
		\caption{Zeitbereich}
	\end{subfigure}
	\begin{subfigure}{\textwidth}
		\includegraphics[width=\textwidth,keepaspectratio]{../fs/FB93908026_frequence_5}
		\caption{Betrag und Phase der Fourierkoeffizienten}
	\end{subfigure}
	\caption{Synthese der Fourierreihe von $x_c(t)$ für $A=1$, $T=4$, $T_h=2$ und
		$-5\leq k\leq5$.}
\end{figure}
\newpage
\subsubsection{Aufgabe d) mit komplexer Fourierreihe}
%
In diesem Abschnitt beschäftigen wir uns mit einer gewichteten Superposition zweier
periodischer Funktionen.
%
%
\begin{align}
	x_d=
		\begin{cases}
			A(a-b)&,-\frac{T_h}{2}\leq t \leq 0 \\
			A(a+b)&,0\leq t \leq +\frac{T_h}{2}\\
			0&,\text{ sonst}
		\end{cases}
\end{align}
%
%
\begin{align}
	c_0&=\int_{t_0}^{t_0+T}x_d(t)\diff t
	=\int_{-\frac{T_h}{2}}^0A(a-b)\diff t+\int_0^{+\frac{T_h}{2}}A(a+b)\diff t\nonumber \\
	&=A(a-b)t\Bigg |_{t=-\frac{T_h}{2}}^{t=0}+A(a+b)\Bigg |_{t=0}^{t=+\frac{T_h}{2}}
	=A(a-b)\frac{T_h}{2}+A(a+b)\frac{T_h}{2}=a\cdot A\frac{T_h}{T}
\end{align}
%
%
%
\begin{align}
	c_k&=\int_{t_0}^{t_0+T}x_{\diff (t)\diff t}
	=\frac{1}{T}\Bigg [\int_{-\frac{T_h}{2}}^0A(a-b)\e^{-\im k \omega t}\diff t+\int_0^{+\frac{T_h}{2}}A(a+b)\e^{-\im k \omega t}\diff t\Bigg ]\nonumber \\
	&=\frac{A(a-b)}{-\im k \omega}\e^{-\im k \omega t}\Bigg |_{t=-\frac{T_h}{2}}^{t=0}+\frac{A(a+b)}{-\im k \omega}\e^{-\im k \omega t}\Bigg |_{t=0}^{t=+\frac{T_h}{2}}\nonumber \\
	&=\frac{A(a-b)}{\im k \omega}\e^{+\im k \omega \frac{T_h}{2}}-\frac{A(a-b)}{\im k \omega}+\frac{A(a+b)}{\im k \omega}-\frac{A(a+b)}{\im k \omega}\e^{-\im k \omega \frac{T_h}{2}}\nonumber \\
	&=A\frac{T}{\im k 2\pi}\left [-a+b+a+b+a\left (\e^{+\im k \frac{2\pi}{T}\frac{T_h}{2}}-\e^{-\im k \frac{2\pi}{T}\frac{T_h}{2}}\right )-b\left (\e^{+\im k \frac{2\pi}{T}}-\e^{-\im k \frac{2\pi}{T}\frac{T_h}{2}}\right )\right ]\nonumber \\
	&=A\frac{T}{\im k\pi}\left [a\im\sin\left(k\pi\frac{T_h}{T}\right)+b\frac{2}{2}\left(1-\cos\left(k\pi\frac{T_h}{T}\right)\right)\right ]\nonumber\\
	&\underset{\ref{eq:sinsquaredcosinus}}{=}A\frac{T}{\im k\pi}\left[a\im\sin\left(k\pi\frac{T_h}{T}\right)+2b\sin^2\left(k\pi\frac{T_h}{2T}\right)\right]\nonumber\\
	&=Aa\frac{T}{k\pi}\frac{T_h}{T_h}\sin\left(k\pi\frac{T_h}{T}\right)+2Ab\frac{T}{\im k\pi}\frac{2k\pi T_h^2T}{2k\pi T_h^2T}\sinc\left(k\pi\frac{T_h}{2T}\right)\nonumber\\
	&=AaT_h\sinc\left(k\pi\frac{T_h}{T}\right)+Abk\pi\frac{T_h^2}{2T}\sinc^2\left(k\pi\frac{T_h}{2T}\right)\e^{-\im\frac{\pi}{2}}
\end{align}
%
%
%
Die Fourierreihe lautet
%
\begin{gather}
	x_d(t)=\frac{1}{T}\sum_{k=-\infty}^{+\infty}\left[AaT_h\sinc\left(k\pi\frac{T_h}{T}\right)+Abk\pi\frac{T_h^2}{2T}\sinc^2\left(k\pi\frac{T_h}{2T}\right)\e^{-\im\frac{\pi}{2}}\right]\e^{+\im k \omega t}
\end{gather}
%
Bei genauerer Betrachtung erkennt man, dass es sich hierbei um die mit $a$ bzw. $b$
multiplizierte Addition der Koeffizienten von $x_a(t)$ und $x_{b(t)}$ handelt.
%
Wir prüfen, ob dies auch immer so stimmt.
%
\begin{align}
	c_k=\int_{t_0}^{t_0+T}\left (a\cdot x(t)+b\cdot y(t)\right )\e^{-\im k \omega t}\diff t
\end{align}
%
Wir können nun die Linearität des Integrals anwenden sowie Konstanten nach vorne ziehen:
\begin{align}
	c_k=a\int_{t_0}^{t_0+T}x(t)\e^{-\im k \omega t}\diff t+b\int_{t_0}^{t_0+T}y(t)\e^{-\im k \omega t}\diff t=aX_k+bY_k.
\end{align}
%
Wir merken uns also folgende Korrespondenz:
%
\begin{align}
	Ax(t)+By(t)\quad\fourier\quad AX_k+BY_k
\end{align}
%
%
\begin{figure}
	\centering
	\begin{subfigure}{\textwidth}
		\includegraphics[width=\textwidth,keepaspectratio]{../fs/FB93908026_timedomain_6}
		\caption{Zeitbereich}
	\end{subfigure}
	\begin{subfigure}{\textwidth}
		\includegraphics[width=\textwidth,keepaspectratio]{../fs/FB93908026_frequence_6}
		\caption{Betrag und Phase der Fourierkoeffizienten}
	\end{subfigure}
	\caption{Synthese der Fourierreihe von $x_d(t)$ für $A=1$, $T=4$, $T_h=2$ und
		$-5\leq k\leq5$.}
\end{figure}
\newpage
\subsubsection{Aufgabe e) mit komplexer Fourierreihe}
%
Nun wollen wir die Koeffizienten der komplexen Fourierreihe von
%
\begin{gather}
	x_{e}(t)=
		\begin{cases}
			A &,0\leq t \leq T_h \\
			0 &,\text{ sonst}
		\end{cases}\\
		t \in [0,T]\nonumber
\end{gather}
berechnen, wobei $0 < \frac{T_h}{T} \leq 1$ gelten soll, beschäftigen.\\
Folgend ist es noch einmal als Graphik dargestellt:
%
%
\begin{figure}[h!]
	\centering
	\begin{tikzpicture}[domain=0:2]
		\def\T{0.4}
		\draw[->] (-3,0) -- (3,0) node[below right] {$t$};
		\draw[->] (0,0) -- (0,1.5) node[above] {$x_{e}(t)$};
		\foreach \pos in {-2,...,2} {
			\draw[-, C0, ultra thick] (\pos,0) -- (\pos,1) -- (\pos+2*\T,1) -- (\pos+2*\T,0) -- (\pos+1,0);
		};
		%\draw[-, C0, ultra thick] (-1+2*\T,0) node[below] {$-T+T_h$};
		\draw[-, C0, ultra thick] (2*\T,-0.5) node[below] {$T_h$};
		\draw[-, C0, ultra thick] (1,0) node[below] {$T$};
		\draw[-, C0, ultra thick] (2,0) node[below] {$2 T$};
		\draw[-, C0, ultra thick] (-2-\T,1) node[left] {$A$};
	\end{tikzpicture}
\end{figure}
%
Wir erkennen schnell, dass dies die Funktion aus Aufgabe a) ist, jedoch um
$\frac{T_h}{2}$ nach rechts verschoben.
%
Das bedeutet, dass $x_{e}(t) = x_{a}(t-\frac{T_h}{2})$.
%
Die Koeffizienten würden sich an dieser Stelle ziemlich leicht berechnen,
indem man das Integrationsintervall klug von $0$ bis $T_h$ wählen würde.
%
Wir wollen es uns aber an dieser Stelle nicht ganz so einfach machen und
zunächst den allgemeinen verschobene Fall, also $x_{a}(t-\tau)$, berechnen
und dann $\frac{T_h}{2}$ für $\tau$ einsetzen, um $x_{e}(t)$ zu erhalten.
%
Für das Lösen der Aufgabe wäre natürlich die direkte Berechnung der
Koeffizienten für $x_{e}(t)$ ausreichend.
%
%
\begin{gather}
	x_{a}(t-\tau)=
		\begin{cases}
			A &,-\frac{T_h}{2}+\tau \leq t \leq +\frac{T_h}{2}+\tau \\
			0 &,\text{sonst}
		\end{cases}\\
		t \in [-\frac{T}{2}+\tau,+\frac{T}{2}+\tau]\nonumber
\end{gather}
%
%
Wir schauen uns noch einmal die Graphik von $x_{a}(t)$ an, bloß nun um $\tau$
nach rechts verschoben.
%
%
%
\begin{figure}[h!]
	\centering
	\begin{tikzpicture}[domain=0:2]
		\def\T{0.4}
		\draw[->] (-3,0) -- (3,0) node[below right] {$t$};
		\draw[->] (0,0) -- (0,1.5) node[above] {$x_{a}(t-\tau)$};
		\foreach \pos in {-2,...,2} {
				\draw[-, C0, ultra thick] (\pos-\T,0) -- (\pos-\T,1) -- (\pos+\T,1) -- (\pos+\T,0) -- (\pos+1-\T,0);
		};
		\draw[-, C0, ultra thick] (-\T,0) node[below] {$\frac{-T_h}{2}+\tau$};
		\draw[-, C0, ultra thick] (+\T,-0.5) node[below] {$\frac{+T_h}{2}+\tau$};
		\draw[-, C0, ultra thick] (1,0) node[below] {$T+\tau$};
		\draw[-, C0, ultra thick] (2,0) node[below] {$2 T+\tau$};
		\draw[-, C0, ultra thick] (-2-\T,1) node[left] {$A$};
	\end{tikzpicture}
\end{figure}
%
%
%
Für die Berechnung der Koeffizienten ändert sich nun nicht viel.
%
Wir müssen bloß die verschobenen Grenzen beachten.
%
%
%
\begin{align}
	c_k&\underset{k\neq 0}{=}\int_{t0}^{t_0+T}x_{e}(t)\e^{-\im k \omega t}\diff t
	=\int_{-\frac{T}{2}+\tau}^{+\frac{T}{2}+\tau}x_a(t-\tau)\e^{-\im k \omega t}\diff t\nonumber \\
	&=\int_{-\frac{T_h}{2}+\tau}^{+\frac{T_h}{2}+\tau}A\diff t\nonumber \\
	&=A\frac{T}{-\im  k 2\pi }\e^{-\im k \frac{2\pi}{T}t}\Bigg |_{t=-\frac{T_h}{2}+\tau}^{t=+\frac{T_h}{2}+\tau}\nonumber \\
	&=A\frac{T}{\im k 2 \pi }\bigg [\e^{\im k \pi \frac{T_h}{T}-\im k \frac{2\pi}{T}\tau}-\e^{-\im k \pi \frac{T_h}{T}-\im k \frac{2\pi}{T}\tau}\bigg ]
\end{align}
%
%
Wir können nun den Faktor $\e^{-\im k \frac{2\pi}{T}}\tau$ ausklammern und die
restliche komplexe Exponentialfunktionen mit Hilfe der eulerschen Identität
zusammenfassen.
%
\begin{align}
	c_k=\e^{-\im k \omega \tau}A\frac{T}{k\pi}\frac{T_h}{T_h}\sin(k \pi \frac{T_h}{T})=\e^{-\im k \omega t}AT_h\sinc\left(k\pi\frac{T_h}{T}\right)
\end{align}
%
Es fehlt nur noch der Gleichanteil.
%
%
\begin{align}
	c_0&=\int_{-\frac{T_h}{2}+\tau}^{+\frac{T_h}{2}+\tau}A\diff t
	=At\Bigg |_{t=-\frac{T_h}{2}+\tau}^{t=+\frac{T_h}{2}+\tau} \nonumber \\
	&=A\bigg [\bigg (+\frac{T_h}{2}+\tau\bigg )- \bigg (-\frac{T_h}{2}+\tau\bigg )\bigg ]
	=AT_h\nonumber \\
	&=\e^{-\im 0 \omega \tau}AT_h
\end{align}
%
%
Wir erkennen also, dass eine zeitliche Verschiebung der Funktion einer
Modulation der Koeffizienten der komplexen Fourierreihe entspricht.
%
Außerdem können alle Koeffizienten in $c_k$ zusammengefasst werden
wegen der Konvergenz von $\sinc$.
%
Falls $x(t)$ eine Funktion ist und $X_k$ der zu dieser Funktion entsprechende
$k$-te Koeffizient der Komplexen Fourierreihe, können wir folgende
Korrespondenz aufstellen:
%
%
\begin{align}
	x(t-\tau)\quad\fourier\quad\e^{-\im k \omega \tau}X_k\quad\quad\tau\in\mathbb{R}
\end{align}
%
%
Beweis:
%
%
\begin{align}
	c_k&=\int_{t_0}^{t_0+T}x(t-\tau)\e^{-\im k \omega t}\diff t
	=\int_{(t_0-\tau)}^{(t_0-\tau)+T}x(l)\e^{-\im k \omega (l+\tau)}\diff l
	= \e^{-\im k \omega \tau}\int_{(t_0-\tau)}^{(t_0-\tau)+T}x(l)\e^{-\im k \omega l}x(l)\diff l\nonumber \\
	&=\e^{-\im k \omega \tau}X_k
\end{align}
%
%
Wir setzen nun $\frac{T_h}{2}$ für $\tau$ ein und erhalten die Fourierreihe für
$x_{e}(t)$.
%
Wir kürzen $\e^{-\im k \omega \frac{T_h}{2}}$ zu $\e^{-\im k \pi
\frac{T_h}{T}}$.
%
%
\begin{align}
	x_{e}(t)=x_a(t-\frac{T_h}{2})
	=\frac{1}{T}\sum_{k=-\infty}^{+\infty}\e^{-\im k\omega \frac{T_h}{2}}AT_h\sinc\left(k\pi\frac{T_h}{T}\right)\e^{\im k \omega t}
\end{align}
%
\begin{figure}
	\centering
	\begin{subfigure}{\textwidth}
		\includegraphics[width=\textwidth,keepaspectratio]{../fs/FB93908026_timedomain_7}
		\caption{Zeitbereich}
	\end{subfigure}
	\begin{subfigure}{\textwidth}
		\includegraphics[width=\textwidth,keepaspectratio]{../fs/FB93908026_frequence_7}
		\caption{Betrag und Phase der Fourierkoeffizienten}
	\end{subfigure}
	\caption{Synthese der Fourierreihe von $x_e(t)$ für $A=1$, $T=4$, $T_h=2$ und
		$-5\leq k\leq5$.}
\end{figure}

\cleardoublepage
\subsubsection{Aufgabe f) mit komplexer Fourierreihe}
%
\label{sec:TaskDComplex}
%
Die nächste Funktion ergibt sich aus der Modulation von $x_{a}(t)$.
%
%
%
\begin{align}
	\label{eq:TaskDSubstituteXA}
	c_k&=\int_{t_0}^{t_0+T}x_{f}(t)\e^{-\im k \omega t}\diff t
	=\int_{-\frac{T_h}{2}}^{+\frac{T_h}{2}}\e^{+\im\omega_0 t}A\e^{-\im k \omega t}x_a(t)\diff t \\
	&=\int_{-\frac{T_h}{2}}^{+\frac{T_h}{2}}A\e^{-\im t(k\omega -\omega_0)}\diff t
\end{align}
%
Sei nun $\omega_0=n\cdot\omega$, wobei $n\in\mathbb{Z}$.
%
Dann lässt sich der Ausdruck $k\omega -\omega_0$ zu $(k-n)\omega$
zusammenfassen.
%
%
\begin{align}
	c_k&=\int_{-\frac{T_h}{2}}^{+\frac{T_h}{2}}A\e^{-\im (k-n)\omega t} \nonumber \\
	&\underset{k\neq n}{=}A\frac{T}{-\im (k-n) 2\pi}\e^{-\im (k-n)\omega t}\Bigg |_{t=-\frac{T_h}{2}}^{t=+\frac{T_h}{2}} \nonumber \\
	&=A\frac{T}{\im (k-n)2 \pi}\Bigg [\e^{\im (k-n)\pi\frac{T_h}{T}}-\e^{-\im (k-n)\pi \frac{T_h}{T}}\Bigg ]
	= A\frac{T}{(k-n)\pi}\frac{T_h}{T_h}\sin\left((k-n)\pi\frac{T_h}{T}\right)\nonumber\\
	&=AT_h\sinc\left((k-n)\pi\frac{T_h}{T}\right)
\end{align}
%
bzw.
%
\begin{align}
	c_k\underset{k=n}{=}\int_{-\frac{T_h}{2}}^{+\frac{T_h}{2}}A\diff t=AT_h
\end{align}
%
Tatsächlich sind die Koeffizienten von $x_{f}(t)$ die gleichen wie von
$x_{a}(t)$, jedoch um $n$ nach rechts verschoben.
%
Wir hätten $x_{a}(t)$ in \eqref{eq:TaskDSubstituteXA} gar nicht einsetzen brauchen.
%
Wir könnten auch $m=k-n$ substituieren.
%
\begin{align}
	c_{k+m}=\int_{x_0}^{x_0+T}x_{a}(t)\e^{-\im m \omega t}\diff t
\end{align}
Also ist der $m+n$-te Koeffizient von $x_{f}(t)$ der $m$-te Koeffizient von
$x_{a}(t)$.
%
Wenn wir nun wieder zurück substituieren ($m=k-n$), dann sehen wir, dass der $k$-te Koeffizient von $x_{f}(t)$ der $k-n$-te Koeffizient von $x_{a}(t)$ ist.
%
Also können wir uns als nächste allgemeine Korrespondenz folgendes notieren:
%
%
\begin{align}
	\label{eq:CorrespondenceModulation}
	\e^{+\im n \omega t}x(t)\quad\fourier\quad X_{k-n}\quad\quad n\in\mathbb{Z}.
\end{align}
%
%
Die Fourierreihe können wir nun wie folgt aufschreiben:
%
%
\begin{align}
	x_{f}(t)=\frac{1}{T}\sum_{k=-\infty}^{+\infty}AT_h\sinc\left((k-n)\pi\frac{T_h}{T}\right)\e^{+\im k \omega t}
\end{align}
%
%
\begin{figure}
	\centering
	\begin{subfigure}{\textwidth}
		\includegraphics[width=\textwidth,keepaspectratio]{../fs/FB93908026_timedomain_8}
		\caption{Zeitbereich}
	\end{subfigure}
	\begin{subfigure}{\textwidth}
		\includegraphics[width=\textwidth,keepaspectratio]{../fs/FB93908026_frequence_8}
		\caption{Betrag und Phase der Fourierkoeffizienten}
	\end{subfigure}
	\caption{Synthese der Fourierreihe von $x_f(t)$ für $A=1$, $T=4$, $T_h=2$ und
		$-5\leq k\leq5$.}
\end{figure}
\newpage
%
\subsubsection{Aufgabe g) mit komplexer Fourierreihe}
Als nächstes berechnen wir die die komplexen Fourierkoeffizienten von
%
\begin{gather}
	x_{g}(t) =
		\begin{cases}
			A &, -\frac{T_h}{2}\leq t \leq 0 \\
			-A &, 0<t\leq \frac{T_h}{2}\\
			0 &,\text{ sonst}
		\end{cases} \\
	t\in[-\frac{T}{2},+\frac{T}{2}]\nonumber
\end{gather}
%
%
%
\begin{figure}[h!]
	\centering
	\begin{tikzpicture}[domain=0:2]
		\def\T{0.4}
		\draw[->] (-3,0) -- (3,0) node[below right] {$t$};
		\draw[->] (0,0) -- (0,1.5) node[above] {$x_{g}(t)$};
		\foreach \pos in {-2,...,2} {
			\draw[-, red, ultra thick] (\pos-\T,0) -- (\pos-\T,1)--(\pos,1)--(\pos,-1) -- (\pos+\T,-1) -- (\pos+\T,0) -- (\pos+1-\T,0);
		};
		\draw[-, black, ultra thick] (-\T,0) node[below] {$\frac{-T_h}{2}$};
		\draw[-, black, ultra thick] (+\T,0) node[below] {$\frac{+T_h}{2}$};
		\draw[-, black, ultra thick] (1,0) node[below] {$T$};
		\draw[-, black, ultra thick] (2,0) node[below] {$2 T$};
		\draw[-, black, ultra thick] (-2-\T,1) node[left] {$A$};
	\end{tikzpicture}
\end{figure}
%
%
%
Die Berechnung des Gleichanteils sparen wir uns an dieser Stelle, denn man
sieht an Hand der Graphik, dass sich innerhalb einer Periode die Flächen unter den Rechtecken wieder aufheben bzw. es sich um eine ungerade Funktion handelt
und somit $a_0=0$ (siehe \ref{eq:KoeffUngerade})  wäre, womit auch $c_0=0$
ist.
%

%
Wir berechnen also nur die Koeffizienten für $k\neq0$:
%
%
%
\begin{align}
	c_k&=\int_{t_0}^{t_0+T}x_{g}(t)\diff t
	=\int_{-\frac{T_h}{2}}^{+\frac{T_h}{2}}x_{g}(t)\diff t \nonumber \\
	&=\int_{-\frac{T_h}{2}}^0A\e^{-\im k \omega t}\diff t+ \int_0^{-\frac{T_h}{2}}-A\e^{-\im k\omega t}\diff t \nonumber \\
	&=\frac{A}{-\im k \omega}\e^{-\im k \omega t}\Bigg |_{t=-\frac{T_h}{2}}^{t=0}+\frac{-A}{-\im k \omega }\e^{-\im k \omega t}\Bigg |_{t=0}^{t=+\frac{T_h}{2}}\nonumber \\
	&=A\frac{T}{\im k2\pi}\left [\e^{\im k \frac{2\pi}{T} \frac{T_h}{2}}-1-1+\e^{-\im k \frac{2\pi}{T} \frac{T_h}{2}}\right ] \nonumber \\
	\label{eq:CKTaskEComplex}
	&=A\frac{T}{\im k\pi}\frac{2}{2}\left (\cos(k\pi\frac{T_h}{T})-1\right )
	=\underset{\ref{eq:sinsquaredcosinus}}{=}-2A\frac{T}{\im k\pi}\frac{2k\pi T}{2k\pi T}\frac{T_h^2}{T_h^2}\sin^2\left(k\pi\frac{T_h}{2T}\right)\nonumber\\
	&=jAk\pi \frac{T_h^2}{2T}\sinc\left(k\pi\frac{T_h}{2T}\right)
\end{align}
%
%
%
Zum Schluss schreiben wir $x_{g}(t)$ noch als Fourierreihe auf:
%
\begin{align}
	x_{g}(t)=\sum_{k=-\infty}^{+\infty}jAk\pi\frac{T_h^2}{2T}\sinc\left(k\pi\frac{T_h}{2T}\right)\e^{+\im k \omega t}
\end{align}
%
Wir sollten bemerkt haben, dass sich $x_{b}(t)$ und $x_{g}(t)$ sehr stark
ähneln.
%
Tatsächlich ergibt sich $x_{g}(t)$ daraus, dass man $x_{b}(t)$ an der Ordinate
(bzw. der $y$-Achse) spiegelt.
%
$x_{g}(t)$ erhält man also durch Zeitumkehr aus $x_{b}(t)$, d.h. $x_{g}(t)$ =
$x_{b}(-t)$.
%
Wir wollen an dieser Stelle schauen, ob sich die Koeffizienten von $x_{g}(t)$
evtl. aus $x_{b}(t)$ ergeben und wir auf die Berechnung verzichten hätten
können.
%
\eqref{eq:CKTaskEComplex} und \eqref{eq:CKTaskBComplex} gleichen sich schon
sehr, jedoch muss man einen Ausdruck jeweils mit $(-1)$ multiplizieren, um den
anderen zu erhalten.
%
Wir versuchen wie in \ref{sec:TaskDComplex} $x_{g}(t)$ mit einer anderen
Funktion, von dem die Fourierkoeffizienten schon bekannt sind, zu
substituieren, um eine Korrespondenz aufzustellen.
%
\begin{align}
	c_k=\int_{t_0}^{t_0+T}x_{g}(t)\e^{-\im k \omega t}\diff t
	= \int_{t_0}^{t_0+T}x_{b}(-t)\e^{-\im k \omega t}\diff t
\end{align}
%
Wir substituieren nun $t$ mit $-\tau$.
%
\begin{align}
	c_k=-\int_{-t_0}^{-t_0-T}x_{b}(\tau)\e^{+\im k \omega \tau}\diff \tau
\end{align}
%
Nun müssen wir die Grenzen tauschen. Somit wird wieder über eine Periode $T$
integriert.
%
\begin{align}
	c_k=\int_{-t_0-T}^{-t_0}x_{b}(\tau)\e^{+\im k \omega \tau}\diff \tau
\end{align}
%
Anschließend substituieren wir $k$ mit $-m$, um wieder ein Minus-Zeichen in der
komplexen Exponentialfunktion zu erhalten.
%
\begin{align}
	c_{-m}=\int_{-t_0-T}{-t_0}x_{b}(\tau)\e^{-\im m \omega \tau}\diff \tau
\end{align}
%
Also ist der $-m$-te Koeffizient von $x_{g}(t)$ der $m$-te Koeffizient von
$x_{b}(t)$.
%
Wir substituieren wieder zurück ($k=-m$) und merken uns folgende Korrespondenz.
\begin{align}
	x(-t)\quad\fourier\quad X_{-k}.
\end{align}
\begin{figure}
	\centering
	\begin{subfigure}{\textwidth}
		\includegraphics[width=\textwidth,keepaspectratio]{../fs/FB93908026_timedomain_9}
		\caption{Zeitbereich}
	\end{subfigure}
	\begin{subfigure}{\textwidth}
		\includegraphics[width=\textwidth,keepaspectratio]{../fs/FB93908026_frequence_9}
		\caption{Betrag und Phase der Fourierkoeffizienten}
	\end{subfigure}
	\caption{Synthese der Fourierreihe von $x_g(t)$ für $A=1$, $T=4$, $T_h=2$ und
		$-5\leq k\leq5$.}
\end{figure}
\newpage
\subsubsection{Aufgabe h) mit komplexer Fourierreihe}
In diesem Abschnitt wollen wir uns die Fourierkoeffizienten einer Funktion
angucken, die aus der zyklischen Faltung zweier $T$-periodischer Funktionen
entsteht.
%
Dabei ist die zyklische Faltung wie folgt definiert:
%
\begin{align}
	x(t)\circledast h(t)=\int_{t_0}^{t_0+T}x(\tau)h(t-\tau)\diff \tau=\int_{t_0}^{t_0+T}x(t-\tau)h(\tau)\diff \tau.
\end{align}
%
Wir werden zunächst die Fourierkoeffizienten direkt aus dem Faltungsprodukt
berechnen.
%
Anschließend schauen wir, ob sich aus den gegebenen Fourierkoeffizienten der
beteiligten Funktionen direkt der Fourierkoeffizient des Faltungsproduktes
ermitteln lässt.

Wir zeichnen uns zunächst noch einmal die Funktionen $x_{a}(t)$ und $x_{b}(t)$
über eine Periode auf.
%
%
\begin{figure}[h!]
	\centering
	\begin{tikzpicture}[domain=0:2]
		\def\T{0.4}
		\draw[->] (-3,0) -- (3,0) node[below right] {$t$};
		\draw[->] (0,0) -- (0,1.5) node[above] {$x_{a}(t)$};
		\foreach \pos in {0} {
			\draw[-, C0, ultra thick] (\pos - 1+\T,0) -- (\pos-\T,0) -- (\pos-\T,1) -- (\pos+\T,1) -- (\pos+\T,0) -- (\pos+1-\T,0);
		};
		\draw[-, C0, ultra thick] (-\T,0) node[below] {$\frac{-T_h}{2}$};
		\draw[-, C0, ultra thick] (+\T,0) node[below] {$\frac{+T_h}{2}$};
		\draw[-, C0, ultra thick] (1,0) node[below] {$T$};
		\draw[-, C0, ultra thick] (-1,0) node[below] {$-T$};
		\draw[-, C0, ultra thick] (-2-\T,1) node[left] {$A$};
	\end{tikzpicture}
\end{figure}
%
%
\begin{figure}[h!]
	\centering
	\begin{tikzpicture}[domain=0:2]
		\def\T{0.4}
		\draw[->] (-3,0) -- (3,0) node[below right] {$t$};
		\draw[->] (0,0) -- (0,1.5) node[above] {$x_{b}(t)$};
		\foreach \pos in {0} {
			\draw[-, red, ultra thick] (\pos-1+\T,0) -- (\pos-\T,0) -- (\pos-\T,-1)--(\pos,-1)--(\pos,1) -- (\pos+\T,1) -- (\pos+\T,0) -- (\pos+1-\T,0);
		};
		\draw[-, black, ultra thick] (-\T,0) node[below] {$\frac{-T_h}{2}$};
		\draw[-, black, ultra thick] (+\T,0) node[below] {$\frac{+T_h}{2}$};
		\draw[-, black, ultra thick] (1,0) node[below] {$T$};
		\draw[-, black, ultra thick] (-1,0) node[below] {$-T$};
		\draw[-, black, ultra thick] (-2-\T,1) node[left] {$A$};
	\end{tikzpicture}
\end{figure}

Wir substituieren nun $t$ mit $\tau$ und verschieben danach noch $x_{a}(\tau)$
um $t$.
%
%
\begin{figure}[h!]
	\centering
	\begin{tikzpicture}[domain=0:2]
		\def\T{0.4}
		\draw[->] (-3,0) -- (3,0) node[below right] {$\tau$};
		\draw[->] (0,0) -- (0,1.5) node[above] {$x_{a}(t-\tau)$};
		\foreach \pos in {-2} {
			\draw[-, C0, ultra thick] (\pos - 1+\T,0) -- (\pos-\T,0) -- (\pos-\T,1) -- (\pos+\T,1) -- (\pos+\T,0) -- (\pos+1-\T,0);
		};
		\draw[-, C0, ultra thick] (-\T-2,-0.5) node[below] {$\frac{-T_h}{2}+t$};
		\draw[-, C0, ultra thick] (+\T-2,-1) node[below] {$\frac{+T_h}{2}+t$};
		\draw[-, C0, ultra thick] (1-2,0) node[below] {$T+t$};
		\draw[-, C0, ultra thick] (-1-2,0) node[below] {$-T+t$};
		\draw[-, C0, ultra thick] (-2-\T,1) node[left] {$A$};
	\end{tikzpicture}
\end{figure}
%
%
\begin{figure}[h!]
	\centering
	\begin{tikzpicture}[domain=0:2]
		\def\T{0.4}
		\draw[->] (-3,0) -- (3,0) node[below right] {$\tau$};
		\draw[->] (0,0) -- (0,1.5) node[above] {$x_{b}(\tau)$};
		\foreach \pos in {0} {
			\draw[-, red, ultra thick] (\pos-1+\T,0) -- (\pos-\T,0) -- (\pos-\T,-1)--(\pos,-1)--(\pos,1) -- (\pos+\T,1) -- (\pos+\T,0) -- (\pos+1-\T,0);
		};
		\draw[-, black, ultra thick] (-\T,0) node[below] {$\frac{-T_h}{2}$};
		\draw[-, black, ultra thick] (+\T,0) node[below] {$\frac{+T_h}{2}$};
		\draw[-, black, ultra thick] (1,0) node[below] {$T$};
		\draw[-, black, ultra thick] (-1,0) node[below] {$-T$};
		\draw[-, black, ultra thick] (-2-\T,1) node[left] {$A$};
	\end{tikzpicture}
\end{figure}
%
%
%

Es lassen sich vier verschiedene Fälle betrachten.

Der erste Fall liegt vor, wenn bei der Faltung sich nur in einer Periode überlappt wird.
%
Dafür muss die linke Seite ( mit der linken Seite von $x_{a}(t)$ wird an dieser Stelle vom Sprung von $0$ zu $A$ und bei der rechten vom Sprung von $A$ zu $0$ geredet) zu Beginn schon hinter $-T+\frac{T_h}{2}$ liegen, damit $x_a(t)$ sich nicht mit der vorherigen Periode von $x_b(t)$ überlappt.
%
Dies kann man auch als Ungleichung ausdrücken.
%
%
\begin{gather}
	-T+\frac{T_h}{2}\leq -\frac{T_h}{2}+t\Bigg |_{t=-\frac{T}{2}}\nonumber\\
	-\frac{T_h}{2}-\frac{T}{2}\quad\Bigg |\quad +T+\frac{T_h}{2}\nonumber \\
	T_h\leq \frac{T}{2}\quad\Bigg |\quad \cdot\frac{1}{T}\nonumber \\
	\frac{T_h}{T}\leq\frac{1}{2}
\end{gather}
%
%
Beim zweiten Fall liegt die linke Seite von $x_{a}(t)$ bereits vor
$-T+\frac{T_h}{2}$, bei der Faltung überlappt $x_{a}(t)$ mit der vorherigen
Periode von $x_b(t)$.
%
Nun wollen wir, dass beim schieben nach rechts die linke Seite von $x_{a}(t)$
$-T+\frac{T_h}{2}$ erreicht und die rechte Seite höchstens bei $0$ liegt.
%
Die linke Seite liegt auf $-T+\frac{T_h}{2} \Leftrightarrow
-T+\frac{T_h}{2}=-\frac{T_h}{2}+t$, also $t=T_h-T$. Wir können wieder eine
Ungleichung aufstellen.
%
%
%
\begin{gather}
	+\frac{T_h}{2}+t\Bigg |_{t=T_h-T}\nonumber\\
	\frac{3}{2}T_h-T\leq 0\quad\Bigg |\quad+T\nonumber \\
	\frac{3}{2}T_h\leq T\quad\Bigg | \quad\cdot\frac{2}{3}\cdot\frac{1}{T}\nonumber\\
	\frac{T_h}{T}\leq \frac{2}{3}
\end{gather}
%
%
%
Der dritte Fall ist schnell erledigt, wir begrenzen $\frac{T_h}{T}$ nach oben,
also $\frac{2}{3}<\frac{T_h}{T}<1$, denn für $\frac{T_h}{T}$ wäre $x_a(t)=1$,
und die Faltung würde $0$ ergeben (wird später erklärt).

\textbf{Fall 1:}

Von $-\frac{T}{2}\leq t \leq -T_h$ und $T_h \leq t \leq \frac{T}{2}$ überlappen
sich $x_{a}(t-\tau)$ und $x_{b}(\tau)$ nicht.

Wir können folgende Fälle unterscheiden.
\begin{gather}
	x_{h}(t)=
		\begin{cases}
			0 &, -\frac{T}{2}\leq t \leq -T_h \\
			y_1(t) &,-T_h\leq t \leq -\frac{T_h}{2} \\
			y_2(t) &, -\frac{T_h}{2} \leq t \leq 0 \\
			y_3(t) &, 0 \leq t \leq +\frac{T_h}{2} \\
			y_4(t) &, \frac{T_h}{2}\leq T_h \\
			0 &, T_h \leq t \leq \frac{T}{2}
		\end{cases}
\end{gather}
%
%
\begin{gather}
	y_1(t)=\int_{-\frac{T_h}{2}}^{+\frac{T_h}{2}+t}-A^2\diff \tau
	= -A^2\tau \Bigg |_{\tau=-\frac{T_h}{2}}^{\tau=\frac{T_h}{2}+t}
	=-A^2(\frac{T_h}{2}+t+\frac{T_h}{2})
	=-A^2(t+T_h)
\end{gather}
%
%
%
\begin{align}
	y_2(t)=-A^2\frac{T_h}{2}+\int_{0}^{\frac{T_h}{2}+t}A^2\diff \tau
	= -A^2\frac{T_h}{2}+A^2\tau\Bigg |_{\tau=0}^{\tau=\frac{T_h}{2}+t}
	=-A^2\frac{T_h}{2}+A^2\left(\frac{T_h}{2}+t\right)
	=A^2t
\end{align}
%
%
\begin{align}
	y_3(t)=A^2\frac{T_h}{2}+\int_{t-\frac{T_h}{2}}^{0}-A^2\diff \tau
	=A^2\frac{T_h}{2}-A^2\tau\Bigg |_{\tau=t-\frac{T_h}{2}}^{\tau=0}
	=A^2\frac{T_h}{2}+A^2\left(t-\frac{T_h}{2}\right)
	=A^2t
\end{align}
%
%
\begin{align}
	y_4(t)=\int_{t-\frac{T_h}{2}}^{\frac{T_h}{2}}A^2\diff \tau
	=A^2\Bigg |_{\tau=t-\frac{T_h}{2}}^{\tau=\frac{T_h}{2}}
	=A^2\left(\frac{T_h}{2}-t+\frac{T-h}{2}\right)
	=A^2(-t+T_h)
\end{align}
%
%
Somit ergibt sich folgende Funktion.
\begin{gather}
	x_{h}(t)=x_{a}(t)\circledast x_{b}(t)
	\underset{0<\frac{T_h}{T}\leq\frac{1}{2}}{=}
		\begin{cases}
			0 &, -\frac{T}{2}\leq t \leq -T_h \\
			-A^2(t+T_h) &,-T_h\leq t \leq -\frac{T_h}{2} \\
			A^2t &, -\frac{T_h}{2} \leq t \leq +\frac{T_h}{2} \\
			A^2(-t+T_h) &, \frac{T_h}{2}\leq T_h \\
			0 &, T_h \leq t \leq \frac{T}{2}
		\end{cases}
\end{gather}
%
%
Nun folgt die Berechnung der Koeffizienten der Fourierreihe.
%
%
%
\begin{align}
	c_0&=\int_{t_0}^{t_0+T}x_{h}(t)\diff t=\frac{1}{T}\int_{-T_h}^{+T_h}x_{a}(t)\circledast x_{b}(t)\diff t \nonumber \\
	&=\int_{-T_h}^{-\frac{T_h}{2}}-A^2(t+T_h)\diff t+\int_{-\frac{T_h}{2}}^{+\frac{T_h}{2}}A^2t\diff t+\int_{+\frac{T_h}{2}}^{+T_h}A^2(-t+T_h)\diff t\nonumber \\
	&=-A^2\left(\frac{t^2}{2}+T_ht\right)\Bigg |_{t=-T_h}^{t=-\frac{T_h}{2}}+A^2\frac{t^2}{2}\Bigg |_{t=-\frac{T_h}{2}}^{t=\frac{T_h}{2}} + A^2\left(-\frac{t^2}{2}+T_ht\right)\Bigg |_{t=+\frac{T_h}{2}}^{t=+T_h} \nonumber \\
	&=-A^2\left(\frac{T_h^2}{8}-\frac{T_h^2}{2}-\frac{T_h^2}{2}+T_h^2\right)+A^2\left(\frac{T_h^2}{8}-\frac{T_h^2}{8}\right)+A^2\left(-\frac{T_h^2}{2}+T_h^2+\frac{T_h^2}{8}-\frac{T_h^2}{2}\right) \nonumber \\
	&=A^2\left (-\frac{T_h^2}{8}+\frac{T_h^2}{8}\right )=0
\end{align}
%
%
%
\begin{align}
	c_k&=\int_{t_0}^{t_0+T}x_{h}(t)\e^{-\im k \omega t}\diff t
	\underset{0<\frac{T_h}{T}\leq \frac{1}{2}}{=}\int_{-T_h}^{+T_h}x_{a}(t)\circledast x_{b}(t)\e^{-\im k \omega t}\diff t \nonumber \\
	&=\underbrace{\int_{-T_h}^{-\frac{T_h}{2}}-A^2(t+T_h)\e^{-\im k \omega t}\diff t}_A+\underbrace{\int_{-\frac{T_h}{2}}^{+\frac{T_h}{2}}A^2t\e^{-\im k \omega t}\diff t}_B+\underbrace{\int_{+\frac{T_h}{2}}^{+T_h}A^2(-t+T_h)\e^{-\im k \omega t}\diff t}_C
\end{align}
%

%
\begin{align}
	A&=\int_{-T_h}^{-\frac{T_h}{2}}-A^2(t+T_h)\e^{-\im k \omega t}\diff t=\frac{A^2(t+T_h)}{\im k \omega }\e^{-\im k \omega t}\Bigg |_{t=-T_h}^{t=-\frac{T_h}{2}}-\int_{-T_h}^{-\frac{T_h}{2}}\frac{A^2}{\im k \omega }\e^{-\im k \omega t}\diff t\nonumber \\
	&=\frac{A^2T_h}{2\im k \omega }\e^{\im k \omega \frac{T_h}{2}}-\frac{A^2}{k^2\omega^2}\e^{\im k \omega \frac{T_h}{2}}+\frac{A^2}{k^2\omega^2}\e^{\im k \omega T_h}
\end{align}
%

%
\begin{align}
	B&=\int_{-\frac{T_h}{2}}^{+\frac{T_h}{2}}A^2t\e^{-\im k \omega t}\diff t
	=\frac{A^2t}{-\im k \omega}\e^{-\im k \omega t}\Bigg |_{t=-\frac{T_h}{2}}^{t=+\frac{T_h}{2}}-\int_{-\frac{T_h}{2}}^{+\frac{T_h}{2}}\frac{A^2}{-\im k \omega}\e^{-\im k \omega t}\diff t\nonumber \\
	&=-\frac{A^2T_h}{2\im k \omega}\e^{-\im k \omega \frac{T_h}{2}}-\frac{A^2T_h}{2\im k \omega}\e^{+\im k \omega \frac{T_h}{2}}+\frac{A^2}{k^2\omega^2}\e^{-\im k \omega \frac{T_h}{2}}-\frac{A^2}{ k^2 \omega^2}\e^{+\im k \omega \frac{T_h}{2}}
\end{align}
%

%
\begin{align}
	C&=\int_{+\frac{T_h}{2}}^{+T_h}A^2(-t+T_h)\e^{-\im k \omega t}\diff t
	=\frac{A^2(-t+T_h)}{-\im k \omega }\e^{-\im k \omega t}\Bigg |_{t=+\frac{T_h}{2}}^{t=+T_h}-\int_{+\frac{T_h}{2}}^{+T_h}\frac{-A^2}{-\im k \omega }\e^{-\im k \omega t}\diff t\nonumber \\
	&=\frac{A^2T_h}{2\im k \omega}\e^{-\im k \omega \frac{T_h}{2}}-\frac{A^2}{k^2\omega^2}\e^{-\im k \omega T_h}+\frac{A^2}{k^2\omega^2}\e^{-\im k \omega \frac{T_h}{2}}
\end{align}
%

%
\begin{align}
	A+C&=\frac{A^2T_h}{2\im k \omega }\e^{\im k \omega \frac{T_h}{2}}-\frac{A^2}{k^2\omega^2}\e^{\im k \omega \frac{T_h}{2}}+\frac{A^2}{k^2\omega^2}\e^{\im k \omega T_h}+\frac{A^2T_h}{2\im k \omega}\e^{-\im k \omega \frac{T_h}{2}}-\frac{A^2}{k^2\omega^2}\e^{-\im k \omega T_h}+\frac{A^2}{k^2\omega^2}\e^{-\im k \omega \frac{T_h}{2}}\nonumber \\
	&=\e^{\im k \omega \frac{T_h}{2}}\left (\frac{A^2T_h}{2\im k \omega}-\frac{A^2}{k^2\omega^2}\right )+\e^{-\im k \omega \frac{T_h}{2}}\left (\frac{A^2T_h}{2\im k \omega}+\frac{A^2}{k^2\omega^2}\right )+\e^{\im k \omega T_h}\frac{A^2}{k^2\omega^2}-\e^{-\im k \omega T_h}\frac{A^2}{k^2\omega^2}
\end{align}
%

%
\begin{align}
	(A+C)+B&=\e^{\im k \omega \frac{T_h}{2}}\left (\frac{A^2T_h}{2\im k \omega}-\frac{A^2}{k^2\omega^2}\right )+\e^{-\im k \omega \frac{T_h}{2}}\left (\frac{A^2T_h}{2\im k \omega}+\frac{A^2}{k^2\omega^2}\right )+\e^{\im k \omega T_h}\left (\frac{A^2}{k^2\omega^2}\right )-\e^{-\im k \omega T_h}\left (\frac{A^2}{k^2\omega^2}\right ) \nonumber \\
	&-\frac{A^2T_h}{2\im k \omega}\e^{-\im k \omega \frac{T_h}{2}}-\frac{A^2T_h}{2\im k \omega}\e^{+\im k \omega \frac{T_h}{2}}+\frac{A^2}{k^2\omega^2}\e^{-\im k \omega \frac{T_h}{2}}-\frac{A^2}{ k^2 \omega^2}\e^{+\im k \omega \frac{T_h}{2}}\nonumber \\
	&=\e^{\im k \omega \frac{T_h}{2}}\left (\frac{A^2T_h}{2\im k \omega}-\frac{A^2}{k^2\omega^2}-\frac{A^2T_h}{2\im k \omega}-\frac{A^2}{k^2\omega^2}\right )+\e^{-\im k \omega \frac{T_h}{2}}\left (\frac{A^2T_h}{2\im k \omega}+\frac{A^2}{k^2\omega^2}-\frac{A^2T_h}{2 \im k \omega}+\frac{A^2}{k^2\omega^2}\right )\nonumber \\
	&+\e^{\im k \omega T_h}\left (\frac{A^2}{k^2\omega^2}\right )-\e^{-\im k \omega T_h}\left (\frac{A^2}{k^2\omega^2}\right )=\frac{A^2}{k^2\omega^2}\left [2\im\sin\left(k\omega T_h\right)-4\im\sin\left(k\omega \frac{T_h}{2}\right)\right ]
\end{align}
%

%
\begin{align}
	c_k=\frac{A^2T^2}{k^24\pi^2}\left[2\im\sin\left(k\omega T_h\right)-4\im\sin\left(k\omega \frac{T_h}{2}\right)\right ]
	=\im\frac{A^2T^2}{2k^2\pi^2}\left[\sin\left(2\pi k \frac{T_h}{T}\right)-2\sin\left(\pi k \frac{T_h}{T}\right)\right]
\end{align}
%
%
Es gilt (vgl. \cite[Kap. 2 S. 81, 84]{Bronstein2015}):
%
\begin{gather}
	\sin(2\alpha)=2\sin(\alpha)\cos(\alpha),\\
	\sin^2(\alpha)=\frac{1}{2}\left(1-\cos(2\alpha)\right).
\end{gather}
%
%
Somit können wir $c_k$ vereinfachen.
%
%
\begin{align}
	c_k
	&=\im\frac{A^2T^2}{2k^2\pi^2}\left[2\sin\left(k\pi\frac{T_h}{T}\right)\cos\left(k\pi\frac{T_h}{T}\right)-2\sin\left(k\pi\frac{T_h}{T}\right)\right]\nonumber\\
	%
	&=\im\frac{A^2T^2}{k^2\pi^2}\sin\left(k\pi\frac{T_h}{T}\right)\left[\cos\left(k\pi\frac{T_h}{T}\right)-1\right]\nonumber\\
	%
	&=-\im2\frac{A^2T^2}{k^2\pi^2}\frac{T}{T}\frac{T_h^3}{T_h^3}\frac2{k\pi}{2k\pi}\sin\left(k\pi\frac{T_h}{T}\right)\sin^2\left(k\pi\frac{T_h}{2T}\right)\nonumber\\
	%
	&=\e^{-\im\frac{\pi}{2}}A^2T_h^2\frac{T_h}{2T}k\pi\sinc\left(k\pi\frac{T_h}{T}\right)\sinc^2\left(k\pi\frac{T_h}{2T}\right)
\end{align}
%
%
\newpage
\textbf{Fall 2:}\\
Nun gilt, dass $\frac{1}{2}\leq \frac{T_h}{T} \leq \frac{2}{3}$.
%
Dadurch überlappt eine jeweilige Periode sich jeweils mit der unmittelbar rechts
und links daneben liegenden.
%
Beim schieben nach rechts verschwindet aber die Überlappung mit der linken
Periode eher, bevor die rechte Kante von $x_{a}(t-\tau)$ an der $y$-Achse liegt.
%
%
\begin{figure}[h!]
	\centering
	\begin{tikzpicture}[domain=0:2]
		\def\T{0.4}
		\draw[->] (-3,0) -- (3,0) node[below right] {$\tau$};
		\draw[->] (0,0) -- (0,1.5) node[above] {$x_{a}(t-\tau)$};
		\foreach \pos in {-2} {
			\draw[-, C0, ultra thick] (\pos - 1+\T,0) -- (\pos-\T,0) -- (\pos-\T,1) -- (\pos+\T,1) -- (\pos+\T,0) -- (\pos+1-\T,0);
		};
		\draw[-, C0, ultra thick] (-\T-2,-0.5) node[below] {$\frac{-T_h}{2}+t$};
		\draw[-, C0, ultra thick] (+\T-2,-1) node[below] {$\frac{+T_h}{2}+t$};
		\draw[-, C0, ultra thick] (1-2,0) node[below] {$T+t$};
		\draw[-, C0, ultra thick] (-1-2,0) node[below] {$-T+t$};
		\draw[-, C0, ultra thick] (-2-\T,1) node[left] {$A$};
	\end{tikzpicture}
\end{figure}
%
%
\begin{figure}[h!]
	\centering
	\begin{tikzpicture}[domain=0:2]
		\def\T{0.4}
		\draw[->] (-3,0) -- (3,0) node[below right] {$\tau$};
		\draw[->] (0,0) -- (0,1.5) node[above] {$x_{b}(\tau)$};
		\foreach \pos in {-1,0,1} {
			\draw[-, red, ultra thick] (\pos-1+\T,0) -- (\pos-\T,0) -- (\pos-\T,-1)--(\pos,-1)--(\pos,1) -- (\pos+\T,1) -- (\pos+\T,0) -- (\pos+1-\T,0);
		};
		\draw[-, black, ultra thick] (-\T,0) node[below] {$\frac{-T_h}{2}$};
		\draw[-, black, ultra thick] (+\T,0) node[below] {$\frac{+T_h}{2}$};
		\draw[-, black, ultra thick] (1,0) node[below] {$T$};
		\draw[-, black, ultra thick] (-1,0) node[below] {$-T$};
		\draw[-, black, ultra thick] (-2-\T,1) node[left] {$A$};
	\end{tikzpicture}
\end{figure}
%
%
Es ergeben sich folgende Teilfunktionen:
%
%
\begin{align}
	x_{h}(t)=x_{a}(t)\circledast x_{b}(t)
	\underset{\frac{1}{2}<\frac{T_h}{T}\leq \frac{2}{3}}{=}
	\begin{cases}
		y_5(t)&,-\frac{T}{2}\leq t \leq T_h-T, \\
		y_6(t)&,T_h-T\leq t \leq -\frac{T_h}{2}, \\
		y_7(t)&,-\frac{T_h}{2}\leq t\leq 0, \\
		y_8(t)&,0\leq t \leq \frac{T_h}{2}, \\
		y_9(t)&,\frac{T_h}{2}\leq t \leq T-T_h,\\
		y_{10}(t)&,T-T_h\leq t \leq \frac{T}{2}.
	\end{cases}
\end{align}
%
%
Man sieht direkt, dass $y_7(t)=y_8(t)=y_3(t)=y_4(t)$. Wir können also zwei
Berechnungen aussparen.
%
\begin{align}
	y_5(t)&=\int_{-\frac{T_h}{2}+t}^{\frac{T_h}{2}-T}A^2\diff \tau+\int_{-\frac{T_h}{2}}^{\frac{T_h}{2}+t}-A^2\diff \tau=A^2\tau \Bigg |_{\tau=-\frac{T_h}{2}+t}^{\frac{T_h}{2}-T}+-A^2\tau \Bigg |_{\tau=-\frac{T_h}{2}}^{\tau=\frac{T_h}{2}+t}\nonumber \\
	&=A^2\left(\frac{T_h}{2}-T+\frac{T_h}{2}-t\right)-A^2\left(\frac{T_h}{2}+t+\frac{T_h}{2}\right)\nonumber \\
	&=A^2 \left(T_h-T-t-t-T_h\right)=A^2\left(-2t-T\right)
\end{align}
%
%
\begin{align}
	y_6(t)=\int_{-\frac{T_h}{2}}^{+\frac{T_h}{2}+t}-A^2\diff \tau=-A^2\tau \Bigg |_{\tau=-\frac{T_h}{2}}^{\tau=+\frac{T_h}{2}+t}=-A^2 (t+T_h)
\end{align}
%
%
\begin{align}
	y_9(t)=\int_{-\frac{T_h}{2}+t}^{\frac{T_h}{2}}A^2\diff \tau=A^2\tau\Bigg |_{-\frac{T_h}{2}+t}^{\frac{T_h}{2}}=A^2(-t+T_h)
\end{align}
%
%
\begin{align}
	y_{10}(t)&=\int_{-\frac{T_h}{2}+t}^{\frac{T_h}{2}}A^2\diff \tau+\int_{T-\frac{T_h}{2}}^{+\frac{T_h}{2}+t}-A^2\diff \tau=A^2\tau\Bigg |_{\tau=-\frac{T_h}{2}+t}^{+\frac{T_h}{2}}-A^2\tau\Bigg |_{T-\frac{T_h}{2}}^{+\frac{T_h}{2}+t}\nonumber \\
	&=A^2(-t+T_h)-A^2(t-T+T_h)=A^2(-2t+T)
\end{align}
%
%
Die endgültige Funktion lautet :
%
%
\begin{align}
	x_{h}(t)=x_{a}(t)\circledast x_{b}(t)
	\underset{\frac{1}{2}<\frac{T_h}{T}\leq \frac{2}{3}}{=}
	\begin{cases}
		A^2(-2t-T) &,-\frac{T}{2}\leq t\leq  T_h-T, \\
		A^2(-t-T_h)&,T_h-T\leq t \leq -\frac{T_h}{2},\\
		A^2t &,-\frac{T_h}{2}\leq t \leq +\frac{T_h}{2},\\
		A^2(-t+T_h) &,+\frac{T_h}{2}\leq t \leq T-T_h, \\
		A^2(-2t+T) &,T-T_h \leq t \leq +\frac{T}{2}.
	\end{cases}
\end{align}
%
%
Wir berechnen nun die Koeffizienten. Wie gewohnt fangen wir mit dem Gleichanteil an.
%
%
\begin{align}
	c_0&=\int_{t_0}^{t_0+T}x_{h}(t)\diff t=
	\int_{-\frac{T}{2}}^{+\frac{T}{2}}x_{a}(t)\circledast x_{b}(t)\diff t \nonumber \\
	&=\int_{-\frac{T}{2}}^{T_h-T}A^2(-2t-T)\diff t+\int_{T_h-T}^{-\frac{T_h}{2}}A^2(-t-T_h)\diff t
	+\int_{-\frac{T_h}{2}}^{+\frac{T_h}{2}}A^2t\diff t
	+\int_{+\frac{T_h}{2}}^{T-T_h}A^2(-t+T_h)\diff t\nonumber \\
	&+\int_{T-T_h}^{+\frac{T}{2}}A^2(-2t+T)\diff t\nonumber \\
	&=A^2(-t^2-Tt)\Bigg |_{t=-\frac{T}{2}}^{t=T_h-T}
	+A^2\left(-\frac{t^2}{2}-T_ht\right)\Bigg |_{t=T_h-T}^{t=-\frac{T_h}{2}}
	+A^2\frac{t^2}{2}\Bigg |_{t=-\frac{T_h}{2}}^{t=+\frac{T_h}{2}}
	+A^2\left(-\frac{t^2}{2}+T_ht\right)\Bigg |_{t=+\frac{T_h}{2}}^{T-T_h}\nonumber \\
	&+A^2(-t^2+Tt)\Bigg |_{t=T-T_h}^{t=+\frac{T}{2}} \nonumber \\
	&=A^2 \ured{-T_h^2}+\ublue{2T_hT}-\ugreen{T^2}-\uorange{TT_h}+\ugreen{T^2}+\umagenta{\frac{T^2}{4}}-\ublack{\frac{T^2}{2}}-\ubrown{\frac{T_h^2}{8}}+\frac{T_h^2}{2}+\frac{T_h^2}{2}-T_hT+\ublack{\frac{T^2}{2}}+T_h^2-TT_h\nonumber \\
	&+\frac{T_h^2}{8}-\frac{T_h^2}{8}-\frac{T^2}{2}+T_hT-\frac{T_h^2}{2}+T_hT-T_h^2+\ubrown{\frac{T_h^2}{8}}-\frac{T_h^2}{2}-\umagenta{\frac{T^2}{4}}+\frac{T^2}{2}+T^2-\ublue{2T_hT}+\ured{T_h^2}-T^2+\uorange{TT_h}\nonumber \\
	&=A^2\ured{\frac{T_h^2}{2}}+\ublue{\frac{T_h^2}{2}}-\ugreen{T_hT}+\uorange{T_h^2}-\umagenta{TT_h}+\ubrown{\frac{T_h^2}{8}}-\ubrown{\frac{T_h^2}{8}}-\frac{T^2}{2}+\ugreen{T_hT}-\ured{\frac{T_h^2}{2}}+\umagenta{T_hT}-\uorange{T_h^2}-\ublue{\frac{T_h^2}{2}}+\frac{T^2}{2}+T^2-T^2 \nonumber \\
	&=A^2\left [-\ured{\frac{T^2}{2}}+\ured{\frac{T^2}{2}}+\ublue{T^2}-\ublue{T^2}\right] = 0
\end{align}
%
%
%
Es folgt die Berechnung der Koeffizienten $c_k$ für $k\neq 0$.
%
%
%
\begin{align}
	c_k&=\int_{t_0}^{t_0+T}x_{h}(t)\e^{-\im k \omega t}\diff t
	=\int_{-\frac{T}{2}}^{+\frac{T}{2}}x_{a}(t)\circledast x_{b}(t)\e^{-\im k \omega t}\diff t\nonumber \\
	&=\underbrace{\int_{-\frac{T}{2}}^{T_h-T}A^2(-2t-T)\e^{-\im k \omega t}\diff t}_A+\underbrace{\int_{T_h-T}^{-\frac{T_h}{2}}A^2(-t-T_h)\e^{-\im k \omega t}\diff t}_B+\underbrace{\int_{-\frac{T_h}{2}}^{+\frac{T_h}{2}}A^2t\e^{-\im k \omega t}\diff t}_C\nonumber \\
	&+\underbrace{\int_{+\frac{T_h}{2}}^{T-T_h}A^2(-t+T_h)\e^{-\im k \omega t}\diff t}_D +\underbrace{\int_{T-T_h}^{+\frac{T}{2}}A^2(-2t+T)\e^{-\im k \omega t}\diff t}_E
\end{align}
%
%
%
\begin{align}
	A&=\int_{-\frac{T}{2}}^{T_h-T}A^2(-2t-T)\e^{-\im k \omega t}\diff t=\frac{A^2(-2t-T)}{-\im k \omega }\e^{-\im k \omega t}\Bigg |_{t=-\frac{T}{2}}^{t=T_h-T}-\int_{-\frac{T}{2}}^{T_h-T}\frac{2A^2}{\im k \omega}\e^{-\im k \omega t}\diff t\nonumber \\
	&=\frac{A^2}{\im k \omega}(2T_h-T)\e^{-\im k \omega (T_h-T)}-\frac{2A^2}{k^2\omega^2}\e^{-\im k \omega t}\Bigg |_{t=-\frac{T
		}{2}}^{t=T_h-T}\nonumber \\
	&=\frac{A^2}{\im k \omega}(2T_h-T)\e^{-\im k \omega (T_h-T)}-\frac{2A^2}{k^2\omega^2}\e^{-\im k \omega (T_h-T)}+\frac{2A^2}{k^2\omega^2}\e^{+\im k \omega \frac{T}{2}}
\end{align}
%
%
%
\begin{align}
	B&=\int_{T_h-T}^{-\frac{T_h}{2}}A^2(-t-T_h)\e^{-\im k \omega t}\diff t=\frac{A^2(-t-T_h)}{-\im k \omega}\e^{-\im k \omega t}\e^{-\im k \omega t}\Bigg |_{t=T_h-T}^{t=-\frac{T_h}{2}}-\int_{T_h-T}^{-\frac{T_h}{2}}\frac{A^2}{\im k \omega}\e^{-\im k \omega t}\diff t\nonumber \\
	&=\frac{A^2T_h}{2\im k \omega}\e^{+\im k \omega \frac{T_h}{2}}+\frac{A^2(-2T_h+T)}{\im k \omega}\e^{-\im k \omega (T_h-T)}-\frac{A^2}{k^2\omega^2}\e^{+\im k \omega \frac{T_h}{2}}+\frac{A^2}{k^2\omega^2}\e^{-\im k \omega (T_h-T)}
\end{align}
%
%
%
\begin{align}
	C&=\int_{-\frac{T_h}{2}}^{+\frac{T_h}{2}}A^2t\e^{-\im k \omega }\diff t=\frac{A^2t}{-\im k \omega t}\e^{-\im k \omega t}\Bigg |_{t=-\frac{T_h}{2}}^{t=+\frac{T_h}{2}}+\int_{-\frac{T_h}{2}}^{+\frac{T_h}{2}}\frac{A^2}{\im k \omega}\e^{-\im k \omega t}\diff t\nonumber \\
	&=-\frac{A^2T_h}{2\im k \omega}\e^{-\im k \omega \frac{T_h}{2}}-\frac{A^2T_h}{2\im k \omega}\e^{+\im k \omega \frac{T_h}{2}}+\frac{A^2}{k^2\omega^2}\e^{-\im k \omega \frac{T_h}{2}}-\frac{A^2}{k^2\omega^2}\e^{+\im k \omega \frac{T_h}{2}}
\end{align}
%
%
%
\begin{align}
	D&=\int_{+\frac{T_h}{2}}^{T-T_h}A^2(-t+T_h)\e^{-\im k \omega t}\diff t=\frac{A^2(-t+T_h)}{-\im k \omega}\e^{-\im k \omega t}\Bigg |_{t=+\frac{T_h}{2}}^{t=T-T_h}-\int_{+\frac{T_h}{2}}^{T-T_h}\frac{A^2}{\im k \omega}\e^{-\im k \omega t}\diff t\nonumber \\
	&=\frac{A^2T_h}{2\im k \omega}\e^{-\im k \omega \frac{T_h}{2}}-\frac{A^2(2T_h-T)}{\im k \omega}\e^{-\im k \omega (T-T_h)}-\frac{A^2}{k^2\omega^2}\e^{-\im k \omega (T-T_h)}+\frac{A^2}{ k^2 \omega^2}\e^{-\im k \omega \frac{T_h}{2}}
\end{align}
%
%
%
\begin{align}
	E&=\int_{T-T_h}^{+\frac{T}{2}}A^2(-2t+T)\e^{-\im k \omega t}\diff t=\frac{A^2(-2t+T)}{-\im k \omega}\e^{-\im k \omega t}\Bigg |_{t=T-T_h}^{+\frac{T}{2}}\int_{T-T_h}^{+\frac{T}{2}}\frac{2A^2}{\im k \omega}\e^{-\im k \omega t}\diff t\nonumber \\
	&=\frac{A^2}{\im k \omega}(2T_h-T)\e^{-\im k \omega (T-T_h)}-\frac{2A^2}{k^2\omega^2}\e^{-\im k \omega \frac{T}{2}}+\frac{2A^2}{k^2\omega^2}\e^{-\im k \omega (T-T_h)}
\end{align}
%
%
%
\begin{align}
	A+B&=\ured{\frac{A^2}{\im k \omega}(2T_h-T)\e^{-\im k \omega (T_h-T)}}-\ublue{\frac{2A^2}{k^2\omega^2}\e^{-\im k \omega (T_h-T)}}+\frac{2A^2}{k^2\omega^2}\e^{+\im k \omega \frac{T}{2}}+\frac{A^2T_h}{2\im k \omega}\e^{+\im k \omega \frac{T_h}{2}}\nonumber \\
	&+\ured{\frac{A^2(-2T_h+T)}{\im k \omega}\e^{-\im k \omega (T_h-T)}}-\frac{A^2}{k^2\omega^2}\e^{+\im k \omega \frac{T_h}{2}}+\ublue{\frac{A^2}{k^2\omega^2}\e^{-\im k \omega (T_h-T)}}\nonumber \\
	&=\frac{2A^2}{k^2\omega^2}\e^{+\im k \omega \frac{T}{2}}+\frac{A^2T_h}{2\im k \omega}\e^{+\im k \omega \frac{T_h}{2}}-\frac{A^2}{k^2\omega^2}\e^{+\im k \omega\frac{T_h}{2}}-\frac{A^2}{k^2\omega^2}\e^{-\im k \omega (T_h-T)}
\end{align}
%
%
%
\begin{align}
	(D+E)&=\frac{A^2T_h}{2\im k \omega}\e^{-\im k \omega \frac{T_h}{2}}-\ured{\frac{A^2(2T_h-T)}{\im k \omega}\e^{-\im k \omega (T-T_h)}}-\ublue{\frac{A^2}{k^2\omega^2}\e^{-\im k \omega (T-T_h)}}+\frac{A^2}{ k^2 \omega^2}\e^{-\im k \omega \frac{T_h}{2}}\nonumber \\
	&+\ured{\frac{A^2}{\im k \omega}(2T_h-T)\e^{-\im k \omega (T-T_h)}}-\frac{2A^2}{k^2\omega^2}\e^{-\im k \omega \frac{T}{2}}+\ublue{\frac{2A^2}{k^2\omega^2}\e^{-\im k \omega (T-T_h)}}\nonumber\\
	&=\frac{A^2T_h}{2\im k \omega}\e^{-\im k \omega \frac{T_h}{2}}+\frac{A^2}{ k^2 \omega^2}\e^{-\im k \omega \frac{T_h}{2}}-\frac{2A^2}{k^2\omega^2}\e^{-\im k \omega \frac{T}{2}}+\frac{A^2}{k^2\omega^2}\e^{-\im k \omega (T-T_h)}
\end{align}
%
%
%
\begin{align}
	(A+B)+C+(D+E)&=\frac{2A^2}{k^2\omega^2}\e^{+\im k \omega \frac{T}{2}}+\ured{\frac{A^2T_h}{2\im k \omega}\e^{+\im k \omega \frac{T_h}{2}}}-\umagenta{\frac{A^2}{k^2\omega^2}\e^{+\im k \omega\frac{T_h}{2}}}-\frac{A^2}{k^2\omega^2}\e^{-\im k \omega (T_h-T)}\nonumber \\
	&+\ublue{\frac{A^2T_h}{2\im k \omega}\e^{-\im k \omega \frac{T_h}{2}}}+\uorange{\frac{A^2}{ k^2 \omega^2}\e^{-\im k \omega \frac{T_h}{2}}}-\frac{2A^2}{k^2\omega^2}\e^{-\im k \omega \frac{T}{2}}+\frac{A^2}{k^2\omega^2}\e^{-\im k \omega (T-T_h)}\nonumber \\
	&-\ublue{\frac{A^2T_h}{2\im k \omega}\e^{-\im k \omega \frac{T_h}{2}}}-\ured{\frac{A^2T_h}{2\im k \omega}\e^{+\im k \omega \frac{T_h}{2}}}+\uorange{\frac{A^2}{k^2\omega^2}\e^{-\im k \omega \frac{T_h}{2}}}-\umagenta{\frac{A^2}{k^2\omega^2}\e^{+\im k \omega \frac{T_h}{2}}}\nonumber \\
	&=\frac{A^2}{k^2\omega^2}\left [2\e^{+\im k \omega\frac{T}{2}}-2\e^{-\im k \omega \frac{T}{2}}-\left (2\e^{+\im k \omega \frac{T_h}{2}}-2\e^{-\im k \omega \frac{T_h}{2}}\right )+\e^{+\im k \omega(T-h-T)}-\e^{-\im k \omega(T_h-T)}\right ]\nonumber \\
	&=\frac{A^2}{k^2\omega^2}\left [\underbrace{4\im\sin\left(k\frac {2\pi}{T}\frac{T}{2}\right)}_{=0}-4\im\sin\left(k \frac{2\pi}{T}\frac{T_h}{2}\right)+2\im\sin\left(k\frac{2\pi}{T}T_h-2k\pi\frac{T}{T}\right) \right ]\nonumber \\
	&=\frac{A^2T^2}{k^24\pi^2}\left [2\im\sin\left(2\pi k \frac{T_h}{T}\right)-4\im\sin\left(k\pi\frac{T_h}{T}\right)\right ]\nonumber \\
	&=\frac{A^2T^2}{2k^2\pi^2}\left [\im\sin\left(k2\pi\frac{T_h}{T}\right)-2\im\sin\left(k\pi\frac{T_h}{T}\right)\right ]
\end{align}
%
%
%
\begin{align}
	c_k=\im\frac{A^2T}{2k^2\pi^2}\left [\sin\left(k2\pi\frac{T_h}{T}\right)-2\sin\left(k\pi\frac{T_h}{T}\right)\right]=\e^{-\im\frac{\pi}{2}}A^2T_h^2\frac{T_h}{2T}k\pi\sinc\left(k\pi\frac{T_h}{T}\right)\sinc^2\left(k\pi\frac{T_h}{2T}\right)
\end{align}
%
%
Interessanterweise sind die Koeffizienten aus \textbf{Fall 1} und
\textbf{Fall 2} gleich.
%
Das könnte bedeuten, dass die Koeffizienten nur von den Koeffizienten von
$x_{a}(t)$ und $x_{b}(t)$ abhängen, aber nicht vom Verhältnis $\frac{T_h}{T}$.
%
\\
\textbf{Fall 3}
Es gilt nun $\frac{2}{3} < \frac{T_h}{T} < 1$. Wie immer bilden wir zunächst die
Grenzen für die Faltung:
%
%
\begin{align}
	x_{h}(t)=x_{a}(t)\circledast x_{b}(t)
	\underset{\frac{2}{3}<\frac{T_h}{T}\leq\frac{2}{3}}{=}
	\begin{cases}
		y_{11}(t)&,-\frac{T}{2}\leq t \leq -\frac{T_h}{2}, \\
		y_{12}(t)&,-\frac{T_h}{2}\leq t \leq T_h-T, \\
		y_{13}(t)&,T_h-T\leq t \leq 0, \\
		y_{14}(t)&,0\leq t \leq T-T_h, \\
		y_{15}(t)&,T-T_h\leq t \leq +\frac{T_h}{2}, \\
		y_{16}(t)&,+\frac{T_h}{2}\leq t \leq +\frac{T}{2}. \\
	\end{cases}
\end{align}
%
%
\begin{align}
	y_{11}(t)=\int_{-\frac{T_h}{2}+t}^{-T+\frac{T_h}{2}}A^2\diff \tau+\int_{-\frac{T_h}{2}}^{+\frac{T_h}{2}+t}-A^2\diff \tau=A^2(-2t-T)
\end{align}
%
%
\begin{align}
	y_{12}(t)&=\int_{-\frac{T_h}{2}+t}^{-T+\frac{T_h}{2}}A^2\diff \tau-\frac{A^2T_h}{2}+\int_0^{+\frac{T_h}{2}+t}A^2\diff \tau\nonumber \\
	&=A^2\left(-\frac{T_h}{2}-T+\frac{T_h}{2}+\frac{T_h}{2}-t+\frac{T_h}{2}+t\right)=A^2(T_h-T)
\end{align}
%
%
\begin{align}
	y_{13}(t)=-\frac{A^2T_h}{2}+\int_0^{\frac{T_h}{2}+t}A^2\diff \tau=A^2\left(-\frac{T_h}{2}+\frac{T_h}{2}+t\right)=A^2t
\end{align}
%
%
\begin{align}
	y_{14}(t)=\int_{-\frac{T_h}{2}+t}^{0}-A^2\diff \tau+\frac{A^2T_h}{2}=A^2\left(-\frac{T_h}{2}+t+\frac{T_h}{2}\right)=A^2t
\end{align}
%
%
\begin{align}
	y_{15}(t)&=\int_{-\frac{T_h}{2}+t}^0-A^2\diff \tau+\frac{A^2T_h}{2}+\int_{T-\frac{T_h}{2}}^{\frac{T_h}{2}+t}-A^2\diff \tau\nonumber \\
	&=A^2\left(-\frac{T_h}{2}+t+\frac{T_h}{2}-\frac{T_h}{2}-t+T-\frac{T_h}{2}\right)=A^2(T-T_h)
\end{align}
%
%
\begin{align}
	y_{16}(t)=\int_{-\frac{T_h}{2}+t}^{+\frac{T_h}{2}}A^2\diff \tau+\int_{T-\frac{T_h}{2}}^{+\frac{T_h}{2}+t}-A^2\diff \tau=A^2(-2t+T)
\end{align}
%
%
Wir schreiben es noch einmal komplett auf:
%
%
\begin{align}
	x_{\mathring{h}}(t)=x_{a}(t)\circledast x_{b}(t)
	=
	\begin{cases}
		A^2(-2t-T)&,-\frac{T}{2}\leq t \leq -\frac{T_h}{2}, \\
		A^2(T_h-T)&,-\frac{T_h}{2}\leq t \leq T_h-T, \\
		A^2t &,T_h-T\leq t \leq T-T_h, \\
		A^2(T-T_h)&,T-T_h\leq t \leq +\frac{T_h}{2}, \\
		A^2(-2t+T)&,+\frac{T_h}{2}\leq t \leq +\frac{T}{2}.
	\end{cases}
\end{align}
%
%
Es folgt wieder die Berechnung der Koeffizienten $c_k$, zunächst der
Gleichanteil, danach für $k\neq 0$.
%
%
\begin{align}
	c_0&=\int_{t_0}^{t_0+T}x_{h}(t)\diff t
	=\int_{-\frac{T}{2}}^{+\frac{T}{2}}x_{a}(t)\circledast x_{b}(t)\diff t\nonumber \\
	&=\int_{-\frac{T}{2}}^{-\frac{T_h}{2}}A^2(-2t-T)\diff t+\int_{-\frac{T_h}{2}}^{T_h-T}A^2(T_h-T)\diff t+\int_{T_h-T}^{T-T_h}A^2t\diff t\nonumber \\
	&+\int_{T-T_h}^{+\frac{T_h}{2}}A^2(T-T_h)\diff t+\int_{+\frac{T_h}{2}}^{+\frac{T}{2}}A^2(-2t+T)\diff t\nonumber \\
	&=A^2\Bigg[\left(-t^2-Tt\right)\Bigg |_{t=-\frac{T}{2}}^{-\frac{T_h}{2}}+\left(T_ht-Tt\right)\Bigg |_{t-\frac{T_h}{2}}^{t=T_h-T}+\left(\frac{t^2}{2}\right)\Bigg |_{t=T_h-T}^{T-T_h}\nonumber \\
	&+\left(Tt-T_ht\right)\Bigg |_{t=T-T_h}^{t=+\frac{T_h}{2}}+\left(-t^2+Tt\right)\Bigg |_{t=+\frac{T_h}{2}}^{+\frac{T}{2}}\Bigg]\nonumber \\
	%
	%
	=A^2\Bigg [&\underbrace{-\frac{T_h^2}{4}+\frac{TT_h}{2}+\frac{T^2}{4}-\frac{T^2}{2}}_{-\frac{T_h^2}{4}+\frac{TT_h}{2}-\frac{T^2}{4}}\nonumber \\
	&\underbrace{+T_h^2-TT_h-TT_h+T^2+\frac{T_h^2}{2} -\frac{TT_h}{2}}_{T^2+\frac{3T_h^2}{2}-\frac{5TT_h}{2}}\nonumber \\
	&\underbrace{+\frac{T^2}{2}-TT_h+\frac{T_h^2}{2}-\frac{T_h^2}{2}+TT_h-\frac{T^2}{2}}_{0}\nonumber \\
	&\underbrace{+\frac{TT_h}{2}-\frac{T_h^2}{2}-T^2+TT_h+TT_h-T_h^2}_{-T^2-\frac{3T_h^2}{2}+\frac{5TT_h}{2}}\nonumber \\
	&\underbrace{-\frac{T^2}{4}+\frac{T^2}{2}+\frac{T_h^2}{4}-\frac{TT_h}{2}}_{\frac{T_h^2}{4}-\frac{TT_h}{2}+\frac{T^2}{4}}\Bigg ]\nonumber \\
	&= 0
\end{align}
%
%
%
\begin{align}
	c_k&=\int_{t_0}^{t_0+T}x_{h}(t)\e^{-\im k \omega t}\diff t=\frac{1}{T}\int_{-\frac{T}{2}}^{+\frac{T}{2}}x_{a}(t)\circledast x_{b}(t)\e^{-\im k \omega t}\diff t\nonumber \\
	&=\underbrace{\int_{-\frac{T}{2}}^{-\frac{T_h}{2}}A^2(-2t-T)\e^{-\im k \omega t}\diff t}_A+\underbrace{\int_{-\frac{T_h}{2}}^{T_h-T}A^2(T_h-T)\e^{-\im k \omega t}\diff t}_B+\underbrace{\int_{T_h-T}^{T-T_h}A^2t\e^{-\im k \omega t}\diff t}_C\nonumber \\
	&+\underbrace{\int_{T-T_h}^{+\frac{T_h}{2}}A^2(T-T_h)\e^{-\im k \omega t}\diff t}_D+\underbrace{\int_{+\frac{T_h}{2}}^{+\frac{T}{2}}A^2(-2t+T)\e^{-\im k \omega t}\diff t}_E
\end{align}
%
%
%
\begin{align}
	A&=\int_{-\frac{T}{2}}^{-\frac{T_h}{2}}A^2(-2t-T)\e^{-\im k \omega t}\diff t=\frac{A^2(-2t-T)}{-\im k \omega}\e^{-\im k \omega t}\Bigg |_{t=-\frac{T}{2}}^{t=-\frac{T_h}{2}}-\int_{-\frac{T}{2}}^{-\frac{T_h}{2}}\frac{2A^2}{\im k \omega}\e^{-\im k \omega t}\diff t\nonumber \\
	&=\frac{A^2(T_h-T)}{-\im k \omega}\e^{+\im k \omega \frac{T_h}{2}}+\frac{A^2(T-T)}{\im k \omega}\e^{+\im k \omega \frac{T}{2}}-\frac{2A^2}{k^2\omega^2}\e^{\im k \omega \frac{T_h}{2}}+\frac{2A^2}{k^2\omega^2}\e^{+\im k \omega \frac{T}{2}}\nonumber \\
	&=-\frac{A^2(T_h-T)}{\im k \omega}\e^{+\im k \omega \frac{T_h}{2}}-\frac{2A^2}{k^2\omega^2}\e^{+\im k \omega \frac{T_h}{2}}+\frac{2A^2}{k^2\omega^2}\e^{+\im k \omega \frac{T}{2}}
\end{align}
%
%
%
\begin{align}
	B&=\int_{-\frac{T_h}{2}}^{T_h-T}A^2(T_h-T)\e^{-\im k \omega t}\diff t = \frac{A^2(T_h-T)}{-\im k \omega}\e^{-\im k \omega t}\Bigg |_{t=-\frac{T_h}{2}}^{T_h-T}\nonumber \\
	&=\frac{A^2(T_h-T)}{\im k \omega}\e^{+\im k \omega \frac{T_h}{2}}-\frac{A^2(T_h-T)}{\im k \omega}\e^{-\im k \omega (T_h-T)}
\end{align}
%
%
%
\begin{align}
	C&=\int_{T_h-T}^{T-T_h}A^2t\e^{-\im k \omega t}\diff t=\frac{A^2t}{-\im k \omega }\e^{-\im k \omega t}\Bigg |_{t=T_h-T}^{t=T-T_h}+\int_{T_h-T}^{T-T_h}\frac{A^2}{\im k \omega}\e^{-\im k \omega t}\diff t\nonumber \\
	&=\frac{A^2(T_h-T)}{\im k \omega}\e^{-\im k \omega (T_h-T)}-\frac{A^2(T-T_h)}{\im k \omega}\e^{-\im k \omega (T-T_h)}+\frac{A^2}{k^2\omega^2}\e^{-\im k \omega (T-T_h)}-\frac{A^2}{k^2\omega^2}\e^{-\im k \omega (T_h-T)}
\end{align}
%
%
%
\begin{align}
	D&=\int_{T-T_h}{+\frac{T_h}{2}}A^2(T-T_h)\e^{-\im k \omega t}\diff t=\frac{A^2(T-T_h)}{-\im k \omega}\e^{-\im k \omega t}\Bigg |_{t=T-T_h}^{t=+\frac{T_h}{2}}\nonumber \\
	&=\frac{A^2(T-T_h)}{\im k \omega}\e^{-\im k \omega(T-T_h)}-\frac{A^2(T-T_h)}{\im k \omega}\e^{-\im k \omega \frac{T_h}{2}}
\end{align}
%
%
%
\begin{align}
	E&=\int_{+\frac{T_h}{2}}^{+\frac{T}{2}}A^2(-2t+T)\e^{-\im k \omega t}\diff t=\frac{A^2(-2t+T)}{-\im k \omega }\e^{-\im k \omega t}\Bigg |_{t=+\frac{T_h}{2}}^{t=+\frac{T}{2}}-\int_{+\frac{T_h}{2}}^{+\frac{T}{2}}\frac{2A^2}{\im k \omega}\e^{-\im k \omega t}\diff t\nonumber \\
	&=\frac{A^2(-T_h+T)}{\im k \omega}\e^{-\im k \omega \frac{T_h}{2}}-\frac{A^2(-T+T)}{\im k \omega}\e^{-\im k \omega \frac{T}{2}}-\frac{2A^2}{k^2\omega^2}\e^{-\im k \omega \frac{T}{2}}+\frac{2A^2}{k^2\omega^2}\e^{-\im k \omega \frac{T_h}{2}}\nonumber \\
	&=\frac{A^2(-T_h+T)}{\im k \omega}\e^{-\im k\omega \frac{T_h}{2}}-\frac{2A^2}{k^2\omega^2}\e^{-\im k \omega \frac{T}{2}}+\frac{2A^2}{k^2\omega^2}\e^{-\im k \omega \frac{T_h}{2}}
\end{align}
%
%
%
\begin{align}
	A+B&=\ured{-\frac{A^2(T_h-T)}{\im k \omega}\e^{+\im k \omega \frac{T_h}{2}}}-\frac{2A^2}{k^2\omega^2}\e^{+\im k \omega \frac{T_h}{2}}+\frac{2A^2}{k^2\omega^2}\e^{+\im k \omega \frac{T}{2}}\nonumber \\
	&\ured{+\frac{A^2(T_h-T)}{\im k \omega}\e^{+\im k \omega \frac{T_h}{2}}}-\frac{A^2(T_h-T)}{\im k \omega}\e^{-\im k \omega (T_h-T)}\nonumber \\
	&=-\frac{2A^2}{k^2\omega^2}\e^{+\im k \omega \frac{T_h}{2}}+\frac{2A^2}{k^2\omega^2}\e^{+\im k \omega \frac{T}{2}}-\frac{A^2(T_h-T)}{\im k \omega}\e^{-\im k \omega (T_h-T)}
\end{align}
%
%
%
\begin{align}
	D+E&=\frac{A^2(T-T_h)}{\im k \omega}\e^{-\im k \omega(T-T_h)}-\ured{\frac{A^2(T-T_h)}{\im k \omega}\e^{-\im k \omega \frac{T_h}{2}}}\nonumber \\
	&\ured{+\frac{A^2(-T_h+T)}{\im k \omega}\e^{-\im k\omega \frac{T_h}{2}}}-\frac{2A^2}{k^2\omega^2}\e^{-\im k \omega \frac{T}{2}}+\frac{2A^2}{k^2\omega^2}\e^{-\im k \omega \frac{T_h}{2}}\nonumber \\
	&=\frac{A^2(T-T_h)}{\im k \omega}\e^{-\im k \omega(T-T_h)}-\frac{2A^2}{k^2\omega^2}\e^{-\im k \omega \frac{T}{2}}+\frac{2A^2}{k^2\omega^2}\e^{-\im k \omega \frac{T_h}{2}}
\end{align}
%
%
%
\begin{align}
	(A+B)+C+(D+E)&=-\frac{2A^2}{k^2\omega^2}\e^{+\im k \omega \frac{T_h}{2}}+\frac{2A^2}{k^2\omega^2}\e^{+\im k \omega \frac{T}{2}}\ured{-\frac{A^2(T_h-T)}{\im k \omega}\e^{-\im k \omega (T_h-T)}}\nonumber \\
	&\umagenta{+\frac{A^2(T-T_h)}{\im k \omega}\e^{-\im k \omega(T-T_h)}}-\frac{2A^2}{k^2\omega^2}\e^{-\im k \omega \frac{T}{2}}+\frac{2A^2}{k^2\omega^2}\e^{-\im k \omega \frac{T_h}{2}}\nonumber\\
	&\ublue{+\frac{A^2(T_h-T)}{\im k \omega}\e^{-\im k \omega (T_h-T)}}\umagenta{-\frac{A^2(T-T_h)}{\im k \omega}\e^{-\im k \omega (T-T_h)}}\nonumber \\
	&+\frac{A^2}{k^2\omega^2}\e^{-\im k \omega (T-T_h)}-\frac{A^2}{k^2\omega^2}\e^{-\im k \omega (T_h-T)}\nonumber \\
	&=\frac{A^2}{k^2\omega^2}\Bigg [-2\left (\e^{+\im k \omega \frac{T_h}{2}}-\e^{-\im k \omega \frac{T_h}{2}}\right )+2\left (\e^{+\im k \omega \frac{T}{2}}-\e^{-\im k \omega \frac{T}{2}}\right )\nonumber \\
	&\left (\e^{+\im k \omega (T_h-T)}-\e^{-\im k \omega (T_h-T)}\right )\Bigg ]\nonumber \\
	&=\frac{A^2T^2}{k^24\pi^2}\left [-4\im \sin\left(k\frac{2\pi}{T}\frac{T_h}{2}\right)+4\im\underbrace{\sin\left(k\frac{2\pi}{T}\frac{T}{2}\right)}_{=0}+2\im\sin\left(k\frac{2\pi}{T}T_h-k\frac{2\pi}{T}T\right)\right ]\nonumber \\
	&=\im\frac{A^2T^2}{2k^2\pi^2}\left [\sin\left(2\pi k\frac{T_h}{T}\right)-2\sin\left(k\pi \frac{T_h}{T}\right)\right ]
\end{align}
%
%
%
\begin{align}
	c_k=\im\frac{A^2T^2}{2k^2\pi^2}\left [\sin\left(k2\pi\frac{T_h}{T}\right)-2\sin\left(k\pi\frac{T_h}{T}\right)\right ]
	=\e^{-\im\frac{\pi}{2}}A^2T_h^2\frac{T_h}{2T}k\pi\sinc\left(k\pi\frac{T_h}{T}\right)\sinc^2\left(k\pi\frac{T_h}{2T}\right)
\end{align}
%
\textbf{Fall 4}
%
Zum Schluss schauen wir uns noch kurz an, was passiert, wenn $\frac{T_h}{T}=1$
gilt.
%
Dann ist nämlich $x_{a}(t)$ eine Konstante und wir integrieren im
Faltungsintegral
%
\begin{align}
	x_{h}(t)=x_{a}(t)\circledast x_{b}(t)
	=\int_{t_0}^{t_0+T}x_{a}(t-\tau)x_{b}(\tau)\diff \tau
	=\int_{t_0}^{t_0+T}1\cdot x_{b}(t)\diff \tau = 0
\end{align}
%
nur eine periodische Funktion über eine Periode.
%
Daraus ergibt sich der Gleichanteil, skaliert um die Periode $T$: $c_0T$. Im Fall $x_b(t)$ ist $T\cdot c_0=0$, denn $c_0=0$.\
Die Fourierkoeffizienten sind also alle $0$.
%
%
\begin{align}
	c_k\underset{\frac{T_h}{T}=1}{=}0
\end{align}
%
%
Die Koeffizienten haben nun für alle $4$ Fälle die gleiche Form (für
$\frac{T_h}{T}=1$ steht in den Sinus-Termen nur noch ein Vielfaches von $\pi$).
%
Wir vermuten daher an dieser Stelle, dass es einen Zusammenhang zwischen
$x_{a}(t)$, $x_{b}(t)$ und $c_k$ gibt.
%
Tatsächlich ergibt sich der Fourierkoeffizient einer Faltung zweier gleich
periodischer Funktionen aus der Multiplikation der Koeffizienten (in \cite[Kap. 3, S. 208]{Oppenheim1997} wird das Produkt noch mit $T$ skaliert, denn die Koeffizienten werden im Gegensatz zu hier mit dem Faktor $\frac{1}{T}$ versehen, während wir diesen erst wieder bei der Synthese der Fourierreihe beachten).
%
Dies ist auch bei uns der Fall.
%
Wir hätten uns also das (lästige) Falten sparen und direkt die Koeffizienten
multiplizieren können. Wir halten also als nächste Korrespondenz folgendes fest:
%
\begin{align}
	x(t)\circledast y(t)\quad\fourier\quad X_k\cdot Y_k
\end{align}
%
Diese Korrespondenz ist sehr wichtig in der Signal- und Systemtheorie und wird
uns später noch oft über den Weg laufen.
%
Wir wollen uns an dieser Stelle die Zeit nehmen und die Korrektheit der
Korrespondenz zeigen.
%
%
\begin{align}
	y(t)&=x(t)\circledast h(t)
	=\int_{-\frac{T}{2}}^{\frac{T}{2}}x(\tau)h(t-\tau)\diff \tau
	=\int_{-\frac{T}{2}}^{\frac{T}{2}}\left (\frac{1}{T}\sum_{k=-\infty}^{\infty}X_k\e^{+\im k \omega \tau}\right )\cdot \left (\frac{1}{T}\sum_{i=-\infty}^{+\infty}H_i\e^{+\im i \omega (t-\tau)}\right )\diff \tau\nonumber \\
	&=\frac{1}{T^2}\int_{-\frac{T}{2}}^{\frac{T}{2}}\sum_{k=-\infty}^{+\infty}\sum_{i=-\infty}^{+\infty}\left (X_kH_i\e^{+\im k \omega \tau}\e^{+\im i \omega (t-\tau)}\right )\diff \tau
	=\frac{1}{T^2}\sum_{k=-\infty}^{+\infty}\sum_{i=-\infty}^{+\infty}\left (\int_{-\frac{T}{2}}^{\frac{T}{2}}X_kH_i\e^{+\im (k-i) \omega \tau}\e^{+\im i \omega t}\diff \tau\right )\nonumber \\
	&=\frac{1}{T^2}\sum_{k=-\infty}^{+\infty}\sum_{i=-\infty}^{+\infty}\left (X_kH_i\e^{-\im i \omega t}\int_{-\frac{T}{2}}^{+\frac{T}{2}}\e^{+\im (k-i)\omega t}\diff \tau \right )
\end{align}
%
%
Für $k=i$ ergibt das Integral $T$.
%
%
\begin{align}
	\int_{-\frac{T}{2}}^{+\frac{T}{2}}\e^{+\im (k-i)\omega \tau}\diff \tau \underset{k=i}{=}\int_{-\frac{T}{2}}^{+\frac{T}{2}}1\diff \tau =\tau \Bigg |_{\tau=-\frac{T}{2}}^{\tau=+\frac{T}{2}}=T
\end{align}
%
%
Für $k\neq i$ ergibt das Integral $0$.
%
%
\begin{align}
	\int_{-\frac{T}{2}}^{+\frac{T}{2}}\e^{+\im (k-i)\omega \tau}\diff \tau &\underset{k\neq i}{=}\frac{\e^{+\im (k-i)\omega \tau}}{+\im (k-i)\omega}\Bigg |_{\tau=-\frac{T}{2}}^{\tau=+\frac{T}{2}}
	=\frac{1}{\im (k-i)\omega }\left (\e^{+\im (k-i)\frac{2\pi}{T}\frac{T}{2}}-\e^{-\im (k-i)\frac{2\pi}{T}\frac{T}{2}}\right )\nonumber \\
	&=\frac{2}{ (k-i)\omega}\sin((k-i)\pi)=0
\end{align}
%
%
Da sich also nur die Koeffizientenpaare addieren, bei denen der Index gleich ist, können wir die Doppelsumme auch als eine einfache Summe schreiben.
%
%
\begin{align}
	\frac{1}{T^2}\sum_{k=-\infty}^{+\infty}\sum_{i=-\infty}^{+\infty}\left (X_kH_i\e^{-\im i \omega t}\int_{-\frac{T}{2}}^{+\frac{T}{2}}\e^{+\im (k-i)\omega t}\diff \tau \right )
	=\frac{1}{T^2}\sum_{k=i=-\infty}^{+\infty}TX_kH_i\e^{+\im i\omega t}
	=\frac{1}{T^2}\sum_{k=-\infty}^{+\infty}X_kH_k\e^{+\im k \omega t}
\end{align}
Somit haben wir die Korrespondenz bewiesen.
%
%
\begin{figure}
	\centering
	\begin{subfigure}{\textwidth}
		\includegraphics[width=\textwidth,keepaspectratio]{../fs/FB93908026_timedomain_10}
		\caption{Zeitbereich}
	\end{subfigure}
	\begin{subfigure}{\textwidth}
		\includegraphics[width=\textwidth,keepaspectratio]{../fs/FB93908026_frequence_10}
		\caption{Betrag und Phase der Fourierkoeffizienten}
	\end{subfigure}
	\caption{Synthese der Fourierreihe von $x_h(t)$ für $A=1$, $T=4$, $T_h=2$ und
		$-5\leq k\leq5$.}
\end{figure}
%
%
\newpage
\subsubsection{Aufgabe i) mit komplexer Fourierreihe}
%
%
In diesem Abschnitt untersuchen wir die Fourierreihe in Verbindung mit der
zeitlichen Ableitung.
%
%
\begin{align}
	x_{i}(t)=\frac{\diff x_{c}(t)}{\diff t}
	=
	\begin{cases}
		\frac{2A}{T_h}&,-\frac{T_h}{2}\leq t < 0\\
		-\frac{2A}{T_h}&,0<t\leq \frac{T_h}{2} \\
		0&,\text{ sonst}
	\end{cases}
\end{align}
%
%
Wir nutzen die aufgestellten Korrespondenz (denn $x_{i}(t)=-\frac{2}{T_h}x_{b})$
%
\begin{gather}
	Ax(t)\quad\fourier\quad AX_{k}
\end{gather}
%
und erhalten
%
\begin{align}
	x_{i}(t)=\frac{1}{T}\sum_{k=-\infty}^{+\infty}\im Ak\pi\frac{T_h}{T} \sinc^2\left(k\pi\frac{T_h}{2T}\right)\e^{\im k \omega t}
\end{align}
%
%
Mit der vorhandenen Korrespondenz konnten wir uns ziemlich viel Rechnerei
ersparen.
%
Die Koeffizienten von $x_{i}(t)$ und $x_{c}(t)$ ähneln sich sehr stark.
%
Wir überprüfen daher, ob es einen Zusammenhang zwischen den beiden gibt.
%
Dafür differenzieren wir einmal direkt die Fourierreihe von $x_{c}(t)$.
%
%
\begin{align}
	x_{i}(t)=\frac{\diff x_{c}(t)}{\diff t}
	=\frac{\diff }{\diff t}\left [\frac{1}{T}\sum_{k=-\infty}^{+\infty}X_k\e^{+\im k \omega t}\right ]
	=\frac{1}{T}\sum_{k=-\infty}^{+\infty}\frac{\diff }{\diff t}X_k\e^{+\im k \omega t}
	=\frac{1}{T}\sum_{k=-\infty}^{+\infty}\im k \omega X_k\e^{+\im k \omega t}
\end{align}
%
%
Also können wir ganz einfach die Fourierreihe einer Ableitung aufstellen,
in dem wir die Koeffizienten der zu ableitenden Funktion mit $\im k \omega$
multiplizieren.
%
Wir merken uns:
%
\begin{align}
	\frac{\diff x(t)}{\diff t}\quad\fourier\quad\im k \omega X_k.
\end{align}
%
Wir überprüfen es einmal mit den Koeffizienten von $x_c(t)$.
%
\begin{gather}
	c_k=\im k\omega \frac{AT_h}{2}\sinc^2\left(k\pi\frac{T_h}{2T}\right)
	=\im k\frac{2\pi}{T}\frac{AT_h}{2}\sinc^2\left(k\pi\frac{T_h}{2T}\right)
	=\im Ak\pi\frac{T_h}{T} \sinc^2\left(k\pi\frac{T_h}{2T}\right)
\end{gather}
%
\begin{figure}
	\centering
	\begin{subfigure}{\textwidth}
		\includegraphics[width=\textwidth,keepaspectratio]{../fs/FB93908026_timedomain_11}
		\caption{Zeitbereich}
	\end{subfigure}
	\begin{subfigure}{\textwidth}
		\includegraphics[width=\textwidth,keepaspectratio]{../fs/FB93908026_frequence_11}
		\caption{Betrag und Phase der Fourierkoeffizienten}
	\end{subfigure}
	\caption{Synthese der Fourierreihe von $x_i(t)$ für $A=1$, $T=4$, $T_h=2$ und
		$-5\leq k\leq5$.}
\end{figure}
%
\newpage
\subsubsection{Aufgabe j) mit komplexer Fourierreihe}
In diesem Abschnitt untersuchen wir, wie sich eine Zeitskalierung auf die
Fourierreihe auswirkt.
%
\begin{align}
	x_{j}(t)=x_{a}(at)\quad\quad a\in\mathbb{R}\setminus\{0\}
\end{align}
%
Dabei sind zwei Fälle zu beachten: $a>0$ und $a<0$.
%
Wir starten mit $a>0$.
%
\begin{figure}[h!]
	\centering
	\begin{tikzpicture}[domain=0:2]
		\def\T{0.4}
		\draw[->] (-3,0) -- (3,0) node[below right] {$t$};
		\draw[->] (0,0) -- (0,1.5) node[above] {$x_{j}(t)$};
		\foreach \pos in {-2,...,2} {
			\draw[-, C0, ultra thick] (\pos-\T,0) -- (\pos-\T,1) -- (\pos+\T,1) -- (\pos+\T,0) -- (\pos+1-\T,0);
		};
		\draw[-, C0, ultra thick] (-\T,0) node[below] {$\frac{-T_h}{2a}$};
		\draw[-, C0, ultra thick] (+\T,0) node[below] {$\frac{+T_h}{2a}$};
		\draw[-, C0, ultra thick] (1,0) node[below] {$\frac{T}{a}$};
		\draw[-, C0, ultra thick] (2,0) node[below] {$\frac{2T}{a}$};
		\draw[-, C0, ultra thick] (-2-\T,1) node[left] {$A$};
	\end{tikzpicture}
\end{figure}
%
Nun ist auch zu beachten, dass sich die Periode verändert hat.
%
Sie beträgt nun $\frac{T}{a}$.
%
Den Gleichanteil berechnen wir nun durch Ausrechnen des Flächeninhalts des
Rechteckes innerhalb einer Periode.
%
\begin{align}
	c_0=aA\cdot\frac{T_h}{a}=AT_h
\end{align}
%
Es folgen noch die Koeffizienten $c_k$ für $k\neq0$.
%
\begin{align}
	c_k&=a\int_{t_0}^{t_0+\frac{T}{a}}x_{j}(t)\e^{-\im k a\omega t}\diff t
	=a\int_{-\frac{T_h}{2a}}^{+\frac{T_h}{2a}}A\e^{-\im k a\omega }\diff t
	=\frac{aA}{-\im k a\omega }\e^{-\im k a \omega t}\Bigg |_{t=-\frac{T_h}{2a}}^{t=+\frac{T_h}{2a}}\nonumber \\
	&=\frac{AT}{\im k 2\pi}\left (\e^{+\im k a \frac{2\pi}{T}\frac{T_h}{2a}}-\e^{-\im k a \frac{2\pi}{T}\frac{T_h}{2a}}\right )
	=\frac{AT}{k\pi}\frac{T_h}{T_h}\sin\left(k\pi\frac{T_h}{T}\right)
	=AT_h\sinc\left(k\pi\frac{T_h}{T}\right)
\end{align}
%
%
Für $a<0$ hat die Funktion $x_{j}(t)$ die gleiche Form, weil sie gerade ist.
%
Der Gleichanteil ist gleich.
%
Die Periode beträgt nun $\frac{T}{|a|}$.
%
%
\begin{align}
	c_k&=|a|\int_{t_0}^{t_0+\frac{T}{|a|}}A\e^{-\im k |a|\omega t}\diff t
	=\frac{|a|A}{-\im k |a| \omega }\e^{-\im k |a| \omega t}\Bigg |_{t=-\frac{T_h}{2|a|}}^{t=+\frac{T_h}{2|a|}}
	=\frac{AT}{\im k 2\pi}\left (\e^{+\im k |a|\frac{2\pi}{T}\frac{T_h}{2a}}-\e^{-\im k |a|\frac{2\pi}{T}\frac{T_h}{2|a|}}\right )\nonumber\\
	&=\frac{AT}{k\pi}\frac{T_h}{T_h}\sin\left(k\pi\frac{T_h}{T}\right)=AT_h\sinc\left(k\pi\frac{T_h}{T}\right)
\end{align}
%
%
Die beiden Fourierkoeffizienten gleichen sich und stimmen sogar mit den
Koeffizienten von $x_{a}(t)$ überein.
%
Den Fall $a<0$ hätten wir gar berechnen müssen, denn das wäre das selbe wie
$a>0$ zu lassen und mit $-1$ zu multiplizieren, was eine Zeitumkehr ist,
für die wir schon eine Korrespondenz haben.
%
Wir setzen nun $a$ direkt in die Fourierreihe ein, um zu untersuchen,
ob sich auch eine Korrespondenz für Zeitskalierungen aufstellen lässt.
%
\begin{align}
	x_{j}(t)=x_{a}(at)=\frac{1}{T}\sum_{k=-\infty}^{+\infty}c_k\e^{+\im k \omega at}
\end{align}
%
Für $a>0$ müssen wir die Koeffizienten also nicht ändern, sondern nur die
Winkelgeschwindigkeit mit $a$ skalieren.
%
Für $a<0$ müssen wir die Winkelgeschwindigkeit mit $|a|$ skalieren und die
positiven Koeffizienten mit den negativen Koeffizienten vertauschen, was der
Zeitumkehr entspricht.
%
Wir stellen also nur eine Korrespondenz für $a>0$ auf, denn für $a<0$ haben wir
schon eine, die Zeitumkehr (zuerst mit $|a|$ skalieren, danach Zeitumkehr).
%
%
\begin{align}
	x(at)\quad a\in\mathbb{R}^+\setminus\{0\}\quad\fourier\quad \frac{T}{a}
\end{align}
%
%
\begin{figure}
	\centering
	\begin{subfigure}{\textwidth}
		\includegraphics[width=\textwidth,keepaspectratio]{../fs/FB93908026_timedomain_12}
		\caption{Zeitbereich}
	\end{subfigure}
	\begin{subfigure}{\textwidth}
		\includegraphics[width=\textwidth,keepaspectratio]{../fs/FB93908026_frequence_12}
		\caption{Betrag und Phase der Fourierkoeffizienten}
	\end{subfigure}
	\caption{Synthese der Fourierreihe von $x_j(t)$ für $A=1$, $T=4$, $T_h=2$ und
		$-5\leq k\leq5$.}
\end{figure}
%
\newpage
\subsubsection{Aufgabe k) mit komplexer Fourierreihe}
%
%
Die nächste Funktion unterscheidet sich doch ziemlich von den anderen, denn hier
haben wir es mit einer Konstante zu tun.
%
Wie gewohnt, berechnen wir zuerst den Gleichanteil, danach die übrigen
Koeffizienten $c_k$ für $k\neq 0$.
%
\begin{align}
	c_0=\int_{t_0}^{t_0+T}x_{k}\diff t
	=\int_{t_0}^{t_0+T}1\diff t
	=T+t_0-t_0=T
\end{align}
%
\begin{align}
	c_k&=\frac{1}{T}\int_{t_0}^{t_0+T}x_{k}\e^{-\im k \omega t}\diff t
	=\int_{t_0}^{t_0+T}1\e^{-\im k \omega t}\diff t
	=\frac{\e^{-\im k \omega t}}{-\im k \omega T}\Bigg |_{t=t_0}^{t=t_0+T}
	=\frac{1}{\im k \omega }\left (\e^{-\im k \omega t_0}-\e^{-\im k \frac{2\pi}{T}(t_0+T)}\right )\nonumber \\
	&=\frac{1}{\im k \omega}\left (\e^{-\im k \omega t_0}-\e^{-\im (k \omega t_0+k2\pi )}\right )=0
\end{align}
%
Wir stellen folgende Korrespondenz auf:
%
\begin{align}
	1 \quad \fourier \quad c_0 = T,c_k=0\quad\forall k \neq 0.
\end{align}
%
In Verbindung mit Gleichung \eqref{eq:CorrespondenceModulation} lässt sich auch
folgende Korrespondenz aufstellen:
%
\begin{align}
	\e^{+n\omega t}\quad n\in\mathbb{Z}\quad \fourier\quad c_n=T,c_k=0\quad\forall k\neq n
\end{align}
%
Wenn wir nun bspw. $\cos(t)$ mit Hilfe der Eulerschen Identität betrachten,
dann ist das bereits eine Fourierreihe.
%
\begin{align}
	\cos(t)=\frac{1}{2}\e^{+\im t}+\frac{1}{2}\e^{-\im t}=\frac{1}{2\pi}\left(\pi\e^{+\im t}+\pi\e^{-\im t}\right)
\end{align}
%
Hier bei ist $c_{-1}=c_1=\pi$ und $c_k=0\forall k \not\in\{-1,1\}$.
%
\begin{figure}
	\centering
	\begin{subfigure}{\textwidth}
		\includegraphics[width=\textwidth,keepaspectratio]{../fs/FB93908026_timedomain_13}
		\caption{Zeitbereich}
	\end{subfigure}
	\begin{subfigure}{\textwidth}
		\includegraphics[width=\textwidth,keepaspectratio]{../fs/FB93908026_frequence_13}
		\caption{Betrag und Phase der Fourierkoeffizienten}
	\end{subfigure}
	\caption{Synthese der Fourierreihe von $x_k(t)$ für $A=1$, $T=4$, $T_h=2$ und
		$-5\leq k\leq5$.}
\end{figure}
%
\newpage
\subsubsection{Aufgabe l) mit komplexer Fourierreihe}
%
Nun untersuchen wir die Multiplikation zweier $T$-periodischer Funktionen.
%
\begin{align}
	x_{l}(t)=x_{b}(t)\cdot x_{c}(t)=\begin{cases}
		-\frac{2A^2}{T_h}t-A^2&,-\frac{T_h}{2}\leq t <0 \\
		-\frac{2A^2}{T_h}t+A^2&,0\leq t \leq \frac{T_h}{2}\\
		0&,\text{ sonst}
	\end{cases}
\end{align}
%
\begin{figure}[h!]
	\centering
	\begin{tikzpicture}[domain=0:2]
		\def\T{0.4}
		\draw[->] (-3,0) -- (3,0) node[below right] {$t$};
		\draw[->] (0,0) -- (0,1.5) node[above] {$x_{l}(t)$};
		\foreach \pos in {-2,...,2} {
			\draw[-, red, ultra thick] (\pos-\T,0) --(\pos,-1)--(\pos,1) -- (\pos+\T,0) -- (\pos+1-\T,0);
		};
		\draw[-, black, ultra thick] (-\T,0) node[below] {$\frac{-T_h}{2}$};
		\draw[-, black, ultra thick] (+\T,0) node[below] {$\frac{+T_h}{2}$};
		\draw[-, black, ultra thick] (1,0) node[below] {$T$};
		\draw[-, black, ultra thick] (2,0) node[below] {$2 T$};
		\draw[-, black, ultra thick] (-2-\T,1) node[left] {$A^2$};
	\end{tikzpicture}
\end{figure}
%
Der Gleichanteil $c_0=0$, denn innerhalb einer Periode heben sich die
Flächeninhalte unter den Dreiecken gegenseitig auf.
%
%
%
\begin{align}
	c_k&=\int_{t_0+T}^{t_0+T}x_{l}(t)\e^{-\im k \omega t}\diff t= \frac{1}{T}\int_{t_0}^{t_0+T}x_{b}(t)\cdot x_{c}\e^{-\im k \omega t}\diff t\nonumber \\
	&=\int_{-\frac{T_h}{2}}^0\left(-\frac{2A^2}{T_h}t-A^2\right)\e^{-\im k \omega t}\diff t+\int_0^{+\frac{T_h}{2}}\left(-\frac{2A^2}{T_h}t+A^2\right)\e^{-\im k \omega t}\diff t\nonumber \\
	&=\frac{-\frac{2A^2}{T_h}t-A^2}{-\im k \omega}\e^{-\im k \omega t}\Bigg |_{t=-\frac{T_h}{2}}^{t=0}-\int_{-\frac{T_h}{2}}^0\frac{2A^2}{T_h\im k \omega}\e^{-\im k \omega t}\diff t +\frac{-\frac{2A^2}{T_h}t+A^2}{-\im k \omega}\e^{-\im k \omega t}\Bigg |_{t=0}^{t=+\frac{T_h}{2}}-\int_0^{+\frac{T_h}{2}}\frac{2A^2}{T_h\im k \omega}\e^{-\im k \omega t}\diff t\nonumber \\
	&=\frac{A^2}{\im k \omega}-\frac{2A^2}{T_hk^2\omega^2}+\frac{2A^2}{T_hk^2\omega^2}\e^{+\im k \omega \frac{T_h}{2}}+\frac{A^2}{\im k \omega}-\frac{2A^2}{T_h k^2 \omega^2}\e^{-\im k \omega \frac{T_h}{2}}+\frac{2A^2}{T_hk^2\omega^2}\nonumber \\
	&=\frac{2A^2}{\im k \omega}+\frac{2A^2}{T_h k^2\omega^2}\left (\e^{+\im k \frac{2\pi}{T}\frac{T_h}{2}}-\e^{-\im k \frac{2\pi}{T}\frac{T_h}{2}}\right )
	=\frac{A^2T}{\im k \pi}+\im \frac{A^2T^2}{T_hk^2\pi^2}\sin\left(k\pi\frac{T_h}{T}\right)\nonumber\\
	&=\im\frac{A^2T}{k\pi}\left[\sinc\left(k\pi\frac{T_h}{T}\right)-1\right]
\end{align}
%
%
%
Bei einer Multiplikation zweier Funktionen mit gleicher Periode $T$ ergeben sich
der Fourierkoeffizient aus der diskreten Faltung der Koeffizienten beider
Funktionen, skaliert mit dem Faktor $\frac{1}{T}$ (vgl. auch \cite[Kap. 3, S. 208]{Oppenheim1997}).

Die diskrete Faltung ist folgendermaßen definiert:
\begin{align}
	X_k\ast H_k=\sum_{\kappa = -\infty}^{+\infty}X_{\kappa}\cdot H_{k-\kappa}=\sum_{\kappa =-\infty}^{+\infty}X_{k-\kappa}\cdot H_{\kappa }
\end{align}
%
Also merken wir uns folgende Korrespondenz:
%
\begin{align}
	x(t)\cdot h(t)\quad\fourier\quad \frac{1}{T}X_k \ast H_k.
\end{align}
%
Wir prüfen das einmal nach.
%
%
\begin{align}
	y(t)&=x(t)\cdot h(t)
	=\left (\frac{1}{T}\sum_{k=-\infty}^{+\infty}X_k\e^{+\im k \omega t}\right )\cdot\left ( \frac{1}{T}\sum_{\kappa=-\infty}^{+\infty}H_{\kappa}\e^{+\im \kappa \omega t}\right )
	=\frac{1}{T^2}\sum_{k=-\infty}^{+\infty}\sum_{\kappa=-\infty}^{+\infty}X_kH_{\kappa}\e^{\im (k+\kappa)\omega t}\nonumber \\
	&=\cdots +\frac{1}{T^2}\sum_{\kappa=-\infty}^{+\infty}X_{-1}H_{\kappa}\e^{+\im(\kappa-1)\omega t}+\frac{1}{T^2}\sum_{\kappa=-\infty}^{+\infty}X_0H_{\kappa}\e^{+\im \kappa \omega t}+\frac{1}{T^2}\sum_{\kappa=-\infty}^{+\infty}X_1H_{\kappa}\e^{+\im(i+1)\omega t}+\cdots\nonumber \\
	&=\frac{1}{T}\Bigg [\cdots+\frac{1}{T}X_{-1}H_{-1}\e^{-2\im \omega }+\frac{1}{T}X_{-1}H_0\e^{-\im \omega t}+\frac{1}{T}X_{-1}H_1+\cdots+\frac{1}{T}X_0H_{-1}\e^{-\im \omega t}+\frac{1}{T}X_0H_0\nonumber\\
	&+\frac{1}{T}X_0H_1\e^{+\im  \omega t}
	+\cdots+\frac{1}{T}X_1H_{-1}+\frac{1}{T}X_1H_0\e^{+\im \omega t}
	+\frac{1}{T}X_1H_1\e^{+2\im \omega t}+\cdots\Bigg]
\end{align}
%
%
Noch einmal die allgemeine Form der Fourierreihe:
%
\begin{align}
	x(t)=\frac{1}{T}\sum_{n=-\infty}^{+\infty}c_n\e^{+\im n \omega t}.
\end{align}
%
Für $n=0$ beträgt $c_n$
%
\begin{align}
	c_n=\cdots + \frac{1}{T0}X_{-1}H_{1}+\frac{1}{T}X_0H_0+\frac{1}{T}X_1H_{-1}.
\end{align}
%
Dabei durchläuft der Index von $X$ $\mathbb{Z}$ und der Index von $H$ ist eine
Lösung der Gleichung
%
\begin{align}
	\kappa = n-k.
\end{align}
%
Also
%
\begin{align}
	y(t)=x(t)\cdot h(t)=\frac{1}{T}\sum_{n=-\infty}^{+\infty}\sum_{k=-\infty}^{+\infty}\frac{1}{T}X_kH_{n-k}\e^{\im n \omega t}.
\end{align}
%
\begin{figure}
	\centering
	\begin{subfigure}{\textwidth}
		\includegraphics[width=\textwidth,keepaspectratio]{../fs/FB93908026_timedomain_14}
		\caption{Zeitbereich}
	\end{subfigure}
	\begin{subfigure}{\textwidth}
		\includegraphics[width=\textwidth,keepaspectratio]{../fs/FB93908026_frequence_14}
		\caption{Betrag und Phase der Fourierkoeffizienten}
	\end{subfigure}
	\caption{Synthese der Fourierreihe von $x_l(t)$ für $A=1$, $T=4$, $T_h=2$ und
		$-5\leq k\leq5$.}
\end{figure}
%
\newpage
\subsubsection{Aufgabe m) mit komplexer Fourierreihe}
%
Wir setzen die Integration direkt in die Definition der Fourierreihe ein, um zu
schauen, ob es eine Korrespondenz gibt.
%
%
\begin{align}
	x_m(t)&=\int x_b(t)\diff t
	=\int\left ( \frac{1}{T}\sum_{k=-\infty}^{+\infty}X_k\e^{+\im k \omega t}\right )\diff t
	=\frac{1}{T}\sum_{k=-\infty}^{+\infty}\left (\int X_k\e^{+\im k \omega t}\diff t\right )\nonumber \\
	&=\frac{1}{T}\sum_{k=-\infty}^{+\infty}c_k\e^{+\im k \omega t}\quad\text{mit }c_k
	=
	\begin{cases}
		C\in\mathbb{R}&,X_0=0,k=0,\\
		X_0t&,X_0\neq 0,k=0,\\
		\frac{1}{\im k \omega}X_k&,k\neq 0.
	\end{cases}
\end{align}
%
%
$c_{k\neq 0}$ entspricht also $\frac{1}{\im k \omega }X_k$, wobei $X_k$ der
Koeffizient der komplexen Fourierreihe der zu integrierenden Funktion ist.
%
Wir merken uns folgende Korrespondenz:
%
\begin{align}
	\int x(t)\diff t\quad\fourier\quad \frac{1}{\im k \omega }X_k\quad k\neq 0.
\end{align}
%
Für können also für die Stammfunktion durch eine Korrespondenz grundsätzlich nur
die Koeffizienten für $k\neq 0$ berechnen.
%
Für $X_0\neq 0$ muss der Gleichanteil nur mit $t$ multipliziert werden für die
neue Funktion, bei $X_0=0$ muss der neue Mittelwert händisch ausgerechnet
werden.
%
\newpage
\subsubsection{Aufgabe n) mit komplexer Fourierreihe}
%
Als vorletzten Punkt untersuchen wir den Zusammenhang zwischen den Koeffizienten
einer Funktion und ihrer komplex konjugierten.
%
%
\begin{align}
	x^*(t)&=\left [\frac{1}{T}\sum_{k=-\infty}^{+\infty}c_k\e^{+\im k \omega t}\right ]^*=\left [\frac{1}{T}\sum_{k=-\infty}^{+\infty}(\Re\{c_k\}+\im\Im\{c_k\})(\cos(k\omega t)+\im\sin(k\omega t))\right ]^*\nonumber \\
	&=\left [\frac{1}{T}\sum_{k=-\infty}^{+\infty}\Re\{c_k\}\cos(k\omega t)-\Im\{c_k\}\sin(k\omega t)+\im (\Re\{c_k\}\sin(k\omega t)+\Im\{c_k\}\cos(k\omega t)) \right ]^*\nonumber \\
	&=\left [\frac{1}{T}\sum_{k=-\infty}^{+\infty}\left (\Re\{c_k\}\cos(k\omega t)-\Im\{c_k\}\sin(k\omega t)\right )+\im\frac{1}{T}\sum_{k=-\infty}^{+\infty}\left (\Re\{c_k\}\sin(k\omega t)+\Im\{c_k\}\cos(k\omega t)\right )\right ]^*\nonumber \\
	&=\frac{1}{T}\sum_{k=-\infty}^{+\infty}\left (\Re\{c_k\}\cos(k\omega t)-\Im\{c_k\}\sin(k\omega t)\right )-\im\sum_{k=-\infty}^{+\infty}\left (\Re\{c_k\}\sin(k\omega t)+\Im\{c_k\}\cos(k\omega t)\right )\nonumber \\
	&=\frac{1}{T}\sum_{k=-\infty}^{+\infty}\left (\Re\{c_k\}\cos(k\omega t)-\Im\{c_k\}\sin(k\omega t-\im(\Re\{c_k\}\sin(k\omega t)+\Im\{c_k\}\cos(k\omega t))\right )\nonumber \\
	\label{eq:ComplexConjugatedFourierSeries}
	&=\frac{1}{T}\sum_{k=-\infty}^{+\infty}\left(\Re\{c_k\}-\im\Im\{c_k\}\right)\left(\cos(\omega k t)-\im\sin(k\omega t) \right)=\frac{1}{T}\sum_{k=-\infty}^{+\infty}c_k^*\e^{-\im k \omega t}
\end{align}
%
%
%
Wenn wir den Index von $c_k$ zu $-k$ verändern, können wir in der komplexen
Exponentialfunktion wieder ein +-Zeichen schreiben.
%
%
\begin{align}
	x^*(t)=\frac{1}{T}\sum_{k=-\infty}^{+\infty}c_{-k}^*\e^{+\im k \omega t}
\end{align}
%
%
Wir merken uns also als Korrespondenz (auch zu finden in \cite[Kap. 3, S. 208]{Oppenheim1997}):
%
\begin{align}
	x^*(t)\quad\fourier\quad X_{-k}^*
\end{align}
%
\newpage
\subsubsection{Aufgabe o) mit komplexer Fourierreihe}
%
Zum Schluss wollen wir noch ein Integral einer periodischen Funktion berechnen.
%
Dabei lösen wir dieses erst im Zeitbereich und prüfen danach, ob sich das
Integral auch mittels der Fourierreihe lösen lässt.
%
\begin{align}
	\int_{t_0}^{t_0+T}|x_b(t)|^2\diff t
	=\int_{-\frac{T_h}{2}}^{+\frac{T_h}{2}}A^2\diff t
	= A^2t\Bigg |_{t=-\frac{T_h}{2}}^{t=+\frac{T_h}{2}}=A^2T_h
\end{align}
%
Nun setzen wir für $x_b(t)$ die Fourierreihe ein.
%
Dabei nutzen wir folgende Identität:
%
\begin{align}
	z\cdot z^*=(a+\im b)\cdot (a-\im b)=a^2+b^2=|z|^2.
\end{align}
%
%
\begin{align}
	\int_{t_0}^{t_0+T}|x_b(t)|^2\diff t
	&=\int_{t_0}^{t_0+T}x_b(t)\cdot x_b^*(t)\diff t
	=\int_{t_0}^{t_0+T}\left (\frac{1}{T}\sum_{k=-\infty}^{+\infty}c_k\e^{+\im k \omega t}\right )\cdot \left (\frac{1}{T}\sum_{i=-\infty}^{+\infty}c_i^*\e^{-\im i \omega t}\right )\diff t\nonumber \\
	&=\frac{1}{T^2}\int_{t_0}^{t_0+T}\sum_{k=-\infty}^{+\infty}\sum_{i=-\infty}^{+\infty}c_kc_i^*\e^{\im (k-i)\omega t}\diff t =\frac{1}{T^2}\sum_{k=-\infty}^{+\infty}\sum_{i=-\infty}^{+\infty}\int_{t_0}^{t_0+T}c_kc_i^*\e^{\im (k-i)\omega t}\diff t
\end{align}
%
Es müssen 2 Fälle unterschieden werden: $k=i$ und $k\neq i$.
%
\begin{align}
	\int_{t_0}^{t_0+T}c_kc_i^*\e^{+\im (k-i)\omega t}\diff t \underset{k=i}{=}c_kc_k^*t\Bigg |_{t=t_0}^{t=t_0+T}=Tc_kc_k^*=T|c_k|^2
\end{align}
%
\begin{align}
	\int_{t_0}^{t_0+T}c_kc_i^*\e^{+\im (k-i)\omega t}\diff t
	&\underset{k\neq i}{=}\frac{c_kc_i^*}{\im k \omega }\e^{+\im (k-i)\omega t}\Bigg |_{t=t_0}^{t=t_0+T}
	=\frac{c_kc_i^*}{\im k \omega }\left (\e^{+\im (k-i)\omega (t_0+T)}-\e^{+\im (k-i)\omega t}\right )\nonumber \\
	&=\frac{c_kc_i^*}{\im k \omega }\left (\e^{+\im ((k-i)\omega t_0+(k-i)2\pi)}-\e^{+\im (k-i)\omega t}\right )=0
\end{align}
Es werden also nur die Terme addiert, bei denen $k=i$ gilt.
%

Wir können die Doppelsumme vereinfachen und folgende Korrespondenz aufstellen:
%
\begin{align}
	\int_{t_0}^{t_0+T}|x(t)|^2\diff t
	\quad\fourier\quad\frac{1}{T}\sum_{k=-\infty}^{+\infty}|c_k|^2.
\end{align}
Diese Korrespondenz ist auch als Parselvalsches Theorem bekannt.




%------------------------------------------------------------------------------
\clearpage
\bibliography{../tutorial_latex_deu/literatur}
\end{document}

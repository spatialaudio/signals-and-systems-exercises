\documentclass[11pt,a4paper,DIV=12]{scrartcl}
\usepackage{scrlayer-scrpage}
\usepackage[utf8]{inputenc}
\usepackage{fouriernc}
\usepackage[T1]{fontenc}
\usepackage[german]{babel}
\usepackage[hidelinks]{hyperref}
\usepackage{natbib}
\usepackage{url}
\usepackage{amsmath}
\usepackage{amsfonts}
\usepackage{amssymb}
\usepackage{trfsigns}
\usepackage{marvosym}
\usepackage{nicefrac}
\usepackage{graphicx}
\usepackage{subcaption}
\usepackage{xcolor}
\usepackage{comment}
\usepackage{mdframed}
\usepackage{tikz}
\usepackage{circuitikz}
\usepackage{pgfplots}
\usepackage{bm}
\usepackage{cancel}
\bibliographystyle{dinat}

\usepackage{../sig_sys_macros}

%------------------------------------------------------------------------------
\ohead{Signal- und Systemtheorie Übung}
\cfoot{\pagemark}
\ofoot{\tiny\url{https://github.com/spatialaudio/signals-and-systems-exercises}}

\begin{document}
%
\noindent Signal- und Systemtheorie Übung\footnote{This tutorial is provided as
Open Educational Resource (OER), to be found at
\url{https://github.com/spatialaudio/signals-and-systems-exercises}
accompanying the OER lecture
\url{https://github.com/spatialaudio/signals-and-systems-lecture}.
%
Both are licensed under a) the Creative Commons Attribution 4.0 International
License for text and graphics and b) the MIT License for source code.
%
Please attribute material from the tutorial as \textit{Frank Schultz,
Continuous- and Discrete-Time Signals and Systems - A Tutorial Featuring
Computational Examples, University of Rostock} with
\texttt{github URL, commit number and/or version tag, year, (file name and/or
content)}.}---Frank Schultz, Sascha Spors,
Institut für Nachrichtentechnik (INT),
Fakultät für Informatik und Elektrotechnik (IEF),
Universität Rostock \&
Robert Hauser, Universität Rostock---Sommersemester 2022, Version: \today

\noindent Main author: https://github.com/robhau, check: https://github.com/fs446

\tableofcontents

\section{Übung: Inverse z-Transformation}

Tasks \ref{sec:taske1}, \ref{sec:taskf1} and \ref{sec:taskg1} were originally created by Till Rettberg
%\subsection{Task}
\subsection{Aufgabe}
%Transform
Transformieren Sie die folgenden z-Transformierten
\begin{itemize}
	\item[a )] $X(z)=2\cdot\frac{z\cdot(z-\frac{1}{\sqrt{2}})}{(z-\e^{+\im\frac{\pi}{4}})\cdot(z-\e^{-\im\frac{\pi}{4}})}$\quad \text{KB: }$|z|>1$
	\item[b )] $X(z)=\frac{z}{z-\frac{1}{2}}$\quad \text{KB: }$|z|>\frac{1}{2}$
	\item[c )] $X(z)=\frac{z}{z-\frac{1}{2}}\cdot\frac{z}{z-1}$\quad \text{KB: }$|z|>1$
	\item[d )] $X(z)=\frac{z^2-z+2}{z^2-\frac{1}{2}z+\frac{1}{4}}$\quad \text{KB: }$|z|>\frac{1}{2}$
	\item[e )] $X(z)=\frac{z^2}{z^2+1}$\quad \text{KB: }$|z|>1$
	\item[f )] $X(z)=\frac{z^4+z^3-6z^2+6z-1}{z^2-2z+1}$\quad \text{KB: }$|z|>\frac{1}{2}$
	\item[g )] $X(z)=\frac{z\cdot(z-1)}{z^2-\sqrt{2}z+1}$\quad \text{KB: }$|z|>1$
\end{itemize}
%into time-discret signals.
%The task shall be performed with help of the
in zeit-diskrete Signale $x[k]$. %Der Konvergenzbereich sei bei allen Aufgaben bis auf c) $|z|>1$ und bei Aufgabe c)?! Typo ?! $|z|>\frac{1}{2}$. KB besser bei den indiviudellen Tasks
Die Aufgabe soll gelöst werden mit Hilfe
\begin{itemize}
	%\item \textbf{correspondence table},
	%\item \textbf{Residue theorem}.
	\item[i )] der \textbf{Korrespondenztabelle} in der Formelsammlung
	\item[ii )] des \textbf{Residuensatz}es.
\end{itemize}
Berechen Sie außerdem die inverse z-Transformation von
\begin{itemize}
	\item[h )] $X(z)=\e^{1/z}$
\end{itemize}
mit Hilfe des \textbf{Residuensatz}es.
%\begin{itemize}
%	\item[ii )] \textbf{Residuensatz}.
%\end{itemize}
%For a quick view there is a solution table at the end of the document.

Am Ende des Dokumentes in Anhang \ref{sec:AnhangB} findet sich eine Übersicht mit Lösungen.
%\subsection{Solution Using Correspondence Table}
\subsection{Lösung mit Hilfe der Korrespondenztabelle}
Es wird die Korrespondenztabelle der Vorlesung/Übung, die im StudIp zur Verfügung steht, genutzt, sie sind in Anhang \ref{sec:AnhangC} nochmal abgedruckt. Diese deckt zwar nur allgemeine, einfache Fälle ab, ist damit jedoch übersichtlicher. In anderen Büchern (z.B. \cite[Kap. 21, S. 1128-1130]{Bronstein2015} oder \cite[S. 237-238]{UlrichWeber2017} ) gibt es weit umfangreichere Korrespondenztabellen, diese können jedoch nicht in der Klausur verwendet werden.
%\subsubsection{Task a)}
\subsubsection{Aufgabe a)}
%We shall find the time-discret signal for
Wir wollen das zeit-diskrete Signal für
\begin{align}
	X(z)&=2\cdot\frac{z\cdot(z-\frac{1}{\sqrt{2}})}{(z-\e^{+\im\frac{\pi}{4}})\cdot(z-\e^{-\im\frac{\pi}{4}})}=2\cdot\frac{z\cdot(z-\frac{1}{\sqrt{2}})}{z^2-z(\e^{+\im\frac{\pi}{4}}+\e^{-\im\frac{\pi}{4}})+1}=2\cdot\frac{z\cdot(z-\frac{1}{\sqrt{2}})}{z^2-z\bigg[\cos(\frac{\pi}{4})+\im\sin(\frac{\pi}{4})+\cos(\frac{\pi}{4})-\im\sin(\frac{\pi}{4})\bigg ]+1}\nonumber\\
	&=2\cdot\frac{z\cdot(z-\frac{1}{\sqrt{2}})}{z^2-2z\cos(\frac{\pi}{4})+1}=2\cdot\frac{z^2-z\cos(\frac{\pi}{4})}{z-2z\cos(\frac{\pi}{4})+1}
\end{align}
berechnen.
%If we take a look at the correspondence table, we find following correspondence:
Durch die Umformungen gelangen wir zu einer Korrespondenz, die sich direkt in der Tabelle findet:
\begin{align}
	\cos[\Omega_0k]\epsilon[k]\quad\ztransf\quad\frac{z^2-z\cos(\Omega_0)}{z^2-2z\cos(\Omega_0)+1}.
\end{align}
%So our solution is
Also ist unsere Lösung
\begin{align}
	X(z)=2\cdot\frac{z^2-z\cos(\frac{\pi}{4})}{z-2z\cos(\frac{\pi}{4})+1}\quad\Ztransf\quad x[k]=2\cdot\cos[\frac{\pi}{4}k]\epsilon[k].
\end{align}
%\subsubsection{Task b)}
\subsubsection{Aufgabe b)}
%We shall find the time-discret signal for
Wir wollen das zeit-diskrete Signal für
\begin{align}
	X(z)=\frac{z}{z-\frac{1}{2}}
\end{align}
berechnen.
%If we take a look at the correspondence table, we find following correspondence:
Wenn wir einen Blick auf die Korrespondenztabelle werfen, finden wir direkt folgende Korrespondenz:
\begin{align}
	\label{eq:GeometricProgression}
	a^k\epsilon[k]\quad\ztransf\quad\frac{z}{z-a}.
\end{align}
%So our solution is
Also ist unsere Lösung
\begin{align}
	X(z)=\frac{z}{z-\frac{1}{2}}\quad\Ztransf\quad x[k]=\bigg (\frac{1}{2}\bigg)^k\epsilon[k].
\end{align}
%\subsubsection{Task c)}
\subsubsection{Aufgabe c)}
%We shall find the time-discret signal for
Wir wollen das zeit-diskrete Signal für
\begin{align}
	X(z)=\frac{z}{z-\frac{1}{2}}\cdot\frac{z}{z-1}
\end{align}
berechnen.
%Now we can not find a correspondence in our table. But it seems that a partial fraction decomposition could be helpful.
Zunächst können wir keine passende Korrespondenz in unserer Tabelle finden. Aber eine Partialbruchzerlegung könnte hilfreich sein.
\begin{align}
	\frac{X(z)}{z}=\frac{1}{z-\frac{1}{2}}\cdot\frac{z}{z-1}=\frac{A}{z-\frac{1}{2}}+\frac{B}{z-1} \quad \Bigg | \quad\cdot (z-\frac{1}{2})\cdot(z-1)\nonumber\\
	z=A\cdot(z-1)+B\cdot(z-\frac{1}{2})=z(A+B)-A-\frac{B}{2}
\end{align}
%We get a system of linear equations. In matrix notation:
Wir erhalten ein lineares Gleichungssystem, hier dargestellt in der Matrixnotation:
\begin{align}
	\begin{pmatrix}
		1 & 1\\
		-1 & -\frac{1}{2}
	\end{pmatrix}
	\begin{pmatrix}
		A \\
		B
	\end{pmatrix}
	=
	\begin{pmatrix}
		1 \\
		0
	\end{pmatrix}.
\end{align}
%With techniques like Gaussian elimination or Cramer's rule we can solute our system of linear equations and achieve
Das lineare Gleichungssystem lässt sich mit Techniken wie dem Gaußschem Eliminationsverfahren oder der Cramerschen Regel lösen und wir erhalten
\begin{align}
	A = -1 \nonumber \\
	B = 2.
\end{align}
\begin{align}
	\frac{X(z)}{z}=-1\cdot\frac{1}{z-\frac{1}{2}}+2\cdot\frac{1}{z-1} \quad\Bigg | \quad \cdot z
\end{align}

\begin{align}
	X(z)=-1\cdot\frac{z}{z-\frac{1}{2}}+2\cdot\frac{z}{z-1}
\end{align}
%Using the correspondence from \eqref{eq:GeometricProgression} and
Wir nutzen die Korrespondenz aus Glg. \eqref{eq:GeometricProgression}, aus der sich zudem für $a=1$
\begin{align}
	\epsilon[k]\quad\ztransf\quad\frac{z}{z-1}.
\end{align}
ableiten lässt.
%We can find the time-discret signal:
Damit können wir das zeit-diskrete Signal angeben:
\begin{align}
	X(z)=-1\cdot\frac{1}{z-\frac{1}{2}}+2\cdot\frac{1}{z-1}\quad\Ztransf\quad x[k]=-\bigg(\frac{1}{2}\bigg)^k\epsilon[k]+2\epsilon[k].
\end{align}
%\subsubsection{Task d)}
\subsubsection{Aufgabe d)}
%We shall find the time-discret signal for
Wir wollen das zeit-diskrete Signal für
\begin{align}
	X(z)=\frac{z^2-z+2}{z^2-\frac{1}{2}z+\frac{1}{4}}
\end{align}
berechnen.
%We use polynomial division and partial fraction decomposition to achieve a term like in our correspondence table.
Wir nutzen Polynomdivision und Partialbruchzerlegung, um Terme aus unserer Korrespondenztabelle zu erhalten.
\begin{align}
	X(z)=\frac{z^2-z+2}{z^2-\frac{1}{2}z+\frac{1}{4}}=1+\frac{-\frac{1}{2}z+\frac{7}{4}}{z^2-\frac{1}{2}z+\frac{1}{4}}
\end{align}
\begin{align}
	\label{eq:TaskDPartialFractionDecomposition}
	\frac{-\frac{1}{2}z+\frac{7}{4}}{(z-(\frac{1}{4}+\im\frac{\sqrt{3}}{4}))\cdot(z-(\frac{1}{4}-\im\frac{\sqrt{3}}{4}))}=\frac{A}{z-(\frac{1}{4}+\im\frac{\sqrt{3}}{4})}+\frac{B}{z-(\frac{1}{4}-\im\frac{\sqrt{3}}{4})}\quad\Bigg | \quad \cdot (z-(\frac{1}{4}+\im\frac{\sqrt{3}}{4}))\cdot(z-(\frac{1}{4}-\im\frac{\sqrt{3}}{4}))\nonumber\\
	-\frac{1}{2}z+\frac{7}{4}=A(z-(\frac{1}{4}-\im\frac{\sqrt{3}}{4}))+B(z-(\frac{1}{4}+\im\frac{\sqrt{3}}{4}))=z(A+B)+A(-\frac{1}{4}+\im\frac{\sqrt{3}}{4})+B(-\frac{1}{4}-\im\frac{\sqrt{3}}{4})
\end{align}
%We get a system of linear equations. In matrix notation:
Wir erhalten ein lineares Gleichungssystem, hier dargestellt in der Matrixnotation:
\begin{align}
	\begin{pmatrix}
		1 & 1 \\
		-\frac{1}{4}+\im\frac{\sqrt{3}}{4} & -\frac{1}{4}-\im\frac{\sqrt{3}}{4}
	\end{pmatrix}
	\begin{pmatrix}
		A \\
		B
	\end{pmatrix}
	=
	\begin{pmatrix}
		-\frac{1}{2} \\
		\frac{7}{4}
	\end{pmatrix}
\end{align}
%With techniquest like Gaussian elimination or Cramer's rule, we can solute our system of linear equations and achieve
Das lineare Gleichungssystem lässt sich mit Techniken wie dem Gaußschem Eliminationsverfahren oder der Cramerschen Regel lösen und wir erhalten
\begin{align}
	A=-\frac{1}{4}-\im\frac{13}{4\cdot\sqrt{3}}, \nonumber \\
	B=-\frac{1}{4}+\im\frac{13}{4\cdot\sqrt{3}}.
\end{align}
\begin{align}
	X(z)=1+\frac{1}{z}\cdot\frac{(-\frac{1}{4}-\im\frac{13}{4\cdot\sqrt{3}})\cdot z}{z-(\frac{1}{4}+\im\frac{\sqrt{3}}{4})}+\frac{1}{z}\cdot\frac{(-\frac{1}{4}+\im\frac{13}{4\cdot\sqrt{3}})\cdot z}{z-(\frac{1}{4}-\im\frac{\sqrt{3}}{4})}=1+\frac{1}{4z}\cdot\frac{(-1-\im\frac{13}{\sqrt{3}})\cdot z}{z-\frac{1}{2}\e^{+\im\frac{\pi}{3}}}+\frac{1}{4z}\cdot\frac{(-1+\im\frac{13}{\sqrt{3}})\cdot z}{z-\frac{1}{2}\e^{-\im\frac{\pi}{3}}}
\end{align}
%We use the correspondences
Wir nutzen die Korrespondenzen
\begin{align}
	\delta[k]\quad\ztransf\quad1\quad\mathrm{ROC}=\mathbb{C},
\end{align}
\begin{align}
	a^k\epsilon[k]\quad\ztransf\quad\frac{z}{z-a}\quad|z|>1
\end{align}
%and
und
\begin{align}
	x[k-\kappa]\quad\ztransf\quad z^{-\kappa}.
\end{align}
%This leads to
Dies führt zu
\begin{align}
	x[k]&=\delta[k]+\frac{-1-\im\frac{13}{\sqrt{3}}}{2\cdot2}\bigg(\frac{1}{2}\bigg)^{k-1}\e^{+\im\frac{\pi}{3}(k-1)}\epsilon[k-1]+\frac{-1+\im\frac{13}{\sqrt{3}}}{2\cdot2}\bigg(\frac{1}{2}\bigg)^{k-1}\e^{-\im\frac{\pi}{3}(k-1)}\epsilon[k-1] \nonumber\\
	&=\delta[k]+\bigg(\frac{1}{2}\bigg)^{k}\Bigg[-\frac{1}{2}\bigg(\e^{+\im\frac{\pi}{3}(k-1)}+\e^{-\im\frac{\pi}{3}(k-1)}\bigg)+\frac{13}{\sqrt{3}\cdot2\im}\bigg(\e^{+\im\frac{\pi}{3}(k-1)}-\e^{-\im\frac{\pi}{3}(k-1)}\bigg)\Bigg ]\epsilon[k-1]\nonumber\\
	&=\delta[k]+\bigg(\frac{1}{2}\bigg)^k\Bigg[\frac{13}{\sqrt{3}}\sin(\frac{\pi}{3}(k-1))-\cos(\frac{\pi}{3}(k-1))\Bigg]\epsilon[k-1]
\end{align}

%\subsubsection{Task e)}
\subsubsection{Aufgabe e)}
\label{sec:taske1}
%We shall find the time-discret signal for
Wir wollen das zeit-diskrete Signal für
\begin{align}
	X(z)=\frac{z^2}{z^2+1}
\end{align}
berechnen.
%If we look at the correspondence
Wir werfen einen Blick auf die Korrespondenz
\begin{align}
	\cos [\Omega_0k]\epsilon[k]\quad\ztransf\quad \frac{z^2-z\cos(\Omega_0))}{z^2-2z\cos(\Omega_0)+1}\quad\quad |z| > 1
\end{align}
%we have the left site if we set
und erkennen, dass wir diese erreichen, indem wir
\begin{align}
	\cos(\Omega_0)=0
\end{align}
setzen.
%We get
Wir erhalten
\begin{align}
	\Omega_0 = \frac{\pi}{2} \pm \pi \cdot m \quad m \in \mathbb{Z}
\end{align}
%At the end, our time-discret signal is
und somit unser zeit-diskretes Signal
\begin{align}
	&x[k]=\cos[(\frac{\pi}{2}\pm \pi\cdot m)k]\epsilon[k]\quad m \in \mathbb{Z}\\
	&x[k] = 1\cdot\delta[k] + 0\cdot \delta[k-1] -1\cdot \delta[k-2] + 0\cdot \delta[k-3] + 1\cdot \delta[k-4] + 0\cdot \delta[k-5] -1\cdot \delta[k-6] + 0\cdot \delta[k-7] ...
\end{align}
%\subsubsection{Task f)}
\subsubsection{Aufgabe f)}
\label{sec:taskf1}
%We shall find the time-discret signal for
Wir wollen das zeit-diskrete Signal für
\begin{align}
	X(z)=\frac{z^4+z^3-6z^2+6z-1}{z^2-2z+1}
\end{align}
berechnen.
%We can simplify the denominator to
Wir können den Nenner vereinfachen:
\begin{align}
	(z-1)^2.
\end{align}
%If we take a look at the correspondences
Mit den Korrespondenzen
\begin{align}
	k\epsilon[k]\quad\ztransf\quad\frac{z}{(z-1)^2}\quad |z|>1
\end{align}
%and
und
\begin{align}
	x[k-\kappa]\quad\ztransf\quad z^{-\kappa}X(z)
\end{align}
%we rewrite our function to use these two:
können wir unsere Funktion umformen:
\begin{align}
	X(z)=z^3\cdot\frac{z}{(z-1)^2}+z^2\cdot\frac{z}{(z-1)^2}-6z\cdot\frac{z}{(z-1)^2}+6\frac{z}{(z-1)^2}-z^{-1}\cdot\frac{z}{(z-1)^2}.
\end{align}
%So our time-discret signal is
Somit können wir die Funktion in ein zeit-diskretes Signal transformieren:
\begin{align}
	&X(z)=z^3\cdot\frac{z}{(z-1)^2}+z^2\cdot\frac{z}{(z-1)^2}-6z\cdot\frac{z}{(z-1)^2}+6\frac{z}{(z-1)^2}-z^{-1}\cdot\frac{z}{(z-1)^2} \nonumber\\
%\end{align}
%\begin{align}
	&\Ztransf \nonumber\\
%\end{align}
%\begin{align}
	&x[k]=(k+3)\epsilon[k+3]+(k+2)\epsilon[k+2]-6(k+1)\epsilon[k+1]+6k\epsilon[k]-(k-1)\epsilon[k-1].
\end{align}
%
%
Alternativer Weg wegen Zählergrad $>$ Nennergrad \quad $\Rightarrow$\quad zunächst
Durchdividieren
\begin{align}
X(z) = \frac{z^4 + z^3 -6z^2 + 6z -1}{z^2-2z+1} = z^2 + 3 z - 1 + \frac{z}{(z-1)^2}.
\end{align}%
und mit Korrespondenzen zu
\begin{align}
x[k] = \delta[k+2] +3 \delta[k+1] -\delta[k] +k\epsilon[k]
\end{align}
%
Die beiden Lösungen sollten wir uns mal grafisch veranschaulichen bzw. ineinander
überführen, damit klar wird, dass das identische Lösungen sind. Siehe dazu auch
der Anhang A in Kap. \ref{sec:AnhangA}.

%\subsubsection{Task g)}
\subsubsection{Aufgabe g)}
\label{sec:taskg1}
%We shall find the time-discret signal for
Wir wollen das zeit-diskrete Signal für
\begin{align}
	X(z)=\frac{z\cdot(z-1)}{z^2-\sqrt{2}z+1}
\end{align}
berechnen.
%We can reshape the function:
Wir können die Funktion umformen:
\begin{align}
	X(z)=\frac{z^2-z}{z^2-2z\cos(\frac{\pi}{4})+1}=\frac{z^2-z\cos(\frac{\pi}{4})}{z^2-2z\cos(\frac{\pi}{4})+1}-(1-\frac{\sqrt{2}}{2})\frac{z}{z^2-2z\cos(\frac{\pi}{4})+1}\cdot\frac{\sin(\frac{\pi}{4})}{\sin(\frac{\pi}{4})}.
\end{align}
%Using the correspondeces
Wir nutzen die Korrespondenzen
\begin{align}
	\sin[\Omega_0k]\epsilon[k]\quad\ztransf\quad\frac{z\sin(\Omega_0)}{z^2-2z\cos(\Omega_0)+1}\quad |z| > 1
\end{align}
%and
und
\begin{align}
	\cos[\Omega_0k]\epsilon[k]\quad\ztransf\quad\frac{z^2-z\cos(\Omega_0)}{z^2-2z\cos(\Omega_0)+1}\quad |z| > 1
\end{align}
%leads to
und erhalten
\begin{align}
	X(z)\quad\Ztransf\quad x[k]=\cos[\frac{\pi}{4}k]\epsilon[k]+\sin[\frac{\pi}{4}k]\epsilon[k]-\sqrt{2}\sin[\frac{\pi}{4}k]\epsilon[k].
\end{align}
\newpage
%\subsection{Solution Using Residue Theorem}
\subsection{Lösung mit Hilfe des Residuensatzes}
%If all isolated singularities are surrounded by the ROC, then we can calculate the inverse z-Transformation using the Residue theorem.
Allgemein berechnet sich die inverse z-Transformation durch ein komplexes Kurvenintegral entlang einer geschlossenen Kurve $C$ im Konvergenzbereich (vgl. \cite[S. 192]{UlrichWeber2017} ).
\begin{align}
	x[k]=\mathcal{Z}^{-1}\{X(z)\}=\frac{1}{2\pi\im}\oint_C X(z)z^{k-1}\mathrm{d}z
\end{align}
Das komplexe Kurvenintegral kann mit Hilfe des Residuensatz berechnet werden.\\
Für die Berechnung der Residuen finden sich z.B. in \cite[K. 14, S. 753-754]{Bronstein2015}, \cite[S. 37-38]{UlrichWeber2017} oder \cite[S. 137-138]{Fritzsche2019}.
%\subsubsection{Task a)}
\subsubsection{Aufgabe a)}
%We shall find the time-discret signal for
Wir wollen das zeit-diskrete Signal für
\begin{align}
	X(z)=2\cdot\frac{z\cdot(z-\frac{1}{\sqrt{2}})}{(z-\e^{+\im\frac{\pi}{4}})\cdot(z-\e^{-\im\frac{\pi}{4}})}
\end{align}
berechnen.
%$\e^{+\im\frac{\pi}{4}}$ and $\e^{-\im\frac{\pi}{4}}$ are poles of order 1.\\
$\e^{+\im\frac{\pi}{4}}$ und $\e^{-\im\frac{\pi}{4}}$ sind Polstellen erster Ordnung.\\
%\begin{align}
%	0\text{ is }
%	\begin{cases}
%	\text{no singularity}, &k > 0, \\
%	\text{removable singularity}, &k = 0, \\
%	\text{pole of order }|k|, &k <0.
%\end{cases}
%\end{align}
\begin{align}
	0\text{ ist }
	\begin{cases}
		\text{keine isolierte Singularität}, &k > 0, \\
		\text{eine hebbare Singularität}, &k = 0, \\
		\text{eine Polstelle }|k|\text{-ter Ordnung}, &k <0.
	\end{cases}
\end{align}
\begin{align}
	\mathrm{Res}(X(z)z^{k-1},\e^{+\im\frac{\pi}{4}})&=\lim\limits_{z\rightarrow\e^{+\im\frac{\pi}{4}}}2\cdot\frac{z\cdot(z-\frac{1}{\sqrt{2}})}{(z-\e^{-\im\frac{\pi}{4}})}z^{k-1}=2\cdot\frac{\e^{+\im\frac{\pi}{4}}\cdot(\e^{+\im\frac{\pi}{4}}-\frac{1}{\sqrt{2}})}{\e^{+\im\frac{\pi}{4}}-\e^{-\im\frac{\pi}{4}}}\e^{+\im\frac{\pi}{4}(k-1)} \nonumber \\
	&=2\cdot\frac{\e^{+\im\frac{\pi}{2}}-\frac{1}{\sqrt{2}}\e^{+\im\frac{\pi}{4}}}{\cos(\frac{\pi}{4})+\im\sin(\frac{\pi}{4})-(\cos(\frac{\pi}{4})-\im\sin(\frac{\pi}{4}))}\e^{+\im\frac{\pi}{4}(k-1)}\nonumber\\
	&=2\cdot\frac{\im-\frac{1}{2}-\frac{1}{2}\im}{2\im\sin(\frac{\pi}{4})}\e^{+\im\frac{\pi}{4}(k-1)}=\frac{-1+\im}{\im\sqrt{2}}\e^{+\im\frac{\pi}{4}(k-1)}
\end{align}
\begin{align}
	\mathrm{Res}(X(z)z^{k-1},\e^{-\im\frac{\pi}{4}})&=\lim\limits_{z\rightarrow\e^{-\im\frac{\pi}{4}}}2\cdot\frac{z\cdot(z-\frac{1}{\sqrt{2}})}{(z-\e^{+\im\frac{\pi}{4}})}z^{k-1}=2\cdot\frac{\e^{-\im\frac{\pi}{4}}\cdot(\e^{-\im\frac{\pi}{4}}-\frac{1}{\sqrt{2}})}{\e^{-\im\frac{\pi}{4}}-\e^{+\im\frac{\pi}{4}}}\e^{-\im\frac{\pi}{4}(k-1)} \nonumber \\
	&=2\cdot\frac{\e^{-\im\frac{\pi}{2}}-\frac{1}{\sqrt{2}}\e^{-\im\frac{\pi}{4}}}{\cos(\frac{\pi}{4})-\im\sin(\frac{\pi}{4})-(\cos(\frac{\pi}{4})+\im\sin(\frac{\pi}{4}))}\e^{-\im\frac{\pi}{4}(k-1)}\nonumber\\
	&=2\cdot\frac{-\im-\frac{1}{2}+\frac{1}{2}\im}{-2\im\sin(\frac{\pi}{4})}\e^{-\im\frac{\pi}{4}(k-1)}=\frac{1+\im}{\im\sqrt{2}}\e^{-\im\frac{\pi}{4}(k-1)}
\end{align}
\begin{align}
	\underset{k=0}{\mathrm{Res}(X(z)z^{k-1},0)}=0
\end{align}
\begin{align}
	\underset{k<0}{\mathrm{Res}(X(z)z^{k-1}),0}&=\lim\limits_{z\rightarrow0}\frac{1}{(|k|-1)!}\frac{\mathrm{d}^{|k|-1}}{\mathrm{d}z^{|k|-1}}\Bigg [2\cdot\frac{z\cdot(z-\frac{1}{\sqrt{2}})}{z^2-\sqrt{2}z+1}\cdot\frac{z^{|k|}}{z^{|k|+1}}\Bigg]\nonumber\\
	&=\lim\limits_{z\rightarrow0}\frac{2}{(|k|-1)!}\frac{\mathrm{d}^{|k|-1}}{\mathrm{d}^{|k|-1}}\Bigg [\frac{z-\frac{1}{\sqrt{2}}}{(z-\e^{+\im\frac{\pi}{4}})\cdot(z-\e^{-\im\frac{\pi}{4}})}\Bigg]
\end{align}
%We use partial fraction decomposition to make the differentiation easier.
Wir nutzen Partialbruchzerlegung, um uns das Ableiten zu vereinfachen.
\begin{align}
	\frac{z-\frac{1}{\sqrt{2}}}{(z-\e^{+\im\frac{\pi}{4}})\cdot(z-\e^{-\im\frac{\pi}{4}})}=\frac{A}{z-\e^{+\im\frac{\pi}{4}}}+\frac{B}{z-\e^{-\im\frac{\pi}{4}}}\quad\Bigg | \quad \cdot(z-\e^{+\im\frac{\pi}{4}})\cdot(z-\e^{-\im\frac{\pi}{4}}) \nonumber\\
	z-\frac{}{\sqrt{2}}=A(z-\e^{-\im\frac{\pi}{4}})+B(z-\e^{-\im\frac{\pi}{4}})=z(A+B)+(-A\e^{-\im\frac{\pi}{4}}-B\e^{+\im\frac{\pi}{4}})
\end{align}
%We get a system of linear equations. In matrix notation:
Wir erhalten ein lineares Gleichungssystem, hier dargestellt in der Matrixnotation:
\begin{align}
	\begin{pmatrix}
		1 & 1 \\
		-\e^{-\im\frac{\pi}{4}} & -\e^{+\im\frac{\pi}{4}}
	\end{pmatrix}
	\begin{pmatrix}
		A \\
		B
	\end{pmatrix}
	=
	\begin{pmatrix}
		1 \\
		-\frac{1}{\sqrt{2}}
	\end{pmatrix}
\end{align}
%With techniques like Gaussian elimination or Cramer's rule we can solute our system of linear equations and achieve
Das lineare Gleichungssystem lässt sich mit Techniken wie dem Gaußschem Eliminationsverfahren oder der Cramerschen Regel lösen und wir erhalten
\begin{align}
	A=\frac{1}{2} \nonumber \\
	B=\frac{1}{2}
\end{align}
\begin{align}
	\underset{k<0}{\mathrm{Res}(X(z)z^{k-1},0)}&=\lim\limits_{z\rightarrow0}\frac{2}{(|k|-1)!}\frac{\mathrm{d}^{|k|-1}}{\mathrm{d}z^{|k|-1}}\Bigg[\frac{1}{2}\frac{1}{z-\e^{+\im\frac{\pi}{4}}}+\frac{1}{2}\frac{1}{z-\e^{-\im\frac{\pi}{4}}}\Bigg]\nonumber\\
	&=\lim\limits_{z\rightarrow0}\frac{2}{(|k|-1)!}\Bigg[\frac{1}{2}\frac{(-1)^{|k|-1}\cdot(|k|-1)!}{(z-\e^{+\im\frac{\pi}{4}})^{|k|}}+\frac{1}{2}\frac{(-1)^{|k|-1}\cdot(|k|-1)!}{(z-\e^{-\im\frac{\pi}{4}})^{|k|}}\Bigg]\nonumber\\
	&=\frac{(-1)^{|k|-1}}{(-1)^{|k|}\e^{+\im\frac{\pi}{4}|k|}}\frac{(-1)^{|k|-1}}{(-1)^{|k|}\e^{-\im\frac{\pi}{4}}|k|}\nonumber\\
	&=-\e^{-\im\frac{\pi}{4}|k|}-\e^{+\im\frac{\pi}{4}|k|}\nonumber\\
	&=-\frac{2}{2}\bigg(\e^{+\im\frac{\pi}{4}|k|}+\e^{-\im\frac{\pi}{4}|k|}\bigg)\nonumber\\
	&=-2\cos(\frac{\pi}{4}|k|)
\end{align}
\begin{align}
	x[k]=\begin{cases}
		x_1[k] &k\geq0 \\
		x_2[k] &k<0
	\end{cases}
\end{align}
\begin{align}
	x_1[k]&=\mathrm{Res}(X(z)z^{k-1},\e^{+\im\frac{\pi}{4}})+\mathrm{Res}(X(z)z^{k-1},\e^{-\im\frac{\pi}{4}})=\frac{-1+\im}{\im\sqrt{2}}\e^{+\im\frac{\pi}{4}(k-1)}+\frac{1+\im}{\im\sqrt{2}}\e^{-\im\frac{\pi}{4}(k-1)}\nonumber\\
	&=\frac{\sqrt{2}}{2}\bigg (\e^{+\im\frac{\pi}{4}(k-1)}+\e^{+\im\frac{\pi}{4}(k-1)}\bigg )-\frac{\sqrt{2}}{2\im} \bigg (\e^{+\im\frac{\pi}{4}(k-1)}+\e^{+\im\frac{\pi}{4}(k-1)}\bigg)\nonumber\\
	&=\sqrt{2}\bigg(\cos(\frac{\pi}{4}(k-1))-\sin(\frac{\pi}{4}(k-1))\bigg)
\end{align}
\begin{align}
	\underset{k<0}{x_2[k]}&=\mathrm{Res}(X(z)z^{k-1},\e^{+\im\frac{\pi}{4}})+\mathrm{Res}(X(z)z^{k-1},\e^{-\im\frac{\pi}{4}})+\mathrm{Res}(X(z)z^{k-1},0)\nonumber\\
	&=x_1[k]+-2\cos(\frac{\pi}{4}|k|) \nonumber\\
	&=\sqrt{2}\bigg(\cos(\frac{\pi}{4}(k-1))-\sin(\frac{\pi}{4}(k-1))\bigg)-2\cos(\frac{\pi}{4}|k|)
\end{align}
%It is (cf.~\cite[Ch. 2, p. 81]{Bronstein2015}):
Es gelten (vgl. \cite[Kap.2, S. 81]{Bronstein2015}):
\begin{align}
	\label{eq:SinAlphaPlusMinusBeta}
	\sin(\alpha\pm\beta)=\sin\alpha\cos\beta\pm\cos\alpha\sin\beta
\end{align}
%and
und
\begin{align}
	\label{eq:CosAlphaPlusMinusBeta}
	\cos(\alpha\pm\beta)=\cos\alpha\cos\beta\mp\sin\alpha\sin\beta.
\end{align}
\begin{align}
	\underset{k<0}{x_2[k]}&=\sqrt{2}\bigg(\cos(\frac{\pi}{4}k-\frac{\pi}{4})-\sin(\frac{\pi}{4}k-\frac{\pi}{4})\bigg)-2\cos(\frac{\pi}{4}|k|)\nonumber\\
	&=\sqrt{2}\bigg(\frac{\sqrt{2}}{2}\cos(\frac{\pi}{4}k)+\frac{\sqrt{2}}{2}\sin(\frac{\pi}{4}k)-\frac{\sqrt{2}}{2}\sin(\frac{\pi}{4}k)+\frac{\sqrt{2}}{2}\cos(\frac{\pi}{4}k)\bigg)-2\cos(\frac{\pi}{4}|k|)\nonumber\\
	&=2\cos(\frac{\pi}{4}k)-2\cos(\frac{\pi}{4}|k|)\nonumber \\
	&\underset{k<0}{=}2\cos(\frac{\pi}{4}k)-2\cos(-\frac{\pi}{4}k)\nonumber\\
	&=2\cos(\frac{\pi}{4}k)-2\cos(\frac{\pi}{4}k)=0
\end{align}
\begin{align}
	x[k]=\begin{cases}
		\sqrt{2}\bigg(\cos(\frac{\pi}{4}k-\frac{\pi}{4})-\sin(\frac{\pi}{4}k-\frac{\pi}{4})\bigg)=2\cos(\frac{\pi}{4}k),&k\geq0 \\
		0, &k<0
	\end{cases}
\end{align}
%\subsubsection{Task b)}
\subsubsection{Aufgabe b)}
%We shall find the time-discret signal for
Wir wollen das zeit-diskrete Signal für
\begin{align}
	X(z)=\frac{z}{z-\frac{1}{2}}
\end{align}
berechnen.
%$\frac{1}{2}$ is a pole of order 1. \\
$\frac{1}{2}$ ist eine Polstelle erster Ordnung. \\
%\begin{align}
%	0\text{ is }
%	\begin{cases}
%		\text{no isolated singularity}, &k > 0 \\
%		\text{removable singularity}, &k=0 \\
%		\text{pole of order }|k|, &k < 0.
%	\end{cases}
%\end{align}
\begin{align}
	0\text{ ist }
	\begin{cases}
		\text{keine isolierte Singularität}, &k > 0 \\
		\text{eine hebbare Singularität}, &k=0 \\
		\text{eine Polstelle }|k|\text{-ter Ordnung}, &k < 0.
	\end{cases}
\end{align}
\begin{align}
	\mathrm{Res}(X(z)z^{k-1},\frac{1}{2})=\lim\limits_{z\rightarrow\frac{1}{2}}z\cdot z^{k-1}=\bigg (\frac{1}{2} \bigg)^k
\end{align}
\begin{align}
	\underset{k<0}{\mathrm{Res}(X(z)z^{k-1},0)}&=\lim\limits_{z\rightarrow0}\frac{1}{(|k|-1)!}\frac{\mathrm{d}^{|k|-1}}{\mathrm{d}z^{|k|-1}}\Bigg [\frac{z}{(z-\frac{1}{2})\cdot z^{|k|+1}}\cdot z^{|k|}\Bigg]=\lim\limits_{z\rightarrow0}\frac{1}{(|k|-1)!}\frac{\mathrm{d}^{|k|-1}}{\mathrm{d}z^{|k|-1}}\Bigg [\frac{1}{z-\frac{1}{2}}\Bigg]\nonumber\\
	&=\frac{1}{(|k|-1)!}\frac{(-1)^{|k|-1}\cdot (|k|-1)!}{(-\frac{1}{2})^{|k|}}=-2^{|k|}
\end{align}
\begin{align}
	x[k]=\begin{cases}
		x_1[k], &k\geq 0 \\
		x_2[k], &k<0
	\end{cases}
\end{align}
\begin{align}
	x_1[k]=\mathrm{Res}(X(z)z^{k-1},\frac{1}{2})=\bigg( \frac{1}{2} \bigg )^k
\end{align}
\begin{align}
	\underset{k<0}{x_2[k]}=\mathrm{Res}(X(z)z^{k-1},\frac{1}{2})+\mathrm{Res}(X(z)z^{k-1},0)=\bigg ( \frac{1}{2} \bigg)^k-2^{|k|}=0
\end{align}
\begin{align}
	x[k]=
	\begin{cases}
		\bigg (\frac{1}{2} \bigg)^k, &k\geq0 \\
		0, &k<0
	\end{cases}
\end{align}
%\subsubsection{Task c)}
\subsubsection{Aufgabe c)}
%We shall find the time-discret signal for
Wir wollen das zeit-diskrete Signal für
\begin{align}
	X(z)=\frac{z}{z-\frac{1}{2}}\cdot\frac{z}{z-1}
\end{align}
berechnen.
%$\frac{1}{2}$ and $1$ are poles of order $1$.
$\frac{1}{2}$ und $1$ sind Polstellen erster Ordnung.
%\begin{align}
%	0\text{ is }
%	\begin{cases}
%		\text{no isolated singularity}, &k>0 \\
%		\text{removeable singularity}, &-1\leq k \leq0 \\
%		\text{pole of order }|k|-1, &k < -1.
%	\end{cases}
%\end{align}
\begin{align}
	0\text{ ist }
	\begin{cases}
		\text{keine isolierte Singularität}, &k>0 \\
		\text{eine hebbare Singularität}, &-1\leq k \leq0 \\
		\text{eine Polstelle }(|k|-1)\text{-ter Ordnung}, &k < -1.
	\end{cases}
\end{align}
\begin{align}
	\mathrm{Res}(X(z)z^{k-1},\frac{1}{2})=\lim\limits_{z\rightarrow\frac{1}{2}}\frac{z^2}{z-1}z^{k-1}=\frac{\bigg (\frac{1}{2}\bigg)^{k+1}}{\frac{1}{2}-1}=-\bigg (\frac{1}{2}\bigg)^k
\end{align}
\begin{align}
	\mathrm{Res}(X(z)z^{k-1},1)=\lim\limits_{z\rightarrow1}\frac{z^2}{z-\frac{1}{2}}z^{k-1}=1^{k+1}{1-\frac{1}{2}}=2
\end{align}
\begin{align}
	\underset{k<-1}{\mathrm{Res}(X(z)z^{k-1},0)}=\lim\limits_{z\rightarrow0}\frac{1}{(|k|-2)!}\frac{\mathrm{d}^{|k|-2}}{\mathrm{d}z^{|k|-2}}\Bigg [\frac{z}{z-\frac{1}{2}}\cdot\frac{z}{z-1}\frac{1}{z^{|k|+1}}\cdot z^{|k|-1}\Bigg]=\lim\limits_{z\rightarrow0}\frac{1}{(|k|-2)!}\frac{\mathrm{d}^{|k|-2}}{\mathrm{d}z^{|k|-2}}\Bigg [\frac{1}{(z-1)\cdot(z-\frac{1}{2})}\Bigg]
\end{align}
%We use partial fraction decomposition to make the differentiation easier.
Wir nutzen Partialbruchzerlegung, um uns das Ableiten zu vereinfachen.
\begin{align}
	\frac{1}{(z-\frac{1}{2})\cdot(z-1)}=\frac{A}{z-\frac{1}{2}}+\frac{B}{z-1}\quad \Bigg | \quad \cdot(z-1)\cdot(z-\frac{1}{2}) \nonumber \\
	1 = A(z-1)+B(z-\frac{1}{2})=z(A+B)+(-A-\frac{B}{2})
\end{align}
%We get a system of linear equations. In matrix notation:
Wir erhalten ein lineares Gleichungssystem, hier dargestellt in der Matrixnotation:
\begin{align}
	\begin{pmatrix}
		1 & 1 \\
		-1 & -\frac{1}{2}
	\end{pmatrix}
	\begin{pmatrix}
		A \\
		B
	\end{pmatrix}
	=
	\begin{pmatrix}
		0 \\
		1
	\end{pmatrix}
\end{align}
%With techniques like Gaussian elimination or Cramer's rule we can solute our system of linear equations and achieve
Das lineare Gleichungssystem lässt sich mit Techniken wie dem Gaußschem Eliminationsverfahren oder der Cramerschen Regel lösen und wir erhalten
\begin{align}
	A = -2, \nonumber \\
	B = 2.
\end{align}
\begin{align}
	\underset{k<-1}{\mathrm{Res}(X(z)z^{k-1},0)}&=\lim\limits_{z\rightarrow0}\frac{1}{(|k|-2)!}\cdot2\frac{\mathrm{d}^{|k|-2}}{\mathrm{d}z^{|k|-2}}\Bigg [\frac{-1}{z-\frac{1}{2}}+\frac{1}{z-1}\Bigg]\nonumber \\
	&=\lim\limits_{z\rightarrow0}\frac{2}{(|k|-2)!}\Bigg [-\frac{(-1)^{|k|}\cdot(|k|-2)!}{(z-\frac{1}{2})^{|k|-1}\cdot(|k|-2)!}+\frac{(-1)^{|k|}}{(z-1)^{|k|-1}}\Bigg] \nonumber \\
	&=\frac{2}{(|k|-2)!}\Bigg [-\frac{(-1)^{|k|}(|k|-2)!}{(-\frac{1}{2})^{|k|-1}}+\frac{(-1)^{|k|}(|k|-2)!}{(-1)^{|k|-1}}\Bigg] \nonumber \\
	&=2\cdot \Bigg [2^{|k|-1}-1\Bigg]=2^{|k|}-2
\end{align}
\begin{align}
	x[k]=
	\begin{cases}
		x_1[k], &k\geq-1 \\
		x_2[k], &k-1
	\end{cases}
\end{align}
\begin{align}
	x_1[k]=\mathrm{Res}(X(z)z^{k-1},\frac{1}{2})+\mathrm{Res}(X(z)z^{k-1},1)=2-\bigg (\frac{1}{2}\bigg)^k
\end{align}
\begin{align}
	\underset{k<0}{x_2[k]}=\mathrm{Res}(X(z)z^{k-1},\frac{1}{2})+\mathrm{Res}(X(z)z^{k-1},1)+\mathrm{Res}(X(z)z^{k-1},0)=2-\bigg (\frac{1}{2})^k+2^{|k|}-2=0
\end{align}
%It also is $2-\bigg (\frac{1}{2}\bigg)^{-1}=0$. \\
Es gilt außerdem: $2-\bigg (\frac{1}{2}\bigg)^{-1}=0$.
Zusammenfassung:
\begin{align}
	x[k]=
	\begin{cases}
		2-\bigg (\frac{1}{2}\bigg)^k, &k\geq 0 \\
		0, &k< 0
	\end{cases}
\end{align}
%\subsubsection{Task d)}
\subsubsection{Aufgabe d)}
%We shall find the time-discret signal for
Wir wollen das zeit-diskrete Signal für
\begin{align}
	X(z)=\frac{z^2-z+2}{z^2-\frac{1}{2}z+\frac{1}{4}}
\end{align}
berechnen.
%$\frac{1}{4}+\im\frac{\sqrt{3}}{4}$ and $\frac{1}{4}-\im\frac{\sqrt{3}}{4}$ are poles of order 1.
$\frac{1}{4}+\im\frac{\sqrt{3}}{4}$ und $\frac{1}{4}-\im\frac{\sqrt{3}}{4}$ sind Polstellen erster Ordnung.
%\begin{align}
%	0 \text{ is }\begin{cases}
%		\text{no isolated singularity}, &k>0 \\
%		\text{pole of order } |k|+1, &k\leq 0.
%	\end{cases}
%\end{align}
\begin{align}
	0 \text{ ist }\begin{cases}
		\text{keine Singularität}, &k>0 \\
		\text{eine Polstelle} (|k|+1)\text{-ter Ordnung}, &k\leq 0.
	\end{cases}
\end{align}
\begin{align}
	\mathrm{Res}(X(z)z^{k-1},\frac{1}{4}+\im\frac{\sqrt{3}}{4})&=\lim\limits_{z\rightarrow\frac{1}{4}+\im\frac{\sqrt{3}}{4}}\frac{z^2-z+2}{z-(\frac{1}{4}-\im\frac{\sqrt{3}}{4})}z^{k-1}=\lim\limits_{z\rightarrow\frac{1}{2}\e^{+\im\frac{\pi}{3}}}\frac{z^2-z+2}{z-(\frac{1}{2}\e^{-\im\frac{\pi}{3}})}z^{k-1}\nonumber\\
	&=\frac{\frac{1}{4}\e^{+\im\frac{2\pi}{3}}-\frac{1}{2}\e^{+\im\frac{\pi}{3}}+2}{\frac{1}{2}\e^{+\im\frac{\pi}{3}}-\frac{1}{2}\e^{-\im\frac{\pi}{3}}}\bigg(\frac{1}{2}\bigg )^{k-1}\e^{+\im\frac{\pi}{3}(k-1)}\nonumber\\
	&=\frac{\frac{1}{4}(\cos(\frac{2\pi}{3})+\im\sin(\frac{2\pi}{3}))-\frac{1}{2}(\cos(\frac{\pi}{3})+\im\sin(\frac{\pi}{3}))+2}{\frac{1}{2}(\cos(\frac{\pi}{3})+\im\sin(\frac{\pi}{3})-\cos(\frac{\pi}{3})+\im\sin(\frac{\pi}{3}))}\bigg(\frac{1}{2}\bigg )^{k-1}\e^{+\im\frac{\pi}{3}(k-1)}\nonumber\\
	&=\frac{\frac{1}{4}(-\frac{1}{2}+\im\frac{\sqrt{3}}{2})-\frac{1}{2}(\frac{1}{2}+\im\frac{\sqrt{3}}{2})+2}{\im\frac{\sqrt{3}}{2}}\bigg(\frac{1}{2}\bigg )^{k-1}\e^{+\im\frac{\pi}{3}(k-1)}\nonumber\\
	&=2\cdot\frac{1}{\im\sqrt{3}}\cdot\bigg [-\frac{1}{8}+\im\frac{\sqrt{3}}{8}-\frac{1}{4}-\im\frac{\sqrt{3}}{4}+2\bigg]\bigg(\frac{1}{2}\bigg )^{k-1}\e^{+\im\frac{\pi}{3}(k-1)}\nonumber\\
	&=\frac{1}{\im\sqrt{3}}\cdot\bigg [\frac{13}{4}-\im\frac{\sqrt{3}}{4}\bigg]\bigg(\frac{1}{2}\bigg)^{k-1}\e^{+\im\frac{\pi}{3}(k-1)}
\end{align}
\begin{align}
	\mathrm{Res}(X(z)z^{k-1},\frac{1}{4}-\im\frac{\sqrt{3}}{4})&=\lim\limits_{z\rightarrow\frac{1}{4}-\im\frac{\sqrt{3}}{4}}\frac{z^2-z+2}{z-(\frac{1}{4}+\im\frac{\sqrt{3}}{4})}z^{k-1}=\lim\limits_{z\rightarrow\frac{1}{2}\e^{-\im\frac{\pi}{3}}}\frac{z^2-z+2}{z-(\frac{1}{2}\e^{+\im\frac{\pi}{3}})}z^{k-1}\nonumber\\
	&=\frac{\frac{1}{4}\e^{-\im\frac{2\pi}{3}}-\frac{1}{2}\e^{-\im\frac{\pi}{3}}+2}{\frac{1}{2}\e^{-\im\frac{\pi}{3}}-\frac{1}{2}\e^{+\im\frac{\pi}{3}}}\bigg(\frac{1}{2}\bigg )^{k-1}\e^{-\im\frac{\pi}{3}(k-1)}\nonumber\\
	&=\frac{\frac{1}{4}(\cos(\frac{2\pi}{3})-\im\sin(\frac{2\pi}{3}))-\frac{1}{2}(\cos(\frac{\pi}{3})-\im\sin(\frac{\pi}{3}))+2}{\frac{1}{2}(\cos(\frac{\pi}{3})-\im\sin(\frac{\pi}{3})-\cos(\frac{\pi}{3})-\im\sin(\frac{\pi}{3}))}\bigg(\frac{1}{2}\bigg )^{k-1}\e^{-\im\frac{\pi}{3}(k-1)}\nonumber\\
	&=\frac{\frac{1}{4}(-\frac{1}{2}-\im\frac{\sqrt{3}}{2})-\frac{1}{2}(\frac{1}{2}-\im\frac{\sqrt{3}}{2})+2}{-\im\frac{\sqrt{3}}{2}}\bigg(\frac{1}{2}\bigg )^{k-1}\e^{+\im\frac{\pi}{3}(k-1)}\nonumber\\
	&=2\cdot\frac{1}{\sqrt{3}}\cdot\bigg [\frac{1}{8}+\im\frac{\sqrt{3}}{8}+\frac{1}{4}-\im\frac{\sqrt{3}}{4}-2\bigg]\bigg(\frac{1}{2}\bigg )^{k-1}\e^{-\im\frac{\pi}{3}(k-1)}\nonumber\\
	&=\frac{1}{\im\sqrt{3}}\cdot\bigg [-\frac{13}{4}-\im\frac{\sqrt{3}}{4}\bigg]\bigg(\frac{1}{2}\bigg)^{k-1}\e^{-\im\frac{\pi}{3}(k-1)}
\end{align}
%We use partial fraction decomposition to make the differentiation easier. Take a look at \eqref{eq:TaskDPartialFractionDecomposition} for the solution.
Wir nutzen Partialbruchzerlegung, um uns das Ableiten zu vereinfachen. Wir nutzen die Lösung aus \eqref{eq:TaskDPartialFractionDecomposition}.
\begin{align}
	\underset{k\leq0}{\mathrm{Res}(X(z)z^{k-1},0)}=\lim\limits_{z\rightarrow0}\frac{1}{|k|!}\frac{\mathrm{d}^{|k|}}{\mathrm{d}z^{|k|}} \Bigg [ \frac{z^2-z+2}{z^2-\frac{1}{2}z+\frac{1}{4}}\cdot\frac{z^{|k|+1}}{z^{|k|+1}}\Bigg ]=\lim\limits_{z\rightarrow0}\frac{1}{|k|!}\frac{\mathrm{d}^{|k|}}{\mathrm{d}z^{|k|}}\Bigg [1+\frac{-\frac{1}{2}z+\frac{7}{4}}{z^2-\frac{1}{2}z+\frac{1}{4}} \Bigg]\nonumber\\
\end{align}
\begin{align}
	\underset{k\leq0}{\mathrm{Res}(X(z)z^{k-1},0)}&=\lim\limits_{z\rightarrow0}\frac{1}{|k|!}\frac{\mathrm{d}^{|k|}}{\mathrm{d}z^{|k|}}\Bigg [1+\frac{-\frac{1}{4}-\im\frac{13}{4\cdot\sqrt{3}}}{z-(\frac{1}{4}+\im\frac{\sqrt{3}}{4})}+\frac{-\frac{1}{4}+\im\frac{13}{4\cdot\sqrt{3}}}{z-(\frac{1}{4}-\im\frac{\sqrt{3}}{4})}\Bigg]\nonumber\\
	&=\frac{\mathrm{d}^{|k|}1}{\mathrm{d}z^{|k|}}+\lim\limits_{z\rightarrow0}\frac{1}{|k|}\frac{\mathrm{d}^{|k|}}{\mathrm{d}z^{|k|}}\Bigg[\frac{-\frac{1}{4}-\im\frac{13}{4\cdot\sqrt{3}}}{z-(\frac{1}{4}+\im\frac{\sqrt{3}}{4})}+\frac{-\frac{1}{4}+\im\frac{13}{4\cdot\sqrt{3}}}{z-(\frac{1}{4}-\im\frac{\sqrt{3}}{4})}\Bigg] \nonumber \\
	&=\begin{cases}
		1+\frac{-\frac{1}{4}-\im\frac{13}{4\cdot\sqrt{3}}}{-\frac{1}{4}-\im\frac{\sqrt{3}}{4}}+\frac{-\frac{1}{4}+\im\frac{13}{4\cdot\sqrt{3}}}{-\frac{1}{4}+\im\frac{\sqrt{3}}{4}}, &k = 0 \\
		\frac{-\frac{1}{4}-\im\frac{13}{4\cdot\sqrt{3}}}{(-\frac{1}{4}+\im\frac{\sqrt{3}}{4})^{|k|+1}}\cdot(-1)^{|k|}+\frac{-\frac{1}{4}+\im\frac{13}{4\cdot\sqrt{3}}}{(-\frac{1}{4}-\im\frac{\sqrt{3}}{4})^{|k|+1}}\cdot(-1)^{|k|}, &k <0
	\end{cases}
\end{align}
\begin{align}
		\mathrm{Res}(X(z)z^{-1},0)&= 1+\frac{-\frac{1}{4}-\im\frac{13}{4\cdot\sqrt{3}}}{-\frac{1}{4}-\im\frac{\sqrt{3}}{4}}+\frac{-\frac{1}{4}+\im\frac{13}{4\cdot\sqrt{3}}}{-\frac{1}{4}+\im\frac{\sqrt{3}}{4}}=1+\frac{1+\im\frac{13}{\sqrt{3}}}{1+\im\sqrt{3}}\cdot\frac{1-\im\sqrt{3}}{1-\im\sqrt{3}}+\frac{1-\im\frac{13}{\sqrt{3}}}{1-\im\sqrt{3}}\cdot\frac{1+\im\sqrt{3}}{1+\im\sqrt{3}}\nonumber\\
		&=1+\frac{1-\im\sqrt{3}+13+\im\frac{13}{\sqrt{3}}+1+\im\sqrt{3}-\im\frac{13}{\sqrt{3}}+13}{1+3}=8
\end{align}
\begin{align}
	\underset{k<0}{\mathrm{Res}(X(z)z^{|k|+1},0)}&=4^{|k|}\Bigg[\frac{1+\im\frac{13}{\sqrt{3}}}{(1-\im\sqrt{3})^{|k|+1}}+\frac{1-\im\frac{13}{\sqrt{3}}}{(1+\im\sqrt{3})^{|k|+1}} \Bigg]=4^{|k|}\frac{(1+\im\frac{13}{\sqrt{3}})\cdot2^{|k|+1}\e^{+\im\frac{\pi}{3}(|k|+1)}+(1-\im\frac{13}{\sqrt{3}})2^{|k|+1}\e^{-\im\frac{\pi}{3}(|k|+1)}}{2^{|k|+1}\e^{-\im\frac{\pi}{3}(|k|+1)}2^{|k|+1}\e^{+\im\frac{\pi}{3}(|k|+1)}}\nonumber\\
	&=\frac{2^{|k|}}{4}\Bigg [4\cdot\frac{1}{2}\bigg(\e^{+\im\frac{\pi}{3}(|k|+1)}+\e^{-\im\frac{\pi}{3}(|k|+1)}\bigg)+\frac{52}{\sqrt{3}}\cdot\frac{1}{2\im}\bigg(\e^{+\im\frac{\pi}{3}(|k|+1)}-\e^{-\im\frac{\pi}{3}(|k|+1)}\bigg)\Bigg]\nonumber\\
	&=2^{|k|}\Bigg[\cos(\frac{\pi}{3}(|k|+1))+\frac{13}{\sqrt{3}}\sin(\frac{\pi}{3}(|k|+1))\Bigg]
\end{align}
\begin{align}
	x[k]=\begin{cases}
		x_1[k], &k>0 \\
		x_2[k], &k=0 \\
		x_3[k], &k<0
	\end{cases}
\end{align}
\begin{align}
	x_1[k]&=\mathrm{Res}(X(z)z^{k-1},\frac{1}{4}+\im\frac{\sqrt{3}}{4})+\mathrm{Res}(X(z)z^{k-1},\frac{1}{4}-\im\frac{\sqrt{3}}{4})\nonumber\\
	&=\frac{1}{\im\sqrt{3}}\cdot\bigg [\frac{13}{4}-\im\frac{\sqrt{3}}{4}\bigg]\bigg(\frac{1}{2}\bigg)^{k-1}\e^{+\im\frac{\pi}{3}(k-1)}+\frac{1}{\im\sqrt{3}}\cdot\bigg [-\frac{13}{4}-\im\frac{\sqrt{3}}{4}\bigg]\bigg(\frac{1}{2}\bigg)^{k-1}\e^{-\im\frac{\pi}{3}(k-1)}\nonumber\\
	&=\bigg(\frac{1}{2}\bigg)^{k-1}\Bigg[-\frac{1}{2\cdot2}\bigg(\e^{+\im\frac{\pi}{3}(k-1)}+\e^{-\im\frac{\pi}{3}(k-1)}\bigg)+\frac{13}{\im\sqrt{3}\cdot2\cdot2}\bigg(\e^{+\im\frac{\pi}{3}(k-1)}-\e^{-\im\frac{\pi}{3}(k-1)}\bigg)\Bigg]\nonumber\\
	&=\bigg(\frac{1}{2}\bigg)^{k}\Bigg[\frac{13}{\sqrt{3}}\sin(\frac{\pi}{3}(k-1))-\cos(\frac{\pi}{3}(k-1))\Bigg]
\end{align}
\begin{align}
	x_2[k]&=\mathrm{Res}(X(z)z^{-1},0)+\mathrm{Res}(X(z)z^{k-1},\frac{1}{4}+\im\frac{\sqrt{3}}{4})\Bigg |_{k=0}+\mathrm{Res}(X(z)z^{k-1},\frac{1}{4}-\im\frac{\sqrt{3}}{4})\Bigg |_{k=0}\nonumber\\
	&=8+x_1[k=0]=8+\frac{13}{\sqrt{3}}\sin(-\frac{\pi}{3})-\cos(-\frac{\pi}{3})=8+\frac{13}{\sqrt{3}}\cdot\frac{-\sqrt{3}}{2}-\frac{1}{2}=8-\frac{13}{2}-\frac{1}{2}\nonumber\\
	&=8-\frac{14}{2}=8-7=1
\end{align}
\begin{align}
	\underset{k<0}{x_3[k]}&=\mathrm{Res}(X(z)z^{|k|+1},0)+\mathrm{Res}(X(z)z^{k-1},\frac{1}{4}+\im\frac{\sqrt{3}}{4})+\mathrm{Res}(X(z)z^{k-1},\frac{1}{4}-\im\frac{\sqrt{3}}{4})\nonumber\\
	&=x_1[k]+2^{|k|}\Bigg[\cos(\frac{\pi}{3}(|k|+1))+\frac{13}{\sqrt{3}}\sin(\frac{\pi}{3}(|k|+1))\Bigg]\nonumber\\
	&=2^{|k|}\Bigg[\cos(\frac{\pi}{3}(|k|+1))+\frac{13}{\sqrt{3}}\sin(\frac{\pi}{3}(|k|+1))\Bigg]+\bigg(\frac{1}{2}\bigg)^{k}\Bigg[\frac{13}{\sqrt{3}}\sin(\frac{\pi}{3}(k-1))-\cos(\frac{\pi}{3}(k-1))\Bigg]\nonumber\\
	&\underset{k<0}{=}2^{|k|}\Bigg[\cos(\frac{\pi}{3}(1-k))+\frac{13}{\sqrt{3}}\sin(\frac{\pi}{3}(1-k))+\frac{13}{\sqrt{3}}\sin(\frac{\pi}{3}(k-1))-\cos(\frac{\pi}{3}(k-1))\Bigg]\nonumber\\
	&=2^{|k|}\Bigg[\cos(\frac{\pi}{3}(-(k-1)))+\frac{13}{\sqrt{3}}\sin(\frac{\pi}{3}(-(k-1)))+\frac{13}{\sqrt{3}}\sin(\frac{\pi}{3}(k-1))-\cos(\frac{\pi}{3}(k-1))\Bigg]\nonumber\\
	&=2^{|k|}\Bigg[\cos(\frac{\pi}{3}(k-1))-\frac{13}{\sqrt{3}}\sin(\frac{\pi}{3}(k-1))+\frac{13}{\sqrt{3}}\sin(\frac{\pi}{3}(k-1))-\cos(\frac{\pi}{3}(k-1))\Bigg] \nonumber\\
	&=0
\end{align}
\begin{align}
	x_[k]=\begin{cases}
		\bigg(\frac{1}{2}\bigg)^{k}\Bigg[\frac{13}{\sqrt{3}}\sin(\frac{\pi}{3}(k-1))-\cos(\frac{\pi}{3}(k-1))\Bigg], &k>0 \\
		1, &k=0 \\
		0, &k<0
	\end{cases}
\end{align}
%\subsubsection{Task e)}
\subsubsection{Aufgabe e )}
%We shall find the time-discret signal for
Wir wollen das zeit-diskrete Signal für
\begin{align}
	X(z)=\frac{z^2}{z^2+1}
\end{align}
berechnen.
%$\im$ and $-\im$ are poles of order 1.
$\im$ und $-\im$ sind Polstellen erster Ordnung.
%\begin{align}
%	0 \text{ is }\begin{cases}
%		\text{no isolated singularity}, &k> 0,\\
%		\text{removeable singularity}, &-1\leq k \leq 0,\\
%		\text{pole of order } |k|-1, &k<-1.
%	\end{cases}
%\end{align}
\begin{align}
	0 \text{ ist }\begin{cases}
		\text{keine isolierte Singularität}, &k> 0,\\
		\text{eine hebbare Singularität}, &-1\leq k \leq 0,\\
		\text{eine Polstelle } (|k|-1)\text{-ter Ordnung}, &k<-1.
	\end{cases}
\end{align}
\begin{align}
	\mathrm{Res}(X(z)z^{k-1},\im)=\lim\limits_{z\rightarrow\im}\frac{z^2}{z+\im}z^{k-1}=\lim\limits_{z\rightarrow\e^{+\im\frac{\pi}{2}}}\frac{z^{k+1}}{z+\im}=\frac{1}{2\im}\e^{+\im\frac{\pi}{2}(k+1)}
\end{align}
\begin{align}
	\mathrm{Res}(X(z)z^{k-1},-\im)=\lim\limits_{z\rightarrow-\im}\frac{z^2}{z-\im}z^{k-1}=\lim\limits_{z\rightarrow\e^{-\im\frac{\pi}{2}}}\frac{z^{k+1}}{z-\im}=\frac{1}{-2\im}\e^{-\im\frac{\pi}{2}(k+1)}
\end{align}
\begin{align}
	\underset{k<-1}{\mathrm{Res}(X(z)z^{k-1},0})=\lim\limits_{z\rightarrow0}\frac{1}{(|k|-2)!}\frac{\mathrm{d}^{|k|-2}}{\mathrm{d}z^{|k|-2}}\Bigg [\frac{z^2}{z^2+1} \cdot\frac{z^{|k|-1}}{z^{|k|+1}}\Bigg]=\lim\limits_{z\rightarrow0}\frac{1}{(|k|-2)!}\frac{\mathbb{d}^{|k|-2}}{\mathrm{d}z^{|k|-2}}\Bigg [\frac{1}{z^2+1}\Bigg]
\end{align}
%We us partial fraction decomposition to make the differentiation easier.
Wir nutzen Partialbruchzerlegung, um uns das Ableiten zu vereinfachen.
\begin{align}
	\frac{1}{(z-\im)\cdot(z+\im)}=\frac{A}{z-\im}+\frac{B}{z+\im}\quad\Bigg | \quad \cdot(z-\im)\cdot(z+\im) \nonumber \\
	1=A(z+\im)+B(z-\im)=z(A+B)+(\im A -\im B)
\end{align}
%We get a system of linear equations. In matrix notation:
Wir erhalten ein lineares Gleichungssystem, hier dargestellt in der Matrixnotation:
\begin{align}
	\begin{pmatrix}
		1 & 1 \\
		\im & -\im
	\end{pmatrix}
	\begin{pmatrix}
		A \\
		B
	\end{pmatrix}
	=
	\begin{pmatrix}
		0 \\
		1
	\end{pmatrix}.
\end{align}
%With techniques like Gaussian elimination or Cramer's rule, we can solute our system of linear equations and achieve
Das lineare Gleichungssystem lässt sich mit Techniken wie dem Gaußschem Eliminationsverfahren oder der Cramerschen Regel lösen und wir erhalten
\begin{align}
	A= -\frac{1}{2}\im, \nonumber \\
	B = \frac{1}{2}\im.
\end{align}
\begin{align}
	\underset{k<-1}{\mathrm{Res}(X(z)z^{k-1},0)}&=\lim\limits_{z\rightarrow0}\frac{\im}{2\cdot(|k|-2)!}\frac{\mathrm{d}^{|k|-2}}{\mathrm{d}z^{|k|-2}}\Bigg [\frac{-1}{z-\im}+\frac{1}{z+\im} \Bigg ] \nonumber \\
	&=\lim\limits_{z\rightarrow0}\frac{\im}{2\cdot(|k|-2)!}\Bigg [-\frac{(-1)^{|k|}\cdot(|k|-2)!}{(z-\im)^{|k|-1}}+\frac{(1)^{|k|}}{(z+\im)^{|k|-1}} \Bigg]\nonumber\\
	&=\frac{1}{2}\im \cdot\frac{1}{(|k|-2)!}\Bigg [-\frac{(-1)^{|k|}\cdot(|k|-2)!}{(-\im)^{|k|-1}}+\frac{(-1)^{|k|}\cdot(|k|-2)!}{(\im)^{|k|-1}}\Bigg] \nonumber \\
	&=\frac{1}{2}\im\Bigg[\frac{1}{(\im)^{|k|-1}}+\frac{(-1)^{k}}{(\im)^{|k|-1}}\Bigg ] \nonumber \\
	&=\frac{1}{2}\im\cdot\frac{1+(-1)^{|k|}}{(\im)^{|k|-1}}=\frac{1}{2}\cdot\frac{1+(-1)^{|k|}}{(\im)^{|k|-2}}=-\frac{1}{2}(1+(-1)^{|k|})\im^{|k|}\cdot\im^{-2}=-\frac{1}{2}\cdot(1+(-1)^{|k|})\im^{|k|} \nonumber \\
%	&=\begin{cases}
%		-(-1)^{\frac{|k|}{2}}, &|k|\text{ even} \\
%		0, &|k|\text{ odd}
%	\end{cases}
	&=\begin{cases}
		-(-1)^{\frac{|k|}{2}}, &|k|\text{ gerade} \\
		0, &|k|\text{ ungerade}
	\end{cases}
\end{align}
\begin{align}
	x[k]=\begin{cases}
		x_1[k], &k\geq -1 \\
		x_2[k], &k< -1
	\end{cases}
\end{align}
\begin{align}
	x_1[k]=\mathrm{Res}(X(z)z^{k-1},\im)+\mathrm{Res}(X(z)z^{k-1},-\im)=\frac{1}{2\im}\bigg(\e^{+\im\frac{\pi}{2}(k+1)}-\e^{-\im\frac{\pi}{2}(k+1)}\bigg )=\sin(\frac{\pi}{2}(k+1))
\end{align}
\begin{align}
	\underset{k<-1}{x_2[k]}&=\mathrm{Res}(X(z)z^{k-1},\im)+\mathrm{Res}(X(z)z^{k-1},-\im)+\mathrm{Res}(X(z)z^{k-1},0) \nonumber \\
	&=\begin{cases}
		\sin(\frac{\pi}{2}(k+1))-(-1)^{|k|}, &|k| \text{ even} \\
		\sin(\frac{\pi}{2}(k+1)), &|k| \text{ odd}
	\end{cases}
\end{align}
%If $|k|$ is odd, $\frac{\pi}{2}(k+1)$ a multiple of $\pi$, so that $\sin(\frac{\pi}{2}(k+1))=0$.
%If $k<-1$ is even, $\sin(\frac{\pi}{2}(k+1))=(-1)^{\frac{|k|}{2}}$.\\
%It is:
Falls $|k|$ gerade ist, ist $\frac{\pi}{2}(k+1)$ ein Vielfaches von $\pi$, sodass $\sin(\frac{\pi}{2}(k+1))=0$ ist.
Falls $k<-1$ ungerade ist, ist $\sin(\frac{\pi}{2}(k+1))=(-1)^{\frac{|k|}{2}}$.\\
Somit gilt:
\begin{align}
	x_2[k]=0.
\end{align}
Es ist außerdem $x_1[-1] =\sin(\frac{\pi}{2}(-1+1))=0$. \\
Zusammenfassung:
\begin{align}
	x[k]=\begin{cases}
		\sin(\frac{\pi}{2}(k+1))\underset{\text{\eqref{eq:SinAlphaPlusMinusBeta}}}{=}\cos(\frac{\pi}{2}k), &k\geq 0 \\
		0, &k<0
	\end{cases}.
\end{align}
%\subsubsection{Task f)}
\subsubsection{Aufgabe f)}
Wir wollen das zeit-diskrete Signal für
\begin{align}
	X(z)=\frac{z^4+z^3-6z^2+6z-1}{z^2-2z+1}
\end{align}
berechnen.
%$1$ is a pole of order 2.
$1$ ist eine Polstelle zweiter Ordnung.
%\begin{align}
%	0\text{ is }\begin{cases}
%		\text{no isolated singularity}, &k>0\\
%		\text{pole of order }|k|+1, &k\leq0.
%	\end{cases}
%\end{align}
\begin{align}
	0\text{ ist }\begin{cases}
		\text{keine isolierte Singularität}, &k>0\\
		\text{eine Polstelle } (|k|+1)\text{-ter Ordnung}, &k\leq0.
	\end{cases}
\end{align}
\begin{align}
	\mathrm{Res}(X(z)z^{k-1},1)&=\lim\limits_{z\rightarrow1}\frac{\mathrm{d}}{\mathrm{d}z}\bigg [(z^4+z^3-6z^2+6z-1)z^{k-1}\bigg]\nonumber\\
	&=\lim\limits_{z\rightarrow1}\frac{\mathrm{d}}{\mathrm{d}z}\bigg[z^{k+3}+z^{k+2}-6z^{k+1}+6z^k-z^{k-1}\bigg]\nonumber\\
	&=\lim\limits_{z\rightarrow1}\bigg[(k+3)z^{k+2}+(k+2)z^{k+1}-6(k+1)z^{k}+6kz^{k-1}-(k-1)z^{k-2}\bigg ]\nonumber\\
	&=(k+3)+(k+2)-6(k+1)+6k-(k-1)=k
\end{align}
\begin{align}
	\underset{\leq0}{\mathrm{Res}(X(z)z^{k-1},0)}=\lim\limits_{z\rightarrow0}\frac{1}{(|k|)!}\frac{\mathrm{d}^{|k|}}{\mathrm{d}z^{|k|}}\Bigg[\frac{z^4+z^3-6z^2+6z-1}{(z-1)^2}\cdot\frac{z^{|k|+1}}{z^{|k|+1}}\Bigg]=\lim\limits_{z\rightarrow0}\frac{1}{(|k|)!}\frac{\mathrm{d}^{|k|}}{\mathrm{d}z^{|k|}}\Bigg[z^2+3z-1+\frac{z}{(z-1)^2}\Bigg]
\end{align}
%We use partial fraction decomposition to make the differentiation of the last term easier.
Wir nutzen Partialbruchzerlegung, um uns das Ableiten des letzten Terms zu vereinfachen.
\begin{align}
	\frac{z}{(z-1)\cdot(z-1)}=\frac{A}{z-1}+\frac{B}{(z-1)^2}\quad\Bigg | \quad \cdot(z-1)^2\nonumber\\
	z=A(z-1)+B=z\cdot A+1(-A+B)
\end{align}
%We get a system of linear equations. In matrix notation:
Wir erhalten ein lineares Gleichungssystem, hier dargestellt in der Matrixnotation:
\begin{align}
	\begin{pmatrix}
		1 & 0 \\
		-1 & 1
	\end{pmatrix}
	\begin{pmatrix}
		A \\
		B
	\end{pmatrix}
	=
	\begin{pmatrix}
		1 \\
		0
	\end{pmatrix}
\end{align}
%With techniques like Gaussian elimination or Cramer's rule (or in this case, with a good look), we can solute our system of linear equations and achieve
Das lineare Gleichungssystem lässt sich mit Techniken wie dem Gaußschem Eliminationsverfahren oder der Cramerschen Regel lösen und wir erhalten
\begin{align}
	A = 1, \nonumber \\
	B = 1.
\end{align}
\begin{align}
	\underset{k\leq0}{\mathrm{Res}(X(z)z^{k-1},0)}&=\lim\limits_{z\rightarrow0}\frac{1}{(|k|)!}\frac{\mathrm{d}^{|k|}}{\mathrm{d}z^{|k|}}\Bigg[z^2+3z-1+\frac{1}{z-1}+\frac{1}{(z-1)^2}\Bigg]\nonumber\\
	&=\begin{cases}
		0^2+3\cdot0-1+\frac{1}{0-1}+\frac{1}{(0-1)^2}=-1, &k=0 \\
		2\cdot0+3-\frac{1}{(0-1)^2}-\frac{2}{(0-1)^3}=4,&k=-1 \\
		\frac{1}{2}\bigg(2+\frac{2}{(0-1)^3}+\frac{6}{(0-1)^4}\bigg)=3,&k=-2\\
		\frac{1}{|k|}\bigg(\frac{(-1)^{|k|}\cdot (|k|)!}{(0-1)^{|k|+1}}+\frac{(-1)^{|k|}\cdot(|k|+1)!}{(0-1)^{|k|+2}}\bigg)=-1+(|k|+1)=|k|,&k<-2
	\end{cases}
\end{align}
\begin{align}
	x[k]=\begin{cases}
		x_1[k],&k>0 \\
		x_2[k], &k=0 \\
		x_3[k], &k=-1 \\
		x_4[k], &k=-2 \\
		x_5[k], &k<-2
	\end{cases}
\end{align}
\begin{align}
	x_1[k]=\mathrm{Res}(X(z)z^{k-1},1)=k
\end{align}
\begin{align}
	x_2[k]=\mathrm{Res}(X(z)z^{k-1},1)\Bigg |_{k=0}+\mathrm{Res}(X(z)z^{k-1},0)\Bigg |_{k=0}=x_1[k=0]-1=-1
\end{align}
\begin{align}
	x_3[k]=\mathrm{Res}(X(z)z^{k-1},1)\Bigg |_{k=-1}+\mathrm{Res}(X(z)z^{k-1},0)\Bigg |_{k=-1}=x_1[k=-1]+4=3
\end{align}
\begin{align}
	x_4[k]=\mathrm{Res}(X(z)z^{k-1},1)\Bigg |_{k=-2}+\mathrm{Res}(X(z)z^{k-1},0)\Bigg |_{k=-2}=x_1[k=-2]+3=1
\end{align}
\begin{align}
	x_5[k]\underset{k <-2}{=}\mathrm{Res}(X(z)z^{k-1},1)+\mathrm{Res}(X(z)z^{k-1},0)=k+|k|=-|k|+|k|=0
\end{align}
\begin{align}
	x[k]=\begin{cases}
		k, &k>0 \\
		-1, &k=0 \\
		3, &k=-1 \\
		1, &k=-2 \\
		0, &k<-2
	\end{cases}
\end{align}
%\subsubsection{Task g)}
\subsubsection{Aufgabe g)}
%We shall find the time-discret signal for
Wir wollen das zeit-diskrete Signal für
\begin{align}
	X(z)=\frac{z\cdot(z-1)}{z^2-\sqrt{2}z+1}
\end{align}
berechnen.
%$\frac{\sqrt{2}}{2}+\im\frac{\sqrt{2}}{2}$ and $\frac{\sqrt{2}}{2}-\im\frac{\sqrt{2}}{2}$ are poles of order 1.
$\frac{\sqrt{2}}{2}+\im\frac{\sqrt{2}}{2}$ und $\frac{\sqrt{2}}{2}-\im\frac{\sqrt{2}}{2}$ sind Polstellen erster Ordnung.
%\begin{align}
%	0\text{ is }\begin{cases}
%		\text{no isolated singularity}, &k>0 \\
%		\text{removeable singularity}, &k=0 \\
%		\text{pole of order }|k|, &k<0
%	\end{cases}
%\end{align}
\begin{align}
	0\text{ ist }\begin{cases}
		\text{keine isolierte Singularität}, &k>0 \\
		\text{hebbare Singularität}, &k=0 \\
		\text{eine Polstelle } (|k|)\text{-ter Ordnung}, &k<0
	\end{cases}
\end{align}
\begin{align}
	\mathrm{Res}(X(z)z^{k-1},\frac{\sqrt{2}}{2}+\im\frac{\sqrt{2}}{2})&=\lim\limits_{z\rightarrow\frac{\sqrt{2}}{2}+\im\frac{\sqrt{2}}{2}}\frac{z^2-z}{z-(\frac{\sqrt{2}}{2}-\im\frac{\sqrt{2}}{2})}z^{k-1}=\lim\limits_{z\rightarrow\e^{+\im\frac{\pi}{4}}}\frac{z^2-z}{z-\e^{-\im\frac{\pi}{4}}}z^{k-1}\nonumber\\
	&=\frac{\e^{+\im\frac{\pi}{2}}-\e^{+\im\frac{\pi}{4}}}{\e^{+\im\frac{\pi}{4}}-\e^{-\im\frac{\pi}{4}}}\e^{+\im\frac{\pi}{4}(k-1)}=\frac{\im-\cos(\frac{\pi}{4})-\im\sin(\frac{\pi}{4})}{\cos(\frac{\pi}{4})+\im\sin(\frac{\pi}{4})-\cos(\frac{\pi}{4})+\im\sin(\frac{\pi}{4})}\e^{+\im\frac{\pi}{4}(k-1)}\nonumber\\
	&=\frac{-\frac{\sqrt{2}}{2}+\im(1-\frac{\sqrt{2}}{2})}{\im\sqrt{2}}\e^{+\im\frac{\pi}{4}(k-1)}
\end{align}
\begin{align}
	\mathrm{Res}(X(z)z^{k-1},\frac{\sqrt{2}}{2}-\im\frac{\sqrt{2}}{2})&=\lim\limits_{z\rightarrow\frac{\sqrt{2}}{2}-\im\frac{\sqrt{2}}{2}}\frac{z^2-z}{z-(\frac{\sqrt{2}}{2}+\im\frac{\sqrt{2}}{2})}z^{k-1}=\lim\limits_{z\rightarrow\e^{-\im\frac{\pi}{4}}}\frac{z^2-z}{z-\e^{+\im\frac{\pi}{4}}}z^{k-1}\nonumber\\
	&=\frac{\e^{-\im\frac{\pi}{2}}-\e^{-\im\frac{\pi}{4}}}{\e^{-\im\frac{\pi}{4}}-\e^{+\im\frac{\pi}{4}}}\e^{-\im\frac{\pi}{4}(k-1)}=\frac{-\im-\cos(\frac{\pi}{4})+\im\sin(\frac{\pi}{4})}{\cos(\frac{\pi}{4})-\im\sin(\frac{\pi}{4})-\cos(\frac{\pi}{4})-\im\sin(\frac{\pi}{4})}\e^{+\im\frac{\pi}{4}(k-1)}\nonumber\\
	&=-\frac{-\frac{\sqrt{2}}{2}-\im(1-\frac{\sqrt{2}}{2})}{\im\sqrt{2}}\e^{-\im\frac{\pi}{4}(k-1)}
\end{align}
\begin{align}
	\underset{k<0}{\mathrm{Res}(X(z)z^{k-1},0)}&=\lim\limits_{z\rightarrow0}\frac{1}{(|k|-21!}\frac{\mathrm{d}^{|k|-1}}{\mathrm{d}z^{|k|-1}}\Bigg[\frac{z\cdot(z-1)}{z^2-\sqrt{2}z+1}\cdot\frac{z^{|k|}}{z^{|k|+1}}\Bigg ]\nonumber\\
	&=\lim\limits_{z\rightarrow0}\frac{1}{(|k|-1)!}\frac{\mathrm{d}^{|k|-1}}{\mathrm{d}z^{|k|-1}}\Bigg[\frac{z-1}{(z-(\frac{\sqrt{2}}{2}+\im\frac{\sqrt{2}}{2}))\cdot(z-(\frac{\sqrt{2}}{2}-\im\frac{\sqrt{2}}{2}))}\Bigg ]
\end{align}
%We use partial fraction decomposition to make the differentiation easier.
Wir nutzen Partialbruchzerlegung, um uns das Ableiten zu vereinfachen.
\begin{align}
	\frac{z-1}{(z-(\frac{\sqrt{2}}{2}+\im\frac{\sqrt{2}}{2}))\cdot(z-(\frac{\sqrt{2}}{2}-\im\frac{\sqrt{2}}{2}))}=\frac{A}{(z-(\frac{\sqrt{2}}{2}+\im\frac{\sqrt{2}}{2}))}+\frac{B}{(z-(\frac{\sqrt{2}}{2}-\im\frac{\sqrt{2}}{2}))}\quad\Bigg|\quad\cdot (z-\e^{+\im\frac{\pi}{4}})\cdot(z-\e^{-\im\frac{\pi}{4}} \nonumber
\end{align}
\begin{align}
	z-1&=A(z-(\frac{\sqrt{2}}{2}-\im\frac{\sqrt{2}}{2}))+B(z-(\frac{\sqrt{2}}{2}+\im\frac{\sqrt{2}}{2}))\nonumber \\
	&=z(A+B)+(-A(\frac{\sqrt{2}}{2}-\im\frac{\sqrt{2}}{2})-B(\frac{\sqrt{2}}{2}+\im\frac{\sqrt{2}}{2}))
\end{align}
%We get a system of linear equations. In matrix notation:
Wir erhalten ein lineares Gleichungssystem, hier dargestellt in der Matrixnotation:
\begin{align}
	\begin{pmatrix}
		1 & 1 \\
		-\e^{-\im\frac{\pi}{4}} & -\e^{+\im\frac{\pi}{4}}
	\end{pmatrix}
	\begin{pmatrix}
		A \\
		B
	\end{pmatrix}
	=
	\begin{pmatrix}
		1 \\
		-1
	\end{pmatrix}
\end{align}
%With techniques like Gaussian elimination or Cramer's rule, we can solute our system of linear equations and achieve
Das lineare Gleichungssystem lässt sich mit Techniken wie dem Gaußschem Eliminationsverfahren oder der Cramerschen Regel lösen und wir erhalten
\begin{align}
	A=\frac{-\e^{+\im\frac{\pi}{4}}+1}{-\sqrt{2}\im}=\frac{1}{2}+\frac{1-\sqrt{2}}{2\im}, \nonumber\\
	B=-1+\e^{-\im\frac{\pi}{4}}=\frac{1}{2}+\frac{\sqrt{2}-1}{2\im}.
\end{align}
\begin{align}
	\underset{k<0}{\mathrm{Res}(X(z)z^{k-1},0)}&=\lim\limits_{z\rightarrow0}\frac{1}{(|k|-1)!}\frac{\mathrm{d}^{|k|-1}}{\mathrm{d}z^{|k|-1}}\Bigg[\bigg(\frac{1}{2}+\frac{1-\sqrt{2}}{2\im}\bigg)\frac{1}{z-\e^{+\im\frac{\pi}{4}}}+\bigg(\frac{1}{2}+\frac{\sqrt{2}-1}{2\im}\bigg)\frac{1}{z-\e^{-\im\frac{\pi}{4}}}\Bigg ]\nonumber\\
	&=\lim\limits_{z\rightarrow0}\frac{1}{(|k|-1)!}\Bigg[\bigg( \frac{1}{2}+\frac{1-\sqrt{2}}{2\im}\bigg)\frac{(-1)^{|k|-1}\cdot(|k|-1)!}{(z-\e^{+\im\frac{\pi}{4}})^{|k|}}+\bigg(\frac{1}{2}+\frac{\sqrt{2}-1}{2\im}\bigg)\frac{(-1)^{|k|-1}\cdot(|k|-1)!}{(z-\e^{-\im\frac{\pi}{4}})^{|k|}}\Bigg] \nonumber\\
	&=\bigg(\frac{1}{2}+\frac{1-\sqrt{2}}{2\im}\bigg)\frac{(-1)^{|k|-1}}{(-1)^{|k|}\e^{+\im\frac{\pi}{4}|k|}}\bigg(\frac{1}{2}+\frac{\sqrt{2}-1}{2\im}\bigg)\frac{(-1)^{|k|-1}}{(-1)^{|k|}\e^{-\im\frac{\pi}{4}|k|}}\nonumber\\
	&=-\bigg(\frac{1}{2}+\frac{1-\sqrt{2}}{2\im}\e^{-\im\frac{\pi}{4}(|k|)}-\bigg(\frac{1}{2}+\frac{\sqrt{2}-1}{2\im}\bigg)\e^{+\im\frac{\pi}{4}(|k|)}\nonumber\\
	&=-\frac{1}{2}\bigg(\e^{+\im\frac{\pi}{4}(|k|)}+\e^{-\im\frac{\pi}{4}(|k|)}\bigg)+\frac{1-\sqrt{2}}{2\im}\bigg(\e^{+\im\frac{\pi}{4}(|k|)}-\e^{-\im\frac{\pi}{4}(|k|)}\bigg) \nonumber\\
	&=(1-\sqrt{2})\sin(\frac{\pi}{4}(|k|))-\cos(\frac{\pi}{4}(|k|))
\end{align}
\begin{align}
	x[k]=\begin{cases}
		x_1[k], &k>=0\\
		x_2[k], &k<0
	\end{cases}
\end{align}
\begin{align}
	x_1[k]=&=\mathrm{Res}(X(z)z^{k-1},\frac{\sqrt{2}}{2}+\im\frac{\sqrt{2}}{2})+\mathrm{Res}(X(z)z^{k-1},\frac{\sqrt{2}}{2}-\im\frac{\sqrt{2}}{2})\nonumber\\
	&=\frac{-\frac{\sqrt{2}}{2}+\im(1-\frac{\sqrt{2}}{2})}{\im\sqrt{2}}\e^{+\im\frac{\pi}{4}(k-1)}-\frac{-\frac{\sqrt{2}}{2}-\im(1-\frac{\sqrt{2}}{2})}{\im\sqrt{2}}\e^{-\im\frac{\pi}{4}(k-1)}\nonumber\\
	&=\frac{1-\frac{\sqrt{2}}{2}}{\sqrt{2}}\cdot\frac{2}{2}\bigg[\e^{+\im\frac{\pi}{4}(k-1)}+\e^{-\im\frac{\pi}{4}(k-1)}\bigg ]-\frac{1}{2\im}\bigg[\e^{+\im\frac{\pi}{4}(k-1)}-\e^{-\im\frac{\pi}{4}(k-1)}\bigg]\nonumber\\
	&=(\sqrt{2}-1)\cos(\frac{\pi}{4}(k-1))-\sin(\frac{\pi}{4}(k-1))
\end{align}
\begin{align}
	\underset{k<0}{x_2[k]}&=\mathrm{Res}(X(z)z^{k-1},\frac{\sqrt{2}}{2}+\im\frac{\sqrt{2}}{2})+\mathrm{Res}(X(z)z^{k-1},\frac{\sqrt{2}}{2}-\im\frac{\sqrt{2}}{2})+\mathrm{Res}(X(z)z^{k-1},0)\nonumber\\
	&=x_1[k]+(1-\sqrt{2})\sin(\frac{\pi}{4}(|k|))-\cos(\frac{\pi}{4}(|k|))\nonumber \\
	&\underset{k<0}{=}(\sqrt{2}-1)\cos(\frac{\pi}{4}(k-1))-\sin(\frac{\pi}{4}(k-1))+(1-\sqrt{2})\sin(\frac{-\pi}{4}k)-\cos(\frac{-\pi}{4}k)\nonumber\\
	&=(\sqrt{2}-1)\cos(\frac{\pi}{4}(k-1))-\sin(\frac{\pi}{4}(k-1))-(1-\sqrt{2})\sin(\frac{\pi}{4}k)-\cos(\frac{\pi}{4}k)
\end{align}
%Remember \eqref{eq:SinAlphaPlusMinusBeta} and \ref{eq:CosAlphaPlusMinusBeta}.
Wir erinnern uns an \eqref{eq:SinAlphaPlusMinusBeta} und \eqref{eq:CosAlphaPlusMinusBeta}.
\begin{align}
	\underset{k<0}{x_2[k]}&=(\sqrt{2}-1)\cos(\frac{\pi}{4}k-\frac{\pi}{4}))-\sin(\frac{\pi}{4}k-\frac{\pi}{4})-(1-\sqrt{2})\sin(\frac{\pi}{4}k)-\cos(\frac{\pi}{4}k)\nonumber\\
	&=(\sqrt{2}-1)\bigg[\frac{\sqrt{2}}{2}\cos(\frac{\pi}{4}k)+\frac{\sqrt{2}}{2}\sin(\frac{\pi}{4}k)\bigg]-\bigg[\frac{\sqrt{2}}{2}\sin(\frac{\pi}{4}k)-\frac{\sqrt{2}}{2}\cos(\frac{\pi}{4}k)\bigg]\nonumber\\
	&-(1-\sqrt{2})\sin(\frac{\pi}{4}k)-\cos(\frac{\pi}{4}k)\nonumber\\
	&=\cos(\frac{\pi}{4}k)+\sin(\frac{\pi}{4}k)-\sqrt{2}\sin(\frac{\pi}{4}k))-(1-\sqrt{2})\sin(\frac{\pi}{4}k)-\cos(\frac{\pi}{4}k)\nonumber\\
	&=\cos(\frac{\pi}{4}k)+(1-\sqrt{2})\sin(\frac{\pi}{4}k)-(1-\sqrt{2})\sin(\frac{\pi}{4}k)-\cos(\frac{\pi}{4}k)=0
\end{align}
\begin{align}
	x[k]=\begin{cases}
		(\sqrt{2}-1)\cos(\frac{\pi}{4}(k-1))-\sin(\frac{\pi}{4}(k-1))=\cos(\frac{\pi}{4}k)+(1-\sqrt{2})\sin(\frac{\pi}{4}k), &k\geq0\\
		0, &k<0
	\end{cases}
\end{align}
\subsubsection{Aufgabe h)}
Wir wollen das zeit-diskrete Signal für
\begin{align}
	X(z)=\e^{\frac{1}{z}}
\end{align}
finden. $0$ ist hier eine wesentliche Singularität, weil es eine isolierte Singularität und weder eine hebbare Singularität noch eine Polstelle ist. Wir müssen also die Laurent Reihe bilden und uns den Koeffizient $c_{-1}$ anschauen. \\
Die Exponentialfunktion lässt sich auch als Potenzreihe schreiben (vgl. \cite[Kap. 21, S. 1059]{Bronstein2015}):
\begin{align}
	\e^{z}=1+z+\frac{z^2}{2!}+\frac{z^3}{3!}+\cdot\cdot\cdot+\frac{z^{n}}{n!}+\cdot\cdot\cdot=\sum_{i=0}^{+\infty}\frac{z^i}{i!}\quad |z| < \infty.
\end{align}
Wenn wir nun $z$ durch $\frac{1}{z}$ substituieren, können wir unsere Funktion als Laurent-Reihe schreiben:
\begin{align}
	X(z)=\sum_{i=0}^{+\infty}\frac{z^{-i}}{i!}=1+z^{-1}+\frac{z^{-2}}{2!}+\frac{z^{-3}}{3!}+\cdot\cdot\cdot+\frac{z^{-n}}{n!}\quad|z|>0.
\end{align}
Das Residuum von $X(z)z^{k-1}$ finden wir nun, indem wir uns für alle $k$ den Koeffizienten $c_{-1}$ der Laurent Reihe, die um $0$ entwickelt wurde, anschauen.
\begin{align}
	X(z)z^{k-1}=\sum_{i=0}^{\infty}\frac{z^k}{z^{i+1}\cdot i!}
\end{align}
Wir betrachten zunächst den Fall, dass $k\geq0$ ist. In diesem Fall ist liegt der Koeffizient $c_{-1}$ an der Stelle, an dem $k=i$ ist, denn
\begin{align}
	\frac{z^k}{z^{i+1}}\underset{k=i}{=}\frac{1}{z}=z^{-1}.
\end{align}
Somit ist
\begin{align}
	\mathrm{Res}(X(z)z^{k-1},0)\underset{k\geq}{=}\frac{1}{k!}
\end{align}
Betrachten wir nun den Fall, dass $k<0$ ist.
\begin{align}
	X(z)z^{k-1}\underset{k<0}{=}\sum_{i=0}^{\infty}\frac{1}{z^{|k|+i+1}\cdot i!}
\end{align}
Damit in der Reihe $c_{-1}\neq 0$ ist, muss folgende Gleichung erfüllt sein:
\begin{align}
	|k|+i+1 = 1.
\end{align}
Also muss $|k| = -i$ erfüllt sein. Weil $|k|$ immer positiv ist und $k<0$, müsste in diesem Fall gelten:
\begin{align}
	k = i.
\end{align}
Damit wäre $i$ aber negativ. $i$ beginnt in unserer Reihe jedoch mit $0$. Somit ist für $k<0$ der Koeffizient $c_{-1}=0$ und somit ist auch $\mathrm{Res}(X(z)z^{k-1},0)\underset{k<0}{=}0$.
Zusammenfassung:
\begin{align}
	x[k]=\mathrm{Res}(X(z)z^{k-1},0)=\begin{cases}
		\frac{1}{k!}, &k\geq0 \\
		0, &k<0
	\end{cases}
\end{align}
\newpage
\subsection{Anhang A: Alternative Systemrepräsentationen}
\label{sec:AnhangA}
Falls die selbst berechneten Lösungen syntaktisch nicht den nachfolgenden Lösungen in \ref{sec:AnhangB} entsprechen sollten, ist das noch kein Grund, dass die Lösung falsch sein muss. Es folgt ein kleines Beispiel, um zu zeigen, dass man schnell auf syntaktisch unterschiedliche Lösungen kommen kann.
Sei
\begin{align}
	X(z) = \frac{z-1}{z-1}.
\end{align}
Ein geübter Mensch erkennt schnell, dass man durch Polynomdivision
\begin{align}
	X(z)= 1
\end{align}
erhält, was dem zeit-diskreten Signal
\begin{align}
	x[k] = \delta[k]
\end{align}
entspricht.
%
Geübter SigSys-Blick auf $X(z)$ verrät uns, dass die einzige Polstelle mit der einzigen Nullstelle kompensiert wird. Das System macht also 'nix', es muss daher als Impulsantwort das neutrale Signal der Faltung, der Dirac Impuls, rauskommen.
%

Mit Blick auf den Nenner könnte man jedoch auch versuchen, die z-Transformierte auf die Korrespondenz
\begin{align}
	\epsilon[k]\quad\ztransf\quad\frac{z}{z-1}\quad |z| > 1
\end{align}
zurückzuführen:
\begin{align}
	X(z)=\frac{z-1}{z-1}=\frac{z}{z-1}-\frac{1}{z}\cdot\frac{z}{z-1}.
\end{align}
Wenn nun noch die Korrespondenz
\begin{align}
	x[k-\kappa]\quad\ztransf\quad\frac{1}{z^{\kappa}}X(z)
\end{align}
mit einbezogen wird, erhält man
\begin{align}
	x[k]=\epsilon[k]-\epsilon[k-1].
\end{align}
Die Lösungen sehen zunächst sehr unterschiedlich aus, jedoch erkennt man bei genauerem Hinsehen, dass nur für $k=0$ das Signal $1$ ist, während sich für $k>0$ die Sprungfunktionen aufheben und für $k<0$ beide noch $0$ sind.

Wir kennen den Zusammenhang zwischen Dirac Impuls und Sprungfunktion als
\begin{align}
&\epsilon[k] = \sum_{\kappa=-\infty}^{k} \delta[\kappa]\\
&\delta[k]=\epsilon[k]-\epsilon[k-1],
\end{align}
bzw. im zeit-kontinuierlichen
\begin{align}
&\epsilon(t) = \int_{\tau=-\infty}^{t} \delta(\tau) \mathrm{d}\tau\\
&\delta(t)=\frac{\mathrm{d} \epsilon(t)}{\mathrm{d} t}.
\end{align}
%
Das können wir hier benutzen, um beide Lösungen ineinander zu überführen.


\newpage
\subsection{Anhang B: Lösungstabelle}
\label{sec:AnhangB}
\begin{center}
	\begin{tabular}{|l|l|}
		\hline
		$\mathbf{X(z)}$ & $\mathbf{x[k]}$ \\
		\hline
		$\frac{z\cdot(z-\frac{1}{\sqrt{2}})}{(z-\e^{+\im\frac{\pi}{4}})\cdot(z-\e^{-\im\frac{\pi}{4}})}$ & $2\cos[\frac{\pi}{4}k]\epsilon[k]$ \\
		\hline
		$\frac{z}{z-\frac{1}{2}}$ & $\bigg (\frac{1}{2} \bigg)^{k}\epsilon[k]$ \\
		\hline
		$\frac{z}{z-\frac{1}{2}}\cdot\frac{z}{z-1} $ & $2-\bigg(\frac{1}{2}\bigg)^k\epsilon[k] $\\[2ex]
		\hline
		$\frac{z^2-z+2}{z^2-\frac{1}{2}z+\frac{1}{4}}$ & $\delta[k]+\bigg(\frac{1}{2}\bigg)^k\Bigg[\frac{13}{\sqrt{3}}\sin[\frac{\pi}{3}(k-1)]-\cos[\frac{\pi}{3}(k-1)]\Bigg]\epsilon[k-1]$ \\
		\hline
		$\frac{z^2}{z^2+1}$ & $\cos[\frac{\pi}{2}k]\epsilon[k-1]$\\
		\hline
		$\frac{z^4+z^3-6z^2+6z-1}{z^2-2z+1}$ & $\delta[k+2]+3\delta[k+1]-\delta[k]+k\epsilon[k-1]$\\
		\hline
		$\frac{z\cdot(z-1)}{z^2-\sqrt{2}z+1}$ & $\Bigg[\cos[\frac{\pi}{4}k]+(1-\sqrt{2})\sin[\frac{\pi}{4}k]\Bigg] \epsilon[k]$\\
		\hline
		$\e^{\frac{1}{z}}$ & $\frac{1}{k!}\epsilon[k]$ \\
		\hline
	\end{tabular}
\end{center}


\subsection{Anhang C: Ausgesuchte Korrespondenzen}
\label{sec:AnhangC}

\noindent $\delta[k] \laplace 1$ für $z \in \mathbb{C}$

\noindent $\epsilon[k] \laplace \frac{z}{z-1}$  für $|z| > 1$

\noindent $a^k \, \epsilon[k] \laplace \frac{z}{z-a}$ für $|z| > |a|$

\noindent $-a^k \, \epsilon[-k-1] \laplace \frac{z}{z-a}$ für $|z| < |a|$

\noindent $k \, \epsilon[k] \laplace \frac{z}{(z-1)^2}$ für $|z| > 1$

\noindent $k \, a^k \, \epsilon[k] \laplace \frac{a \, z}{(z-a)^2}$  für $|z| > |a|$

\noindent $\sin[\Omega_0 k] \, \epsilon[k] \laplace \frac{\quad\, z\sin(\Omega_0)}{z^2 -2z \cos(\Omega_0) +1}$ für $|z| > 1$

\noindent $\cos[\Omega_0 k] \, \epsilon[k] \laplace \frac{z^2-z\cos(\Omega_0)}{z^2 -2z \cos(\Omega_0) +1}$ für $|z| > 1$


%\section*{Acknowledgement}
%Thanks to Robert Hauser (https://github.com/robhau) for adding this task.

\clearpage
\bibliography{../tutorial_latex_deu/literatur}
\end{document}

\documentclass[11pt,a4paper,DIV=12]{scrartcl}
\usepackage{scrlayer-scrpage}
\usepackage[utf8]{inputenc}
\usepackage{fouriernc}
\usepackage[T1]{fontenc}
\usepackage[english]{babel}
\usepackage[hidelinks]{hyperref}
\usepackage{natbib}
\usepackage{url}
\usepackage{amsmath}
\usepackage{amsfonts}
\usepackage{amssymb}
\usepackage{trfsigns}
\usepackage{marvosym}
\usepackage{nicefrac}
\usepackage{graphicx}
\usepackage{subcaption}
\usepackage{xcolor}
\usepackage{comment}
\usepackage{mdframed}
\usepackage{tikz}
\usepackage{circuitikz}
\usepackage{pgfplots}
\usepackage{bm}
\usepackage{cancel}
\bibliographystyle{dinat}

\usepackage{../sig_sys_macros}

%------------------------------------------------------------------------------
\ohead{Signal- und Systemtheorie Übung}
\cfoot{\pagemark}
\ofoot{\tiny\url{https://github.com/spatialaudio/signals-and-systems-exercises}}

\begin{document}
%
\noindent Signal- und Systemtheorie Übung\footnote{This tutorial is provided as
Open Educational Resource (OER), to be found at
\url{https://github.com/spatialaudio/signals-and-systems-exercises}
accompanying the OER lecture
\url{https://github.com/spatialaudio/signals-and-systems-lecture}.
%
Both are licensed under a) the Creative Commons Attribution 4.0 International
License for text and graphics and b) the MIT License for source code.
%
Please attribute material from the tutorial as \textit{Frank Schultz,
Continuous- and Discrete-Time Signals and Systems - A Tutorial Featuring
Computational Examples, University of Rostock} with
\texttt{github URL, commit number and/or version tag, year, (file name and/or
content)}.}---Frank Schultz, Sascha Spors,
Institut für Nachrichtentechnik (INT),
Fakultät für Informatik und Elektrotechnik (IEF),
Universität Rostock---Sommersemester 2023, Version: \today\\
\verb|laplace_transform_2A2B3CDD89|

Draft, WIP...

\tableofcontents

\section{Solving 1st Order, Linear, Ordinary Differential Equation with
Constant Coefficients}
\subsection{Problem Statement}

Calculate the solutions of the 1st order ODE
\begin{align}
\tau \frac{\fsd y(t)}{\fsd t} + y(t) = x(t)
\end{align}
for
\begin{itemize}
\item[a)] $x(t) = 0$, $y(0^-)=\frac{1}{\tau}$ (here we actually solve for the impulse response $y(t)=h(t)$)
\item[b)] $x(t) = \epsilon(t)$, $y(0^-)=0$ (here we solve for the step response $y(t)=h_\epsilon(t)$)
\item[c)] $x(t) = \delta(t)$, $y(0^-)=0$ (here we solve for the impulse response $y(t)=h(t)$)
\item[d)] $x(t) = \e^{s_0 t}$, $y(0^-)=y_0$, $s_0 \neq \frac{1}{\tau}$ (here we solve for the general exponential response)
\end{itemize}

We might check the brilliant textbook \cite{Strang2014} and the MIT OCW course 18.009 at \url{https://ocw.mit.edu/courses/res-18-009-learn-differential-equations-up-close-with-gilbert-strang-and-cleve-moler-fall-2015/pages/differential-equations-and-linear-algebra/}; for german readers \cite{Burg2013} might be another useful ressource.



\subsection{Solutions using Fundamental System}

As task d) is a general solution including some special cases, we should start with it.

\textbf{Task d)}
We solve
\begin{align}
\tau \frac{\fsd y(t)}{\fsd t} + y(t) = \e^{s_0 t} \qquad y(0^-) = y_0
\end{align}
We use short notation for signals and derivatives, i.e. $y(t)\rightarrow y$, $\frac{\fsd y(t)}{\fsd t} \rightarrow y'$.
%
We obtain the \underline{homogeneous solution} / null solution $y_h$, i.e. for
\begin{align}
\tau y'_h + y_h = 0
\end{align}
with the ansatz
\begin{align}
y_h = C_h \e^{\lambda t}.
\end{align}
Inserting yields
\begin{align}
\lambda \tau C_h \e^{\lambda t} + C_h \e^{\lambda t} = 0\\
(\lambda \tau +1 ) \, C_h \e^{\lambda t} = 0
\end{align}
We never can force $\e^{\lambda t} \rightarrow 0$, thus our only chance is $(\lambda \tau + 1)=0$, which leads (note that this is the zero of the characteristic polynomial)
\begin{align}
\lambda = -\frac{1}{\tau}
\end{align}
and hence
\begin{align}
y_h = C_h \e^{- \frac{t}{\tau}}
\end{align}
%
We obtain the \underline{particular solution} $y_p$ with the ansatz (we can use this because $s_0 \neq \frac{1}{\tau}$)
\begin{align}
y_p = C_p \e^{s_0 t}.
\end{align}
%
This ansatz was chosen deliberately, as the source term is precisely an exponential term $\e^{s_0 t}$ and this ODE actually only can comply with this function, as it is precisely one eigensolution of this ODE.
%
Inserting yields
\begin{align}
\tau y'_p + y_p = \e^{s_0 t}\\
\tau \frac{\fsd}{\fsd t}(C_p \e^{s_0 t}) + C_p \e^{s_0 t} = \e^{s_0 t}\\
s_0 \tau C_p \e^{s_0 t} + C_p \e^{s_0 t} = \e^{s_0 t}\\
s_0 \tau C_p + C_p = 1\\
C_p = \frac{1}{s_0 \tau + 1}
\end{align}
and hence
\begin{align}
y_p = C_p \e^{s_0 t} = \frac{1}{s_0 \tau + 1} \e^{s_0 t}.
\end{align}

The superposition of homogeneous and inhomogeneous solutions $y = y_h + y_p$ yields
\begin{align}
y = C_h \e^{- \frac{t}{\tau}} + \frac{1}{s_0 \tau + 1} \e^{s_0 t}
\end{align}
and the initial condition $y(0^-)=y_0$ solves for the still unknown $C_h$
%
\begin{align}
y_0 = C_h + \frac{1}{s_0 \tau + 1} \rightarrow C_h = y_0 - \frac{1}{s_0 \tau + 1},
\end{align}
which is inserted to find the final solution
\begin{align}
y =& (y_0 - \frac{1}{s_0 \tau + 1}) \e^{-\frac{t}{\tau}} + \frac{1}{s_0 \tau + 1} \e^{s_0 t}\\
\label{eq:taskd_final}
y =& y_0 \e^{-\frac{t}{\tau}} + \frac{\e^{s_0 t} - \e^{-\frac{t}{\tau}}}{s_0 \tau + 1} \qquad t \geq 0.
\end{align}
%
Solving \textbf{task b)}: for $y_0=0$ und $s_0 = 0$ we immediately find the well known \underline{step response} solution from the general exponential solution \eqref{eq:taskd_final}
\begin{align}
h_\epsilon(t) = y = 1 - \e^{-\frac{t}{\tau}} \qquad t \geq 0.
\end{align}
Solving \textbf{task a)}: we recall that $C_p=\frac{1}{s_0 \tau + 1}$ was the factor for the
particular solution, i.e. to apply the source term. We could also set $C_p=0$ to get $y_p=0$. Then the solution reduces to
the homogeneous solution plus initial condition. Choosing $y(0^-)=\frac{1}{\tau}$ to solve for task a) results in the \underline{impulse response}
\begin{align}
h(t) = y = \frac{1}{\tau} \e^{-\frac{t}{\tau}} + \underbrace{\frac{1}{s_0 \tau + 1}}_{C_p=0} \cdot (\e^{s_0 t} - \e^{-\frac{t}{\tau}})   \qquad t \geq 0
\end{align}
We could perform $h(t) = \frac{\fsd h_\epsilon(t)}{\fsd t}$ to find this result as well.

Solving \textbf{task c)}: this derivation is rather tricky, as the Dirac impulse needs to be treated as a distribution. We can conveniently calculate this via the Laplace transform. See \cite{Strang2014} for a solution with engineering driven mindset.


\newpage
\begin{figure}[h!]
\centering
	\begin{tikzpicture}
	\def\taun{1}
	\def\yo{1/2};
	\def\so{-1/10};
	\def\Cp{1/(\so*\taun+1)}
	\begin{axis}[
	width=0.45\textwidth,
	height=0.3\textwidth,
	domain=0:30,
	samples=50,
	legend pos=outer north east,
	xlabel = {t},
	ylabel = {$y(t)$},
	title = {$y_0=\yo,\,\tau=\taun,\,s_0=\so,\,C_p=\frac{1}{s_0 \tau + 1}$},
	xmin=0, xmax=30,
	ymin=0, ymax=1.5,
	xtick={0,5,10,15,20,25,30},
	ytick={0,0.5,1,1.5},
	ymajorgrids=true,
	xmajorgrids=true
	]
	\addplot[mark=None, color=C3, ultra thick] {\yo * exp(-x/\taun) + \Cp*exp(\so*x)  - \Cp*exp(-x/\taun)};
	\addplot[mark=None, color=C2, thick] {\yo * exp(-x/\taun)};
	\addplot[mark=None, color=C1, thick] {\Cp*exp(-x/\taun)};
	\addplot[mark=None, color=C0, thick] {\Cp*exp(\so*x)};
	\legend{$y(t)$, $y_0 \cdot \e^{-t/\tau}$, $C_p \cdot \e^{-t/\tau}$, $C_p \cdot \e^{-s_0 t}$};
	\end{axis}
	%
	\end{tikzpicture}
\caption{Task c): solution $y(t)$ \eqref{eq:taskd_final} and its contributing signals for chosen parameters.}
\end{figure}
%
\begin{figure}[h!]
\centering
	\begin{tikzpicture}
	\def\taun{1}
	\def\yo{0};
	\def\so{0};
	\def\Cp{1/(\so*\taun+1)}
	\begin{axis}[
	width=0.45\textwidth,
	height=0.3\textwidth,
	domain=0:5,
	samples=50,
	legend pos=outer north east,
	xlabel = {t},
	ylabel = {$y(t)$},
	title = {$y_0=\yo,\,\tau=\taun,\,s_0=\so,\,C_p=\frac{1}{s_0 \tau + 1}$},
	xmin=0, xmax=5,
	ymin=0, ymax=1.5,
	xtick={0,1,2,3,4,5},
	ytick={0,0.5,1,1.5},
	ymajorgrids=true,
	xmajorgrids=true
	]
	\addplot[mark=None, color=C3, ultra thick] {\yo * exp(-x/\taun) + \Cp*exp(\so*x)  - \Cp*exp(-x/\taun)};
	\addplot[mark=None, color=C2, thick] {\yo * exp(-x/\taun)};
	\addplot[mark=None, color=C1, thick] {\Cp*exp(-x/\taun)};
	\addplot[mark=None, color=C0, thick] {\Cp*exp(\so*x)};
	\legend{$y(t)$, $y_0 \cdot \e^{-t/\tau}$, $C_p \cdot \e^{-t/\tau}$, $C_p \cdot \e^{-s_0 t}$};
	\end{axis}
	%
	\end{tikzpicture}
\caption{Task b) Step response: solution $y(t)$ \eqref{eq:taskd_final} and its contributing signals for chosen parameters.}
\end{figure}
%
\begin{figure}[h!]
\centering
	\begin{tikzpicture}
	\def\taun{5}
	\def\yo{1/\taun};
	\def\so{0};
	\def\Cp{0}
	\begin{axis}[
	width=0.45\textwidth,
	height=0.3\textwidth,
	domain=0:30,
	samples=50,
	legend pos=outer north east,
	xlabel = {t},
	ylabel = {$y(t)$},
	title = {$y_0=\yo,\,\tau=\taun,\,s_0=\so,\,C_p=0$},
	xmin=0, xmax=30,
	ymin=0, ymax=0.2,
	xtick={0,5,10,15,20,25,30},
	ytick={0,0.05,0.1,0.15,0.2},
	ymajorgrids=true,
	xmajorgrids=true
	]
	\addplot[mark=None, color=C3, ultra thick] {\yo * exp(-x/\taun) + \Cp*exp(\so*x)  - \Cp*exp(-x/\taun)};
	\addplot[mark=None, color=C2, thick] {\yo * exp(-x/\taun)};
	\addplot[mark=None, color=C1, thick] {\Cp*exp(-x/\taun)};
	\addplot[mark=None, color=C0, thick] {\Cp*exp(\so*x)};
	\legend{$y(t)$, $y_0 \cdot \e^{-t/\tau}$, $C_p \cdot \e^{-t/\tau}$, $C_p \cdot \e^{-s_0 t}$};
	\end{axis}
	%
	\end{tikzpicture}
\caption{Task a) Impulse response: solution $y(t)$ \eqref{eq:taskd_final} and its contributing signals for chosen parameters.}
\end{figure}





\subsection{Solutions using Laplace Transform}
%
We use\footnote{\cite{Lundberg2007} draft at \url{https://math.mit.edu/~hrm/papers/lmt.pdf}}
\begin{align}
&\mathcal{L}_-\{y(t)\}	= \int\limits_{0^-}^{\infty} y(t) \e^{-s t} \fsd t\\
&\mathcal{L}_-\{y'(t)\} = s Y(s) - y(0^-)\\
&\lim_{s \to \infty \cdot 1} s Y(s) = y(0^+).
\end{align}
for our given problem
\begin{align}
\tau y' + y = x, \qquad y(0^-) = y_0
\end{align}
With notation $y(t) \laplace Y(s)$ the ODE's laplace transfrom reads
\begin{align}
\tau \cdot \large(s \, Y(s) - y_0\large) + Y(s) = X(s)
\end{align}
Sort terms
\begin{align}
\tau s Y(s) + Y(s) = X(s) + \tau y_0
\label{eq:laplace_ansatz}
\end{align}
such that system characteristics (w.r.t. $Y(s)$) are on the left side and source terms (w.r.t. $X(s)$) / initial conditions (w.r.t. $y_0$) are on the right side.

The transfer function $H(s)$ is defined for vanishing initial condition, i.e. setting $y_0=0$ here, as
\begin{align}
H(s) := \frac{Y(s)}{X(s)} = \frac{1}{s \tau + 1}
\end{align}
This transfer function has one pole
\begin{align}
s_\infty = -\frac{1}{\tau},
\end{align}
which is precisely the same number as the above derived zero of the characteristic polynomial $\lambda=-\frac{1}{\tau}$.
This is not by accident: the zeros of the characteristic polynomial and the poles of the transfer function equivalently describe the system inherent characteristics.

\textbf{Task c)} We know that the \underline{impulse response} and the transfer function are connected as $h(t) \laplace H(s)$, hence by inverse Laplace transform
\begin{align}
h(t) = \mathcal{L}^{-1}(\frac{1}{s \tau + 1}) = \mathcal{L}^{-1}(\frac{1}{\tau}\cdot\frac{1}{s + \frac{1}{\tau}})=
\frac{1}{\tau}\mathcal{L}^{-1}(\frac{1}{s -  (-\frac{1}{\tau})}) = \frac{1}{\tau}\e^{-\frac{t}{\tau}} \qquad t \geq 0.
\end{align}
%
We would find this also by a more step-by-step mindset:
Recall that the Dirac impulse is the neutral element of the convolution operation, i.e.
\begin{align}
h(t) \ast \delta(t) = h(t).
\end{align}
The convolution (in time domain) is connected with multiplication (in Laplace domain)
\begin{align}
y(t) = h(t) \ast x(t) \laplace Y(s) = H(s) \cdot X(s).
\end{align}
Hence, by
\begin{align}
h(t) = h(t)\ast \delta(t)	 \laplace H(s) = H(s) \cdot 1
\end{align}
we see that the Dirac impulse must have $\delta(t) \laplace 1$,
such that $H(s)$ is not changed in the Laplace domain by multiplication with another spectrum, trivially spectrum '1' is the neutral element in multiplication.
%
Therefore, task c) is actually solved with $y_0=0$ and $X(s)=1$ as input signal for eq.~\eqref{eq:laplace_ansatz}
\begin{align}
s \tau \, Y(s) + Y(s) = X(s) + \tau y_0\\
s \tau \, Y(s) + Y(s) = 1 + \tau \cdot 0
\end{align}
Simple rearrangement leads to
\begin{align}
Y(s) = \frac{1}{s \tau + 1}
\end{align}
and since we applied a Dirac impulse as input and zero initial conditions, we
calculate the impulse response by inverse Laplace transform
\begin{align}
h(t) = \mathcal{L}^{-1}(\frac{1}{s \tau + 1}),
\end{align}
the solution already known above.

\textbf{Task a)} The choices $x(t)=0$ and $y(0^-)=\frac{1}{\tau}$
do actually the very same.
We have (trivially) $x(t)=0 \laplace X(s)=0$ and thus for eq.~\eqref{eq:laplace_ansatz}
\begin{align}
s \tau  Y(s) + Y(s) = X(s) + \tau y_0\\
s \tau  Y(s) + Y(s) = 0 + \tau \cdot \frac{1}{\tau}
\longrightarrow Y(s) = \frac{1}{s \tau + 1}
\end{align}
again resulting in the impulse response. Here, this is not so obvious, as
we did not trigger the ODE's input with a Dirac impulse.
However, when comparing the
two approaches i) $x(t)=\delta(t), y(0^-)=0$ vs. ii) $x(t)=0, y(0^-)=\frac{1}{\tau}$
we see that they do the very same on the right side of eq.~\eqref{eq:laplace_ansatz}.

\textbf{Task b)} For $x(t)=\epsilon(t), y(0^-)=0$ we calculate the step response.
The Laplace transform $x(t)=\epsilon(t) \laplace X(s) = \frac{1}{s}$ is introduced to eq.~\eqref{eq:laplace_ansatz}
\begin{align}
s \tau Y(s) + Y(s) = \frac{1}{s} + \tau \cdot 0\\
Y(s) = \frac{1}{s} \cdot \frac{1}{s \tau + 1} = \frac{1}{\tau} \cdot \frac{1}{s - (-\frac{1}{\tau})} \cdot \frac{1}{s}
\end{align}
A partial fraction expansion helps here
\begin{equation}
\frac{1}{\tau} (\frac{1}{s + \frac{1}{\tau}} \cdot \frac{1}{s} ) =
\frac{1}{\tau} (\frac{A}{s + \frac{1}{\tau}} + \frac{B}{s})
\end{equation}
solving for the coefficients
\begin{equation}
  A=-B\qquad B = \tau
\end{equation}
\begin{align}
Y(s) = \frac{1}{\tau} (\frac{-\tau}{s - (-\frac{1}{\tau})} + \frac{\tau}{s})
= \frac{1}{s} - \frac{1}{s - (-\frac{1}{\tau})}
\end{align}
Both terms have simple inverse Laplace transforms, which yields the \underline{step response}
\begin{align}
h_\epsilon(t) = \mathcal{L}^{-1}(Y(s)) = 1 - \e^{-\frac{t}{\tau}}\qquad t \geq 0
\end{align}

\textbf{Task d)}
The general exponential response is calculated from \eqref{eq:laplace_ansatz}
\begin{align}
\tau s Y(s) + Y(s) = X(s) + \tau y_0
\end{align}
using the Laplace transform correspondence for $\Re(s)>\Re(s_0)$ (causal signal)
$$x(t) = \e^{s_0 t}, \quad t \geq 0 \quad\laplace\quad X(s) = \frac{1}{s-s_0}.$$
Inserting this
\begin{align}
\tau s Y(s) + Y(s) &= \frac{1}{s-s_0} + \tau y_0\\
(\tau s + 1 ) Y(s) &= \frac{1}{s-s_0} + \tau y_0\\
Y(s) &= \frac{\frac{1}{s-s_0}}{\tau s + 1 } + \frac{\tau y_0}{\tau s + 1}\\
Y(s) &= \frac{1}{s-s_0} \cdot \frac{1}{\tau s + 1} + \frac{y_0}{s - (-\frac{1}{\tau})}\\
Y(s) &= \frac{1}{\tau}\cdot\frac{1}{s-s_0} \cdot \frac{1}{s -(-\frac{1}{\tau})} + \frac{y_0}{s - (-\frac{1}{\tau})}
\end{align}
We should rearrange the term $\frac{1}{s-s_0} \cdot \frac{1}{s -(-\frac{1}{\tau})}$ by a partial fraction expansion, and for more convenient notation let us define $\frac{-1}{\tau} := a$:
%
\begin{align}
\frac{1}{s-s_0} \cdot \frac{1}{s - a} &= \frac{A}{s-s_0} + \frac{B}{s - a}\\
1 &= A (s-a) + B (s-s_0)\\
0 s^1 + 1 s^0 &= A s - A a + B s - B s_0
\end{align}
\begin{align}
0 &= A + B\\
1 &= -A a - B s_0\\
A &= -B\\
%1 = -A a + A s_0 = A (-a + s_0) \rightarrow
A &= \frac{1}{-a + s_0}
\end{align}
%
We now can insert this sum into the expression for $Y(s)$ (and re-insert $\tau$ in third line)
\begin{align}
Y(s) &= \frac{1}{\tau}
\left(
\frac{\frac{1}{-a + s_0}}{s-s_0} + \frac{-\frac{1}{-a + s_0}}{s - a}
\right) + \frac{y_0}{s - (-\frac{1}{\tau})}\\
&=
\frac{1}{\tau} (\frac{1}{-a + s_0}) (\frac{1}{s-s_0} - \frac{1}{s - a}) + \frac{y_0}{s - (-\frac{1}{\tau})}\\
&=\frac{1}{\tau} (\frac{1}{-\frac{-1}{\tau} + s_0}) (\frac{1}{s-s_0} - \frac{1}{s - \frac{-1}{\tau}}) + \frac{y_0}{s - (-\frac{1}{\tau})}\\
&= (\frac{1}{1 + \tau s_0}) (\frac{1}{s-s_0} - \frac{1}{s - (-\frac{1}{\tau})}) + \frac{y_0}{s - (-\frac{1}{\tau})}
\end{align}
We can rewrite even more extensively
\begin{align}
Y(s) = 	(\frac{1}{1 + \tau s_0}) \cdot \frac{1}{s-s_0} - (\frac{1}{1 + \tau s_0}) \cdot \frac{1}{s - (-\frac{1}{\tau})} +y_0 \cdot \frac{1}{s - (-\frac{1}{\tau})}
\end{align}
to apply linearity (each term in a sum can be independently treated) and inverse Laplace transforms of the one-pole spectra, to yield the continuous-time signal
\begin{align}
y(t) = \frac{1}{1 + \tau s_0} \cdot \e^{s_0 t} - \frac{1}{1 + \tau s_0} \cdot \e^{-\frac{t}{\tau}} + y_0 \cdot \e^{-\frac{t}{\tau}} \qquad t \geq 0,
\end{align}
or rearranged
\begin{align}
y(t) = \frac{1}{1 + \tau s_0} \left( \e^{s_0 t} - \e^{-\frac{t}{\tau}} \right) + y_0 \, \e^{-\frac{t}{\tau}} \qquad t \geq 0,
\end{align}
matching above derived solution.




\subsection{Solutions using Convolution}
TBD...












\newpage
\bibliography{../tutorial_latex_deu/literatur}
\end{document}

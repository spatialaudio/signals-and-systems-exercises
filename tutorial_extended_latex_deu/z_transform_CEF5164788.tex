\documentclass[11pt,a4paper,DIV=12]{scrartcl}
\usepackage{scrlayer-scrpage}
\usepackage[utf8]{inputenc}
\usepackage{fouriernc}
\usepackage[T1]{fontenc}
\usepackage[german]{babel}
\usepackage[hidelinks]{hyperref}
\usepackage{natbib}
\usepackage{url}
\usepackage{amsmath}
\usepackage{amsfonts}
\usepackage{amssymb}
\usepackage{trfsigns}
\usepackage{marvosym}
\usepackage{nicefrac}
\usepackage{graphicx}
\usepackage{subcaption}
\usepackage{xcolor}
\usepackage{comment}
\usepackage{mdframed}
\usepackage{tikz}
\usepackage{circuitikz}
\usepackage{pgfplots}
\usepackage{bm}
\usepackage{cancel}
\usepackage{mathtools}
\bibliographystyle{dinat}

\usepackage{../sig_sys_macros}

%------------------------------------------------------------------------------
\ohead{Signal- und Systemtheorie Übung}
\cfoot{\pagemark}
\ofoot{\tiny\url{https://github.com/spatialaudio/signals-and-systems-exercises}}

\begin{document}
%
\noindent Signal- und Systemtheorie Übung\footnote{This tutorial is provided as
Open Educational Resource (OER), to be found at
\url{https://github.com/spatialaudio/signals-and-systems-exercises}
accompanying the OER lecture
\url{https://github.com/spatialaudio/signals-and-systems-lecture}.
%
Both are licensed under a) the Creative Commons Attribution 4.0 International
License for text and graphics and b) the MIT License for source code.
%
Please attribute material from the tutorial as \textit{Frank Schultz,
Continuous- and Discrete-Time Signals and Systems - A Tutorial Featuring
Computational Examples, University of Rostock} with
\texttt{main file, github URL, commit SHA number and/or version tag, year}
.}---Frank Schultz, Sascha Spors,
Institut für Nachrichtentechnik (INT),
Fakultät für Informatik und Elektrotechnik (IEF),
Universität Rostock \&
Robert Hauser, Universität Rostock---Sommersemester 2022, Version: \today
%
%
% random hex to link tasks and jupyter notebooks:
% CEF5164788	<- i used this for this standalone file
% C75A2BFFBA
% FB93908026
% 687C91AB3B
% 79446829E3
% 802DCFC440
% 8E7F7674AE
% C7D49E1546
% F4236112D3
% C05FCB36EF
% 30D32CBA97
% 5BDC418471
% 0A04E3FD9B
% 910FE14236
% 3D29E13E26
% 1BD323D2B3
% C3AB7BCE83
% CD0DFC919B
% 2D8C8747E6
% 7F0FA6A2AF
% BD4740E5E3
% D846423FBD
% FE7335DCB0
% 1087B93852
% B7A1099DA9
% 513D86FD55
% 1AE0F6563C
% 399986BA92
% 09F4B2D1F4
% E9F56F88A6

\section*{Aufgabe: Kausale Systeme im z-Bereich (Klausuraufgabe aus dem Sommersemester 2021)}
\vspace{-0.4cm}{\tiny 41B4F6D3A2} %insert unique hex hash from above
\\
\quad
\begin{minipage}{0.5\textwidth}
	\begin{tikzpicture}[thick]
		\newcommand{\filtDelay}{$z^{-1}$}
		\tikzstyle{filtBlock} = [draw, rectangle, minimum height=2em, minimum width=2em,anchor=center]
		\tikzstyle{filtBranch}=[fill,shape=circle,minimum size=4pt,inner sep=0pt,anchor=center]
		\tikzstyle{filtSum} = [draw, circle, inner sep=1pt,node distance=0.8cm,anchor=center]
		%\begin{footnotesize}
		
		\def \bNull {$1$}
		\def \bEins {$0$}
		\def \bZwei {$-\frac{9}{16}$}
		\def \einsDurchANull {$1$}
		\def \minusAEins {$1$}
		\def \minusAZwei {$\frac{3}{4}$}
		
		\matrix (m) [matrix of nodes, row sep=1em, column sep=1em]{
			\node (input) [anchor=center]{$x[k]$}; & \node (ffsplit0) [filtBranch]{}; & \node (b0) [filtBlock]{\bNull}; & \node (join0) [filtSum]{$+$};
			& \node (a0) [filtBlock]{\einsDurchANull}; & \node (fbsplit0) [filtBranch]{}; & \node (output)[anchor=center]{$y[k]$}; \\
			& \node [filtBlock](ffdelay1) {\filtDelay}; & & & & \node (fbdelay1)[filtBlock] {\filtDelay};  \\
			&  &  & \node (join1) [filtSum]{$+$};
			& \node (a1) [filtBlock]{\minusAEins}; & \node (fbsplit1) [filtBranch]{}; \\
			& \node [filtBlock](ffdelay2) {\filtDelay}; & & & & \node (fbdelay2)[filtBlock] {\filtDelay};  \\
			& \node (ffsplit2) [coordinate]{}; & \node (b2) [filtBlock]{\bZwei}; & \node (join2) [filtSum]{$+$};
			& \node (a2) [filtBlock]{\minusAZwei}; & \node (fbsplit2) [coordinate]{};  \\
		};
		\draw [->] (input) -- (ffsplit0) -- (b0);
		\draw [->] (b0) -- (join0);
		\draw [->] (ffsplit0) -- (ffdelay1);
		\draw [->] (ffdelay1) -- (ffdelay2);
		\draw [->] (join1) -- (join0);
		\draw[->] (ffdelay2) -- (ffsplit2) -- (b2);
		\draw [->] (b2) -- (join2);
		\draw [->] (join2) -- (join1);
		\draw [->] (join0) -- (a0);
		\draw [->] (a0) -- (fbsplit0) -- (output);
		\draw [->] (fbsplit0) -- (fbdelay1);
		\draw[->] (fbdelay1) -- (fbsplit1) -- (a1);
		\draw [->] (a1) -- (join1);
		\draw [->] (fbsplit1) -- (fbdelay2);
		\draw[->] (fbdelay2) -- (fbsplit2) -- (a2);
		\draw [->] (a2) -- (join2);
		%\end{footnotesize}
	\end{tikzpicture}
\end{minipage}
%
\quad
\begin{minipage}{0.5\textwidth}
	\begin{itemize}
		\item[Geg.:] Linkes Blockschaltbild eines zeitdiskreten\\Systems ohne Anfangsbedingungen
		\item[Ges.: a)] Zeigen Sie analytisch, dass
		\begin{align*}
			h_1[k] =& \frac{3}{4} \delta[k] - \frac{5}{16}\left(-\frac{1}{2}\right)^k \epsilon[k] + \frac{9}{16}\left(\frac{3}{2}\right)^k \epsilon[k]\\
			h_2[k] =& \delta[k] + \frac{5}{32}\left(-\frac{1}{2}\right)^{k-1} \epsilon[k-1] + \frac{27}{32}\left(\frac{3}{2}\right)^{k-1} \epsilon[k-1]
		\end{align*}
		die Impulsantwort $h_1[k]=h_2[k]=h[k]$ des Systems ist.
		%
		Es reicht die Rechnung für $h_1[k]$ \underline{oder} $h_2[k]$ durchzuführen.
		%
		\item[      b)] Nennen Sie mindestens zwei Kriterien, woran wir sehen, dass das System \underline{in}stabil ist.
	\end{itemize}
\end{minipage}
\quad
\newpage
%
\section*{Lösung}
%
%
%
\subsection*{Lösung Aufgabe a) Vorwärts}
\subsubsection*{Lösung mit $\frac{1}{z}$ Trick}
Zunächst leiten wir die Differenzengleichung aus dem Blockschaltbild ab. 
Dabei nutzen wir folgende Korrespondenzen:
%
%
%
\begin{gather}
	Ax[k]+By[k]\quad\ztransf\quad AX(z)+BY(z),\nonumber\\
	x[k-\kappa]\quad\ztransf\quad z^{-\kappa}X(z).\nonumber
\end{gather}
%
%
%
Ein $z^{-1}$ steht bspw. für einen Time-Shift nach rechts, das $+$ bedeutet
eine Addition und $-\frac{9}{16}$ eine Multiplikation des diskreten Signals mit
$-\frac{9}{16}$. 
%
In dem Blockschaltbild gucken wir uns nun an, aus welchen Signalen das 
Ausgangssystem $y[k]$ gebildet wird.\\
%
Zunächst wird einmal das gewöhnliche Eingangssignal $x[k]$ einbezogen. 
%
Es spaltet sich bei $x[k]$ auch ein Signalstrang ab, der zweimal das Kästchen 
mit $z^{-1}$ passiert. Das bedeutet einer Verschiebung des diskreten Signals um 
$2$ nach rechts, bzw. anders gesagt: das Eingangssignal von vor $2$ 
Abtastpunkten hat auch eine Auswirkung auf das aktuelle Ausgangssignal. 
%
Mit der $-\frac{9}{16}$ wird $x[k-2]$ noch multipliziert und dann mit $x[k]$ 
addiert.\\
%
Von der rechten Seite, dem Ausgangssignal, gehen aber auch zwei Stränge ab.
%
Das System ist also rückgekoppelt, ältere Ausgangssignale werden auch 
berücksichtigt. 
%
Das Ausgangssignal passiert einmal ein $z^{-1}$ Kästchen und spaltet sich dann
wieder auf.
%
$y[k-1]$ wird einmal hinzuaddiert und auch noch einmal um $1$ nach rechts
verschoben, mit $\frac{3}{4}$ multipliziert und noch einmal hinzuaddiert.
%
Die Differenzengleichung lautet also:
%
%
%
\begin{gather}
	y[k] = x[k] - \frac{9}{16} x[k-2] + y[k-1] + \frac{3}{4} y[k-2]
\end{gather}
%
%
%
bzw. mit Umsortieren
%
%
%
\begin{gather}
	y[k] - y[k-1] - \frac{3}{4} y[k-2] = x[k] - \frac{9}{16} x[k-2].
\end{gather}
%
%
%
Nun transformieren wird die DGL in den $z$-Bereich:
%
%
%
\begin{gather}
	Y(z) - Y(z) z^{-1} - \frac{3}{4} Y(z) z^{-2} = X(z) - \frac{9}{16} X(z) z^{-2}.
\end{gather}
%
%
%
Die Übertragungsfunktion $H(z)=\frac{Y(z)}{X(z)}$ erreichen wir dadurch, dass 
wir $Y(z)$ und $X(z)$ ausklammern, sodass folgender Ausdruck erzielt wird:
%
\begin{gather}
	Y(z)p(z)=X(z)q(z).\\
	%
	Y(z) \underbrace{\left(1 - z^{-1} - \frac{3}{4} z^{-2}\right)}_{p(z)} 
	= X(z) \underbrace{\left(1 - \frac{9}{16} z^{-2}\right)}_{q(z)}
\end{gather}
%
%
Nun muss noch auf beiden Seiten durch $X(z)$ und $p(z)$ dividiert werden:
%
%
\begin{gather}
	H(z)=\frac{Y(z)}{X(z)}=\frac{q(z)}{p(z)}
\end{gather}
%
%
bzw.
%
%
\begin{gather}
	H(z) = \frac{Y(z)}{X(z)} 
	=\frac{1 - \frac{9}{16} z^{-2}}{1 - z^{-1} - \frac{3}{4} z^{-2}}
	= \frac{z^2 - \frac{9}{16}}{z^2 - z - \frac{3}{4}}.
\end{gather}
%
%
Nullstellen:
%
%
$z_{0,1,2} = \pm\frac{3}{4}$,
Polstellen
%
%
$z_{\infty,1} = -\frac{1}{2}\quad z_{\infty,2} = +\frac{3}{2}$,
%
Verstärkungsfaktor $g=1$
%
%
\begin{center}
	\begin{tikzpicture}[scale=1]
		\def \tic {0.075}
		\begin{scope}
			\def \rocmax{2.5}
			\filldraw[even odd rule,C2!50] (0,0) circle(1.5) decorate
			[decoration={snake, segment length=15pt, amplitude=1pt}]
			{(0,-3pt) circle(\rocmax)}; % sketch the roc domain
			%\draw[help lines, C7!50, step=0.25cm] (-\rocmax,-\rocmax) grid (\rocmax,\rocmax);
			%
			\draw[C7, thick] (0,0) circle(1);  % unit circle, i.e. DTFT domain
			\draw (1+2*\tic,-2*\tic) node{$1$}; % indicate that this is the unit circle
			\draw[->] (-1.25,0)--(2.5,0) node[right]{$\Re\{z\}$}; % axis label
			\draw[->] (0,-1.25)--(0,1.25) node[above]{$\Im\{z\}$}; % axis label
			%
			\draw (1.2*0.86602540378*\rocmax,1.2*0.5*\rocmax) node[C2!75]{KB}; % indicate the roc
			%
			% the z-transfer function specific stuff:
			\draw (1,1.2) node[black]{$g=1$}; % indicate gain factor
			%
			% draw the poles / zeros and if desired ticks:
			\draw[C0, ultra thick] (-1/2,0) node{\Huge $\times$};
			\draw[C0, ultra thick] (+3/2,0) node{\Huge $\times$};
			\draw[C0, ultra thick] (+3/4,0) node{\Huge $\circ$};
			\draw[C0, ultra thick] (-3/4,0) node{\Huge $\circ$};
			%
			\draw (-1/2,+\tic) -- (-1/2,-\tic) node[below]{$-\frac{1}{2}$};
			\draw (3/2,+\tic) -- (3/2,-\tic) node[below]{$\frac{3}{2}$};
			\draw (3/4,+\tic) -- (3/4,-\tic) node[below]{$\frac{3}{4}$};
			%
		\end{scope}
	\end{tikzpicture}
\end{center}
Nun müssen wir die z-Transformation unserer Systemfunktion wieder zurück 
transformieren, um sie mit der gegebenen Impulsantwort zu vergleichen. 
%
Wir nutzen die Partialbruchzerlegung, um einfache Korrespondenzen aus unserer
Formelsammlung zu erhalten.\\
%
Wir nutzen hier den $H(z)$/$z$-Trick. 
%
Bei diesem Trick wird zunächst die Partialbruchzerlegung aus $\frac{H(z)}{z}$
%
gebildet, denn der Grad des Nennerpolynoms ist gleich dem Grad des 
Zählerpolynoms.\\
%
Die Übertragungsfunktion umgeschrieben für die Partialbruchzerlegung lautet:
%
\begin{gather}
	H(z) = 1 \cdot \frac{\left(z - \frac{3}{4}\right)\left(z + \frac{3}{4}\right)}
	{\left(z+\frac{1}{2}\right)\left(z-\frac{3}{2}\right)}.
\end{gather}
%
Ansatz PZB
%
%
%
\begin{gather}
	\frac{H(z)}{z} = 
	1 \cdot \frac{\left(z - \frac{3}{4}\right)\left(z + \frac{3}{4}\right)}
	{z \cdot \left(z+\frac{1}{2}\right)\left(z-\frac{3}{2}\right)} 
	=1 \cdot \frac{z^2 - \frac{9}{16}}{z \cdot \left(z+\frac{1}{2}\right)\left(z-\frac{3}{2}\right)} 
	=\frac{A}{z} + \frac{B}{z+\frac{1}{2}} + \frac{C}{z-\frac{3}{2}}\\
	%
	z^2 - \frac{9}{16} 
	=A \left(z+\frac{1}{2}\right)\left(z-\frac{3}{2}\right) 
	+ B z \left(z-\frac{3}{2}\right) + C z \left(z+\frac{1}{2}\right)\\
	%
	z^2 - \frac{9}{16} 
	=A z^2 - A z - \frac{3}{4} A + B z^2 - \frac{3}{2} B z + C z^2 + \frac{1}{2} C z
\end{gather}
%
%
%
Durch den Koeffizientenvergleich ergeben sich $3$ Gleichungen für $3$ 
Unbekannte.
%
%
\begin{gather}
	z^2:\quad 1 = A+B+C\\
	%
	z^1:\quad 0 = -A -\frac{3}{2}B + \frac{1}{2}C\\
	%
	z^0:\quad -\frac{9}{16} = -\frac{3}{4}A \rightarrow \qquad A = \frac{3}{4}
\end{gather}
%
%
Wir setzen $A$ in den $z^2$-Term ein und Formen nach $C$ um.
%
%
\begin{gather}
	1 = \frac{3}{4}+B+C \rightarrow \frac{1}{4} = B + C \rightarrow C = \frac{1}{4} - B
\end{gather}
%
%
Anschließend setzen wir $A$ und das $C$ aus dem $z^2$-Term in den $z^1$-Term
ein, um eine Lösung für $B$ zu finden.
%
\begin{gather}
	0 = - \frac{3}{4} -\frac{3}{2}B + \frac{1}{2} (\frac{1}{4} - B) 
	=- \frac{3}{4} -\frac{3}{2}B + \frac{1}{8} - \frac{1}{2} B\\
	%
	\frac{3}{4} - \frac{1}{8} =  - \frac{3}{2}B - \frac{1}{2} B \rightarrow\frac{5}{8} =  -  2 B
\end{gather}
%
%
daher
%
\begin{gather}
	B = -\frac{5}{16}
\end{gather}
%
Mit $B$ lässt sich nun $C$ berechnen.
%
\begin{gather}
	C = \frac{1}{4} - B = \frac{4}{16} + \frac{5}{16} = \frac{9}{16}
\end{gather}
%
Alternativ können wir das LGS in Matrixform aufschreiben und mit Hilfe der 
Cramerschen Regel lösen.
\begin{gather}
	\begin{pmatrix}
		1 & 1 & 1 \\
		-1 & -\nicefrac{3}{2} & \nicefrac{1}{2} \\
		-\nicefrac{3}{4} & 0 & 0
	\end{pmatrix}
	%
	\begin{pmatrix}
		A \\
		B \\
		C
	\end{pmatrix}
	=
	\begin{pmatrix}
		1 \\
		0 \\
		-\nicefrac{9}{16}
	\end{pmatrix}
\end{gather}
%
%
%
\begin{align}
	A=
	\frac{
		\begin{vmatrix}
			1 & 1 & 1 \\
			0 & -\nicefrac{3}{2} & \nicefrac{1}{2} \\
			-\nicefrac{9}{16} &0& 0
		\end{vmatrix}
	}
	{
		\begin{vmatrix}
			1 & 1 & 1 \\
			-1 & -\nicefrac{3}{2} & \nicefrac{1}{2} \\
			-\nicefrac{3}{4} & 0 & 0
		\end{vmatrix}
	}
	%Zähler
	&=\frac{1\cdot \left(-\frac{3}{2}\right) \cdot 0 + 1\cdot \frac{1}{2} \cdot\left(-\frac{9}{16}\right) +1\cdot 0 \cdot 0- \left(-\frac{9}{16}\right)\cdot \left(-\frac{3}{2}\right)\cdot 1-0\cdot \frac{1}{2}\cdot 1 - 0 \cdot 0 \cdot 1}
	%Nenner
	{1\cdot \left(-\frac{3}{2}\right)\cdot 0 +1\cdot \frac{1}{2}\cdot \left(-\frac{3}{4}\right)+1\cdot \left(-1\right) \cdot 0- \left(-\frac{3}{4}\right)\cdot \left(-\frac{3}{2}\right)\cdot 1 -0\cdot \frac{1}{2}\cdot 1 + 0\cdot \left(-1\right) \cdot 1}\nonumber \\
	%
	&=\frac{-\frac{4}{2}\cdot\frac{9}{16}}{-\frac{4}{2}\cdot\frac{3}{4}}=\frac{9}{16}\cdot\frac{4}{3}=\frac{3}{4}
\end{align}
%
%
%
\begin{align}
	B=
	\frac{
		\begin{vmatrix}
			1 & 1 & 1 \\
			-1 & 0 & \nicefrac{1}{2} \\
			-\nicefrac{3}{4} & -\nicefrac{9}{16}&0
		\end{vmatrix}
	}
	{
		\begin{vmatrix}
			1 & 1 & 1 \\
			-1 & -\nicefrac{3}{2} & \nicefrac{1}{2} \\
			-\nicefrac{3}{4} & 0 & 0
		\end{vmatrix}
	}
	&=
	%Zähler
	\frac{1\cdot 0 \cdot 0+ 1 \cdot \frac{1}{2}\cdot \left(-\frac{3}{4}\right)+1\cdot \left(-1\right)\cdot \left(-\frac{9}{16}\right)-\left(-\frac{3}{4}\right)\cdot 0 \cdot 1 -\left(-\frac{9}{16}\right)\cdot \frac{1}{2}\cdot 1 - 0\cdot \left(-1\right)\cdot 1}
	%Nenner
	{1\cdot \left(-\frac{3}{2}\right)\cdot 0 +1\cdot \frac{1}{2}\cdot \left(-\frac{3}{4}\right)+1\cdot \left(-1\right) \cdot 0- \left(-\frac{3}{4}\right)\cdot \left(-\frac{3}{2}\right)\cdot 1 -0\cdot \frac{1}{2}\cdot 1 + 0\cdot \left(-1\right) \cdot 1}\nonumber \\
	%
	&=\frac{-\frac{3}{8}+\frac{27}{32}}{-\frac{4}{2}\cdot\frac{3}{4}}=\frac{\frac{-12+27}{32}}{-\frac{3}{2}}=-\frac{15}{32}\cdot\frac{2}{3}=-\frac{5}{16}
\end{align}
\begin{align}
	C=
	\frac{
		\begin{vmatrix}
			1 & 1 & 1 \\
			-1 & -\nicefrac{3}{2} & 0 \\
			-\nicefrac{3}{4}& 0 & -\nicefrac{9}{16}
		\end{vmatrix}
	}
	{
		\begin{vmatrix}
			1 & 1 & 1 \\
			-1 & -\nicefrac{3}{2} & \nicefrac{1}{2} \\
			-\nicefrac{3}{4} & 0 & 0
		\end{vmatrix}
	}
	&=
	%Zähler
	\frac{1\cdot\left(-\frac{3}{2}\right)\cdot\left(-\frac{9}{16}\right)+1\cdot 0 \cdot \left(-\frac{3}{4}\right)+1\cdot \left(-1\right)\cdot 0-\left(-\frac{3}{4}\right)\cdot \left(-\frac{3}{2}\right)\cdot 1 -0 \cdot 0 \cdot 1 - \left(-\frac{9}{16}\right)\cdot \left(-1\right)\cdot 1}
	%Nenner
	{1\cdot \left(-\frac{3}{2}\right)\cdot 0 +1\cdot \frac{1}{2}\cdot \left(-\frac{3}{4}\right)+1\cdot \left(-1\right) \cdot 0- \left(-\frac{3}{4}\right)\cdot \left(-\frac{3}{2}\right)\cdot 1 -0\cdot \frac{1}{2}\cdot 1 + 0\cdot \left(-1\right) \cdot 1}\nonumber \\
	%
	&=
	\frac{\frac{9}{32}-\frac{9}{8}}{-\frac{4}{2}\cdot\frac{3}{4}}=\frac{\frac{9-36}{32}}{-\frac{3}{2}}=\frac{27}{32}\cdot \frac{2}{3}=\frac{9}{16}
\end{align}
%
%
%
Auch mit dem Gauß-Algorithmus lässt sich das LGS lösen.
%
%
%
\begin{gather}
	\left\lparen
	\begin{matrix}
		1 & 1 & 1 \\
		-1 & -\nicefrac{3}{2} & \nicefrac{1}{2} \\
		-\nicefrac{3}{4} & 0 & 0
	\end{matrix}
	\middle\vert
	\begin{matrix}
		1 \\
		0 \\
		-\nicefrac{9}{16}
	\end{matrix}
	\right\rparen
\end{gather}
%
%
%
Zuerst werden die 3. und die 1. Spalte vertauscht.
%
%
%
\begin{gather}
	\left\lparen
	\begin{matrix}
		-\nicefrac{3}{4} & 0 & 0 \\
		-1 & -\nicefrac{3}{2} & \nicefrac{1}{2} \\
		1 & 1 & 1
	\end{matrix}
	\middle\vert
	\begin{matrix}
		-\nicefrac{9}{16} \\
		0 \\
		1
	\end{matrix}
	\right\rparen
\end{gather}
%
%
%
Nun ziehen wir die 1. Spalte $\frac{4}{3}$-mal von der 2. ab

$\left(\mathrm{II}:=\mathrm{II}-\frac{4}{3}\mathrm{I}\right)$ und $\mathrm{III}:=\mathrm{III}+\frac{4}{3}\mathrm{I}$.
%
%
%
\begin{gather}
	\left\lparen
	\begin{matrix}
		-\nicefrac{3}{4} & 0 & 0 \\
		0 & -\nicefrac{3}{2} & \nicefrac{1}{2} \\
		0 & 1 & 1
	\end{matrix}
	\middle\vert
	\begin{matrix}
		-\nicefrac{9}{16} \\
		\nicefrac{3}{4} \\
		\nicefrac{1}{4}
	\end{matrix}
	\right\rparen
\end{gather}
%
%
%
Anschließend rechnen wir $\mathrm{III}:=\mathrm{III}+\frac{2}{3}\mathrm{II}$
%
%
%
\begin{gather}
	\left\lparen
	\begin{matrix}
		-\nicefrac{3}{4} & 0 & 0 \\
		0 & -\nicefrac{3}{2} & \nicefrac{1}{2} \\
		0 & 0 & \nicefrac{4}{3}
	\end{matrix}
	\middle\vert
	\begin{matrix}
		-\nicefrac{9}{16} \\
		\nicefrac{3}{4} \\
		\nicefrac{3}{4}
	\end{matrix}
	\right\rparen
\end{gather}
%
%
%
Nun folgt $\mathrm{II}:=\mathrm{II}-\frac{3}{8}\mathrm{III}$
%
%
%
\begin{gather}
	\left\lparen
	\begin{matrix}
		-\nicefrac{3}{4} & 0 & 0 \\
		0 & -\nicefrac{3}{2} & 0 \\
		0 & 0 & \nicefrac{4}{3}
	\end{matrix}
	\middle\vert
	\begin{matrix}
		-\nicefrac{9}{16} \\
		\nicefrac{15}{32} \\
		\nicefrac{3}{4}
	\end{matrix}
	\right\rparen
\end{gather}
%
%
%
Wir teilen zum Schluss jede Zeile noch durch den Faktor, der auf 
der linken Seite steht
%
%
%
\begin{gather}
	\left\lparen
	\begin{matrix}
		1 & 0 & 0 \\
		0 & 1 & 0 \\
		0 & 0 & 1
	\end{matrix}
	\middle\vert
	\begin{matrix}
		\nicefrac{3}{4} \\
		-\nicefrac{5}{16} \\
		\nicefrac{9}{16}
	\end{matrix}
	\right\rparen
\end{gather}
%
%
%
und können nun die Lösungen aus dem rechten Spaltenvektor auslesen:
%
\begin{gather}
	A=\frac{3}{4},\\
	B=-\frac{5}{16},\\
	C=\frac{9}{16}.
\end{gather}
%
Wir schreiben uns die Systemfunktion noch einmal auf. 
%
Dabei darf nicht vergessen werden, dass sie durch $z$ dividiert wurde.
%
\begin{gather}
	\frac{H(z)}{z}=
	\frac{\frac{3}{4}}{z} - \frac{\frac{5}{16}}{z+\frac{1}{2}} + \frac{\frac{9}{16}}{z-\frac{3}{2}}\\
	%
	H(z) =\frac{3}{4} - \frac{\frac{5}{16} z}{z+\frac{1}{2}} + \frac{\frac{9}{16} z}{z-\frac{3}{2}}
\end{gather}
%
Durch die Korrespondenzen
%
\begin{gather}
	Ax[k]+B[k]\quad\ztransf\quad AX(z)+BY(z),\\
	%
	a^k\epsilon[k]\quad\ztransf\quad \frac{z}{z-a},
\end{gather}
%
und
%
\begin{gather}
	1\quad\ztransf\quad \delta[k]
\end{gather}
%
erhalten wir die Impulsantwort
%
\begin{gather}
	h[k] = 
	\frac{3}{4} \delta[k] - \frac{5}{16}\left(-\frac{1}{2}\right)^k \epsilon[k] + \frac{9}{16}\left(\frac{3}{2}\right)^k \epsilon[k].
\end{gather}
%
\newpage
\subsubsection*{Lösung mit Polynomdivision und $\frac{z}{z}$-Erweiterung}
%
Alternativ kann man auch zunächst eine Polynomdivision durchführen, 
um den Grad des Zählerpolynoms kleiner als den Grad des Nennerpolynoms 
zu kriegen.
%
\begin{gather}
	H(z)=
	%
	\frac{1-\frac{9}{16}z^{-2}}{1-z^{-1}-\frac{3}{4}z^{-2}}\cdot\frac{z^2}{z^2}
	%
	=\frac{z^2-\frac{9}{16}}{z^2-z-\frac{3}{4}}
	%
	\underset{\text{Polynomdivision}}{=}1+\frac{z+\frac{3}{16}}{z^2-z-\frac{3}{4}}
\end{gather}
%
Nun können wir eine Partialbruchzerlegung durchführen.
%
\begin{gather}
	\frac{z+\frac{3}{16}}{z^2-z-\frac{3}{4}}=\frac{A}{z+\frac{1}{2}}+\frac{B}{z-\frac{3}{2}}\\
	%
	%
	z+\frac{3}{16}=A\left(z-\frac{3}{2}\right)+B\left(z+\frac{1}{2}\right)=z\left(A+B\right)+\left(-\frac{3}{2}A+\frac{1}{2}B\right)
\end{gather}
%
Durch den Koeffizientenvergleich können folgende Formeln aufgestellt werden:
%
\begin{gather}
	z^1: A+B=1\\
	z^0: -\frac{3}{2}A+\frac{1}{2}B=\frac{3}{16}
\end{gather}
%
Aus dem $z^1$ Term folgt
%
\begin{gather}
	A=1-B.
\end{gather}
%
Das setzen wir nun in den $z^0$ Term ein:
%
\begin{gather}
	-\frac{3}{2}(1-B)+\frac{1}{2}B=-\frac{3}{2}+\frac{3}{2}B+\frac{1}{2}B=-\frac{3}{2}+2B=\frac{3}{16}.
\end{gather}
%
Also
%
\begin{gather}
	2B=\frac{3}{16}+\frac{3}{2}=\frac{3+24}{16}=\frac{27}{16}
\end{gather}
%
sowie
%
\begin{gather}
	B=\frac{27}{16}\cdot\frac{1}{2}=\frac{27}{32}.
\end{gather}
%
Daraus folgt auch
%
\begin{gather}
	A=1-B=1-\frac{27}{32}=\frac{5}{32}.
\end{gather}
%
Das LGS lässt sich auch in Matrixform darstellen:
%
\begin{gather}
	\begin{pmatrix}
		1 & 1\\
		-\nicefrac{3}{2} & \nicefrac{1}{2}
	\end{pmatrix}
	\begin{pmatrix}
		A \\
		B
	\end{pmatrix}
	=
	\begin{pmatrix}
		1 \\
		\nicefrac{3}{16}
	\end{pmatrix}
\end{gather}
%
Mit der Cramerschen Regel ist die Lösung schnell gefunden.
%
\begin{align}
	A = 
	\frac{
		\begin{vmatrix}
			1 & 1\\
			\nicefrac{3}{16} & \nicefrac{1}{2}
		\end{vmatrix}
	}
	{
		\begin{vmatrix}
			1 & 1\\
			-\nicefrac{3}{2} & \nicefrac{1}{2}
	\end{vmatrix}
	}
	=\frac{\frac{1}{2}-\frac{3}{16}}{\frac{1}{2}+\frac{3}{2}}=\frac{\frac{8-3}{16}}{2}=\frac{5}{32}\\
	B=
	\frac{
		\begin{vmatrix}
			1 &  1\\
			-\nicefrac{3}{2} & \nicefrac{3}{16}
		\end{vmatrix}
	}
	{
		\begin{vmatrix}
			1 & 1\\
			-\nicefrac{3}{2} & \nicefrac{1}{2}
	\end{vmatrix}
	}
	=\frac{\frac{3}{16}+\frac{3}{2}}{\frac{1}{2}+\frac{3}{2}}=\frac{\frac{3+24}{16}}{2}=\frac{27}{32}
\end{align}
%
%
%
Auch der Gauß-Algorithmus führt zur Lösung.
%
%
\begin{gather}
	\left\lparen
	\begin{matrix}
		1 & 1 \\
		-\nicefrac{3}{2}&\nicefrac{1}{2}
	\end{matrix}
	\middle\vert
	\begin{matrix}
		1 \\
		\nicefrac{3}{16}
	\end{matrix}
	\right\rparen
\end{gather}
%
%
Zuerst rechnen wir $\mathrm{II}:=\mathrm{II}+\frac{3}{2}\mathrm{I}$.
%
%
\begin{gather}
	\left\lparen
	\begin{matrix}
		1 & 1 \\
		0&2
	\end{matrix}
	\middle\vert
	\begin{matrix}
		1 \\
		\nicefrac{27}{16}
	\end{matrix}
	\right\rparen
\end{gather}
%
%
Nun folgt $\mathrm{I}:=\mathrm{I}-\frac{1}{2}\mathrm{II}$.
%
%
\begin{gather}
	\left\lparen
	\begin{matrix}
		1 & 0 \\
		0&2
	\end{matrix}
	\middle\vert
	\begin{matrix}
		\nicefrac{5}{32} \\
		\nicefrac{27}{16}
	\end{matrix}
	\right\rparen
\end{gather}
%
%
%
Die zweite Zeile muss noch durch $2$ dividiert werden
%
%
%
\begin{gather}
	\left\lparen
	\begin{matrix}
		1 & 0 \\
		0&1
	\end{matrix}
	\middle\vert
	\begin{matrix}
		\nicefrac{5}{32} \\
		\nicefrac{27}{32}
	\end{matrix}
	\right\rparen
\end{gather}
%
%
und wir können die Lösungen aus dem rechten Spaltenvektor ablesen:
%
%
\begin{gather}
	A=\frac{5}{32},\\
	B=\frac{27}{32}.
\end{gather}
%
Die Systemfunktion lautet also
%
\begin{gather}
	H(z)=1+\frac{5}{32}\cdot\frac{1}{z+\frac{1}{2}}+\frac{27}{32}\cdot\frac{1}{z-\frac{3}{2}}.
\end{gather}
%
Wir nutzen folgende Korrespondenzen zum rücktransformieren:
%
\begin{gather}
	Ax[k]+By[k]\quad \ztransf\quad AX(z)+BY(z),\\
	%
	\label{eq:CorrespondenceExponentialFunction}
	a^k\epsilon[k]\quad\ztransf\quad \frac{z}{z-a},\\
	%
	\delta[k]\quad\ztransf\quad 1,\\
	%
	x[k-\kappa]\quad\ztransf\quad z^{-\kappa}X(z).
\end{gather}
%
Wir erweitern die Brüche mit $\frac{z}{z}$, um die Korrespondenz aus \eqref{eq:CorrespondenceExponentialFunction} zu erhalten.
%
%
\begin{gather}
	H(z)=1+ \frac{1}{z}\cdot\frac{5}{32}\cdot\frac{z}{z+\frac{1}{2}}+\frac{1}{z}\cdot\frac{27}{32}\cdot\frac{z}{z-\frac{3}{2}}\nonumber\\
	%
	\Ztransf\nonumber\\
	%
	h_2[k]=\delta[k]+\frac{5}{32}\bigg (-\frac{1}{2}\bigg )^{k-1}\epsilon[k-1]+\frac{27}{32}\bigg (\frac{3}{2}\bigg )^{k-1}\epsilon[k-1]\nonumber
\end{gather}
%
%
\newpage
\subsubsection*{Lösen der linearen Differenzengleichung}
Wir haben zu Beginn folgende Rekursionsgleichung aufgestellt:
%
\begin{gather}
	y[k]-y[k-1]-\frac{3}{4}y[k-2]=x[k]-\frac{9}{16}x[k-2].
\end{gather}
%
Durch das Faltungstheorem der z-Transformation
%
\begin{gather}
	x[k]\ast h[k]\quad\ztransf\quad X(z)\cdot H(z)
\end{gather}
%
sowie der Gleichung
%
\begin{gather}
	y[k]=x[k]\ast h[k]=\mathcal{Z}^{-1}\left[\mathcal{Z}\left[x[k]\right]\cdot\mathcal{Z}\left[h[k]\right]\right]
\end{gather}
%
ist $y[k]=h[k]$, wenn $x[k]=\delta[k]$, denn $\delta[k]\quad\ztransf\quad 1$.
%
Äquivalent zu den kontinuierlichen Systemen müssen wir also nur einen 
Impuls/eine Anregung zum Zeitpunkt $t=0$ (denn $k=\frac{t}{T}=0$) haben, 
um am Ausgang die Impulsantwort des diskreten Systems zu beobachten. 
%
Weil es sich um ein kausales System handelt, ist $y[k]\underset{k<0}{=}0$. 
%
Somit können wir für das Lösen der Differenzengleichung auch die Technik zum 
Lösen von Differenzengleichungen für Zahlenfolgen 
$y:\mathbb{N}\cup\{0\}\rightarrow\mathbb{R}$ anwenden.
%
Die nun folgende angewandte Technik zum Lösen von Differenzengleichungen stammt 
aus \cite[Kap. 5]{Witt2013}.\\
%
Eine lineare Differenzengleichung $n$-ter Ordnung hat die Form
%
\begin{gather}
	\sum_{i=0}^na_iy[k+i]=g[k]\quad \text{vgl. \cite[Kap. 5 S. 80]{Witt2013}}.
\end{gather}
%
Bei uns ist es die zweite Ordnung, denn durch einen Time-Shift hat unsere 
Differenzengleichung die Form 
$$y[k+2]-y[k+1]-\frac{3}{4}y[k]=x[k+2]-\frac{9}{16}x[k].$$
%
Die rechte Seite der Differenzengleichung ist ungleich $0$, 
somit handelt es sich um eine inhomogene Differenzengleichung. 
%
Des weiteren ergibt sich die allgemeine Lösung der inhomogenen 
Differenzengleichung aus der Addition der homogenen Lösung der 
Differenzengleichung sowie einer speziellen Lösung der inhomogenen 
Differenzengleichung (vgl. \cite[Kap. 5 S. 83]{Witt2013}).\\
%
Das Lösungsverfahren für homogene Differenzen ist ausführlich 
in \cite[Kap. 5 S. 84-85]{Witt2013} beschrieben.\\
%
Zuerst stellen wir die charakteristische Gleichung auf. 
%
Dies erfolgt äquivalent zur Differentialgleichung.\\
%
Wenn die homogene Differenzengleichung die Form
%
\begin{gather}
	a_ny[k+n]+a_{n-1}y[k+n-1]+\dots+a_1y[k+1]+a_0y[k]=0
\end{gather}
%
besitzt, dann lautet die charakteristische Gleichung
%
\begin{gather}
	a_n\lambda^n+a_{n-1}\lambda^{n-1}+\dots+a_1\lambda+a_0=0.
\end{gather}
%
Diese muss dann nach $\lambda$ gelöst werden. 
%
Wenn die charakteristische Gleichung $n$ paarweise verschiedene Nullstellen hat, 
dann lautet die homogene Lösung der Differenzengleichung
%
\begin{gather}
	y_h[k]=C_1\lambda_1^k+C_2\lambda_2^k+\dots+C_n\lambda_n^k,
\end{gather}
%
wobei $C_1$, $C_2$, $\dots$, $C_n$ frei wählbare Parameter sind 
(vgl. \cite[Kap. 5, S. 85]{Witt2013}).
%
Wenn nun die Nullstelle $\lambda_0$ $m$-fach vorkommt, 
dann sind folgende Terme Lösungen der homogenen Differenzengleichung:
%
\begin{gather}
	\lambda_0^k,k\lambda_0^k,k^2\lambda_0^k,\dots,k^{m-1}\lambda_0^k\quad\text{vgl. \cite[Kap. 5, S. 91]{Witt2013}}.
\end{gather}
%
\textbf{Beispiel}\\
%
Nehmen wir an, wir haben eine charakteristische Gleichung aufgestellt und 
die Nullstellen berechnet. 
%
Diese lauten
%
\begin{gather}
	\lambda_1 = \frac{1}{2},\nonumber\\
	\lambda_2 = \frac{1}{2},\nonumber\\
	\lambda_3 = \frac{1}{2},\nonumber\\
	\lambda_4 = \frac{2}{3},\nonumber\\
	\lambda_5 = \frac{2}{3},\nonumber\\
	\lambda_6 = \frac{7}{9}.\nonumber
\end{gather}
%
Dann lautet unsere Lösung
%
\begin{gather}
	y_h[k]=C_1\left(\frac{1}{2}\right)^k+C_2k\left(\frac{1}{2}\right)^k+C_3k^2\left(\frac{1}{2}\right)^k+C_4\left(\frac{2}{3}\right)^k+C_5k\left(\frac{2}{3}\right)^k+C_6\left(\frac{7}{9}\right)^k.
\end{gather}
%
Unsere charakteristische Gleichung lautet
%
\begin{gather}
	\lambda^2-\lambda-\frac{3}{4}=0.
\end{gather}
%
Mit der $p$-$q$-Formel aus Schulzeiten ist die Lösung schnell gefunden.
%
\begin{gather}
	\lambda_{1,2}=\frac{1}{2}\pm \sqrt{\frac{1}{4}+\frac{3}{4}}=\frac{1}{2}\pm 1\\
	\lambda_1=\frac{3}{2}\\
	\lambda_2=-\frac{1}{2}
\end{gather}
%
Die allgemeine Lösung der homogenen Differenzengleichung lautet also
%
\begin{gather}
	y[k]=C_1\left(\frac{3}{2}\right)^k+C_2\left(-\frac{1}{2}\right)^k.
\end{gather}
%
Nun müssen wir eine spezielle Lösung der inhomogenen Differenzengleichung 
finden. 
%
Dabei hilft folgender Satz (vgl. \cite[Kap. 5 S. 83]{Witt2013}):\\
%
Ist $y_1$ eine Lösung der Differenzengleichung $\sum_{i=0}^na_iy[k+i]=x_1[k]$ 
und $y_2$ eine Lösung der Differenzengleichung $\sum_{i=0}^na_iy[k+i]=x_2[k]$, 
dann ist $y_1+y_2$ eine Lösung der Differenzengleichung 
$\sum_{i=0}^{n}a_iy[k+i]=x_1[k]+x_2[k]$.\\
%
Wir suchen zunächst eine Lösung für
%
\begin{gather}
	y[k+2]-y[k+1]-\frac{3}{4}y[k]=x[k+2]=\delta[k+2].
\end{gather}
%
Für $k<0$ ist $y[k]=0$, da es sich um kausale Systeme/Signale handelt. 
%
Ab $k=0$ ist die rechte Seite immer $0$, denn der verschobene Dirac Impuls wäre 
nur bei $k=-2$ $1$. Wir müssen tatsächlich also nur eine spezielle Lösung für 
eine homogene Differenzengleichung finden. 
%
Wir wählen die triviale Lösung,
%
\begin{gather}
	y_1[k]=0.
\end{gather}
%
Die zweite inhomogene Differenzengleichung lautet
%
\begin{gather}
	y[k+2]-y[k+1]-\frac{3}{4}y[k]=-\frac{9}{16}x[k]=-\frac{9}{16}\delta[k].
\end{gather}
%
Für $k<0$ ist $y[k]$ wieder $0$ und für $k>0$ ist die rechte Seite $0$. 
%
Somit wäre für $k>0$ auch wieder $y_2[k]$ eine spezielle Lösung. 
%
Für $k=0$ ist die rechte Seite jedoch $-\frac{9}{16}$. 
%
Wir nehmen also an, dass eine spezielle Lösung von $y_2[k]$ ein 
Dirac-Impuls ist.
%
\begin{gather}
	y_2[k]=a\delta[k]
\end{gather}
%
Diesen Ansatz setzen wir nun in die inhomogene Differenzengleichung ein.
%
\begin{gather}
	y[k+2]-y[k+1]-\frac{3}{4}y[k]=a\delta[k+2]-a\delta[k+1]-a\frac{3}{4}\delta[k]=-\frac{9}{16}\delta[k]
\end{gather}
%
Wir wissen, dass $y_2[k<0]=y_2[k>0]=0$ ist. Wir untersuchen die 
Differenzengleichung also für $k=0$.
%
%
\begin{gather}
	y[2]-y[1]-\frac{3}{4}y[k]=\delta[2]-\delta[1]-a\frac{3}{4}\delta[0]=-a\frac{3}{4}=-\frac{9}{16}\delta[0]=-\frac{9}{16}\\
	%
	a = \frac{3}{4}\\
	%
	y_2[k]=\frac{3}{4}\delta[k]
\end{gather}
%
%
Die allgemeine Lösung der inhomogenen Differenzengleichung lautet
%
\begin{gather}
	y_a[k]=y_h[k]+y_s[k]=y_h[k]+y_1[k]+y_2[k]=
	\frac{3}{4}\delta[k]+C_1\left(\frac{3}{2}\right)^k+C_2\left(-\frac{1}{2}\right)^k.
\end{gather}
%
Wir betrachten nun noch einmal die ursprüngliche Differenzengleichung 
(also wieder Timeshift um 2 nach rechts):
%
\begin{gather}
	y[k]-y[k-1]-\frac{3}{4}y[k-2]=x[k]-\frac{9}{16}x[k-2].
\end{gather}
%
Wir wissen wegen Kausalität, dass $y[k<0]=0$ gilt. 
%
Außerdem ist $x[k]=\delta[k]$. Somit lassen sich Anfangsbedingungen finden.\\
%
$k=0$:
%
\begin{gather}
	y[0]-\underbrace{y[-1]}_{=0}-\frac{3}{4}\underbrace{y[-2]}_{=0}=\delta[0]-\frac{9}{16}\delta[-2]=1
\end{gather}
%
$k=1$:
%
\begin{gather}
	y[1]-\underbrace{y[0]}_{=1}-\frac{3}{4}\underbrace{y[-1]}_{=0}=\delta[1]-\frac{9}{16}\delta[-1]=0
\end{gather}
%
Zusammengefasst:
%
\begin{gather}
	y[0]=1,\\
	y[1]=1.
\end{gather}
%
Wir setzen nun $k=0$ und $k=1$ in die allgemeine Lösung der inhomogenen 
Differenzengleichung ein und erhalten dann ein LGS, um die Koeffizienten 
auszurechnen.
%
%
\begin{gather}
	y[k=0]=1=\frac{3}{4}\delta[0]+C_1\left(\frac{3}{2}\right)^0+C_2\left(-\frac{1}{2}\right)^0=\frac{3}{4}+C_1+C_2\\
	%
	\Longleftrightarrow\nonumber \\
	%
	C_1+C_2=\frac{1}{4}\\
	%
	y[k=1]=1=\frac{3}{4}\delta[1]+C_1\left(\frac{3}{2}\right)^1+C_2\left(-\frac{1}{2}\right)^1=\frac{3}{2}C_1-\frac{1}{2}C_2\\
	%
	\Longleftrightarrow\nonumber\\
	%
	\frac{3}{2}C_1-\frac{1}{2}C_2=1
\end{gather}
%
Aus der ersten Gleichung folgt 
%
\begin{gather}
	C_1=\frac{1}{4}-C_2.
\end{gather}
%
Dies kann in die zweite Gleichung eingesetzt werden:
%
%
\begin{gather}
	\frac{3}{2}(\frac{1}{4}-C_2)-\frac{1}{2}C_2=\frac{3}{8}-\frac{3}{2}C_2-\frac{1}{2}C_2=\frac{3}{8}-2C_2=1\\
	%
	\Longleftrightarrow\\
	%
	-2C=1-\frac{3}{8}=\frac{8-3}{8}=\frac{5}{8}\\
	%
	\Longleftrightarrow\nonumber\\
	%
	C_2=-\frac{5}{16}
\end{gather}
%
%
Das nun in die erste Gleichung einsetzen:
%
\begin{gather}
	C_1-\frac{5}{16}=\frac{1}{4}\\
	%
	\Longleftrightarrow\nonumber\\
	%
	C_1=\frac{1}{4}+\frac{5}{16}=\frac{4+5}{16}=\frac{9}{16}
\end{gather}
%
Das LGS lässt sich auch in Matrixform aufschreiben,
%
%
\begin{gather}
	\begin{pmatrix}
		1 & 1 \\
		\nicefrac{3}{2} & -\nicefrac{1}{2}
	\end{pmatrix}
	\begin{pmatrix}
		C_1 \\
		C_2
	\end{pmatrix}
	=
	\begin{pmatrix}
		\nicefrac{1}{4} \\
		1
	\end{pmatrix}
\end{gather}
%
%
, und mit der Cramerschen Regel bzw. dem Gauß-Algorithmus lösen.\\
%
Cramersche Regel:
%
\begin{gather}
	C_1=
	\frac{
		\begin{vmatrix}
			\nicefrac{1}{4} & 1\\
			1 & -\nicefrac{1}{2}
		\end{vmatrix}
	}
	{
		\begin{vmatrix}
			1 & 1 \\
			\nicefrac{3}{2} & -\frac{1}{2}
		\end{vmatrix}
	}
	=
	%Zähler
	\frac{\frac{1}{4}\cdot\left(-\frac{1}{2}\right)-1\cdot 1}
	%Nenner
	{1\cdot\left(-\frac{1}{2}\right)-\frac{3}{2}\cdot1}
	%
	=\frac{-\frac{1}{8}-1}{-\frac{1}{2}-\frac{3}{2}}=\frac{-\frac{9}{8}}{-2}=\frac{9}{16}\\
	%
	%
	C_2=
	\frac{
		\begin{vmatrix}
			1 & \nicefrac{1}{4}\\
			\nicefrac{3}{2} & 1
		\end{vmatrix}
	}
	{
		\begin{vmatrix}
			1 & 1 \\
			\nicefrac{3}{2} & -\frac{1}{2}
		\end{vmatrix}
	}
	=
	%Zähler
	\frac{1\cdot 1-\frac{3}{2}\cdot\frac{1}{4}}
	%Nenner
	{1\cdot\left(-\frac{1}{2}\right)-\frac{3}{2}\cdot1}
	=\frac{1-\frac{3}{8}}{-\frac{1}{2}-\frac{3}{2}}=\frac{\frac{5}{8}}{-2}=-\frac{5}{16}
\end{gather}
%
%
Gauß-Algorithmus:
%
%
\begin{gather}
	\left\lparen
	\begin{matrix}
		1 & 1 \\
		\nicefrac{3}{2} & -\nicefrac{1}{2}
	\end{matrix}
	\middle\vert
	\begin{matrix}
		\nicefrac{1}{4}\\
		1
	\end{matrix}
	\right\rparen\\
	%
	\mathrm{II}:=\mathrm{II}-\frac{3}{2}\mathrm{I}\\
	%
	%
	\left\lparen
	\begin{matrix}
		1 & 1 \\
		0 & -2
	\end{matrix}
	\middle\vert
	\begin{matrix}
		\nicefrac{1}{4}\\
		\frac{5}{8}
	\end{matrix}
	\right\rparen\\
	%
	%
	\mathrm{I}:=\mathrm{I}+\frac{1}{2}\mathrm{II}
	%
	%
	\left\lparen
	\begin{matrix}
		1 & 0 \\
		0 & -2
	\end{matrix}
	\middle\vert
	\begin{matrix}
		\nicefrac{9}{16}\\
		\frac{5}{8}
	\end{matrix}
	\right\rparen\\
	%
	%
	\mathrm{II}:=-\frac{1}{2}\mathrm{II}\\
	%
	%
	\left\lparen
	\begin{matrix}
		1 & 0 \\
		0 & 1
	\end{matrix}
	\middle\vert
	\begin{matrix}
		\nicefrac{9}{16}\\
		-\frac{5}{16}
	\end{matrix}
	\right\rparen
\end{gather}
%
%
%
Die Lösungen lassen sich aus dem rechten Spaltenvektor ablesen:
%
\begin{gather}
	C_1=\frac{9}{16},\nonumber \\
	C_2=-\frac{5}{16}.\nonumber
\end{gather}
%
Die Lösung der inhomogenen Differenzengleichung mit Anfangsbedingungen lautet 
somit
%
%
\begin{gather}
	y[k]=h[k]=
	\frac{3}{4}\delta[k]+\frac{9}{16}\left(\frac{3}{2}\right)^k\epsilon[k]
	-\frac{5}{16}\left(-\frac{1}{2}\right)^k\epsilon[k]
\end{gather}
%
%
\newpage
\subsection*{Lösung Aufgabe a) Vorwärts Rückwärts}
%
Wir verfolgen nun die Aufgabe a) ab dem Zeitpunkt, ab dem wir die 
Systemfunktion im z-Bereich hergeleitet haben. 
%
Wir transformieren diese nun nicht zurück, sondern bilden die z-Transformierte 
der vorgegebenen Impulsantworten, um diese dann mit der Hergeleiteten zu 
vergleichen. 
%
Am Ende muss $H(z)-\mathcal{Z}[h[k]]=0$ gelten.
%
\begin{gather}
	H_1(z)=\frac{3}{4}-\frac{5}{16}\frac{z}{z+\frac{1}{2}}+\frac{9}{16}\frac{z}{z-\frac{3}{2}}\\
	H_2(z)=1+\frac{5}{32}\frac{1}{z+\frac{1}{2}}+\frac{27}{32}\frac{1}{z-\frac{3}{2}}
\end{gather}
%
Zum transformieren von $h_1[k]$ und $h_2[k]$ nutzen wir folgende 
Korrespondenzen:
%
\begin{gather}
	Ax[k]+By[k]\quad\ztransf\quad AX(z)+BY(z),\\
	%
	\delta[k]\quad\ztransf\quad 1,\\
	%
	a^k\epsilon[k]\quad\ztransf\quad \frac{z}{z-1},\\
	%
	x[k-\kappa]\quad\ztransf\quad z^{-\kappa}X(z).
\end{gather}
%
\begin{gather}
	H_1(z)\overset{!}{=}\mathcal{Z}\left[h_1[k]\right]\\
	\frac{3}{4}-\frac{5}{16}\frac{z}{z+\frac{1}{2}}+\frac{9}{16}\frac{z}{z-\frac{3}{2}}
	=\frac{3}{4}-\frac{5}{16}\frac{z}{z+\frac{1}{2}}+\frac{9}{16}\frac{z}{z-\frac{3}{2}}\\
\end{gather}
%
%
Die Terme auf beiden Seiten sind offensichtlich gleich. 
%
Wir wiederholen das ganze mit $h_2[k]$.
%
%
\begin{gather}
	H_1(z)\overset{!}{=}\mathcal{Z}\left[h_2[k]\right]\\
	\frac{3}{4}-\frac{5}{16}\frac{z}{z+\frac{1}{2}}+\frac{9}{16}\frac{z}{z-\frac{3}{2}}
	=1+\frac{5}{32}\frac{1}{z+\frac{1}{2}}+\frac{27}{32}\frac{1}{z-\frac{3}{2}}
\end{gather}
%
%
Es werden nun beide Seiten auf einen gemeinsamen Nenner gebracht.
%
%
\begin{gather}
	\frac{
		%Zähler
		\frac{3}{4}(z+\frac{1}{2})(z-\frac{3}{2})-\frac{5}{16}z(z-\frac{3}{2})+\frac{9}{16}z(z+\frac{1}{2})
	}
	{
		%Nenner
		(z+\frac{1}{2})(z-\frac{3}{2})
	}
	=\frac{
		%Zähler
		(z+\frac{1}{2})(z-\frac{3}{2})+\frac{5}{32}(z-\frac{3}{2})+\frac{27}{32}(z+\frac{1}{2})
	}
	{
		%Nenner
		(z+\frac{1}{2})(z-\frac{3}{2})
	}
\end{gather}
%
%
Wir kürzen den Nenner weg und multiplizieren aus.
%
%
\begin{gather}
	\frac{3}{4}z^2-\frac{3}{4}z-\frac{9}{16}-\frac{5}{16}z^2+\frac{15}{32}z+\frac{9}{16}z^2+\frac{9}{32}z
	=z^2-z-\frac{3}{4}+\frac{5}{32}z-\frac{15}{64}+\frac{27}{32}z+\frac{27}{64}
\end{gather}
%
%
Es folgt eine Sortierung anhand der $z^k$ Terme.
%
%
\begin{gather}
	z^2(\frac{3}{4}-\frac{5}{16}+\frac{9}{16})+z(-\frac{3}{4}+\frac{15}{32}+\frac{9}{32})-\frac{9}{16}
	=z^2+z(-1+\frac{5}{32}+\frac{27}{32})-\frac{3}{4}-\frac{15}{16}+\frac{27}{64}
\end{gather}
%
%
Nun noch die Terme zu den jeweiligen $z^k$ Termen zusammenfassen.
%
\begin{gather}
	z^2-\frac{9}{16}=z^2-\frac{9}{16}
\end{gather}
%
Man sieht auch, dass $\mathcal{Z}\left[h_2[k]\right]=H_2(z)$ ist. 
%
Somit brauchen wir den Test hiernach nicht nochmal mit $H_2(z)$ wiederholen. 
%
Weil $H_1(z)=\mathcal{Z}\left[h_1[k]\right]$ und 
$H_1(z)=\mathcal{Z}\left[h_2[k]\right]$, ist auch $h_1[k]=h_2[k]$ und wir haben
somit die Zusatzaufgabe gelöst.\\
%
\newpage
\subsection*{Lösung a) Rückwärts}
\subsubsection*{Einsetzen in die lineare Differenzengleichung}
Als wir die Impulsantwort mit der Differenzengleichung gelöst haben, besprachen 
wir schon, dass am Systemausgang die Impulsantwort beobachtet werden kann, wenn 
am Eingang ein Dirac-Impuls anliegt, also falls $x[k]=\delta[k]$, dann 
$y[k]=h[k]$. 
%
Wenn die Differenzengleichung das System beschreibt, dann lassen sich 
$y[k]=h[k]$ und $x[k]=\delta[k]$ in die Gleichung einsetzen und es entsteht 
kein Widerspruch.\\
%
Es  gilt:
%
\begin{gather}
	y[k]-y[k-1]-\frac{3}{4}y[k-2]=x[k]-\frac{9}{16}x[k-2]
\end{gather}
%
%
%
\begin{align}
	&\frac{3}{4}\delta[k]-\frac{5}{16}\left(-\frac{1}{2}\right)^k\epsilon[k]+\frac{9}{16}\left(\frac{3}{2}\right)^k\epsilon[k]-\frac{3}{4}\delta[k-1]+\frac{5}{16}\left(-\frac{1}{2}\right)^{k-1}\epsilon[k-1]-\frac{9}{16}\left(\frac{3}{2}\right)^{k-1}\epsilon[k-1]\nonumber \\
	%
	&-\frac{9}{16}\delta[k-2]+\frac{15}{64}\left(-\frac{1}{2}\right)^{k-2}\epsilon[k-2]-\frac{27}{64}\left(\frac{3}{2}\right)^{k-2}\epsilon[k-2]=\delta[k]-\frac{9}{16}\delta[k]
\end{align}
%
%
%
Für $k=0$ ist die rechte Seite $1$, für $k=2$ ist die rechte Seite 
$-\frac{9}{16}$, ansonsten ist die rechte Seite $0$.\\
%
Die linke Seite ist für $k<0$ offensichtlich auch 0.\\
%
Für $k=0$ beträgt die linke Seite
%
%
\begin{align}
	\frac{3}{4}\delta[0]-\frac{5}{16}\left(-\frac{1}{2}\right)^0\epsilon[0]+\frac{9}{16}\left(\frac{3}{2}\right)^0\epsilon[0]
	=\frac{3}{4}-\frac{5}{16}+\frac{9}{16}=\frac{3}{4}+\frac{4}{16}=1.
\end{align}
%
%
Für $k=0$ gleichen sich linke und rechte Seite.\\
%
Für $k=1$ beträgt die linke Seite
%
%
\begin{align}
	&\frac{3}{4}\delta[1]-\frac{5}{16}\left(-\frac{1}{2}\right)^1\epsilon[1]+\frac{9}{16}\left(\frac{3}{2}\right)^1\epsilon[1]-\frac{3}{4}\delta[0]+\frac{5}{16}\left(-\frac{1}{2}\right)^{0}\epsilon[0]-\frac{9}{16}\left(\frac{3}{2}\right)^{0}\epsilon[0]\nonumber\nonumber \\
	%
	&=\left(-\frac{5}{16}\right)\cdot\left(-\frac{1}{2}\right)+\frac{9}{16}\cdot\frac{3}{2}-\frac{3}{4}+\frac{5}{16}-\frac{9}{16}=\frac{5}{32}+\frac{27}{32}-\frac{3}{4}+\frac{5}{16}-\frac{9}{16}=\frac{5+27-24+10-18}{32}=\frac{0}{32}=0.
\end{align}
%
%
Für $k=1$ gleichen sich linke und rechte Seite.\\
%
Für $k=2$ beträgt die linke Seite
%
%
%
\begin{align}
	&\frac{3}{4}\delta[2]-\frac{5}{16}\left(-\frac{1}{2}\right)^2\epsilon[2]+\frac{9}{16}\left(\frac{3}{2}\right)^2\epsilon[2]-\frac{3}{4}\delta[1]+\frac{5}{16}\left(-\frac{1}{2}\right)^{1}\epsilon[1]-\frac{9}{16}\left(\frac{3}{2}\right)^{1}\epsilon[1]\nonumber \\
	%
	&-\frac{9}{16}\delta[0]+\frac{15}{64}\left(-\frac{1}{2}\right)^{0}\epsilon[0]-\frac{27}{64}\left(\frac{3}{2}\right)^{0}\epsilon[0]\nonumber\\
	%
	&=\left(\frac{5}{16}\right)\cdot\frac{1}{4}+\frac{9}{16}\cdot\frac{9}{4}+\frac{5}{16}\cdot\left(-\frac{1}{2}\right)-\frac{9}{16}\cdot\frac{3}{2}-\frac{9}{16}+\frac{15}{64}-\frac{27}{64}\nonumber\\
	%
	&=-\frac{5}{64}+\frac{81}{64}-\frac{5}{32}-\frac{27}{32}-\frac{9}{16}+\frac{15}{64}-\frac{27}{64}\nonumber\\
	%
	&=\frac{-5+81-10-54-36+15-27}{64}=-\frac{36}{64}=-\frac{18}{32}=-\frac{9}{16}.
\end{align}
%
%
%
Für $k=2$ gleichen sich linke und rechte Seite.\\
%
Für $k>2$ beträgt die rechte Seite
%
%
%
\begin{align}
	&\frac{3}{4}\delta[k]-\frac{5}{16}\left(-\frac{1}{2}\right)^k\epsilon[k]+\frac{9}{16}\left(\frac{3}{2}\right)^k\epsilon[k]-\frac{3}{4}\delta[k-1]+\frac{5}{16}\left(-\frac{1}{2}\right)^{k-1}\epsilon[k-1]-\frac{9}{16}\left(\frac{3}{2}\right)^{k-1}\epsilon[k-1]\nonumber \\
	%
	&-\frac{9}{16}\delta[k-2]+\frac{15}{64}\left(-\frac{1}{2}\right)^{k-2}\epsilon[k-2]-\frac{27}{64}\left(\frac{3}{2}\right)^{k-2}\epsilon[k-2]\nonumber\\
	%
	&\underset{k>2}{=}-\frac{5}{16}\left(-\frac{1}{2}\right)^k+\frac{9}{16}\left(\frac{3}{2}\right)^k+\frac{5}{16}\left(-\frac{1}{2}\right)^{k-1}-\frac{9}{16}\left(\frac{3}{2}\right)^{k-1}+\frac{15}{64}\left(-\frac{1}{2}\right)^{k-2}-\frac{27}{64}\left(\frac{3}{2}\right)^{k-2}\nonumber \\
	%
	&=-\frac{5}{16}\left(-\frac{1}{2}\right)^k+\frac{9}{16}\left(\frac{3}{2}\right)^k+\frac{5}{16}\left(-\frac{1}{2}\right)^k\cdot\left(-\frac{1}{2}\right)^{-1}-\frac{9}{16}\left(\frac{3}{2}\right)^k\cdot\left(\frac{3}{2}\right)^{-1}\nonumber \\
	%
	&+\frac{15}{64}\left(-\frac{1}{2}\right)^k\cdot\left(-\frac{1}{2}\right)^{-2}-\frac{27}{64}\left(\frac{3}{2}\right)^k\cdot\left(\frac{3}{2}\right)^{-2}\nonumber\\
	%
	&=-\frac{5}{16}\left(-\frac{1}{2}\right)^k+\frac{9}{16}\left(\frac{3}{2}\right)^k+\frac{5}{16}\left(-\frac{1}{2}\right)^k\cdot \left(-2\right)-\frac{9}{16}\left(\frac{3}{2}\right)^k\cdot\left(\frac{2}{3}\right)+\frac{15}{64}\left(-\frac{1}{2}\right)^k\cdot 4-\frac{27}{64}\left(\frac{3}{2}\right)^k\cdot\left(\frac{4}{9}\right)\nonumber\\
	%
	&=-\frac{5}{16}\left(-\frac{1}{2}\right)^k+\frac{9}{16}\left(\frac{3}{2}\right)^k-\frac{5}{8}\left(-\frac{1}{2}\right)^k-\frac{3}{8}\left(\frac{3}{2}\right)^k+\frac{15}{16}\left(-\frac{1}{2}\right)^k-\frac{3}{16}\left(\frac{3}{2}\right)^k\nonumber\\
	%
	&=\left(-\frac{1}{2}\right)^k\Bigg [-\frac{5}{16}-\frac{5}{8}+\frac{15}{16}\Bigg ]+\left(\frac{3}{2}\right)^k\Bigg [\frac{9}{16}-\frac{3}{8}-\frac{3}{16}\Bigg ]\nonumber\\
	%
	&=\left(-\frac{1}{2}\right)^k\Bigg [\frac{-5-10+15}{16}\Bigg ]+\left(\frac{3}{2}\right)^k\Bigg [\frac{9-6-3}{16}\Bigg ]=0.
\end{align}
%
%
%
Auch für $k>2$ sind beide Seiten gleich. Somit ist $h[k]$ eine Lösung der 
Differenzengleichung mit $x[k]=\delta[k]$ und damit die Impulsantwort.\\
%
Für $h_2[k]$ ist das Vorgehen analog.
%
\newpage
\subsubsection*{Herleiten der Differenzengleichung aus der Impulsantwort}
%
Nun wollen wir die Differenzengleichung aus der Impulsantwort $h_1[k]$ 
herleiten. 
%
Für $h_2[k]$ würde man analog vorgehen.\\
%
Zunächst bilden wir die z-Transformierte der Impulsantwort:
%
\begin{gather}
	h[k]=\frac{3}{4}\delta[k]-\frac{5}{16}\left(-\frac{1}{2}\right)^k\epsilon[k]+\frac{9}{16}\left(\frac{3}{2}\right)^k\epsilon[k]\\
	%
	\ztransf\nonumber\\
	H(z)=\frac{3}{4}-\frac{5}{16}\frac{z}{z+\frac{1}{2}}+\frac{9}{16}\frac{z}{z-\frac{3}{2}}
\end{gather}
%
Wir bringen nun die Systemfunktion auf einen gemeinsamen Nenner.
%
\begin{align}
	H(z)=
	\frac{
		%Zähler
		\frac{3}{4}\left(z+\frac{1}{2}\right)\left(z-\frac{3}{2}\right)-\frac{5}{16}z\left(z-\frac{3}{2}\right)+\frac{9}{16}z\left(z+\frac{1}{2}\right)
	}
	{
		%Nenner
		\left(z+\frac{1}{2}\right)\left(z-\frac{3}{2}\right)
	}
\end{align}
%
Wir multiplizieren aus und sortieren ein weg.
%
\begin{gather}
	H(z)=\frac{z^2-\frac{9}{16}}{z^2-z-\frac{3}{4}}
\end{gather}
%
Die Systemfunktion ergab sich ja auch aus der Division der $z$-Transformierten 
des Ausgangssignals durch die $z$-Transformierte des Eingangsignals.
%
\begin{gather}
	H(z)=\frac{Y(z)}{X(z)}=\frac{z^2-\frac{9}{16}}{z^2-z-\frac{3}{4}}\Bigg |\cdot X(z)(z^2-z-\frac{3}{4})\\
	Y(z)(z^2-z-\frac{3}{4})=X(z)(z^2-\frac{9}{16})\\
	\Ztransf\nonumber \\
	y[k+2]-y[k+1]-\frac{3}{4}y[k]=x[k+2]-\frac{9}{16}x[k]
\end{gather}
%
Wir schieben Differenzengleichung noch um $2$ nach rechts,
%
\begin{align}
	y[k]-y[k-1]-\frac{3}{4}y[k]=x[k]-\frac{9}{16}x[k-2]
\end{align}
%
, und erhalten unsere am Anfang aufgestellte Differenzengleichung. 
%
Wenn man $H(z)$ noch mit $\frac{z^{-2}}{z^{-2}}$ erweitert hätte, 
dann würde später die Verschiebung wegfallen.
%
\newpage
\subsection*{b)}
%
Instabilität sieht man
%
\begin{itemize}
	\item ein Pol außerhalb des Einheitskreises, KB schliesst Einheitskreis nicht ein
	%
	\item $\left(\frac{3}{2}\right)^k \epsilon[k]$ ist eine aufklingende Folge, die nicht von anderen Termen in $h[k]$ kompensiert wird, daher wächst $h[k]$ und auch $h_\epsilon[k]$ über alle Grenzen
	%
	\item in \url{https://nbviewer.jupyter.org/github/spatialaudio/digital-signal-processing-exercises/blob/outputs/filter_design/iir_biquad.ipynb}
	findet man ein drittes Kriterium anhand von $a$-Koeffizienten in $H(z)$, $a_2>|a_1|-1$ wird verletzt
\end{itemize}
%
\newpage
\subsection*{Zusatzaufgabe}
%
Hier soll $h_1[k]=h_2[k]$ gezeigt werden.
%
\subsubsection*{Direkter Vergleich der Impulsantworten}
%
Für $k<0$ gilt offensichtlich $h_1[k]=h_2[k]$.\\
%
Nun $k=0$:
%
\begin{gather}
	h_1[k=0]=\frac{3}{4}-\frac{5}{16}+\frac{9}{16}=\frac{3}{4}+\frac{4}{16}=1\\
	h_2[k=0]=1
\end{gather}
%
%
Für $k=0$ gilt also $h_1[k]=h_2[k]$.\\
%
Zum Schluss $k>0$. Dafür formen wir $h_2[k]$ etwas um.
%
%
\begin{gather}
	h_1[k>0]=-\frac{5}{16}\bigg (-\frac{1}{2}\bigg )^k+\frac{9}{16}\bigg (\frac{3}{2}\bigg )^k\\
	%
	h_2[k>0]=\frac{5}{32}\bigg (-\frac{1}{2}\bigg )^{k-1}+\frac{27}{32}\bigg (\frac{3}{2}\bigg )^{k-1}
	=\frac{5}{32}\cdot(-2)\bigg (-\frac{1}{2}\bigg )^k+\frac{27}{32}\cdot\frac{2}{3}\bigg (\frac{3}{2}\bigg )^k\\
	%
	h_2[k>0]=-\frac{5}{16}\bigg (-\frac{1}{2}\bigg )^k+\frac{9}{16}\bigg (\frac{3}{2}\bigg )^k
\end{gather}
%
%
Also gilt $h_2[k]=h_1[k]$
%
\subsubsection*{Vergleich der z-Transformierten der Impulsantworten}
%
\begin{gather}
	h_1[k]\overset{!}{=}h_2[k]\\
	%
	\mathcal{Z}[h_1[k]]\overset{!}{=}\mathcal{Z}[h_2[k]]\\
	%
	\frac{3}{4}-\frac{5}{16}\frac{z}{z+\frac{1}{2}}+\frac{9}{16}\frac{z}{z-\frac{3}{2}}=1+\frac{5}{32}\frac{1}{z+\frac{1}{2}}+\frac{27}{32}\frac{1}{z-\frac{3}{2}}
\end{gather}
%
Nun werden beiden Seiten auf den gleichen Nenner gebracht.
%
\begin{gather}
	\frac{
		%Zähler
		\frac{3}{4}(z+\frac{1}{2})(z-\frac{3}{2})-\frac{5}{16}z(z-\frac{3}{2})+\frac{9}{16}z(z+\frac{1}{2})
	}
	{
		%Nenner
		(z+\frac{1}{2})(z-\frac{3}{2})
	}
	=\frac{
		%Zähler
		(z+\frac{1}{2})(z-\frac{3}{2})+\frac{5}{32}(z-\frac{3}{2})+\frac{27}{32}(z+\frac{1}{2})
	}
	{
		%Nenner
		(z+\frac{1}{2})(z-\frac{3}{2})
	}
\end{gather}
%
Der Nenner kann weggekürzt werden und wir multiplizieren den Zähler aus.
%
\begin{gather}
	\frac{3}{4}(z-^2-z-\frac{3}{4})-\frac{5}{16}z^2+\frac{15}{32}z+\frac{9}{16}z^2+\frac{9}{32}z
	=z^2-z-\frac{3}{4}+\frac{5}{32}z-\frac{15}{64}+\frac{27}{32}z+\frac{27}{64}
\end{gather}
%
Wir können noch ein wenig sortieren und erhalten
%
\begin{gather}
	z^2-\frac{9}{16}=z^2-\frac{9}{16}.
\end{gather}
%
Beide Seiten sind gleich. Somit haben wir gezeigt, dass $h_1[k]=h_2[k]$.
%
\renewcommand{\refname}{Buchzitate}
\clearpage
\bibliography{../tutorial_latex_deu/literatur}
\end{document}

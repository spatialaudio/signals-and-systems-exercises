\documentclass[11pt,a4paper,DIV=12]{scrartcl}
\usepackage{scrlayer-scrpage}
\usepackage[utf8]{inputenc}
\usepackage{fouriernc}
\usepackage[T1]{fontenc}
\usepackage[german]{babel}
\usepackage[hidelinks]{hyperref}
\usepackage{natbib}
\usepackage{url}
\usepackage{amsmath}
\usepackage{amsfonts}
\usepackage{amssymb}
\usepackage{trfsigns}
\usepackage{bm}
\usepackage{marvosym}
\usepackage{nicefrac}
\usepackage{graphicx}
\usepackage{subcaption}
\usepackage{xcolor}
%\usepackage{comment}
%\usepackage{mdframed}
%\usepackage{tikz}
%\usepackage{circuitikz}
%\usepackage{pgfplots}
\bibliographystyle{dinat}

%\numberwithin{equation}{section}
%\numberwithin{figure}{section}

% \usetikzlibrary{calc}
% \usetikzlibrary{positioning}
% \usetikzlibrary{matrix}
% \usetikzlibrary{chains}
% \usetikzlibrary{shapes.misc}
% \tikzset{cross/.style={cross out, draw,minimum size=2*(#1-\pgflinewidth),inner sep=0pt, outer sep=0pt}}
% \pgfkeys{/pgfplots/axis style/.code={\pgfkeysalso{/pgfplots/every axis/.append style={#1}}}}
% \pgfplotsset{
% mathaxis/.style={
% axis lines=center,
% xtick=\empty,
% ytick=\empty,
% xlabel style=right,
% ylabel style=above,
% % Make sure the origin is shown (http://tex.stackexchange.com/a/91253)
% before end axis/.code={
% \addplot [draw=none, forget plot] coordinates {(0,0)};
% },
% anchor=origin,
% },
% stemaxis/.style={
% mathaxis,
% x=.8em,
% y=6ex,
% enlarge x limits={abs=1.2em},
% enlarge y limits={abs=1.2em},
% }
% }
% \tikzstyle{stem}=[ycomb,mark=*,mark size=10\pgflinewidth,color=C0,ultra thick]
% \tikzstyle{stem2}=[ycomb,mark=*,mark size=5\pgflinewidth,color=C0,ultra thick]

%\newcommand\fsd{\mathrm{d}} %der/int operator
%\newcommand{\sysH}[1]{\mathcal{H}{\{#1\}}}  % system operator
%\renewcommand{\vec}[1]{\mathbf{#1}} %vector
%\newcommand{\eq}[1]{Glg. (\ref{#1})} %ref equation
%\newcommand{\fig}[1]{Abb. \ref{#1}} %ref figure
% \newcommand{\red}{\textcolor{red}}
% \newcommand{\blue}{\textcolor{blue}}
% \newcommand{\ured}[1]{\textcolor{red}{\underline{#1}}}
% \newcommand{\ublue}[1]{\textcolor{blue}{\underline{#1}}}
% \newcommand{\ugreen}[1]{\textcolor{green}{\underline{#1}}}
% \newcommand{\uorange}[1]{\textcolor{orange}{\underline{#1}}}
% \newcommand{\umagenta}[1]{\textcolor{magenta}{\underline{#1}}}
% \newcommand{\ublack}[1]{\textcolor{black}{\underline{#1}}}
% \newcommand{\ubrown}[1]{\textcolor{brown}{\underline{#1}}}
\newcommand{\diff}{\mathrm{d}}
%Sha symbol:
% \DeclareFontFamily{U}{wncy}{}
% \DeclareFontShape{U}{wncy}{m}{n}{<->wncyr10}{}
% \DeclareSymbolFont{mcy}{U}{wncy}{m}{n}
% \DeclareMathSymbol{\Sha}{\mathord}{mcy}{"58}
%\newcommand{\Sha}{$\bot \!\! \bot \!\! \bot$}

%\DeclareMathOperator{\rectOP}{\text{rect}}
%\newcommand{\rectN}[2]{\ensuremath{\rectOP_{#1} \left[ #2 \right]}}

%matplotlib colors:
%\definecolor{C0}{HTML}{1f77b4}
%\definecolor{C1}{HTML}{ff7f0e}
%\definecolor{C2}{HTML}{2ca02c}
%\definecolor{C3}{HTML}{d62728}
%\definecolor{C7}{HTML}{7f7f7f}

%\specialcomment{Ziel}{\begin{mdframed}[backgroundcolor=C2!10] \textit{Lernziel}\\\noindent}{\end{mdframed}\noindent}
%\specialcomment{Werkzeug}{\begin{mdframed}[backgroundcolor=C7!10] \textit{Werkzeug}\\\noindent}{\end{mdframed}\noindent}
%\specialcomment{Ansatz}{\begin{mdframed}[backgroundcolor=C3!10] \textit{Ansatz}\\\noindent}{\end{mdframed}\noindent}
%\specialcomment{ExCalc}{\begin{mdframed}[backgroundcolor=C1!10]\noindent \textit{Ausführliche Rechnung}\\\noindent}{\end{mdframed}\noindent}
%\specialcomment{Loesung}{\begin{mdframed}[backgroundcolor=C0!10] \textit{Lösung}\\\noindent}{\end{mdframed}\noindent}

%https://ctan.org/tex-archive/macros/latex/contrib/trfsigns?lang=en
%needs \rmfamily instead of \rm in trfsigns package
\renewcommand{\ztransf}{\mbox{\setlength{\unitlength}{0.1em}%
                            \begin{picture}(20,10)%
                              \put(2,3){\circle{4}}%
                              \put(4,3){\line(1,0){4.75}}%
                              \multiput(8.625,3.15)(0.25,0.25){11}{%
                                \makebox(0,0){\rmfamily\tiny .}}%
                              \put(17,3){\line(-1,0){5.75}}%
                              \put(18,3){\circle*{4}}%
                            \end{picture}%
                           }
                      }
\renewcommand{\Ztransf}{\mbox{\setlength{\unitlength}{0.1em}%
                            \begin{picture}(20,10)%
                              \put(2,3){\circle*{4}}%
                              \put(3,3){\line(1,0){5.75}}%
                              \multiput(11.375,3.15)(-0.25,0.25){11}{%
                                \makebox(0,0){\rmfamily\tiny .}}%
                              \put(16,3){\line(-1,0){4.75}}%
                              \put(18,3){\circle{4}}%
                            \end{picture}%
                           }
                      }

% \newcommand{\dtft}{\mbox{\setlength{\unitlength}{0.1em}%
%                             \begin{picture}(20,10)%
%                               \put(2,3){\circle{4}}%
%                               \put(4,3){\line(1,0){4.75}}%
%                               \multiput(8.625,3.15)(0.25,0.25){11}{%
%                                 \makebox(0,0){\rmfamily\tiny .}}%
%                               \put(17,3){\line(-1,0){5.75}}%
%                               %\put(18,3){\circle*{4}}%
%                             \end{picture}%
%                            }
%                       }
% \newcommand{\DTFT}{\mbox{\setlength{\unitlength}{0.1em}%
%                             \begin{picture}(20,10)%
%                               %\put(2,3){\circle*{4}}%
%                               \put(3,3){\line(1,0){5.75}}%
%                               \multiput(11.375,3.15)(-0.25,0.25){11}{%
%                                 \makebox(0,0){\rmfamily\tiny .}}%
%                               \put(16,3){\line(-1,0){4.75}}%
%                               \put(18,3){\circle{4}}%
%                             \end{picture}%
%                            }
%                       }


% \newcommand{\mydft}{\mbox{\setlength{\unitlength}{0.1em}%
%                             \begin{picture}(20,10)%
%                               \put(2,3){\circle{4}}%
%                               \put(4,3){\line(1,0){4.75}}%
%                               \multiput(8.625,3.15)(0.25,0.25){11}{%
%                                 \makebox(0,0){\rmfamily\tiny .}}%
%                               \put(17,3){\line(-1,0){5.75}}%
%                               %\put(18,3){\circle*{4}}%
%                               \put(6,-4){\scriptsize $N$}
%                             \end{picture}%
%                            }
%                       }
%
% \newcommand{\myDFT}{\mbox{\setlength{\unitlength}{0.1em}%
%                             \begin{picture}(20,10)%
%                               %\put(2,3){\circle*{4}}%
%                               \put(3,3){\line(1,0){5.75}}%
%                               \multiput(11.375,3.15)(-0.25,0.25){11}{%
%                                 \makebox(0,0){\rmfamily\tiny .}}%
%                               \put(16,3){\line(-1,0){4.75}}%
%                               \put(18,3){\circle{4}}%
%                               \put(6,-4){\scriptsize $N$}
%                             \end{picture}%
%                            }
%                       }

%\excludecomment{Ziel}
%\excludecomment{Werkzeug}
%\excludecomment{ExCalc}

%------------------------------------------------------------------------------
% \title{Übung Signal- und Systemtheorie\thanks{
% This tutorial is provided as Open Educational Resource (OER), to be found at
% \url{https://github.com/spatialaudio/signals-and-systems-exercises}
% accompanying the OER lecture
% \url{https://github.com/spatialaudio/signals-and-systems-lecture}.
% %
% Both are licensed under a) the Creative Commons Attribution 4.0 International
% License for text and graphics and b) the MIT License for source code.
% %
% Please attribute material from the tutorial as \textit{Frank Schultz,
% Continuous- and Discrete-Time Signals and Systems - A Tutorial Featuring
% Computational Examples, University of Rostock} with
% \texttt{main file, github URL, commit SHA number and/or version tag, year}.
% }
% \\
% \small Vst.-Nr. 24015}
% %
% \author{Frank Schultz, Sascha Spors\\
% \small Institut für Nachrichtentechnik (INT)\\
% \small Fakultät für Informatik und Elektrotechnik (IEF)\\
% \small Universität Rostock
% }
% %
% \date{Sommersemester 2021, Version II: \today}
%------------------------------------------------------------------------------

%\ihead{Aufgabe \thesubsection}
\ohead{Signal- und Systemtheorie Übung}
\cfoot{\pagemark}
\ofoot{\tiny\url{https://github.com/spatialaudio/signals-and-systems-exercises}}

\begin{document}
\noindent Signal- und Systemtheorie Übung\footnote{This tutorial is provided as
Open Educational Resource (OER), to be found at
\url{https://github.com/spatialaudio/signals-and-systems-exercises}
accompanying the OER lecture
\url{https://github.com/spatialaudio/signals-and-systems-lecture}.
%
Both are licensed under a) the Creative Commons Attribution 4.0 International
License for text and graphics and b) the MIT License for source code.
%
Please attribute material from the tutorial as \textit{Frank Schultz,
Continuous- and Discrete-Time Signals and Systems - A Tutorial Featuring
Computational Examples, University of Rostock} with
\texttt{main file, github URL, commit SHA number and/or version tag, year}
.}---Frank Schultz, Sascha Spors,
Institut für Nachrichtentechnik (INT),
Fakultät für Informatik und Elektrotechnik (IEF),
Universität Rostock \&
Robert Hauser, Universität Rostock---Version: \today

\section{Übung: Inverse z-Transformation}

%\subsection{Task}
\subsection{Aufgabe}
%Transform
Transformieren Sie die folgenden z-Transformierten
\begin{itemize}
	\item[a )] $X(z)=2\cdot\frac{z\cdot(z-\frac{1}{\sqrt{2}})}{(z-\e^{+\im\frac{\pi}{4}})\cdot(z-\e^{-\im\frac{\pi}{4}})}$\quad \text{KB: }$|z|>1$
	\item[b )] $X(z)=\frac{z}{z-\frac{1}{2}}$\quad \text{KB: }$|z|>\frac{1}{2}$
	\item[c )] $X(z)=\frac{z}{z-\frac{1}{2}}\cdot\frac{z}{z-1}$\quad \text{KB: }$|z|>1$
	\item[d )] $X(z)=\frac{z^2-z+2}{z^2-\frac{1}{2}z+\frac{1}{4}}$\quad \text{KB: }$|z|>\frac{1}{2}$
	\item[e )] $X(z)=\frac{z^2}{z^2+1}$\quad \text{KB: }$|z|>1$
	\item[f )] $X(z)=\frac{z^4+z^3-6z^2+6z-1}{z^2-2z+1}$\quad \text{KB: }$|z|>\frac{1}{2}$
	\item[g )] $X(z)=\frac{z\cdot(z-1)}{z^2-\sqrt{2}z+1}$\quad \text{KB: }$|z|>1$
\end{itemize}
%into time-discret signals.
%The task shall be performed with help of the
in zeit-diskrete Signale $x[k]$. %Der Konvergenzbereich sei bei allen Aufgaben bis auf c) $|z|>1$ und bei Aufgabe c)?! Typo ?! $|z|>\frac{1}{2}$. KB besser bei den indiviudellen Tasks
Die Aufgabe soll gelöst werden mit Hilfe
\begin{itemize}
	%\item \textbf{correspondence table},
	%\item \textbf{Residue theorem}.
	\item[i )] der \textbf{Korrespondenztabelle} in der Formelsammlung
	\item[ii )] des \textbf{Residuensatz}es.
\end{itemize}
Berechen Sie außerdem die inverse z-Transformation von
\begin{itemize}
	\item[h )] $X(z)=\e^{1/z}$
\end{itemize}
mit Hilfe des \textbf{Residuensatz}es.
%\begin{itemize}
%	\item[ii )] \textbf{Residuensatz}.
%\end{itemize}
%For a quick view there is a solution table at the end of the document.

Am Ende des Dokumentes in Anhang \ref{sec:AnhangB} findet sich eine Übersicht mit Lösungen.
%\subsection{Solution Using Correspondence Table}
\subsection{Lösung mit Hilfe der Korrespondenztabelle}
Es wird die Korrespondenztabelle der Vorlesung/Übung, die im StudIp zur Verfügung steht, genutzt, sie sind in Anhang \ref{sec:AnhangC} nochmal abgedruckt. Diese deckt zwar nur allgemeine, einfache Fälle ab, ist damit jedoch übersichtlicher. In anderen Büchern (z.B. \cite[Kap. 21, S. 1147-1149]{Bronstein} oder \cite[S. 237-238]{UlrichWeber2017} ) gibt es weit umfangreichere Korrespondenztabellen, diese können jedoch nicht in der Klausur verwendet werden.
%\subsubsection{Task a)}
\subsubsection{Aufgabe a)}
%We shall find the time-discret signal for
Wir wollen das zeit-diskrete Signal für
\begin{align}
	X(z)&=2\cdot\frac{z\cdot(z-\frac{1}{\sqrt{2}})}{(z-\e^{+\im\frac{\pi}{4}})\cdot(z-\e^{-\im\frac{\pi}{4}})}=2\cdot\frac{z\cdot(z-\frac{1}{\sqrt{2}})}{z^2-z(\e^{+\im\frac{\pi}{4}}+\e^{-\im\frac{\pi}{4}})+1}=2\cdot\frac{z\cdot(z-\frac{1}{\sqrt{2}})}{z^2-z\bigg[\cos(\frac{\pi}{4})+\im\sin(\frac{\pi}{4})+\cos(\frac{\pi}{4})-\im\sin(\frac{\pi}{4})\bigg ]+1}\nonumber\\
	&=2\cdot\frac{z\cdot(z-\frac{1}{\sqrt{2}})}{z^2-2z\cos(\frac{\pi}{4})+1}=2\cdot\frac{z^2-z\cos(\frac{\pi}{4})}{z-2z\cos(\frac{\pi}{4})+1}
\end{align}
berechnen.
%If we take a look at the correspondence table, we find following correspondence:
Durch die Umformungen gelangen wir zu einer Korrespondenz, die sich direkt in der Tabelle findet:
\begin{align}
	\cos[\Omega_0k]\epsilon[k]\quad\ztransf\quad\frac{z^2-z\cos(\Omega_0)}{z^2-2z\cos(\Omega_0)+1}.
\end{align}
%So our solution is
Also ist unsere Lösung
\begin{align}
	X(z)=2\cdot\frac{z^2-z\cos(\frac{\pi}{4})}{z-2z\cos(\frac{\pi}{4})+1}\quad\Ztransf\quad x[k]=2\cdot\cos[\frac{\pi}{4}k]\epsilon[k].
\end{align}
%\subsubsection{Task b)}
\subsubsection{Aufgabe b)}
%We shall find the time-discret signal for
Wir wollen das zeit-diskrete Signal für
\begin{align}
	X(z)=\frac{z}{z-\frac{1}{2}}
\end{align}
berechnen.
%If we take a look at the correspondence table, we find following correspondence:
Wenn wir einen Blick auf die Korrespondenztabelle werfen, finden wir direkt folgende Korrespondenz:
\begin{align}
	\label{eq:GeometricProgression}
	a^k\epsilon[k]\quad\ztransf\quad\frac{z}{z-a}.
\end{align}
%So our solution is
Also ist unsere Lösung
\begin{align}
	X(z)=\frac{z}{z-\frac{1}{2}}\quad\Ztransf\quad x[k]=\bigg (\frac{1}{2}\bigg)^k\epsilon[k].
\end{align}
%\subsubsection{Task c)}
\subsubsection{Aufgabe c)}
%We shall find the time-discret signal for
Wir wollen das zeit-diskrete Signal für
\begin{align}
	X(z)=\frac{z}{z-\frac{1}{2}}\cdot\frac{z}{z-1}
\end{align}
berechnen.
%Now we can not find a correspondence in our table. But it seems that a partial fraction decomposition could be helpful.
Zunächst können wir keine passende Korrespondenz in unserer Tabelle finden. Aber eine Partialbruchzerlegung könnte hilfreich sein.
\begin{align}
	\frac{X(z)}{z}=\frac{1}{z-\frac{1}{2}}\cdot\frac{z}{z-1}=\frac{A}{z-\frac{1}{2}}+\frac{B}{z-1} \quad \Bigg | \quad\cdot (z-\frac{1}{2})\cdot(z-1)\nonumber\\
	z=A\cdot(z-1)+B\cdot(z-\frac{1}{2})=z(A+B)-A-\frac{B}{2}
\end{align}
%We get a system of linear equations. In matrix notation:
Wir erhalten ein lineares Gleichungssystem, hier dargestellt in der Matrixnotation:
\begin{align}
	\begin{pmatrix}
		1 & 1\\
		-1 & -\frac{1}{2}
	\end{pmatrix}
	\begin{pmatrix}
		A \\
		B
	\end{pmatrix}
	=
	\begin{pmatrix}
		1 \\
		0
	\end{pmatrix}.
\end{align}
%With techniques like Gaussian elimination or Cramer's rule we can solute our system of linear equations and achieve
Das lineare Gleichungssystem lässt sich mit Techniken wie dem Gaußschem Eliminationsverfahren oder der Cramerschen Regel lösen und wir erhalten
\begin{align}
	A = -1 \nonumber \\
	B = 2.
\end{align}
\begin{align}
	\frac{X(z)}{z}=-1\cdot\frac{1}{z-\frac{1}{2}}+2\cdot\frac{1}{z-1} \quad\Bigg | \quad \cdot z
\end{align}

\begin{align}
	X(z)=-1\cdot\frac{z}{z-\frac{1}{2}}+2\cdot\frac{z}{z-1}
\end{align}
%Using the correspondence from \eqref{eq:GeometricProgression} and
Wir nutzen die Korrespondenz aus Glg. \eqref{eq:GeometricProgression}, aus der sich zudem für $a=1$
\begin{align}
	\epsilon[k]\quad\ztransf\quad\frac{z}{z-1}.
\end{align}
ableiten lässt.
%We can find the time-discret signal:
Damit können wir das zeit-diskrete Signal angeben:
\begin{align}
	X(z)=-1\cdot\frac{1}{z-\frac{1}{2}}+2\cdot\frac{1}{z-1}\quad\Ztransf\quad x[k]=-\bigg(\frac{1}{2}\bigg)^k\epsilon[k]+2\epsilon[k].
\end{align}
%\subsubsection{Task d)}
\subsubsection{Aufgabe d)}
%We shall find the time-discret signal for
Wir wollen das zeit-diskrete Signal für
\begin{align}
	X(z)=\frac{z^2-z+2}{z^2-\frac{1}{2}z+\frac{1}{4}}
\end{align}
berechnen.
%We use polynomial division and partial fraction decomposition to achieve a term like in our correspondence table.
Wir nutzen Polynomdivision und Partialbruchzerlegung, um Terme aus unserer Korrespondenztabelle zu erhalten.
\begin{align}
	X(z)=\frac{z^2-z+2}{z^2-\frac{1}{2}z+\frac{1}{4}}=1+\frac{-\frac{1}{2}z+\frac{7}{4}}{z^2-\frac{1}{2}z+\frac{1}{4}}
\end{align}
\begin{align}
	\label{eq:TaskDPartialFractionDecomposition}
	\frac{-\frac{1}{2}z+\frac{7}{4}}{(z-(\frac{1}{4}+\im\frac{\sqrt{3}}{4}))\cdot(z-(\frac{1}{4}-\im\frac{\sqrt{3}}{4}))}=\frac{A}{z-(\frac{1}{4}+\im\frac{\sqrt{3}}{4})}+\frac{B}{z-(\frac{1}{4}-\im\frac{\sqrt{3}}{4})}\quad\Bigg | \quad \cdot (z-(\frac{1}{4}+\im\frac{\sqrt{3}}{4}))\cdot(z-(\frac{1}{4}-\im\frac{\sqrt{3}}{4}))\nonumber\\
	-\frac{1}{2}z+\frac{7}{4}=A(z-(\frac{1}{4}-\im\frac{\sqrt{3}}{4}))+B(z-(\frac{1}{4}+\im\frac{\sqrt{3}}{4}))=z(A+B)+A(-\frac{1}{4}+\im\frac{\sqrt{3}}{4})+B(-\frac{1}{4}-\im\frac{\sqrt{3}}{4})
\end{align}
%We get a system of linear equations. In matrix notation:
Wir erhalten ein lineares Gleichungssystem, hier dargestellt in der Matrixnotation:
\begin{align}
	\begin{pmatrix}
		1 & 1 \\
		-\frac{1}{4}+\im\frac{\sqrt{3}}{4} & -\frac{1}{4}-\im\frac{\sqrt{3}}{4}
	\end{pmatrix}
	\begin{pmatrix}
		A \\
		B
	\end{pmatrix}
	=
	\begin{pmatrix}
		-\frac{1}{2} \\
		\frac{7}{4}
	\end{pmatrix}
\end{align}
%With techniquest like Gaussian elimination or Cramer's rule, we can solute our system of linear equations and achieve
Das lineare Gleichungssystem lässt sich mit Techniken wie dem Gaußschem Eliminationsverfahren oder der Cramerschen Regel lösen und wir erhalten
\begin{align}
	A=-\frac{1}{4}-\im\frac{13}{4\cdot\sqrt{3}}, \nonumber \\
	B=-\frac{1}{4}+\im\frac{13}{4\cdot\sqrt{3}}.
\end{align}
\begin{align}
	X(z)=1+\frac{1}{z}\cdot\frac{(-\frac{1}{4}-\im\frac{13}{4\cdot\sqrt{3}})\cdot z}{z-(\frac{1}{4}+\im\frac{\sqrt{3}}{4})}+\frac{1}{z}\cdot\frac{(-\frac{1}{4}+\im\frac{13}{4\cdot\sqrt{3}})\cdot z}{z-(\frac{1}{4}-\im\frac{\sqrt{3}}{4})}=1+\frac{1}{4z}\cdot\frac{(-1-\im\frac{13}{\sqrt{3}})\cdot z}{z-\frac{1}{2}\e^{+\im\frac{\pi}{3}}}+\frac{1}{4z}\cdot\frac{(-1+\im\frac{13}{\sqrt{3}})\cdot z}{z-\frac{1}{2}\e^{-\im\frac{\pi}{3}}}
\end{align}
%We use the correspondences
Wir nutzen die Korrespondenzen
\begin{align}
	\delta[k]\quad\ztransf\quad1\quad\mathrm{ROC}=\mathbb{C},
\end{align}
\begin{align}
	a^k\epsilon[k]\quad\ztransf\quad\frac{z}{z-a}\quad|z|>1
\end{align}
%and
und
\begin{align}
	x[k-\kappa]\quad\ztransf\quad z^{-\kappa}.
\end{align}
%This leads to
Dies führt zu
\begin{align}
	x[k]&=\delta[k]+\frac{-1-\im\frac{13}{\sqrt{3}}}{2\cdot2}\bigg(\frac{1}{2}\bigg)^{k-1}\e^{+\im\frac{\pi}{3}(k-1)}\epsilon[k-1]+\frac{-1+\im\frac{13}{\sqrt{3}}}{2\cdot2}\bigg(\frac{1}{2}\bigg)^{k-1}\e^{-\im\frac{\pi}{3}(k-1)}\epsilon[k-1] \nonumber\\
	&=\delta[k]+\bigg(\frac{1}{2}\bigg)^{k}\Bigg[-\frac{1}{2}\bigg(\e^{+\im\frac{\pi}{3}(k-1)}+\e^{-\im\frac{\pi}{3}(k-1)}\bigg)+\frac{13}{\sqrt{3}\cdot2\im}\bigg(\e^{+\im\frac{\pi}{3}(k-1)}-\e^{-\im\frac{\pi}{3}(k-1)}\bigg)\Bigg ]\epsilon[k-1]\nonumber\\
	&=\delta[k]+\bigg(\frac{1}{2}\bigg)^k\Bigg[\frac{13}{\sqrt{3}}\sin(\frac{\pi}{3}(k-1))-\cos(\frac{\pi}{3}(k-1))\Bigg]\epsilon[k-1]
\end{align}

%\subsubsection{Task e)}
\subsubsection{Aufgabe e)}
%We shall find the time-discret signal for
Wir wollen das zeit-diskrete Signal für
\begin{align}
	X(z)=\frac{z^2}{z^2+1}
\end{align}
berechnen.
%If we look at the correspondence
Wir werfen einen Blick auf die Korrespondenz
\begin{align}
	\cos [\Omega_0k]\epsilon[k]\quad\ztransf\quad \frac{z^2-z\cos(\Omega_0))}{z^2-2z\cos(\Omega_0)+1}\quad\quad |z| > 1
\end{align}
%we have the left site if we set
und erkennen, dass wir diese erreichen, indem wir
\begin{align}
	\cos(\Omega_0)=0
\end{align}
setzen.
%We get
Wir erhalten
\begin{align}
	\Omega_0 = \frac{\pi}{2} \pm \pi \cdot m \quad m \in \mathbb{Z}
\end{align}
%At the end, our time-discret signal is
und somit unser zeit-diskretes Signal
\begin{align}
	&x[k]=\cos[(\frac{\pi}{2}\pm \pi\cdot m)k]\epsilon[k]\quad m \in \mathbb{Z}\\
	&x[k] = 1\cdot\delta[k] + 0\cdot \delta[k-1] -1\cdot \delta[k-2] + 0\cdot \delta[k-3] + 1\cdot \delta[k-4] + 0\cdot \delta[k-5] -1\cdot \delta[k-6] + 0\cdot \delta[k-7] ...
\end{align}
%\subsubsection{Task f)}
\subsubsection{Aufgabe f)}
%We shall find the time-discret signal for
Wir wollen das zeit-diskrete Signal für
\begin{align}
	X(z)=\frac{z^4+z^3-6z^2+6z-1}{z^2-2z+1}
\end{align}
berechnen.
%We can simplify the denominator to
Wir können den Nenner vereinfachen:
\begin{align}
	(z-1)^2.
\end{align}
%If we take a look at the correspondences
Mit den Korrespondenzen
\begin{align}
	k\epsilon[k]\quad\ztransf\quad\frac{z}{(z-1)^2}\quad |z|>1
\end{align}
%and
und
\begin{align}
	x[k-\kappa]\quad\ztransf\quad z^{-\kappa}X(z)
\end{align}
%we rewrite our function to use these two:
können wir unsere Funktion umformen:
\begin{align}
	X(z)=z^3\cdot\frac{z}{(z-1)^2}+z^2\cdot\frac{z}{(z-1)^2}-6z\cdot\frac{z}{(z-1)^2}+6\frac{z}{(z-1)^2}-z^{-1}\cdot\frac{z}{(z-1)^2}.
\end{align}
%So our time-discret signal is
Somit können wir die Funktion in ein zeit-diskretes Signal transformieren:
\begin{align}
	&X(z)=z^3\cdot\frac{z}{(z-1)^2}+z^2\cdot\frac{z}{(z-1)^2}-6z\cdot\frac{z}{(z-1)^2}+6\frac{z}{(z-1)^2}-z^{-1}\cdot\frac{z}{(z-1)^2} \nonumber\\
%\end{align}
%\begin{align}
	&\Ztransf \nonumber\\
%\end{align}
%\begin{align}
	&x[k]=(k+3)\epsilon[k+3]+(k+2)\epsilon[k+2]-6(k+1)\epsilon[k+1]+6k\epsilon[k]-(k-1)\epsilon[k-1].
\end{align}
%
%
Alternativer Weg wegen Zählergrad $>$ Nennergrad \quad $\Rightarrow$\quad zunächst
Durchdividieren
\begin{align}
X(z) = \frac{z^4 + z^3 -6z^2 + 6z -1}{z^2-2z+1} = z^2 + 3 z - 1 + \frac{z}{(z-1)^2}.
\end{align}%
und mit Korrespondenzen zu
\begin{align}
x[k] = \delta[k+2] +3 \delta[k+1] -\delta[k] +k\epsilon[k]
\end{align}
%
Die beiden Lösungen sollten wir uns mal grafisch veranschaulichen bzw. ineinander
überführen, damit klar wird, dass das identische Lösungen sind. Siehe dazu auch
der Anhang A in Kap. \ref{sec:AnhangA}.

%\subsubsection{Task g)}
\subsubsection{Aufgabe g)}
%We shall find the time-discret signal for
Wir wollen das zeit-diskrete Signal für
\begin{align}
	X(z)=\frac{z\cdot(z-1)}{z^2-\sqrt{2}z+1}
\end{align}
berechnen.
%We can reshape the function:
Wir können die Funktion umformen:
\begin{align}
	X(z)=\frac{z^2-z}{z^2-2z\cos(\frac{\pi}{4})+1}=\frac{z^2-z\cos(\frac{\pi}{4})}{z^2-2z\cos(\frac{\pi}{4})+1}-(1-\frac{\sqrt{2}}{2})\frac{z}{z^2-2z\cos(\frac{\pi}{4})+1}\cdot\frac{\sin(\frac{\pi}{4})}{\sin(\frac{\pi}{4})}.
\end{align}
%Using the correspondeces
Wir nutzen die Korrespondenzen
\begin{align}
	\sin[\Omega_0k]\epsilon[k]\quad\ztransf\quad\frac{z\sin(\Omega_0)}{z^2-2z\cos(\Omega_0)+1}\quad |z| > 1
\end{align}
%and
und
\begin{align}
	\cos[\Omega_0k]\epsilon[k]\quad\ztransf\quad\frac{z^2-z\cos(\Omega_0)}{z^2-2z\cos(\Omega_0)+1}\quad |z| > 1
\end{align}
%leads to
und erhalten
\begin{align}
	X(z)\quad\Ztransf\quad x[k]=\cos[\frac{\pi}{4}k]\epsilon[k]+\sin[\frac{\pi}{4}k]\epsilon[k]-\sqrt{2}\sin[\frac{\pi}{4}k]\epsilon[k].
\end{align}
\newpage
%\subsection{Solution Using Residue Theorem}
\subsection{Lösung mit Hilfe des Residuensatzes}
%If all isolated singularities are surrounded by the ROC, then we can calculate the inverse z-Transformation using the Residue theorem.
Allgemein berechnet sich die inverse z-Transformation durch ein komplexes Kurvenintegral entlang einer geschlossenen Kurve $C$ im Konvergenzbereich (vgl. \cite[S. 192]{UlrichWeber2017} ).
\begin{align}
	x[k]=\mathcal{Z}^{-1}\{X(z)\}=\frac{1}{2\pi\im}\oint_C X(z)z^{k-1}\mathrm{d}z
\end{align}
Das komplexe Kurvenintegral kann mit Hilfe des Residuensatz berechnet werden.\\
Für die Berechnung der Residuen finden sich z.B. in \cite[K. 14, S. 767-768]{Bronstein}, \cite[S. 37-38]{UlrichWeber2017} oder \cite[S. 137-138]{Fritzsche2019}.
%\subsubsection{Task a)}
\subsubsection{Aufgabe a)}
%We shall find the time-discret signal for
Wir wollen das zeit-diskrete Signal für
\begin{align}
	X(z)=2\cdot\frac{z\cdot(z-\frac{1}{\sqrt{2}})}{(z-\e^{+\im\frac{\pi}{4}})\cdot(z-\e^{-\im\frac{\pi}{4}})}
\end{align}
berechnen.
%$\e^{+\im\frac{\pi}{4}}$ and $\e^{-\im\frac{\pi}{4}}$ are poles of order 1.\\
$\e^{+\im\frac{\pi}{4}}$ und $\e^{-\im\frac{\pi}{4}}$ sind Polstellen erster Ordnung.\\
%\begin{align}
%	0\text{ is }
%	\begin{cases}
%	\text{no singularity}, &k > 0, \\
%	\text{removable singularity}, &k = 0, \\
%	\text{pole of order }|k|, &k <0.
%\end{cases}
%\end{align}
\begin{align}
	0\text{ ist }
	\begin{cases}
		\text{keine isolierte Singularität}, &k > 0, \\
		\text{eine hebbare Singularität}, &k = 0, \\
		\text{eine Polstelle }|k|\text{-ter Ordnung}, &k <0.
	\end{cases}
\end{align}
\begin{align}
	\mathrm{Res}(X(z)z^{k-1},\e^{+\im\frac{\pi}{4}})&=\lim\limits_{z\rightarrow\e^{+\im\frac{\pi}{4}}}2\cdot\frac{z\cdot(z-\frac{1}{\sqrt{2}})}{(z-\e^{-\im\frac{\pi}{4}})}z^{k-1}=2\cdot\frac{\e^{+\im\frac{\pi}{4}}\cdot(\e^{+\im\frac{\pi}{4}}-\frac{1}{\sqrt{2}})}{\e^{+\im\frac{\pi}{4}}-\e^{-\im\frac{\pi}{4}}}\e^{+\im\frac{\pi}{4}(k-1)} \nonumber \\
	&=2\cdot\frac{\e^{+\im\frac{\pi}{2}}-\frac{1}{\sqrt{2}}\e^{+\im\frac{\pi}{4}}}{\cos(\frac{\pi}{4})+\im\sin(\frac{\pi}{4})-(\cos(\frac{\pi}{4})-\im\sin(\frac{\pi}{4}))}\e^{+\im\frac{\pi}{4}(k-1)}\nonumber\\
	&=2\cdot\frac{\im-\frac{1}{2}-\frac{1}{2}\im}{2\im\sin(\frac{\pi}{4})}\e^{+\im\frac{\pi}{4}(k-1)}=\frac{-1+\im}{\im\sqrt{2}}\e^{+\im\frac{\pi}{4}(k-1)}
\end{align}
\begin{align}
	\mathrm{Res}(X(z)z^{k-1},\e^{-\im\frac{\pi}{4}})&=\lim\limits_{z\rightarrow\e^{-\im\frac{\pi}{4}}}2\cdot\frac{z\cdot(z-\frac{1}{\sqrt{2}})}{(z-\e^{+\im\frac{\pi}{4}})}z^{k-1}=2\cdot\frac{\e^{-\im\frac{\pi}{4}}\cdot(\e^{-\im\frac{\pi}{4}}-\frac{1}{\sqrt{2}})}{\e^{-\im\frac{\pi}{4}}-\e^{+\im\frac{\pi}{4}}}\e^{-\im\frac{\pi}{4}(k-1)} \nonumber \\
	&=2\cdot\frac{\e^{-\im\frac{\pi}{2}}-\frac{1}{\sqrt{2}}\e^{-\im\frac{\pi}{4}}}{\cos(\frac{\pi}{4})-\im\sin(\frac{\pi}{4})-(\cos(\frac{\pi}{4})+\im\sin(\frac{\pi}{4}))}\e^{-\im\frac{\pi}{4}(k-1)}\nonumber\\
	&=2\cdot\frac{-\im-\frac{1}{2}+\frac{1}{2}\im}{-2\im\sin(\frac{\pi}{4})}\e^{-\im\frac{\pi}{4}(k-1)}=\frac{1+\im}{\im\sqrt{2}}\e^{-\im\frac{\pi}{4}(k-1)}
\end{align}
\begin{align}
	\underset{k=0}{\mathrm{Res}(X(z)z^{k-1},0)}=0
\end{align}
\begin{align}
	\underset{k<0}{\mathrm{Res}(X(z)z^{k-1}),0}&=\lim\limits_{z\rightarrow0}\frac{1}{(|k|-1)!}\frac{\mathrm{d}^{|k|-1}}{\mathrm{d}z^{|k|-1}}\Bigg [2\cdot\frac{z\cdot(z-\frac{1}{\sqrt{2}})}{z^2-\sqrt{2}z+1}\cdot\frac{z^{|k|}}{z^{|k|+1}}\Bigg]\nonumber\\
	&=\lim\limits_{z\rightarrow0}\frac{2}{(|k|-1)!}\frac{\mathrm{d}^{|k|-1}}{\mathrm{d}^{|k|-1}}\Bigg [\frac{z-\frac{1}{\sqrt{2}}}{(z-\e^{+\im\frac{\pi}{4}})\cdot(z-\e^{-\im\frac{\pi}{4}})}\Bigg]
\end{align}
%We use partial fraction decomposition to make the differentiation easier.
Wir nutzen Partialbruchzerlegung, um uns das Ableiten zu vereinfachen.
\begin{align}
	\frac{z-\frac{1}{\sqrt{2}}}{(z-\e^{+\im\frac{\pi}{4}})\cdot(z-\e^{-\im\frac{\pi}{4}})}=\frac{A}{z-\e^{+\im\frac{\pi}{4}}}+\frac{B}{z-\e^{-\im\frac{\pi}{4}}}\quad\Bigg | \quad \cdot(z-\e^{+\im\frac{\pi}{4}})\cdot(z-\e^{-\im\frac{\pi}{4}}) \nonumber\\
	z-\frac{}{\sqrt{2}}=A(z-\e^{-\im\frac{\pi}{4}})+B(z-\e^{-\im\frac{\pi}{4}})=z(A+B)+(-A\e^{-\im\frac{\pi}{4}}-B\e^{+\im\frac{\pi}{4}})
\end{align}
%We get a system of linear equations. In matrix notation:
Wir erhalten ein lineares Gleichungssystem, hier dargestellt in der Matrixnotation:
\begin{align}
	\begin{pmatrix}
		1 & 1 \\
		-\e^{-\im\frac{\pi}{4}} & -\e^{+\im\frac{\pi}{4}}
	\end{pmatrix}
	\begin{pmatrix}
		A \\
		B
	\end{pmatrix}
	=
	\begin{pmatrix}
		1 \\
		-\frac{1}{\sqrt{2}}
	\end{pmatrix}
\end{align}
%With techniques like Gaussian elimination or Cramer's rule we can solute our system of linear equations and achieve
Das lineare Gleichungssystem lässt sich mit Techniken wie dem Gaußschem Eliminationsverfahren oder der Cramerschen Regel lösen und wir erhalten
\begin{align}
	A=\frac{1}{2} \nonumber \\
	B=\frac{1}{2}
\end{align}
\begin{align}
	\underset{k<0}{\mathrm{Res}(X(z)z^{k-1},0)}&=\lim\limits_{z\rightarrow0}\frac{2}{(|k|-1)!}\frac{\mathrm{d}^{|k|-1}}{\mathrm{d}z^{|k|-1}}\Bigg[\frac{1}{2}\frac{1}{z-\e^{+\im\frac{\pi}{4}}}+\frac{1}{2}\frac{1}{z-\e^{-\im\frac{\pi}{4}}}\Bigg]\nonumber\\
	&=\lim\limits_{z\rightarrow0}\frac{2}{(|k|-1)!}\Bigg[\frac{1}{2}\frac{(-1)^{|k|-1}\cdot(|k|-1)!}{(z-\e^{+\im\frac{\pi}{4}})^{|k|}}+\frac{1}{2}\frac{(-1)^{|k|-1}\cdot(|k|-1)!}{(z-\e^{-\im\frac{\pi}{4}})^{|k|}}\Bigg]\nonumber\\
	&=\frac{(-1)^{|k|-1}}{(-1)^{|k|}\e^{+\im\frac{\pi}{4}|k|}}\frac{(-1)^{|k|-1}}{(-1)^{|k|}\e^{-\im\frac{\pi}{4}}|k|}\nonumber\\
	&=-\e^{-\im\frac{\pi}{4}|k|}-\e^{+\im\frac{\pi}{4}|k|}\nonumber\\
	&=-\frac{2}{2}\bigg(\e^{+\im\frac{\pi}{4}|k|}+\e^{-\im\frac{\pi}{4}|k|}\bigg)\nonumber\\
	&=-2\cos(\frac{\pi}{4}|k|)
\end{align}
\begin{align}
	x[k]=\begin{cases}
		x_1[k] &k\geq0 \\
		x_2[k] &k<0
	\end{cases}
\end{align}
\begin{align}
	x_1[k]&=\mathrm{Res}(X(z)z^{k-1},\e^{+\im\frac{\pi}{4}})+\mathrm{Res}(X(z)z^{k-1},\e^{-\im\frac{\pi}{4}})=\frac{-1+\im}{\im\sqrt{2}}\e^{+\im\frac{\pi}{4}(k-1)}+\frac{1+\im}{\im\sqrt{2}}\e^{-\im\frac{\pi}{4}(k-1)}\nonumber\\
	&=\frac{\sqrt{2}}{2}\bigg (\e^{+\im\frac{\pi}{4}(k-1)}+\e^{+\im\frac{\pi}{4}(k-1)}\bigg )-\frac{\sqrt{2}}{2\im} \bigg (\e^{+\im\frac{\pi}{4}(k-1)}+\e^{+\im\frac{\pi}{4}(k-1)}\bigg)\nonumber\\
	&=\sqrt{2}\bigg(\cos(\frac{\pi}{4}(k-1))-\sin(\frac{\pi}{4}(k-1))\bigg)
\end{align}
\begin{align}
	\underset{k<0}{x_2[k]}&=\mathrm{Res}(X(z)z^{k-1},\e^{+\im\frac{\pi}{4}})+\mathrm{Res}(X(z)z^{k-1},\e^{-\im\frac{\pi}{4}})+\mathrm{Res}(X(z)z^{k-1},0)\nonumber\\
	&=x_1[k]+-2\cos(\frac{\pi}{4}|k|) \nonumber\\
	&=\sqrt{2}\bigg(\cos(\frac{\pi}{4}(k-1))-\sin(\frac{\pi}{4}(k-1))\bigg)-2\cos(\frac{\pi}{4}|k|)
\end{align}
%It is (cf.~\cite[Ch. 2, p. 81]{Bronstein}):
Es gelten (vgl. \cite[Kap.2, S. 81]{Bronstein}):
\begin{align}
	\label{eq:SinAlphaPlusMinusBeta}
	\sin(\alpha\pm\beta)=\sin\alpha\cos\beta\pm\cos\alpha\sin\beta
\end{align}
%and
und
\begin{align}
	\label{eq:CosAlphaPlusMinusBeta}
	\cos(\alpha\pm\beta)=\cos\alpha\cos\beta\mp\sin\alpha\sin\beta.
\end{align}
\begin{align}
	\underset{k<0}{x_2[k]}&=\sqrt{2}\bigg(\cos(\frac{\pi}{4}k-\frac{\pi}{4})-\sin(\frac{\pi}{4}k-\frac{\pi}{4})\bigg)-2\cos(\frac{\pi}{4}|k|)\nonumber\\
	&=\sqrt{2}\bigg(\frac{\sqrt{2}}{2}\cos(\frac{\pi}{4}k)+\frac{\sqrt{2}}{2}\sin(\frac{\pi}{4}k)-\frac{\sqrt{2}}{2}\sin(\frac{\pi}{4}k)+\frac{\sqrt{2}}{2}\cos(\frac{\pi}{4}k)\bigg)-2\cos(\frac{\pi}{4}|k|)\nonumber\\
	&=2\cos(\frac{\pi}{4}k)-2\cos(\frac{\pi}{4}|k|)\nonumber \\
	&\underset{k<0}{=}2\cos(\frac{\pi}{4}k)-2\cos(-\frac{\pi}{4}k)\nonumber\\
	&=2\cos(\frac{\pi}{4}k)-2\cos(\frac{\pi}{4}k)=0
\end{align}
\begin{align}
	x[k]=\begin{cases}
		\sqrt{2}\bigg(\cos(\frac{\pi}{4}k-\frac{\pi}{4})-\sin(\frac{\pi}{4}k-\frac{\pi}{4})\bigg)=2\cos(\frac{\pi}{4}k),&k\geq0 \\
		0, &k<0
	\end{cases}
\end{align}
%\subsubsection{Task b)}
\subsubsection{Aufgabe b)}
%We shall find the time-discret signal for
Wir wollen das zeit-diskrete Signal für
\begin{align}
	X(z)=\frac{z}{z-\frac{1}{2}}
\end{align}
berechnen.
%$\frac{1}{2}$ is a pole of order 1. \\
$\frac{1}{2}$ ist eine Polstelle erster Ordnung. \\
%\begin{align}
%	0\text{ is }
%	\begin{cases}
%		\text{no isolated singularity}, &k > 0 \\
%		\text{removable singularity}, &k=0 \\
%		\text{pole of order }|k|, &k < 0.
%	\end{cases}
%\end{align}
\begin{align}
	0\text{ ist }
	\begin{cases}
		\text{keine isolierte Singularität}, &k > 0 \\
		\text{eine hebbare Singularität}, &k=0 \\
		\text{eine Polstelle }|k|\text{-ter Ordnung}, &k < 0.
	\end{cases}
\end{align}
\begin{align}
	\mathrm{Res}(X(z)z^{k-1},\frac{1}{2})=\lim\limits_{z\rightarrow\frac{1}{2}}z\cdot z^{k-1}=\bigg (\frac{1}{2} \bigg)^k
\end{align}
\begin{align}
	\underset{k<0}{\mathrm{Res}(X(z)z^{k-1},0)}&=\lim\limits_{z\rightarrow0}\frac{1}{(|k|-1)!}\frac{\mathrm{d}^{|k|-1}}{\mathrm{d}z^{|k|-1}}\Bigg [\frac{z}{(z-\frac{1}{2})\cdot z^{|k|+1}}\cdot z^{|k|}\Bigg]=\lim\limits_{z\rightarrow0}\frac{1}{(|k|-1)!}\frac{\mathrm{d}^{|k|-1}}{\mathrm{d}z^{|k|-1}}\Bigg [\frac{1}{z-\frac{1}{2}}\Bigg]\nonumber\\
	&=\frac{1}{(|k|-1)!}\frac{(-1)^{|k|-1}\cdot (|k|-1)!}{(-\frac{1}{2})^{|k|}}=-2^{|k|}
\end{align}
\begin{align}
	x[k]=\begin{cases}
		x_1[k], &k\geq 0 \\
		x_2[k], &k<0
	\end{cases}
\end{align}
\begin{align}
	x_1[k]=\mathrm{Res}(X(z)z^{k-1},\frac{1}{2})=\bigg( \frac{1}{2} \bigg )^k
\end{align}
\begin{align}
	\underset{k<0}{x_2[k]}=\mathrm{Res}(X(z)z^{k-1},\frac{1}{2})+\mathrm{Res}(X(z)z^{k-1},0)=\bigg ( \frac{1}{2} \bigg)^k-2^{|k|}=0
\end{align}
\begin{align}
	x[k]=
	\begin{cases}
		\bigg (\frac{1}{2} \bigg)^k, &k\geq0 \\
		0, &k<0
	\end{cases}
\end{align}
%\subsubsection{Task c)}
\subsubsection{Aufgabe c)}
%We shall find the time-discret signal for
Wir wollen das zeit-diskrete Signal für
\begin{align}
	X(z)=\frac{z}{z-\frac{1}{2}}\cdot\frac{z}{z-1}
\end{align}
berechnen.
%$\frac{1}{2}$ and $1$ are poles of order $1$.
$\frac{1}{2}$ und $1$ sind Polstellen erster Ordnung.
%\begin{align}
%	0\text{ is }
%	\begin{cases}
%		\text{no isolated singularity}, &k>0 \\
%		\text{removeable singularity}, &-1\leq k \leq0 \\
%		\text{pole of order }|k|-1, &k < -1.
%	\end{cases}
%\end{align}
\begin{align}
	0\text{ ist }
	\begin{cases}
		\text{keine isolierte Singularität}, &k>0 \\
		\text{eine hebbare Singularität}, &-1\leq k \leq0 \\
		\text{eine Polstelle }(|k|-1)\text{-ter Ordnung}, &k < -1.
	\end{cases}
\end{align}
\begin{align}
	\mathrm{Res}(X(z)z^{k-1},\frac{1}{2})=\lim\limits_{z\rightarrow\frac{1}{2}}\frac{z^2}{z-1}z^{k-1}=\frac{\bigg (\frac{1}{2}\bigg)^{k+1}}{\frac{1}{2}-1}=-\bigg (\frac{1}{2}\bigg)^k
\end{align}
\begin{align}
	\mathrm{Res}(X(z)z^{k-1},1)=\lim\limits_{z\rightarrow1}\frac{z^2}{z-\frac{1}{2}}z^{k-1}=1^{k+1}{1-\frac{1}{2}}=2
\end{align}
\begin{align}
	\underset{k<-1}{\mathrm{Res}(X(z)z^{k-1},0)}=\lim\limits_{z\rightarrow0}\frac{1}{(|k|-2)!}\frac{\mathrm{d}^{|k|-2}}{\mathrm{d}z^{|k|-2}}\Bigg [\frac{z}{z-\frac{1}{2}}\cdot\frac{z}{z-1}\frac{1}{z^{|k|+1}}\cdot z^{|k|-1}\Bigg]=\lim\limits_{z\rightarrow0}\frac{1}{(|k|-2)!}\frac{\mathrm{d}^{|k|-2}}{\mathrm{d}z^{|k|-2}}\Bigg [\frac{1}{(z-1)\cdot(z-\frac{1}{2})}\Bigg]
\end{align}
%We use partial fraction decomposition to make the differentiation easier.
Wir nutzen Partialbruchzerlegung, um uns das Ableiten zu vereinfachen.
\begin{align}
	\frac{1}{(z-\frac{1}{2})\cdot(z-1)}=\frac{A}{z-\frac{1}{2}}+\frac{B}{z-1}\quad \Bigg | \quad \cdot(z-1)\cdot(z-\frac{1}{2}) \nonumber \\
	1 = A(z-1)+B(z-\frac{1}{2})=z(A+B)+(-A-\frac{B}{2})
\end{align}
%We get a system of linear equations. In matrix notation:
Wir erhalten ein lineares Gleichungssystem, hier dargestellt in der Matrixnotation:
\begin{align}
	\begin{pmatrix}
		1 & 1 \\
		-1 & -\frac{1}{2}
	\end{pmatrix}
	\begin{pmatrix}
		A \\
		B
	\end{pmatrix}
	=
	\begin{pmatrix}
		0 \\
		1
	\end{pmatrix}
\end{align}
%With techniques like Gaussian elimination or Cramer's rule we can solute our system of linear equations and achieve
Das lineare Gleichungssystem lässt sich mit Techniken wie dem Gaußschem Eliminationsverfahren oder der Cramerschen Regel lösen und wir erhalten
\begin{align}
	A = -2, \nonumber \\
	B = 2.
\end{align}
\begin{align}
	\underset{k<-1}{\mathrm{Res}(X(z)z^{k-1},0)}&=\lim\limits_{z\rightarrow0}\frac{1}{(|k|-2)!}\cdot2\frac{\mathrm{d}^{|k|-2}}{\mathrm{d}z^{|k|-2}}\Bigg [\frac{-1}{z-\frac{1}{2}}+\frac{1}{z-1}\Bigg]\nonumber \\
	&=\lim\limits_{z\rightarrow0}\frac{2}{(|k|-2)!}\Bigg [-\frac{(-1)^{|k|}\cdot(|k|-2)!}{(z-\frac{1}{2})^{|k|-1}\cdot(|k|-2)!}+\frac{(-1)^{|k|}}{(z-1)^{|k|-1}}\Bigg] \nonumber \\
	&=\frac{2}{(|k|-2)!}\Bigg [-\frac{(-1)^{|k|}(|k|-2)!}{(-\frac{1}{2})^{|k|-1}}+\frac{(-1)^{|k|}(|k|-2)!}{(-1)^{|k|-1}}\Bigg] \nonumber \\
	&=2\cdot \Bigg [2^{|k|-1}-1\Bigg]=2^{|k|}-2
\end{align}
\begin{align}
	x[k]=
	\begin{cases}
		x_1[k], &k\geq-1 \\
		x_2[k], &k-1
	\end{cases}
\end{align}
\begin{align}
	x_1[k]=\mathrm{Res}(X(z)z^{k-1},\frac{1}{2})+\mathrm{Res}(X(z)z^{k-1},1)=2-\bigg (\frac{1}{2}\bigg)^k
\end{align}
\begin{align}
	\underset{k<0}{x_2[k]}=\mathrm{Res}(X(z)z^{k-1},\frac{1}{2})+\mathrm{Res}(X(z)z^{k-1},1)+\mathrm{Res}(X(z)z^{k-1},0)=2-\bigg (\frac{1}{2})^k+2^{|k|}-2=0
\end{align}
%It also is $2-\bigg (\frac{1}{2}\bigg)^{-1}=0$. \\
Es gilt außerdem: $2-\bigg (\frac{1}{2}\bigg)^{-1}=0$.
Zusammenfassung:
\begin{align}
	x[k]=
	\begin{cases}
		2-\bigg (\frac{1}{2}\bigg)^k, &k\geq 0 \\
		0, &k< 0
	\end{cases}
\end{align}
%\subsubsection{Task d)}
\subsubsection{Aufgabe d)}
%We shall find the time-discret signal for
Wir wollen das zeit-diskrete Signal für
\begin{align}
	X(z)=\frac{z^2-z+2}{z^2-\frac{1}{2}z+\frac{1}{4}}
\end{align}
berechnen.
%$\frac{1}{4}+\im\frac{\sqrt{3}}{4}$ and $\frac{1}{4}-\im\frac{\sqrt{3}}{4}$ are poles of order 1.
$\frac{1}{4}+\im\frac{\sqrt{3}}{4}$ und $\frac{1}{4}-\im\frac{\sqrt{3}}{4}$ sind Polstellen erster Ordnung.
%\begin{align}
%	0 \text{ is }\begin{cases}
%		\text{no isolated singularity}, &k>0 \\
%		\text{pole of order } |k|+1, &k\leq 0.
%	\end{cases}
%\end{align}
\begin{align}
	0 \text{ ist }\begin{cases}
		\text{keine Singularität}, &k>0 \\
		\text{eine Polstelle} (|k|+1)\text{-ter Ordnung}, &k\leq 0.
	\end{cases}
\end{align}
\begin{align}
	\mathrm{Res}(X(z)z^{k-1},\frac{1}{4}+\im\frac{\sqrt{3}}{4})&=\lim\limits_{z\rightarrow\frac{1}{4}+\im\frac{\sqrt{3}}{4}}\frac{z^2-z+2}{z-(\frac{1}{4}-\im\frac{\sqrt{3}}{4})}z^{k-1}=\lim\limits_{z\rightarrow\frac{1}{2}\e^{+\im\frac{\pi}{3}}}\frac{z^2-z+2}{z-(\frac{1}{2}\e^{-\im\frac{\pi}{3}})}z^{k-1}\nonumber\\
	&=\frac{\frac{1}{4}\e^{+\im\frac{2\pi}{3}}-\frac{1}{2}\e^{+\im\frac{\pi}{3}}+2}{\frac{1}{2}\e^{+\im\frac{\pi}{3}}-\frac{1}{2}\e^{-\im\frac{\pi}{3}}}\bigg(\frac{1}{2}\bigg )^{k-1}\e^{+\im\frac{\pi}{3}(k-1)}\nonumber\\
	&=\frac{\frac{1}{4}(\cos(\frac{2\pi}{3})+\im\sin(\frac{2\pi}{3}))-\frac{1}{2}(\cos(\frac{\pi}{3})+\im\sin(\frac{\pi}{3}))+2}{\frac{1}{2}(\cos(\frac{\pi}{3})+\im\sin(\frac{\pi}{3})-\cos(\frac{\pi}{3})+\im\sin(\frac{\pi}{3}))}\bigg(\frac{1}{2}\bigg )^{k-1}\e^{+\im\frac{\pi}{3}(k-1)}\nonumber\\
	&=\frac{\frac{1}{4}(-\frac{1}{2}+\im\frac{\sqrt{3}}{2})-\frac{1}{2}(\frac{1}{2}+\im\frac{\sqrt{3}}{2})+2}{\im\frac{\sqrt{3}}{2}}\bigg(\frac{1}{2}\bigg )^{k-1}\e^{+\im\frac{\pi}{3}(k-1)}\nonumber\\
	&=2\cdot\frac{1}{\im\sqrt{3}}\cdot\bigg [-\frac{1}{8}+\im\frac{\sqrt{3}}{8}-\frac{1}{4}-\im\frac{\sqrt{3}}{4}+2\bigg]\bigg(\frac{1}{2}\bigg )^{k-1}\e^{+\im\frac{\pi}{3}(k-1)}\nonumber\\
	&=\frac{1}{\im\sqrt{3}}\cdot\bigg [\frac{13}{4}-\im\frac{\sqrt{3}}{4}\bigg]\bigg(\frac{1}{2}\bigg)^{k-1}\e^{+\im\frac{\pi}{3}(k-1)}
\end{align}
\begin{align}
	\mathrm{Res}(X(z)z^{k-1},\frac{1}{4}-\im\frac{\sqrt{3}}{4})&=\lim\limits_{z\rightarrow\frac{1}{4}-\im\frac{\sqrt{3}}{4}}\frac{z^2-z+2}{z-(\frac{1}{4}+\im\frac{\sqrt{3}}{4})}z^{k-1}=\lim\limits_{z\rightarrow\frac{1}{2}\e^{-\im\frac{\pi}{3}}}\frac{z^2-z+2}{z-(\frac{1}{2}\e^{+\im\frac{\pi}{3}})}z^{k-1}\nonumber\\
	&=\frac{\frac{1}{4}\e^{-\im\frac{2\pi}{3}}-\frac{1}{2}\e^{-\im\frac{\pi}{3}}+2}{\frac{1}{2}\e^{-\im\frac{\pi}{3}}-\frac{1}{2}\e^{+\im\frac{\pi}{3}}}\bigg(\frac{1}{2}\bigg )^{k-1}\e^{-\im\frac{\pi}{3}(k-1)}\nonumber\\
	&=\frac{\frac{1}{4}(\cos(\frac{2\pi}{3})-\im\sin(\frac{2\pi}{3}))-\frac{1}{2}(\cos(\frac{\pi}{3})-\im\sin(\frac{\pi}{3}))+2}{\frac{1}{2}(\cos(\frac{\pi}{3})-\im\sin(\frac{\pi}{3})-\cos(\frac{\pi}{3})-\im\sin(\frac{\pi}{3}))}\bigg(\frac{1}{2}\bigg )^{k-1}\e^{-\im\frac{\pi}{3}(k-1)}\nonumber\\
	&=\frac{\frac{1}{4}(-\frac{1}{2}-\im\frac{\sqrt{3}}{2})-\frac{1}{2}(\frac{1}{2}-\im\frac{\sqrt{3}}{2})+2}{-\im\frac{\sqrt{3}}{2}}\bigg(\frac{1}{2}\bigg )^{k-1}\e^{+\im\frac{\pi}{3}(k-1)}\nonumber\\
	&=2\cdot\frac{1}{\sqrt{3}}\cdot\bigg [\frac{1}{8}+\im\frac{\sqrt{3}}{8}+\frac{1}{4}-\im\frac{\sqrt{3}}{4}-2\bigg]\bigg(\frac{1}{2}\bigg )^{k-1}\e^{-\im\frac{\pi}{3}(k-1)}\nonumber\\
	&=\frac{1}{\im\sqrt{3}}\cdot\bigg [-\frac{13}{4}-\im\frac{\sqrt{3}}{4}\bigg]\bigg(\frac{1}{2}\bigg)^{k-1}\e^{-\im\frac{\pi}{3}(k-1)}
\end{align}
%We use partial fraction decomposition to make the differentiation easier. Take a look at \eqref{eq:TaskDPartialFractionDecomposition} for the solution.
Wir nutzen Partialbruchzerlegung, um uns das Ableiten zu vereinfachen. Wir nutzen die Lösung aus \eqref{eq:TaskDPartialFractionDecomposition}.
\begin{align}
	\underset{k\leq0}{\mathrm{Res}(X(z)z^{k-1},0)}=\lim\limits_{z\rightarrow0}\frac{1}{|k|!}\frac{\mathrm{d}^{|k|}}{\mathrm{d}z^{|k|}} \Bigg [ \frac{z^2-z+2}{z^2-\frac{1}{2}z+\frac{1}{4}}\cdot\frac{z^{|k|+1}}{z^{|k|+1}}\Bigg ]=\lim\limits_{z\rightarrow0}\frac{1}{|k|!}\frac{\mathrm{d}^{|k|}}{\mathrm{d}z^{|k|}}\Bigg [1+\frac{-\frac{1}{2}z+\frac{7}{4}}{z^2-\frac{1}{2}z+\frac{1}{4}} \Bigg]\nonumber\\
\end{align}
\begin{align}
	\underset{k\leq0}{\mathrm{Res}(X(z)z^{k-1},0)}&=\lim\limits_{z\rightarrow0}\frac{1}{|k|!}\frac{\mathrm{d}^{|k|}}{\mathrm{d}z^{|k|}}\Bigg [1+\frac{-\frac{1}{4}-\im\frac{13}{4\cdot\sqrt{3}}}{z-(\frac{1}{4}+\im\frac{\sqrt{3}}{4})}+\frac{-\frac{1}{4}+\im\frac{13}{4\cdot\sqrt{3}}}{z-(\frac{1}{4}-\im\frac{\sqrt{3}}{4})}\Bigg]\nonumber\\
	&=\frac{\mathrm{d}^{|k|}1}{\mathrm{d}z^{|k|}}+\lim\limits_{z\rightarrow0}\frac{1}{|k|}\frac{\mathrm{d}^{|k|}}{\mathrm{d}z^{|k|}}\Bigg[\frac{-\frac{1}{4}-\im\frac{13}{4\cdot\sqrt{3}}}{z-(\frac{1}{4}+\im\frac{\sqrt{3}}{4})}+\frac{-\frac{1}{4}+\im\frac{13}{4\cdot\sqrt{3}}}{z-(\frac{1}{4}-\im\frac{\sqrt{3}}{4})}\Bigg] \nonumber \\
	&=\begin{cases}
		1+\frac{-\frac{1}{4}-\im\frac{13}{4\cdot\sqrt{3}}}{-\frac{1}{4}-\im\frac{\sqrt{3}}{4}}+\frac{-\frac{1}{4}+\im\frac{13}{4\cdot\sqrt{3}}}{-\frac{1}{4}+\im\frac{\sqrt{3}}{4}}, &k = 0 \\
		\frac{-\frac{1}{4}-\im\frac{13}{4\cdot\sqrt{3}}}{(-\frac{1}{4}+\im\frac{\sqrt{3}}{4})^{|k|+1}}\cdot(-1)^{|k|}+\frac{-\frac{1}{4}+\im\frac{13}{4\cdot\sqrt{3}}}{(-\frac{1}{4}-\im\frac{\sqrt{3}}{4})^{|k|+1}}\cdot(-1)^{|k|}, &k <0
	\end{cases}
\end{align}
\begin{align}
		\mathrm{Res}(X(z)z^{-1},0)&= 1+\frac{-\frac{1}{4}-\im\frac{13}{4\cdot\sqrt{3}}}{-\frac{1}{4}-\im\frac{\sqrt{3}}{4}}+\frac{-\frac{1}{4}+\im\frac{13}{4\cdot\sqrt{3}}}{-\frac{1}{4}+\im\frac{\sqrt{3}}{4}}=1+\frac{1+\im\frac{13}{\sqrt{3}}}{1+\im\sqrt{3}}\cdot\frac{1-\im\sqrt{3}}{1-\im\sqrt{3}}+\frac{1-\im\frac{13}{\sqrt{3}}}{1-\im\sqrt{3}}\cdot\frac{1+\im\sqrt{3}}{1+\im\sqrt{3}}\nonumber\\
		&=1+\frac{1-\im\sqrt{3}+13+\im\frac{13}{\sqrt{3}}+1+\im\sqrt{3}-\im\frac{13}{\sqrt{3}}+13}{1+3}=8
\end{align}
\begin{align}
	\underset{k<0}{\mathrm{Res}(X(z)z^{|k|+1},0)}&=4^{|k|}\Bigg[\frac{1+\im\frac{13}{\sqrt{3}}}{(1-\im\sqrt{3})^{|k|+1}}+\frac{1-\im\frac{13}{\sqrt{3}}}{(1+\im\sqrt{3})^{|k|+1}} \Bigg]=4^{|k|}\frac{(1+\im\frac{13}{\sqrt{3}})\cdot2^{|k|+1}\e^{+\im\frac{\pi}{3}(|k|+1)}+(1-\im\frac{13}{\sqrt{3}})2^{|k|+1}\e^{-\im\frac{\pi}{3}(|k|+1)}}{2^{|k|+1}\e^{-\im\frac{\pi}{3}(|k|+1)}2^{|k|+1}\e^{+\im\frac{\pi}{3}(|k|+1)}}\nonumber\\
	&=\frac{2^{|k|}}{4}\Bigg [4\cdot\frac{1}{2}\bigg(\e^{+\im\frac{\pi}{3}(|k|+1)}+\e^{-\im\frac{\pi}{3}(|k|+1)}\bigg)+\frac{52}{\sqrt{3}}\cdot\frac{1}{2\im}\bigg(\e^{+\im\frac{\pi}{3}(|k|+1)}-\e^{-\im\frac{\pi}{3}(|k|+1)}\bigg)\Bigg]\nonumber\\
	&=2^{|k|}\Bigg[\cos(\frac{\pi}{3}(|k|+1))+\frac{13}{\sqrt{3}}\sin(\frac{\pi}{3}(|k|+1))\Bigg]
\end{align}
\begin{align}
	x[k]=\begin{cases}
		x_1[k], &k>0 \\
		x_2[k], &k=0 \\
		x_3[k], &k<0
	\end{cases}
\end{align}
\begin{align}
	x_1[k]&=\mathrm{Res}(X(z)z^{k-1},\frac{1}{4}+\im\frac{\sqrt{3}}{4})+\mathrm{Res}(X(z)z^{k-1},\frac{1}{4}-\im\frac{\sqrt{3}}{4})\nonumber\\
	&=\frac{1}{\im\sqrt{3}}\cdot\bigg [\frac{13}{4}-\im\frac{\sqrt{3}}{4}\bigg]\bigg(\frac{1}{2}\bigg)^{k-1}\e^{+\im\frac{\pi}{3}(k-1)}+\frac{1}{\im\sqrt{3}}\cdot\bigg [-\frac{13}{4}-\im\frac{\sqrt{3}}{4}\bigg]\bigg(\frac{1}{2}\bigg)^{k-1}\e^{-\im\frac{\pi}{3}(k-1)}\nonumber\\
	&=\bigg(\frac{1}{2}\bigg)^{k-1}\Bigg[-\frac{1}{2\cdot2}\bigg(\e^{+\im\frac{\pi}{3}(k-1)}+\e^{-\im\frac{\pi}{3}(k-1)}\bigg)+\frac{13}{\im\sqrt{3}\cdot2\cdot2}\bigg(\e^{+\im\frac{\pi}{3}(k-1)}-\e^{-\im\frac{\pi}{3}(k-1)}\bigg)\Bigg]\nonumber\\
	&=\bigg(\frac{1}{2}\bigg)^{k}\Bigg[\frac{13}{\sqrt{3}}\sin(\frac{\pi}{3}(k-1))-\cos(\frac{\pi}{3}(k-1))\Bigg]
\end{align}
\begin{align}
	x_2[k]&=\mathrm{Res}(X(z)z^{-1},0)+\mathrm{Res}(X(z)z^{k-1},\frac{1}{4}+\im\frac{\sqrt{3}}{4})\Bigg |_{k=0}+\mathrm{Res}(X(z)z^{k-1},\frac{1}{4}-\im\frac{\sqrt{3}}{4})\Bigg |_{k=0}\nonumber\\
	&=8+x_1[k=0]=8+\frac{13}{\sqrt{3}}\sin(-\frac{\pi}{3})-\cos(-\frac{\pi}{3})=8+\frac{13}{\sqrt{3}}\cdot\frac{-\sqrt{3}}{2}-\frac{1}{2}=8-\frac{13}{2}-\frac{1}{2}\nonumber\\
	&=8-\frac{14}{2}=8-7=1
\end{align}
\begin{align}
	\underset{k<0}{x_3[k]}&=\mathrm{Res}(X(z)z^{|k|+1},0)+\mathrm{Res}(X(z)z^{k-1},\frac{1}{4}+\im\frac{\sqrt{3}}{4})+\mathrm{Res}(X(z)z^{k-1},\frac{1}{4}-\im\frac{\sqrt{3}}{4})\nonumber\\
	&=x_1[k]+2^{|k|}\Bigg[\cos(\frac{\pi}{3}(|k|+1))+\frac{13}{\sqrt{3}}\sin(\frac{\pi}{3}(|k|+1))\Bigg]\nonumber\\
	&=2^{|k|}\Bigg[\cos(\frac{\pi}{3}(|k|+1))+\frac{13}{\sqrt{3}}\sin(\frac{\pi}{3}(|k|+1))\Bigg]+\bigg(\frac{1}{2}\bigg)^{k}\Bigg[\frac{13}{\sqrt{3}}\sin(\frac{\pi}{3}(k-1))-\cos(\frac{\pi}{3}(k-1))\Bigg]\nonumber\\
	&\underset{k<0}{=}2^{|k|}\Bigg[\cos(\frac{\pi}{3}(1-k))+\frac{13}{\sqrt{3}}\sin(\frac{\pi}{3}(1-k))+\frac{13}{\sqrt{3}}\sin(\frac{\pi}{3}(k-1))-\cos(\frac{\pi}{3}(k-1))\Bigg]\nonumber\\
	&=2^{|k|}\Bigg[\cos(\frac{\pi}{3}(-(k-1)))+\frac{13}{\sqrt{3}}\sin(\frac{\pi}{3}(-(k-1)))+\frac{13}{\sqrt{3}}\sin(\frac{\pi}{3}(k-1))-\cos(\frac{\pi}{3}(k-1))\Bigg]\nonumber\\
	&=2^{|k|}\Bigg[\cos(\frac{\pi}{3}(k-1))-\frac{13}{\sqrt{3}}\sin(\frac{\pi}{3}(k-1))+\frac{13}{\sqrt{3}}\sin(\frac{\pi}{3}(k-1))-\cos(\frac{\pi}{3}(k-1))\Bigg] \nonumber\\
	&=0
\end{align}
\begin{align}
	x_[k]=\begin{cases}
		\bigg(\frac{1}{2}\bigg)^{k}\Bigg[\frac{13}{\sqrt{3}}\sin(\frac{\pi}{3}(k-1))-\cos(\frac{\pi}{3}(k-1))\Bigg], &k>0 \\
		1, &k=0 \\
		0, &k<0
	\end{cases}
\end{align}
%\subsubsection{Task e)}
\subsubsection{Aufgabe e )}
%We shall find the time-discret signal for
Wir wollen das zeit-diskrete Signal für
\begin{align}
	X(z)=\frac{z^2}{z^2+1}
\end{align}
berechnen.
%$\im$ and $-\im$ are poles of order 1.
$\im$ und $-\im$ sind Polstellen erster Ordnung.
%\begin{align}
%	0 \text{ is }\begin{cases}
%		\text{no isolated singularity}, &k> 0,\\
%		\text{removeable singularity}, &-1\leq k \leq 0,\\
%		\text{pole of order } |k|-1, &k<-1.
%	\end{cases}
%\end{align}
\begin{align}
	0 \text{ ist }\begin{cases}
		\text{keine isolierte Singularität}, &k> 0,\\
		\text{eine hebbare Singularität}, &-1\leq k \leq 0,\\
		\text{eine Polstelle } (|k|-1)\text{-ter Ordnung}, &k<-1.
	\end{cases}
\end{align}
\begin{align}
	\mathrm{Res}(X(z)z^{k-1},\im)=\lim\limits_{z\rightarrow\im}\frac{z^2}{z+\im}z^{k-1}=\lim\limits_{z\rightarrow\e^{+\im\frac{\pi}{2}}}\frac{z^{k+1}}{z+\im}=\frac{1}{2\im}\e^{+\im\frac{\pi}{2}(k+1)}
\end{align}
\begin{align}
	\mathrm{Res}(X(z)z^{k-1},-\im)=\lim\limits_{z\rightarrow-\im}\frac{z^2}{z-\im}z^{k-1}=\lim\limits_{z\rightarrow\e^{-\im\frac{\pi}{2}}}\frac{z^{k+1}}{z-\im}=\frac{1}{-2\im}\e^{-\im\frac{\pi}{2}(k+1)}
\end{align}
\begin{align}
	\underset{k<-1}{\mathrm{Res}(X(z)z^{k-1},0})=\lim\limits_{z\rightarrow0}\frac{1}{(|k|-2)!}\frac{\mathrm{d}^{|k|-2}}{\mathrm{d}z^{|k|-2}}\Bigg [\frac{z^2}{z^2+1} \cdot\frac{z^{|k|-1}}{z^{|k|+1}}\Bigg]=\lim\limits_{z\rightarrow0}\frac{1}{(|k|-2)!}\frac{\mathbb{d}^{|k|-2}}{\mathrm{d}z^{|k|-2}}\Bigg [\frac{1}{z^2+1}\Bigg]
\end{align}
%We us partial fraction decomposition to make the differentiation easier.
Wir nutzen Partialbruchzerlegung, um uns das Ableiten zu vereinfachen.
\begin{align}
	\frac{1}{(z-\im)\cdot(z+\im)}=\frac{A}{z-\im}+\frac{B}{z+\im}\quad\Bigg | \quad \cdot(z-\im)\cdot(z+\im) \nonumber \\
	1=A(z+\im)+B(z-\im)=z(A+B)+(\im A -\im B)
\end{align}
%We get a system of linear equations. In matrix notation:
Wir erhalten ein lineares Gleichungssystem, hier dargestellt in der Matrixnotation:
\begin{align}
	\begin{pmatrix}
		1 & 1 \\
		\im & -\im
	\end{pmatrix}
	\begin{pmatrix}
		A \\
		B
	\end{pmatrix}
	=
	\begin{pmatrix}
		0 \\
		1
	\end{pmatrix}.
\end{align}
%With techniques like Gaussian elimination or Cramer's rule, we can solute our system of linear equations and achieve
Das lineare Gleichungssystem lässt sich mit Techniken wie dem Gaußschem Eliminationsverfahren oder der Cramerschen Regel lösen und wir erhalten
\begin{align}
	A= -\frac{1}{2}\im, \nonumber \\
	B = \frac{1}{2}\im.
\end{align}
\begin{align}
	\underset{k<-1}{\mathrm{Res}(X(z)z^{k-1},0)}&=\lim\limits_{z\rightarrow0}\frac{\im}{2\cdot(|k|-2)!}\frac{\mathrm{d}^{|k|-2}}{\mathrm{d}z^{|k|-2}}\Bigg [\frac{-1}{z-\im}+\frac{1}{z+\im} \Bigg ] \nonumber \\
	&=\lim\limits_{z\rightarrow0}\frac{\im}{2\cdot(|k|-2)!}\Bigg [-\frac{(-1)^{|k|}\cdot(|k|-2)!}{(z-\im)^{|k|-1}}+\frac{(1)^{|k|}}{(z+\im)^{|k|-1}} \Bigg]\nonumber\\
	&=\frac{1}{2}\im \cdot\frac{1}{(|k|-2)!}\Bigg [-\frac{(-1)^{|k|}\cdot(|k|-2)!}{(-\im)^{|k|-1}}+\frac{(-1)^{|k|}\cdot(|k|-2)!}{(\im)^{|k|-1}}\Bigg] \nonumber \\
	&=\frac{1}{2}\im\Bigg[\frac{1}{(\im)^{|k|-1}}+\frac{(-1)^{k}}{(\im)^{|k|-1}}\Bigg ] \nonumber \\
	&=\frac{1}{2}\im\cdot\frac{1+(-1)^{|k|}}{(\im)^{|k|-1}}=\frac{1}{2}\cdot\frac{1+(-1)^{|k|}}{(\im)^{|k|-2}}=-\frac{1}{2}(1+(-1)^{|k|})\im^{|k|}\cdot\im^{-2}=-\frac{1}{2}\cdot(1+(-1)^{|k|})\im^{|k|} \nonumber \\
%	&=\begin{cases}
%		-(-1)^{\frac{|k|}{2}}, &|k|\text{ even} \\
%		0, &|k|\text{ odd}
%	\end{cases}
	&=\begin{cases}
		-(-1)^{\frac{|k|}{2}}, &|k|\text{ gerade} \\
		0, &|k|\text{ ungerade}
	\end{cases}
\end{align}
\begin{align}
	x[k]=\begin{cases}
		x_1[k], &k\geq -1 \\
		x_2[k], &k< -1
	\end{cases}
\end{align}
\begin{align}
	x_1[k]=\mathrm{Res}(X(z)z^{k-1},\im)+\mathrm{Res}(X(z)z^{k-1},-\im)=\frac{1}{2\im}\bigg(\e^{+\im\frac{\pi}{2}(k+1)}-\e^{-\im\frac{\pi}{2}(k+1)}\bigg )=\sin(\frac{\pi}{2}(k+1))
\end{align}
\begin{align}
	\underset{k<-1}{x_2[k]}&=\mathrm{Res}(X(z)z^{k-1},\im)+\mathrm{Res}(X(z)z^{k-1},-\im)+\mathrm{Res}(X(z)z^{k-1},0) \nonumber \\
	&=\begin{cases}
		\sin(\frac{\pi}{2}(k+1))-(-1)^{|k|}, &|k| \text{ even} \\
		\sin(\frac{\pi}{2}(k+1)), &|k| \text{ odd}
	\end{cases}
\end{align}
%If $|k|$ is odd, $\frac{\pi}{2}(k+1)$ a multiple of $\pi$, so that $\sin(\frac{\pi}{2}(k+1))=0$.
%If $k<-1$ is even, $\sin(\frac{\pi}{2}(k+1))=(-1)^{\frac{|k|}{2}}$.\\
%It is:
Falls $|k|$ gerade ist, ist $\frac{\pi}{2}(k+1)$ ein Vielfaches von $\pi$, sodass $\sin(\frac{\pi}{2}(k+1))=0$ ist.
Falls $k<-1$ ungerade ist, ist $\sin(\frac{\pi}{2}(k+1))=(-1)^{\frac{|k|}{2}}$.\\
Somit gilt:
\begin{align}
	x_2[k]=0.
\end{align}
Es ist außerdem $x_1[-1] =\sin(\frac{\pi}{2}(-1+1))=0$. \\
Zusammenfassung:
\begin{align}
	x[k]=\begin{cases}
		\sin(\frac{\pi}{2}(k+1))\underset{\text{\eqref{eq:SinAlphaPlusMinusBeta}}}{=}\cos(\frac{\pi}{2}k), &k\geq 0 \\
		0, &k<0
	\end{cases}.
\end{align}
%\subsubsection{Task f)}
\subsubsection{Aufgabe f)}
Wir wollen das zeit-diskrete Signal für
\begin{align}
	X(z)=\frac{z^4+z^3-6z^2+6z-1}{z^2-2z+1}
\end{align}
berechnen.
%$1$ is a pole of order 2.
$1$ ist eine Polstelle zweiter Ordnung.
%\begin{align}
%	0\text{ is }\begin{cases}
%		\text{no isolated singularity}, &k>0\\
%		\text{pole of order }|k|+1, &k\leq0.
%	\end{cases}
%\end{align}
\begin{align}
	0\text{ ist }\begin{cases}
		\text{keine isolierte Singularität}, &k>0\\
		\text{eine Polstelle } (|k|+1)\text{-ter Ordnung}, &k\leq0.
	\end{cases}
\end{align}
\begin{align}
	\mathrm{Res}(X(z)z^{k-1},1)&=\lim\limits_{z\rightarrow1}\frac{\mathrm{d}}{\mathrm{d}z}\bigg [(z^4+z^3-6z^2+6z-1)z^{k-1}\bigg]\nonumber\\
	&=\lim\limits_{z\rightarrow1}\frac{\mathrm{d}}{\mathrm{d}z}\bigg[z^{k+3}+z^{k+2}-6z^{k+1}+6z^k-z^{k-1}\bigg]\nonumber\\
	&=\lim\limits_{z\rightarrow1}\bigg[(k+3)z^{k+2}+(k+2)z^{k+1}-6(k+1)z^{k}+6kz^{k-1}-(k-1)z^{k-2}\bigg ]\nonumber\\
	&=(k+3)+(k+2)-6(k+1)+6k-(k-1)=k
\end{align}
\begin{align}
	\underset{\leq0}{\mathrm{Res}(X(z)z^{k-1},0)}=\lim\limits_{z\rightarrow0}\frac{1}{(|k|)!}\frac{\mathrm{d}^{|k|}}{\mathrm{d}z^{|k|}}\Bigg[\frac{z^4+z^3-6z^2+6z-1}{(z-1)^2}\cdot\frac{z^{|k|+1}}{z^{|k|+1}}\Bigg]=\lim\limits_{z\rightarrow0}\frac{1}{(|k|)!}\frac{\mathrm{d}^{|k|}}{\mathrm{d}z^{|k|}}\Bigg[z^2+3z-1+\frac{z}{(z-1)^2}\Bigg]
\end{align}
%We use partial fraction decomposition to make the differentiation of the last term easier.
Wir nutzen Partialbruchzerlegung, um uns das Ableiten des letzten Terms zu vereinfachen.
\begin{align}
	\frac{z}{(z-1)\cdot(z-1)}=\frac{A}{z-1}+\frac{B}{(z-1)^2}\quad\Bigg | \quad \cdot(z-1)^2\nonumber\\
	z=A(z-1)+B=z\cdot A+1(-A+B)
\end{align}
%We get a system of linear equations. In matrix notation:
Wir erhalten ein lineares Gleichungssystem, hier dargestellt in der Matrixnotation:
\begin{align}
	\begin{pmatrix}
		1 & 0 \\
		-1 & 1
	\end{pmatrix}
	\begin{pmatrix}
		A \\
		B
	\end{pmatrix}
	=
	\begin{pmatrix}
		1 \\
		0
	\end{pmatrix}
\end{align}
%With techniques like Gaussian elimination or Cramer's rule (or in this case, with a good look), we can solute our system of linear equations and achieve
Das lineare Gleichungssystem lässt sich mit Techniken wie dem Gaußschem Eliminationsverfahren oder der Cramerschen Regel lösen und wir erhalten
\begin{align}
	A = 1, \nonumber \\
	B = 1.
\end{align}
\begin{align}
	\underset{k\leq0}{\mathrm{Res}(X(z)z^{k-1},0)}&=\lim\limits_{z\rightarrow0}\frac{1}{(|k|)!}\frac{\mathrm{d}^{|k|}}{\mathrm{d}z^{|k|}}\Bigg[z^2+3z-1+\frac{1}{z-1}+\frac{1}{(z-1)^2}\Bigg]\nonumber\\
	&=\begin{cases}
		0^2+3\cdot0-1+\frac{1}{0-1}+\frac{1}{(0-1)^2}=-1, &k=0 \\
		2\cdot0+3-\frac{1}{(0-1)^2}-\frac{2}{(0-1)^3}=4,&k=-1 \\
		\frac{1}{2}\bigg(2+\frac{2}{(0-1)^3}+\frac{6}{(0-1)^4}\bigg)=3,&k=-2\\
		\frac{1}{|k|}\bigg(\frac{(-1)^{|k|}\cdot (|k|)!}{(0-1)^{|k|+1}}+\frac{(-1)^{|k|}\cdot(|k|+1)!}{(0-1)^{|k|+2}}\bigg)=-1+(|k|+1)=|k|,&k<-2
	\end{cases}
\end{align}
\begin{align}
	x[k]=\begin{cases}
		x_1[k],&k>0 \\
		x_2[k], &k=0 \\
		x_3[k], &k=-1 \\
		x_4[k], &k=-2 \\
		x_5[k], &k<-2
	\end{cases}
\end{align}
\begin{align}
	x_1[k]=\mathrm{Res}(X(z)z^{k-1},1)=k
\end{align}
\begin{align}
	x_2[k]=\mathrm{Res}(X(z)z^{k-1},1)\Bigg |_{k=0}+\mathrm{Res}(X(z)z^{k-1},0)\Bigg |_{k=0}=x_1[k=0]-1=-1
\end{align}
\begin{align}
	x_3[k]=\mathrm{Res}(X(z)z^{k-1},1)\Bigg |_{k=-1}+\mathrm{Res}(X(z)z^{k-1},0)\Bigg |_{k=-1}=x_1[k=-1]+4=3
\end{align}
\begin{align}
	x_4[k]=\mathrm{Res}(X(z)z^{k-1},1)\Bigg |_{k=-2}+\mathrm{Res}(X(z)z^{k-1},0)\Bigg |_{k=-2}=x_1[k=-2]+3=1
\end{align}
\begin{align}
	x_5[k]\underset{k <-2}{=}\mathrm{Res}(X(z)z^{k-1},1)+\mathrm{Res}(X(z)z^{k-1},0)=k+|k|=-|k|+|k|=0
\end{align}
\begin{align}
	x[k]=\begin{cases}
		k, &k>0 \\
		-1, &k=0 \\
		3, &k=-1 \\
		1, &k=-2 \\
		0, &k<-2
	\end{cases}
\end{align}
%\subsubsection{Task g)}
\subsubsection{Aufgabe g)}
%We shall find the time-discret signal for
Wir wollen das zeit-diskrete Signal für
\begin{align}
	X(z)=\frac{z\cdot(z-1)}{z^2-\sqrt{2}z+1}
\end{align}
berechnen.
%$\frac{\sqrt{2}}{2}+\im\frac{\sqrt{2}}{2}$ and $\frac{\sqrt{2}}{2}-\im\frac{\sqrt{2}}{2}$ are poles of order 1.
$\frac{\sqrt{2}}{2}+\im\frac{\sqrt{2}}{2}$ und $\frac{\sqrt{2}}{2}-\im\frac{\sqrt{2}}{2}$ sind Polstellen erster Ordnung.
%\begin{align}
%	0\text{ is }\begin{cases}
%		\text{no isolated singularity}, &k>0 \\
%		\text{removeable singularity}, &k=0 \\
%		\text{pole of order }|k|, &k<0
%	\end{cases}
%\end{align}
\begin{align}
	0\text{ ist }\begin{cases}
		\text{keine isolierte Singularität}, &k>0 \\
		\text{hebbare Singularität}, &k=0 \\
		\text{eine Polstelle } (|k|)\text{-ter Ordnung}, &k<0
	\end{cases}
\end{align}
\begin{align}
	\mathrm{Res}(X(z)z^{k-1},\frac{\sqrt{2}}{2}+\im\frac{\sqrt{2}}{2})&=\lim\limits_{z\rightarrow\frac{\sqrt{2}}{2}+\im\frac{\sqrt{2}}{2}}\frac{z^2-z}{z-(\frac{\sqrt{2}}{2}-\im\frac{\sqrt{2}}{2})}z^{k-1}=\lim\limits_{z\rightarrow\e^{+\im\frac{\pi}{4}}}\frac{z^2-z}{z-\e^{-\im\frac{\pi}{4}}}z^{k-1}\nonumber\\
	&=\frac{\e^{+\im\frac{\pi}{2}}-\e^{+\im\frac{\pi}{4}}}{\e^{+\im\frac{\pi}{4}}-\e^{-\im\frac{\pi}{4}}}\e^{+\im\frac{\pi}{4}(k-1)}=\frac{\im-\cos(\frac{\pi}{4})-\im\sin(\frac{\pi}{4})}{\cos(\frac{\pi}{4})+\im\sin(\frac{\pi}{4})-\cos(\frac{\pi}{4})+\im\sin(\frac{\pi}{4})}\e^{+\im\frac{\pi}{4}(k-1)}\nonumber\\
	&=\frac{-\frac{\sqrt{2}}{2}+\im(1-\frac{\sqrt{2}}{2})}{\im\sqrt{2}}\e^{+\im\frac{\pi}{4}(k-1)}
\end{align}
\begin{align}
	\mathrm{Res}(X(z)z^{k-1},\frac{\sqrt{2}}{2}-\im\frac{\sqrt{2}}{2})&=\lim\limits_{z\rightarrow\frac{\sqrt{2}}{2}-\im\frac{\sqrt{2}}{2}}\frac{z^2-z}{z-(\frac{\sqrt{2}}{2}+\im\frac{\sqrt{2}}{2})}z^{k-1}=\lim\limits_{z\rightarrow\e^{-\im\frac{\pi}{4}}}\frac{z^2-z}{z-\e^{+\im\frac{\pi}{4}}}z^{k-1}\nonumber\\
	&=\frac{\e^{-\im\frac{\pi}{2}}-\e^{-\im\frac{\pi}{4}}}{\e^{-\im\frac{\pi}{4}}-\e^{+\im\frac{\pi}{4}}}\e^{-\im\frac{\pi}{4}(k-1)}=\frac{-\im-\cos(\frac{\pi}{4})+\im\sin(\frac{\pi}{4})}{\cos(\frac{\pi}{4})-\im\sin(\frac{\pi}{4})-\cos(\frac{\pi}{4})-\im\sin(\frac{\pi}{4})}\e^{+\im\frac{\pi}{4}(k-1)}\nonumber\\
	&=-\frac{-\frac{\sqrt{2}}{2}-\im(1-\frac{\sqrt{2}}{2})}{\im\sqrt{2}}\e^{-\im\frac{\pi}{4}(k-1)}
\end{align}
\begin{align}
	\underset{k<0}{\mathrm{Res}(X(z)z^{k-1},0)}&=\lim\limits_{z\rightarrow0}\frac{1}{(|k|-21!}\frac{\mathrm{d}^{|k|-1}}{\mathrm{d}z^{|k|-1}}\Bigg[\frac{z\cdot(z-1)}{z^2-\sqrt{2}z+1}\cdot\frac{z^{|k|}}{z^{|k|+1}}\Bigg ]\nonumber\\
	&=\lim\limits_{z\rightarrow0}\frac{1}{(|k|-1)!}\frac{\mathrm{d}^{|k|-1}}{\mathrm{d}z^{|k|-1}}\Bigg[\frac{z-1}{(z-(\frac{\sqrt{2}}{2}+\im\frac{\sqrt{2}}{2}))\cdot(z-(\frac{\sqrt{2}}{2}-\im\frac{\sqrt{2}}{2}))}\Bigg ]
\end{align}
%We use partial fraction decomposition to make the differentiation easier.
Wir nutzen Partialbruchzerlegung, um uns das Ableiten zu vereinfachen.
\begin{align}
	\frac{z-1}{(z-(\frac{\sqrt{2}}{2}+\im\frac{\sqrt{2}}{2}))\cdot(z-(\frac{\sqrt{2}}{2}-\im\frac{\sqrt{2}}{2}))}=\frac{A}{(z-(\frac{\sqrt{2}}{2}+\im\frac{\sqrt{2}}{2}))}+\frac{B}{(z-(\frac{\sqrt{2}}{2}-\im\frac{\sqrt{2}}{2}))}\quad\Bigg|\quad\cdot (z-\e^{+\im\frac{\pi}{4}})\cdot(z-\e^{-\im\frac{\pi}{4}} \nonumber
\end{align}
\begin{align}
	z-1&=A(z-(\frac{\sqrt{2}}{2}-\im\frac{\sqrt{2}}{2}))+B(z-(\frac{\sqrt{2}}{2}+\im\frac{\sqrt{2}}{2}))\nonumber \\
	&=z(A+B)+(-A(\frac{\sqrt{2}}{2}-\im\frac{\sqrt{2}}{2})-B(\frac{\sqrt{2}}{2}+\im\frac{\sqrt{2}}{2}))
\end{align}
%We get a system of linear equations. In matrix notation:
Wir erhalten ein lineares Gleichungssystem, hier dargestellt in der Matrixnotation:
\begin{align}
	\begin{pmatrix}
		1 & 1 \\
		-\e^{-\im\frac{\pi}{4}} & -\e^{+\im\frac{\pi}{4}}
	\end{pmatrix}
	\begin{pmatrix}
		A \\
		B
	\end{pmatrix}
	=
	\begin{pmatrix}
		1 \\
		-1
	\end{pmatrix}
\end{align}
%With techniques like Gaussian elimination or Cramer's rule, we can solute our system of linear equations and achieve
Das lineare Gleichungssystem lässt sich mit Techniken wie dem Gaußschem Eliminationsverfahren oder der Cramerschen Regel lösen und wir erhalten
\begin{align}
	A=\frac{-\e^{+\im\frac{\pi}{4}}+1}{-\sqrt{2}\im}=\frac{1}{2}+\frac{1-\sqrt{2}}{2\im}, \nonumber\\
	B=-1+\e^{-\im\frac{\pi}{4}}=\frac{1}{2}+\frac{\sqrt{2}-1}{2\im}.
\end{align}
\begin{align}
	\underset{k<0}{\mathrm{Res}(X(z)z^{k-1},0)}&=\lim\limits_{z\rightarrow0}\frac{1}{(|k|-1)!}\frac{\mathrm{d}^{|k|-1}}{\mathrm{d}z^{|k|-1}}\Bigg[\bigg(\frac{1}{2}+\frac{1-\sqrt{2}}{2\im}\bigg)\frac{1}{z-\e^{+\im\frac{\pi}{4}}}+\bigg(\frac{1}{2}+\frac{\sqrt{2}-1}{2\im}\bigg)\frac{1}{z-\e^{-\im\frac{\pi}{4}}}\Bigg ]\nonumber\\
	&=\lim\limits_{z\rightarrow0}\frac{1}{(|k|-1)!}\Bigg[\bigg( \frac{1}{2}+\frac{1-\sqrt{2}}{2\im}\bigg)\frac{(-1)^{|k|-1}\cdot(|k|-1)!}{(z-\e^{+\im\frac{\pi}{4}})^{|k|}}+\bigg(\frac{1}{2}+\frac{\sqrt{2}-1}{2\im}\bigg)\frac{(-1)^{|k|-1}\cdot(|k|-1)!}{(z-\e^{-\im\frac{\pi}{4}})^{|k|}}\Bigg] \nonumber\\
	&=\bigg(\frac{1}{2}+\frac{1-\sqrt{2}}{2\im}\bigg)\frac{(-1)^{|k|-1}}{(-1)^{|k|}\e^{+\im\frac{\pi}{4}|k|}}\bigg(\frac{1}{2}+\frac{\sqrt{2}-1}{2\im}\bigg)\frac{(-1)^{|k|-1}}{(-1)^{|k|}\e^{-\im\frac{\pi}{4}|k|}}\nonumber\\
	&=-\bigg(\frac{1}{2}+\frac{1-\sqrt{2}}{2\im}\e^{-\im\frac{\pi}{4}(|k|)}-\bigg(\frac{1}{2}+\frac{\sqrt{2}-1}{2\im}\bigg)\e^{+\im\frac{\pi}{4}(|k|)}\nonumber\\
	&=-\frac{1}{2}\bigg(\e^{+\im\frac{\pi}{4}(|k|)}+\e^{-\im\frac{\pi}{4}(|k|)}\bigg)+\frac{1-\sqrt{2}}{2\im}\bigg(\e^{+\im\frac{\pi}{4}(|k|)}-\e^{-\im\frac{\pi}{4}(|k|)}\bigg) \nonumber\\
	&=(1-\sqrt{2})\sin(\frac{\pi}{4}(|k|))-\cos(\frac{\pi}{4}(|k|))
\end{align}
\begin{align}
	x[k]=\begin{cases}
		x_1[k], &k>=0\\
		x_2[k], &k<0
	\end{cases}
\end{align}
\begin{align}
	x_1[k]=&=\mathrm{Res}(X(z)z^{k-1},\frac{\sqrt{2}}{2}+\im\frac{\sqrt{2}}{2})+\mathrm{Res}(X(z)z^{k-1},\frac{\sqrt{2}}{2}-\im\frac{\sqrt{2}}{2})\nonumber\\
	&=\frac{-\frac{\sqrt{2}}{2}+\im(1-\frac{\sqrt{2}}{2})}{\im\sqrt{2}}\e^{+\im\frac{\pi}{4}(k-1)}-\frac{-\frac{\sqrt{2}}{2}-\im(1-\frac{\sqrt{2}}{2})}{\im\sqrt{2}}\e^{-\im\frac{\pi}{4}(k-1)}\nonumber\\
	&=\frac{1-\frac{\sqrt{2}}{2}}{\sqrt{2}}\cdot\frac{2}{2}\bigg[\e^{+\im\frac{\pi}{4}(k-1)}+\e^{-\im\frac{\pi}{4}(k-1)}\bigg ]-\frac{1}{2\im}\bigg[\e^{+\im\frac{\pi}{4}(k-1)}-\e^{-\im\frac{\pi}{4}(k-1)}\bigg]\nonumber\\
	&=(\sqrt{2}-1)\cos(\frac{\pi}{4}(k-1))-\sin(\frac{\pi}{4}(k-1))
\end{align}
\begin{align}
	\underset{k<0}{x_2[k]}&=\mathrm{Res}(X(z)z^{k-1},\frac{\sqrt{2}}{2}+\im\frac{\sqrt{2}}{2})+\mathrm{Res}(X(z)z^{k-1},\frac{\sqrt{2}}{2}-\im\frac{\sqrt{2}}{2})+\mathrm{Res}(X(z)z^{k-1},0)\nonumber\\
	&=x_1[k]+(1-\sqrt{2})\sin(\frac{\pi}{4}(|k|))-\cos(\frac{\pi}{4}(|k|))\nonumber \\
	&\underset{k<0}{=}(\sqrt{2}-1)\cos(\frac{\pi}{4}(k-1))-\sin(\frac{\pi}{4}(k-1))+(1-\sqrt{2})\sin(\frac{-\pi}{4}k)-\cos(\frac{-\pi}{4}k)\nonumber\\
	&=(\sqrt{2}-1)\cos(\frac{\pi}{4}(k-1))-\sin(\frac{\pi}{4}(k-1))-(1-\sqrt{2})\sin(\frac{\pi}{4}k)-\cos(\frac{\pi}{4}k)
\end{align}
%Remember \eqref{eq:SinAlphaPlusMinusBeta} and \ref{eq:CosAlphaPlusMinusBeta}.
Wir erinnern uns an \eqref{eq:SinAlphaPlusMinusBeta} und \eqref{eq:CosAlphaPlusMinusBeta}.
\begin{align}
	\underset{k<0}{x_2[k]}&=(\sqrt{2}-1)\cos(\frac{\pi}{4}k-\frac{\pi}{4}))-\sin(\frac{\pi}{4}k-\frac{\pi}{4})-(1-\sqrt{2})\sin(\frac{\pi}{4}k)-\cos(\frac{\pi}{4}k)\nonumber\\
	&=(\sqrt{2}-1)\bigg[\frac{\sqrt{2}}{2}\cos(\frac{\pi}{4}k)+\frac{\sqrt{2}}{2}\sin(\frac{\pi}{4}k)\bigg]-\bigg[\frac{\sqrt{2}}{2}\sin(\frac{\pi}{4}k)-\frac{\sqrt{2}}{2}\cos(\frac{\pi}{4}k)\bigg]\nonumber\\
	&-(1-\sqrt{2})\sin(\frac{\pi}{4}k)-\cos(\frac{\pi}{4}k)\nonumber\\
	&=\cos(\frac{\pi}{4}k)+\sin(\frac{\pi}{4}k)-\sqrt{2}\sin(\frac{\pi}{4}k))-(1-\sqrt{2})\sin(\frac{\pi}{4}k)-\cos(\frac{\pi}{4}k)\nonumber\\
	&=\cos(\frac{\pi}{4}k)+(1-\sqrt{2})\sin(\frac{\pi}{4}k)-(1-\sqrt{2})\sin(\frac{\pi}{4}k)-\cos(\frac{\pi}{4}k)=0
\end{align}
\begin{align}
	x[k]=\begin{cases}
		(\sqrt{2}-1)\cos(\frac{\pi}{4}(k-1))-\sin(\frac{\pi}{4}(k-1))=\cos(\frac{\pi}{4}k)+(1-\sqrt{2})\sin(\frac{\pi}{4}k), &k\geq0\\
		0, &k<0
	\end{cases}
\end{align}
\subsubsection{Aufgabe h)}
Wir wollen das zeit-diskrete Signal für
\begin{align}
	X(z)=\e^{\frac{1}{z}}
\end{align}
finden. $0$ ist hier eine wesentliche Singularität, weil es eine isolierte Singularität und weder eine hebbare Singularität noch eine Polstelle ist. Wir müssen also die Laurent Reihe bilden und uns den Koeffizient $c_{-1}$ anschauen. \\
Die Exponentialfunktion lässt sich auch als Potenzreihe schreiben (vgl. \cite[Kap. 21, S. 1077]{Bronstein}):
\begin{align}
	\e^{z}=1+z+\frac{z^2}{2!}+\frac{z^3}{3!}+\cdot\cdot\cdot+\frac{z^{n}}{n!}+\cdot\cdot\cdot=\sum_{i=0}^{+\infty}\frac{z^i}{i!}\quad |z| < \infty.
\end{align}
Wenn wir nun $z$ durch $\frac{1}{z}$ substituieren, können wir unsere Funktion als Laurent-Reihe schreiben:
\begin{align}
	X(z)=\sum_{i=0}^{+\infty}\frac{z^{-i}}{i!}=1+z^{-1}+\frac{z^{-2}}{2!}+\frac{z^{-3}}{3!}+\cdot\cdot\cdot+\frac{z^{-n}}{n!}\quad|z|>0.
\end{align}
Das Residuum von $X(z)z^{k-1}$ finden wir nun, indem wir uns für alle $k$ den Koeffizienten $c_{-1}$ der Laurent Reihe, die um $0$ entwickelt wurde, anschauen.
\begin{align}
	X(z)z^{k-1}=\sum_{i=0}^{\infty}\frac{z^k}{z^{i+1}\cdot i!}
\end{align}
Wir betrachten zunächst den Fall, dass $k\geq0$ ist. In diesem Fall ist liegt der Koeffizient $c_{-1}$ an der Stelle, an dem $k=i$ ist, denn
\begin{align}
	\frac{z^k}{z^{i+1}}\underset{k=i}{=}\frac{1}{z}=z^{-1}.
\end{align}
Somit ist
\begin{align}
	\mathrm{Res}(X(z)z^{k-1},0)\underset{k\geq}{=}\frac{1}{k!}
\end{align}
Betrachten wir nun den Fall, dass $k<0$ ist.
\begin{align}
	X(z)z^{k-1}\underset{k<0}{=}\sum_{i=0}^{\infty}\frac{1}{z^{|k|+i+1}\cdot i!}
\end{align}
Damit in der Reihe $c_{-1}\neq 0$ ist, muss folgende Gleichung erfüllt sein:
\begin{align}
	|k|+i+1 = 1.
\end{align}
Also muss $|k| = -i$ erfüllt sein. Weil $|k|$ immer positiv ist und $k<0$, müsste in diesem Fall gelten:
\begin{align}
	k = i.
\end{align}
Damit wäre $i$ aber negativ. $i$ beginnt in unserer Reihe jedoch mit $0$. Somit ist für $k<0$ der Koeffizient $c_{-1}=0$ und somit ist auch $\mathrm{Res}(X(z)z^{k-1},0)\underset{k<0}{=}0$.
Zusammenfassung:
\begin{align}
	x[k]=\mathrm{Res}(X(z)z^{k-1},0)=\begin{cases}
		\frac{1}{k!}, &k\geq0 \\
		0, &k<0
	\end{cases}
\end{align}
\newpage
\subsection{Anhang A}
\label{sec:AnhangA}
Falls die selbst berechneten Lösungen syntaktisch nicht den nachfolgenden Lösungen in \ref{sec:AnhangB} entsprechen sollten, ist das noch kein Grund, dass die Lösung falsch sein muss. Es folgt ein kleines Beispiel, um zu zeigen, dass man schnell auf syntaktisch unterschiedliche Lösungen kommen kann.
Sei
\begin{align}
	X(z) = \frac{z-1}{z-1}.
\end{align}
Ein geübter Mensch erkennt schnell, dass man durch Polynomdivision
\begin{align}
	X(z)= 1
\end{align}
erhält, was dem zeit-diskreten Signal
\begin{align}
	x[k] = \delta[k]
\end{align}
entspricht.
%
Geübter SigSys-Blick auf $X(z)$ verrät uns, dass die einzige Polstelle mit der einzigen Nullstelle kompensiert wird. Das System macht also 'nix', es muss daher als Impulsantwort das neutrale Signal der Faltung, der Dirac Impuls, rauskommen.
%

Mit Blick auf den Nenner könnte man jedoch auch versuchen, die z-Transformierte auf die Korrespondenz
\begin{align}
	\epsilon[k]\quad\ztransf\quad\frac{z}{z-1}\quad |z| > 1
\end{align}
zurückzuführen:
\begin{align}
	X(z)=\frac{z-1}{z-1}=\frac{z}{z-1}-\frac{1}{z}\cdot\frac{z}{z-1}.
\end{align}
Wenn nun noch die Korrespondenz
\begin{align}
	x[k-\kappa]\quad\ztransf\quad\frac{1}{z^{\kappa}}X(z)
\end{align}
mit einbezogen wird, erhält man
\begin{align}
	x[k]=\epsilon[k]-\epsilon[k-1].
\end{align}
Die Lösungen sehen zunächst sehr unterschiedlich aus, jedoch erkennt man bei genauerem Hinsehen, dass nur für $k=0$ das Signal $1$ ist, während sich für $k>0$ die Sprungfunktionen aufheben und für $k<0$ beide noch $0$ sind.

Wir kennen den Zusammenhang zwischen Dirac Impuls und Sprungfunktion als
\begin{align}
&\epsilon[k] = \sum_{\kappa=-\infty}^{k} \delta[\kappa]\\
&\delta[k]=\epsilon[k]-\epsilon[k-1],
\end{align}
bzw. im zeit-kontinuierlichen
\begin{align}
&\epsilon(t) = \int_{\tau=-\infty}^{t} \delta(\tau) \mathrm{d}\tau\\
&\delta(t)=\frac{\mathrm{d} \epsilon(t)}{\mathrm{d} t}.
\end{align}
%
Das können wir hier benutzen, um beide Lösungen ineinander zu überführen.


\newpage
\subsection{Anhang B: Lösungstabelle}
\label{sec:AnhangB}
\begin{center}
	\begin{tabular}{|l|l|}
		\hline
		$\mathbf{X(z)}$ & $\mathbf{x[k]}$ \\
		\hline
		$\frac{z\cdot(z-\frac{1}{\sqrt{2}})}{(z-\e^{+\im\frac{\pi}{4}})\cdot(z-\e^{-\im\frac{\pi}{4}})}$ & $2\cos[\frac{\pi}{4}k]\epsilon[k]$ \\
		\hline
		$\frac{z}{z-\frac{1}{2}}$ & $\bigg (\frac{1}{2} \bigg)^{k}\epsilon[k]$ \\
		\hline
		$\frac{z}{z-\frac{1}{2}}\cdot\frac{z}{z-1} $ & $2-\bigg(\frac{1}{2}\bigg)^k\epsilon[k] $\\[2ex]
		\hline
		$\frac{z^2-z+2}{z^2-\frac{1}{2}z+\frac{1}{4}}$ & $\delta[k]+\bigg(\frac{1}{2}\bigg)^k\Bigg[\frac{13}{\sqrt{3}}\sin[\frac{\pi}{3}(k-1)]-\cos[\frac{\pi}{3}(k-1)]\Bigg]\epsilon[k-1]$ \\
		\hline
		$\frac{z^2}{z^2+1}$ & $\cos[\frac{\pi}{2}k]\epsilon[k-1]$\\
		\hline
		$\frac{z^4+z^3-6z^2+6z-1}{z^2-2z+1}$ & $\delta[k+2]+3\delta[k+1]-\delta[k]+k\epsilon[k-1]$\\
		\hline
		$\frac{z\cdot(z-1)}{z^2-\sqrt{2}z+1}$ & $\Bigg[\cos[\frac{\pi}{4}k]+(1-\sqrt{2})\sin[\frac{\pi}{4}k]\Bigg] \epsilon[k]$\\
		\hline
		$\e^{\frac{1}{z}}$ & $\frac{1}{k!}\epsilon[k]$ \\
		\hline
	\end{tabular}
\end{center}


\subsection{Anhang C: Ausgesuchte Korrespondenzen}
\label{sec:AnhangC}

\noindent $\delta[k] \laplace 1$ für $z \in \mathbb{C}$

\noindent $\epsilon[k] \laplace \frac{z}{z-1}$  für $|z| > 1$

\noindent $a^k \, \epsilon[k] \laplace \frac{z}{z-a}$ für $|z| > |a|$

\noindent $-a^k \, \epsilon[-k-1] \laplace \frac{z}{z-a}$ für $|z| < |a|$

\noindent $k \, \epsilon[k] \laplace \frac{z}{(z-1)^2}$ für $|z| > 1$

\noindent $k \, a^k \, \epsilon[k] \laplace \frac{a \, z}{(z-a)^2}$  für $|z| > |a|$

\noindent $\sin[\Omega_0 k] \, \epsilon[k] \laplace \frac{\quad\, z\sin(\Omega_0)}{z^2 -2z \cos(\Omega_0) +1}$ für $|z| > 1$

\noindent $\cos[\Omega_0 k] \, \epsilon[k] \laplace \frac{z^2-z\cos(\Omega_0)}{z^2 -2z \cos(\Omega_0) +1}$ für $|z| > 1$


%\section*{Acknowledgement}
%Thanks to Robert Hauser (https://github.com/robhau) for adding this task.

%------------------------------------------------------------------------------
%\clearpage
\section{Appendix A: Dirac Impuls: Austast-/Ausblendeigenschaft und Multiplikationseigenschaft}
%
\subsection{Zeitkontinuierlich}
Definition, kein eigentliches Riemannintegral:
\begin{mdframed}
Austasteigenschaft/Ausblendeigenschaft \circled{1}, englisch: sifting property (nicht: shifting)
\begin{align}
\int\limits_{t=-\infty}^{+\infty} x(t) \, \delta(t-\tau) \, \fsd t \stackrel{\mathrm{def}}= x(t=\tau)
\label{eq:AppA_SifitingCT}
\end{align}
\end{mdframed}
Speziell für $\tau=0$ folgt
\begin{align}
\int\limits_{t=-\infty}^{+\infty} x(t) \, \delta(t) \, \fsd t \stackrel{\mathrm{def}}= x(t=0),
\end{align}
und für $x(t)=1$ und $\tau=0$ folgt
\begin{align}
\int\limits_{t=-\infty}^{+\infty} \delta(t) \, \fsd t \stackrel{\mathrm{def}}= 1.
\end{align}
Aus der obigen Definition finden wir, dass
\begin{align}
\int\limits_{t=-\infty}^{+\infty} x(t) \, \delta(t-\tau) \, \fsd t \stackrel{\mathrm{def}}= x(\tau)\qquad
\int\limits_{t=-\infty}^{+\infty} x(\tau) \, \delta(t-\tau) \, \fsd t \stackrel{\mathrm{def}}= x(\tau)
\end{align}
zum gleichen Ergebnis führen, die Ausdrücke innerhalb des 'Integrals' bzgl.
der Definition also das gleiche machen, damit finden wir durch Vergleich der Integranden beider Integrale die
\begin{mdframed}
Multiplikationseigenschaft
\begin{align}
x(t) \, \delta(t-\tau) = x(\tau) \, \delta(t-\tau)
\end{align}
\end{mdframed}
Speziell für $\tau=0$ folgt
\begin{align}
x(t) \, \delta(t) = x(0) \, \delta(t).
\end{align}
%
Beispiel:
Austasteigenschaft
\begin{align}
\int\limits_{t=-\infty}^{+\infty} \cos(\frac{\pi}{4} t) \, \delta(t-\tau) \, \fsd t \stackrel{\mathrm{def}}=\cos(\frac{\pi}{4} \tau)
\end{align}
Multiplikationseigenschaft
\begin{align}
\cos(\frac{\pi}{4} t) \, \delta(t-\tau) = \cos(\frac{\pi}{4} \tau) \, \delta(t-\tau)
\end{align}
Für spezielles $\tau=4$ Austasteigenschaft:
\begin{align}
\int\limits_{t=-\infty}^{+\infty} \cos(\frac{\pi}{4} t) \, \delta(t-4) \, \fsd t \stackrel{\mathrm{def}}=\cos(\pi) = -1
\end{align}
%
Für spezielles $\tau=4$ Multiplikationseigenschaft:
\begin{align}
\cos(\frac{\pi}{4} t) \, \delta(t-4) = \cos(\pi) \, \delta(t-4) = -\delta(t-4)
\end{align}

\begin{mdframed}
Von der Austasteigenschaft \circled{1} zur Faltung:
\begin{align}
&\int\limits_{\tau=-\infty}^{+\infty} x(\tau) \, \delta(+(\tau-t)) \, \fsd \tau \stackrel{\mathrm{def}}= x(\tau=t)\text{ , ausgehend von \eq{eq:AppA_SifitingCT} mit Variablentausch } t \leftrightarrow \tau\nonumber\\
&\int\limits_{\tau=-\infty}^{+\infty} x(\tau) \, \delta(-(\tau-t)) \, \fsd \tau \stackrel{\mathrm{def}}= x(\tau=t)\text{ , weil gerade Funktion }\delta(t)=\delta(-t)\nonumber\\
&\int\limits_{\tau=-\infty}^{+\infty} x(\tau) \, \delta(-\tau+t) \, \fsd \tau \stackrel{\mathrm{def}}= x(t) \text{ , Austasteigenschaft \circled{2}=Faltung mit Neutralelement, vgl. \textbf{Formelsammlung}}\nonumber\\
&\int\limits_{\tau=-\infty}^{+\infty} x(\tau) \, h(-\tau+t) \, \fsd \tau = y(t) \text{ , Faltungsintegral mit LTI-System Impulsantwort } h(t)
\end{align}
\end{mdframed}



%
\newpage
\subsection{Zeitdiskret}
Dirac Impuls (Folge) diesmal als exaktes Signal definiert als
\begin{align}
\delta[k] =
\begin{cases}
1 & k=0\\
0 & k \neq 0
\end{cases}
\end{align}
und damit sind 'tatsächliche' Berechnungen möglich.
\begin{mdframed}
Austasteigenschaft/Ausblendeigenschaft \circled{1}, englisch: sifting property (nicht: shifting)
\begin{align}
\sum\limits_{k=-\infty}^{+\infty} x[k] \, \delta[k-\kappa] = x[k=\kappa]
\label{eq:AppA_SifitingDT}
\end{align}
\end{mdframed}
Speziell für $\kappa=0$ folgt
\begin{align}
\sum\limits_{k=-\infty}^{+\infty} x[k] \, \delta[k] = x[k=0],
\end{align}
und für $x[k]=1$ und $\kappa=0$ folgt (hier sofort einzusehen)
\begin{align}
\sum\limits_{\kappa=-\infty}^{+\infty} \delta[k] = 1.
\end{align}
Aus der obigen Definition finden wir, dass
\begin{align}
\sum\limits_{k=-\infty}^{+\infty} x[k] \, \delta[k-\kappa] = x[\kappa]\qquad
\sum\limits_{k=-\infty}^{+\infty} x[\kappa] \, \delta[k-\kappa] = x[\kappa]
%\int\limits_{-\infty}^{+\infty} f(t) \, \delta(t-\tau) \, \fsd t \stackrel{\mathrm{def}}= f(\tau)\qquad
%\int\limits_{-\infty}^{+\infty} f(\tau) \, \delta(t-\tau) \, \fsd t \stackrel{\mathrm{def}}= f(\tau)
\end{align}
zum gleichen Ergebnis führen, die Ausdrücke innerhalb der Summe also das gleiche
machen, damit finden wir die
\begin{mdframed}
Multiplikationseigenschaft
\begin{align}
x[k] \, \delta[k-\kappa] = x[\kappa] \, \delta[k-\kappa]
\end{align}
\end{mdframed}
Speziell für $\kappa=0$ folgt
\begin{align}
x[k] \, \delta[k] = x[0] \, \delta[k].
\end{align}
%
Beispiel:
\begin{align}
\sum\limits_{k=-\infty}^{+\infty} \cos(\frac{\pi}{4} k) \, \delta[k-\kappa] =\cos(\frac{\pi}{4} \kappa)\\
\cos(\frac{\pi}{4} k) \, \delta[k-\kappa] = \cos(\frac{\pi}{4} \kappa) \, \delta[k-\kappa]
\end{align}
für spezielles $\kappa=4$
\begin{align}
\sum\limits_{k=-\infty}^{+\infty} \cos(\frac{\pi}{4} k) \, \delta[k-4] =\cos(\pi) = -1\\
\cos(\frac{\pi}{4} k) \, \delta[k-4] = \cos(\pi) \, \delta[k-4] = -\delta[k-4]
\end{align}

\begin{mdframed}
Von der Austasteigenschaft \circled{1} zur zeitdiskreten Faltung:
\begin{align}
&\sum\limits_{\kappa=-\infty}^{+\infty} x[\kappa] \, \delta[\kappa-k] = x[\kappa=k]\text{ , ausgehend von \eq{eq:AppA_SifitingDT} mit } k \leftrightarrow \kappa\nonumber\\
&\sum\limits_{\kappa=-\infty}^{+\infty} x[\kappa] \, \delta[-(\kappa-k)] = x[\kappa=k]\text{ , weil }\delta[k]=\delta[-k]\nonumber\\
&\sum\limits_{\kappa=-\infty}^{+\infty} x[\kappa] \, \delta[-\kappa+k] = x[k] \text{ , Austasteigenschaft \circled{2}=Faltung mit Neutralelement, vgl. \textbf{Formelsammlung}}\nonumber\\
&\sum\limits_{\kappa=-\infty}^{+\infty} x[\kappa] \, h[-\kappa+k] = y[k] \text{ , Faltungsintegral mit LTI-System Impulsantwort } h[k]
\end{align}
\end{mdframed}

%\clearpage
\section{Appendix B: Basics Elektrotechnik (ET)}
%
Dieser Appendix ersetzt auf keinen Fall ein mehrsemestriges intensives Studieren
der ET, vielleicht hilft er aber sich schneller zurechtzufinden und die Links
zur SigSys zu erkennen.
%
Sehr hilfreiches Buch für Einsteiger*innen in ET ist das Buch
\cite{Marinescu2020}, im Uni Netz als freies E-Book unter

\url{https://link.springer.com/book/10.1007/978-3-658-28884-6}

\noindent Wenn wir bisher wirklich gar nichts mit ET zu tun hatten und uns
der Link zur SigSys interessiert, helfen vielleicht die Basics
in den \cite[Kapitel 1-3]{Marinescu2020}
\begin{itemize}
  \item Ohmsches Gesetz
  \item Kirchhoffsche Gleichungen/Regeln, also Maschenumlauf ergibt Spannung Null,
  Summe alle Ströme rein/raus per Knoten muss Null sein
  \item für Widerstand: Reihenschaltung (R1+R2) mit gleichem Strom,
  Parallelschaltung (1/R1+1/R2) mit gleicher Spannung
\end{itemize}
Dann in \cite[Kapitel 9]{Marinescu2020}
\begin{itemize}
\item Konzept Induktivität u = L di/dt, ideales Bauelement Spule
\item Konzept Kapazität i = C du/dt, ideales Bauelement Kondensator
\item Sinusschwingung, Effektivwert
\item Phasenunterschied zwischen Strom/Spannung bei Widerstand, Spule, Kondensator
\item Kapitel 9.7 / 9.8 beinhaltet Rechnerei an
einfachen R,L,C Schaltungen mit DGL-Ansätzen.
\end{itemize}
Danach vielleicht hilfreich \cite[Kap. 10.2 und 10.3]{Marinescu2020}, d.h.
die komplexe Wechselstromtechnik an R,L,C-Netzwerken, hier steht
der Zusammenhang zwischen Strom und Spannung im Vordergrund.
In \cite[Kap. 13]{Marinescu2020} schaut man sich dann den Zusammenhang zwischen
wechselförmigen Ausgangs- und Eingangsspannung an, während
\cite[Kap. 15]{Marinescu2020} auf Schaltvorgänge eingeht.

Im Wesen schauen wir uns in ET an, wie sich Gleich- oder Wechselgrößen
an elektrischen Netzwerken verhalten, entweder im stationären Zustand
oder speziell den Ein-/Ausschaltvorgang von Gleich-/Wechselgrößen.
%
Dies sind in der Welt der SigSys Spezialfälle für die Analyse des LTI-Systems
'elektrisches Netzwerk mit idealen, konzentrierten Bauelementen'.
%
Im Folgenden ein Versuch die Essenz kompakt zusammenzufassen.

\subsection{Strom i(t) / Spannung u(t) an passiven Bauelementen}

vgl. \cite[Kap. 10.2.2]{Marinescu2020}, Integrationskonstanten Null, d.h. ohne Anfangsbedingungen (also keine Energie gespeichert)

\begin{align}
\text{Widerstand} \qquad & u(t) = R i(t) \qquad& i(t) = \frac{1}{R} u(t)\\
\text{Spule} \qquad & u(t) = L \frac{\fsd i(t)}{\fsd t} \qquad& i(t) = \frac{1}{L} \int u(t) \fsd t\\
\text{Kondensator} \qquad & u(t) = \frac{1}{C} \int i(t) \fsd t \qquad& i(t) = C \frac{\fsd u(t)}{\fsd t}
\end{align}

\subsection{Symbolische Rechnung mit komplexen Zeigern für harmonische Wechselspannung}

vgl. \cite[Kap. 10.3]{Marinescu2020}

Wir machen den Ansatz für komplexwertige, harmonische Signale
\begin{align}
u(t) = \sqrt{2} \, U_\text{eff} \cdot \e^{\im(\omega t + \phi_u)}\\
i(t) = \sqrt{2} \, I_\text{eff} \cdot \e^{\im(\omega t + \phi_i)},
\end{align}
mit den Effektivwerten $U_\text{eff}$, $I_\text{eff}$ und Phasenoffsetwinkeln
$\phi_u$, $\phi_i$.
%
Die obigen Bauteile können in der Praxis nur reellwertige Signale
verarbeiten, daher müssen wir uns entscheiden, ob wir mit dem Realteil oder dem
Imaginärteil von $u(t)$ und $i(t)$ Rückschlüsse zur tatsächlich stattfindenden
Physik ziehen wollen.
Im Grunde ist das Geschmacksfrage, wir entscheiden uns hier 'ja nur', ob wir
physikalisch mit dem Cosinus oder dem Sinus operieren, aber diese Entscheidung
muss dann konsequent beibehalten werden.

Falls wir nun nur eine einzige Kreisfrequenz $\omega$, also z.B. die
EU-typische Stromnetzfrequenz $f=50$ Hz $\rightarrow$ $\omega = 2\pi \cdot 50$ rad/s
in einem elektrischen Netzwerk betrachten, lässt sich die Schreibweise vereinfachen,
wenn wir die Terme $\sqrt{2}$ und $\e^{\im \omega t}$ im Kopf zwar berücksichtigen,
aber eben nicht mehr explizit notieren, weil wir wissen, dass es mit $\omega$
schwingt und wir mit Effektivwerten operieren.
Wir schreiben also kurz die symbolischen Zeiger (sehr gängig ist ein Unterstrich
unter einem Großbuchstaben um das klar zu machen) für Strom und Spannung
\begin{align}
\underline{U} = U_\text{eff} \cdot \e^{\im \phi_u}\\
\underline{I} = I_\text{eff} \cdot \e^{\im \phi_i},
\end{align}
meinen damit aber den vollständigen obigen Ansatz und beachten, ob wir der
$\Re$-Teil oder der $\Im$-Teil Konvention folgen wollen.
%
Diese Zeiger werden oft auch als ruhende Effektivwertzeiger bezeichnet.
%
Die sogenannte komplexe Impedanz $\underline{Z}$ ist dann definiert als
\begin{align}
\underline{Z} =
\frac{\underline{U}}{\underline{I}} =
\frac{U_\text{eff} \cdot \e^{\im \phi_u}}{I_\text{eff} \cdot \e^{\im \phi_i}} =
\frac{U_\text{eff}}{I_\text{eff}} \e^{\im (\phi_u-\phi_i)};
\end{align}
es ist konsistent mit dem
obigen vollständigen Ansatz, weil sich $\sqrt{2}$ und $\e^{\im\omega t}$ kürzen.

\subsection{Komplexe Impedanzen für passive Bauelemente}

vgl. \cite[Kap. 10.3.5]{Marinescu2020}

Wenn wir für den Zusammenhang Strom/Spannung am Kondensator
$u(t) = \frac{1}{C} \int i(t) \fsd t$
die Ansätze
\begin{align}
u(t) = \sqrt{2} \, U_\text{eff} \cdot \e^{\im(\omega t + \phi_u)}\\
i(t) = \sqrt{2} \, I_\text{eff} \cdot \e^{\im(\omega t + \phi_i)}
\end{align}
einsetzen, erhalten wir zunächst
\begin{align}
\sqrt{2} \, U_\text{eff} \cdot \e^{\im(\omega t + \phi_u)} = \frac{1}{C} \int \sqrt{2} \, I_\text{eff} \cdot \e^{\im(\omega t + \phi_i)} \fsd t
\end{align}
und integriert (Hinweis: wir betrachten harmonische, also eingeschwungene Vorgänge, daher auch uneigentliche Integrale, also Integrale ohne explizite Grenzen)
\begin{align}
\sqrt{2} \, U_\text{eff} \cdot \e^{\im(\omega t + \phi_u)} = \frac{1}{\im \omega C} \sqrt{2} \, I_\text{eff} \cdot \e^{\im(\omega t + \phi_i)}
\end{align}
und zur komplexen Impedanz für den Kondensator umgestellt
\begin{align}
\label{eq:appb_ZC}
\underline{Z} = \frac{\underline{U}}{\underline{I}} = \frac{U_\text{eff}}{I_\text{eff}} \e^{\im (\phi_u-\phi_i)} = \frac{1}{\im \omega C} = \frac{1}{\omega C} \cdot \e^{\im \frac{-\pi}{2}}
\end{align}
%
Für die Spule führt der gleiche Ansatz (diesmal ist statt Interal eine Ableitung nach der Zeit zu rechnen)
zur komplexen Impedanz
\begin{align}
\underline{Z} = \frac{\underline{U}}{\underline{I}} = \frac{U_\text{eff}}{I_\text{eff}} \e^{\im (\phi_u-\phi_i)} = \im \omega L = \omega L \cdot \e^{\im \frac{+\pi}{2}}
\end{align}
%
Für den Widerstand erhält man die komplexe Impedanz
\begin{align}
\underline{Z} = \frac{\underline{U}}{\underline{I}} = \frac{U_\text{eff}}{I_\text{eff}} \e^{\im (\phi_u-\phi_i)} = R.
\end{align}
%
Wir sehen, dass die komplexe Impedanz des Widerstands eigentlich gar nicht komplexwertig ist.
Zudem ist sie offensichtlich nicht von $\omega$ abhängig, d.h. ein (idealer) Widerstand
verhält sich bzgl. direkter Proportionalität zwischen Strom und Spannung frequenz\underline{un}abhängig.
%
Für die Spule erkennen wir ein frequenzabhängiges Verhalten zwischen Strom und Spannung, zudem
ist der Phasenwinkel zwischen Spannung und der Referenzgröße Strom $\frac{\pi}{2} \rightarrow 90^\circ$, man sagt,
dass die Spannung dem Strom $90^\circ$ voraus ist.
%
Für den Konsensator erkennen wir auch frequenzabhängiges Verhalten zwischen Strom und Spannung.
Der Phasenwinkel ist diesmal $-\frac{\pi}{2}$, die Spannung ist dem Strom um $90^\circ$ hinterher.



\subsection{Beispiel Wechselspannung am einfachen RC-Glied}

Für das abgebildete RC-Glied sollen zwei Fälle diskutiert werden
\begin{itemize}
  \item 1. Zusammenhang zwischen Eingangsspannung $u_e(t)$ und Strom $i(t)$
  \item 2. Zusammenhang zwischen Eingangsspannung $u_e(t)$ und Ausgangsspannung $u_a(t)$
\end{itemize}
%
\begin{center}
\begin{circuitikz}[european, scale=0.75]
\node (in) at (0,0){};
\node (in_ground) at (0,-3){};
\node (out) at (4,0){};
\node (out_ground) at (4,-3){};
\draw (in) to [R,l^=$R$,o-] (3,0);
\draw (3,0) to [short,-o,] (out);
\draw (3,0) to [C,l_=$C$,*-*] (3,-3);
\draw (in_ground) to [short,o-o] (out_ground);
\path[draw, bend right, ->, >=latex] (in) edge node[left]{Eingangsspannung $u_e(t)$} (in_ground);
\path[draw, bend left, ->, >=latex] (out) edge node[right]{Ausgangsspannung $u_a(t)$} (out_ground);
\draw (3,-3) to [short,i=${i(t)}$] (1.5,-3);
\node (inp) at (-0.5,0){$+$};
\node (inm) at (-0.5,-3){$\bot$};
\end{circuitikz}
\end{center}

\subsubsection*{Zusammenhang zwischen Eingangsspannung $u_e(t)$ und Strom $i(t)$}

\cite[Kap. 9.8.2]{Marinescu2020}

Wir wählen die $\Re$-Konvention (Cosinussignal).
%
Es gibt nur einen Strom $i(t)$ im Netzwerk, der durch $R$ und $C$ in der angezeigten
Richtung fließt. Diese Fließrichtung ist eine (wenn man von der Physik kommt und den tatsächlichen Elektronenstrom anschaut anfangs vielleicht sehr gewöhnungsbedürftige) ET-Konvention mit dem sperrigen Namen
Verbraucherzählpfeilsystem \cite[S. 7]{Marinescu2020}. Für das gewählte Potentialgefälle
der Quelle $u_e(t)$ von Pluspol zu Masse führt diese Konvention dazu, dass
Spannungsabfälle über Verbrauchern die gleiche Richtung wie der technische
Stromfluss haben. Das sieht man z.B. am Spannungspfeil über dem Kondensator,
diese Spannung ist gleichzeitig auch die uns interessierende Ausgangsspannung.
%
Der tatsächliche Elektronenstrom im elektrischen Netzwerk ist entgegengesetzt
zur getroffenen Konvention!
%
Aus dem Maschenumlauf \cite[Kap. 2.5]{Marinescu2020} für Spannungen folgt
\begin{align}
\label{eq:appb:mascheRC}
u_e(t) = R i(t) +  \frac{1}{C} \int i(t) \fsd t
\end{align}
und mit der symbolischen Methode und komplexen Impedanzen kann elegant (deswegen hat man sich symbolische Methode 'ausgedacht')
\begin{align}
\underline{U}_e = R \underline{I} +  \frac{1}{\im \omega C} \underline{I}
= (R +  \frac{1}{\im \omega C}) \underline{I}
\end{align}
aufgeschrieben werden.
%
Die resultierende komplexe Impedanz zwischen $\underline{U}_e$ und $\underline{I}$
lässt sich in Betrag/Phase darstellen
\begin{align}
\underline{U}_e = \sqrt{R^2 +  \left(\frac{1}{\omega C}\right)^2} \e^{\im \phi_Z} \cdot \underline{I}
\end{align}
mit dem Phasenwinkel
\begin{align}
\phi_Z = \text{atan2}({-\frac{1}{\omega R C}}).
\end{align}
Falls nun die cosinusförmige, stationär einwirkende Eingangsspannung als die
Ursache und der resultierende Stromfluss durch R und C als
Wirkung betrachtet wird, fügen wir den eingestellten (gegebenen = bekannte Ursache)
Effektivwert und Phasenoffsetwinkel der Eingangsspannung ein
\begin{align}
U_{e,\text{eff}} \cdot \e^{\im \phi_u} = \sqrt{R^2 +  \left(\frac{1}{\omega C}\right)^2} \e^{\im \phi_Z} \cdot \underline{I}
\end{align}
und lösen nach dem ruhendem Effektivwertzeiger des Stroms auf
\begin{align}
\underline{I} =
I_\text{eff} \cdot \e^{\im \phi_i} =
\frac{U_{e,\text{eff}}}{\sqrt{R^2 +  \left(\frac{1}{\omega C}\right)^2}}
\cdot \e^{\im (\phi_u-\phi_Z)}
\end{align}
aus dem wir sehr elegant erkennen, wie groß der Effektivwert und zugehörige
Phasenoffsetwinkel des Stroms ist.
%
Wenn wir uns nun noch für den tatsächlichen, harmonischen Stromverlauf
interessieren, müssen wir
die gewonnenen Informationen in den vollständigen Ansatz übertragen
\begin{align}
\sqrt{2} \underline{I} \e^{\im \omega t}=
\sqrt{2}  I_\text{eff} \cdot \e^{\im \phi_i} \e^{\im \omega t} =
\sqrt{2} \frac{U_{e,\text{eff}}}{\sqrt{R^2 +  \left(\frac{1}{\omega C}\right)^2}}
\cdot \e^{\im (\phi_u-\phi_Z)} \e^{\im \omega t}
\end{align}
und den Zeitverlauf durch 'Rücktransformation', also Benutzung der
gewählten Cosinus-Konvention ($\Re$-Teil) bilden
\begin{align}
i(t) =
\Re\{\sqrt{2} \frac{U_{e,\text{eff}}}{\sqrt{R^2 +  \left(\frac{1}{\omega C}\right)^2}}
\cdot \e^{\im (\phi_u-\phi_Z)} \e^{\im \omega t}\}=
\frac{\sqrt{2} \, U_{e,\text{eff}}}{\sqrt{R^2 +  \left(\frac{1}{\omega C}\right)^2}}
\cdot \cos(\omega t + \phi_u-\phi_Z)
\end{align}
%
In \cite[Kap. 9.8.2]{Marinescu2020} ist dann der Funktionsverlauf von $i(t)$
und der Zusammenhang zu $u_e(t)$ zu Ende diskutiert.

\subsubsection*{Zusammenhang zwischen Eingangsspannung $u_e(t)$ und Ausgangsspannung $u_a(t)$}

\cite[Kap. 13.1]{Marinescu2020}

Sehr oft interessieren wir uns für das Verhältnis zwischen Eingangs- und Ausgangsgröße
des Netzwerks (i.e. Annahme: LTI-Systems), hier also zwischen der Eingangsspannung
(i.e. eine Quellenspannung) und der Ausgangsspannung, die hier über den Kondensator
abfällt.
Ein wichtiges Werkzeug dafür ist der sogenannte Spannungsteiler
\cite[Kap. 3.1.2, 11.2.1]{Marinescu2020}, den wir immer dann anwenden können,
wenn wir uns für Spannungsverhältnisse interessieren in Schaltungsbereichen mit
gleichem Stromfluss.
Hier im sehr einfachen Beispiel gibt es nur einen Strom und wir können
zwei komplexe Spannungsteiler aufstellen, einmal Kondensatorspannung bezogen auf
Gesamtspannung (vgl. \eqref{eq:appb_ZC}, \eqref{eq:appb:mascheRC})
\begin{align}
\frac{\underline{U}_a}{\underline{U}_e} = \frac{\frac{1}{\im \omega C}\underline{I}}{R \underline{I}+\frac{1}{\im \omega C} \underline{I}}
\end{align}
und einmal Widerstandsspannung bezogen auf Gesamtspannung
\begin{align}
\frac{\underline{U}_a}{\underline{U}_e} = \frac{R\underline{I}}{R \underline{I}+\frac{1}{\im \omega C} \underline{I}}.
\end{align}
Wir sehen, dass sich in beiden Spannungsteilern der Stromzeiger kürzen lässt,
das ist die Idee beim Spannungsteiler.
Wir sind am ersten Spannungsteiler interessiert und schreiben den ein wenig
um
\begin{align}
\frac{\underline{U}_a}{\underline{U}_e} = \frac{\frac{1}{\im \omega C}}{R +\frac{1}{\im \omega C}}=
\frac{1}{\im \omega R C + 1}
\end{align}
Es ist sinnvoll für $RC$ eine eigene Variable einzuführen, weil sie ganz entscheidend
das Verhalten des RC-Glieds bestimmt.
Gemäß der Einheit Sekunde des Produkts $R C$ führen wie die sogenannte Zeitkonstante
$T_\text{RC} = R C$ ein.
%
Sehr hilfreich ist auch der reziproke Wert, den wir als Grenzkreisfrequenz
$\omega_\text{RC} = \frac{1}{R C}$ in rad/s kennenlernen werden.
%
Mit beiden Variablen lässt sich der Spannungsteiler schreiben zu
\begin{align}
\frac{\underline{U}_a}{\underline{U}_e} = \frac{\frac{1}{\im \omega C}}{R +\frac{1}{\im \omega C}}=
\frac{1}{\im \omega R C + 1} =
\frac{1}{\im \omega T_\text{RC} + 1} =
\frac{1}{\im \frac{\omega}{\omega_\text{RC}} + 1}.
\end{align}
In Übung~\ref{sec:4408E33353} (3.4) sehen wir genau dieses System wieder, dort aus Sicht der SigSys, aber
wir begegnen dem Spannungsteiler sozusagen als Spezialfall wieder.
%
In SigSys Sprech haben wir Ausgang $\underline{U}_a \rightarrow y$ und Eingang
$\underline{U}_e \rightarrow x$ und wir könnten uns z.B. anschauen, was
stationäre harmonische Signalen (bzw. eine Superposition solcher, weil wir
LTI-Eigenschaften annehmen) passiert, wenn sie durch das System RC-Glied geschickt
werden. Wir können dann finden, dass die Fälle $\omega\ll\omega_\text{RC}$ und
$\omega\gg\omega_\text{RC}$ asymptotisch wichtige Grenzfälle sind, der Speziallfall
$\omega=\omega_\text{RC}$ sehr wichtig ist und zur Erkenntnis kommen, dass
das System nur die Amplitude (Effektivwert) und Phasenlage zwischen Aus- und Eingang
ändern kann. All das machen wir auch in einer Elektrotechnik VL/UE, in SigSys
ist es aber eingebettet in den größeren Kontext. Vielleicht spannender Vorgriff:
In Abb.~\ref{fig:bodeplot_examples_pt1_element_AppB}
sieht man, wie das Systemverhalten SigSys-typisch grafisch aufbereitet wird,
das kommt in Übung~\ref{sec:ue5_levelresponse} (5) ausführlich.

\begin{figure}
  \includegraphics[width=\textwidth]{../laplace_system_analysis/bodeplot_examples_pt1_element.pdf}
  \caption{Full SigSys picture of the \textbf{1st order lowpass system from
  exercise~\ref{sec:4408E33353} (3.4)} with $T_\mathrm{RC} = \frac{1}{\omega_\mathrm{RC}} = 1$ s.
  \texttt{bodeplot\_examples.ipynb}}
  \label{fig:bodeplot_examples_pt1_element_AppB}
\end{figure}



\subsection{Beispiel Auf- und Entladevorgang am einfachen RC-Glied}

Für das abgebildete RC-Glied sollen zwei Fälle diskutiert werden
\begin{itemize}
  \item 1. Aufladevorgang Spannung des Kondensators
  \item 2. Entladevorgang Spannung des Kondensators
\end{itemize}

\begin{center}
\begin{circuitikz}[european, scale=0.75]
\node (in) at (0,0){};
\node (in_ground) at (0,-3){};
\node (out) at (4,0){};
\node (out_ground) at (4,-3){};
\draw (in) to [R,l^=$R$,o-] (3,0);
\draw (3,0) to [short,-o,] (out);
\draw (3,0) to [C,l_=$C$,*-*] (3,-3);
\draw (in_ground) to [short,o-o] (out_ground);
\path[draw, bend right, ->, >=latex] (in) edge node[left]{Eingangsspannung $u_e(t)$} (in_ground);
\path[draw, bend left, ->, >=latex] (out) edge node[right]{Ausgangsspannung $u_a(t)$} (out_ground);
\draw (3,-3) to [short,i=${i(t)}$] (1.5,-3);
\node (inp) at (-0.5,0){$+$};
\node (inm) at (-0.5,-3){$\bot$};
\end{circuitikz}
\end{center}


\subsubsection*{Aufladevorgang am einfachen RC-Glied}

Zum Zeitpunkt $t=0$ soll die Gleichspannung $u_e(t) = U$ als Eingang auf das vorher in
Ruhe befindliche RC Glied geschaltet werden.
%
Wie verhält sich die Spannung über dem Kondensator $u_a(t)$ über die Zeit?

Wir kennen den Maschenumlauf ab $t\geq 0$
\begin{align}
U = R i(t) + u_a(t)
\end{align}
und wissen, dass der Stromfluss vom Kondensator gemäß
\begin{align}
i(t) = C \frac{\fsd u_a(t)}{\fsd t}
\end{align}
beeinflusst wird.
Die zweite Gleichung in die erste eingesetzt mit Benutzung der Zeitkonstante
$T_\text{RC} = R C$ ergibt eine lineare Differentialgleichung erster Ordnung
\begin{align}
U = R C \frac{\fsd u_a(t)}{\fsd t} + u_a(t) =
T_\text{RC} \frac{\fsd u_a(t)}{\fsd t} + u_a(t)
\end{align}
Wir könnten das (und sollten auch) noch etwas sauberer aufschreiben mit Anfangsbedingungen
und Inhomogenität
\begin{align}
u_e(t) = T_\text{RC} \frac{\fsd u_a(t)}{\fsd t} + u_a(t),
\qquad u_a(0_-)=0,\quad u_e(t\geq 0) = U.
\end{align}
Diese DGL lässt sich mit Trennung der Variablen lösen (dann sieht man leider nicht, dass es eigentlich eine Faltung ist) und die Anfangsbedingungen
können direkt mit Wahl des Integrationsbereichs berücksichtigt werden.
%
Die Lösung führt auf
\begin{align}
u_a(t \geq 0) = U(1-\e^{-\frac{t}{T_\text{RC}}}) = U - U\e^{-\frac{t}{T_\text{RC}}}
\end{align}
und ist in Abb.~\ref{fig:4408E33353_AppB} links schematisch dargestellt.
%
Die Lösung wird in \cite[Lap. 15.4.4]{Marinescu2020} ausführlich diskutiert.

\subsubsection*{Entladevorgang am einfachen RC-Glied}

Zum Zeitpunkt $t=0$ sei der Kondensator auf die Spannung $U$ aufgeladen und
der Eingang der Schaltung wird bei $t=0$ mit einem Kabel kurzgeschlossen.
Der daraufhin einsetzende Stromfluss entlädt den Kondensator.
Wie verhält sich die Spannung $u_a(t)$ über dem Kondensator?

Diese Lösung wird auch in \cite[Lap. 15.4.4, S. 371]{Marinescu2020} diskutiert.

Maschenumlauf für $t\geq 0$ ergibt
\begin{align}
  u_a(t) + R i(t) = 0
\end{align}
und Strom wie schon vorher
\begin{align}
i(t) = C \frac{\fsd u_a(t)}{\fsd t}.
\end{align}
Dies ergibt
\begin{align}
  u_a(t) = - R C \frac{\fsd u_a(t)}{\fsd t}
\end{align}
und die Anfangsbedingung berücksichtigt
\begin{align}
  u_a(t) = - R C \frac{\fsd u_a(t)}{\fsd t},
  \quad u_a(0_-) = U.
\end{align}
Auch diese lineare DGL erster Ordnung lässt sich mit Trennung der Variablen lösen
zu
\begin{align}
u_a(t) = U \e^{-\frac{t}{T_\text{RC}}}
\end{align}
%
Die Funktion ist in Abb.~\ref{fig:4408E33353_AppB} rechts schematisch dargestellt.


\begin{figure}[h]
\begin{center}
%sigma < 0
\begin{tikzpicture}
%
\def \axisLength {4}
\def \tic {0.05}
\def \sigmaz {1}
\def \omegaz {1}
\def \convAbsz {-\sigmaz}
%
\begin{scope}[]
\begin{axis}[
width=0.45\textwidth,
height=0.3\textwidth,
domain=0:7,
samples=64,
legend pos=outer north east,
xlabel = {$t\rightarrow$},
ylabel = {$u_a(t)$},
title = {Aufladevorgang Kondensatorspannung},
xmin=0, xmax=7,
ymin=-0.1, ymax=1.1,
xtick={0, 1, 3, 5},
ytick={0,0.63,0.95, 1},
xticklabels={$0$, $T_\mathrm{RC}$, $3 T_\mathrm{RC}$, $5 T_\mathrm{RC}$},
yticklabels={$0$, $0.63 U$, $0.95 U\quad$, $U$},
ymajorgrids=true,
xmajorgrids=true
]
\addplot[mark=None, color=C0, ultra thick]
coordinates {(-4,0)(0,0)};
\addplot[mark=None, color=C0, ultra thick]
{1-exp(-\sigmaz*x)};
\end{axis}
\end{scope}
%
\begin{scope}[shift={(7,0)}]
\begin{axis}[
width=0.45\textwidth,
height=0.3\textwidth,
domain=0:7,
samples=64,
legend pos=outer north east,
xlabel = {$t\rightarrow$},
ylabel = {$u_a(t)$},
title = {Entladevorgang Kondensatorspannung},
xmin=0, xmax=7,
ymin=-0.1, ymax=1.1,
xtick={0,1,3,5},
ytick={0, 0.3678, 1},
xticklabels={$0$, $T_\mathrm{RC}$, $3 T_\mathrm{RC}$, $5 T_\mathrm{RC}$},
yticklabels={$0$, $0.37 U$, $U$},
ymajorgrids=true,
xmajorgrids=true
]
\addplot[mark=None, color=C0, ultra thick]
coordinates {(-4,0)(0,0)(0,1)};
\addplot[mark=None, color=C0, ultra thick]
{exp(-\sigmaz*x)};
\end{axis}
\end{scope}
%
\end{tikzpicture}
\end{center}

\caption{Auf- und Entladekurven für Gleichspannung am Kondensator}
\label{fig:4408E33353_AppB}
\end{figure}

%
%Laplace Trafo macht jetzt etwas sehr elegantes: anstatt jedesmal das Matrix Problem aufzustellen (DGL n-ter Ordnung ergibt nxn Matrix und n Eigenwerte/Eigenvektoren), hat man das Problem
%mit Rechenregeln direkt in ein algebriasches Problem überführt, und nicht nur das es erlaubt das elebante Handling von Eingangsgrößen und Anfangs/Randbedingungen. Dadurch werden DGL erst bequem oder überhauot lösbar. Die Einführung der Laplace vor ca. 100 Jahren und ihre Formalisierung in den 1960/70er Jahren hat maßgeblich zum Weiterentwiklung analoger Signalverabrietung beigetragen und muss auch heute noch als wichtige Essenz des IngHandwerks verstanden werden.

%\begin{comment}
%------------------------------------------------------------------------------
\newpage
\section{UE 2: Basics: Elementarsignale, Lineare Systeme}
Zielsetzung / Objectives:
Basics Signale / Lineare Systeme


Fahrplan
\begin{itemize}
\item Elementarsignale rect, sprung, exp(jwt), Superposition, Zeit/Amplitudenskalierung, Zeitverschiebung/inversion
\item Systeme Test auf Linearität, i.e. Superposition, Zeit Verschiebung, Amplitudenskalierung
\item einfache aber plakative Beispiele, vlt. Corona Zeitreihe
\item Ausblick: Verknüfung Signal -> System -> Signal, vs. DGL, vs. FR, d.h. Rückgriff auf UE1
\item Message System: DGL ist mühsam, viele lineare Systeme lassen sich deutlich einfacher und elegnater lösen
\item Message Signal: manchmal(sehr oft) ist es vorteilhaft das Signal anders darzustellen (Frequenzanalyse), um
an Informationen ranzukommen, BSP: funktechnik, Video
\end{itemize}


%------------------------------------------------------------------------------
\newpage
\section{UE 3: Faltung zeitkontinuierlich}
Zielsetzung / Objectives: Idee der Faltung, Faltungsintegral allgemein, Link zu System: Faltung mit einer Impulsantwort

Fahrplan
\begin{itemize}
\item Tiefpass erster Ordnung, Sprungantowrt analytisch, Impulsantwort analytisch, Faltung für rect analytisch und grafisch
\item Link Sprung / Impuls
\item DGL lösen vs. Falten
\item Beispiel SOS Hochpass analytisch/grafisch, Interpretation/Erwartungshaltung was sehen wir in Immuls/Sprung, vgl. zu Tiefpass
\item Faltung Eigenfunktion mit Impulsantowt, Teaser zu Amplitude und Phase
\item Beispiel: Bildverabreitunt Glättungsfilter = Tiefpass, Kanten
\item Message: Faltungsintegral fundamental zur beschreibung Signale->LTI Systeme->Signale
\item Message: Faltungsintegral bzw. die Impulsantwort analytisch lösen/beschreiben oft kompliziert bis hin zu unlösbar, daher suchen wir elegante adequate beschreibungen, zB im Freqz
vgl. eine einfache 2nd order H(s) vs. komplizierter Ausdruck für Impz
\end{itemize}


%------------------------------------------------------------------------------
\newpage
\section{UE 4: DGL vs. Impulsantwort vs. Laplace Trafo}
Zielsetzung / Objectives: Sinn der Laplace Trafo, HinTrafo, Link Impulsantowrt H(s), Laplace Ebene

Fahrplan
\begin{itemize}
\item von DGL Tiefpass1st zu h(t) zu H(s) mit Def Laplace Trafo
\item Direkter Link DGL konst Koeff zu H(s)
\item Polstellen in Lapalce sind NST des Char Pol, zusätzlich in Laplace: NST
\item Analytisches Rechnen einfache Laplace Hin Trafos, stückweise stetige Signale
\item Korrespondenzen / Eigenschaften motivieren, Sprung / Impulse Zusammenhang in Laplace, Mod, Delay, was passiert jeweils im PZ
\item Faltung vs. Mult
\item Message: bisher vermeintlich noch nicht so viel gewonnen, außer andere Denke, aber in der übernächsten UE wird das alles sehr sinnvoll anzuwenden sein, aus Faltung Mult machen hat  großen Impact
\end{itemize}


%------------------------------------------------------------------------------
\newpage
\section{UE 5: Laplace Rücktrafo, ROC}
Zielsetzung / Objectives: Damit das Tool Laplace vollständig sinnvoll ist, brauchen wir auch die Rücktrafo,
komplexes Integral :-(, meist nicht rechnen, sondern mit Korrespondeze oder der
partialbruchzerlegung (Spezialfall der allg. Rücktrafo Integralsatz), wir wollen aus einer PZ Verteliung eine Impulsanfort/Springantwort/Signal finden

Fahrplan
\begin{itemize}
\item H(s) tiefpass/hochpass rücktransfomrieren, wir wissen aus UE3 was die Lösung sein muss
\item PZB Beispiele
\item KB / ROC Problematik, bisher kausale Systeme/Signale, aber Mehrdeutigekt, daher BSP links/rechts/beidseitg (diese Reihenfolge, wichtig, weil viel weniger verwirrend, also erst Spezialfall, dann allgemeinfall)
\item Message: wir können nun Laplace Hin/Rück trafo und es wurde bahuptet, dass es DGLs einfacher lösbar macht, das schauen wir uns in der nächsten UE im Detail an
\item Message: wir können mit Laplace Dinge machen die mit FT nicht gehen, KOnverhenz des Integrals erzwingen, damit Sprunghafte Signale erlaubt
\end{itemize}


%------------------------------------------------------------------------------
\newpage
\section{UE 6: Beispiel System 2. Ordung Laplace vs. DGL}
Zielsetzung / Objectives: Anhand eines Tief- oder Hochpasssystems 2. Ordnung soll
einmal alles durchgepsielt werden, um den Sinn/Eleganz der Laplace Trafo zu demonstrieren, Falt zu Mult

Fahrplan
\begin{itemize}
\item DGL homo + verschiedene inhom, so dass man Sprung/Impulsantowrt und Aexp(jwt+phi) errechnet
\item verschiedenene Schwingungszustände durch verschiedene NST des char Polynoms, versch Ansätze für part Ansatz mühsam
\item Laplace H(s) + Anfangs/Randbed, Diskussion PZ
\item Inverse Laplace für Sprung/Impuls/exp-Antowrten
\item Message: man sollte hier nun gesehen haben, dass Laplace in der Tat eleganter ist, zumindest für die hier betrachteten LTI 2nd order Systeme
\item Systemstabilität andeuten
\item Message: Statt nur hin/her zu transfomieren, können wir aber im Laplace Bereich noch mehr Info aus der PZ Ebene rausholen, daszu setzen wir sigma=0 und landen bei
einem Spezialfall der Lapace Trafo, nämlich der Fourier Trafo
\end{itemize}


%------------------------------------------------------------------------------
\newpage
\section{UE 7: Fourier Trafo}
Zielsetzung / Objectives: Trafo für eingeschwungene Zustände, Anaylse Frequenzgang Systeme, FrequenzAnaylse spezieller Signaltypen

Fahrplan
\begin{itemize}
\item Begriff Spektrum
\item Anayltische Beisoiele für Mod, Delay, Soektralanalysue
\item Energie
\item Korrespondenzen
\item Gem/Untersch Fouirer / Laplace
\item Link zur FR, periodische Signale vs. Linienspektrum -> wir kennen jetzt 2 von 4 Trafos
\item Message: Rückgriff aus UE1, jetzt macht die Anwendnung vlt mehr Sinn
\end{itemize}


%------------------------------------------------------------------------------
\newpage
\section{UE 8: Bode Plot/Nyquist Plot}
Zielsetzung / Objectives: Ing Tools um aus Laplace
Fourier Abschätzungen zu Systemeigenschaften zu treffen, früher manuell wichtig, heute
Computer, aber Kopfschätzungen auch heute noch sehr relavent, was macht ein Pol, eine NST

Fahrplan
\begin{itemize}
\item Bode Regeln PZ für Magnitude
\item Bode Regeln PZ für Phase
\item Beispiel Bode für das System aus UE 6
\item Typische 'Kurven'diskussion für System ist also: H(s), h(t), he(t), PZ, KB, Mag/Phase
\item Parallel / Reihenschaltund anhand Bode, Ausblick: Feedback Regler
\item Stabilität durch Pole schieben
\item Message: Bode und Nyquist Plots waren in analog Zeiten DIE Tools, Abschätzung
per Hand auch heute noch wichtig, weil Verständnis für die Dinge
\item Message: wir sind damit durch mit zeitkont Signalen, es lohnt sich für alle bisherig behandelten Trafos (FT, FR, Laplace) exolizit nochmal mit den Eigenschaften z ubeschäftigen -> evtl. Sonderrechenblatt wo nur das behandelt wird
\item Schaubild x->h->y vs. X->H->Y
\end{itemize}


%------------------------------------------------------------------------------
\newpage
\section{UE 9: Abtastung / Rekonstruktion}
Zielsetzung / Objectives: Erklärung Abtastung im Spektralbereich mittels Fourier trafo

Fahrplan
\begin{itemize}
\item Rückgriff Fourier Trafo, Dirac Kamm, Exp/Sin/Cos
\item Rückgriff Spektrum / Signal Vershc / Mod-> das brauchen wir jetz wieder
\item Anwendung: Signal Abtastung und Rekonsturktion
\item Sinc Rect Dualität again
\item Ausblick Signal vs. Spektrum Denke (Beweis Abtasttheorem Kotelnikov, FH Lange)
\item Message: WKS-Abtasttheorem ist ein mögliches Szenario mit Signalannahmen und
idealer (d.h. praktisch nicht realisierbarer) Reko, Praxis: andere Interpolatoren, oder aber andere Sampling Schemes, je nach Signaltyp
\end{itemize}


%------------------------------------------------------------------------------
\newpage
\section{UE 10: Basics Elementarsignale, Lineare Systeme}
Zielsetzung / Objectives: Für Folgen kann man ähnliche
Elementarsignale und Eigenschaften für LTI Systeme aufstellen, Eigensignale/lösungen von DiffGl

Fahrplan
\begin{itemize}
\item Rect/Exp/Sin/Cos, Periodizität, Eigensignale, Stem Plots!
\item Denke zu DiffGlg
\item Zeitverschiebung, Modulation, Zeitstreckung/Stauchung, Ampitudenskal, Zeit/Ampl-Inversion
\item Checks auf System Lineariät
\item Message: es gibt fundamentale Gemeinsamkeiten zwischen zeitkont/diskret, aber
auch ein paar sehr wichtige Unterschiede. Das Wesen wie Tools benutzt werden ist aber komplett gleich
\end{itemize}

%------------------------------------------------------------------------------
\newpage
\section{UE 11: The Big Picture FT / FR / DTFT / DFT}
Zielsetzung / Objectives: Erkennen/Erarbeiten der großen Zusammenhänge,
die DFT ist die FR für diskrete Signale, die DTFT ist die FT für diskrete Signale.
Vom Wesen, das was UE1 für zeitkont. Signale war, Zusammenhang für diskrete Siganle aufzeigen
Didaktisch: funktioniert bestens als WDH/Neueinstieg und als Teaser für das Kommende

Fahrplan
\begin{itemize}
\item Gemeinsamkeiten/Unterschiede Signale und deren Spektren (Periodizität, Linienspektrum)
\item Sinc / Rect Dualitäten, Dirac aus Sinc
\item Link als Ausblick: Laplace und z-Trafo als Systembeschreibungs Trafos und Frequenzgang in der FT und DTFT
\item Ausblick: es gibt auch wieder Faltung vs. Mult, Pole/Nullstellen, aus DGL wird DifffGlg
\item Message: jede Signaltyp hat seine Spektraldarstellung in exp()-Eigenfunktionen
für viele Signale ist diese Zerlegung sinnvoll, für andere aber gar nicht (KOnvergenz Fourierreihe)
wichtig ist, die richtige Trafo für das richtige Problem und souveränder Wechsel zwischen den Trafos, ist
Ing Handwerk
\end{itemize}

%------------------------------------------------------------------------------
\newpage
\section{UE 12: Zeitkdiskrete Faltung}
Zielsetzung / Objectives: Fundamentals der Faltung für Zeitdiskrete Signale

Fahrplan
\begin{itemize}
\item Faltungssume
\item Grafisch vs. Analytisch
\item Periodische / Linear Faltung
\item Korr vs. Conv summe (Machine Learning Link)
\item Message: Faltung diskret vtl. sogar einfacher als Integral, weil Summe
zugänglicher, Wesen erfassen: lange zurücliegende Samples werden mit sehr späten Samples der IR verrechnet, Ursache Wirkung plausibilisieren (Dirac Kamm)
\end{itemize}

%------------------------------------------------------------------------------
\newpage
\section{UE 13: z-Trafo}
Zielsetzung / Objectives: Sinn und Eigenschaften der zTrafo erlernen, weil wir ja eigentlich zu Analyse von Systemen nicht immer falten wollen, also Falt vs. Mult auch bei Zeitdiskret erarbeiten, statt Laplace machen wir das mit zT

Fahrplan
\begin{itemize}
\item aus DiffGL wird ein z Polynom, Analogie zu Laplace
\item z-Plane statt s-Plane
\item Korrespondenzen einfache rekursive / nicht rek Systeme, exp(jWt)
\item Beispiele Hin / Rücktrafo / ROC, Pole / NST
\item Signal X(s), Y(s), System H(s)
\item Schaubild x->h->y vs. X->H->Y
\item Message: mit Schaubild udn Vorwissen Laplace sollte wir das Tool zT zu schätzen wissen
\item TBD: DiffGL -> z Vorwärts,Rückwärts, Centered Int, Reihenentwicklung
\end{itemize}


%------------------------------------------------------------------------------
\newpage
\section{UE 14: DTFT / DFT Im Detail}
Zielsetzung / Objectives: vertiefendes Kennenlernen der beiden Neuen Fourier Trafos

Fahrplan
\begin{itemize}
\item einfache Hin/Rück, Korrespondenzen, va. Mod, Mult, Dly
\item DTFT Mag/Phase eines FIR, Was machen Nullstellen im Spektrum, Ausblick: FensterDesign als Spezialfall von Design endlicher Folgen-> Codes mit bestimmtem Spektrum
\item schnelle Faltung über zeropadded DFT
\item Message: DTFT und DFT sind Pendants von FT und FR, spezielles Problem in best Domäne sehr elegant mit richtiger Trafo
\end{itemize}



%------------------------------------------------------------------------------
\newpage
\section{UE 15: z-Trafo großes Beispiel SOS}
Zielsetzung / Objectives: komplette analytische 'Kurvendiskussion' für rek und nichtrek System

Fahrplan
\begin{itemize}
\item rek System 2nd order, zB wieder der Tiefpass, impz, step, H(z), PZ, Mag/Phase usw.
\item nicht rek System, Spezialfall Sym IR und Linearphasigkeit, impz, step, H(z), PZ, Mag/Phase usw.
\item Min/allpass Phase
\item Inversion
\item Ausblick: Bode Approx hier etwas unangenehmer, aber im Prinzip gleiche
\item Wichtig:!!! Code für IIR / FIR Filter, FIR ist Conv, IIR ist DiffGl
\item Message: einmal Systemanalyse z-Trafo/DTFT für IIR/FIR durchgespielt, in der Praxis werden die Systeme nur komplizierte, nicht aber die Tools
\end{itemize}
%\end{comment}

\renewcommand{\refname}{Buchzitate}
\clearpage
\bibliography{literatur}
\end{document}

\newpage
\section*{Einleitung}
%
Aus aktuellem Anlass einer hoffentlich einmaligen COVID-19 Pandemie, entsteht
dieses ausführliche Skript mit Übungsaufgaben, um die Signal- und Systemtheorie
(SigSys) zu vertiefen.
%
Typischerweise ist die in SigSys benutzte Mathematik einfacher als so mancher Stoff
aus Mathematik Grundlagenvorlesungen.
%
Das was wir aber an Mathematik brauchen, muss sicher beherrscht werden.
%
Dies gelingt durch individuelle Übung.
%
Das Klären des \textbf{Was?, Warum? Wie?} und die vorangestellte \textbf{Frage des Wesens}
bei der Vermittlung und Vertiefung von Stoff liegt in der
Verantwortung der Dozierenden.
%
Dies zu beherzigen und dem streng zu folgen ist viel wichtiger als die neuesten
oder die tradiertesten technischen Errungenschaften für die Didaktik zu benutzen.
%
Es gibt unterirdisch schlechte, genauso wie brillante Tafelvorlesungen,
E-Learning Videos, Lehrbücher usw.
%
Am Ende des Tages ist \textbf{Verstehen} ein hoch individueller Erarbeitungs-
und Selbstreflexionsprozess, den jede(r) mit eigener Strategie bewältigt.
%

Ich hab mich nach langem Überlegen für ein klassisches Skript entschieden,
um den nötigen Stoff (komplett neu) aufzubereiten.
%
Jede Übung werde ich in einem kurzen Screencast-Video des vorliegenden Skripts
ankommentieren und teasen (ich hoffe ich komme zeitlich mit allem hinterher).
%
Unterstützt wird dieses Übungsskript durch kleine Python-Skripte in
Jupyter Notebooks (*.ipynb Dateien),
welche alle hier enthaltenen Diagramme erzeugen, sodass Sie
detailliert nachschauen können und mit diesen Beispielen eigene Versuche machen
können.
%
Wenn Ihnen irgendwo in unserem Material das \textbf{Was?, Warum? Wie?} und das
Lernziel fehlt oder ausbaufähig erscheint, geben Sie bitte Bescheid!

Im StudIP haben wir eine \textbf{Formelsammlung} bereitgestellt. Dort sind die
wichtigsten Zusammenhänge der Signal- und Systemtheorie kompakt auf vier DIN A4
Seiten zusammengetragen. Es ist sehr sinnvoll, sich diese Formeln zu erarbeiten
und zu verstehen. Diese Formelsammlung steht während der Klausur zur Verfügung.

Im StudIP, finden sich zudem \textbf{Altklausuren} der letzten Semester,
teilweise mit Lösungen, sowie die \textbf{Übungsblätter} des letzten
Sommersemesters 2019 für Fans den nichtlinearen Lernens.

Das LaTex-Projekt zu diesem vorliegenden Skript und alle Jupyter Notebooks
gibt es unter
\begin{mdframed}[backgroundcolor=C2!10]
\url{https://github.com/spatialaudio/signals-and-systems-exercises}
\end{mdframed}
%
Regelmäßiges Update des Skripts als fertiges PDF erfolgt im StudIP.
%
Unter
\begin{mdframed}[backgroundcolor=C2!10]
\url{https://github.com/spatialaudio/signals-and-systems-lecture}
\end{mdframed}
finden sich die Unterlagen der Vorlesung.
%
Jeder ist eingeladen bei diesen zwei Projekten Fehler zu finden, Verbesserungen
vorzuschlagen oder sogar selber vorzunehmen (mittels neuem git branch).
%
Zudem sollten wir die StudIP-Mittel \textbf{Wiki, Forum, Blubber} usw. zusammen
intensiv nutzen um das kommende Semester bestmöglich zu bestreiten.
%
Ich werde zur planmäßigen Übungszeit Mi 11:00 bis 13:00 online im StudIP sein
und für Blubber etc. zur Verfügung stehen.
%
Gerne jederzeit Fragen ins Forum (dann haben alle etwas davon).

\newpage
\subsection*{Aufbau Aufgabe}
\begin{Ziel}
Das Wesen vorher klären mit Was?, Warum?, Wie? ist essentiell, damit wir das
Lernziel erfassen können und Lernen nicht willkürliches Puzzleteilsammeln ist.
Aus dem Was?, Warum?, Wie? und der gestellten Aufgabe sollten wir immer eine
Erwartungshaltung entwickeln, bevor wir uns rein handwerklich der Lösung nähern.
\end{Ziel}
\textbf{Aufgabe:} Berechnen Sie..., Zeigen Sie..., Falls...warum..., Erläutern
Sie anhand...
\begin{Werkzeug}
Hier meist Formeln, Zusammenhänge, Modelle. Wir benutzen die Formelsammlung
wo es geht und lesen ideal andere Literatur zum Thema, um die nötigen Werkzeuge
aus anderen Blickwinkeln zu erfassen.
\end{Werkzeug}
\begin{Ansatz}
Problemstellung in eine Formel überführt. Als nächstes können wir rechnen, also
reines Handwerk betreiben.
\end{Ansatz}
\begin{ExCalc}
Ein möglicher Rechenweg, wir sollten mit eigenem Stil
rechnen.
Händisches Rechnen, im Vergleich zu reinem Lesen und computergestützten Lösungen,
steigert die kognitive Kompetenz und erleichtert dem Kopf den Zugang zum Thema.
\end{ExCalc}
\begin{Loesung}
Endergebnisse, Grafiken und ganz wichtig ist --- nicht der fragwürdige Antwortsatz
aus der Schulphysik --- sondern vielmehr eine Interpretation der Lösung
für das gestellte Problem. Was ist unser Erkenntnisgewinn? Was sehen wir in den
Formeln? Was lesen wir aus den Grafiken? Wurde unsere Erwartungshaltung erfüllt
oder nicht.

In der Klausur ist typischerweise so viel Stoff abzufragen und so wenig Zeit,
dass so etwas leicht auf der Strecke bleiben kann. Jedoch eine im Kopf
durchgeführte Interpretation ist ein hervorragender Plausibilitätscheck, ob das was
wir da gerechnet haben, sein kann. Und je mehr Aufgaben wir
vorher geübt haben (eigentlich je mehr wir vom Wesen verstanden haben und in
Echtzeit verwenden können; das wird aber gern mit viel Üben oder gar
Auswendiglernen gleichgesetzt),
desto leichter und schneller wird uns das fallen.
\end{Loesung}

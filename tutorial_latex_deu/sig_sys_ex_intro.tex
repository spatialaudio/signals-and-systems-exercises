\newpage
\section*{Einleitung}
%
Dieses Übungsskript entstand als erste Version im Frühjahr/Sommer 2020 der
COVID-19 - pandemie-bedingten Lehr- und Lernsituation Rechnung tragend.
%
Die Grundidee war ein klassisches, sehr ausführliches Skript zur Hand zu haben, mit dem man auch ohne Videokonferenz-Marathon (und damals initialem Chaos) eine gute Lerngrundlage
zur Vertiefung der Signal- und Systemtheorie (SigSys) mit Übungsaufgaben hat.
%
In der anhaltenden Hoffnung, dass dies ein nützliches Format ist, gibt es nun für das Sommersemester 2021 die zweite COVID19-Version. Wir sind für Feedback sehr dankbar.

Typischerweise ist die in SigSys benutzte Mathematik einfacher als so mancher Stoff
aus Mathematik Grundlagenvorlesungen.
%
Das was wir in SigSys aber an Mathematik brauchen, muss sicher beherrscht werden.
%
Dies gelingt durch individuelle Übung und das individuelle erfolgreiche Aneignen
von Wissen.
%
Das Klären des \textbf{Was?, Warum? Wie?} und die vorangestellte \textbf{Frage des Wesens}
bei der Vermittlung und Vertiefung von Stoff liegt in der
Verantwortung der Dozierenden.
%
Dies zu beherzigen und dem streng zu folgen ist viel wichtiger als die neuesten
oder die tradiertesten technischen Errungenschaften der Didaktik zu benutzen.
%
Es gibt unterirdisch schlechte, genauso wie brillante Tafelvorlesungen,
E-Learning Videos, Lehrbücher usw. usw.
%
Am Ende des Tages ist \textbf{Verstehen} ein hoch individueller Erarbeitungs-
und Selbstreflexionsprozess, den jede/r mit eigener Strategie bewältigt.
%
Viel Erfolg dabei. Scheuen Sie keine Fragen, das macht es lebendig!
Nutzen Sie dazu gerne die Zoom-Termine der Übung und Vorlesung und die jeweiligen
Foren im studIP.
%
Wenn Ihnen irgendwo in unserem Material das \textbf{Was?, Warum? Wie?} und das
\textbf{Lernziel} fehlt oder ausbaufähig erscheint, geben Sie bitte Bescheid!

%
Unterstützt wird dieses Übungsskript durch kleine Python-Skripte in
Jupyter Notebooks (*.ipynb Dateien),
welche alle hier enthaltenen Diagramme erzeugen, sodass Sie
detailliert nachschauen können und mit diesen Beispielen eigene Versuche machen
können.
%


Im StudIP haben wir eine \textbf{Formelsammlung} bereitgestellt. Dort sind die
wichtigsten Zusammenhänge der Signal- und Systemtheorie kompakt auf vier DIN A4
Seiten zusammengetragen. Es ist sehr sinnvoll, sich diese Formeln zu erarbeiten
und zu verstehen. Diese Formelsammlung bekommen Sie während der Klausur ausgehändigt.
%
Im StudIP, finden sich zudem \textbf{Altklausuren} der letzten Semester,
teilweise mit Lösungen, sowie die \textbf{Übungsblätter} des
Sommersemesters 2019 für Fans den nichtlinearen Lernens. Zudem das komplette Skript des letzten Sommers (2020), also die Urversion dieses Skript-Konzepts.
Lassen Sie sich bitte nicht von den Seitenzahlen abschrecken: wenn man versucht
alles das zu verschriftlichen, was man während einer Präsenzübung alles erzählt
und an die Tafel schmiert, landet man eben bei knapp 30 Seiten pro Übung.
Wenn Sie das aber halbwegs in Echtzeit mit durcharbeiten und verinnerlichen,
sind Sie gut vorbereitet zum Üben der Altklausuraufgaben und ganz sicher auch zum Bestehen der Klausur.

Das LaTex-Projekt zu diesem vorliegenden Skript und alle gerenderten Jupyter Notebooks gibt es frei verfügbar unter CC BY 4.0 Lizenz unter
\begin{mdframed}[backgroundcolor=C2!10]
\url{https://github.com/spatialaudio/signals-and-systems-exercises/tree/output}
\end{mdframed}
%
Regelmäßiges Update des Skripts als fertiges PDF erfolgt im StudIP.
%
Unter
\begin{mdframed}[backgroundcolor=C2!10]
\url{https://github.com/spatialaudio/signals-and-systems-lecture}
\end{mdframed}
finden sich die Unterlagen der Vorlesung.
%
Jeder ist eingeladen bei diesen zwei Projekten Fehler zu finden, Verbesserungen
vorzuschlagen oder sogar selber vorzunehmen (mittels neuem git branch...lernen Sie auch git, Sie werden es sehr wahrscheinlich brauchen).
%
%


\newpage
\subsection*{Aufbau Aufgabe}
\begin{Ziel}
Das Wesen vorher klären mit Was?, Warum?, Wie? ist essentiell, damit wir das
Lernziel erfassen können und Lernen nicht willkürliches Puzzleteilsammeln ist.
Aus dem Was?, Warum?, Wie? und der gestellten Aufgabe sollten wir immer eine
\textbf{Erwartungshaltung} entwickeln, bevor wir uns rein handwerklich der Lösung nähern.
\end{Ziel}
\textbf{Aufgabe:} Berechnen Sie..., Zeigen Sie..., Falls...warum..., Erläutern
Sie anhand...
\begin{Werkzeug}
Hier meist Formeln, Zusammenhänge, Modelle. Wir benutzen die Formelsammlung
wo es geht und lesen ideal andere Literatur zum Thema, um die nötigen Werkzeuge
aus anderen Blickwinkeln zu erfassen.
\end{Werkzeug}
\begin{Ansatz}
Problemstellung in eine Formel überführt. Als nächstes können wir rechnen, also
reines Handwerk betreiben.
\end{Ansatz}
\begin{ExCalc}
Ein möglicher Rechenweg, wir sollten mit eigenem Stil
rechnen.
Händisches Rechnen und Malen und eigenes Programmieren, im Vergleich zu reinem Lesen und computergestützten Lösungen,
steigert die kognitive Kompetenz und erleichtert dem Kopf den Zugang zum Thema.
%
Fancy neue elektronische Medien und Aufbereitung sind zwar auf ihre Weise elegant, aber der Erlern- und Verstehensprozess bleibt leider (bzw. zum Glück?!?!) ein Akt es Selbermachens nicht des bloßen Anschauens.
%
Man muss sich einfach nur selbst beobachten und fragen, wann genau sich ein befriedigendes Belohnungsgefühl einsetzt.
%
\end{ExCalc}
\begin{Loesung}
Endergebnisse, Grafiken und ganz wichtig ist---eben nicht der fragwürdige Antwortsatz
aus der Schulphysik---sondern vielmehr eine Interpretation der Lösung
für das gestellte Problem. Was ist unser Erkenntnisgewinn? Was sehen wir in den
Formeln? Was lesen wir aus den Grafiken? Wurde unsere Erwartungshaltung erfüllt
oder nicht.

In der Klausur ist typischerweise so viel Stoff abzufragen und so wenig Zeit,
dass so etwas leicht auf der Strecke bleiben kann. Jedoch eine im Kopf
durchgeführte Interpretation ist ein hervorragender Plausibilitätscheck, ob das was
wir da gerechnet haben, sein kann. Und je mehr Aufgaben wir
vorher geübt haben (eigentlich je mehr wir vom Wesen verstanden haben und in
Echtzeit verwenden können; das wird aber gern mit viel Üben oder gar
Auswendiglernen gleichgesetzt),
desto leichter und schneller wird uns das fallen.
%
Auf geht's!
\end{Loesung}

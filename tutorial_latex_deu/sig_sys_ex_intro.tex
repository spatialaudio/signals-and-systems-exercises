\newpage
\section*{Einleitung}
%

Dieses Übungsskript entstand als erste Version während des Frühjahrs und Sommers 2020, der
COVID-19 - pandemie-bedingten Lehr- und Lernsituation Rechnung tragend.
%
Die Intention war ein klassisches, sehr ausführliches Übungsskript zur Hand zu haben,
mit dem man auch ohne Videokonferenz Marathon (und damaligem initialem Chaos)
eine Materialsammlung zur Vertiefung der Signal- und Systemtheorie (SigSys)
hat.
%
In der anhaltenden Hoffnung, dass dies ein nützliches Format ist, gibt es nun
für das Sommersemester 2025 die sechste Auflage, wieder leicht adaptiert,
um der Logik des aktuellen ET/ITTI-Studienplans noch besser zu folgen.
%
Zur kontinuierlichen Verbesserung dieses Skripts sind wir für Feedback sehr dankbar.
%
Wenn Ihnen irgendwo in unserem Material das \textbf{Was?, Warum? Wie?} und das
\textbf{Lernziel} oder die \textbf{Essenz} fehlt oder ausbaufähig erscheint, geben Sie bitte Bescheid!

Typischerweise ist die in SigSys benutzte Mathematik einfacher als so mancher
Stoff aus den Mathematik Grundlagenvorlesungen.
%
Das was wir in SigSys an Mathematik brauchen, muss aber sicher beherrscht werden
und vom Wesen durchdrungen worden sein.
%
Dies gelingt durch individuelle Übung und das individuelle erfolgreiche Aneignen
von Wissen und Rechenhandwerk.
%
Am Ende des Tages ist \textbf{Verstehen} ein hoch individueller---und ganz wichtig: analoger---
Erarbeitungs-und Selbstreflexionsprozess, den jede/r mit eigener Strategie bewältigen muss.
%
Kleine Arbeitsgruppen helfen aber immens sich gegenseitig das Verstandene und
Unverstandene zu reflektieren.

Aus vielen Jahren Lern- und Lehrerfahrung: digitale Hilfsmittel helfen eigentlich immer nur auf den letzten Metern. Textbücher und Papier und Stift, also Kulturtechniken mit mehreren Hundert bzw. Tausend Jahren Historie, sind nicht old school, nur weil uns seit nun 20 Jahren Smartphones, digital (a)social media/networking und digitale Inhalte mit sehr oft unterkomplexen Narrativen beherrschen.

Viel Erfolg beim Lernen und Erarbeiten und scheuen Sie keine Fragen an uns, das macht es lebendig!
Nutzen Sie dazu gerne die Übungen und Vorlesungen und die
jeweiligen Foren im studIP.
%


%
Unterstützt wird dieses Übungsskript durch kleine Python-Skripte in
\textbf{Jupyter Notebooks} (*.ipynb Dateien),
welche alle hier enthaltenen Diagramme erzeugen, sodass Sie
detailliert nachschauen, mit diesen Beispielen eigene Versuche machen
bzw. Fehler aufspüren können.
%
Im StudIP haben wir weiter eine \textbf{Formelsammlung} bereitgestellt.
%
Diese Formelsammlung können Sie in der Klausur benutzen, sie bekommen Sie
dann mit den Aufgaben ausgehändigt.
%
Eigene Ausdrucke sind nicht zugelassen.
%
In der Formelsammlung sind die wichtigsten Zusammenhänge der Signal- und
Systemtheorie kompakt auf vier DIN A4 Seiten zusammengetragen.
%
Es ist sehr sinnvoll, sich diese Formeln zu erarbeiten und vor allem zu verstehen.
%
Im StudIP, finden sich zudem \textbf{Altklausuren} der letzten Semester zur eigenständigen Lösung.

\newpage
Das LaTex-Projekt zu diesem vorliegenden Übungsskript und alle gerenderten Jupyter
Notebooks gibt es frei verfügbar unter CC BY 4.0 und MIT Lizenzen unter
\begin{mdframed}[backgroundcolor=C2!10]
\url{https://github.com/spatialaudio/signals-and-systems-exercises/tree/output}
\end{mdframed}
bzw.
\begin{mdframed}[backgroundcolor=C2!10]
\url{https://nbviewer.jupyter.org/github/spatialaudio/signals-and-systems-exercises/blob/output/index.ipynb}
\end{mdframed}
%

%
Die Unterlagen der Vorlesung finden sich unter
\begin{mdframed}[backgroundcolor=C2!10]
\url{https://github.com/spatialaudio/signals-and-systems-lecture}
\end{mdframed}
%
Jede/r ist eingeladen bei diesen zwei Projekten Fehler zu finden, Verbesserungen
vorzuschlagen (mittels Issue im github System) oder sogar selber vorzunehmen
(mittels neuem git branch und einem Pull Request im githbub System).
%
Lernen Sie auch möglichst bald Versionskontrolle mittels git, Sie werden es sehr
wahrscheinlich brauchen im weiteren Verlauf des Studiums und im Berufsleben.

\newpage
\subsection*{Aufbau Aufgabe}
\begin{Ziel}
Das Wesen vorher klären mit Was?, Warum?, Wie? ist essentiell, damit wir das
Lernziel erfassen können und Lernen nicht willkürliches Puzzleteilsammeln ist.
Aus dem Was?, Warum?, Wie? und der gestellten Aufgabe sollten wir immer eine
\textbf{Erwartungshaltung} entwickeln, bevor wir uns rein handwerklich der Lösung nähern.
\end{Ziel}
\textbf{Aufgabe:} Berechnen Sie..., Zeigen Sie..., Falls...warum..., Erläutern
Sie anhand...
\begin{Werkzeug}
Hier meist Formeln, Zusammenhänge, Modelle. Wir benutzen die Formelsammlung
wo es geht und lesen ideal andere Literatur (Textbücher sind immer noch ein sehr
valides Format für professionelle Wissensdarlegung) zum Thema, um die nötigen
Werkzeuge aus anderen Blickwinkeln zu erfassen.
\end{Werkzeug}
\begin{Ansatz}
Problemstellung in Formeln überführt. Als nächstes können wir rechnen, also
reines Handwerk betreiben. Wenn reines Mathehandwerk holprig, dann wird es natürlich auch schwierig das eigentliche SigSys-Problem mathematisch beschreiben und ausrechnen zu können.
\end{Ansatz}
\begin{ExCalc}
Ein möglicher Rechenweg; wir sollten mit eigenem Stil
rechnen.
Händisches Rechnen und Malen und eigenes Programmieren, im Vergleich zu reinem
Lesen und computergestützten Lösungen, steigert die kognitive Kompetenz und
erleichtert dem Kopf den Zugang zum Thema.
%
Fancy neue elektronische Medien und digitale interaktive Aufbereitung (Gamification, Nudging) sind zwar interessante und wenn gut gemacht auch zielführende Ansätze,
aber der Erlern- und Verstehensprozess bleibt leider (bzw. zum Glück?!) ein
Akt des analogen(!) Selbermachens, nicht des reinen Konsumierens von analogen und digitalen Medien.
%
Wir sollten uns beobachten (gilt im Grunde auch für den Alltag) und uns fragen, wann genau ein
befriedigendes Belohnungsgefühl einsetzt...
%
\end{ExCalc}
\begin{Loesung}
Endergebnisse, Grafiken und ganz wichtig ist---eben nicht der obligatorische Antwortsatz
aus der Schulphysik---sondern vielmehr eine ingenieursgerechte Interpretation der Lösung
für das gestellte Problem.
%
Was ist unser Erkenntnisgewinn?
%
(Vergleiche: Ich kann Faltung rechnen. vs. Ich habe Faltung verstanden. vs. Ich weiss sogar wo Faltung als Konzept herkommt.)
%
\textbf{Was sehen wir in den Formeln}?
%
Was lesen wir aus den Grafiken?
%
Wurde unsere Erwartungshaltung erfüllt oder nicht? \textbf{Ingenieurig denken}...

In der Klausur ist typischerweise so viel Stoff abzufragen und so wenig Zeit,
dass so etwas leicht auf der Strecke bleiben kann. Jedoch eine im Kopf
durchgeführte Interpretation ist ein hervorragender Plausibilitätscheck, ob das was
wir da gerechnet haben, sein kann. Und je mehr Aufgaben wir
vorher geübt haben (eigentlich je mehr wir vom Wesen verstanden haben und in
Echtzeit verwenden können; das wird aber gern mit viel Üben oder gar
Auswendiglernen gleichgesetzt),
desto leichter und schneller wird uns das fallen.
%
Auf geht's!
\end{Loesung}

% \newpage
% \subsection*{Einführung SigSys 13.04.2022, 11-13}
%
% \subsubsection*{Wesen und Einordnung SigSys}
% \begin{itemize}
% \item Mathe: DGLs/ODEs, Fundamentalsystem, Fourierreihe, Fouriertransformation,
% vlt. auch Laplace Transformation, Folgen, Rekursion
% \item Elektrotecchnik: Gleich-/Wechselstromtechnik an RLC-Schaltungen mit
% zwei speziellen Signalen: Gleichspannung und 50 Hz Sinus
% \item Mechanik: Kräfte auf Feder-Masse System, Schwingungen, Resonanz
% \item SigSys (typisch zwischen 2-4 Semester): DGLs und Signale mit
% Ingenieurspragmatik analysieren und synthetisieren, hier speziell: DGLs mit
% konstanten! Koeffizienten und Signale die als Superposition von kontinuierlichen
% Sinus/Cosinus-Schwingungen oder diskreten Folgen darstellbar sind
% \item Wir lernen Tools für: Regelungstechnik, Nachrichtenübertragungstechnik,
% Hochfrequenztechnik, Akustik, aber auch Leistungselektronik, Schaltungstechnik
% usw.
% \item Wir legen das Fundament für: Digitale Signalverarbeitung, aber auch
% Algorithmen des Maschinellen Lernens, vlg. Buzzwords: Convolutional Neural Network,
% Image Processing Black Hole
% \end{itemize}
%
% \subsubsection*{IT für SigSys VL/UE}
% \begin{itemize}
% \item Open Educational Resources (OER), zB MIT, edX
% \item wann/warum: Matlab vs. Python
% \item SigSys Lecture github
% \item SigSys Exercise github
% \item git clone vs. repo download als zip
% \item Static View Using NBViewer
% \item Dynamic \& Interactive Using MyBinder
% \item Local Install
% \item Anaconda Distribution
% \item Anaconda Navigator
% \item operate from Terminal
% \item IDEs:PyCharm / Spyder
% \item Jupyter Lab
% \item Jupyter Notebook
% \item Beispiel numpy vs. sympy
% \end{itemize}

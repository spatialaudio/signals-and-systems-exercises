\newpage
\section*{Einleitung}
%

Dieses Übungsskript entstand als erste Version im Frühjahr/Sommer 2020, der
COVID-19 - pandemie-bedingten Lehr- und Lernsituation Rechnung tragend.
%
Die Intention war ein klassisches, sehr ausführliches Skript zur Hand zu haben,
mit dem man auch ohne Videokonferenz-Marathon (und damaligem initialem Chaos)
eine gute Materialsammlung zur Vertiefung der Signal- und Systemtheorie (SigSys)
hat.
%
In der anhaltenden Hoffnung, dass dies ein nützliches Format ist, gibt es nun
für das Sommersemester 2021 die zweite COVID19-Version. Wir sind für Feedback
sehr dankbar.

Typischerweise ist die in SigSys benutzte Mathematik einfacher als so mancher
Stoff aus den Mathematik Grundlagenvorlesungen.
%
Das was wir in SigSys aber an Mathematik brauchen, muss sicher beherrscht werden.
%
Dies gelingt durch individuelle Übung und das individuelle erfolgreiche Aneignen
von Wissen.
%
Das Klären des \textbf{Was?, Warum? Wie?} und die vorangestellte
\textbf{Frage des Wesens} bei der Vermittlung und Vertiefung von Stoff liegt
in der Verantwortung der Dozierenden.
%
Dies zu beherzigen und dem streng zu folgen ist viel wichtiger als die neuesten
oder ältere technische Errungenschaften der Didaktik zu benutzen.
%
Am Ende des Tages ist \textbf{Verstehen} ein hoch individueller Erarbeitungs-
und Selbstreflexionsprozess, den jede/r mit eigener Strategie bewältigt.
%
Viel Erfolg dabei und scheuen Sie keine Fragen, das macht es lebendig!
Nutzen Sie dazu gerne die Zoom-Termine der Übung und der Vorlesung und die
jeweiligen Foren im studIP.
%
Wenn Ihnen irgendwo in unserem Material das \textbf{Was?, Warum? Wie?} und das
\textbf{Lernziel} fehlt oder ausbaufähig erscheint, geben Sie bitte Bescheid!

%
Unterstützt wird dieses Übungsskript durch kleine Python-Skripte in
\textbf{Jupyter Notebooks} (*.ipynb Dateien),
welche alle hier enthaltenen Diagramme erzeugen, sodass Sie
detailliert nachschauen, mit diesen Beispielen eigene Versuche machen
bzw. Fehler aufspüren können.
%
Im StudIP haben wir weiter eine \textbf{Formelsammlung} bereitgestellt.
%
Diese Formelsammlung können Sie in der Klausur benutzen, sie bekommen Sie
dann mit den Aufgaben ausgehändigt.
%
Eigene Ausdrucke sind nicht zugelassen.
%
In der Formelsammlung sind die wichtigsten Zusammenhänge der Signal- und
Systemtheorie kompakt auf vier DIN A4 Seiten zusammengetragen.
%
Es ist sehr sinnvoll, sich diese Formeln zu erarbeiten und zu verstehen.

%
Im StudIP, finden sich zudem \textbf{Altklausuren} der letzten Semester,
teilweise mit Lösungen, sowie die \textbf{Übungsblätter} des Sommersemesters
2019 für Fans den nichtlinearen Lernens;
%
zudem das komplette Skript des letzten Sommers (2020), also die Urversion
dieses Skript-Konzepts, damit Sie bei Interesse schon vorab eine Idee bekommen,
wohin die Reise während des Semester gehen wird.
%
Die aktuelle Version II wird sich nicht essentiell ändern, wir werden
Didaktik optimieren und Fehler aufspüren.
%
Lassen Sie sich bitte nicht von der Seitenanzahl abschrecken:
%
wenn man versucht alles das detailliert zu verschriftlichen, was man während einer
Präsenzübung so alles erzählt und an die Tafel schmiert, landet man eben bei
knapp 30 Seiten pro Übung.
%
Wenn Sie das aber halbwegs in Echtzeit mit durcharbeiten und verinnerlichen,
sind Sie gut vorbereitet zum Üben der Altklausuraufgaben und ganz sicher auch
zum Bestehen der Klausur.

Regelmäßiges, mindest wöchentliches Update dieses Skripts als fertiges PDF
erfolgt im StudIP im Dateienordner und es wird verlinkt mit dem jeweiligen Thema
im Forum.
%
Nutzen Sie das Forum für Fragen!
%
Jeden \textbf{Mittwoch 11:15 - 12:45} im SS2021 gibt es eine Zoom Konsultation
(Zugangsdaten finden Sie im StudIP).

\newpage
Das LaTex-Projekt zu diesem vorliegenden Übungsskript und alle gerenderten Jupyter
Notebooks gibt es frei verfügbar unter CC BY 4.0 und MIT Lizenzen unter
\begin{mdframed}[backgroundcolor=C2!10]
\url{https://github.com/spatialaudio/signals-and-systems-exercises/tree/output}
\end{mdframed}
bzw.
\begin{mdframed}[backgroundcolor=C2!10]
\url{https://nbviewer.jupyter.org/github/spatialaudio/signals-and-systems-exercises/blob/output/index.ipynb}
\end{mdframed}
%

%
Die Unterlagen der Vorlesung finden sich unter
\begin{mdframed}[backgroundcolor=C2!10]
\url{https://github.com/spatialaudio/signals-and-systems-lecture}
\end{mdframed}
%
Jede/r ist eingeladen bei diesen zwei Projekten Fehler zu finden, Verbesserungen
vorzuschlagen (mittels Issue im github System) oder sogar selber vorzunehmen
(mittels neuem git branch und einem Pull Request im githbub System).
%
Lernen Sie auch möglichst bald Versionskontrolle mittels git, Sie werden es sehr
wahrscheinlich brauchen im weiteren Verlauf des Studiums und im Berufsleben.

\newpage
\subsection*{Aufbau Aufgabe}
\begin{Ziel}
Das Wesen vorher klären mit Was?, Warum?, Wie? ist essentiell, damit wir das
Lernziel erfassen können und Lernen nicht willkürliches Puzzleteilsammeln ist.
Aus dem Was?, Warum?, Wie? und der gestellten Aufgabe sollten wir immer eine
\textbf{Erwartungshaltung} entwickeln, bevor wir uns rein handwerklich der Lösung nähern.
\end{Ziel}
\textbf{Aufgabe:} Berechnen Sie..., Zeigen Sie..., Falls...warum..., Erläutern
Sie anhand...
\begin{Werkzeug}
Hier meist Formeln, Zusammenhänge, Modelle. Wir benutzen die Formelsammlung
wo es geht und lesen ideal andere Literatur zum Thema, um die nötigen Werkzeuge
aus anderen Blickwinkeln zu erfassen.
\end{Werkzeug}
\begin{Ansatz}
Problemstellung in Formeln überführt. Als nächstes können wir rechnen, also
reines Handwerk betreiben.
\end{Ansatz}
\begin{ExCalc}
Ein möglicher Rechenweg, wir sollten mit eigenem Stil
rechnen.
Händisches Rechnen und Malen und eigenes Programmieren, im Vergleich zu reinem
Lesen und computergestützten Lösungen, steigert die kognitive Kompetenz und
erleichtert dem Kopf den Zugang zum Thema.
%
Fancy neue elektronische Medien und Aufbereitung sind zwar auf ihre Weise elegant,
aber der Erlern- und Verstehensprozess bleibt leider (bzw. zum Glück?!?!) ein
Akt es Selbermachens nicht des bloßen Konsumierens.
%
Beobachten Sie sich dazu selber und fragen sich, wann genau ein
befriedigendes Belohnungsgefühl einsetzt...
%
\end{ExCalc}
\begin{Loesung}
Endergebnisse, Grafiken und ganz wichtig ist---eben nicht der obligatorische Antwortsatz
aus der Schulphysik---sondern vielmehr eine Interpretation der Lösung
für das gestellte Problem.
%
Was ist unser Erkenntnisgewinn?
%
(Vergleiche: Ich kann Faltung rechnen vs. Ich habe Faltung verstanden.)
%
Was sehen wir in den Formeln?
%
Was lesen wir aus den Grafiken?
%
Wurde unsere Erwartungshaltung erfüllt oder nicht?

In der Klausur ist typischerweise so viel Stoff abzufragen und so wenig Zeit,
dass so etwas leicht auf der Strecke bleiben kann. Jedoch eine im Kopf
durchgeführte Interpretation ist ein hervorragender Plausibilitätscheck, ob das was
wir da gerechnet haben, sein kann. Und je mehr Aufgaben wir
vorher geübt haben (eigentlich je mehr wir vom Wesen verstanden haben und in
Echtzeit verwenden können; das wird aber gern mit viel Üben oder gar
Auswendiglernen gleichgesetzt),
desto leichter und schneller wird uns das fallen.
%
Auf geht's!
\end{Loesung}

\newpage
\subsection*{Einführung SigSys 07.04.2021, Zoom 11:15}

\subsubsection*{Wesen und Einordnung SigSys}
\begin{itemize}
\item Mathe: DGLs/ODEs, Fundamentalsystem, Fourierreihe, Fouriertransformation,
vlt. auch Laplace Transformation, Folgen, Rekursion
\item Elektrotecchnik: Gleich-/Wechselstromtechnik an RLC-Schaltungen mit
zwei speziellen Signalen: Gleichspannung und 50 Hz Sinus
\item Mechanik: Kräfte auf Feder-Masse System, Schwingungen, Resonanz
\item SigSys (typisch zwischen 2-4 Semester): DGLs und Signale mit
Ingenieurspragmatik analysieren und synthetisieren, hier speziell: DGLs mit
konstanten! Koeffizienten und Signale die als Superposition von kontinuierlichen
Sinus/Cosinus-Schwingungen oder diskreten Folgen darstellbar sind
\item Wir lernen Tools für: Regelungstechnik, Nachrichtenübertragungstechnik,
Hochfrequenztechnik, Akustik, aber auch Leistungselektronik, Schaltungstechnik
usw.
\item Wir legen das Fundament für: Digitale Signalverarbeitung, aber auch
Algorithmen des Maschinellen Lernens, vlg. Buzzwords: Convolutional Neural Network,
Image Processing Black Hole
\end{itemize}

\subsubsection*{IT für SigSys VL/UE}
\begin{itemize}
\item Open Educational Resources (OER), zB MIT, edX
\item wann/warum: Matlab vs. Python
\item SigSys Lecture github
\item SigSys Exercise github
\item git clone vs. repo download als zip
\item Static View Using NBViewer
\item Dynamic \& Interactive Using MyBinder
\item Local Install
\item Anaconda Distribution
\item Anaconda Navigator
\item operate from Terminal
\item IDEs:PyCharm / Spyder
\item Jupyter Lab
\item Jupyter Notebook
\item Beispiel numpy vs. sympy
\end{itemize}

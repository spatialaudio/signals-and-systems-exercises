\clearpage
\section{UE 9: Inverse z-Transformation}

\subsection{Task}
Transform
\begin{itemize}
	\item[a )] $X(z)=2\cdot\frac{z\cdot(z-\frac{1}{\sqrt{2}})}{(z-\e^{+\im\frac{\pi}{4}})\cdot(z-\e^{-\im\frac{\pi}{4}})}$
	\item[b )] $X(z)=\frac{z}{z-\frac{1}{2}}$
	\item[c )] $X(z)=\frac{z}{z-\frac{1}{2}}\cdot\frac{z}{z-1}$
	\item[d )] $X(z)=\frac{z^2-z+2}{z^2-\frac{1}{2}z+\frac{1}{4}}$
	\item[e )] $X(z)=\frac{z^2}{z^2+1}$
	\item[f )] $X(z)=\frac{z^4+z^3-6z^2+6z-1}{z^2-2z+1}$
	\item[g )] $X(z)=\frac{z\cdot(z-1)}{z^2-\sqrt{2}z+1}$
\end{itemize}
into time-discret signals.
The task shall be performed with help of the
\begin{itemize}
	\item \textbf{correspondence table/ partial fraction decomposition},
	\item \textbf{Residue theorem}.
\end{itemize}
For a quick view there is a solution table at the end of the document.
\subsection{Solution Using Correspondence Table/ Partial Fraction Decomposition}
\subsubsection{Task a)}
We shall find the time-discret signal for
\begin{align}
	X(z)&=2\cdot\frac{z\cdot(z-\frac{1}{\sqrt{2}})}{(z-\e^{+\im\frac{\pi}{4}})\cdot(z-\e^{-\im\frac{\pi}{4}})}=2\cdot\frac{z\cdot(z-\frac{1}{\sqrt{2}})}{z^2-z(\e^{+\im\frac{\pi}{4}}+\e^{-\im\frac{\pi}{4}})+1}=2\cdot\frac{z\cdot(z-\frac{1}{\sqrt{2}})}{z^2-z\bigg[\cos(\frac{\pi}{4})+\im\sin(\frac{\pi}{4})+\cos(\frac{\pi}{4})-\im\sin(\frac{\pi}{4})\bigg ]+1}\nonumber\\
	&=2\cdot\frac{z\cdot(z-\frac{1}{\sqrt{2}})}{z^2-2z\cos(\frac{\pi}{4})+1}=2\cdot\frac{z^2-z\cos(\frac{\pi}{4})}{z-2z\cos(\frac{\pi}{4})+1}.
\end{align}
If we take a look at the correspondence table, we find following correspondence:
\begin{align}
	\cos[\Omega_0k]\epsilon[k]\quad\ztransf\quad\frac{z^2-z\cos(\Omega_0)}{z^2-2z\cos(\Omega_0)+1}.
\end{align}
So our solution is
\begin{align}
	X(z)=2\cdot\frac{z^2-z\cos(\frac{\pi}{4})}{z-2z\cos(\frac{\pi}{4})+1}\quad\Ztransf\quad x[k]=2\cdot\cos[\frac{\pi}{4}k]\epsilon[k]
\end{align}
\subsubsection{Task b)}
We shall find the time-discret signal for
\begin{align}
	X(z)=\frac{z}{z-\frac{1}{2}}.
\end{align}
If we take a look at the correspondence table, we find following correspondence:
\begin{align}
	\label{eq:GeometricProgression}
	a^k\epsilon[k]\quad\ztransf\quad\frac{z}{z-a}.
\end{align}
So our solution is
\begin{align}
	X(z)=\frac{z}{z-\frac{1}{2}}\quad\Ztransf\quad x[k]=\bigg (\frac{1}{2}\bigg)^k\epsilon[k].
\end{align}
\subsubsection{Task c)}
We shall find the time-discret signal for
\begin{align}
	X(z)=\frac{z}{z-\frac{1}{2}}\cdot\frac{z}{z-1}.
\end{align}
Now we can not find a correspondence in our table. But it seems that a partial fraction decomposition could be helpful.
\begin{align}
	\frac{X(z)}{z}=\frac{1}{z-\frac{1}{2}}\cdot\frac{z}{z-1}=\frac{A}{z-\frac{1}{2}}+\frac{B}{z-1} \quad \Bigg | \quad\cdot (z-\frac{1}{2})\cdot(z-1)\nonumber\\
	z=A\cdot(z-1)+B\cdot(z-\frac{1}{2})=z(A+B)-A-\frac{B}{2}
\end{align}
We get a system of linear equations. In matrix notation:
\begin{align}
	\begin{pmatrix}
		-1 & -\frac{1}{2} \\
		1 & 1
	\end{pmatrix}
	\cdot
	\begin{pmatrix}
		A & B
	\end{pmatrix}
	=
	\begin{pmatrix}
		0 & 1
	\end{pmatrix}.
\end{align}
With techniques like Gaussian elimination or Cramer's rule we can solute our system of linear equations and achieve
\begin{align}
	A = -1 \nonumber \\
	B = 2.
\end{align}

\begin{align}
	\frac{X(z)}{z}=-1\cdot\frac{1}{z-\frac{1}{2}}+2\cdot\frac{1}{z-1} \quad\Bigg | \quad \cdot z
\end{align}

\begin{align}
	X(z)=-1\cdot\frac{z}{z-\frac{1}{2}}+2\cdot\frac{z}{z-1}
\end{align}
Using the correspondence from \ref{eq:GeometricProgression} and 
\begin{align}
	\epsilon[k]\quad\ztransf\quad\frac{z}{z-1}
\end{align}
we can find the time-discret signal:
\begin{align}
	X(z)=-1\cdot\frac{1}{z-\frac{1}{2}}+2\cdot\frac{1}{z-1}\quad\Ztransf\quad x[k]=-\bigg(\frac{1}{2}\bigg)^k\epsilon[k]+2\epsilon[k].
\end{align}
\subsubsection{Task d)}
We shall find the time-discret signal for
\begin{align}
	X(z)=\frac{z^2-z+2}{z^2-\frac{1}{2}z+\frac{1}{4}}.
\end{align}
\begin{align}
	\frac{X(z)}{z}=\frac{z^2-z+2}{z\cdot(z^2-\frac{1}{2})}
\end{align}
We get a system of linear equations. In matrix notation:
\begin{align}
	\begin{pmatrix}
		-1 & -\frac{1}{2} \\
		1 & 1
	\end{pmatrix}
	\cdot
	\begin{pmatrix}
		A \\ B
	\end{pmatrix}
	=
	\begin{pmatrix}
		0 & 1
	\end{pmatrix}.
\end{align}
With techniques like Gaussian elimination or Cramer's rule we can solute our system of linear equations and achieve
\begin{align}
	A = -1 \nonumber \\
	B = 2.
\end{align}

\begin{align}
	\frac{X(z)}{z}=-1\cdot\frac{1}{z-\frac{1}{2}}+2\cdot\frac{1}{z-1} \quad\Bigg | \quad \cdot z
\end{align}

\begin{align}
	X(z)=-1\cdot\frac{z}{z-\frac{1}{2}}+2\cdot\frac{z}{z-1}
\end{align}
Using the correspondence from \ref{eq:GeometricProgression} and 
\begin{align}
	\epsilon[k]\quad\ztransf\quad\frac{z}{z-1}
\end{align}
we can find the time-discret signal:
\begin{align}
	X(z)=-1\cdot\frac{1}{z-\frac{1}{2}}+2\cdot\frac{1}{z-1}\quad\Ztransf\quad x[k]=-\bigg(\frac{1}{2}\bigg)^k\epsilon[k]+2\epsilon[k].
\end{align}
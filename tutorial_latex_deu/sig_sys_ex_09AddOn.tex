\clearpage
\section{UE 9: Inverse z-Transformation}

\subsection{Task}
Transform
\begin{itemize}
	\item[a )] $X(z)=2\cdot\frac{z\cdot(z-\frac{1}{\sqrt{2}})}{(z-\e^{+\im\frac{\pi}{4}})\cdot(z-\e^{-\im\frac{\pi}{4}})}$
	\item[b )] $X(z)=\frac{z}{z-\frac{1}{2}}$
	\item[c )] $X(z)=\frac{z}{z-\frac{1}{2}}\cdot\frac{z}{z-1}$
	\item[d )] $X(z)=\frac{z^2-z+2}{z^2-\frac{1}{2}z+\frac{1}{4}}$
	\item[e )] $X(z)=\frac{z^2}{z^2+1}$
	\item[f )] $X(z)=\frac{z^4+z^3-6z^2+6z-1}{z^2-2z+1}$
	\item[g )] $X(z)=\frac{z\cdot(z-1)}{z^2-\sqrt{2}z+1}$
\end{itemize}
into time-discret signals.
The task shall be performed with help of the
\begin{itemize}
	\item \textbf{correspondence table/ partial fraction decomposition},
	\item \textbf{Residue theorem}.
\end{itemize}
For a quick view there is a solution table at the end of the document.
\subsection{Solution Using Correspondence Table/ Partial Fraction Decomposition}
\subsubsection{Task a)}
We shall find the time-discret signal for
\begin{align}
	X(z)&=2\cdot\frac{z\cdot(z-\frac{1}{\sqrt{2}})}{(z-\e^{+\im\frac{\pi}{4}})\cdot(z-\e^{-\im\frac{\pi}{4}})}=2\cdot\frac{z\cdot(z-\frac{1}{\sqrt{2}})}{z^2-z(\e^{+\im\frac{\pi}{4}}+\e^{-\im\frac{\pi}{4}})+1}=2\cdot\frac{z\cdot(z-\frac{1}{\sqrt{2}})}{z^2-z\bigg[\cos(\frac{\pi}{4})+\im\sin(\frac{\pi}{4})+\cos(\frac{\pi}{4})-\im\sin(\frac{\pi}{4})\bigg ]+1}\nonumber\\
	&=2\cdot\frac{z\cdot(z-\frac{1}{\sqrt{2}})}{z^2-2z\cos(\frac{\pi}{4})+1}=2\cdot\frac{z^2-z\cos(\frac{\pi}{4})}{z-2z\cos(\frac{\pi}{4})+1}.
\end{align}
If we take a look at the correspondence table, we find following correspondence:
\begin{align}
	\cos[\Omega_0k]\epsilon[k]\quad\ztransf\quad\frac{z^2-z\cos(\Omega_0)}{z^2-2z\cos(\Omega_0)+1}.
\end{align}
So our solution is
\begin{align}
	X(z)=2\cdot\frac{z^2-z\cos(\frac{\pi}{4})}{z-2z\cos(\frac{\pi}{4})+1}\quad\Ztransf\quad x[k]=2\cdot\cos[\frac{\pi}{4}k]\epsilon[k]
\end{align}
\subsubsection{Task b)}
We shall find the time-discret signal for
\begin{align}
	X(z)=\frac{z}{z-\frac{1}{2}}.
\end{align}
If we take a look at the correspondence table, we find following correspondence:
\begin{align}
	\label{eq:GeometricProgression}
	a^k\epsilon[k]\quad\ztransf\quad\frac{z}{z-a}.
\end{align}
So our solution is
\begin{align}
	X(z)=\frac{z}{z-\frac{1}{2}}\quad\Ztransf\quad x[k]=\bigg (\frac{1}{2}\bigg)^k\epsilon[k].
\end{align}
\subsubsection{Task c)}
We shall find the time-discret signal for
\begin{align}
	X(z)=\frac{z}{z-\frac{1}{2}}\cdot\frac{z}{z-1}.
\end{align}
Now we can not find a correspondence in our table. But it seems that a partial fraction decomposition could be helpful.
\begin{align}
	\frac{X(z)}{z}=\frac{1}{z-\frac{1}{2}}\cdot\frac{z}{z-1}=\frac{A}{z-\frac{1}{2}}+\frac{B}{z-1} \quad \Bigg | \quad\cdot (z-\frac{1}{2})\cdot(z-1)\nonumber\\
	z=A\cdot(z-1)+B\cdot(z-\frac{1}{2})=z(A+B)-A-\frac{B}{2}
\end{align}
We get a system of linear equations. In matrix notation:
\begin{align}
	\begin{pmatrix}
		-1 & -\frac{1}{2} \\
		1 & 1
	\end{pmatrix}
	\cdot
	\begin{pmatrix}
		A & B
	\end{pmatrix}
	=
	\begin{pmatrix}
		0 & 1
	\end{pmatrix}.
\end{align}
With techniques like Gaussian elimination or Cramer's rule we can solute our system of linear equations and achieve
\begin{align}
	A = -1 \nonumber \\
	B = 2.
\end{align}

\begin{align}
	\frac{X(z)}{z}=-1\cdot\frac{1}{z-\frac{1}{2}}+2\cdot\frac{1}{z-1} \quad\Bigg | \quad \cdot z
\end{align}

\begin{align}
	X(z)=-1\cdot\frac{z}{z-\frac{1}{2}}+2\cdot\frac{z}{z-1}
\end{align}
Using the correspondence from \ref{eq:GeometricProgression} and 
\begin{align}
	\epsilon[k]\quad\ztransf\quad\frac{z}{z-1}
\end{align}
we can find the time-discret signal:
\begin{align}
	X(z)=-1\cdot\frac{1}{z-\frac{1}{2}}+2\cdot\frac{1}{z-1}\quad\Ztransf\quad x[k]=-\bigg(\frac{1}{2}\bigg)^k\epsilon[k]+2\epsilon[k].
\end{align}
\subsubsection{Task d)}
We shall find the time-discret signal for
\begin{align}
	X(z)=\frac{z^2-z+2}{z^2-\frac{1}{2}z+\frac{1}{4}}.
\end{align}
\begin{align}
	\frac{X(z)}{z}=\frac{z^2-z+2}{z\cdot(z^2-\frac{1}{2})}
\end{align}
We get a system of linear equations. In matrix notation:
\begin{align}
	\begin{pmatrix}
		-1 & -\frac{1}{2} \\
		1 & 1
	\end{pmatrix}
	\cdot
	\begin{pmatrix}
		A \\ B
	\end{pmatrix}
	=
	\begin{pmatrix}
		0 \\ 1
	\end{pmatrix}.
\end{align}
With techniques like Gaussian elimination or Cramer's rule we can solute our system of linear equations and achieve
\begin{align}
	A = -1 \nonumber \\
	B = 2.
\end{align}

\begin{align}
	\frac{X(z)}{z}=-1\cdot\frac{1}{z-\frac{1}{2}}+2\cdot\frac{1}{z-1} \quad\Bigg | \quad \cdot z
\end{align}

\begin{align}
	X(z)=-1\cdot\frac{z}{z-\frac{1}{2}}+2\cdot\frac{z}{z-1}
\end{align}
Using the correspondence from \ref{eq:GeometricProgression} and 
\begin{align}
	\epsilon[k]\quad\ztransf\quad\frac{z}{z-1}
\end{align}
we can find the time-discret signal:
\begin{align}
	X(z)=-1\cdot\frac{1}{z-\frac{1}{2}}+2\cdot\frac{1}{z-1}\quad\Ztransf\quad x[k]=-\bigg(\frac{1}{2}\bigg)^k\epsilon[k]+2\epsilon[k].
\end{align}
\begin{align}
	\underset{s \in \mathrm{Kb} \{ X(s) \} }{\min}\big \{ \Re \{ s \} \big \}
\end{align}
\subsubsection{Task e)}
We shall find the time-discret signal for
\begin{align}
	X(z)=\frac{z^2}{z^2+1}.
\end{align}
If we look at the correspondence 
\begin{align}
	\cos [\Omega_0k]\epsilon[k]\quad\ztransf\quad \frac{z^2-z\cos(\Omega_0))}{z^2-2z\cos(\Omega_0)+1}\quad\quad |z| > 1
\end{align}
we have the left site if we set
\begin{align}
	\cos(\Omega_0)=0.
\end{align}
We get
\begin{align}
	\Omega_0 = \frac{\pi}{2} \pm \pi \cdot m \quad m \in \mathbb{Z}.
\end{align}
At the end, our time-discret signal is
\begin{align}
	x[k]=\cos[(\frac{\pi}{2}\pm \pi\cdot m)k]\epsilon[k]\quad m \in \mathbb{Z}.
\end{align}
\subsubsection{Task f)}
We shall find the time-discret signal for
\begin{align}
	X(z)=\frac{z^4+z^3-6z^2+6z-1}{z^2-2z+1}.
\end{align}
We can simplify the denominator to
\begin{align}
	(z-1)^2.
\end{align}
If we take a look at the correspondences
\begin{align}
	k\epsilon[k]\quad\ztransf\quad\frac{z}{(z-1)^2}\quad |z|>1
\end{align}
and
\begin{align}
	x[k-\kappa]\quad\ztransf\quad z^{-\kappa}X(z)
\end{align}
we rewrite our function to use these two:
\begin{align}
	X(z)=z^3\cdot\frac{z}{z^2+1}+z^2\cdot\frac{z}{z^2+1}-6z\cdot\frac{z}{z^2+1}+6\frac{z}{z^2+1}-z^{-1}\cdot\frac{z}{z^2+1}.
\end{align}
So our time-discret signal is
\begin{align}
	X(z)=z^3\cdot\frac{z}{z^2+1}+z^2\cdot\frac{z}{z^2+1}-6z\cdot\frac{z}{z^2+1}+6\frac{z}{z^2+1}-z^{-1}\cdot\frac{z}{z^2+1} \nonumber
\end{align}
\begin{align}
	\Ztransf \nonumber
\end{align}
\begin{align}
	x[k]=(k+3)\epsilon[k+3]+(k+2)\epsilon[k+2]-6(k+1)\epsilon[k+1]+6k\epsilon[k]-(k-1)\epsilon[k-1].
\end{align}
\subsubsection{Task g)}
We shall find the time-discret signal for
\begin{align}
	X(z)=\frac{z\cdot(z-1)}{z^2-\sqrt{2}z+1}.
\end{align}
We can reshape the function:
\begin{align}
	X(z)=\frac{z^2-z}{z^2-2z\cos(\frac{\pi}{4})+1}=\frac{z^2-z\cos(\frac{\pi}{4})}{z^2-2z\cos(\frac{\pi}{4})+1}-(1-\frac{\sqrt{2}}{2})\frac{z}{z^2-2z\cos(\frac{\pi}{4})+1}\cdot\frac{\sin(\frac{\pi}{4})}{\sin(\frac{\pi}{4})}.
\end{align}
Using the correspondeces
\begin{align}
	\sin[\Omega_0k]\epsilon[k]\quad\ztransf\quad\frac{z\sin(\Omega_0)}{z^2-2z\cos(\Omega_0)+1}\quad |z| > 1
\end{align}
and
\begin{align}
	\cos[\Omega_0k]\epsilon[k]\quad\ztransf\quad\frac{z^2-z\cos(\Omega_0)}{z^2-2z\cos(\Omega_0)+1}\quad |z| > 1
\end{align}
leads to
\begin{align}
	X(z)\quad\Ztransf\quad x[k]=\cos[\frac{\pi}{4}k]\epsilon[k]+\sin[\frac{\pi}{4}k]\epsilon[k]-\sqrt{2}\sin[\frac{\pi}{4}k]\epsilon[k].
\end{align}
\subsection{Solution Using Residue Theorem}
\subsubsection{Task a)}
We shall find the time-discret signal for
\begin{align}
	X(z)=2\cdot\frac{z\cdot(z-\frac{1}{\sqrt{2}})}{(z-\e^{+\im\frac{\pi}{4}})\cdot(z-\e^{-\im\frac{\pi}{4}})}.
\end{align}
There are two singularities: $\e^{+\im\frac{\pi}{4}}$ and $\e^{-\im\frac{\pi}{4}}$.
\begin{align}
	\mathrm{Res}(X(z)z^{k-1},\e^{+\im\frac{\pi}{4}})&=\lim\limits_{z\rightarrow\e^{+\im\frac{\pi}{4}}}2\cdot\frac{z\cdot(z-\frac{1}{\sqrt{2}})}{(z-\e^{-\im\frac{\pi}{4}})}z^{k-1}=2\cdot\frac{\e^{+\im\frac{\pi}{4}}\cdot(\e^{+\im\frac{\pi}{4}}-\frac{1}{\sqrt{2}})}{\e^{+\im\frac{\pi}{4}}-\e^{-\im\frac{\pi}{4}}}\e^{+\im\frac{\pi}{4}(k-1)} \nonumber \\
	&=2\cdot\frac{\e^{+\im\frac{\pi}{2}}-\frac{1}{\sqrt{2}}\e^{+\im\frac{\pi}{4}}}{\cos(\frac{\pi}{4})+\im\sin(\frac{\pi}{4})-(\cos(\frac{\pi}{4})-\im\sin(\frac{\pi}{4}))}\e^{+\im\frac{\pi}{4}(k-1)}\nonumber\\
	&=2\cdot\frac{\im-\frac{1}{2}-\frac{1}{2}\im}{2\im\sin(\frac{\pi}{4})}\e^{+\im\frac{\pi}{4}(k-1)}=\frac{-1+\im}{\im\sqrt{2}}\e^{+\im\frac{\pi}{4}(k-1)}
\end{align}
\begin{align}
	\mathrm{Res}(X(z)z^{k-1},\e^{-\im\frac{\pi}{4}})&=\lim\limits_{z\rightarrow\e^{-\im\frac{\pi}{4}}}2\cdot\frac{z\cdot(z-\frac{1}{\sqrt{2}})}{(z-\e^{+\im\frac{\pi}{4}})}z^{k-1}=2\cdot\frac{\e^{-\im\frac{\pi}{4}}\cdot(\e^{-\im\frac{\pi}{4}}-\frac{1}{\sqrt{2}})}{\e^{-\im\frac{\pi}{4}}-\e^{+\im\frac{\pi}{4}}}\e^{-\im\frac{\pi}{4}(k-1)} \nonumber \\
	&=2\cdot\frac{\e^{-\im\frac{\pi}{2}}-\frac{1}{\sqrt{2}}\e^{-\im\frac{\pi}{4}}}{\cos(\frac{\pi}{4})-\im\sin(\frac{\pi}{4})-(\cos(\frac{\pi}{4})+\im\sin(\frac{\pi}{4}))}\e^{-\im\frac{\pi}{4}(k-1)}\nonumber\\
	&=2\cdot\frac{-\im-\frac{1}{2}+\frac{1}{2}\im}{-2\im\sin(\frac{\pi}{4})}\e^{-\im\frac{\pi}{4}(k-1)}=\frac{1+\im}{\im\sqrt{2}}\e^{-\im\frac{\pi}{4}(k-1)}
\end{align}
\begin{align}
	x[k]&=\mathcal{Z}^{-1}[X(z)]=\mathrm{Res}(X(z)z^{k-1},\e^{+\im\frac{\pi}{4}})+\mathcal{Z}^{-1}=\mathrm{Res}(X(z)z^{k-1},\e^{-\im\frac{\pi}{4}})=\frac{-1+\im}{\im\sqrt{2}}\e^{+\im\frac{\pi}{4}(k-1)}+\frac{1+\im}{\im\sqrt{2}}\e^{-\im\frac{\pi}{4}(k-1)}\nonumber\\
	&=\frac{\sqrt{2}}{2}\bigg (\e^{+\im\frac{\pi}{4}(k-1)}+\e^{+\im\frac{\pi}{4}(k-1)}\bigg )-\frac{\sqrt{2}}{2\im} \bigg (\e^{+\im\frac{\pi}{4}(k-1)}+\e^{+\im\frac{\pi}{4}(k-1)}\bigg)\nonumber\\
	&=\sqrt{2}\bigg(\cos(\frac{\pi}{4}(k-1))-\sin(\frac{\pi}{4}(k-1))\bigg)
\end{align}
\subsubsection{Task b)}
We shall find the time-discret signal for
\begin{align}
	X(z)=\frac{z}{z-\frac{1}{2}}.
\end{align}
There is one singularity: $\frac{1}{2}$.
\begin{align}
	\mathrm{Res}(X(z)z^{k-1},\frac{1}{2})=\lim\limits_{z\rightarrow\frac{1}{2}}z\cdot z^{k-1}=\bigg (\frac{1}{2} \bigg)^k
\end{align}
\begin{align}
	x[k]=\mathcal{Z}^{-1}[X(z)]=\mathrm{Res}(X(z)z^{k-1},\frac{1}{2})=\bigg (\frac{1}{2} \bigg)^k
\end{align}
\subsubsection{Task c)}
We shall find the time-discret signal for 
\begin{align}
	X(z)=\frac{z}{z-\frac{1}{2}}\cdot\frac{z}{z-1}.
\end{align}
There are two singularities: $\frac{1}{2}$ and $1$.
\begin{align}
	\mathrm{Res}(X(z)z^{k-1},\frac{1}{2})=\lim\limits_{z\rightarrow\frac{1}{2}}\frac{z^2}{z-1}z^{k-1}=\frac{\bigg (\frac{1}{2}\bigg)^{k+1}}{\frac{1}{2}-1}=-\bigg (\frac{1}{2}\bigg)^k
\end{align}
\begin{align}
	\mathrm{Res}(X(z)z^{k-1},1)=\lim\limits_{z\rightarrow1}\frac{z^2}{z-\frac{1}{2}}z^{k-1}=1^{k+1}{1-\frac{1}{2}}=2
\end{align}
\begin{align}
	x[k]=\mathcal{Z}^{-1}[X(z)]=\mathrm{Res}(X(z)z^{k-1},\frac{1}{2})+\mathrm{Res}(X(z)z^{k-1},1)=2-\bigg (\frac{1}{2}\bigg)^k
\end{align}
\subsubsection{Task d)}
We shall find the time-discret signal for
\begin{align}
	X(z)=\frac{z^2-z+2}{z^2-\frac{1}{2}z+\frac{1}{4}}.
\end{align}
The poles are $\frac{1}{4}+\im\frac{\sqrt{3}}{4}$ and $\frac{1}{4}-\im\frac{\sqrt{3}}{4}$.
\begin{align}
	\mathrm{Res}(X(z)z^{k-1},\frac{1}{4}+\im\frac{\sqrt{3}}{4})&=\lim\limits_{z\rightarrow\frac{1}{4}+\im\frac{\sqrt{3}}{4}}\frac{z^2-z+2}{z-(\frac{1}{4}-\im\frac{\sqrt{3}}{4})}z^{k-1}=\lim\limits_{z\rightarrow\frac{1}{2}\e^{+\im\frac{\pi}{3}}}\frac{z^2-z+2}{z-(\frac{1}{2}\e^{-\im\frac{\pi}{3}})}z^{k-1}\nonumber\\
	&=\frac{\frac{1}{4}\e^{+\im\frac{2\pi}{3}}-\frac{1}{2}\e^{+\im\frac{\pi}{3}}+2}{\frac{1}{2}\e^{+\im\frac{\pi}{3}}-\frac{1}{2}\e^{-\im\frac{\pi}{3}}}\bigg(\frac{1}{2}\bigg )^{k-1}\e^{+\im\frac{\pi}{3}(k-1)}\nonumber\\
	&=\frac{\frac{1}{4}(\cos(\frac{2\pi}{3})+\im\sin(\frac{2\pi}{3}))-\frac{1}{2}(\cos(\frac{\pi}{3})+\im\sin(\frac{\pi}{3}))+2}{\frac{1}{2}(\cos(\frac{\pi}{3})+\im\sin(\frac{\pi}{3})-\cos(\frac{\pi}{3})+\im\sin(\frac{\pi}{3}))}\bigg(\frac{1}{2}\bigg )^{k-1}\e^{+\im\frac{\pi}{3}(k-1)}\nonumber\\
	&=\frac{\frac{1}{4}(-\frac{1}{2}+\im\frac{\sqrt{3}}{2})-\frac{1}{2}(\frac{1}{2}+\im\frac{\sqrt{3}}{2})+2}{\im\frac{\sqrt{3}}{2}}\bigg(\frac{1}{2}\bigg )^{k-1}\e^{+\im\frac{\pi}{3}(k-1)}\nonumber\\
	&=2\cdot\frac{1}{\im\sqrt{3}}\cdot\bigg [-\frac{1}{8}+\im\frac{\sqrt{3}}{8}-\frac{1}{4}-\im\frac{\sqrt{3}}{4}+2\bigg]\bigg(\frac{1}{2}\bigg )^{k-1}\e^{+\im\frac{\pi}{3}(k-1)}\nonumber\\
	&=\frac{1}{\im\sqrt{3}}\cdot\bigg [\frac{13}{4}-\im\frac{\sqrt{3}}{4}\bigg]\bigg(\frac{1}{2}\bigg)^{k-1}\e^{+\im\frac{\pi}{3}(k-1)}
\end{align}
\begin{align}
	\mathrm{Res}(X(z)z^{k-1},\frac{1}{4}-\im\frac{\sqrt{3}}{4})&=\lim\limits_{z\rightarrow\frac{1}{4}-\im\frac{\sqrt{3}}{4}}\frac{z^2-z+2}{z-(\frac{1}{4}+\im\frac{\sqrt{3}}{4})}z^{k-1}=\lim\limits_{z\rightarrow\frac{1}{2}\e^{-\im\frac{\pi}{3}}}\frac{z^2-z+2}{z-(\frac{1}{2}\e^{+\im\frac{\pi}{3}})}z^{k-1}\nonumber\\
	&=\frac{\frac{1}{4}\e^{-\im\frac{2\pi}{3}}-\frac{1}{2}\e^{-\im\frac{\pi}{3}}+2}{\frac{1}{2}\e^{-\im\frac{\pi}{3}}-\frac{1}{2}\e^{+\im\frac{\pi}{3}}}\bigg(\frac{1}{2}\bigg )^{k-1}\e^{-\im\frac{\pi}{3}(k-1)}\nonumber\\
	&=\frac{\frac{1}{4}(\cos(\frac{2\pi}{3})-\im\sin(\frac{2\pi}{3}))-\frac{1}{2}(\cos(\frac{\pi}{3})-\im\sin(\frac{\pi}{3}))+2}{\frac{1}{2}(\cos(\frac{\pi}{3})-\im\sin(\frac{\pi}{3})-\cos(\frac{\pi}{3})-\im\sin(\frac{\pi}{3}))}\bigg(\frac{1}{2}\bigg )^{k-1}\e^{-\im\frac{\pi}{3}(k-1)}\nonumber\\
	&=\frac{\frac{1}{4}(-\frac{1}{2}-\im\frac{\sqrt{3}}{2})-\frac{1}{2}(\frac{1}{2}-\im\frac{\sqrt{3}}{2})+2}{-\im\frac{\sqrt{3}}{2}}\bigg(\frac{1}{2}\bigg )^{k-1}\e^{+\im\frac{\pi}{3}(k-1)}\nonumber\\
	&=2\cdot\frac{1}{\sqrt{3}}\cdot\bigg [\frac{1}{8}+\im\frac{\sqrt{3}}{8}+\frac{1}{4}-\im\frac{\sqrt{3}}{4}-2\bigg]\bigg(\frac{1}{2}\bigg )^{k-1}\e^{-\im\frac{\pi}{3}(k-1)}\nonumber\\
	&=\frac{1}{\im\sqrt{3}}\cdot\bigg [-\frac{13}{4}-\im\frac{\sqrt{3}}{4}\bigg]\bigg(\frac{1}{2}\bigg)^{k-1}\e^{-\im\frac{\pi}{3}(k-1)}
\end{align}
\begin{align}
	x[k]&=\mathcal{Z}^{-1}[X(z)]=\mathrm{Res}(X(z)z^{k-1},\frac{1}{4}+\im\frac{\sqrt{3}}{4})+\mathrm{Res}(X(z)z^{k-1},\frac{1}{4}-\im\frac{\sqrt{3}}{4})\nonumber\\
	&=\frac{1}{\im\sqrt{3}}\cdot\bigg [\frac{13}{4}-\im\frac{\sqrt{3}}{4}\bigg]\bigg(\frac{1}{2}\bigg)^{k-1}\e^{+\im\frac{\pi}{3}(k-1)}+\frac{1}{\im\sqrt{3}}\cdot\bigg [-\frac{13}{4}-\im\frac{\sqrt{3}}{4}\bigg]\bigg(\frac{1}{2}\bigg)^{k-1}\e^{-\im\frac{\pi}{3}(k-1)}\nonumber\\
	&=\bigg (\frac{1}{2}\bigg)^{k-1}\Bigg[ \frac{13}{2\cdot\sqrt{3}}\cdot\frac{1}{2\im}\bigg [\e^{+\im\frac{\pi}{3}(k-1)}-\e^{-\im\frac{\pi}{3}(k-1)}\bigg]-\frac{1}{2}\cdot\frac{1}{2}\bigg [\e^{+\im\frac{\pi}{3}(k-1)}+\e^{-\im\frac{\pi}{3}(k-1)}\bigg]\Bigg ]\nonumber\\
	&=\bigg(\frac{1}{2}\bigg)^k\bigg[\frac{13}{\sqrt{3}}\sin(\frac{\pi}{3}(k-1))-\cos(\frac{\pi}{3}(k-1))\bigg]
\end{align}
\subsubsection{Task e)}
We shall find the time-discret signal for 
\begin{align}
	X(z)=\frac{z^2}{z^2+1}.
\end{align}
The poles are $\im$ and $-\im$.
\begin{align}
	\mathrm{Res}(X(z)z^{k-1},\im)=\lim\limits_{z\rightarrow\im}\frac{z^2}{z+\im}z^{k-1}=\lim\limits_{z\rightarrow\e^{+\im\frac{\pi}{2}}}\frac{z^{k+1}}{z+\im}=\frac{1}{2\im}\e^{+\im\frac{\pi}{2}(k+1)}
\end{align}
\begin{align}
	\mathrm{Res}(X(z)z^{k-1},-\im)=\lim\limits_{z\rightarrow-\im}\frac{z^2}{z-\im}z^{k-1}=\lim\limits_{z\rightarrow\e^{-\im\frac{\pi}{2}}}\frac{z^{k+1}}{z-\im}=\frac{1}{-2\im}\e^{-\im\frac{\pi}{2}(k+1)}
\end{align}
\begin{align}
	x[k]=\mathrm{Res}(X(z)z^{k-1},\im)+\mathrm{Res}(X(z)z^{k-1},-\im)=\frac{1}{2\im}\bigg(\e^{+\im\frac{\pi}{2}(k+1)}-\e^{-\im\frac{\pi}{2}(k+1)}\bigg )=\sin(\frac{\pi}{2}(k+1))
\end{align}
\subsubsection{Task f)}
We shall find the time discret signal for
\begin{align}
	X(z)=\frac{z^4+z^3-6z^2+6z-1}{z^2-2z+1}.
\end{align}
$1$ is a pole of order 2.
\begin{align}
	\mathrm{Res}(X(z)z^{k-1},1)&=\lim\limits_{z\rightarrow1}\frac{\mathrm{d}}{\mathrm{d}z}\bigg [(z^4+z^3-6z-1)z^{k-1}\bigg]\nonumber\\
	&=\lim\limits_{z\rightarrow1}\frac{\mathrm{d}}{\mathrm{d}z}\bigg[z^{k+3}+z^{k+2}-6z^k-z^{k-1}\bigg]\nonumber\\
	&=\lim\limits_{z\rightarrow1}\bigg[(k+3)z^{k+2}+(k+2)z^{k+1}-6kz^{k-1}-(k-1)z^{k-2}\bigg ]\nonumber\\
	&=(k+3)+(k+2)-6k-(k-1)=-5k+6
\end{align}
\begin{align}
	x[k]=\mathcal{Z}^{-1}[X(z)]=\mathrm{Res}(X(z)z^{k-1},1)=6-5k
\end{align}
\subsubsection{Task g)}
We shall find the time-discret signal for
\begin{align}
	X(z)=\frac{z\cdot(z-1)}{z^2-\sqrt{2}z+1}.
\end{align}
The poles are $\frac{\sqrt{2}}{2}+\im\frac{\sqrt{2}}{2}$ and $\frac{\sqrt{2}}{2}-\im\frac{\sqrt{2}}{2}$.
\begin{align}
	\mathrm{Res}(X(z)z^{k-1},\frac{\sqrt{2}}{2}+\im\frac{\sqrt{2}}{2})&=\lim\limits_{z\rightarrow\frac{\sqrt{2}}{2}+\im\frac{\sqrt{2}}{2}}\frac{z^2-z}{z-(\frac{\sqrt{2}}{2}-\im\frac{\sqrt{2}}{2})}z^{k-1}=\lim\limits_{z\rightarrow\e^{+\im\frac{\pi}{4}}}\frac{z^2-z}{z-\e^{-\im\frac{\pi}{4}}}z^{k-1}\nonumber\\
	&=\frac{\e^{+\im\frac{\pi}{2}}-\e^{+\im\frac{\pi}{4}}}{\e^{+\im\frac{\pi}{4}}-\e^{-\im\frac{\pi}{4}}}\e^{+\im\frac{\pi}{4}(k-1)}=\frac{\im-\cos(\frac{\pi}{4})-\im\sin(\frac{\pi}{4})}{\cos(\frac{\pi}{4})+\im\sin(\frac{\pi}{4})-\cos(\frac{\pi}{4})+\im\sin(\frac{\pi}{4})}\e^{+\im\frac{\pi}{4}(k-1)}\nonumber\\
	&=\frac{(1-\frac{\sqrt{2}}{2})-\im\frac{\sqrt{2}}{2}}{\im\sqrt{2}}\e^{+\im\frac{\pi}{4}(k-1)}
\end{align}
\begin{align}
	\mathrm{Res}(X(z)z^{k-1},\frac{\sqrt{2}}{2}-\im\frac{\sqrt{2}}{2})&=\lim\limits_{z\rightarrow\frac{\sqrt{2}}{2}-\im\frac{\sqrt{2}}{2}}\frac{z^2-z}{z-(\frac{\sqrt{2}}{2}+\im\frac{\sqrt{2}}{2})}z^{k-1}=\lim\limits_{z\rightarrow\e^{-\im\frac{\pi}{4}}}\frac{z^2-z}{z-\e^{+\im\frac{\pi}{4}}}z^{k-1}\nonumber\\
	&=\frac{\e^{-\im\frac{\pi}{2}}-\e^{-\im\frac{\pi}{4}}}{\e^{-\im\frac{\pi}{4}}-\e^{+\im\frac{\pi}{4}}}\e^{-\im\frac{\pi}{4}(k-1)}=\frac{-\im-\cos(\frac{\pi}{4})+\im\sin(\frac{\pi}{4})}{\cos(\frac{\pi}{4})-\im\sin(\frac{\pi}{4})-\cos(\frac{\pi}{4})-\im\sin(\frac{\pi}{4})}\e^{+\im\frac{\pi}{4}(k-1)}\nonumber\\
	&=-\frac{(1-\frac{\sqrt{2}}{2})-\im\frac{\sqrt{2}}{2}}{\im\sqrt{2}}\e^{+\im\frac{\pi}{4}(k-1)}
\end{align}
\begin{align}
	x[k]=\mathcal{Z}^{-1}&=\mathrm{Res}(X(z)z^{k-1},\frac{\sqrt{2}}{2}+\im\frac{\sqrt{2}}{2})+\mathrm{Res}(X(z)z^{k-1},\frac{\sqrt{2}}{2}-\im\frac{\sqrt{2}}{2})\nonumber\\
	&=\frac{(1-\frac{\sqrt{2}}{2})-\im\frac{\sqrt{2}}{2}}{\im\sqrt{2}}\e^{+\im\frac{\pi}{4}(k-1)}-\frac{(1-\frac{\sqrt{2}}{2})-\im\frac{\sqrt{2}}{2}}{\im\sqrt{2}}\e^{+\im\frac{\pi}{4}(k-1)}
\end{align}
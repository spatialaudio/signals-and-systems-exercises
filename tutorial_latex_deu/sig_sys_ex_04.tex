\clearpage
\section{UE 4: Fourier Transformation}
\label{sec:ue4_fouriertransformation}

\textbf{Definitionsgleichungen} der Integraltransformation (SigSys Konvention:
exp(-...) bei der Transformation in den Bildbereich, $2\pi$-Normierung und exp(+...) bei der
Transformation in den Zeitbereich)
\begin{align}
X(\omega) = \int\limits_{-\infty}^{+\infty} x(t) \e^{-\im \omega t} \fsd t
&\qquad
x(t)= \frac{1}{2\pi}\int\limits_{-\infty}^{+\infty} X(\omega) \e^{+\im \omega t} \fsd \omega
\end{align}
Wir benutzen sehr häufig die verkürzte (weitgehend standardisierte)
\textbf{Operator-Schreibweise}
\begin{align}
X(\omega) = \mathcal{F}\{x(t)\}&\qquad x(t) = \mathcal{F}^{-1}\{X(\omega)\}\\
x(t) \,\fourier\, X(\omega) &\qquad X(\omega) \,\Fourier\, x(t)
\end{align}
Hinweis: Hintrafo Laplace $\laplace$, Rücktrafo Laplace $\Laplace$, Hintrafo Fourier
$\fourier$, Rücktrafo Fourier $\Fourier$.
Wir finden oft auch $\laplace$ bei der Fourier Trafo, dann muss im Kontext klar werden,
welche Trafo benutzt wird. In der Formelsammlung und Übung~\ref{sec:ue1_intro} (1)
ist das z.B. eindeutig, weil
wir dort keine Laplace, sondern nur die Fourier Transformation verwenden.
%

\noindent \textbf{Wichtigste Eigenschaften und Korrespondenzen}
\begin{itemize}
\item Linearität (Addition, Superposition)
\begin{align}
a x_1(t) + b x_2(t) &\quad \fourier \quad a X_1(\omega) + b X_2(\omega)
\end{align}

\item Dualität Zeit vs. Frequenzskalierung mit $a\in\mathbb{R}$
\begin{align}
x(a \cdot t) &\quad \fourier \quad \frac{1}{|a|}X\left(\frac{\omega}{a}\right)
\end{align}

\item Dualität Faltung vs. Multiplikation
\begin{align}
x_1(t) \ast_t x_2(t) &\quad \fourier \quad X_1(\omega) \cdot X_2(\omega)\\
x_1(t) \cdot x_2(t) &\quad \fourier \quad \frac{1}{2\pi} X_1(\omega) \ast_\omega X_2(\omega)
\end{align}

\item Dualität Verschiebung vs. Multiplikation mit komplexem Exponentialsignal
\begin{align}
x(t-\tau) &\quad \fourier \quad \e^{-\im\omega\tau} \cdot X(\omega)\\
\e^{+\im\omega_0 t} \cdot x(t) &\quad \fourier \quad X(\omega-\omega_0)
\end{align}

\item Dirac Zeitskalierung vs. Gewichtskalierung
mit $a,b\in\mathbb{R}$
\begin{align}
\delta(t) =& |a| \cdot \delta(a t)\\
\delta(t-b) = & |a| \cdot \delta(a [t-b])
\end{align}

\item Verschiebung und Skalierung in der Zeit vs. Modulation und Skalierung in der Frequenz
mit $a,b\in\mathbb{R}$
\begin{align}
x(\frac{t}{a}-b)  = x(\frac{t-a b}{a}) \quad\fourier\quad
|a| X(a \cdot \im\omega) \e^{-\im\omega a b}\\
x(a t - b)  = x(a [t-\frac{b}{a}]) \quad\fourier\quad
\frac{1}{|a|} X(\frac{\omega}{a}) \e^{-\im\omega \frac{b}{a}}
\end{align}

\item Dualität Dirac Impuls vs. Gleichsignal
\begin{align}
\delta(t) &\quad \fourier \quad 1\\
1 &\quad \fourier \quad 2\pi\delta(\omega)
\end{align}

\item Dualität Rect vs. Sinc ($T\in\mathbb{R}$, $\mathrm{sinc}(\nu) := \frac{\sin(\nu)}{\nu}$)
\begin{align}
\frac{1}{T} \mathrm{rect}(\frac{t}{T}) &\quad \fourier \quad \frac{|T|}{T} \mathrm{sinc}\left(\frac{\omega T}{2}\right)\\
\mathrm{sinc}(\frac{t}{T}) &\quad \fourier \quad \pi |T| \mathrm{rect}\left(\frac{\omega T}{2}\right)
\end{align}

\item Dualität Gaussglocke für $\Re\{T^{-2}\}>0$, Spektrum ohne Nebenzipfel bzw. ohne Seitenkeulen
\begin{align}
\e^{-\left(\frac{t}{T}\right)^2} \quad \fourier \quad \sqrt{\pi T^2} \e^{-\left(\frac{\omega T}{2}\right)^2}
\end{align}


\item komplexe Exponentialschwingung
\begin{align}
\e^{+\im\omega_0 t} &\quad \fourier \quad 2\pi\delta(\omega - \omega_0)
\end{align}

\item Cosinus aus komplexer Exponentialschwingung
\begin{align}
\frac{1}{2} \e^{+\im\omega_0 t} + \frac{1}{2}  \e^{-\im\omega_0 t} &\quad \fourier \quad \frac{1}{2} \cdot 2\pi\delta(\omega -\omega_0) +\frac{1}{2} \cdot 2\pi\delta(\omega + \omega_0)
\end{align}

\item Sinus aus komplexer Exponentialschwingung
\begin{align}
\frac{1}{2\im} \e^{+\im\omega_0 t} - \frac{1}{2\im}  \e^{-\im\omega_0 t} &\quad \fourier \quad \frac{1}{2\im} \cdot 2\pi\delta(\omega -\omega_0) - \frac{1}{2\im} \cdot 2\pi\delta(\omega + \omega_0)
\end{align}
\end{itemize}



\newpage
\begin{figure}[h!]
\centering
%
\begin{tikzpicture}[scale=0.75]
\draw[->] (-2.1 ,0) -- (2.1,0) node[right]{$t$};
\draw[->] (0,-1.1) -- (0,1.25) node[above]{$x_r(t)=\cos(\omega_0 t)$};
\begin{scope}
\draw[C2, ultra thick, domain=-1.9:1.8,variable=\t,samples=100,smooth] plot(\t,{ cos (2*pi*\t r)});
\node at (-0.25,1) {$1$};
\node at (3,0) {$\fourier$};
\end{scope}
\begin{scope}[shift={(6,0)}]
\draw[->] (-2.1,0) -- (2.1,0) node[right] {$\omega$};
\draw[->] (0,-1.1) -- (0,1.25) node[above] {$X_r(\omega)=\textcolor{C1}{\pi\delta(\omega+\omega_0)}+\textcolor{C0}{\pi\delta(\omega-\omega_0)}$};
\draw[->, C1, line width=1mm] (-1,0) -- (-1,1) node[left] {$(\pi)$};
\draw[->, C0, line width=1mm] (+1,0) -- (+1,1) node[left] {$(\pi)$};
\node at (-1,-0.25) {$-\omega_0$};
\node at (+1,-0.25) {$+\omega_0$};
\end{scope}
\end{tikzpicture}
%
%
%
%
\begin{tikzpicture}[scale=0.75]
\draw[->] (-2.1 ,0) -- (2.1,0) node[right]{$t$};
\draw[->] (0,-1.1) -- (0,1.25) node[above]{$x_i(t)=\sin(\omega_0 t)$};
\begin{scope}
\draw[C3, ultra thick, domain=-1.9:1.8,variable=\t,samples=100,smooth] plot(\t,{ sin (2*pi*\t r)});
\node at (3,0) {$\fourier$};
\node at (-0.25,1) {$1$};
\end{scope}
\begin{scope}[shift={(6,0)}]
\draw[->] (-2.1,0) -- (2.1,0) node[right] {$\omega$};
\draw[->] (0,-1.1) -- (0,1.25) node[above] {$X_i(\omega)=\textcolor{C1}{\im\pi\delta(\omega+\omega_0)}-\textcolor{C0}{\im\pi\delta(\omega-\omega_0)}$};
\draw[->, C1, line width=1mm] (-1,0) -- (-1,+1) node[left] {$(\im\pi)$};
\draw[->, C0, line width=1mm] (+1,0) -- (+1,-1) node[left] {$(\im\pi)$};
\node at (-1,-0.25) {$-\omega_0$};
\node at (+1,+0.25) {$+\omega_0$};
\end{scope}
\end{tikzpicture}
%
%
%
\caption{Korrespondenzen der Fourier Transformation für Cosinus (oben) und
Sinus (unten) schematisch für $\omega_0>0$.
Wir können uns in den Grafiken klarmachen, dass für $\omega_0=0$ die Korrespondenzen
$x_r(t)=1 \, \fourier \, X_r(\omega)=2\pi\delta(\omega)$ und
$x_i(t)=0 \, \fourier \, X_i(\omega)=0$ gelten.}
\label{fig:ue4_cos_sin_ft}
\end{figure}










\begin{figure}[h!]
\centering
%
\begin{tikzpicture}[scale=0.75]
\draw[->] (-2.1 ,0) -- (2.1,0) node[right]{$t$};
\draw[->] (0,-1.1) -- (0,1.25) node[above]{$x(t)=\textcolor{C2}{\cos(\omega_0 t)} + \im\,\textcolor{C3}{\sin(\omega_0 t)}$};
\begin{scope}
\draw[C2, ultra thick, domain=-1.9:1.8,variable=\t,samples=100,smooth] plot(\t,{ cos (2*pi*\t r)});
\draw[C3, ultra thick, domain=-1.9:1.8,variable=\t,samples=100,smooth] plot(\t,{ sin (2*pi*\t r)});
\node at (3,0) {$\fourier$};
\node at (-0.25,1) {$1$};
\end{scope}
\begin{scope}[shift={(6,0)}]
\draw[->] (-2.1,0) -- (2.1,0) node[right] {$\omega$};
\draw[->] (0,-1.1) -- (0,1.25) node[above] {$X(\omega)=\textcolor{C0}{2\pi\delta(\omega-\omega_0)}$};
\draw[->, C0, line width=1mm] (+1,0) -- (+1,1) node[right] {$(2\pi)$};
\node at (+1,-0.25) {$+\omega_0$};
\end{scope}
\end{tikzpicture}
%
%
%
\caption{Korrespondenz der Fourier Transformation für komplexe Exponentialschwingung
$\e^{\im\omega_0 t}$ für $\omega_0>0$.
Wir können uns mit dieser Grafik klarmachen, dass für $\omega_0=0$ die Korrespondenz
$x(t)=1 \, \fourier \, X(\omega)=2\pi\delta(\omega)$ gilt.}
\label{fig:ue4_exp_ft}
\end{figure}


\clearpage
\subsection{Fourier-Transformation Beispiel Rechteckfunktion}
\label{sec:8C3958BE4F}
\begin{Ziel}
Wir wollen für die Fouriertransformation die wichtige Dualität zwischen der
Rechteckfunktion und der Spaltfunktion anhand eines fundamentalen Beispiels erfassen.
Zudem enthält auch hier die ausführliche Rechnung Schritte und Umformungen
welche wir wiederkehrend brauchen werden.
\end{Ziel}
\textbf{Aufgabe} {\tiny 8C3958BE4F}: Berechnen Sie die Fourier-Transformation für
\begin{itemize}
\item axialsymmetrische Rechteckfunktion der Breite $T_h$ und Amplitude $A$
\end{itemize}

\begin{figure}[h!]
\centering
\begin{tikzpicture}[domain=0:2]
\draw[->] (-2,0) -- (2,0) node[below right] {$t$};
\draw[->] (0,0) -- (0,1.5) node[above] {$x(t)$};
\draw[-, C0, ultra thick] (-2,0) -- (-0.5,0) node[below] {$\frac{-T_h}{2}$} -- (-0.5,1) node[left] {$A$} -- (0.5,1) -- (0.5,0) node[below] {$\frac{+T_h}{2}$} -- (2,0);
\end{tikzpicture}
\end{figure}

\begin{Werkzeug}
Synthese mit Fouriertransformation
\begin{align}
x(t) = \frac{1}{2\pi} \int\limits_{-\infty}^{+\infty} X(\omega) \, \e^{+\im \omega t} \fsd \omega
\end{align}
%
Analyse mit Fouriertransformation
\begin{align}
X(\omega) = \int\limits_{-\infty}^{+\infty} x(t) \, \e^{-\im \omega t} \fsd t
\end{align}
\end{Werkzeug}

\begin{Ansatz}
\begin{equation}
X(\omega) = \int\limits_{-T_h/2}^{+T_h/2} A \e^{-\im \omega t} \mathrm{d}t
\end{equation}
\end{Ansatz}

\begin{ExCalc}
Integrieren
\begin{equation}
X(\omega) = A \frac{\e^{-\im \omega t}}{-\im \omega} \bigg|_{-T_h/2}^{+T_h/2}
\end{equation}
Grenzen einsetzen
\begin{equation}
X(\omega) = A \frac{\e^{-\im \omega T_h/2}-\e^{+\im \omega T_h/2}}{-\im \omega}
\end{equation}
Vorzeichen
\begin{equation}
X(\omega) = A \frac{\e^{+\im \omega T_h/2}-\e^{-\im \omega T_h/2}}{\im \omega}
\end{equation}
Erweitern mit dem Ziel $\sin(x) = \frac{\e^{+\im x}-\e^{-\im x}}{2\im}$, $\mathrm{sinc}(x):=\frac{\sin(x)}{x}$
\begin{equation}
X(\omega) = A \frac{\e^{+\im \omega T_h/2}-\e^{-\im \omega T_h/2}}{\im \omega} \frac{T_h/2}{T_h/2} \cdot \frac{2}{2}
\end{equation}
Umsortieren
\begin{equation}
X(\omega) = A \frac{\e^{+\im \omega T_h/2}-\e^{-\im \omega T_h/2}}{2 \im \omega T_h/2} \frac{T_h/2}{1} \cdot \frac{2}{1}
\end{equation}
\end{ExCalc}

\begin{Loesung}
\begin{equation}
\label{eq:8C3958BE4F_Loesung}
X(\omega) = A T_h \cdot \mathrm{sinc}(\omega \frac{T_h}{2})
\end{equation}

Wir bekommen wieder eine Spaltfunktion, wie schon bei der Fourierreihe aus Aufgabe
\ref{sec:D1483A84E2}, vgl. \eq{eq:D1483A84E2_Loesung}.
%
Diesmal, weil die Frequenzvariable $\omega$ kontinuierlich ist,
wird die Spaltfunktion 'überall' ausgewertet,
die Fouriertransformation ist also eine kontinuierliche Funktion über $\omega$.
%
Auch hier ist $X(\omega)\in\mathbb{R}$ ein Spezialfall.
%
Der Faktor $\frac{T_h}{2}$ im sinc-Argument bestimmt wieder die Breite der
Spaltfunktion.

Wir merken uns bzw. besser wir sehen das in den Formeln:

Je \textbf{kleiner} $T_h$, also je \textbf{schmal}er der
\textbf{Rechteck}impuls, desto \textbf{breit}er die \textbf{Spalt}funktion.

Je \textbf{größer} $T_h$, also je \textbf{breit}er der
\textbf{Rechteck}impuls, desto \textbf{schmal}er die \textbf{Spalt}funktion.
%
Dieser Zusammenhang ist fundamental in der Signalverarbeitung und der
Kommunikationstechnik und wurde in \fig{fig:8C3958BE4F} grafisch aufbereitet.

Was erwarten wir für $T_h\to 0$ (idealer Impuls) und was für $T_h\to \infty$
(Gleichspannung)? In beiden Fällen bekommen
wir Probleme beim Integrieren. Aber dies sind in der SigSys überaus wichtige
Grenzfälle, für die auch Fouriertransformierte existieren. Wir werden sie bald
kennenlernen.

Wir werden in der Vorlesung und Übung sehr oft das Elementarsignal

\begin{equation}
\text{rect}(t) := \begin{cases} 1 & |t| < \frac{1}{2} \\ \frac{1}{2} & |t| = \frac{1}{2} \\ 0 & |t| > \frac{1}{2} \end{cases}\quad,
\end{equation}
also den Rechteckimpuls mit $T_h=1$ und $A=1$, zum Rechnen benutzen.
%
Die Fouriertransformierte von $\text{rect}(t)$ lautet dann abgeleitet aus
\eq{eq:8C3958BE4F_Loesung} einfach
$X(\omega) = \mathrm{sinc}(\frac{\omega}{2})$.
%
Eine oft benutzte Schreibweise von Fouriertransformationspaaren (sogenannte
Korrespondenzen, siehe Formelsammlung) benutzt den Operator $\laplace$.
%
Wir schreiben für das soeben gefundene Transformationspaar
\begin{equation}
x(t) = \text{rect}(t) \quad \laplace \quad X(\omega) = \mathrm{sinc}(\frac{\omega}{2})
\end{equation}
%
Eine weitere Korrespondenz ist
\begin{equation}
x(t) = \text{sinc}(t) \quad \laplace \quad X(\omega) = \pi \, \mathrm{rect}(\frac{\omega}{2}),
\end{equation}
also bis auf den Faktor $\pi$ vertauscht.
%Diese Korrespondenz direkt zu
%beweisen, also die Spaltfunktion fourierzutransformieren ist nicht trivial, aber
%man kann die inverse Fourier Transformation berechnen. Das ist wieder nur der
%Rechteckimpuls unter dem Integralkern.
%
Die Dualität, also \textbf{Spaltfunktion transformiert ergibt Rechteckfunktion und umgekehrt},
dürfen wir die ganze Zeit nicht aus den Augen verlieren. Das ist so ein bisschen
wie der Satz des Pythagoras bei rechtwinkligen Dreiecken, ohne den wird's auch eher
dünne Rechnerei mit wenig Vertändnis.

Schauen wir noch kurz zurück in die \fig{fig:D1483A84E2_0} bis \fig{fig:D1483A84E2_4}.
Dort wurde in grau immer der Verlauf der hier diskutierten Fouriertransformierten
hinterlegt. Wir sehen, dass die Werte der Fourierkoeffizienten $X_\nu$
\eq{eq:D1483A84E2_Loesung} aus Aufgabe \ref{sec:D1483A84E2}
mit den Werten von $X(\omega)$ \eq{eq:8C3958BE4F_Loesung} übereinstimmen bei
$\frac{2\pi}{T} \nu = \nu \cdot \omega_0 = \omega$.
Dies ist kein Zufall. Wir werden das später als Abtastung von Spektren kennenlernen.



\end{Loesung}

\begin{figure*}[h!]
\centering
\begin{subfigure}{1\textwidth}
\centering
\includegraphics[width=\textwidth]{../ft/8C3958BE4F_1.pdf}
\caption{Rechteckimpulse für verschiedene $T_h$ und gewähltem $A=1/T_h$.
Achtung: Um die resultierenden Amplitudenunterschiede sinnvoll darstellen zu können,
ist $\log_{10}x(t)$ über $t$ aufgetragen. Da $\log_{10}(0)=-\infty$
ist nur der Signalteil mit (positiver) Amplitude $A=1/T_h$ sichtbar.}
\label{fig:8C3958BE4F_1}
\end{subfigure}
\\
\begin{subfigure}{1\textwidth}
\centering
\includegraphics[width=\textwidth]{../ft/8C3958BE4F_0.pdf}
\caption{Fouriertransformation der Rechteck-Impulse aus
\fig{fig:8C3958BE4F_1}. Durch die Wahl $A=1/T_h$ haben alle Rechteckimpulse Fläche
1 und alle Fouriertransformierten
das gleiche globale Maximum (bei $\omega=0$ rad/s) mit Wert 1.}
\label{fig:8C3958BE4F_0}
\end{subfigure}
%
\caption{Rechteckimpulse (oben) und Fouriertransformierte (unten).
Liniendickenvariation zu besseren Unterscheidbarkeit bei Graustufenanzeige.
\texttt{FourierTransformation\_8C3958BE4F.ipynb}}
\label{fig:8C3958BE4F}
\end{figure*}


\begin{figure}[h!]
\includegraphics[width=\textwidth]{../ft/8C3958BE4F_SingleCase_MagPhase.pdf}
  \caption{Betrag (oben) und Phase (unten) der rellwertigen
  Fouriertransformierten \eq{eq:8C3958BE4F_Loesung}. Für eine Variable
  $a\in\mathbb{R}$ gilt $a = -a \e^{\pm\im\pi}$, daher
  die Phasensprünge um $\pm\pi$ zum punktsymmetrisches Spektrum
  bei negativer Amplitude der Spaltfunktion.
\texttt{FourierTransformation\_8C3958BE4F.ipynb}}
  \label{fig:8C3958BE4F_SingleCase_MagPhase}
\end{figure}




\clearpage
\subsection{Fourier-Transformation Signal-Verschiebung}
\label{sec:A8A2DEE53A}
\begin{Ziel}
Wir wollen uns erarbeiten, wie sich eine Zeitverschiebung des Signals auf die
Fouriertransformierte auswirkt und damit das sogenannte Verschiebungsgesetz
kennenlernen und/oder wiederholen.
Merke bzw. besser sieh in den Formeln:
Zeitverschiebung bedeutet immer Phasenänderung in der Fouriertransformierten,
also im Spektrum. Das gilt allgemein für alle anderen Fourier-Transformationen
die wir noch kennenlernen werden, lohnt sich also vom Wesen zu verstehen.
\end{Ziel}
\textbf{Aufgabe} {\tiny A8A2DEE53A}: Berechnen Sie die Fourier-Transformation für
\begin{itemize}
\item dargestellte Rechteckfunktion der Breite $T_h$ und Amplitude $A$
\end{itemize}
%
\begin{figure}[h!]
\centering
\begin{tikzpicture}[domain=0:2]
\draw[->] (-2,0) -- (2,0) node[below right] {$t$};
\draw[->] (0,0) -- (0,1.5) node[above] {$x(t)$};
\draw[-, C0, ultra thick] (-2,0) -- (-0.0,0) node[below] {$0$} -- (-0,1) node[left] {$A$} -- (1,1) -- (1,0) node[below] {$T_h$} -- (2,0);
\end{tikzpicture}
\end{figure}
%
Wir erkennen die in Aufgabe \ref{sec:8C3958BE4F} hier um $T_h/2$, also nach rechts,
verschobene Rechteckfunktion.


\begin{Werkzeug}
nothing really new here...wir machen
Analyse mit Fouriertransformation
\begin{align}
X(\omega) = \int\limits_{-\infty}^{+\infty} x(t) \, \e^{-\im \omega t} \fsd t
\end{align}
\end{Werkzeug}
\begin{Ansatz}
\begin{equation}
X(\omega) = \int\limits_{0}^{T_h} A \e^{-\im \omega t} \mathrm{d}t
\end{equation}
\end{Ansatz}


\begin{ExCalc}
Integrieren
\begin{equation}
X(\omega) = A \frac{\e^{-\im \omega t}}{-\im \omega} \bigg|_{0}^{T_h}
\end{equation}
Grenzen einsetzen
\begin{equation}
X(\omega) = A \frac{\e^{-\im \omega T_h}-1}{-\im \omega}
\end{equation}
Vorzeichen
\begin{equation}
X(\omega) = A \frac{1 - \e^{-\im \omega T_h}}{\im \omega}
\end{equation}
Erweitern mit dem Ziel $\sin(x) = \frac{\e^{+\im x}-\e^{-\im x}}{2\im}$, $\mathrm{sinc}(x):=\frac{\sin(x)}{x}$
\begin{equation}
X(\omega) = A \frac{1 - \e^{-\im \omega T_h}}{\im \omega} \underbrace{\e^{+\im\omega\frac{T_h}{2}} \e^{-\im\omega\frac{T_h}{2}}}_{=1}=
A \frac{\e^{+\im\omega\frac{T_h}{2}} - \e^{-\im\omega\frac{T_h}{2}}}{\im\omega} \e^{-\im\omega\frac{T_h}{2}}
\end{equation}
\begin{equation}
X(\omega) = A \frac{\e^{+\im\omega\frac{T_h}{2}} - \e^{-\im\omega\frac{T_h}{2}}}{\im\omega} \e^{-\im\omega\frac{T_h}{2}}
\cdot \frac{T_h/2}{T_h/2} \cdot \frac{2}{2}
\end{equation}
Umsortieren
\begin{equation}
X(\omega) = A T_h \frac{\e^{+\im\omega\frac{T_h}{2}} - \e^{-\im\omega\frac{T_h}{2}}}{2 \im \cdot \omega T_h/2} \e^{-\im\omega\frac{T_h}{2}} =
A T_h \frac{\sin(\omega\frac{T_h}{2})}{\omega\frac{T_h}{2}} \e^{-\im\omega\frac{T_h}{2}}
\end{equation}
\end{ExCalc}


\begin{Loesung}
\begin{equation}
\label{eq:A8A2DEE53A_Loesung}
X(\omega) = A T_h \cdot \mathrm{sinc}(\omega\frac{T_h}{2}) \cdot  \e^{-\im\omega\frac{T_h}{2}}
\end{equation}
%
Wir erkennen, dass \eq{eq:8C3958BE4F_Loesung} um den komplexen Dreher / Zeiger $\e^{-\im\omega\frac{T_h}{2}}$
erweitert wird und nun $X(\omega)\in\mathbb{C}$.
%
Wir müssen bei grafischen Darstellungen daher eigentlich streng immer mindest
zwei Diagramme anfertigen:
\begin{itemize}
  \item Realteil + Imaginärteil und/oder
  \item Betrag + Phase
\end{itemize}
%
Eine Darstellung in Betrag und Phase ist üblich, weil dies bei praktischen
Problemen, schnell hilfreiche Informationen offenbart. Das haben wir auch schon
bei der komplexen Wechselstromrechnung benutzt.


Im Gegensatz zum vorherigen Beispiel und \fig{fig:8C3958BE4F_SingleCase_MagPhase},
wo $X(\omega)\in\mathbb{R}<0$ Phasensprünge um $\pi$ verursacht haben, müssen
wir hier eine stückweise stetige, linear Phasenfunktion $\phi(\omega)$ berücksichtigen.
Schauen wir schonmal auf \fig{fig:A8A2DEE53A} unten, um uns ein Bild zu machen.
%
Der Anstieg der Geradenteile lässt sich direkt aus dem komplexen Dreher rauslesen
zu $-\frac{T_h}{2}$.
%
Für die Fouriertransformierte in Betrag und Phase dargestellt, gilt mit $m\in\mathbb{Z}$
\begin{equation}
X(\omega) = \bigg|A T_h \cdot \mathrm{sinc}(\omega\frac{T_h}{2})\bigg|
\cdot  \e^{-\im\omega\frac{T_h}{2}}\cdot
\begin{cases}
\e^{\im (0+2\pi \cdot m)}\quad \text{für}\quad \mathrm{sinc}(\omega\frac{T_h}{2})\geq0\\
\e^{\im (\pi + 2\pi \cdot m)}\quad \text{für}\quad\mathrm{sinc}(\omega\frac{T_h}{2})<0.
\end{cases}
\end{equation}
Überall da wo die Spaltfunktion einen Polaritärswechsel macht, entsteht
eine Phasen-Unstetigkeitstelle, weil die Phase springt um $\pi$ bzw. wegen der
$2\pi$-Vieldeutigkeit um $\pi + 2\pi m$.
%
Die Phasenfunktion ist daher
\begin{equation}
\phi(\omega) = -\omega\frac{T_h}{2} +
\begin{cases}
2\pi m\quad \text{für}\quad \mathrm{sinc}(\omega\frac{T_h}{2})\geq0\\
\pi + 2\pi m\quad \text{für}\quad\mathrm{sinc}(\omega\frac{T_h}{2})<0,
\end{cases}
\end{equation}
Die numpy-Funktion \verb|angle()| wertet die Phase zwischen $-\pi$ und $+\pi$ aus,
deswegen bekommen wir den in \fig{fig:A8A2DEE53A} dargestellten Verlauf der Phase.

Daher nochmal: Jenseits der Phasensprünge, ist zu realisieren, dass
eine zeitliche Verschiebung um $\frac{T_h}{2}$ in der Fouriertransformierten
zu einer Phasenänderung mit Anstieg $-\frac{T_h}{2}$ (negativ, fallend)
führt und der Betrag erhalten bleibt.
%
In \fig{fig:A8A2DEE53A} ist daher das Verhalten von \eq{eq:A8A2DEE53A_Loesung} als
Betrag und Phase über $\omega$ dargestellt.

Der Betrag der Spaltfunktion ist einfach zu erhalten: wir klappen einfach alle
negativen Funktionswerte in Richtung positiver Ordinate.
%
Die Phase ist die stückweise stetig Gerade über $\omega$ durch den Ursprung
mit Anstieg $-T_h/2$, also fallend.
%
Diese \textbf{Phase}nfunktion ist bzgl. des Ursprungs \textbf{punktsymmetrisch}.
Dies gilt immer für die Fouriertransformierten $X(\omega)$ von \textbf{reellwertigen}
Signalen $x(t)$.
%
Größere Zeitverschiebung $T_h/2$ bedeutet größere Phasenverschiebung, also steilere
fallende Geradenstücke.

Das was wir anhand eines speziellen (wichtigen) Beispiels gezeigt haben, gilt auch
allgemein. Die Korrespondenz zur \textbf{Zeitverschiebung} um $\tau\in\mathbb{R}$
---bekannt als \textbf{Verschiebungssatz}---lautet
\begin{align}
x(t) & \quad \laplace \quad X(\omega)\\
x(t - \tau) & \quad \laplace \quad X(\omega) \cdot \e^{-\im\omega \tau}.
\end{align}
%
Man kann natürlich auch eine Fouriertransformierte um die Kreisfrequenz
$\omega_0\in\mathbb{R}$ verschieben, vgl. Aufgabe \ref{sec:9D652BE72B}.
%
Die Korrespondenz zur \textbf{Frequenzverschiebung}
---bekannt als \textbf{Modulationssatz}---lautet
\begin{align}
x(t) & \quad \laplace \quad X(\omega)\\
\label{eq:A8A2DEE53A_ModTheoremTime}
x(t) \cdot \e^{+\im\omega_0 t} & \quad \laplace \quad X(\omega-\omega_0).
\end{align}
%
Bemerkenswert ist der gekreuzte Zusammenhang Zeit-/ Frequenzverschiebung
vs.
Frequenz-/ Zeitmodulation,
jedoch mit anderem Vorzeichen
im komplexen Dreher.
\end{Loesung}

\begin{figure}[h!]
\centering
\includegraphics[width=0.9\textwidth]{../ft/A8A2DEE53A.pdf}
  \caption{Betrag (oben) und Phase (unten) der Fouriertransformierten \eq{eq:A8A2DEE53A_Loesung}.
\textbf{Vorsicht:} Die Grafik suggeriert, dass eine zeitliche Rechts-Verschiebung
sowohl Phase als auch Betrag ändert. Dies passiert aber hier speziell, weil wir in der
Aufgabe nach einer Verschiebung um $T_h/2$ gefragt haben, und $T_h$
im Argument der Spaltfunktion die Breite dieser bestimmt.
\texttt{FourierTransformation\_A8A2DEE53A.ipynb}}
  \label{fig:A8A2DEE53A}
\end{figure}


\clearpage
\subsection{Fourier-Transformation Signal-Zeitumkehr}
\label{sec:1CFE5FE3A1}
\begin{Ziel}
Wir wollen uns erarbeiten, wie sich eine Zeitumkehr in der Fouriertransformierten
auswirkt und dabei eine wichtige Symmetrie der Korrespondenzen kennenlernen.

Merke bzw. besser sieh in den Formeln: Zeitumkehr ist Phasenumkehr in der Fouriertransformierten.
\end{Ziel}

\textbf{Aufgabe} {\tiny 1CFE5FE3A1}: Berechnen Sie die Fourier-Transformation für
\begin{itemize}
\item dargestellte Rechteckfunktion der Breite $T_h$ und Amplitude $A$
\end{itemize}
%
\begin{figure}[h!]
\centering
\begin{tikzpicture}[domain=0:2]
\draw[->] (-2,0) -- (2,0) node[below right] {$t$};
\draw[->] (0,0) -- (0,1.5) node[above] {$x(t)$};
\draw[-, C0, ultra thick] (-2,0) -- (-1,0) node[below] {$-T_h$} -- (-1,1) node[left] {$A$} -- (0,1) -- (0,0) -- (2,0);
\end{tikzpicture}
\end{figure}
%
Wir erkennen
\begin{itemize}
\item die in Aufgabe \ref{sec:8C3958BE4F} hier um $-T_h/2$, also nach links,
verschobene Rechteckfunktion
\item die in Aufgabe \ref{sec:A8A2DEE53A} hier zeitgedrehte Rechteckfunktion
\end{itemize}

\begin{Werkzeug}
Korrespondenz Zeitverschiebung um $\tau\in\mathbb{R}$:
\begin{align}
x(t) & \quad \laplace \quad X(\omega)\\
x(t - \tau) & \quad \laplace \quad X(\omega) \cdot \e^{-\im\omega \tau}.
\end{align}
Analyse mit Fouriertransformation:
\begin{align}
X(\omega) = \int\limits_{-\infty}^{+\infty} x(t) \, \e^{-\im \omega t} \fsd t
\end{align}
\end{Werkzeug}

\begin{Ansatz}
Mit Kenntnis der Fouriertransformierten aus Aufgabe \ref{sec:8C3958BE4F}, d.h. der
unverschobenen Zeitfunktion, können wir den Zeitverschiebungssatz anwenden:

\begin{equation}
x(t - (-T_h/2)) \quad \laplace \quad X(\omega) \cdot \e^{-\im\omega (-T_h/2)}
\end{equation}
Das Endergebnis für die hier gestellte Aufgabe lautet daher
\begin{equation}
X(\omega) = A T_h \cdot \mathrm{sinc}(\omega\frac{T_h}{2}) \cdot  \e^{+\im\omega\frac{T_h}{2}}.
\end{equation}
Wir können aber auch wieder zu Fuß das Integral der Fourier Transformation
\begin{equation}
X(\omega) = \int\limits_{-T_h}^{0} A \e^{-\im \omega t} \mathrm{d}t
\end{equation}
mit ähnlichem Vorgehen wie Aufgabe \ref{sec:A8A2DEE53A} lösen und müssen das
gleiche Endergebnis bekommen, wie die folgende ausführliche Rechnung zeigt.
\end{Ansatz}
\begin{ExCalc}
%Integrieren
\begin{equation}
X(\omega) = A \frac{\e^{-\im \omega t}}{-\im \omega} \bigg|_{-T_h}^{0}
\end{equation}
%Grenzen einsetzen
\begin{equation}
X(\omega) = A \frac{1 - \e^{+\im \omega T_h}}{-\im \omega}
\end{equation}
%Vorzeichen
\begin{equation}
X(\omega) = A \frac{\e^{+\im \omega T_h}-1}{\im \omega}
\end{equation}
%Erweitern mit dem Ziel $\sin(x) = \frac{\e^{+\im x}-\e^{-\im x}}{2\im}$, $\mathrm{sinc}(x):=\frac{\sin(x)}{x}$
\begin{equation}
X(\omega) = A \frac{\e^{+\im \omega T_h}-1}{\im \omega} \underbrace{\e^{-\im\omega\frac{T_h}{2}} \e^{+\im\omega\frac{T_h}{2}}}_{=1}=
A \frac{\e^{+\im\omega\frac{T_h}{2}} - \e^{-\im\omega\frac{T_h}{2}}}{\im\omega} \e^{+\im\omega\frac{T_h}{2}}
\end{equation}
\begin{equation}
X(\omega) = A \frac{\e^{+\im\omega\frac{T_h}{2}} - \e^{-\im\omega\frac{T_h}{2}}}{\im\omega} \e^{+\im\omega\frac{T_h}{2}}
\cdot \frac{T_h/2}{T_h/2} \cdot \frac{2}{2}
\end{equation}
%Umsortieren
\begin{equation}
X(\omega) = A T_h \frac{\e^{+\im\omega\frac{T_h}{2}} - \e^{-\im\omega\frac{T_h}{2}}}{2 \im \cdot \omega T_h/2} \e^{+\im\omega\frac{T_h}{2}} =
A T_h \frac{\sin(\omega\frac{T_h}{2})}{\omega\frac{T_h}{2}} \e^{+\im\omega\frac{T_h}{2}}
\end{equation}
\end{ExCalc}
\begin{Loesung}
Das Ergebnis
\begin{equation}
\label{eq:1CFE5FE3A1_Loesung}
X(\omega) = A T_h \cdot \mathrm{sinc}(\omega\frac{T_h}{2}) \cdot  \e^{+\im\omega\frac{T_h}{2}}.
\end{equation}
ist in \fig{fig:1CFE5FE3A1} dargestellt. Der Betrag
$\bigg|A T_h \cdot \mathrm{sinc}(\omega\frac{T_h}{2})\bigg|$
ist gleich wie in Aufgabe \ref{sec:A8A2DEE53A}, weil wir gelernt haben, dass
eine Zeitverschiebung nur die Phase in der Fouriertransformierten ändert.
%

Der wichtige neue Punkt ist, dass bei zeitlicher Linksverschiebung (Voreilen)
die \textbf{unstetige Phasenfunktion} $\phi(\omega)$
\textbf{diesmal ansteigend} ist,  da mit $m\in\mathbb{Z}$
\begin{equation}
X(\omega) = \bigg|A T_h \cdot \mathrm{sinc}(\omega\frac{T_h}{2})\bigg|
\cdot  \e^{+\im\omega\frac{T_h}{2}}\cdot
\begin{cases}
\e^{\im (0+2\pi \cdot m)}\quad \text{für}\quad \mathrm{sinc}(\omega\frac{T_h}{2})\geq0\\
\e^{\im (\pi + 2\pi \cdot m)}\quad \text{für}\quad\mathrm{sinc}(\omega\frac{T_h}{2})<0.
\end{cases}
\end{equation}

\begin{equation}
\phi(\omega) = +\omega\frac{T_h}{2} +
\begin{cases}
2\pi \cdot m\quad \text{für}\quad \mathrm{sinc}(\omega\frac{T_h}{2})\geq0\\
\pi + 2\pi \cdot m\quad \text{für}\quad\mathrm{sinc}(\omega\frac{T_h}{2})<0,
\end{cases}
\end{equation}
Die Phase ist eine stückweise stetige Geradengleichung über $\omega$ durch den
Ursprung und mit Anstieg $+T_h/2$, also diesmal steigend!
%
Die \textbf{Zeitumkehr} erzeugt also eine \textbf{Phasenumkehr}, vgl.
Aufgabe \ref{sec:A8A2DEE53A} fallende Phase, Aufgabe \ref{sec:1CFE5FE3A1} hier steigende Phase.
%
Dies gilt auch wieder verallgemeinert und ist eine wichtige \textbf{Symmetrie}eigenschaft
der Fourier Transformation
%
\begin{align}
x(+t) \quad \laplace \quad X(+\omega) =& |X(\omega)| \, \e^{+\im\angle X(\omega)}\\
x(-t) \quad \laplace \quad X(-\omega) =& |X(\omega)| \, \e^{-\im\angle X(\omega)}
\end{align}
%
Zufällig (naja, die Aufgabe war so gewählt)
in unserem Beispiel konnten wir das Problem auch über Zeitverschiebung
lösen, weil wir die unverschobene Rechteckfunktion bereits gut kennen.
\end{Loesung}
%
\begin{figure}[h!]
\includegraphics[width=\textwidth]{../ft/1CFE5FE3A1.pdf}
  \caption{Betrag (oben) und Phase (unten) der Fouriertransformierten \eq{eq:1CFE5FE3A1_Loesung}.
\textbf{Erneut Vorsicht:} Die Grafik suggeriert, dass eine zeitliche Links-Verschiebung
sowohl Phase als auch Betrag ändert. Dies passiert aber hier speziell, weil wir in der
Aufgabe nach einer Verschiebung um $-T_h/2$ gefragt haben, und $T_h$
im Argument der Spaltfunktion die Breite dieser bestimmt.
\texttt{FourierTransformation\_1CFE5FE3A1.ipynb}}
  \label{fig:1CFE5FE3A1}
\end{figure}





\clearpage
\subsection{Fourier-Transformation Signal-Modulation}
\label{sec:9D652BE72B}
\begin{Ziel}
Anhand eines speziellen Beispiels erfassen wir das Wesen des Modulationstheorems.
\end{Ziel}
\textbf{Aufgabe} {\tiny 9D652BE72B}: Berechnen Sie die Fouriertransformation für
\begin{itemize}
\item axialsymmetrische Rechteckfunktion der Breite $T_h$ und Amplitude $A$
\item die mit der komplexen Schwingung $\e^{+\im \omega_0 t}$ mit Kreisfrequenz $\omega_0>0$
multipliziert wird
\end{itemize}
Wir wollen die Fouriertransformation Korrespondenz \eq{eq:A8A2DEE53A_ModTheoremTime}
\begin{align}
x(t) \cdot \e^{+\im\omega_0 t} & \quad \laplace \quad X(\omega-\omega_0)
\end{align}
nochmal am Beispiel sehen.

\begin{figure}[h!]
\centering
\begin{tikzpicture}[domain=0:2]
\draw[->] (-2,0) -- (2,0) node[below right] {$t$};
\draw[->] (0,0) -- (0,1.5) node[above] {$x(t)$};
\draw[-, C0, ultra thick] (-2,0) -- (-0.5,0) node[below] {$\frac{-T_h}{2}$} -- (-0.5,1) node[left] {$A$} -- (0.5,1) -- (0.5,0) node[below] {$\frac{+T_h}{2}$} -- (2,0);
\end{tikzpicture}
\end{figure}

\begin{Werkzeug}
Analyse mit Fouriertransformation
\begin{align}
X(\omega) = \int\limits_{-\infty}^{+\infty} x(t) \, \e^{-\im \omega t} \fsd t
\end{align}
\end{Werkzeug}
\begin{Ansatz}
\begin{align}
X_m(\omega) = \int\limits_{-T_h/2}^{+T_h/2} \e^{+\im \omega_0 t} \cdot A \, \e^{-\im \omega t} \mathrm{d}t
\end{align}
\end{Ansatz}
\begin{ExCalc}
\begin{align}
X_m(\omega) = A \int\limits_{-T_h/2}^{+T_h/2} \e^{-\im (\omega-\omega_0) t} \mathrm{d}t =
A \frac{\e^{-\im (\omega-\omega_0) t}}{-\im (\omega-\omega_0)}\bigg|_{-T_h/2}^{+T_h/2} =
A \frac{\e^{-\im (\omega-\omega_0) T_h/2} - \e^{+\im (\omega-\omega_0) T_h/2}}{-\im (\omega-\omega_0)}
\end{align}
Vorzeichen drehen, erweitern, $\sin(x) = \frac{\e^{+\im x}-\e^{-\im x}}{2\im}$, $\mathrm{sinc}(x):=\frac{\sin (x)}{x}$
\begin{align}
X_m(\omega) = A \frac{\e^{+\im (\omega-\omega_0) T_h/2} - \e^{-\im (\omega-\omega_0) T_h/2}}{\im (\omega-\omega_0)} \cdot \frac{T_h/2}{T_h/2} \cdot \frac{2}{2}
\end{align}
\end{ExCalc}
\begin{Loesung}
\begin{align}
X_m(\omega) = A T_h \cdot \mathrm{sinc}\left((\omega-\omega_0) \frac{T_h}{2}\right)
\end{align}
Wir können mit Vergleich dieser und  Aufgabe \ref{sec:8C3958BE4F} beobachten, dass
die Korrespondenz
\begin{align}
\e^{+\im \omega_0 t} \cdot x(t) \quad\laplace\quad X\left(\omega-\omega_0\right)
\end{align}
gilt.
%
Dieser Zusammenhang gilt generell und ist als Modulationstheorem der
Fourier Transformation bekannt.
%
Das Spektrum (die Fouriertransformierte) $X(\omega)$ wird im Frequenzbereich
um $\omega_0$ verschoben, wenn $x(t)$ mit der komplexen Trägerschwingung
$\e^{+\im \omega_0 t}$ multipliziert wird. In der Nachrichtentechnik bezeichnet man
diesen Vorgang als Modulation von $x(t)$ auf den Träger $\e^{+\im \omega_0 t}$.
Dies benutzt man um Spektren in einen anderen (freien) Frequenzbereich zu
verschieben. Das werden wir bald im Detail besser verstehen.
\end{Loesung}


\begin{mdframed}
\textit{Ausblick:}
%
\\\noindent Wir könnten an dieser Stelle durchaus schon in der Lage sein, uns die
Fouriertransformierten von $x(t) = \cos(\omega_0 t)$ und $x(t) = \sin(\omega_0 t)$
abzuleiten und in Betrag und Phase zu skizzieren.
%
Dazu ist es nicht ratsam, dass Transformationsintegral lösen zu wollen,
es konvergiert nicht, sondern die bisher bekannten Korrespondenzen zusammen
mit den Euler Identitäten
\begin{align}
\cos(\omega_0 t) = \frac{\e^{+\im\omega_0 t}+\e^{-\im\omega_0 t}}{2}\qquad
\sin(\omega_0 t) = \frac{\e^{+\im\omega_0 t}-\e^{-\im\omega_0 t}}{2\im}
\end{align}
zu benutzen.
%
Diese Idee durchzieht die SigSys: `Suchen nach` und `Arbeiten mit` passenden
Korrespondenzen. Wir werden als Ingenieur*innen sehr wahrscheinlich keine unbekannte
Fouriertransformierte neu finden! Überlassen wir dieses Feld hier zunächst
den Integralnerds (vgl. \cite{Abramowitz1972}, \cite{Gradshteyn2007}) und lernen
stattdessen was wir mit den Tools anstellen können und was sie uns im Wesen sagen.
\end{mdframed}

%
In den bisherigen Aufgaben haben wir uns mit einfachen Signalen und der
Fouriertransformierten beschäftigt, sprich wir haben geschaut, welche Frequenzen
stecken mit welcher Amplitude in den untersuchten Signalen.
%
Als Spezialfall bekamen wir für periodische Signale die komplex-wertigen Fourierkoeffizienten als Linienspektrum-Darstellung.
%
Das ist der \textbf{SigSys-Teil: Signalanalyse und -synthese}. Wir werden insgesamt
\textbf{vier Fouriertransformationen} kennenlernen, alle spezialisiert auf bestimmte
Signaltypen, zwei kennen wir nun schon ein wenig besser: Fourierreihe und
Fouriertransformation.
Die anderen beiden sind

\url{https://en.wikipedia.org/wiki/Discrete_Fourier_transform}

\url{https://en.wikipedia.org/wiki/Discrete-time_Fourier_transform}

Die Idee mit Korrespondenzen zu arbeiten, bleibt für alle Transformationen
erhalten. Auch werden wir die Sinc/Rect-Dualitäten immer wieder finden,
genauso wie Zeit-/Frequenzverschiebung.




%\begin{comment}
\newpage
\subsection{Fourier-Transformation Cosinus-Schwingung}
\label{sec:610482EF57}
\begin{Ziel}
Mit dem (eigentlich noch recht bescheidenen) Wissen zur Fourier-Transformation
können wir trotzdem schon eine ganze Menge fundamental wichtiger Aspekte ableiten,
diese sind in Büchern bereits bestens abgedeckt.
Es kann jedoch nicht schaden, hin und wieder Sachen neu zu erfinden.
Wir werden die Qualität je schmaler/breiter Rechteckfunktion desto breiter/schmaler
Spaltfunktion erfassen und uns das zunutze machen, um die auf direktem Wege
für uns noch nicht lösbare Aufgabenstellung trotzdem zu lösen.
\end{Ziel}
\textbf{Aufgabe} {\tiny 610482EF57}: Berechnen Sie die Fouriertransformation für
\begin{itemize}
\item Cosinus-Schwingung $\cos(\omega_1 t)$ mit Kreisfrequenz $\omega_1>0$
\end{itemize}
\begin{Werkzeug}
Analyse mit Fouriertransformation
\begin{align}
X(\omega) = \int\limits_{-\infty}^{+\infty} x(t) \, \e^{-\im \omega t} \fsd t
\end{align}
\end{Werkzeug}
\begin{Ansatz}
\begin{align}
X(\omega) = \int\limits_{-\infty}^{+\infty} \cos(\omega_1 t) \, \e^{-\im \omega t} \fsd t
\end{align}
Dieses Integral konvergiert nicht und suggeriert, dass es keine Lösung gibt, was
natürlich nicht stimmt, sonst gäbe es die Aufgabenstellung nicht. Wir müssen
anders herangehen.
%
Wir könnten $x(t) = \cos(\omega_1 t)$ definieren und das Modulationstheorem
aus vorheriger Aufgabe mit $\omega_0=0$ zu Hilfe nehmen. Das bringt uns nur keinen
Schritt weiter.
%
Wir könnten $x(t) = 1$ definieren und das Modulationstheorem
aus vorheriger Aufgabe mit $\omega_0=\omega_1$ zu Hilfe nehmen. Dann haben wir
zwei Probleme mehr: a) wir kennen die Transformationsregel für die Modulation
mit dem reellen Cosinussignal noch nicht und b) wir kennen die Lösung der
Fourier Transformation des unmodulierten Signals $x(t)=1$, also
\begin{align}
X(\omega) = \int\limits_{-\infty}^{+\infty} 1 \e^{-\im \omega t} \fsd t = ?
\end{align}
noch nicht. Es stresst hier wieder, dass das Integral nicht konvergiert. Mit
unendlichen Grenzen können wir hier noch nicht gut umgehen.
%
Benutzen wir also unser Handwerkszeug aus den vorigen Übungen und legen zunächst
eine endliche Integrationsgrenze fest, nämlich mit Hilfe der
axialsymmetrischen Rechteckfunktion der Breite $T$ und Amplitude $A$. Diesen
Rechteckverlauf lassen wir dann immer breiter werden, bis er sich im Grenzfall
von $-\infty$ bis
$+\infty$ erstreckt.
%
Unser Ansatz lautet also
\begin{align}
X(\omega) = \int\limits_{-T_h/2}^{+T_h/2} A \cos(\omega_1 t) \, \e^{-\im \omega t} \fsd t
\end{align}
Wir erkennen, dass der Reckteckimpuls auf den Cosinus-Träger moduliert wird, aber
wir kennen das Modulationsgesetz diesbezüglich noch nicht.
Daher lösen wir das Integral zu Fuß.
\end{Ansatz}
\begin{ExCalc}
Wir benutzen $\cos(x) = \frac{1}{2}(\e^{\im x} + \e^{-\im x})$ um
\begin{align}
X(\omega) = \int\limits_{-T_h/2}^{+T_h/2} A \frac{1}{2}(\e^{\im \omega_1} + \e^{-\im \omega_1}) \, \e^{-\im \omega t} \fsd t =
\frac{A}{2} \int\limits_{-T_h/2}^{+T_h/2} (\e^{\im \omega_1} + \e^{-\im \omega_1}) \, \e^{-\im \omega t} \fsd t
\end{align}
%
\begin{align}
X(\omega) = \int\limits_{-T_h/2}^{+T_h/2} \left(\e^{-\im (\omega-\omega_1) t} + \e^{-\im (\omega+\omega_1) t} \right) \fsd t
\end{align}
%
\begin{align}
X(\omega) =
\frac{A}{2} \frac{\e^{-\im (\omega-\omega_1) t}}{-\im (\omega-\omega_1)}\bigg|_{-T_h/2}^{+T_h/2}
+\frac{A}{2} \frac{\e^{-\im (\omega+\omega_1) t}}{-\im (\omega+\omega_1)}\bigg|_{-T_h/2}^{+T_h/2}
\end{align}
%
\begin{align}
X(\omega) =
\frac{A}{2} \frac{\e^{-\im (\omega-\omega_1) T_h/2} - \e^{+\im (\omega-\omega_1) T_h/2}}{-\im (\omega-\omega_1)}
+\frac{A}{2} \frac{\e^{-\im (\omega+\omega_1) T_h/2} - \e^{+\im (\omega+\omega_1) T_h/2}}{-\im (\omega+\omega_1)}
\end{align}
%
Vorzeichen, Erweitern
%
\begin{align}
X(\omega) =
\frac{A}{2} \frac{\e^{+\im (\omega-\omega_1) T_h/2} - \e^{-\im (\omega-\omega_1) T_h/2}}{\im (\omega-\omega_1)} \cdot \frac{T_h/2}{T_h/2}
+\frac{A}{2} \frac{\e^{+\im (\omega+\omega_1) T_h/2} - \e^{-\im (\omega+\omega_1) T_h/2}}{\im (\omega+\omega_1)} \cdot \frac{T_h/2}{T_h/2}
\end{align}
%
Mit $\sin(x) = \frac{\e^{+\im x}-\e^{-\im x}}{2\im}$, $\mathrm{sinc}(x):=\frac{\sin (x)}{x}$
\begin{align}
X(\omega) =
A \frac{T_h}{2} \mathrm{sinc}\left((\omega-\omega_1) \frac{T_h}{2}\right) +
A \frac{T_h}{2} \mathrm{sinc}\left((\omega+\omega_1) \frac{T_h}{2}\right)
\end{align}
Zur Kontrolle: für $\omega_1 = 0$ bekommen wir das Ergebnis aus Aufgabe
\ref{sec:8C3958BE4F}, ein Indiz, dass wir uns nicht verrechnet haben.
\end{ExCalc}
%
\begin{Loesung}
Wir bekommen zwei mit $A \frac{T_h}{2}$ gewichtete Spaltfunktionen,
verschoben um $\pm \omega_1$ auf der Frequenzachse. Das Maximum der ersten Spaltfunktion
liegt bei $+\omega_1$ (positive Frequenz), das Maximum der zweiten Spaltfunktion
liegt bei $-\omega_1$ (negative Frequenz).



Was ist nun mit der ursprünglichen Aufgabe die Fourier Transformierte des Cosinus
zu finden?
%
Wir können sie ehrlicherweise nur mit einem Vorgriff auf kommenden Stoff erarbeiten, das
ist aber so fundamental, dass es sich lohnt, das hier schonmal anzudeuten.
%
Die Grundidee ist bekanntes mathematisches Handwerkszeug: wir lassen
$T_h\to \infty$ gehen. Dann wird sich für beide gewichtete Spaltfunktionen ein
geschlossener Ausdruck finden lassen.

\end{Loesung}

\begin{mdframed}
\textit{Vorgriff}
%
Zunächst nehmen wir unser vorläufiges Endergebnis und schreiben die Spaltfunction
aus
\begin{align}
X(\omega) =
A \frac{T_h}{2} \frac{\sin\left((\omega-\omega_1) \frac{T_h}{2}\right)}{(\omega-\omega_1) \frac{T_h}{2}} +
A \frac{T_h}{2} \frac{\sin\left((\omega+\omega_1) \frac{T_h}{2}\right)}{(\omega+\omega_1) \frac{T_h}{2}}
\end{align}
kürzen ein wenig und erweitern mit $\pi$
\begin{align}
X(\omega) =
A \pi \frac{\sin\left((\omega-\omega_1) \frac{T_h}{2}\right)}{\pi (\omega-\omega_1)} +
A \pi \frac{\sin\left((\omega+\omega_1) \frac{T_h}{2}\right)}{\pi (\omega+\omega_1)}
\end{align}
und stellen im sin-Argument noch ein wenig die Brüche um
\begin{align}
X(\omega) =
A \pi \frac{\sin\left( \frac{(\omega-\omega_1)}{\frac{2}{T_h}}  \right)}{\pi (\omega-\omega_1)} +
A \pi \frac{\sin\left( \frac{(\omega+\omega_1)}{\frac{2}{T_h}}  \right)}{\pi (\omega+\omega_1)}
\end{align}
weil wir jetzt einen Ausdruck ähnlich $\sin(\omega / \xi)$ stehen haben.
%
Die Mathematik schenkt uns nun für folgenden Grenzübergang
\begin{align}
\delta(\omega) = \lim_{\xi\to 0} \frac{\sin(\frac{\omega}{\xi})}{\pi \omega}
\end{align}
einen geschlossenen Ausdruck $\delta(\omega)$.
%
Wenden wir $\xi\to 0$ für unseren spezifischen Term $\frac{2}{T_h}\to 0$ an,
was wie gewünscht $T_h\to \infty$ bedingt,
können wir die obige Definition verwenden und schreiben
\begin{align}
X(\omega) = A \pi \delta(\omega-\omega_1) + A \pi \delta(\omega+\omega_1)
\end{align}
%
Für $A=1$ folgt weiter trivial
\begin{align}
X(\omega) = \pi \, \delta(\omega-\omega_1) + \pi \, \delta(\omega+\omega_1)
\end{align}
%
Machen wir uns klar, dass durch die Wahl $T_h\to \infty$ (also Verbreiterung
der Rechteckfunktion bis ins Unendliche) die gesuchte Korrespondenz
\begin{align}
\cos(\omega_1 t) \quad\laplace\quad \pi \, \delta(\omega-\omega_1) + \pi \, \delta(\omega+\omega_1)
\end{align}
gefunden wurde.
%
Der Ausdruck $\delta(\omega)$ ist bekannt als Dirac-Impuls, es ist aufgrund seiner
Eigenschaften keine normale Funktion, sondern eine Distribution.
Momentan hilft es uns weiter, wenn wir uns den Dirac-Impuls $\delta(\omega)$
als unendlich hohe, unendlich schmale Rechteckfunktion der Fläche 1 an der Stelle
$\omega=0$ vorstellen, siehe Grafiken unten.

Die Fourier-Transformation des Cosinus besteht daher aus zwei mit $\pi$ gewichteten
Dirac-Impulsen, die verschoben sind. Der erste Dirac-Impuls ist bei $+\omega_1$
(positive Kreisfrequenz) und der zweite Dirac ist bei $-\omega_1$
(negative Kreisfrequenz).
%
Wir erkennen eine Analogie zur komplexen Fourierreihe bei der es auch zweier
komplexer Fourierkoeffizienten (bei $\pm k$) bedarf um eine Cosinus-Schwingung
darzustellen bzw. zu synthetisieren.
%
Dies soll als Ausblich des \textbf{Dirac-Impulses} erstmal genügen.
\end{mdframed}
%\end{comment}





\clearpage
\subsection{Fourier Transformation der rechteckbegrenzten Cosinus-Schwingung}
\label{sec:610482EF57}
\begin{Ziel}
In der Übung~\ref{sec:ue1_intro} (1) hatten wir schon ein paar wichtige Eigenschaften der Fourier Transformation
kennengelernt. Eine sehr wichtige Signalklasse, nämlich harmonische
Schwingungen soll nun in dieser Aufgabe im Vordergrund stehen.
%
Weil das Integral der Fourier Transformation für diese Signale auf den ersten
Blick nicht konvergiert, behandeln wir es erst hier, weil wir mittlerweile genug
SigSys-Werkzeug haben, um trotzdem Fourier Korrespondenzen zu finden bzw. zu
benutzen.
%
Später in der Praxis werden wir selten das Integral explizit lösen, sondern
nachdem wir das Wesen vollständig verstanden haben, eher bekannte Korrespondenzen
benutzen. Die allerwichtigsten sind am Anfang dieses Skripts zusammengetragen.

Wir wollen uns erarbeiten, wie das Spektrum einer rechteckbegrenzten Cos-Schwingung
ausschaut. Wir schauen uns dabei drei Lösungswege an, die alle einen leicht anderen
Blickwinkel auf das Wesen erlauben.
\end{Ziel}
\textbf{Aufgabe} {\tiny 610482EF57}: Berechnen Sie die Fouriertransformation für
das zeitbegrenzte Signal
\begin{equation}
x(t) = \mathrm{rect}(\frac{t}{T_h}) \cdot A \cos(\omega_0 t)
\end{equation}
mit $\omega_0>0, A>0, T_h>0$.

\begin{Werkzeug}
Fouriertransformation
\begin{align}
X(\omega) = \int\limits_{-\infty}^{+\infty} x(t) \, \e^{-\im \omega t} \fsd t
\end{align}
\end{Werkzeug}
\begin{Ansatz}
Eine Skizze ist als Einstieg immer gut. Nachdem keine speziellen Zahlen gegeben sind,
die Aufgabe also auf die generellen Zusammenhänge abzielt, müssen wir uns mit einer
schematischen und deswegen allgemeingültigen Skizze behelfen. Für den Fall
$T_0=\frac{2\pi}{\omega_0}< \frac{T_h}{2}$ ist dies unten dargestellt.
%
\begin{center}
\begin{tikzpicture}[scale=01]
\def \T {1.5}
\def \Thhalf {2.4}
\def \xmin {-3}
\def \xmax {3.2}
\def \ymin {-1}
\def \ymax {1.2}
\draw[->] (\xmin-0.1 ,0) -- (\xmax+0.1,0) node[right]{$t$};
\draw[->] (0,\ymin-0.1) -- (0,\ymax+0.1) node[above]{$x(t)$};
\draw (0.05,1) -- (-0.05,1) node[left]{\small $A$};
\draw(\T,0) node[below]{\small $\frac{2\pi}{\omega_0}$};
\begin{scope}
\draw[C0, ultra thick, domain=-\Thhalf:\Thhalf,variable=\t,samples=64,smooth] plot(\t,{ cos (2*pi/\T*\t r)});
\draw[C0, ultra thick] (\xmin,0) -- (-\Thhalf,0) node[above]{$-\frac{T_h}{2}$};
\draw[C0, ultra thick] (\Thhalf,0) node[above]{$+\frac{T_h}{2}$} -- (\xmax,0);
\draw[C7, thin, dashed] (-\Thhalf,0) -- (-\Thhalf,1) -- (+\Thhalf,1) -- (+\Thhalf,0);
\end{scope}
\end{tikzpicture}
\end{center}
%
Bevor wir rechnen, sollten wir uns die Zeit nehmen und die verschiedenen
möglichen Fälle
$T_0<T_h$, $T_0=T_h$, $T_0>T_h$ als Grafik in Gedanken durchspielen.
Das Spektrum, also die Fourier Transformierte von $x(t)$ ist entscheidend davon
geprägt wie groß $T_0$ ist und wie viele Schwingungen die Rechteckfunktion 'heraus'
multipliziert.

Außerdem sollten wir uns schon hier klarmachen, dass für $T_h\to\infty$ das Signal
$x(t) = A \cos(\omega_0 t)$ resultiert und dieses harmonische Signal
mit nur zwei Koeffizienten der komplexen Fourierreihe dargestellt werden
kann.
%
Falls die Fourier Transformation für $x(t) = A \cos(\omega_0 t)$ existiert
(wir wissen ja schon, dass es so ist), müssen wir diesen Fakt hier
irgendwie wieder finden.

Nun aber zur Rechnerei.
%
Unser Ansatz lautet
\begin{align}
X(\omega) = \int\limits_{-T_h/2}^{+T_h/2} A \cos(\omega_0 t) \, \e^{-\im \omega t} \fsd t
\end{align}
\end{Ansatz}
%
%
%
\begin{ExCalc}
\textbf{Lösungsweg I}: Stur das Integral lösen.
Dieser Weg ist der mathematischste, aber wegen des Integraloperators vielleicht auch
der un-intuitivste; und ohne Erwartungshaltung Drauflos-Rechnen ist auch nicht die beste
Vorgehensweise. Anyway...Wir müssen das Integral schon sehr lange anstarren, um das Wesen
dahinter zu entlocken. Das wird uns mit den Lösungswegen II und III eleganter gelingen.
Lösen wir aber erst einmal das Integral, das gelingt ganz ähnlich wie in Übung~\ref{sec:ue1_intro} (1).

Ein Klassiker, sobald wir mit harmonischen Schwingungen im Integral
der Fouriertransformation operieren, ist die Benutzung der Euleridentität
$\cos(\phi) = \frac{1}{2}(\e^{\im \phi} + \e^{-\im \phi})$.
Damit wird das Integral vergleichsweise einfach lösbar, es ist eigentlich mehr
Schreib- als Rechenarbeit:
\begin{align}
&X(\omega) = \frac{A}{2} \int\limits_{-T_h/2}^{+T_h/2} (\e^{\im \omega_0} + \e^{-\im \omega_0}) \, \e^{-\im \omega t} \fsd t
\\
&X(\omega) = \frac{A}{2}\int\limits_{-T_h/2}^{+T_h/2} \left(\e^{-\im (\omega-\omega_0) t} + \e^{-\im (\omega+\omega_0) t} \right) \fsd t
\\
&X(\omega) =
\frac{A}{2} \frac{\e^{-\im (\omega-\omega_0) t}}{-\im (\omega-\omega_0)}\bigg|_{-T_h/2}^{+T_h/2}
+\frac{A}{2} \frac{\e^{-\im (\omega+\omega_0) t}}{-\im (\omega+\omega_0)}\bigg|_{-T_h/2}^{+T_h/2}
\\
&X(\omega) =
\frac{A}{2} \frac{\e^{-\im (\omega-\omega_0) T_h/2} - \e^{+\im (\omega-\omega_0) T_h/2}}{-\im (\omega-\omega_0)}
+\frac{A}{2} \frac{\e^{-\im (\omega+\omega_0) T_h/2} - \e^{+\im (\omega+\omega_0) T_h/2}}{-\im (\omega+\omega_0)}
\end{align}
%
Vorzeichen, Erweitern
%
\begin{align}
X(\omega) =
A\frac{\e^{+\im (\omega-\omega_0) T_h/2} - \e^{-\im (\omega-\omega_0) T_h/2}}{2\im (\omega-\omega_0)} \cdot \frac{T_h/2}{T_h/2}
+A \frac{\e^{+\im (\omega+\omega_0) T_h/2} - \e^{-\im (\omega+\omega_0) T_h/2}}{2\im (\omega+\omega_0)} \cdot \frac{T_h/2}{T_h/2}
\end{align}
%
Mit $\sin(\phi) = \frac{\e^{+\im \phi}-\e^{-\im \phi}}{2\im}$ und Spaltfunktion
$\mathrm{sinc}(\phi):=\frac{\sin \phi}{\phi}$ wird
\begin{align}
\label{eq:610482EF57_Xjw}
X(\omega) =
\frac{A T_h}{2} \mathrm{sinc}\left([\omega-\omega_0] \frac{T_h}{2}\right) +
\frac{A T_h}{2} \mathrm{sinc}\left([\omega+\omega_0] \frac{T_h}{2}\right)
\end{align}
Zur Kontrolle: für $\omega_0 = 0$ bekommen wir das Ergebnis aus Übung~\ref{sec:8C3958BE4F} (1.2),
%
ein Indiz, dass wir uns nicht verrechnet haben.
%
Wir veranschaulichen uns das an einem konkreten Zahlenbeispiel in
\fig{fig:rect_cos_610482EF57_1}. Gewählt wurde $T_0=1$ s, dann ist
$\omega_0 = \frac{2\pi}{T_0}=2\pi$ rad/s. Die Breite der Rechteckfunktion
ist $T_h=3.2$ s. Das Verhältnis $T_0/T_h$ entspricht dann genau der
schematischen Skizze aus dem Ansatz. Der Einfachheit halber und weil
kein wesentlich neuer Erkenntnisgewinn: $A=1$ .
%
Die Grafik oben zeigt das Signal $x(t)$ in grün.
In der unteren Grafik sind die beiden Teilspektren von \eq{eq:610482EF57_Xjw}
getrennt dargestellt.
Superposition ergibt die Fouriertransformierte $X(\omega)$, das ist der schwarze
Verlauf in der mittleren Grafik.

Wir haben hier ein Spezialfall der Aufgabe~\ref{sec:9D652BE72B} (1.5) vorliegen, nämlich die Modulation
eines Signals im Zeitbereich führt zur Verschiebung des Spektrums.
Hier bekommen wir nun zwei mit $\frac{A T_h}{2}$ gewichtete Spaltfunktionen,
verschoben um $\pm \omega_0$ auf der Frequenzachse. Das Maximum der ersten Spaltfunktion
liegt bei $+\omega_0$ (positive Frequenz falls (typischerweise) $\omega_0>0$),
das Maximum der zweiten Spaltfunktion liegt bei $-\omega_0$ (negative Frequenz
falls $\omega_0>0$).


\end{ExCalc}
%
\begin{figure}[h]
\centering
  \includegraphics[width=0.8\textwidth]{../ft/rect_cos_610482EF57_1.pdf}
  \caption{Fouriertransformation des rechteckbegrenzten Cosinussignals
  aus Aufgabe \ref{sec:610482EF57}.}
  \label{fig:rect_cos_610482EF57_1}
\end{figure}









\begin{mdframed}
\textbf{Zwischenbetrachtung Grenzübergang} $T_h\to\infty$ Wir können mit
unserem bisher erlernten SigSys Werkzeug
einen schönen Zusammenhang zur Fouriertransformierten von $x(t)=\cos(\omega_0 t)$
herstellen.
Diese Korrespondenz lässt sich nicht so ohne weiteres direkt aus der Integraltransformation
ableiten, weil das Integral ja streng genommen nicht konvergiert.
%

Nehmen wir das Ergebnis aus \ref{eq:610482EF57_Xjw} und schreiben die Spaltfunktion
aus
\begin{align}
X(\omega) =
\frac{A T_h}{2} \frac{\sin\left([\omega-\omega_0]) \frac{T_h}{2}\right)}{[\omega-\omega_0] \frac{T_h}{2}} +
\frac{A T_h}{2} \frac{\sin\left([\omega+\omega_0]) \frac{T_h}{2}\right)}{[\omega+\omega_0] \frac{T_h}{2}},
\end{align}
kürzen und erweitern mit $\pi$
\begin{align}
X(\omega) =
A \pi \frac{\sin\left([\omega-\omega_0] \frac{T_h}{2}\right)}{\pi [\omega-\omega_0]} +
A \pi \frac{\sin\left([\omega+\omega_0] \frac{T_h}{2}\right)}{\pi [\omega+\omega_0]}
\end{align}
und stellen im Sinus-Argument noch ein wenig die Brüche um
\begin{align}
X(\omega) =
A \pi \frac{\sin\left( \frac{[\omega-\omega_0]}{\frac{2}{T_h}}  \right)}{\pi [\omega-\omega_0]} +
A \pi \frac{\sin\left( \frac{[\omega+\omega_0]}{\frac{2}{T_h}}  \right)}{\pi [\omega+\omega_0]}
\end{align}
damit wir einen Ausdruck ähnlich $\sin(\frac{\omega}{\epsilon})$ bekommen.
%
Die Mathematik schenkt uns den Grenzübergang
\begin{align}
\delta(\omega) = \lim_{\epsilon\to 0} \frac{\sin(\frac{\omega}{\epsilon})}{\pi \omega},
\end{align}
also den Dirac Impuls $\delta(\omega)$, hier brauchen wir ihn im Frequenzbereich.
%
In unserem Fall haben wir $\epsilon=\frac{2}{T_h} \to 0$, was $T_h\to \infty$
erfordert.
Dann können wir für das Zeitsignal
\begin{equation}
x(t) = \lim_{T_h\to\infty} \mathrm{rect}(\frac{t}{T_h}) \cdot A \cos(\omega_0 t) =
A \cos(\omega_0 t)
\end{equation}
und dessen Fourier Transformierte
\begin{align}
X(\omega) =
\lim_{T_h\to\infty}
\left(A \pi \frac{\sin\left( \frac{[\omega-\omega_0]}{\frac{2}{T_h}}  \right)}{\pi [\omega-\omega_0]} +
A \pi \frac{\sin\left( \frac{[\omega+\omega_0]}{\frac{2}{T_h}}  \right)}{\pi [\omega+\omega_0]} \right)=
A \pi \delta(\omega-\omega_0) + A \pi \delta(\omega+\omega_0)
\end{align}
schreiben, woraus dann für die Wahl $A=1$ die uns gut bekannte Korrespondenz
\begin{align}
\cos(\omega_0 t) \quad\fourier\quad \pi \, \delta(\omega-\omega_0) + \pi \, \delta(\omega+\omega_0)
\end{align}
folgt.
%
Die Fourier-Transformation des Cosinus besteht daher aus zwei mit $\pi$ gewichteten
Dirac-Impulsen, die verschoben sind. Falls $\omega_0>0$, ist der erste
Dirac-Impuls $\delta(\omega-\omega_0)$ bei $+\omega_0$ eine positive Kreisfrequenz
(rechtsverschobener Dirac)
und der zweite Dirac $\delta(\omega+\omega_0)$ ist bei $-\omega_0$ eine
negative Kreisfrequenz (linksverschobener Dirac).
%
Grafisch veranschaulichen können wir uns das mit \fig{fig:ue4_cos_sin_ft}
in der Einleitung dieser Übung.
%
Zudem sind die beiden Dirac Impulse in der mittleren Grafik von
\fig{fig:rect_cos_610482EF57_1} für das spezielle Beispiel eingemalt.
%

Wir erkennen eine Analogie zur komplexen Fourierreihe bei der es auch der
zwei komplexen Fourierkoeffizienten $X_{\pm \nu} = \frac{A}{2}$ bedarf, um diese
Cosinus-Schwingung zu analysieren bzw. zu synthetisieren (falls für die
Fourierreihe die 1. Harmonische $\omega_0$ ist, gilt dann $\nu=\pm 1$).
\end{mdframed}

%\url{https://www.wolframalpha.com/input/?i=integrate+1%2FT0++*+cos%282*pi%2FT0*t%29+*+exp%28-1i*2*pi%2FT0*k*t%29+dt+limits+-T0%2F2+to+T0%2F2}


\begin{ExCalc}
\textbf{Lösungsweg II:}
\textcolor{C0}{unendlichen} Cosinus
\textcolor{C3}{zeitlich begrenzen}
heisst
\textcolor{C0}{ideales} Dirac-Spektrum
\textcolor{C3}{verschleifen}, i.e. aus \textbf{Sicht der Signalanalyse, Fensterung}

Für diesen Ansatz benutzen wir die Dualität der Multiplikation / Faltung
\begin{align}
x_1(t) \cdot x_2(t) &\quad \fourier \quad \frac{1}{2\pi} X_1(\omega) \ast_\omega X_2(\omega),
\end{align}
die gerade gefundene Korrespondenz des Cosinus
(wir setzen hier $A=1$ ohne Allgemeingültigkeit zu verlieren)
\begin{align}
x_2(t) = \cos(\omega_0 t) \quad\fourier\quad X_2(\omega) = \pi \, \delta(\omega-\omega_0) + \pi \, \delta(\omega+\omega_0)
\end{align}
und die aus der Übung~\ref{sec:ue1_intro} (1) bekannte Korrespondenz für $T_h>0$
\begin{align}
x_1(t) = \mathrm{rect}\left(\frac{t}{T_h}\right) &\quad \fourier \quad X_1(\omega) = T_h \mathrm{sinc}\left(\frac{\omega T_h}{2}\right)
\end{align}
%
Mit der Skalierungseigenschaft
\begin{align}
x(a \cdot t) &\quad \fourier \quad \frac{1}{|a|}X(\frac{\omega}{a})
\end{align}
(wir rufen uns in Erinnerung: 'schnellere' Zeitfunktion $a>1$ bedeutet breiteres Spektrum,
'langsamere' Zeitfunktion $0<a<1$ bedeutet schmaleres Spektrum)
%
und dem Einheitsrechteckimpuls
\begin{align}
\mathrm{rect}(t) &\quad \fourier \quad \mathrm{sinc}(\frac{\omega}{2})
\end{align}
%
können wir das für $a=\frac{1}{T_h}$ auch nochmal schnell aus den Korrespondenzen
ableiten (das sollten wir in der Klausur sicher beherrschen).

Wir schreiben die Multiplikation / Faltung Dualität aus
\begin{align}
x(t) = \mathrm{rect}\left(\frac{t}{T_h}\right) \cdot \cos(\omega_0 t)
&\quad \fourier \quad
X(\omega) = \textcolor{C3}{\frac{1}{2\pi}}
T_h \mathrm{sinc}\left(\frac{\omega T_h}{2}\right) \ast_\omega
[\pi \, \delta(\omega-\omega_0) + \pi \, \delta(\omega+\omega_0)],
\end{align}
und machen uns nochmal klar, dass
$a ( X_1(\omega) \ast X_2(\omega) ) = a X_1(\omega)\ast X_2(\omega) = X_1(\omega) \ast a X_2(\omega)$.
%
Der Vorfaktor $\textcolor{C3}{\frac{1}{2\pi}}$ bei der Faltung wird gerne vergessen,
den müssen wir aber berücksichtigen, damit alles sauber aufgeht.

In SigSys versuchen wir so oft es geht die Faltung durch eine einfachere Operation
zu ersetzen (siehe Umweg mit der Laplace Transformation, Übung~\ref{sec:ue3_laplace} (3)).
Hier gelingt die Lösung nun aber sehr elegant eben genau mit der Faltung (Grund:
weil wir mit Diracs falten, da ist Faltung immer sehr einfach zu überschauen).
Die Faltung ist distributiv, also können wir schreiben
\begin{align}
X(\omega) = \frac{1}{2\pi}
T_h \mathrm{sinc}\left(\frac{\omega T_h}{2}\right) \ast_\omega
[\pi \, \delta(\omega-\omega_0)]
+
\frac{1}{2\pi}
T_h \mathrm{sinc}\left(\frac{\omega T_h}{2}\right) \ast_\omega
[\pi \, \delta(\omega+\omega_0)]
\end{align}

Die Faltung im Frequenzbereich funktioniert exakt so wie im Zeitbereich, das
Integral selber interessiert sich ja nicht, welche Variablenabhängigkeit die
beiden zu faltenden Funktionen haben.
Führen wir $\nu\in\mathbb{R}$ als Variable für die Frequenzumkehr und
-verschiebung ein, so
gelten die äquivalenten Faltungsintegrale für zwei Fouriertransformierte
$X_1$ und $X_2$
\begin{equation}
X(\omega)  =
\int\limits_{\nu=-\infty}^{+\infty} X_1(\nu) \cdot X_2(-\nu+\omega) \fsd \nu
=
\int\limits_{\nu=-\infty}^{+\infty} X_1(-\nu+\omega) \cdot X_2(\nu) \fsd \nu.
\end{equation}

Das Faltungsintegral für den ersten spektralen Anteil lautet also
\begin{equation}
\int\limits_{\nu=-\infty}^{+\infty}
\frac{1}{2\pi}
T_h \mathrm{sinc}\left(\frac{(-\nu+\omega) T_h}{2}\right)
\cdot
[\pi \, \delta(\nu-\omega_0)] \fsd \nu,
\end{equation}
was wir zunächst kürzen und leicht umformulieren
\begin{equation}
\frac{T_h}{2} \int\limits_{\nu=-\infty}^{+\infty}
\mathrm{sinc}\left(\frac{(-\nu+\omega) T_h}{2}\right)
\cdot
\delta(\nu-\omega_0) \fsd \nu.
\end{equation}
%
Anstatt das obige Faltungsintegral tatsächlich zu lösen, benutzen wir
die Austasteigenschaft des Dirac Impulses. Diese ausgeschrieben für die benutzten Variablen
\begin{equation}
\int\limits_{-\infty}^{+\infty} \delta(\nu-\omega_0) \cdot f(\nu) \, \fsd \nu \stackrel{\mathrm{def}}= f(\omega_0),
\end{equation}
bringt als Ergebnis also
\begin{equation}
\frac{T_h}{2}\mathrm{sinc}\left(\frac{(-\omega_0+\omega) T_h}{2}\right)=
\frac{T_h}{2}\mathrm{sinc}\left(\frac{[\omega-\omega_0] T_h}{2}\right)
\end{equation}
Das mag auf den ersten Blick ungewohnt erscheinen, weil es ja eine ganze Funktion
bzgl. $\omega$ und nicht nur einen Funktionswert zurückliefert.
Aber wir können uns das durchaus so vorstellen, dass wir für ein ganz bestimmtes
$\omega$ das Faltungsintegral lösen wollen, und immer!
wenn $\nu=\omega_0$ die Austasteigenschaft des Dirac Impulses 'zuschlägt',
notieren wir uns das Ergebnis.
Wenn wir das für alle möglichen $\omega$ gemacht haben, werden wir feststellen,
dass wir genau das Teilspektrum
$\frac{T_h}{2}\mathrm{sinc}\left(\frac{[\omega-\omega_0] T_h}{2}\right)$
als Ergebnis bekommen.
%
Für den zweiten Teil des Spektrums gehen wir genauso vor. Das Faltungsintegral
lautet (Dirac anders verschoben)
\begin{equation}
\frac{T_h}{2} \int\limits_{\nu=-\infty}^{+\infty}
\mathrm{sinc}\left(\frac{(-\nu+\omega) T_h}{2}\right)
\cdot
\delta(\nu-(-\omega_0)) \fsd \nu,
\end{equation}
und diesmal wird die Austasteigenschaft für $\nu=-\omega_0$ 'aktiv'.
Das zweite Teilspektrum ist daher
\begin{equation}
\frac{T_h}{2}\mathrm{sinc}\left(\frac{(-(-\omega_0)+\omega) T_h}{2}\right) =
\frac{T_h}{2}\mathrm{sinc}\left(\frac{[\omega+\omega_0] T_h}{2}\right)
\end{equation}
Die beiden Teilspektren lassen sich dann (wegen Linearität!) addieren zum erwarteten
Ergebnis (vgl. Lösungsweg I)
\begin{equation}
  X(\omega) =
  \frac{T_h}{2}\mathrm{sinc}\left(\frac{[\omega-\omega_0] T_h}{2}\right) +
  \frac{T_h}{2}\mathrm{sinc}\left(\frac{[\omega+\omega_0] T_h}{2}\right).
\end{equation}
Wir sollten ein wenig üben, um solche Faltungen mit einem Dirac
im Kopf durchführen zu können. Der Dirac ist ja das Neutralelement der Faltung.
Ein zeitlich verschobener Dirac in der Faltung kann also
nur die zeitlich verschobene Funktion zurückliefern.
Wir haben das hier in aller Ausführlichkeit neu erfunden, wir könnten zum Merken
kurz notieren die Faltung im Zeitbereich (Operator $\ast_t$)
\begin{equation}
  \delta(t-\tau) \ast_t x(t) = x(t-\tau),
\end{equation}
was sich auch in der Formelsammlung findet. Wir haben die Faltung im
Frequenzbereich verwendet (Operator $\ast_\omega$),
also bzgl. der Verschiebung um $\omega_0$
\begin{equation}
  \delta(\omega-\omega_0) \ast_\omega X(\omega) = X(\omega-\omega_0).
\end{equation}
Die Verschiebung von Funktionen (in SigSys: Signale, Spektren) bzgl. des
Arguments (Zeit, (Kreis)-Frequenz) wird also sehr oft gebraucht.

Machen wir uns nun das \textbf{Wesen von Lösungsweg II} klar: Die unendliche
Cosinus-Schwingung bringt als Fourier Transformation, d.h. als Spektrum zwei
mit $\pi$ gewichtete Dirac Impulse hervor an den Kreisfrequenzen $\pm\omega_0$.
Wichtig: Die Frequenzvariable $\omega$ ist kontinuierlich, das Spektrum unterscheidet sich
wegen der Diracs nur an zwei Frequenzen von Null.
Sobald nun dieses Cosinussignal zeitlich begrenzt wird---in unserem Beispiel
symmetrisch bezüglich $t=0$ mit der einfachen Rechteckfunktion---werden diese
'schönen' Dirac Impulse verschliffen zu 'unschönen' Sinc-Funktionen. Das handeln
wir uns durch die Faltung der Spektren $X_1$ (sinc-förmig) und $X_2$ (dirac-förmig)
ein. Die Faltung passiert deswegen, weil zwei Zeitsignale $x_1$ (rect) und $x_2$ (cos)
multipliziert wurden.
Jetzt gilt wieder Zeit-Bandbreite Dualität (siehe auch Übung~\ref{sec:ue1_intro} (1)):
Je breiter das Rechtecksignal, desto schmaler die Spaltfunktion
(im Grenzfall gilt $x_1(t)=1\fourier X_1(\omega) = 2\pi\delta(\omega)$).
Je schmaler das Rechtecksignal, desto breiter die Spaltfunktion (sehr kurzer Impuls
hat ein breitbandiges Spektrum).

Das ist ein wichtiger Aspekt in der Signalanalyse, speziell dann für
digitale Signale die mit Rechnern verarbeitet werden: Eine zeitliche Begrenzung
(auch Fensterung genannt),
welche notwendig ist, damit die Fouriertransformation mit einem Rechner
praktikabel ist,
hat unweigerlich zu Folge, dass das wirkliche Spektrum durch den Faltungsprozess
verschliffen wird. Wir müssen uns daher immer fragen, was ein guter Zeitausschnitt
des zu analysierenden Signals ist und ob die Rechteckfunktion mit dem
Sinc-Spektrum das optimale Signal zur Signalbegrenzung ist. Mehr dazu dann im Kapitel
zeitdiskrete Signale.
\end{ExCalc}


\begin{ExCalc}
\textbf{Lösungsweg III:}
das Spektrum der Rechteckfunktion im Frequenzbereich verschieben
i.e. aus \textbf{Sicht der Nachrichtentechnik, Modulation}

Für diesen Weg lösen wir die Aufgabe mit bekannten Korrespondenzen der
Fouriertransformation, inhärent steckt hier ein Spezialfall der Faltung aus
Lösungsweg II drin.

Diesmal starten wir aber aus Sicht der Rechteckfunktion
\begin{align}
x_1(t) = \mathrm{rect}\left(\frac{t}{T_h}\right) &\quad \fourier \quad X_1(\omega) = T_h \mathrm{sinc}\left(\frac{\omega T_h}{2}\right)
\end{align}
und nehmen mal an, dass dies ein Signal ist, was tatsächlich (wenn auch sehr wenig)
Information codiert und wir zu einem Empfänger (mit dem wir vorher
Alphabet und Sprache abgesprochen haben,
erst dann hat das Signal als Träger von Information
für uns potentielle Bedeutung) übertragen wollen.
In der Praxis hätten wir vielleicht eine Folge von Rechteckimpulsen
mit z.B. positiver und negativer Amplitude als ganz einfache Codierung binärer
Daten, das werden wir im Detail in Nachrichtentechnik lernen. Hier geht es um die
Grundidee im Sinne der SigSys: Dieses Signal könnten wir nun exakt so wie es ist
übertragen mit einer gewünschten elektromagnetischen Welle, also über Kabel, durch Licht,
durch Luft, was jedoch ein sehr verschwenderischer Umgang mit Ressourcen wäre.
Das Spektrum gehorcht der Spaltfunktion und ist daher nicht bandbegrenzt und
konzentriert das Energiemaximum um $\omega=0$ (also um den Gleichanteil). Signale,
die solche Spektren besitzen, nennen wir Basisbandsignale.
Die Nachrichtentechnik ist getrieben von der Idee, sehr, sehr viele Basisbandsignale
über einen einzigen Übertragungskanal zu übertragen, das führte uns Stand heute
z.B. zu 5G bei Mobilfunk. Das geht aber nur, wenn die Spektren von
Signalen sich nicht überlappen, also eigentlich streng bandbegrenzt sind
und sich nicht alle Spektren im gleichen Frequenzband 'tummeln'.
Letzteres erreichen wir durch Modulation, wir hatten den Begriff schon bei der Übung~\ref{sec:ue1_intro} (1) und \ref{sec:ue3_laplace} (3),
es geht um Verschiebung des Spektrums in einen anderen Frequenzbereich.

Das Modulationstheorem kennen wir bereits, siehe Übung~\ref{sec:9D652BE72B} (1.5)
\begin{equation}
  \e^{+\im\omega_0 t} x_1(t) \quad\fourier\quad X_1(\omega-\omega_0)
\end{equation}
Das Signal, dessen Spektrum verschoben werden soll, wird mit einem sogenannten Trägersignal,
hier also $\e^{+\im\omega_0 t}$, moduliert, wobei typischerweise $\omega_0$ sehr viel größer als
die Bandbreite des Signals ist. In unserem Beispiel des nicht bandbegrenzten Sinc-Spektrums
könnten wir annehmen, dass die zunehmend kleiner werdenden Amplituden des Sincs
immer unbedeutender für das Aussehen des Zeitsignals sind, wir könnten also praktisch eine
endliche Bandbreite definieren. Quiz Frage: wie sieht der Rechteckimpuls über der Zeit aus, wenn
wir das Sinc-Spektrum tatsächlich hart band-begrenzen würden? vgl. Analogie Fourierreihe, Gibbsches Phänomen.

Physikalische Übertragungskanäle können in der Praxis nur mit reellen Signalen operieren
(komplexe Zahlen sind ja 'nur' ein Hilfskonstrukt um die Rechnerei wesentlich zu
vereinfachen), daher
erweitern wir das Modulationstheorem gemäß Euleridentität
\begin{equation}
  \frac{1}{2}\e^{+\im\omega_0 t} x_1(t) + \frac{1}{2}\e^{-\im\omega_0 t} x_1(t)
  \quad\fourier\quad
  \frac{1}{2}X_1(\omega-\omega_0) + \frac{1}{2}X_1(\omega+\omega_0)
\end{equation}
und bekommen für unser gewähltes Beispiel
\begin{equation}
  \frac{1}{2}\e^{+\im\omega_0 t} \mathrm{rect}\left(\frac{t}{T_h}\right) +
  \frac{1}{2}\e^{-\im\omega_0 t} \mathrm{rect}\left(\frac{t}{T_h}\right)
  \quad\fourier\quad
  \frac{1}{2}T_h \mathrm{sinc}\left(\frac{[\omega-\omega_0] T_h}{2}\right) +
  \frac{1}{2}T_h \mathrm{sinc}\left(\frac{[\omega+\omega_0] T_h}{2}\right)
\end{equation}
und damit das schon bekannte Ergebnis
\begin{equation}
  x(t) = \cos(\omega_0 t) \mathrm{rect}\left(\frac{t}{T_h}\right)
  \quad\fourier\quad
  X(\omega) = \frac{T_h}{2} \mathrm{sinc}\left(\frac{[\omega-\omega_0] T_h}{2}\right) +
  \frac{T_h}{2} \mathrm{sinc}\left(\frac{[\omega+\omega_0] T_h}{2}\right)
\end{equation}

\end{ExCalc}

\begin{figure}[h]
\centering
  \includegraphics[width=0.8\textwidth]{../ft/rect_cos_610482EF57_2.pdf}
  \caption{Fouriertransformation des rechteckbegrenzten Cosinussignals
  aus Aufgabe \ref{sec:610482EF57} aus Sicht einer Modulation.
  Das Basisbandspektrum (blau, vom Rechteckimpuls mit $T_h=3.2$ s)
  wird für $\omega_0=2\pi$ rad/s mit
  $\e^{+\im\omega_0 t}$ moduliert und mit $\frac{1}{2}$ gewichtet, also einmal
  rechts verschoben (orange) und einmal links verschoben (rot). Dies entspricht
  der Modulation mit $\cos(\omega_0 t)$. Ergebnisspektrum in grün.}
  \label{fig:rect_cos_610482EF57_2}
\end{figure}


\begin{Loesung}
Die drei Lösungsvarianten liefern immer das gleiche Ergebnis
(unter der Annahme $T_h>0$)
\begin{equation}
  x(t) = A\cos(\omega_0 t) \cdot \mathrm{rect}\left(\frac{t}{T_h}\right)
  \quad\fourier\quad
  X(\omega) = \frac{A T_h}{2} \mathrm{sinc}\left(\frac{[\omega-\omega_0] T_h}{2}\right) +
  \frac{A T_h}{2} \mathrm{sinc}\left(\frac{[\omega+\omega_0] T_h}{2}\right)
\end{equation}
%

Mit der Formelsammlung finden wir nun sehr einfach einen neuen Zusammenhang
\begin{equation}
  x(t) = A\cos(\omega_0 t) \cdot \mathrm{\Lambda}\left(\frac{t}{T_h}\right)
  \quad\fourier\quad
  X(\omega) = \frac{A T_h}{2} \mathrm{sinc}^2\left(\frac{[\omega-\omega_0] T_h}{2}\right) +
  \frac{A T_h}{2} \mathrm{sinc}^2\left(\frac{[\omega+\omega_0] T_h}{2}\right)
\end{equation}
also die \textbf{Zeitbegrenzung mit der Dreiecksfunktion}. Machen wir uns hierfür klar,
dass im Vergleich zur Rechteckfunktion hier ein doppelt so langer Zeitausschnitt
dreieckig ausgeschnitten wird und
im Spektrum die \textbf{Spaltfunktion quadriert} vorkommt. Das bedeutet, dass das Spektrum
um die beiden Hauptmaxima asymptotisch schneller abklingt, aber immer noch
lokale Nebenmaxima und -minima aufweist.

Eine weitere Möglichkeit, auch direkt ableitbar aus der Formelsammlung, ergibt sich
mit \textbf{Zeitbegrenzung in Form einer Gaussglocke} (so ähnlich auch mal Klausuraufgabe)
\begin{equation}
  x(t) = A\cos(\omega_0 t) \cdot \e^{-\left(\frac{t}{T_h}\right)^2}
  \quad\fourier\quad
  X(\omega) =
  \sqrt{\pi} \frac{A T_h}{2} \e^{-\left(\frac{[\omega-\omega_0] T_h}{2}\right)^2}
  +\sqrt{\pi} \frac{A T_h}{2} \e^{-\left(\frac{[\omega+\omega_0] T_h}{2}\right)^2}
\end{equation}
Die Gaussglocke hat zwei Besonderheiten bzgl. der Fouriertransformation:
\begin{itemize}
  \item $x(t)$ als breite Gaussglocke, bringt $X(\omega)$ als schmale Gaussglocke hervor.
  $x(t)$ als schmale Gaussglocke, bringt $X(\omega)$ als breite Gaussglocke hervor.
  Es gibt kein anderes Signal was sich in der Fourier Trafo skaliert auf sich selbst abbildet,
  die Gaussglocke ist ein fancy Signal.
  \item das Spektrum hat nur die zwei Hauptmaxima, aber keine lokalen Nebenmaxima und -minima.
  Das ist im Sinne bandbegrenzter Spektren die moduliert werden sollten eine gute Eigenschaft
  und wird in der Nachrichtentechnik benutzt.
\end{itemize}


\end{Loesung}






\newpage
\subsection{Fourier Transformation des Sinus-Quadrat}
\label{sec:932EC251F7}
\begin{Ziel}
Wir wollen anhand eines einfachen Beispiels das Falten mit Dirac Impulsen
vertiefen und nebenbei ein paar nützliche Verknüpfungen finden.
\end{Ziel}
\textbf{Aufgabe} {\tiny 932EC251F7}: Berechnen Sie die Fouriertransformierte, also
das Spektrum, von $x(t)=\sin^2(\omega_0 t)$.
\begin{Werkzeug}
Wir werden an den Punkt kommen, wo wir formal in einem Faltungsintegral
zwei Dirac Impulse multiplizieren wollen. Dies ist nach Definition dieses Signals
aber nicht erlaubt bzw. definiert.
Es hilft sich klarzumachen, dass
\begin{equation}
\label{eq:ue4_dirac_conv_dirac}
  \delta(t-\tau_1) \ast \delta(t-\tau_2) = \delta(t-(\tau_1+\tau_2))
\end{equation}
gilt, was wir mit der Neutralelement-Regel deduzieren könnten. Ausschreiben des
Faltungsintegrals und Sinnieren über die Austasteigenschaft des Diracs
hilft uns sehr für's Verständnis.
\end{Werkzeug}
\begin{Ansatz}
\begin{equation}
  x(t)  = \sin^2(\omega_0 t) \quad\fourier\quad X(\omega) = \mathcal{F}\{\sin^2(\omega_0 t)\}
\end{equation}
Mit Dualität Faltung vs. Multiplikation
\begin{align}
x_1(t) \cdot x_2(t) &\quad \fourier \quad \frac{1}{2\pi} X_1(\omega) \ast_\omega X_2(\omega)
\end{align}
können wir den Ansatz machen
\begin{equation}
  x(t)  = \sin(\omega_0 t)\cdot \sin(\omega_0 t) \quad\fourier\quad X(\omega) = \frac{1}{2\pi} \mathcal{F}\{\sin(\omega_0 t)\} \ast \mathcal{F}\{\sin(\omega_0 t)\}
\end{equation}
was wir mit den Erkenntnissen aus der vorherigen Aufgabe lösen können.
\end{Ansatz}
\begin{ExCalc}
Die Korrespondenz für den Sinus ist (siehe \fig{fig:ue4_cos_sin_ft} in der Einleitung dieser Übung)
\begin{align}
\sin(\omega_0 t) &\quad \fourier \quad \frac{\pi}{\im} \delta(\omega -\omega_0) - \frac{\pi}{\im} \delta(\omega + \omega_0)
\end{align}
Einsetzen, sukzessives Vereinfachen und Dirac Faltungsregel anwenden:
\begin{equation}
  X(\omega) = \frac{1}{2\pi}
  \left(\frac{\pi}{\im} \delta(\omega -\omega_0) - \frac{\pi}{\im} \delta(\omega + \omega_0)\right)
  \ast
  \left(\frac{\pi}{\im} \delta(\omega -\omega_0) - \frac{\pi}{\im} \delta(\omega + \omega_0)\right)
\end{equation}
\begin{equation}
  X(\omega) =
  \left(\frac{1}{2\im} \delta(\omega -\omega_0) - \frac{1}{2\im} \delta(\omega + \omega_0)\right)
  \ast
  \left(\frac{\pi}{\im} \delta(\omega -\omega_0) - \frac{\pi}{\im} \delta(\omega + \omega_0)\right)
\end{equation}
\begin{align}
  X(\omega) =
  &\frac{1}{2\im} \delta(\omega -\omega_0) \ast \frac{\pi}{\im} \delta(\omega -\omega_0)\\\nonumber
  &+ \frac{1}{2\im} \delta(\omega -\omega_0) \ast - \frac{\pi}{\im} \delta(\omega + \omega_0)\\\nonumber
  &- \frac{1}{2\im} \delta(\omega + \omega_0) \ast \frac{\pi}{\im} \delta(\omega -\omega_0)\\\nonumber
  &- \frac{1}{2\im} \delta(\omega + \omega_0) \ast - \frac{\pi}{\im} \delta(\omega + \omega_0)
\end{align}
\begin{align}
  X(\omega) =
  &-\frac{\pi}{2} \delta(\omega -2 \omega_0)\\\nonumber
  &+\frac{\pi}{2} \delta(\omega -(+\omega_0)) \ast \delta(\omega - (-\omega_0))\\\nonumber
  &+\frac{\pi}{2} \delta(\omega -(-\omega_0)) \ast \delta(\omega -(+\omega_0))\\\nonumber
  &-\frac{\pi}{2} \delta(\omega -(-\omega_0)) \ast \delta(\omega -(-\omega_0))
\end{align}
\begin{align}
  X(\omega) =
  &-\frac{\pi}{2} \delta(\omega -2 \omega_0)\\\nonumber
  &+\frac{\pi}{2} \delta(\omega)\\\nonumber
  &+\frac{\pi}{2} \delta(\omega)\\\nonumber
  &-\frac{\pi}{2} \delta(\omega +2 \omega_0)\\\nonumber
  & =\pi \delta(\omega) -\frac{\pi}{2} \delta(\omega -2 \omega_0) - \frac{\pi}{2} \delta(\omega +2 \omega_0)\\\nonumber
  & =\frac{1}{2} \left(2 \pi \delta(\omega) - \pi \delta(\omega -2 \omega_0) - \pi \delta(\omega +2 \omega_0)\right)\\\nonumber
  & \Fourier \quad x(t) = \sin^2(\omega_0 t) = \frac{1}{2} (1 - \cos(2\omega_0 t))
\end{align}

\end{ExCalc}
\begin{Loesung}
Die Lösung ist in \fig{fig:932EC251F7} schematisch dargestellt.
Die Rücktransformierte liefert direkt das
Additionstheorem $x(t) = \sin^2(\omega_0 t) = \frac{1}{2} (1 - \cos(2\omega_0 t))$.
Nachdem das Signal einen Gleichanteil besitzt, muss dieser im Spektrum abgebildet werden:
das ist genau der Dirac Impuls $\delta(\omega)$ mit Gewicht $\pi$, was uns Gleichanteil
von $\frac{1}{2}$ verrät (vgl. $1 \, \fourier \, 2\pi\delta(\omega)$).
\end{Loesung}


\begin{figure}[h!]
\centering
%
\begin{tikzpicture}[scale=0.75]
\draw[->] (-2.1 ,0) -- (2.1,0) node[right]{$t$};
\draw[->] (0,-1.1) -- (0,1.25) node[above]{$x(t)=\sin^2(\omega_0 t)$};
\begin{scope}
\draw[C7, ultra thick, domain=-1.9:1.8,variable=\t,samples=100,smooth] plot(\t,{ sin (2*pi*\t r) * sin (2*pi*\t r)});
\node at (4,0) {$\fourier$};
\node at (-0.25,1) {$1$};
\end{scope}
\begin{scope}[shift={(7,0)}]
\draw[->] (-2.1,0) -- (3,0) node[right] {$\omega$};
\draw[->] (0,-1.1) -- (0,1.25) node[above] {$X(\omega)=\textcolor{C0}{\frac{-\pi}{2}\delta(\omega+2\omega_0)}+\textcolor{C2}{\pi\delta(\omega)}+\textcolor{C3}{\frac{-\pi}{2}\delta(\omega-2\omega_0)}$};
\draw[->, C0, line width=1mm] (-2,0) -- (-2,-0.5) node[left] {$(\frac{-\pi}{2})$};
\draw[->, C2, line width=1mm] (+0,0) -- (+0,1) node[right] {$(\pi)$};
\draw[->, C3, line width=1mm] (+2,0) -- (+2,-0.5) node[right] {$(\frac{-\pi}{2})$};
\node at (+2,0.25) {$+2\omega_0$};
\node at (-2,0.25) {$-2\omega_0$};
\end{scope}
\end{tikzpicture}
%
%
%
\caption{Korrespondenz der Fourier Transformation für $\sin^2(\omega_0 t)$
für $\omega_0>0$.}
%Gewichte der Dirac Impulse entsprechen den relativen Skalierungen der Pfeillängen.
%Zur besseren Übersicht sind negative Gewichte mit Pfeilen nach unten gekennzeichnet.}
\label{fig:932EC251F7}
\end{figure}





\newpage
\subsection{Fourier Transformation bei zeitlicher Ableitung}
\label{sec:04ADD03866}
\begin{Ziel}
Erinnern wir uns, dass die Laplace Transformation genau davon lebt, den
Ableitungsoperator zu vereinfachen $\frac{\fsd}{\fsd t}(\cdot) \laplace s(\cdot)$
(Anfangsbedingungen hier nicht mit notiert!).
Nachdem  die Fourier Transformation der Spezialfall von $s=\sigma+\im\omega$ für
$\sigma=0$ ist, falls!!! die Laplace Transformierte für $\sigma=0$ existiert
(Konvergenzbereich muss die imaginäre Achse der $s$-Ebene, also $\Im\{s\}$ enthalten),
gilt hier $\frac{\fsd}{\fsd t}(\cdot) \fourier \im\omega(\cdot)$. Diesen Zusammenhang
hatten wir auch schon bei eingeschwungenen Zuständen in der komplexen Wechselstromtechnik
bemüht.
Wir wollen das an einem Standardsignal der SigSys mal beispielhaft
durchspielen. Mit dem SigSys Werkzeugkoffer gelingt es uns die Problematik der
Signalunstetigkeiten elegant zu umschiffen.
\end{Ziel}
\textbf{Aufgabe} {\tiny 04ADD03866}: Berechnen Sie unter der Annahme $T_h>0$
\begin{equation}
  X(\omega) = \mathcal{F}\{\frac{\fsd}{\fsd t} \mathrm{rect}\left(\frac{t}{T_h}\right)\}
\end{equation}
\begin{Werkzeug}
\begin{equation}
  \epsilon'(t) = \delta(t)
\end{equation}
\begin{equation}
\frac{\fsd}{\fsd t} x(t) \quad\fourier\quad \im\omega X(\omega)
\end{equation}
\end{Werkzeug}
\begin{Ansatz}
Im \textbf{Zeitbereich} ähnlich wie Aufgabe~\ref{sec:114F06AFAA} (2.2).

Klassiker: Darstellung des rect-Signals als zwei Sprünge (eine gute Gelegenheit
Zeitverschiebung und -skalierung nochmal aufzufrischen)
\begin{equation}
  \mathrm{rect}\left(\frac{t}{T_h}\right) = \epsilon(t+\frac{T_h}{2}) - \epsilon(t-\frac{T_h}{2})
\end{equation}
\end{Ansatz}

\begin{ExCalc}
Nun anwenden von $\epsilon'(t) = \delta(t)$
\begin{equation}
\frac{\fsd}{\fsd t} \mathrm{rect}\left(\frac{t}{T_h}\right) = \delta(t+\frac{T_h}{2}) - \delta(t-\frac{T_h}{2})
\end{equation}
Mit den uns mittlerweile gut bekannten Korrespondenzen
\begin{align}
\delta(t) &\quad \fourier \quad 1\\
x(t-\tau) &\quad \fourier \quad \e^{-\im\omega\tau} \cdot X(\omega)
\end{align}
finden wir
\begin{equation}
X(\omega) = \e^{+\im\omega\frac{T_h}{2}}\cdot 1 - \e^{-\im\omega\frac{T_h}{2}}\cdot 1
= \frac{2\im}{2\im}(\e^{+\im\omega\frac{T_h}{2}} - \e^{-\im\omega\frac{T_h}{2}})
= 2\im \sin(\omega\frac{T_h}{2})
= 2\sin(\omega\frac{T_h}{2}) \e^{\im \nicefrac{\pi}{2}}
\end{equation}
\end{ExCalc}

\begin{Ansatz}
Für die Lösung im \textbf{Frequenzbereich} benutzen wir die Korrespondenzen
\begin{align}
&\mathrm{rect}(\frac{t}{T_h}) \quad \fourier \quad |T_h| \mathrm{sinc}\left(\frac{\omega T}{2}\right)\\
&\frac{\fsd}{\fsd t} x(t) \quad\fourier\quad \im\omega X(\omega)
\end{align}
\end{Ansatz}

\begin{ExCalc}
\begin{align}
\frac{\fsd}{\fsd t} \mathrm{rect}(\frac{t}{T_h}) &\quad \fourier \quad \im\omega |T_h| \mathrm{sinc}\left(\frac{\omega T_h}{2}\right)
\end{align}
Spaltfunktion ausgeschrieben und $T_h>0$ beachtet
\begin{align}
\frac{\fsd}{\fsd t} \mathrm{rect}(\frac{t}{T_h}) &\quad \fourier \quad \im\omega T_h \frac{\sin (\frac{\omega T_h}{2})}{\frac{\omega T_h}{2}}
=2\im\sin (\frac{\omega T_h}{2})
=2\sin(\frac{\omega T_h}{2}) \e^{\im \nicefrac{\pi}{2}}
\end{align}
bringt das gleiche Ergebnis wie im Zeitbereich.
\end{ExCalc}


\begin{Loesung}
\begin{equation}
  \frac{\fsd}{\fsd t} \mathrm{rect}\left(\frac{t}{T_h}\right) \quad\fourier\quad
  2\im\sin (\frac{\omega T_h}{2})
\end{equation}
führt auf ein rein imaginäres Spektrum.
Wir könnten äquivalent auch
\begin{equation}
\label{eq:04ADD03866_Sym1}
- \delta(t-\frac{T_h}{2}) + \delta(t+\frac{T_h}{2})  \quad\fourier\quad
2\im\sin (\frac{\omega T_h}{2})
\end{equation}
schreiben bzw. wenn wir die Amplitudenskalierung anders wählen
\begin{equation}
\label{eq:04ADD03866_Sym2}
\frac{\im}{2}\delta(t-\frac{T_h}{2}) - \frac{\im}{2}\delta(t+\frac{T_h}{2})\quad\fourier\quad
\sin(\frac{\omega T_h}{2})
\end{equation}
Hier können wir uns sehr schön ein Beispiel für Symmetrien der Korrespondenzen
klarmachen, siehe dazu Vorlesungsfolie 5-13 "'Symmetrien für allgemeine Zeitsignale"`.
%
Für \eq{eq:04ADD03866_Sym1} ist das Zeitsignal rein reell und ungerade, korrespondierend
mit rein imaginärer und ungerader Fouriertransformierten.
Für \eq{eq:04ADD03866_Sym2} ist das Zeitsignal rein imaginär und ungerade.
Dies korrespondiert mit einer rein reellen und ungeraden Fouriertransformierten.
Wenn wir uns dann noch merken, dass gerade Funktionen die rein reell/imaginär sind,
korrespondieren mit geraden Transformierten die rein reellen/imaginären sind,
haben wir alle möglichen Fälle der Symmetrien abgebildet.
Bemerkenswert ist die Dualität unseres Ergebnisses zu der bekannten Korrespondenz
\begin{equation}
\sin(\omega_0 t) \quad \fourier \quad \frac{\pi}{\im} \delta(\omega -\omega_0) - \frac{\pi}{\im} \delta(\omega + \omega_0)
\end{equation}
\end{Loesung}
















\newpage
\subsection{Parseval'sches Theorem / Signalenergie im Zeit- und Frequenzbereich}
\label{sec:0A13DD5E57}
\begin{Ziel}
Wir wollen anhand einer wichtigen Korrespondenz der Fouriertransformation das
Parseval'sche Theorem erkunden, also Signalenergieäquivalenz.
In der Praxis ist dies sehr hilfreich,
um die Signalenergie aus dem Signalspektrum zu ermitteln.
In der Aufgabe hier ist die Berechnung der Signalenergie aus dem Sektrum
absichtlich mühsamer als im Zeitbereich, damit wir uns das Parseval'sche Theorem
mit einem plakativen Beispiel merken.
\end{Ziel}
\textbf{Aufgabe} {\tiny 0A13DD5E57}: Zeigen Sie, dass die Korrespondenz
\begin{equation}
x(t) = \frac{1}{2} \mathrm{rect}(\frac{1}{2}t - 1) \quad\fourier\quad X(\omega)=\e^{-2 \im \omega} \cdot \mathrm{sinc}(\omega)
\end{equation}
gilt.
Berechnen Sie dann die Signalenergie, sowohl im Zeit- als auch im Frequenzbereich mit dem
Parseval'schen Theorem.
\begin{Werkzeug}
\begin{align}
\text{Amplituden- und Zeitskalierung:   } A x(a \cdot t) \quad\fourier\quad A \frac{1}{|a|} X(\frac{\omega}{a})\\
\text{Zeitverschiebung:  } x(t - \tau)  \quad\fourier\quad \e^{-\im \omega \tau} \cdot X(\omega)
\end{align}
Signalenergie, i.e. \textbf{Parseval'schen Theorem}
\begin{equation}
  \int\limits_{-\infty}^{+\infty} |x(t)|^2 \fsd t
  \quad\fourier\quad
  \frac{1}{2\pi}\int\limits_{-\infty}^{+\infty} |X(\omega)|^2 \fsd \omega
\end{equation}
\end{Werkzeug}
\begin{Ansatz}
\textbf{Fourier Korrespondenz}

In der Formelsammlung finden wir die uns gut bekannte
\begin{equation}
\mathrm{rect}(t) \quad\fourier\quad \mathrm{sinc}\left(\frac{\omega}{2}\right)
\end{equation}
\end{Ansatz}
\begin{ExCalc}
Wir hangeln uns Schritt für Schritt zurück zu dieser Korrespondenz:
\begin{align}
\frac{1}{2} \mathrm{rect}(\frac{1}{2}t - 1) &\quad\fourier\quad \e^{-2 \im \omega} \cdot \mathrm{sinc}(\omega)\\
\frac{1}{2} \mathrm{rect}(\frac{1}{2} [t - 2]) &\quad\fourier\quad \e^{-2 \im \omega} \cdot \mathrm{sinc}(\omega)\\
\frac{1}{2} \mathrm{rect}(\frac{1}{2} (t)) &\quad\fourier\quad \mathrm{sinc}(\omega)\\
\mathrm{rect}(\frac{1}{2} t) &\quad\fourier\quad 2 \mathrm{sinc}(\omega)\\
\mathrm{rect}(\frac{1}{2} t \cdot 2) &\quad\fourier\quad 2 \frac{1}{|2|} \mathrm{sinc}(\omega \cdot \frac{1}{2})\\
\mathrm{rect}(t) &\quad\fourier\quad \mathrm{sinc}(\frac{\omega}{2})
\end{align}
Vorwärts hangeln also mit obigen Werkzeug-Korrespondenzen für $a=\frac{1}{2}$, A = $\frac{1}{2}$ und $\tau=2$.
Wichtig ist hier wieder, dass wir $(\frac{1}{2} [t - 2])$ für das Argument des rect()
schreiben müssen, um nacheinander
Zeitverschiebung und Zeitskalierung durchzuführen. Dem Ausgangspunkt $\frac{1}{2}t - 1$
sehen wir das nicht sofort an.
\end{ExCalc}

\begin{Ansatz}
\textbf{Signalenergie im Zeitbereich} ist vergleichsweise einfach.

Der Rechteckimpuls $\frac{1}{2} \mathrm{rect}(\frac{1}{2} [t - 2])$ hat Amplitude
$\frac{1}{2}$ für $1 < t < 3$ wegen Zeitverzögerung um $2$ und Skalierung um $\frac{1}{2}$
('langsamere' Funktion, bzw. breiter)
%
\begin{equation}
  E = \int\limits_{1}^{3} |\frac{1}{2}|^2 \fsd t
\end{equation}
%
Hinweis: durch das Quadrat ist das Integral nicht unmittelbar die Fläche
des Rechtecks, daher sollten wir das nicht übereilt per Auge lösen bzw.
wenn schon dann das Quadrat in der Amplitude berücksichtigen.
\end{Ansatz}

\begin{ExCalc}
\begin{equation}
  E = \int\limits_{1}^{3} |\frac{1}{2}|^2 \fsd t = \frac{1}{4} t\bigg|_{1}^{3} =
  \frac{3}{4} - \frac{1}{4} = \frac{1}{2}
\end{equation}
\end{ExCalc}


\begin{Ansatz}
\textbf{Signalenergie im Frequenzbereich} ist vergleichsweise unangenehm.

Das wird uns oft begegnen und genau deswegen benutzen wir Zeit- \textbf{und}
Frequenzbereichslösungen:
in einem Bereich ist eine Lösung fast trivial im anderen Bereich artet das in
 Rechnerei aus.
%
Die bequemere Lösung können wir dann in der Praxis bevorzugen.
%

Dieses ausartende Rechenbeispiel hilft uns das \textbf{Parsevalsche Theorem}
zu erinnern, sprich die Signalenergie ist sowohl im Zeitsignal als auch im
Spektrum vollständig codiert und je nachdem was vorliegt oder einfacher
zu rechnen ist, benutzen wir entweder den Zeit- oder Frequenzbereich.

Hier also wegen gewünschter Didaktik, der Frequenzbereich. Für die Energiebetrachtung spielt die
Phase keine Rolle.
Daher nun unser Ansatz im Frequenzbereich
\begin{align}
\label{eq:0A13DD5E57_Energie_Freq_Ansatz}
E = \frac{1}{2\pi}\int\limits_{-\infty}^{+\infty} |\e^{-2 \im \omega} \mathrm{sinc}(\omega)|^2 \fsd \omega
=\frac{1}{2\pi}\int\limits_{-\infty}^{+\infty} \mathrm{sinc}^2(\omega) \fsd \omega
\end{align}

\end{Ansatz}

\begin{figure}
\centering
\includegraphics[width=0.7\textwidth]{../ft/sine_intergral_0A13DD5E57.pdf}
\caption{Integralsinus $\mathrm{Si}(\omega)$.}
\label{fig:sine_intergral_0A13DD5E57}
\end{figure}


\begin{ExCalc}
Wir brauchen zur Lösung des obigen Integrals den Integralsinus
\begin{equation}
  \mathrm{Si}(\omega) := \int_{0}^{\omega} \frac{\sin \nu}{\nu} \fsd \nu
\end{equation}
mit $\nu\in\mathbb{R}$, dargestellt in \fig{fig:sine_intergral_0A13DD5E57}.
Es gilt
\begin{equation}
  \lim_{\omega\to+\infty} \mathrm{Si}(\omega) = \frac{\pi}{2},
\end{equation}
was wir aus \fig{fig:sine_intergral_0A13DD5E57} auch erahnen könnten.

Wir wollen also das Integral in \eq{eq:0A13DD5E57_Energie_Freq_Ansatz} lösen,
um die Signalenergie im Frequenzbereich zu ermitteln.
%
Dies gelingt mit partieller Integration und nachfolgend einer Variablensubstitution
bei einem Integral, siehe auch \url{https://calculus.subwiki.org/wiki/Sinc-squared_function}.
Mit der Schreibweise $a' = \frac{\fsd}{\fsd \nu} a$
\begin{align}
&\int a b' \fsd \nu + \int a' b \fsd \nu = a\,b\\
&\int a b' \fsd \nu = a\,b - \int a' b \fsd \nu\\
&a = \sin^2 \nu \quad\rightarrow\quad a'=\sin(2 \nu)\\ % über Euleridentität und binomische Formel
&b=-\frac{1}{\nu} \quad\leftarrow\quad b'=\frac{1}{\nu^2}\\
&\int\limits_0^\omega \frac{\sin^2 \nu}{\nu^2} \fsd \nu  =
-\frac{\sin^2 \nu}{\nu}\bigg|_0^\omega
- \int\limits_0^\omega -\frac{\sin(2 \nu)}{\nu} \fsd \nu\\
&\int\limits_0^\omega \frac{\sin^2 \nu}{\nu^2} \fsd \nu  =
-\frac{\sin^2 \omega}{\omega}
+\underbrace{\left(\lim_{\nu\to 0}\frac{\sin^2 \nu}{\nu}\right)}_{0}
+ \int\limits_0^\omega \frac{\sin(2 \nu)}{\nu} \fsd \nu
\end{align}
%
Vergeben wir $a$ nun neu für eine Variablensubstitution
$a = 2 \nu\rightarrow\frac{\fsd a}{\fsd \nu} = 2\rightarrow\fsd a = 2 \fsd \nu\rightarrow\fsd \nu = \frac{\fsd a}{2}$
und substituieren im letzten Integral, dann wird
\begin{align}
&\int\limits_0^\omega \frac{\sin^2 \nu}{\nu^2} \fsd \nu  =
-\frac{\sin^2 \omega}{\omega} + \int\limits_0^{2 \omega} \frac{\sin a}{a} \fsd a \\
&\int\limits_0^\omega \frac{\sin^2 \nu}{\nu^2} \fsd \nu  =
- \frac{\sin^2 \omega}{\omega} + \mathrm{Si}(2 \omega).
\end{align}
In der zweiten Gleichung sind wir das Integral dann 'losgeworden' weil ersetzt
mit der Definition des Integralsinus.
Nun für die Grenzen $0\dots +\infty$, also positive Kreisfrequenzen, zwei
Grenzübergänge (den des Integralsinus kennen wir von oben)
\begin{align}
\int\limits_0^{\omega=\infty} \frac{\sin^2 \nu}{\nu^2} \fsd \nu  =
- \lim_{\omega\to\infty}\frac{\sin^2 \omega}{\omega} + \lim_{\omega\to\infty}
\mathrm{Si}(2 \omega) = -0 + \frac{\pi}{2}.
\end{align}
%
Für den Integrationsbereich $-\infty\dots 0$ legen wir $\omega<0$ fest und machen
ausgehend von der partiellen Integration den gleichen Ansatz:
\begin{align}
&\int\limits_{\omega}^0 \frac{\sin^2 \nu}{\nu^2} \fsd \nu  =
-\frac{\sin^2 \nu}{\nu}\bigg|_{\omega}^0
- \int\limits_{\omega}^0 -\frac{\sin(2 \nu)}{\nu} \fsd \nu\\
&\int\limits_{\omega}^0 \frac{\sin^2 \nu}{\nu^2} \fsd \nu  =
-\lim_{\nu\to 0}\frac{\sin^2 \nu}{\nu}
+\frac{\sin^2 \omega}{\omega}
+ \int\limits_{\omega}^0 \frac{\sin(2 \nu)}{\nu} \fsd \nu\\
&\int\limits_{\omega}^0 \frac{\sin^2 \nu}{\nu^2} \fsd \nu  =
\frac{\sin^2 \omega}{\omega}
+ \int\limits_{\omega}^0 \frac{\sin(2 \nu)}{\nu} \fsd \nu\\
&\int\limits_{\omega}^0 \frac{\sin^2 \nu}{\nu^2} \fsd \nu  =
\frac{\sin^2 \omega}{\omega}
-\int\limits_{0}^\omega \frac{\sin(2 \nu)}{\nu} \fsd \nu\\
&\int\limits_{\omega}^0 \frac{\sin^2 \nu}{\nu^2} \fsd \nu  =
\frac{\sin^2 \omega}{\omega}
-\mathrm{Si}(2\omega)\\
&\int\limits_{\omega=-\infty}^0 \frac{\sin^2 \nu}{\nu^2} \fsd \nu  =
\lim_{\omega \to -\infty}\frac{\sin^2 \omega}{\omega}
-\lim_{\omega \to -\infty}\mathrm{Si}(2\omega)\\
&\int\limits_{\omega=-\infty}^0 \frac{\sin^2 \nu}{\nu^2} \fsd \nu  =
\lim_{\omega \to -\infty}\frac{\sin^2 \omega}{\omega}
+\lim_{\omega \to \infty}\mathrm{Si}(2\omega) = 0 + \frac{\pi}{2}
\end{align}
%
Die letzte Gleichung finden wir weil $\mathrm{Si}(-x)=-\mathrm{Si}(+x)$.
Mit dieser langwierigen Rechnung finden wir schlussendlich
\begin{align}
\int\limits_{-\infty}^{+\infty} \mathrm{sinc}^2(\omega) \fsd \omega = \pi
\end{align}
und damit die Signalenergie
\begin{align}
E = \frac{1}{2\pi}\int\limits_{-\infty}^{+\infty} \mathrm{sinc}^2(\omega) \fsd \omega
= \frac{1}{2\pi} \pi = \frac{1}{2}.
\end{align}
Das Parseval'sche Theorem liefert wie gewünscht das gleiche Ergebnis wie beim
Zeitbereichsansatz, nur hier deutlich komplizierter.


\end{ExCalc}

\begin{Loesung}
Für das gewählte Signal $x(t)$ und die Fouriertransformierte $X(\omega)$ ist die
Signalenergie
\begin{equation}
  \int\limits_{-\infty}^{+\infty} |x(t)|^2 \fsd t
  =\frac{1}{2}\quad\fourier\quad
  \frac{1}{2\pi}\int\limits_{-\infty}^{+\infty} |X(\omega)|^2 \fsd \omega=\frac{1}{2}
\end{equation}
Falls $x(t)$ eine Einheit hat, ist die Signalenergie in Einheit$^2$ $\cdot$ Sekunde
gegeben.
\end{Loesung}


% \noindent Kurzer Rückgriff zur Übung 1:
% Dort hatten wir für $\epsilon\in\mathbb{N}^+$
% \begin{equation}
% \delta_\epsilon(x) := \frac{\sin(\epsilon x)}{\pi x}
% \end{equation}
% definiert als einen mögliche Funktion für den Grenzübergang zum Dirac Impuls.
% Für die Fläche dieser Funktion also
% \begin{equation}
% \int\limits_{-\infty}^{+\infty} \frac{\sin(\epsilon x)}{\pi x} \fsd x =
% \int\limits_{-\infty}^{0} \frac{\sin(\epsilon x)}{\pi x} \fsd x +
% \int\limits_{0}^{\infty} \frac{\sin(\epsilon x)}{\pi x} \fsd x
% \end{equation}
% sehen wir, dass auch wieder der Integralsinus nützlich angewendet werden kann
% und dann tatsächlich
% \begin{equation}
% \int\limits_{-\infty}^{+\infty} \frac{\sin(\epsilon x)}{\pi x} \fsd x = 1
% \end{equation}
% resuliert.

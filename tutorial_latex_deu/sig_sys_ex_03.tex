%------------------------------------------------------------------------------
\clearpage
\section{UE 3: Laplace Transformation}
%
Wir haben Laplace Transformation vielleicht schon in einer Mathe Vorlesung
kennengelernt.
Laplace Transformation ist eine Integraltransformation der allgemeinen Form
\begin{equation}
F(y) = \int\limits_{x_1}^{x_2} f(x) \cdot K(x,y) \fsd x
\end{equation}
mit dem Integralkern $K(x,y) = \e^{-x\,y}$.
Wir verwenden in SigSys vorwiegend für die Variable $x$ die Zeit $t$ und für
$y$ die Laplace/Bild Variable $s\in\mathbb{C}$ (in älterer Literatur finden
wir auch oft die Variable $p$, wahrscheinlich um den Link zur Namensgebung
La\underline{p}lace zu machen).
Als Grenzen benutzen wir oft $x_1=0$ und $x_2=\infty$ um kausale Signale und
Systeme beschreiben zu können.
%
Die Rücktransformation ist wegen $s\in\mathbb{C}$ ein komplexes Wegintegral.

Wir verwenden Signal-Transformationen, in der Hoffnung, dass a) Probleme im
transformierten Signalraum, dem sogenannten Bildbereich einfacher zu lösen
sind als im Originalbereich und b) dass Signale einfacher/kompakter darstellbar
sind (vgl. periodisches Signal vs. Koeffizienten der Fourierreihe).
%
Die Wahl der geeignetsten Transformation ist dabei entscheidend, weil eine
ungeeignete das Problem auch komplizierter machen kann.
%
Die Laplace Transformation ist nun deswegen herausragend geeignet, weil
wir Signale \textbf{und} LTI-Systeme analysieren \textbf{und} synthetisieren
können und deswegen Signale mit Systemen elegant und konsistent verknüpfen können.

Wir sollten uns klarmachen, dass der \textbf{Wunsch nach einem einfacherem Operator
für die zeitliche Ableitung} ein Ausgangspunkt für die Erfindung der Laplace
Transformation ist; das sehr lesenswerte Lehrwerk \cite{LangeSigSys1} enthält
einen Abschnitt über die Entwicklungshistorie bis in die 1970er.
Statt Ableiten, also Multiplizieren
\begin{equation}
  \frac{\fsd }{\fsd t} (\cdot) \rightarrow s(\cdot ) \text{ mit } s \in \mathbb{C}
\end{equation}
unter Beibehaltung der Skalierungseigenschaft (Multiplikation mit Konstanten)
und Additionseigenschaft (Superposition).
%
Viele Mathematiker*innen haben sich an dieser Fragestellung der Operatorentheorie
abgearbeitet, sie ist ja zunächst aus rein mathematischer hochspannend, und
es hat sich herausgestellt, dass ein Spezialfall
(nämlich für $y=s\in\mathbb{C}$\footnote{die Integraltransformation mit $y=\im\omega$
, also $y$ rein imaginär, erfüllt unsere Anforderung an den neuen Operator nicht ganz,
weil eben der für bestimmte Betrachtungen wichtige Realteil in $y$ fehlt.
Wir werden sie aber trotzdem sehr nützlich finden
für SigSys, es ist die Fouriertransformation.})
der obigen
Integraltransformation genau das gewünschte leistet. Es wurde definiert
\begin{align}
\mathcal{L}\{x(t)\} = \int\limits_{t=0}^{\infty} x(t) \cdot \e^{-s\,t} \fsd t,
\end{align}
später deklariert als einseitige Laplace Transformation, benutzbar
für kausale Signale.
%
Die Laplace Transformation ist ein linearer Operator, d.h.
es gilt Additionseigenschaft
\begin{align}
\mathcal{L}\{x_1(t)+x_2(t)\} = \mathcal{L}\{x_1(t)\} + \mathcal{L}\{x_2(t)\}
\end{align}
und die Skalierungseigenschaft
\begin{align}
\mathcal{L}\{a x(t)\} = a \mathcal{L}\{x(t)\}.
\end{align}

Schauen wir uns kurz an, wie der Wunsch nach
$\frac{\fsd }{\fsd t} (\cdot) \rightarrow s(\cdot )$
mittels der Integraltransformation erfüllt wird, also was passiert mit
\begin{align}
\mathcal{L}\{\frac{\fsd }{\fsd t}  x(t)\} &= \int\limits_{t=0}^{\infty} \frac{\fsd }{\fsd t}  x(t) \cdot \e^{-s\,t} \fsd t
\end{align}
Die partielle Integration hilft dies anders darzustellen
\begin{align}
&\int u v' \fsd t + \int v u' \fsd t = u\,v\\
&\int u v' \fsd t = u\,v - \int v u' \fsd t\\
&u = \e^{-s\,t}\qquad u' = -s\,\e^{-s\,t}\\
&v = x(t)\qquad v' = \frac{\fsd }{\fsd t}  x(t)\\
&\int\limits_{t=0}^{\infty} \e^{-s\,t} \frac{\fsd }{\fsd t}  x(t) \, \fsd t
= \e^{-s\,t}\,x(t)\bigg|_{t=0}^{\infty} - \int\limits_{t=0}^{\infty} x(t) \cdot (-s\,\e^{-s\,t}) \, \fsd t\\
&\int\limits_{t=0}^{\infty} \e^{-s\,t} \frac{\fsd }{\fsd t}  x(t) \, \fsd t
= s \cdot \int\limits_{t=0}^{\infty} x(t) \e^{-s\,t} \, \fsd t
+\e^{-s\,t}\,x(t)\bigg|_{t=0}^{\infty}\\
&\int\limits_{t=0}^{\infty} \e^{-s\,t} \frac{\fsd }{\fsd t}  x(t) \, \fsd t
= s \cdot \int\limits_{t=0}^{\infty} x(t) \e^{-s\,t} \, \fsd t - x(0)
\end{align}
und wir stellen fest, dass
\begin{align}
\mathcal{L}\{\frac{\fsd }{\fsd t}  x(t)\}  = s \cdot \mathcal{L}\{x(t)\} - x(0)
\end{align}
also tatsächlich die lineare Operation \textbf{zeitliche Ableitung} zur Operation
\textbf{Multiplikation mit} $s$ gemacht wurde. Dabei müssen wir zusätzlich den
\textbf{Anfangswert} $x(0)$ beachten.

Wie bildet sich nun der Umkehroperator der zeitlichen Ableitung, also die zeitliche
Integration ab? Es gilt
\begin{align}
\mathcal{L}\{\int\limits_{0}^{t} x(\tau)\fsd \tau\}  = \frac{1}{s} \cdot \mathcal{L}\{x(t)\}.
\end{align}
Damit haben wir zwei ganz wichtige Operatoreigenschaften für
$x(t)\quad\laplace\quad X(s)$ zusammengetragen:
\begin{align}
\text{Differentiation:   } &\frac{\fsd }{\fsd t}  x(t) \quad\laplace\quad s\cdot X(s) - x(0)\\
\text{Integration:   } &\int\limits_{0}^{t} x(t) \quad\laplace\quad \frac{1}{s} \cdot X(s)
\end{align}
%
Weitere fundamentale Eigenschaften, die wie oft benötigen sind
\begin{align}
\text{Zeitverschiebung:   } x(t-\tau)& \quad\laplace\quad \e^{-s\,\tau} X(s)\\
\text{Bildbereichsverschiebung:   } \e^{a\,t} \, x(t)& \quad\laplace\quad X(s-a)
\end{align}
Die Bildbereichsverschiebung ist auch als Modulationstheorem bekannt,
und wir werden es in den Aufgaben dieser Übung ausführlich kennenlernen und
benutzen.


Die eigentliche Essenz der Laplace Transformation für die Anwendung in
SigSys steckt in der Eigenschaft
\begin{align}
\label{eq:laplace_intro_ast_mult}
x(t) \ast h(t) \quad\laplace\quad X(s) \cdot H(s),
\end{align}
d.h. das 'Abfallprodukt' für die Einführung eines einfacheren Operators
für die zeitliche Ableitung ist, dass die Faltung im Zeitbereich durch die
Multiplikation im Bildbereich dargestellt wird. Das dürfen wir durchaus fancy finden.

Die mathematisch saubere Verortung des Dirac Impulses war im Erfindungsprozess
der Laplace Transformation nicht einfach. Nachdem wir uns für SigSys mit Einführung
der Definition (zur Erinnerung: kein klassisches Riemann-Integral!)
\begin{equation}
\int\limits_{-\infty}^{+\infty} \delta(t-\tau) \cdot f(t) \, \fsd t \stackrel{\mathrm{def}}= f(\tau)
\end{equation}
eine nützliches, einfaches Werkzeug gebastelt haben, sollten wir das auch
benutzen.
Wir nehmen $\tau=0$
\begin{equation}
\int\limits_{-\infty}^{+\infty} \delta(t) \cdot f(t) \, \fsd t \stackrel{\mathrm{def}}= f(0)
\end{equation}
und für den Integralkern der Laplace Transformation $f(t)=\e^{-s \cdot t}$
ergibt sich
\begin{equation}
\int\limits_{-\infty}^{+\infty} \delta(t) \cdot \e^{-s \cdot t} \, \fsd t \stackrel{\mathrm{def}}= \e^{-s \cdot 0} = 1
\end{equation}
bzw. nochmal in Operatorschreibweise, weil fundamental wichtig:
%
\begin{equation}
  \mathcal{L}\{\delta(t)\} = 1
  \text{ oder anders notiert }
  \delta(t) \quad\laplace\quad 1.
\end{equation}
%
Der Dirac Impuls bildet also die gesamte komplexe $s$-Ebene gleich gewichtet ab.
%
Wir werden in den Aufgaben sehen, dass für den Einheitssprung
\begin{equation}
  \mathcal{L}\{\epsilon(t)\} = \frac{1}{s}
  \text{ oder anders notiert }
  \epsilon(t) \quad\laplace\quad \frac{1}{s}
\end{equation}
gilt.
Die Integrationsregel der Laplace Transformation liefert uns hier sehr bequem
den einfachen Zusammenhang
\begin{equation}
  \int\limits_{0}^{t} \delta(\tau)\fsd \tau = \epsilon(t),
\end{equation}
mit dem wir uns wegen der Signalunstetigkeiten immer ein wenig schwer tun.
%
Wenn wir $x(t)=\delta(t)$ in \eq{eq:laplace_intro_ast_mult} einsetzen
\begin{align}
\delta(t) \ast h(t) \quad\laplace\quad 1 \cdot H(s)
\end{align}
und beachten, dass der Dirac Impuls das Neutralelement der Faltung ist,
also $h(t) = \delta(t) \ast h(t)$ bekommen wir
\begin{align}
h(t) \quad\laplace\quad H(s).
\end{align}
Anders herum gedacht, die Faltung eines Signals mit einem Dirac Impuls
ändert nicht das Signal (weil Dirac Neutralelement der Faltung) und
daher auch nicht die Laplace Transformierte des Signals.


Soweit der kurze (mathematisch nicht rigoroseste, es geht hier um's Wesen)
Abriss zur Laplace Transformation.
Wir müssen lernen, die Laplace Transformierte von verschiedenen
Signalen $X(s), Y(s)$ und Systemen $H(s)$ zu interpretieren. Was sehen wir in den
komplexwertigen Funktionen über die komplexe Ebene $s$?!
Ziel bzw. zunächst nur Behauptung bevor wir das nicht selbst eingesehen haben,
war ja zunächst die Rechnerei mit LTI-System zu vereinfachen.
Mindestens genauso wichtig ist, dass der Bildbereich, also die Laplace Ebene
sehr viele schöne Interpretationen zulässt, die wir im Zeitbereich nicht
anstellen können. Nur deswegen machen wir das alles, nicht weil wir so viel
Spass beim Üben des Residuensatzes haben ;-).


%\red{zeitbegrenztes Signal Band ROC Bänder}
%\red{Mod theorem gilt für s0 in C}
%\red{was passiert wenn man sigma 0 und w0 ändert}
%\red{lim schön schreiben}












\newpage
\subsection{Laplace Transformation des Sprungsignals, Konvergenzbereich}
\label{sec:A0F7C530F3}
\begin{Ziel}
Für die zweiseitige Laplace Transformation müssen wir für
$x(t) \quad \laplace \quad X(s)$
den Konvergenzbereich (Kb) der $s$-Ebene definieren, weil erst das eindeutig
macht, wie das Signal $x(t)$ über die Zeit definiert ist.
Anhand des Sprungsignals wollen wir das exemplarisch mit der
\textbf{Laplace Hintransformation} durchspielen.
\end{Ziel}
\textbf{Aufgabe} {\tiny A0F7C530F3}: Berechnen Sie für
\begin{itemize}
  \item $x(t)=\epsilon(t)$
  \item $x(t)=-\epsilon(t)$
  \item $x(t)=\epsilon(-t)$
  \item $x(t)=-\epsilon(t)$
\end{itemize}
die Laplace Transformierte und geben Sie den zugehörigen Konvergenzbereich
der $s$-Ebene an.
\begin{Werkzeug}
Zweiseitige Laplace Transformation
\begin{align}
X(s) = \int\limits_{t=-\infty}^{+\infty} x(t) \cdot \e^{-s\,t} \fsd t
\end{align}
\end{Werkzeug}
\begin{Ansatz}
Signale ins Integral einsetzen, spezifische Grenzen berücksichtigen,
Konvergenzbereich definieren, Integral lösen.
\end{Ansatz}
\begin{ExCalc}
Einheitssprung $x(t)=\epsilon(t)$, rechtsseitig:
\begin{align}
  &X(s) = \int\limits_{t=-\infty}^{+\infty} \epsilon(t) \cdot \e^{-s\,t} \fsd t
  = \int\limits_{t=0}^{+\infty} \e^{-s\,t} \fsd t
  \rightarrow \text{Konvergenz nur, wenn}\,\,\,\Re\{s\}>0
  \rightarrow\\
  &X(s) = \frac{1}{-s}\e^{-s\,t}\bigg|_{t=0}^{\infty}
  = \frac{1}{-s}\e^{-s \,\cdot\, \infty} - \frac{1}{-s}\e^{-s\,\cdot\, 0} = \frac{1}{s}
\end{align}
%
Sprung $x(t)=-\epsilon(t)$, rechtsseitig::
\begin{align}
  &X(s) = \int\limits_{t=-\infty}^{+\infty} -\epsilon(t) \cdot \e^{-s\,t} \fsd t
  = \int\limits_{t=0}^{\infty} - \e^{-s\,t} \fsd t
  \rightarrow \text{Konvergenz nur, wenn}\,\,\,\Re\{s\}>0
  \rightarrow\\
  &X(s) = -\frac{1}{-s}\e^{-s\,t}\bigg|_{t=0}^{\infty}
  = \frac{1}{s}\e^{-s\,\cdot\, \infty} - \frac{1}{s}\e^{-s\,\cdot\, 0} = -\frac{1}{s}
\end{align}
%
Sprung $x(t)=\epsilon(-t)$, linksseitig:
\begin{align}
  &X(s) = \int\limits_{t=-\infty}^{+\infty} \epsilon(-t) \cdot \e^{-s\,t} \fsd t
  = \int\limits_{t=-\infty}^{0} \e^{-s\,t} \fsd t
  \rightarrow \text{Konvergenz nur, wenn}\,\,\,\Re\{s\}<0
  \rightarrow\\
  &X(s) = \frac{1}{-s}\e^{-s\,t}\bigg|_{t=-\infty}^{0}
  = \frac{1}{-s}\e^{-s\,\cdot\, 0} - \frac{1}{-s}\e^{s\,\cdot\, \infty} = -\frac{1}{s}
\end{align}
%
Sprung $x(t)=-\epsilon(-t)$, linksseitig:
\begin{align}
  &X(s) = \int\limits_{t=-\infty}^{+\infty} -\epsilon(-t) \cdot \e^{-s\,t} \fsd t
  = \int\limits_{t=-\infty}^{0} - \e^{-s\,t} \fsd t
  \rightarrow \text{Konvergenz nur, wenn}\,\,\,\Re\{s\}<0
  \rightarrow\\
  &X(s) = -\frac{1}{-s}\e^{-s\,t}\bigg|_{t=-\infty}^{0}
  = \frac{1}{s}\e^{-s \, \cdot \, 0} - \frac{1}{s}\e^{s \, \cdot \, \infty} = \frac{1}{s}
\end{align}
\end{ExCalc}
\begin{Loesung}
%
Zusammenfassend also
\begin{align}
+\epsilon(t) \quad &\laplace \quad +\frac{1}{s} \quad\text{ für } \quad\Re\{s\} > 0\\
-\epsilon(t) \quad &\laplace \quad -\frac{1}{s} \quad\text{ für } \quad\Re\{s\} > 0\\
%\epsilon(-t) \quad &\laplace \quad \frac{1}{-s} \quad\text{ für }\quad \Re\{-s\} > 0\\
+\epsilon(-t) \quad &\laplace \quad -\frac{1}{s} \quad\text{ für }\quad \Re\{s\} < 0\\
-\epsilon(-t) \quad &\laplace \quad +\frac{1}{s} \quad\text{ für }\quad \Re\{s\} < 0,
\end{align}
wobei wir die ersten beiden und die letzten beiden Beziehungen
direkt mit der Linearitätseigenschaft
\begin{equation}
a \cdot x(t) \quad \laplace \quad a \cdot X(s)
\end{equation}
der Laplace Transformation
verknüpfen können.

Wir sehen, dass
$X(s)=\frac{1}{s}$ die Laplacetransformierte entweder von $x(t) = \epsilon(t)$
oder von $x(t)=-\epsilon(-t)$ ist, je nachdem welchen Teil der Laplace-Ebene ($s$-Ebene)
wir betrachten, also für welche $\Re\{s\}$ das Laplace Integral konvergieren sollte.
%
In \fig{fig:A0F7C530F3} sind die vier Varianten veranschaulicht.

Wir können auch die Abbildungen \fig{fig:0B03A693AD_rightsided} und
\fig{fig:0B03A693AD_leftsided} jeweils in der Mitte zu Rate ziehen, der hier
diskutierte Fall von $x(t)=\epsilon(t)$ und $x(t)=-\epsilon(-t)$ ist dort der
Spezialfall $s_0=0$.
%
\end{Loesung}


\begin{figure*}[h!]
\centering
\begin{subfigure}{0.45\textwidth}
\begin{tikzpicture}
\begin{axis}[
width=1\textwidth,
height=0.5\textwidth,
domain=-4:4,
samples=50,
legend pos=outer north east,
xlabel = {t},
ylabel = {$\epsilon(t)$},
title = {$\epsilon(t) \quad \laplace \quad \frac{1}{s} \quad\text{ für } \quad\Re\{s\} > 0$},
xmin=-4, xmax=4,
ymin=-1.1, ymax=1.1,
xtick={-4,-2,0,2,4},
ytick={-1,0,1},
ymajorgrids=true,
xmajorgrids=true
]
\addplot[mark=None, color=C0, ultra thick]
coordinates {(-4,0)(0,0)(0,1)(4,1)};
\end{axis}
\end{tikzpicture}
\caption{rechtsseitiges Signal.}
%\label{fig:}
\end{subfigure}
%
\begin{subfigure}{0.45\textwidth}
\begin{tikzpicture}
\begin{axis}[
width=1\textwidth,
height=0.5\textwidth,
domain=-4:4,
samples=50,
legend pos=outer north east,
xlabel = {t},
ylabel = {$\epsilon(-t)$},
title = {$\epsilon(-t) \quad \laplace \quad -\frac{1}{s} \quad\text{ für } \quad\Re\{s\} < 0$},
xmin=-4, xmax=4,
ymin=-1.1, ymax=1.1,
xtick={-4,-2,0,2,4},
ytick={-1,0,1},
ymajorgrids=true,
xmajorgrids=true
]
\addplot[mark=None, color=C0, ultra thick]
coordinates {(-4,1)(0,1)(0,0)(4,0)};
\end{axis}
\end{tikzpicture}
\caption{linksseitiges Signal.}
%\label{fig:}
\end{subfigure}
%

\begin{subfigure}{0.45\textwidth}
\begin{tikzpicture}
\begin{axis}[
width=1\textwidth,
height=0.5\textwidth,
domain=-4:4,
samples=50,
legend pos=outer north east,
xlabel = {t},
ylabel = {$-\epsilon(t)$},
title = {$-\epsilon(t) \quad \laplace \quad -\frac{1}{s} \quad\text{ für } \quad\Re\{s\} > 0$},
xmin=-4, xmax=4,
ymin=-1.1, ymax=1.1,
xtick={-4,-2,0,2,4},
ytick={-1,0,1},
ymajorgrids=true,
xmajorgrids=true
]
\addplot[mark=None, color=C0, ultra thick]
coordinates {(-4,0)(0,0)(0,-1)(4,-1)};
\end{axis}
\end{tikzpicture}
\caption{rechtsseitiges Signal.}
%\label{fig:}
\end{subfigure}
%
\begin{subfigure}{0.45\textwidth}
\begin{tikzpicture}
\begin{axis}[
width=1\textwidth,
height=0.5\textwidth,
domain=-4:4,
samples=50,
legend pos=outer north east,
xlabel = {t},
ylabel = {$-\epsilon(-t)$},
title = {$-\epsilon(-t) \quad \laplace \quad \frac{1}{s} \quad\text{ für } \quad\Re\{s\} < 0$},
xmin=-4, xmax=4,
ymin=-1.1, ymax=1.1,
xtick={-4,-2,0,2,4},
ytick={-1,0,1},
ymajorgrids=true,
xmajorgrids=true
]
\addplot[mark=None, color=C0, ultra thick]
coordinates {(-4,-1)(0,-1)(0,0)(4,0)};
\end{axis}
\end{tikzpicture}
\caption{linksseitiges Signal.}
%\label{fig:}
\end{subfigure}
%
%
%
\caption{Einheitssprung, Zeit- und Amplitudenskalierung mit $\pm 1$,
Aufgabe \ref{sec:A0F7C530F3}.}
\label{fig:A0F7C530F3}
\end{figure*}




\clearpage
\subsection{Laplace Transformation des modulierten Einheitssprungs,
Konvergenzbereich}
\label{sec:0B03A693AD}
\begin{Ziel}
Wir hatten schon bei der Fourier Transformation das Modulationstheorem
kennengelernt. Für die Laplace Transformation gibt es dies auch.
Wir wollen uns dies an dem Beispiel mit dem Sprungsignal erarbeiten.
Hier müssen wir auch wieder den rechts- und linksseitigen Fall über den
Konvergenzbereich definieren.
\end{Ziel}
\textbf{Aufgabe} {\tiny 0B03A693AD}: Berechnen Sie für $\e^{s_0 \, t} \cdot \epsilon(t)$
und $\e^{-s_0 \, t} \cdot -\epsilon(-t)$ die Laplace Transformierte unter Angabe
des Konvergenzbereichs.

\begin{Werkzeug}
Zweiseitige Laplace Transformation
\begin{align}
X(s) = \int\limits_{t=-\infty}^{+\infty} x(t) \cdot \e^{-s\,t} \fsd t
\end{align}
\end{Werkzeug}
\begin{Ansatz}
Wir modulieren das Sprungsignal mit $\e^{s_0 \, t}$, wobei hier
$s_0 \in \mathbb{C}$ sein darf, weil die Laplace-Ebene auch komplexwertig ist.
Falls $s_0$ (a) pur reellwertig oder (b) pur imaginär handelt es sich um die
Spezialfälle Modulation mit (a) exponentiellem Anstieg/Abfall oder (b)
harmonische, komplexe Schwingung links-/rechtsdrehend.
\end{Ansatz}

\begin{ExCalc}
Rechtsseitiges (hier sogar zusätzlich kausales) Signal:
\begin{align}
&x(t) = \e^{s_0 \, t} \cdot \epsilon(t) \qquad
X(s) = \int\limits_{t = -\infty}^{+\infty} x(t) \, \e^{-s\,t}  \fsd t\\
&X(s) = \int\limits_{t = -\infty}^{+\infty} \e^{s_0 \, t} \, \epsilon(t) \, \e^{-s\,t}  \fsd t
= \int\limits_{t = 0}^{+\infty} \e^{[s_0-s]\,t} \fsd t = \int\limits_{t = 0}^{+\infty} \e^{-[s-s_0]\,t} \fsd t\\
&X(s) = -\frac{1}{[s-s_0]}\e^{-[s-s_0]\,t}\bigg|_{t=0}^{\infty}=
-\frac{1}{[s-s_0]}\e^{-[s-s_0]\,\cdot\,\infty }
-(-\frac{1}{[s-s_0]}\e^{-[s-s_0]\,\cdot\,0})\\
&X(s) = -\frac{1}{[s-s_0]}\e^{-[s-s_0]\,\cdot\,\infty }
+\frac{1}{s-s_0}
\end{align}
Nur wenn $\Re\{s\}-\Re\{s_0\} > 0$, also wenn $\Re\{s\}>\Re\{s_0\}$, geht
der Term $\e^{-[s-s_0]\,\cdot\,\infty }$ gegen Null und das bestimmte Integral
konvergiert.
Unter dieser gestellten Bedingung lautet das Endergebnis und damit die
Korrespondenz der Laplace Transformierten dann
\begin{align}
\e^{+s_0 \, t} \cdot \epsilon(t) \quad \laplace \quad \frac{1}{s-s_0} \quad\text{ für } \quad\Re\{s\} > \Re\{+s_0\}\\
\e^{-s_0 \, t} \cdot \epsilon(t) \quad \laplace \quad \frac{1}{s+s_0} \quad\text{ für } \quad\Re\{s\} > \Re\{-s_0\}
\end{align}

Linksseitiges (hier sogar zusätzlich antikausales) Signal:
\begin{align}
&x(t) = \e^{s_0 \, t} \cdot - \epsilon(-t) \qquad
X(s) = \int\limits_{t = -\infty}^{+\infty} x(t) \, \e^{-s\,t}  \fsd t\\
&X(s) = \int\limits_{t = -\infty}^{+\infty} -\epsilon(-t) \cdot \e^{s_0 \, t} \, \e^{-s\,t}  \fsd t
= \int\limits_{t = -\infty}^{0} -\e^{[s_0-s]\,t} \fsd t = \int\limits_{t = -\infty}^{0} -\e^{-[s-s_0]\,t} \fsd t\\
&X(s) = \frac{1}{[s-s_0]}\e^{-[s-s_0]\,t}\bigg|_{t=-\infty}^{0}=
\frac{1}{[s-s_0]}\e^{-[s-s_0]\,\cdot\, 0} - \frac{1}{[s-s_0]}\e^{-[s-s_0]\,\cdot\, -\infty}\\
&X(s) = \frac{1}{s-s_0} - \frac{1}{[s-s_0]}\e^{[s-s_0]\,\cdot\, \infty}
\end{align}
Nur wenn $\Re\{s\}-\Re\{s_0\} < 0$, also wenn $\Re\{s\}<\Re\{s_0\}$, geht
der Term $\e^{[s-s_0]\,\cdot\,\infty }$ gegen Null und das bestimmte Integral
konvergiert.
Unter dieser gestellten Bedingung lautet das Endergebnis und damit die
Korrespondenz der Laplace Transformierten dann
\begin{align}
\e^{+s_0 \, t} \cdot -\epsilon(-t) \quad \laplace \quad \frac{1}{s-s_0} \quad\text{ für } \quad\Re\{s\} < \Re\{+s_0\}\\
\e^{-s_0 \, t} \cdot -\epsilon(-t) \quad \laplace \quad \frac{1}{s+s_0} \quad\text{ für } \quad\Re\{s\} < \Re\{-s_0\}
\end{align}
\end{ExCalc}
\begin{Loesung}
%
Für $s_0=\sigma_0 + \im \omega_0$ mit $\sigma_0\in\mathbb{R}$ und $\omega_0\in\mathbb{R}$
gelten die Korrespondenzen für die Laplace Transformation mit dem zugehörigen
Konvergenzbereich (KB), englisch: region of convergence (ROC).
%
\begin{itemize}
  \item kausales 1-Pol Signal / 1-Pol System:
  \begin{align}
  \e^{+s_0 \, t} \cdot \epsilon(t) \quad \laplace \quad \frac{1}{s-s_0} \quad\text{ für } \quad\Re\{s\} > \Re\{+s_0\}\\
  \e^{-s_0 \, t} \cdot \epsilon(t) \quad \laplace \quad \frac{1}{s+s_0} \quad\text{ für } \quad\Re\{s\} > \Re\{-s_0\}
  \end{align}

  siehe \fig{fig:0B03A693AD_rightsided}

  \item antikausales 1-Pol Signal / 1-Pol System:
  \begin{align}
  \e^{+s_0 \, t} \cdot -\epsilon(-t) \quad \laplace \quad \frac{1}{s-s_0} \quad\text{ für } \quad\Re\{s\} < \Re\{+s_0\}\\
  \e^{-s_0 \, t} \cdot -\epsilon(-t) \quad \laplace \quad \frac{1}{s+s_0} \quad\text{ für } \quad\Re\{s\} < \Re\{-s_0\}
  \end{align}

  siehe \fig{fig:0B03A693AD_leftsided}

\end{itemize}
%
Dieser wichtige Link geht zuweilen am Anfang unter: $s_0=0$ führt wieder zu Aufgabe \ref{sec:A0F7C530F3},
mittlere Grafik in \fig{fig:0B03A693AD_rightsided} und \fig{fig:0B03A693AD_rightsided}.
Wir schieben Pole (oder später auch Nullstellen) in der Laplace Ebene
(auch $s$-Ebene, also in der Ebene $\Im\{s\}$ über $\Re\{s\}$) herum,
und müssen uns fragen, wie sich das auf das Signal im Zeitbereich auswirkt.
%
Speziell sollten wir uns mal fragen, wie sich jeweils in
\fig{fig:0B03A693AD_rightsided} und \fig{fig:0B03A693AD_leftsided} der
Signalverlauf weiter ändern würde, wenn
wir $\sigma_0$ gegen $\pm\infty$ laufen lassen würden.
Was heißt das für den Konvergenzbereich.

Das sogenannte Modulationstheorem gilt auch allgemein, also wenn
$x(t) \,\laplace\, X(s)$
%
dann
\begin{align}
\e^{+s_0 \, t} \cdot x(t) \quad \laplace \quad X(s-s_0) \quad\text{ für } \quad\Re\{s\} > \Re\{+s_0\}\textrm{ und }\in\mathrm{KB}\{X(s)\}
\end{align}
%
\textbf{Und nochmal eine Verständnis-Quizfrage}:
Wir sollten uns klarmachen wie der Signalverlauf wäre, wenn wir
für \fig{fig:0B03A693AD_rightsided} und \fig{fig:0B03A693AD_leftsided}
$s_0$ nicht auf $\sigma_0$ beschränken, sondern $s_0\in\mathbb{C}$ zulassen!
Wenn wir für ein solchen Fall den Pol richtig in der $s$-Ebene verorten können
und den zugehörigen Signalverlauf korrekt darstellen können, haben wir
schon sehr viel vom Wesen der Laplace Transformation verstanden, so wie wir es
für SigSys als Tool brauchen.
%
Die nächste Aufgabe möge helfen.
\end{Loesung}

\begin{figure}
\begin{center}
%sigma < 0
\begin{tikzpicture}
%
\def \axisLength {4}
\def \tic {0.05}
\def \convAbsz {-1}
\fill[C2!50] (\convAbsz,-\axisLength/2)--(\convAbsz,\axisLength/2)
decorate [decoration={snake,segment length=15pt,amplitude=1pt}]
{(\convAbsz,\axisLength/2)--
(\axisLength/2,\axisLength/2)--
(\axisLength/2,-\axisLength/2)--
(\convAbsz,-\axisLength/2)};
\draw[->] (-\axisLength/2,0)--(\axisLength/2,0) node[right]{\small$\Re\{s\}$};
\draw[->] (0,-\axisLength/2)--(0,\axisLength/2) node[above]{\small$\Im\{s\}$};
\draw[C0, ultra thick] (-1,0) node{\Huge $\times$};
\draw (-1,\tic)--(-1,-\tic) node[below]{$\sigma_0$};
%\draw (-\tic,1) -- (\tic,1) node[right]{$\omega_0$};
%\draw (-\tic,-1) -- (\tic,-1) node[right]{$-\omega_0$};
\draw (1.25,+2.25) node[C2!75]{KB};
\draw (1.25,-1.75) node[draw,outer sep=0pt]{$\sigma_0<0, \omega_0=0$};
\draw (1.25,1.75) node[]{$g=+1$};
%
\begin{scope}[shift={(5,-1.5)}]
\begin{axis}[
width=0.45\textwidth,
height=0.3\textwidth,
domain=0:4,
samples=50,
legend pos=outer north east,
xlabel = {t},
ylabel = {$\e^{+s_0 t} \cdot \epsilon(t)$},
title = {$\e^{+s_0 t} \cdot \epsilon(t) \, \laplace \, \frac{1}{s-s_0}\text{ für }\Re\{s\} > \Re\{+s_0\}$},
xmin=-4, xmax=4,
ymin=-1.1, ymax=1.1,
xtick={0},
ytick={-1,0,1},
ymajorgrids=true,
xmajorgrids=true
]
\addplot[mark=None, color=C0, ultra thick]
coordinates {(-4,0)(0,0)(0,1)};
\addplot[mark=None, color=C0, ultra thick]
{exp(-x)};
\end{axis}
\end{scope}
%
\end{tikzpicture}
%
%
%
% sigma = 0
\begin{tikzpicture}
%
\def \axisLength {4}
\def \tic {0.05}
\def \convAbsz {0}
\fill[C2!50] (\convAbsz,-\axisLength/2)--(\convAbsz,\axisLength/2)
decorate [decoration={snake,segment length=15pt,amplitude=1pt}]
{(\convAbsz,\axisLength/2)--
(\axisLength/2,\axisLength/2)--
(\axisLength/2,-\axisLength/2)--
(\convAbsz,-\axisLength/2)};
\draw[->] (-\axisLength/2,0)--(\axisLength/2,0) node[right]{\small$\Re\{s\}$};
\draw[->] (0,-\axisLength/2)--(0,\axisLength/2) node[above]{\small$\Im\{s\}$};
\draw[C0, ultra thick] (0,0) node{\Huge $\times$};
\draw (0,\tic)--(0,-\tic) node[below]{$\sigma_0$};
%\draw (-\tic,1) -- (\tic,1) node[right]{$\omega_0$};
%\draw (-\tic,-1) -- (\tic,-1) node[right]{$-\omega_0$};
\draw (1.25,+2.25) node[C2!75]{KB};
\draw (1.25,-1.75) node[draw,outer sep=0pt]{$\sigma_0=0, \omega_0=0$};
\draw (1.25,1.75) node[]{$g=+1$};
%
\begin{scope}[shift={(5,-1.5)}]
\begin{axis}[
width=0.45\textwidth,
height=0.3\textwidth,
domain=0:4,
samples=50,
legend pos=outer north east,
xlabel = {t},
ylabel = {$\e^{+s_0 t} \cdot \epsilon(t)$},
title = {$\e^{+s_0 t} \cdot \epsilon(t) \, \laplace \, \frac{1}{s-s_0}\text{ für }\Re\{s\} > \Re\{+s_0\}$},
xmin=-4, xmax=4,
ymin=-1.1, ymax=1.1,
xtick={0},
ytick={-1,0,1},
ymajorgrids=true,
xmajorgrids=true
]
\addplot[mark=None, color=C0, ultra thick]
coordinates {(-4,0)(0,0)(0,1)};
\addplot[mark=None, color=C0, ultra thick]
{exp(0*x)};
\end{axis}
\end{scope}
%
\end{tikzpicture}
%
%
%
%sigma >0
\begin{tikzpicture}
%
\def \axisLength {4}
\def \tic {0.05}
\def \convAbsz {+1}
\fill[C2!50] (\convAbsz,-\axisLength/2)--(\convAbsz,\axisLength/2)
decorate [decoration={snake,segment length=15pt,amplitude=1pt}]
{(\convAbsz,\axisLength/2)--
(\axisLength/2,\axisLength/2)--
(\axisLength/2,-\axisLength/2)--
(\convAbsz,-\axisLength/2)};
\draw[->] (-\axisLength/2,0)--(\axisLength/2,0) node[right]{\small$\Re\{s\}$};
\draw[->] (0,-\axisLength/2)--(0,\axisLength/2) node[above]{\small$\Im\{s\}$};
\draw[C0, ultra thick] (1,0) node{\Huge $\times$};
\draw (1,\tic)--(1,-\tic) node[below]{$\sigma_0$};
%\draw (-\tic,1) -- (\tic,1) node[right]{$\omega_0$};
%\draw (-\tic,-1) -- (\tic,-1) node[right]{$-\omega_0$};
\draw (1.25,+2.25) node[C2!75]{KB};
\draw (1.25,-1.75) node[draw,outer sep=0pt]{$\sigma_0>0, \omega_0=0$};
\draw (1.25,1.75) node[]{$g=+1$};
%
\begin{scope}[shift={(5,-1.5)}]
\begin{axis}[
width=0.45\textwidth,
height=0.3\textwidth,
domain=0:4,
samples=50,
legend pos=outer north east,
xlabel = {t},
ylabel = {$\e^{+s_0 t} \cdot \epsilon(t)$},
title = {$\e^{+s_0 t} \cdot \epsilon(t) \, \laplace \, \frac{1}{s-s_0}\text{ für }\Re\{s\} > \Re\{+s_0\}$},
xmin=-4, xmax=4,
ymin=-1.1, ymax=12.1,
xtick={0},
ytick={0,1},
ymajorgrids=true,
xmajorgrids=true
]
\addplot[mark=None, color=C0, ultra thick]
coordinates {(-4,0)(0,0)(0,1)};
\addplot[mark=None, color=C0, ultra thick]
{exp(x)};
\end{axis}
\end{scope}
%
\end{tikzpicture}
\end{center}
%
%
%
\caption{\textbf{Kausales} 1-Pol Signal.
Links: $s$-Ebene, rechts: zugehöriges \textbf{rechtsseitiges}
Signal $\e^{s_0 t} \cdot \epsilon(t)$ für $s_0 = \sigma_0 + \im \omega_0$ mit
$\omega_0=0$ und Variation von $\sigma_0$.
Das Signal ganz unten ist nicht beschränkt.}
\label{fig:0B03A693AD_rightsided}
\end{figure}

\begin{figure}
\begin{center}
%sigma < 0
\begin{tikzpicture}
%
\def \axisLength {4}
\def \tic {0.05}
\def \convAbsz {-1}
\fill[C2!50] (\convAbsz,-\axisLength/2)--(\convAbsz,\axisLength/2)
decorate [decoration={snake,segment length=15pt,amplitude=1pt}]
{(\convAbsz,\axisLength/2)--
(-\axisLength/2,\axisLength/2)--
(-\axisLength/2,-\axisLength/2)--
(\convAbsz,-\axisLength/2)};
\draw[->] (-\axisLength/2,0)--(\axisLength/2,0) node[right]{\small$\Re\{s\}$};
\draw[->] (0,-\axisLength/2)--(0,\axisLength/2) node[above]{\small$\Im\{s\}$};
\draw[C0, ultra thick] (-1,0) node{\Huge $\times$};
\draw (-1,\tic)--(-1,-\tic) node[below]{$\sigma_0$};
%\draw (-\tic,1) -- (\tic,1) node[right]{$\omega_0$};
%\draw (-\tic,-1) -- (\tic,-1) node[right]{$-\omega_0$};
\draw (-1.75,+2.25) node[C2!75]{KB};
\draw (1.25,-1.75) node[draw,outer sep=0pt]{$\sigma_0<0, \omega_0=0$};
\draw (1.25,1.75) node[]{$g=+1$};
%
\begin{scope}[shift={(5,-1.5)}]
\begin{axis}[
width=0.45\textwidth,
height=0.3\textwidth,
domain=-4:0,
samples=50,
legend pos=outer north east,
xlabel = {t},
ylabel = {$\e^{+s_0 t} \cdot -\epsilon(-t)$},
title = {$\e^{+s_0 t} \cdot -\epsilon(-t) \, \laplace \, \frac{1}{s-s_0}\text{ für }\Re\{s\} < \Re\{+s_0\}$},
xmin=-4, xmax=4,
ymin=-12.1, ymax=1.1,
xtick={0},
ytick={-1,0},
ymajorgrids=true,
xmajorgrids=true
]
\addplot[mark=None, color=C0, ultra thick]
coordinates {(0,-1)(0,0)(4,0)};
\addplot[mark=None, color=C0, ultra thick]
{-exp(-x)};
\end{axis}
\end{scope}
%
\end{tikzpicture}
%
%
%
% sigma = 0
\begin{tikzpicture}
%
\def \axisLength {4}
\def \tic {0.05}
\def \convAbsz {0}
\fill[C2!50] (\convAbsz,-\axisLength/2)--(\convAbsz,\axisLength/2)
decorate [decoration={snake,segment length=15pt,amplitude=1pt}]
{(\convAbsz,\axisLength/2)--
(-\axisLength/2,\axisLength/2)--
(-\axisLength/2,-\axisLength/2)--
(\convAbsz,-\axisLength/2)};
\draw[->] (-\axisLength/2,0)--(\axisLength/2,0) node[right]{\small$\Re\{s\}$};
\draw[->] (0,-\axisLength/2)--(0,\axisLength/2) node[above]{\small$\Im\{s\}$};
\draw[C0, ultra thick] (0,0) node{\Huge $\times$};
\draw (0,\tic)--(0,-\tic) node[below]{$\sigma_0$};
%\draw (-\tic,1) -- (\tic,1) node[right]{$\omega_0$};
%\draw (-\tic,-1) -- (\tic,-1) node[right]{$-\omega_0$};
\draw (-1.75,+2.25) node[C2!75]{KB};
\draw (1.25,-1.75) node[draw,outer sep=0pt]{$\sigma_0=0, \omega_0=0$};
\draw (1.25,1.75) node[]{$g=+1$};
%
\begin{scope}[shift={(5,-1.5)}]
\begin{axis}[
width=0.45\textwidth,
height=0.3\textwidth,
domain=-4:0,
samples=50,
legend pos=outer north east,
xlabel = {t},
ylabel = {$\e^{+s_0 t} \cdot -\epsilon(-t)$},
title = {$\e^{+s_0 t} \cdot -\epsilon(-t) \, \laplace \, \frac{1}{s-s_0}\text{ für }\Re\{s\} < \Re\{+s_0\}$},
xmin=-4, xmax=4,
ymin=-1.1, ymax=1.1,
xtick={0},
ytick={-1,0,1},
ymajorgrids=true,
xmajorgrids=true
]
\addplot[mark=None, color=C0, ultra thick]
coordinates {(0,-1)(0,0)(4,0)};
\addplot[mark=None, color=C0, ultra thick]
{-exp(0*x)};
\end{axis}
\end{scope}
%
\end{tikzpicture}
%
%
%
%sigma >0
\begin{tikzpicture}
%
\def \axisLength {4}
\def \tic {0.05}
\def \convAbsz {+1}
\fill[C2!50] (\convAbsz,-\axisLength/2)--(\convAbsz,\axisLength/2)
decorate [decoration={snake,segment length=15pt,amplitude=1pt}]
{(\convAbsz,\axisLength/2)--
(-\axisLength/2,\axisLength/2)--
(-\axisLength/2,-\axisLength/2)--
(\convAbsz,-\axisLength/2)};
\draw[->] (-\axisLength/2,0)--(\axisLength/2,0) node[right]{\small$\Re\{s\}$};
\draw[->] (0,-\axisLength/2)--(0,\axisLength/2) node[above]{\small$\Im\{s\}$};
\draw[C0, ultra thick] (1,0) node{\Huge $\times$};
\draw (1,\tic)--(1,-\tic) node[below]{$\sigma_0$};
%\draw (-\tic,1) -- (\tic,1) node[right]{$\omega_0$};
%\draw (-\tic,-1) -- (\tic,-1) node[right]{$-\omega_0$};
\draw (-1.75,+2.25) node[C2!75]{KB};
\draw (1.25,-1.75) node[draw,outer sep=0pt]{$\sigma_0>0, \omega_0=0$};
\draw (1.25,1.75) node[]{$g=+1$};
%
\begin{scope}[shift={(5,-1.5)}]
\begin{axis}[
width=0.45\textwidth,
height=0.3\textwidth,
domain=-4:0,
samples=50,
legend pos=outer north east,
xlabel = {t},
ylabel = {$\e^{+s_0 t} \cdot -\epsilon(-t)$},
title = {$\e^{+s_0 t} \cdot -\epsilon(-t) \, \laplace \, \frac{1}{s-s_0}\text{ für }\Re\{s\} < \Re\{+s_0\}$},
xmin=-4, xmax=4,
ymin=-1.1, ymax=1.1,
xtick={0},
ytick={-1,0,1},
ymajorgrids=true,
xmajorgrids=true
]
\addplot[mark=None, color=C0, ultra thick]
coordinates {(0,-1)(0,0)(4,0)};
\addplot[mark=None, color=C0, ultra thick]
{-exp(x)};
\end{axis}
\end{scope}
%
\end{tikzpicture}
\end{center}
%
%
%
\caption{\textbf{Antikausales} 1-Pol Signal.
Links: $s$-Ebene, rechts: zugehöriges \textbf{linksseitiges}
Signal $\e^{+s_0 t} \cdot -\epsilon(-t)$ für $s_0 = \sigma_0 + \im \omega_0$ mit
$\omega_0=0$ und Variation von $\sigma_0$.
Das Signal ganz oben ist nicht beschränkt.}
\label{fig:0B03A693AD_leftsided}
\end{figure}



\clearpage
\subsection{Vom Modulationstheorem zur Essenz}
\label{sec:31AEFEF90B}
\begin{Ziel}
Cosinus-Einschaltfunktion aus moduliertem Einheitssprung. Von der
Laplace Transformation einfacher Zeitsignale zur Veranschaulichung was wir damit
in SigSys sinnvollerweise machen. Was sehen wir eigentlich in der Laplace-Funktion
über $s$. Was sehen wir in der Laplace Ebene.
\end{Ziel}
\textbf{Aufgabe} {\tiny 31AEFEF90B}: Leiten Sie mit Hilfe von
$\e^{+s_0 \, t} \cdot \epsilon(t)$ für
$s_0=\sigma_0 + \im \omega_0$,
$\sigma_0\in\mathbb{R}$,
$\omega_0\in\mathbb{R}$
die Laplace Transformierte
für das Zeitsignal $\cos(\omega_0 t) \cdot \epsilon(t)$ her.

\begin{Werkzeug}
Wir wissen aus Aufgabe \ref{sec:0B03A693AD}, dass das kausale 1-Pol Signal / 1-Pol System
die Laplace Korrespondenz
\begin{align}
\e^{+s_0 \, t} \cdot \epsilon(t) \quad \laplace \quad \frac{1}{s-s_0} \quad\text{ für } \quad\Re\{s\} > \Re\{+s_0\}
\end{align}
hat.
%
Wir brauchen die Additions- und Superpositionseigenschaft der Laplace Transformation:
\begin{align}
a_1 x_1(t) + a_2 x_2(t) \quad\laplace\quad  a_1 X_1(s) + a_2 X_2(s)
\end{align}
Nebenbemerkung zum späteren Einordnen: es existiert keine Fouriertransformierte für $\cos(\omega_0 t) \cdot \epsilon(t)$.
\end{Werkzeug}
\begin{Ansatz}
Willkürlich erscheinender Ansatz, der aber zum gewünschten Ergebnis führt
\begin{align}
\frac{1}{2}\cdot\e^{+s_0 \, t}\cdot \epsilon(t)+
\frac{1}{2}\cdot\e^{+s_0^* \, t}\cdot \epsilon(t)
\quad \laplace \quad
\frac{1}{2}\cdot\frac{1}{s-s_0} + \frac{1}{2}\cdot\frac{1}{s-s_0^*}
\quad\text{ für } \quad\Re\{s\} > \Re\{+s_0\}
\end{align}
$s_0^*$ ist die konjugiert komplexe Zahl zu $s_0$.
\end{Ansatz}
\begin{ExCalc}
Wir machen eine Superposition von Exponentialsignalen mit jeweils $s_0$ und
$s_0^*$ (also der konjugiert komplexen 'Laplace-Frequenz',
$s_0$ darf ja komplexwertig sein).

Jetzt für die linke Seite $s_0/s_0^*$ splitten in $\sigma_0$ und $\pm\im\omega_0$
Anteile, rechts ist einfache Bruchrechnung:
\begin{align}
\frac{1}{2}\cdot\e^{+\sigma_0 t}
\left(
\e^{+\im\omega_0 \, t}+
\e^{-\im\omega_0 \, t}
\right) \epsilon(t)
\quad \laplace \quad
\frac{1}{2}\cdot\frac{(s-s_0) + (s-s_0^*)}{(s-s_0)\cdot(s-s_0^*)}
\quad\text{ für } \quad\Re\{s\} > +\sigma_0
\end{align}
Damit erhalten wir links einen cos(), dann auf der rechten Seite ausformulieren
\begin{align}
\e^{+\sigma_0 t} \cos(\omega_0 \, t) \epsilon(t)
\quad \laplace \quad
\frac{1}{2}\cdot\frac{s-(\sigma_0+\im\omega_0) + s-(\sigma_0-\im\omega_0)}
{(s-(\sigma_0+\im\omega_0)) \cdot (s-(\sigma_0-\im\omega_0))}
\quad\text{ für } \quad\Re\{s\} > +\sigma_0
\end{align}
Mit Vereinfachung kommen wir zu einer wichtigen Laplace Korrespondenz
\begin{align}
\e^{+\sigma_0 t} \cos(\omega_0 \, t) \epsilon(t)
\quad \laplace \quad
\frac{s-\sigma_0}{(s-\sigma_0)^2+\omega_0^2}
\quad\text{ für } \quad\Re\{s\} >  +\sigma_0,
\end{align}
weil die sehr oft bei der Analyse von DGLs 2. Ordnung (z.B. RLC-Schaltungen)
benötigt wird.
%
Ein ähnlicher Ansatz mit der Euleridentität für den sin() liefert
\begin{align}
\e^{+\sigma_0 t} \sin(\omega_0 \, t) \epsilon(t)
\quad \laplace \quad
\frac{\omega_0}{(s-\sigma_0)^2+\omega_0^2}
\quad\text{ für } \quad\Re\{s\} >  +\sigma_0,
\end{align}

\end{ExCalc}
\begin{Loesung}
Für $s_0=\sigma_0 + \im \omega_0$,
$\sigma_0\in\mathbb{R}$,
$\omega_0\in\mathbb{R}$
gelten die Laplace Transformierten
\begin{align}
\label{eq:31AEFEF90B}
\e^{+\sigma_0 t} \cos(\omega_0 \, t) \epsilon(t)
\quad \laplace \quad
\frac{s-\sigma_0}{(s-\sigma_0)^2+\omega_0^2}
\quad\text{ für } \quad\Re\{s\} >  +\sigma_0
\\
\e^{+\sigma_0 t} \sin(\omega_0 \, t) \epsilon(t)
\quad \laplace \quad
\frac{\omega_0}{(s-\sigma_0)^2+\omega_0^2}
\quad\text{ für } \quad\Re\{s\} >  +\sigma_0
\end{align}
Speziell für $\sigma_0=0$ ist dann
\begin{align}
\cos(\omega_0 \, t) \epsilon(t)
\quad \laplace \quad
\frac{s}{s^2+\omega_0^2}
\quad\text{ für } \quad\Re\{s\} > 0
\\
\sin(\omega_0 \, t) \epsilon(t)
\quad \laplace \quad
\frac{\omega_0}{s^2+\omega_0^2}
\quad\text{ für } \quad\Re\{s\} > 0
\end{align}
%
Damit haben wir die eigentliche Aufgabenstellung gelöst und machen uns klar:
\begin{itemize}
\item wir haben mit einer einzigen bekannten Laplace Korrespondenz
quasi fast im Vorbeigehen viele weitere gefunden
ohne das Laplace Integral nochmal explizit zu lösen, sondern vielmehr Superposition
und ein bisschen Rechnen mit komplexen Zahlen angewandt
\item das war deswegen möglich, weil das 1-Pol Signal
$\e^{+s_0 \, t} \cdot \epsilon(t)$ eine Eigenlösung von LTI-Systemen ist und
daher viele Spezialfälle beinhaltet
\end{itemize}

In \fig{fig:31AEFEF90B} ist \eq{eq:31AEFEF90B}---also der praxisnahe Fall von
kausalen Signalen---für verschiedene $\sigma_0$ schematisch veranschaulicht.

In \fig{fig:31AEFEF90B_3D_surface} sind für konkrete Zahlen die Laplace
Transformierten $X(s)$ des Fundamentalsignals $x(t)=\e^{s_0 t}\,\epsilon(t)$
über die komplexe Laplace Ebene visualisiert.
Die $z$-Achse ist der Betrag in Dezibel, also
$20 \log_{10}|X(s)|$.

\textbf{}
Die Laplace Transformation hat nun die einzigartige Eigenschaft, dass sie
sowohl Signale UND Systeme \textbf{vollständig} darstellen kann
(die Fouriertransformation kann das nicht).
%
Wir haben gelernt, dass in der \textbf{Impulsantwort} $h(t)$ eines LTI-Systems die
\textbf{komplette Information zur Systembeschreibung} steckt.
%
Wir können also davon ausgehen, dass die Laplace Transformierte der Impulsantwort
auch alle Information codiert, nur halt anders. Es ist natürlich genau so, diese
\textbf{Laplace Transformierte} hat auch einen speziellen Namen, nämlich
\textbf{Übertragungsfunktion}.
Es ist sehr gängige Konvention diese mit $H(s)$ zu bezeichnen, also
\begin{equation}
h(t) \, \laplace \, H(s)
\end{equation}
%

Für ein LTI-System $\mathcal{H}$ mit Impulsantwort $h(t)$ gilt
$y(t) = \mathcal{H}\{x(t)\}$.
Für den speziellen Fall von $x(t)=\e^{+s_0 t}$ gilt (und nur dann!,
der allgemeine Fall ist \eq{eq:4408E33353_duality})
\begin{equation}
y(t) = H(s)\bigg|_{s=s_0} \cdot \e^{+s_0 t}.
\end{equation}
%
Nehmen wir also mal an, dass die in \fig{fig:31AEFEF90B_3D_surface_one_pole}
dargestellte Laplace Transformierte von einer Impulsantwort eines LTI-Systems resultiert,
konkret also die Bildunterschrift erweitert werden kann
\begin{equation}
\label{eq_31AEFEF90B_LaplaceExample_hH}
h(t) = \e^{-10 t} \epsilon(t) \, \laplace \, H(s) = \frac{1}{s+10}\text{ für }\Re\{s\} > -10.
\end{equation}
Dann können wir für ein speziell gewähltes $x(t)$ (solange wir nur $\Re\{s_0\} > -10$
für Erfüllung der Konvergenz sichergestellt haben) in der Grafik \fig{fig:31AEFEF90B_3D_surface_one_pole}
nachschauen, wie das System den Betrag des Signals ändert, also ob das Ausgangssignal
verstärkt (>0dB) oder gedämpft (<0dB) ist. Gleiches geht auch für die Phase, was
wir uns für später aufheben.


\end{Loesung}

\begin{figure}
\begin{center}
%sigma < 0
\begin{tikzpicture}
%
\def \axisLength {4}
\def \tic {0.05}
\def \sigmaz {1/4}
\def \omegaz {5/4}
\def \convAbsz {-\sigmaz}
\fill[C2!50] (\convAbsz,-\axisLength/2)--(\convAbsz,\axisLength/2)
decorate [decoration={snake,segment length=15pt,amplitude=1pt}]
{(\convAbsz,\axisLength/2)--
(\axisLength/2,\axisLength/2)--
(\axisLength/2,-\axisLength/2)--
(\convAbsz,-\axisLength/2)};
\draw[->] (-\axisLength/2,0)--(\axisLength/2,0) node[right]{\small$\Re\{s\}$};
\draw[->] (0,-\axisLength/2)--(0,\axisLength/2) node[above]{\small$\Im\{s\}$};
\draw[C0, ultra thick] (-\sigmaz,+\omegaz) node{\Huge $\times$};
\draw[C0, ultra thick] (-\sigmaz,-\omegaz) node{\Huge $\times$};
\draw[C0, ultra thick] (-\sigmaz,0) node{\Huge $\circ$};
\draw (-\sigmaz,\tic)--(-\sigmaz,-\tic) node[below]{$\sigma_0$};
\draw (-\tic,\omegaz) -- (\tic,\omegaz) node[right]{$+\omega_0$};
\draw (-\tic,-\omegaz) -- (\tic,-\omegaz) node[right]{$-\omega_0$};
\draw (1.25,+2.25) node[C2!75]{KB};
\draw (1.25,-1.75) node[draw,outer sep=0pt]{$\sigma_0<0, \omega_0>0$};
\draw (1.25,1.75) node[]{$g=+1$};
%
\begin{scope}[shift={(5,-1.5)}]
\begin{axis}[
width=0.45\textwidth,
height=0.3\textwidth,
domain=0:4,
samples=64,
legend pos=outer north east,
xlabel = {t},
ylabel = {$\e^{+\sigma_0 t} \, \cos(\omega_0 t) \, \epsilon(t)$},
title = {$\e^{+\sigma_0 t} \cos(\omega_0 t) \epsilon(t)
\, \laplace \,
\frac{s-\sigma_0}{(s-\sigma_0)^2+\omega_0^2}
\text{ für }\Re\{s\} > +\sigma_0$},
xmin=-0.1, xmax=4,
ymin=-1.1, ymax=1.1,
xtick={0},
ytick={-1,0,1},
ymajorgrids=true,
xmajorgrids=true
]
\addplot[mark=None, color=C0, ultra thick]
coordinates {(-4,0)(0,0)(0,1)};
\addplot[mark=None, color=C0, ultra thick]
{exp(-\sigmaz*4*x) * cos(deg(\omegaz*4*x)))};
\end{axis}
\end{scope}
%
\end{tikzpicture}
%
%
%

% sigma = 0
\begin{tikzpicture}
%
\def \axisLength {4}
\def \tic {0.05}
\def \sigmaz {0}
\def \omegaz {5/4}
\def \convAbsz {-\sigmaz}
\fill[C2!50] (\convAbsz,-\axisLength/2)--(\convAbsz,\axisLength/2)
decorate [decoration={snake,segment length=15pt,amplitude=1pt}]
{(\convAbsz,\axisLength/2)--
(\axisLength/2,\axisLength/2)--
(\axisLength/2,-\axisLength/2)--
(\convAbsz,-\axisLength/2)};
\draw[->] (-\axisLength/2,0)--(\axisLength/2,0) node[right]{\small$\Re\{s\}$};
\draw[->] (0,-\axisLength/2)--(0,\axisLength/2) node[above]{\small$\Im\{s\}$};
\draw[C0, ultra thick] (-\sigmaz,+\omegaz) node{\Huge $\times$};
\draw[C0, ultra thick] (-\sigmaz,-\omegaz) node{\Huge $\times$};
\draw[C0, ultra thick] (-\sigmaz,0) node{\Huge $\circ$};
\draw (-\sigmaz,\tic)--(-\sigmaz,-\tic) node[below]{$\sigma_0$};
\draw (-\tic,\omegaz) -- (\tic,\omegaz) node[right]{$+\omega_0$};
\draw (-\tic,-\omegaz) -- (\tic,-\omegaz) node[right]{$-\omega_0$};
\draw (1.25,+2.25) node[C2!75]{KB};
\draw (1.25,-1.75) node[draw,outer sep=0pt]{$\sigma_0=0, \omega_0>0$};
\draw (1.25,1.75) node[]{$g=+1$};
%
\begin{scope}[shift={(5,-1.5)}]
\begin{axis}[
width=0.45\textwidth,
height=0.3\textwidth,
domain=0:4,
samples=64,
legend pos=outer north east,
xlabel = {t},
ylabel = {$\e^{+\sigma_0 t} \, \cos(\omega_0 t) \, \epsilon(t)$},
title = {$\e^{+\sigma_0 t} \cos(\omega_0 t) \epsilon(t)
\, \laplace \,
\frac{s-\sigma_0}{(s-\sigma_0)^2+\omega_0^2}
\text{ für }\Re\{s\} > +\sigma_0$},
xmin=-0.1, xmax=4,
ymin=-1.1, ymax=1.1,
xtick={0},
ytick={-1,0,1},
ymajorgrids=true,
xmajorgrids=true
]
\addplot[mark=None, color=C0, ultra thick]
coordinates {(-4,0)(0,0)(0,1)};
\addplot[mark=None, color=C0, ultra thick]
{exp(-\sigmaz*4*x) * cos(deg(\omegaz*4*x)))};
\end{axis}
\end{scope}
%
\end{tikzpicture}
%
%
%

%sigma >0
\begin{tikzpicture}
%
\def \axisLength {4}
\def \tic {0.05}
\def \sigmaz {-1/4}
\def \omegaz {5/4}
\def \convAbsz {-\sigmaz}
\fill[C2!50] (\convAbsz,-\axisLength/2)--(\convAbsz,\axisLength/2)
decorate [decoration={snake,segment length=15pt,amplitude=1pt}]
{(\convAbsz,\axisLength/2)--
(\axisLength/2,\axisLength/2)--
(\axisLength/2,-\axisLength/2)--
(\convAbsz,-\axisLength/2)};
\draw[->] (-\axisLength/2,0)--(\axisLength/2,0) node[right]{\small$\Re\{s\}$};
\draw[->] (0,-\axisLength/2)--(0,\axisLength/2) node[above]{\small$\Im\{s\}$};
\draw[C0, ultra thick] (-\sigmaz,+\omegaz) node{\Huge $\times$};
\draw[C0, ultra thick] (-\sigmaz,-\omegaz) node{\Huge $\times$};
\draw[C0, ultra thick] (-\sigmaz,0) node{\Huge $\circ$};
\draw (-\sigmaz,\tic)--(-\sigmaz,-\tic) node[below]{$\sigma_0$};
\draw (-\tic,\omegaz) -- (\tic,\omegaz) node[right]{$+\omega_0$};
\draw (-\tic,-\omegaz) -- (\tic,-\omegaz) node[right]{$-\omega_0$};
\draw (1.25,+2.25) node[C2!75]{KB};
\draw (1.25,-1.75) node[draw,outer sep=0pt]{$\sigma_0>0, \omega_0>0$};
\draw (1.25,1.75) node[]{$g=+1$};
%
\begin{scope}[shift={(5,-1.5)}]
\begin{axis}[
width=0.45\textwidth,
height=0.3\textwidth,
domain=0:4,
samples=64,
legend pos=outer north east,
xlabel = {t},
ylabel = {$\e^{+\sigma_0 t} \, \cos(\omega_0 t) \, \epsilon(t)$},
title = {$\e^{+\sigma_0 t} \cos(\omega_0 t) \epsilon(t)
\, \laplace \,
\frac{s-\sigma_0}{(s-\sigma_0)^2+\omega_0^2}
\text{ für }\Re\{s\} > +\sigma_0$},
xmin=-0.1, xmax=4,
ymin=-10.1, ymax=10.1,
xtick={0},
ytick={-1,0,1},
ymajorgrids=true,
xmajorgrids=true
]
\addplot[mark=None, color=C0, ultra thick]
coordinates {(-4,0)(0,0)(0,1)};
\addplot[mark=None, color=C0, ultra thick]
{exp(-\sigmaz*4*x) * cos(deg(\omegaz*4*x)))};
\end{axis}
\end{scope}
%
\end{tikzpicture}
\end{center}
%
%
%
\caption{\textbf{Kausales} 2-Pol/1-NST Signal für
$\sigma_0\in\mathbb{R}$ und $\omega_0\in\mathbb{R}, >0$.
Links: $s$-Ebene, rechts: zugehöriges \textbf{rechtsseitiges}
Signal $\e^{+\sigma_0 t} \, \cos(\omega_0 t) \, \epsilon(t)$
mit Variation von $\sigma_0$.
Das Signal ganz oben geht wegen exp() und $\sigma_0<0$ asymptotisch gegen Null.
Das Signal in der Mitte ist eine harmonische cos()-Schwingung für $t>0$, weil
$\sigma_0=0$.
Das Signal ganz unten ist nicht beschränkt, weil exp() wegen $\sigma_0>0$ wächst.
}
\label{fig:31AEFEF90B}
\end{figure}


\begin{figure}
\centering
\begin{subfigure}{0.49\textwidth}
\includegraphics[width=\textwidth]{../laplace_transform/fundamental_signals_laplace_plane_31AEFEF90B_single_pole.pdf}
\caption{$\e^{-10 t} \epsilon(t) \, \laplace \, \frac{1}{s+10}\text{ für }\Re\{s\} > -10$}
\label{fig:31AEFEF90B_3D_surface_one_pole}
\end{subfigure}



\begin{subfigure}{0.49\textwidth}
\includegraphics[width=\textwidth]{../laplace_transform/fundamental_signals_laplace_plane_31AEFEF90B_cos.pdf}
\caption{$\cos(5 t) \epsilon(t)
\, \laplace \,
\frac{s}{s^2+25}
\text{ für }\Re\{s\} > 0$}
%\label{}
\end{subfigure}
\begin{subfigure}{0.5\textwidth}
\includegraphics[width=\textwidth]{../laplace_transform/fundamental_signals_laplace_plane_31AEFEF90B_sin.pdf}
%\label{}
\caption{$\sin(5 \, t) \epsilon(t)
\, \laplace \,
\frac{5}{s^2+25}
\quad\text{ für } \quad\Re\{s\} >  0$}
\end{subfigure}



\begin{subfigure}{0.49\textwidth}
\includegraphics[width=\textwidth]{../laplace_transform/fundamental_signals_laplace_plane_31AEFEF90B_double_pole_single_zero.pdf}
\caption{$\e^{-10 t} \cos(5 t) \epsilon(t)
\, \laplace \,
\frac{s+10}{(s+10)^2+25}
\text{ für }\Re\{s\} > -10$}
%\label{}
\end{subfigure}
\begin{subfigure}{0.5\textwidth}
\includegraphics[width=\textwidth]{../laplace_transform/fundamental_signals_laplace_plane_31AEFEF90B_double_pole.pdf}
%\label{}
\caption{$\e^{-10 t} \sin(5 \, t) \epsilon(t)
\, \laplace \,
\frac{5}{(s+10)^2+25}
\quad\text{ für } \quad\Re\{s\} >  -10$}
\end{subfigure}
%
%
%
\caption{Betrag von Laplace Transformierten in dB über die komplexe $s$-Ebene.}
\label{fig:31AEFEF90B_3D_surface}
\end{figure}
















\cleardoublepage
\subsection{Von der Essenz zur Anwendung: Elektrotechnik RC-Glied Tiefpass}
\label{sec:4408E33353}
\begin{Ziel}
Wir wollen mit dem SigSys-Werkzeugkoffer anhand des uns gut bekannten RC-Gliedes
bestimmte Ein-/Ausgangssignalpaare berechnen.
Anstatt zu Falten, also das Faltungsintegral zu lösen (dafür siehe Übung 2), wollen
wir hier die Laplace Transformation zu Hilfe nehmen.
Ziel ist zu erkennen, dass der beschrittene Lösungsansatz generisch funktioniert und
vergleichsweise einfach und vor allem konsistent
die Spezialfälle der Auf-/Entlade-/Wechselstromfälle, die
wir in ET kennengelernt haben, abbildet.
\end{Ziel}
\textbf{Aufgabe} {\tiny 4408E33353}: Berechnen Sie mit Werkzeugen der Systemtheorie
\begin{itemize}
  \item a) die Impulsantwort
  \item b) die Sprungantwort
  \item c) den eingeschwungenen Zustand auf eine eingeschaltete harmonische
  cos()-Anregung mit der Kreisfrequenz $\omega_\mathrm{RC}=\frac{1}{R \cdot C}$
\end{itemize}
für das unten dargestellte RC-Glied mit idealen Bauelementen (also Annahme, dass dies
ein LTI-System darstellt). $x(t)$ und $y(t)$ sollen Spannungen darstellen.
%
Das System befindet sich zu $t=0$ in Ruhe, $C$ ist also nicht aufgeladen, $y(0)=0$.
%
\begin{center}
\begin{circuitikz}[european, scale=0.75]
\node (in) at (1,0){};
\node (in_ground) at (1,-3){};
\node (out) at (4,0){};
\node (out_ground) at (4,-3){};
\draw (in) to [R,l_=$R$,o-] (3,0);
\draw (3,0) to [short,-o,] (out);
\draw (3,0) to [C,l_=$C$,*-*] (3,-3);
\draw (in_ground) to [short,o-o] (out_ground);
\path[draw, bend right, ->, >=latex] (in) edge node[left]{$x(t)$} (in_ground);
\path[draw, bend left, ->, >=latex] (out) edge node[right]{$y(t)$} (out_ground);
\end{circuitikz}
\end{center}
%
\begin{Werkzeug}
Wir öffnen den großen SigSys-Koffer, den wir schon ganz gut gefüllt haben...\\
Wir könnten die Aufgabe (bis vielleicht die Impulsantwort) auch mit Mitteln
lösen, die wir bisher in Mathe und ET gelernt haben.
%
Wir wollen aber hier genau mal alles das anwenden was wir bisher an SigSys-Tools
kennengelernt haben.
%
Weil wir für dieses System wissen, wie die Lösungen ausschauen müssen und
diese auch schon mal in ET interpretiert haben, fällt es uns hier leicht die
Rechenwege vergleichend einzuordnen.
%
\end{Werkzeug}
\begin{Ansatz}
Ausgangspunkt ist die bekannte DGL für dieses RC-Gied
\begin{align}
T_\mathrm{RC}\frac{\fsd}{\fsd t} y(t) + y(t) = x(t)
\end{align}
mit der Systemzeitkonstante $T_\mathrm{RC}=R \cdot C$ bzw. der systemspezifischen
Kreisgrenzfrequenz $\omega_\mathrm{RC}=\frac{1}{R \cdot C}$.
%

\noindent\textbf{Hinweis 1}: Dieses System hat, weil 1. Ordnung, nur eine
Zeitkonstante bzw. eine Systemgrenzfrequenz.
Systeme höherer Ordnung können mehrere Zeitkonstanten haben.

\noindent\textbf{Hinweis 2}: es gilt die
Relation $f=\frac{1}{T}$ zwischen physikalischer Frequenz $f$ und Periodendauer $T$ ABER
es gilt die Relation $\omega_\tau = \frac{1}{T_\tau}$ zwischen der System-\textbf{Kreis}grenz\textbf{frequenz}
$\omega_\tau $ und der System-\textbf{Zeitkonstante} $T_\tau $.

\noindent \textbf{Hinweis 3}:
An dieser Stelle machen wir uns den großen Vorteil der SigSys klar:
%
Wir behandeln hier zwar ein elektrotechnisches System (weil didaktisch
Verknüpfungen zu Sachen aus den vorangegangenen Semestern hergestellt werden
sollen), aber im Grunde geht es darum diese DGL losgelöst vom naturwissenschaftlichen
Ursprung mit SigSys-Denke zu lösen und Verhaltensweisen dieser DGL elegant
interpretierbar zu machen.
%
Diese DGL modelliert als einfachste Modellannahme den Wärmestrom von einem Körper
zu einem benachbarten, also z.B. das Abkühlen von Bier in einer wasserbefüllten
Badewanne ;-), that's the message. Fragen wir bei der nächsten
Gelegenheit Physiker*innen nach der genauen DGL...
\end{Ansatz}

\begin{ExCalc}
Die DGL
\begin{align}
T_\mathrm{RC}\frac{\fsd}{\fsd t} y(t) + y(t) = x(t)\\
\end{align}
wird Laplace transformiert (Achtung: $Y(s)$ vs. $y(t)$ beachten,
gerade handschriftlich eine beliebte Fehlerquelle)
\begin{align}
T_\mathrm{RC} \cdot s \cdot Y(s) - y(t=0) + Y(s) = X(s)
\end{align}
Weil $y(t=0)=0$ als Anfangsbedingung gegeben ist, können wir vereinfachen
\begin{align}
T_\mathrm{RC} \cdot s \cdot Y(s) + Y(s) = X(s)
\end{align}
und haben die DGL damit in eine algebraische Gleichung überführt: das ist die
elegante Idee der Laplace Transformation als Operator auf eine DGL.
%
Schreiben wir um zu
\begin{align}
&(T_\mathrm{RC} \cdot s + 1 ) \cdot Y(s) = X(s)
& (\frac{s}{\omega_\mathrm{RC}} + 1 ) \cdot Y(s) = X(s) \\
&H(s) = \frac{Y(s)}{X(s)} = \frac{1}{T_\mathrm{RC} \cdot s + 1}
&H(s) = \frac{Y(s)}{X(s)} = \frac{1}{\frac{s}{\omega_\mathrm{RC}} + 1}
\end{align}
%
$H(s)$ ist die sogenannte Übertragungsfunktion, welche
die Relation zwischen Aus- und Eingangs-Laplacetransformierter beschreibt.

Links die Gleichungen mit Benutzung der Zeitkonstante $T_\mathrm{RC}$ werden in der
Regelungstechnik bevorzugt, weil das Zeitverhalten der Systeme im Vordergrund
der Diskussion steht (vgl. Anlaufen einer Rolltreppe).
%
Rechts die Gleichungen mit Benutzung der Grenzfrequenz $\omega_\mathrm{RC}$ werden
in der Nachrichtentechnik, Akustik, Video, Audio genutzt, weil das Frequenzverhalten
der Systeme eher im Vordergrund steht und Systeme oft im Sinne eines Filters
aufgefasst werden. Ein digitales Foto in bestimmten Bereichen unschärfer machen und
Bassanhebung wären zwei typische Alltagsanwendungen die mit Filtern
(also LTI-Systemen) realisiert werden.

Folgen wir hier mal der Schreibweise mit Zeitkonstante, also
\begin{align}
Y(s) = H(s) \cdot X(s) = \frac{1}{T_\mathrm{RC} \cdot s + 1} \cdot X(s)
\end{align}
was wir noch umschreiben, damit wir es später einfacher haben (hier hilft Übung und Erfahrung)
\begin{align}
Y(s) = H(s) \cdot X(s) = \frac{1}{T_\mathrm{RC}} \cdot \frac{1}{s - (-\frac{1}{T_\mathrm{RC}})} \cdot X(s)
\end{align}

Wir können jetzt alle drei Teilaufgaben immer mit
dem gleichen Ansatz lösen, nämlich

\begin{itemize}
  \item das entsprechende $X(s)$ zum gewünschten
  Eingangssignal $x(t)$ ermitteln
  \item in die Gleichung $Y(s)=H(s) \cdot X(s)$ einsetzen
  \item Laplace Rücktransformieren um mit $Y(s) \Laplace y(t)$
\end{itemize}
das Ausgangssignal zu erhalten.
%
Das erscheint im Gegensatz zum direkten Lösen der DGL mit drei eigenen Ansätzen
für die verschiedene Inhomogenitäten elegant.
%
Es ist auch geringfügigere Rechnerei, als das Faltungsintegral zu lösen.
%
Wir können argumentieren, dass wir dann eine komplexes Integral bei der Rücktrafo
lösen müssen, was stimmt, aber das haben zum Glück viele Leute vor uns schon
in aller Ausführlichkeit betrieben. Die Wahrscheinlichkeit, dass eine
unbekannte Laplace Rücktransformation in der Praxis  vorkommt ist sehr gering.
%
In den vorangegangenen Übungsaufgaben haben wir alle Laplace Korrespondenzen
zusammengetragen, um das hier vorliegende Problem direkt lösen zu können.
%

Wir müssen uns klar machen, dass wir anstatt die Faltung
\begin{align}
y(t) = h(t) \ast x(t)
\end{align}
zu berechnen, hier die Multiplikation im Laplace Bereich
\begin{align}
Y(s) = H(s) \cdot X(s)
\end{align}
benutzen, um zur Lösung zu gelangen. Beide Operationen sind (absolut nicht zufällig)
miteinander durch
\begin{align}
\label{eq:4408E33353_duality}
y(t) = h(t) \ast x(t) \quad \laplace \quad Y(s) = H(s) \cdot X(s).
\end{align}
verknüpft. Diese \textbf{Dualität Faltung} Signalbereich A (hier Zeit)
\textbf{vs. Multiplikation} Signalbereich B (hier Laplace/Bild) müssen wir auf dem Schirm haben! Das ist
eine der wichtigsten Beziehungen von SigSys und im Grunde die Essenz.
Diese Dualität gilt auch für andere Transformationen, wie die vier Fourier
Transformationen die wir kennengelernt haben bzw. noch werden.
%

%Vorrausgeschickt seien die Gemeinsamkeiten / Unterschiede der verschiedenen
%Transformationen:

%Fourier Transformationen: Signal Analyse / Synthese, System Analyse ABER keine
%System Synthese (dafür braucht es die komplexe $s$-Ebene)

%Laplace Transformation: es geht alles, also Signal Analyse / Synthese,
%System Analyse / Synthese, deswegen auch das mächtigste Tool aus dem die anderen
%im Grunde abgeleitet werden können. Das macht man leider didaktisch noch zu selten.

%Nun aber zur Lösung der Aufgabe.

\end{ExCalc}
\begin{Loesung}


\begin{itemize}
  \item a) \textbf{Impulsantwort}

  Die Impulsantwort $h(t)$ ist die Antwort des LTI-Systems auf einen Dirac Impuls unter
  verschwindenden Anfangsbedingungen (hier müssen wir $y(0_-)=0$ ansetzen, weil
  nach unserer Konvention der Dirac Impuls exakt zu $t=0$ 'explodiert').

  Die Laplace Korrespondenz ist
  \begin{equation}
  x(t)=\delta(t) \quad\laplace\quad X(s) = 1
  \end{equation}
  %
  Es ergibt sich für die Übertragungsfunktion $H(s)=\frac{Y(s)}{X(s)}$
  der Spezialfall $H(s)=Y(s)$.

  Wenn wir nun allgemein rücktransformieren $Y(s) \Laplace y(t)$, erkennen wir
  für diesen Spezialfall, dass $H(s) \Laplace h(t)$.

  Diesen weiteren fundamentalen Zusammenhang dürfen wir nie aus den Augen verlieren:
  die \textbf{Laplace Transformierte der Impulsantwort ist die Übertragungsfunktion}

  Wir lösen zu

  \begin{align}
  Y(s) = \frac{1}{T_\mathrm{RC}}\cdot\frac{1}{s - (-\frac{1}{T_\mathrm{RC}})} \cdot 1
  \quad\Laplace\quad
  y(t) = \frac{1}{T_\mathrm{RC}} \cdot \e^{-\frac{t}{T_\mathrm{RC}}} \epsilon(t)
  \end{align}
  für eine kausale Impulsantwort $h(t)=y(t)$, weil wir gut vorgearbeitet hatten
  mit der bekannten Korrespondenz
  \begin{align}
  \e^{+s_0 \, t} \cdot \epsilon(t) \quad \laplace \quad \frac{1}{s-s_0} \quad\text{ für } \quad\Re\{s\} > \Re\{+s_0\}
  \end{align}

  Mit dieser Impulsantwort haben wir schon einmal in Übung 2.8 operiert, dort
  hatten wir diese gefaltet mit einem Rechteckimpuls, um zu schauen, wie das RC-Glied
  den Rechteckimpuls verschleift/filtert.

  Machen wir uns klar, dass \fig{fig:31AEFEF90B_3D_surface_one_pole}
  die Laplace Transformierte der gefundenen Impulsantwort für ein konkretes
  Zahlenbeispiel veranschaulicht, es fehlt dort aber der Zeitkonstanten-Gewichtungsfaktor.

  \item \textbf{b) Sprungantwort (i.e. Kondensatorspannung beim Aufladevorgang)}

Die Sprungantwort $h_\epsilon(t)$ ist die Antwort des LTI-Systems auf einen Einheitssprung.

Machen wir uns klar, dass bei LTI-Systemen alle Systeminformation in der Impulsantwort
steckt, siehe Übung 1.

Der aus Übung 2 bekannte Zusammenhang $\epsilon(t)' = \delta(t)$ führt bei LTI-Systemen
direkt zur Erkenntnis, dass auch
%(das können wir mit der Austasteigenschaft des
%Dirac Impulses formal für jedes $t$ zeigen)
\begin{eqnarray}
h_\epsilon(t)' = h(t)\quad\text{weil}\quad
h(t)=\mathcal{H}\{\delta(t)\}
\quad\text{und}\quad
h_\epsilon(t)=\mathcal{H}\{\epsilon(t)\}
\end{eqnarray}
gilt, also die zeitliche Ableitung der Sprungantwort die Impulsantwort darstellt.
%
Nachdem die zeitliche Ableitung ein linearer Operator ist, geht also formal keine
Information verloren: in der Sprungantwort stecken daher auch alle Systeminformationen
eines LTI-Systems, halt in anders verpackter Form.
%
Beide Anregungssignale, also der Dirac Impuls als auch der Einheitssprung sind in
der Praxis nicht exakt zu realisieren. Dirac Impuls bräuchte zu einem unendlich kurzen
Zeitpunkt unendlich viel Energie (vgl. Urknall der SigSys), der Einheitssprung
dauert nach Definition bis in alle Ewigkeit.
%
Mit einem Einheitssprung der nur sehr lange genug gegenüber der Systemzeitkonstanten
ist, bilden wir die Sprungantwort in der Praxis mit guter Näherung ab.
%
Bei Systemen die mit Gleichsignalen
(Strom, Spannung) gut klar kommen
(in der Regelungstechnik ist das oft der Fall), wäre diese Methode dann auch
praktisch umsetzbar.
%
Es gibt Systeme die nicht gut auf Gleichsignale zu sprechen sind (Lautsprecher
knacken und die darin befindliche Schwingspule wird heiß),
dort sollten wir auf die direkte Messung der Sprungantwort mittels
Einheitssprung verzichten.
%

Quizfrage: ein Lautsprecher mag eigentlich auch keinen kurzen Impuls hoher Amplitude
mit dem wir die Impulsantwort näherungsweise messen könnten. Was machen wir stattdessen
? Ist die ganze SigSys-Denke schon am Ende, weil wir es in der Praxis eh nicht
anwenden können?


Weiter mit der Lösung: die Korrespondenz des Einheitssprungs
haben wir uns vorher erarbeitet (Konvergenzbereich (KB) immer stillschweigend für kausale Signale)
\begin{equation}
x(t)=\epsilon(t) \quad \laplace \quad X(s)=\frac{1}{s}
\end{equation}

Setzen wir diese Korrespondenz ein, bekommen wir zunächst
\begin{align}
Y(s) = \frac{1}{T_\mathrm{RC}}\cdot\frac{1}{s + \frac{1}{T_\mathrm{RC}}} \cdot \frac{1}{s}
\end{align}
Ziel ist diesen Ausdruck als additive Überlagerung von bekannten Korrespondenzen
darzustellen, genau genommen in eine additive Überlagerung von Polstellen mit
unterschiedlicher Ausprägung
(reell, reell und doppelt, einfach komplex, konjugiert komplex, usw.) zu zerlegen.
Im Grunde ist das die Kochrezept-Anwendung des Residuensatzes.

Wir müssen also für die beiden Polstellen
\begin{equation}
\frac{1}{s + \frac{1}{T_\mathrm{RC}}} \cdot \frac{1}{s} =
\frac{A}{s + \frac{1}{T_\mathrm{RC}}} + \frac{B}{s}
\end{equation}
die äquivalente Darstellung über die Koeffizienten $A$, $B$ finden, bekannt als
Partialbruchzerlegung.
Wir bekommen zunächst die Ausdrücke
\begin{align}
&0\cdot s^1 + 1\cdot s^0 =
A\cdot s + B \cdot (s + \frac{1}{T_\mathrm{RC}})\\
&0\cdot s^1 + 1\cdot s^0 =
(A+B)\cdot s^1 + B \frac{1}{T_\mathrm{RC}} \cdot s^0
\end{align}
und für die Koeffizienten per Vergleich
\begin{equation}
  A=-B\qquad B = T_\mathrm{RC}
\end{equation}
Eingesetzt ergibt das
\begin{align}
&Y(s) = \frac{1}{T_\mathrm{RC}}\cdot\frac{1}{s + \frac{1}{T_\mathrm{RC}}} \cdot \frac{1}{s}\\
&Y(s) = \frac{1}{T_\mathrm{RC}}\cdot \left(\frac{A}{s + \frac{1}{T_\mathrm{RC}}} + \frac{B}{s}\right)\\
&Y(s) = \frac{1}{T_\mathrm{RC}}\cdot \left(\frac{-T_\mathrm{RC}}{s + \frac{1}{T_\mathrm{RC}}} + \frac{T_\mathrm{RC}}{s}\right)\\
&Y(s) = \frac{-1}{s - (-\frac{1}{T_\mathrm{RC}})} + \frac{1}{s}
\end{align}
Die Rücktransformation ist mit den vorher erarbeiteten Korrespondenzen schnell
erledigt
\begin{align}
Y(s) = \frac{1}{s} - \frac{1}{s - (-\frac{1}{T_\mathrm{RC}})} \quad \Laplace \quad
y(t) = \left(1-\e^{-\frac{t}{T_\mathrm{RC}}}\right)\cdot \epsilon(t),
\end{align}
und damit ist die Sprungantwort $h_\epsilon(t) = y(t)$ gefunden.
In Übung 2.8 haben wir als Abfallprodukt die Sprungantwort an einem dimensionierten
RC-Glied schon mal mit berechnet. Zudem sollte uns die Formel und der Verlauf des
Signals $h_\epsilon(t)$ sehr bekannt vorkommen aus der ET,
im großen Kapitel Gleichspannungen und -ströme und Auf-/Entladevorgänge
an passiven Energiespeichern. Wir haben das
da (mutmaßlich) nicht als Sprungantwort bezeichnet.


\item c) \textbf{Steady State (Verhältnis Aus- zu Eingangsspannung des
RC-Gliedes bei Harmonischer Anregung)}

Wir setzten fort und lösen im Wesen das was wir in der ET als weiteres großes
Kapitel behandelt haben: eingeschwungene Zustände mit komplexer Wechselstromtechnik.
Rechnen mit komplexen Zeigern ist eine gutes Tool, aber the bigger picture erschließt
sich damit nicht notwendig als Selbstläufer.

Wir interessieren uns für den Fall, was das System im eingeschwungenen Zustand
treibt, wenn wir das RC-Glied mit einer harmonischer Schwingung (z.B. mit typischer
Stromnetzfrequenz 50 Hz) anregen.
Also eine ganz klassische Aufgabe für komplexe Spannungsteiler, Zeigerbild usw.
Das ist das einfachste und uns schon bekannte Lösungstool, aber es geht eben auch
nur! das.

Für SigSys ist die gestellte Aufgabe nun just another Eingangssignal, und wir lösen
mit dem gleichen Weg wie schon zuvor.

Machen wir den Ansatz eines zu $t=0$ eingeschalteten Cosinussignals, also benutzen
\begin{equation}
x(t)=\cos(\omega_\mathrm{RC} t ) \epsilon(t) \quad \laplace \quad X(s)=\frac{s}{s^2+\omega_\mathrm{RC}^2}
\end{equation}

Einsetzen in $Y(s) = H(s)\cdot X(s)$ bringt
%
\begin{align}
Y(s) = H(s) \cdot X(s) = \frac{1}{T_\mathrm{RC}} \cdot \frac{1}{s - (-\frac{1}{T_\mathrm{RC}})} \cdot \frac{s}{s^2+\omega_\mathrm{RC}^2}
\end{align}
Für die manuelle Rechnung ist es hier ratsam entweder nur $\omega_\mathrm{RC}$ oder
$T_\mathrm{RC}$ zu benutzen, um nicht durcheinander zu kommen.
%
Eine Partialbruchzerlegung (\textbf{das sollten wir für die Klausur ein wenig üben
anhand der Klausuraufgaben zur Laplace Transformation oder der Aufgaben 3.7 und 3.8
aus dem SS2019})
 führt wieder zum Ziel, d.h. additive Überlagerung von
'Polstellen' (bekannte Korrespondenzen, es sind genau genommen keine reinen Polstellen,
sondern es können auch Nullstellen vorkommen, also gebrochen rationale Funktionen).

Der Ansatz für eine reelle Polstelle und eine konjugiert-komplexe Polstelle (
$T_\mathrm{RC}$ auf die linke Seite bringen macht sich gut
)
\begin{align}
% Y(s) = H(s) \cdot X(s) = \frac{1}{T_\mathrm{RC}} \cdot \frac{1}{s - (-\frac{1}{T_\mathrm{RC}})} \cdot \frac{s}{s^2+\omega_\mathrm{RC}^2}\\
% Y(s) = H(s) \cdot X(s) = \frac{1}{T_\mathrm{RC}} \cdot \frac{1}{s + \omega_\mathrm{RC}} \cdot \frac{s}{s^2+\omega_\mathrm{RC}^2}\\
T_\mathrm{RC} Y(s) = \frac{1}{s + \omega_\mathrm{RC}} \cdot \frac{s}{s^2+\omega_\mathrm{RC}^2} = \frac{A}{s+\omega_\mathrm{RC}} + \frac{B s + C}{s^2+\omega_\mathrm{RC}^2}
\end{align}
verlangt Koeffizientenvergleich für
\begin{align}
s =& A (s^2+\omega_\mathrm{RC}^2) + (B s + C) (s+\omega_\mathrm{RC})
%s =& A s^2 + A \omega_\mathrm{RC}^2 + B s^2 + B s \omega_\mathrm{RC} + C s + C \omega_\mathrm{RC}
\end{align}
und führt zu Gleichungssystem
\begin{align}
0 =& A + B\\
1 =& B \omega_\mathrm{RC} + C\\
0 =& A \omega_\mathrm{RC}^2 + C \omega_\mathrm{RC}
\end{align}
% \begin{align}
% A =& -B\\
% 1 =& -A \omega_\mathrm{RC} + C \rightarrow 1 + A \omega_\mathrm{RC} = C\\
% 0 =& A \omega_\mathrm{RC}^2 + C \omega_\mathrm{RC}
% \end{align}
% \begin{align}
% A =& -B\\
% C = & 1 + A \omega_\mathrm{RC}\\
% 0 =& A \omega_\mathrm{RC}^2 + (1 + A \omega_\mathrm{RC}) \omega_\mathrm{RC}
% \end{align}
% \begin{align}
% A =& -B\\
% C = & 1 + A \omega_\mathrm{RC}\\
% 0 =& A \omega_\mathrm{RC}^2 + A \omega_\mathrm{RC}^2 + \omega_\mathrm{RC}
% \end{align}
% \begin{align}
% A =& -B\\
% C = & 1 + A \omega_\mathrm{RC}\\
% -\omega_\mathrm{RC} =& 2 A \omega_\mathrm{RC}^2 + A \omega_\mathrm{RC}^2
% \end{align}
% \begin{align}
% A =& -B\\
% C = & 1 + A \omega_\mathrm{RC}\\
% A =& -\frac{1}{2\omega_\mathrm{RC}}
% \end{align}
was wir lösen können zu
\begin{align}
A = -\frac{1}{2\omega_\mathrm{RC}},\,B = +\frac{1}{2\omega_\mathrm{RC}},\,C = \frac{1}{2}
\end{align}
Nun setzen wir die Koeffizienten ein
\begin{align}
T_\mathrm{RC} Y(s) = \frac{-\frac{1}{2\omega_\mathrm{RC}}}{s+\omega_\mathrm{RC}} + \frac{\frac{1}{2\omega_\mathrm{RC}} s + \frac{1}{2}}{s^2+\omega_\mathrm{RC}^2}
%T_\mathrm{RC} Y(s) = -\frac{1}{2\omega_\mathrm{RC}}\cdot\frac{1}{s+\omega_\mathrm{RC}} + \frac{\frac{1}{2\omega_\mathrm{RC}} s + \frac{1}{2}}{s^2+\omega_\mathrm{RC}^2}\\
%T_\mathrm{RC} Y(s) = -\frac{1}{2\omega_\mathrm{RC}}\cdot\frac{1}{s+\omega_\mathrm{RC}} + \frac{1}{2\omega_\mathrm{RC}}\cdot\frac{s + \omega_\mathrm{RC}}{s^2+\omega_\mathrm{RC}^2}\\
%Y(s) = -\frac{1}{2\omega_\mathrm{RC} T_\mathrm{RC} }\cdot\frac{1}{s+\omega_\mathrm{RC}} + \frac{1}{2\omega_\mathrm{RC}T_\mathrm{RC} }\cdot\frac{s + \omega_\mathrm{RC}}{s^2+\omega_\mathrm{RC}^2}\\
%Y(s) = -\frac{1}{2}\cdot\frac{1}{s+\omega_\mathrm{RC}} + \frac{1}{2}\cdot\frac{s + \omega_\mathrm{RC}}{s^2+\omega_\mathrm{RC}^2}\\
%Y(s) = -\frac{1}{2}\cdot\frac{1}{s-(-\frac{1}{T_\mathrm{RC}})} + \frac{1}{2}\cdot\frac{s + \omega_\mathrm{RC}}{s^2+\omega_\mathrm{RC}^2}
\end{align}
und stellen um ($T_\mathrm{RC}$ wieder zurück auf die rechte Seite),
bis wir bekannte Korrespondenzen entdecken, also hier
\begin{align}
Y(s) = \frac{1}{2} \left(
\frac{s}{s^2+\omega_\mathrm{RC}^2} + \frac{\omega_\mathrm{RC}}{s^2+\omega_\mathrm{RC}^2} - \frac{1}{s-(-\frac{1}{T_\mathrm{RC}})}
\right)
\end{align}
%
Die Rücktransformation der einzelnen Terme führt zum Ausgangssignal (also der
Spannung über dem Kondensator)
\begin{align}
  y(t) = \frac{1}{2}\left(
  \cos(\omega_\mathrm{RC} t) + \sin(\omega_\mathrm{RC} t) - \e^{-\frac{t}{T_\mathrm{RC}}}
  \right) \epsilon(t).
\end{align}
Dieses Ergebnis sollte noch sinnvoll vereinfacht werden, weil wir dann erst
den schönen Link zur ET sehen.
%
Wir könnten jetzt Additionstheoreme für die Addition von cos() und sin() nachschlagen
oder selber neu erfinden mittels Euleridentitäten.
%
Es geht in dem Fall aber auch maximal bequem mit einem einfachen Zeigerdiagramm
weil sin und cos mit gleicher Kreisfrequenz schwingen, die Zeigermethode ist hier
sehr nützlich. Wir legen den cos() auf die Phasenbezugslinie.
Das machen wir intentional, weil wir als Endergebnis
einen cos()-Term haben wollen, damit wir die Relation zum cos()-Eingangssignal
einfach sehen. Das Sinussignal ist dem Cosinussignal $\frac{\pi}{2}$ nachfolgend.
Das Zeigerdiagramm ist also
\begin{center}
\begin{tikzpicture}[scale=2]
\draw[-] (0,0)--(1.5,0) node[right]{Phasen-Bezugslinie};
\draw[->, C0, ultra thick] (0,0)--(1,0) node[above]{$A_\textrm{cos}=1$};
\draw[->, C0, ultra thick] (0,0)--(0,-1) node[left]{$A_\textrm{sin}=1$};
\draw[-, C7, thin] (0,-1)--(1,-1);
\draw[-, C7, thin] (1,0)--(1,-1);
\draw[->, C1, ultra thick] (0,0)--(1,-1) node[below]{$A_{\sum_{}^{}}=\sqrt{2}$};
\end{tikzpicture}
\end{center}
und daraus können wir zunächst aus der Vektoraddition die resultierende Amplitude und
die Lage bzgl. der Phasenbezugslinie feststellen zu
\begin{equation}
  \sin(\phi) + \cos(\phi) = \sqrt{2} \cos(\phi-\frac{\pi}{4})
\end{equation}
Damit ergibt sich
\begin{align}
  y(t) = \frac{1}{2}\left(
  \sqrt{2}\,\cos(\omega_\mathrm{RC} t - \frac{\pi}{4}) - \e^{-\frac{t}{T_\mathrm{RC}}}
  \right) \epsilon(t)
\end{align}
Ein nun wichtiger \textbf{Speziallfall} ist der sogenannte \textbf{eingeschwungene
Zustand (steady state)} des Systems.
%
Das ist der Zustand wo das Ausgangssignal keine Artefakte mehr beinhaltet, die
vom Einschalten herrühren.
In unserem speziellen Fall ist das alles was mit dem sprunghaften Einschalten
des Cosinus zu $t=0$ zu tun hat, also der
$\e^{-\frac{t}{T_\mathrm{RC}}}$-Anteil.
%
Eingeschwungener Zustand gilt ganz streng genommen, wenn $t\to\infty$.
In der Praxis erreicht ein System den eingeschwungenen Zustand wenn wir nur viel
länger (aber eben nicht unendlich) als die längste Zeitkonstante des Systems
abwarten (beim RC-Glied haben wir nur eine, nämlich $T_\mathrm{RC}$).
%
Wenn also $t \gg T_\mathrm{RC}$,
dann geht $\e^{-\frac{t}{T_\mathrm{RC}}}$ mit guter Näherung gegen Null
und es verbleibt
\begin{align}
  y(t \gg T_\mathrm{RC}) =& \frac{1}{\sqrt{2}} \,\cos(\omega_\mathrm{RC} t - \frac{\pi}{4})\\
  y(t \gg T_\mathrm{RC}) =& \frac{1}{\sqrt{2}} \,\cos(\omega_\mathrm{RC} \left[t - \frac{\pi}{4} T_\mathrm{RC}\right])
\end{align}
Zweite Gleichung weil immer $T_\mathrm{RC}\cdot \omega_\mathrm{RC} =1$ gilt, und diese Darstellung ist nun ideal für erneute Übung
zur Zeitverschiebung und Zeitskalierung.

Die spezielle Amplitudenabnahme um Faktor $\frac{1}{\sqrt{2}}$ und Phasenverschiebung
um $-\frac{\pi}{4}$ sollte uns bekannt vorkommen. Wir haben als Signalfrequenz
die sogenannte Grenzfrequenz des Systems gewählt, also
$\omega_\mathrm{RC} = \frac{1}{T_\mathrm{RC}} = \frac{1}{R \cdot C}$.
Das war natürlich didaktisch so gewählt.
%
Der Vorteil der vollständigen Lösung mit Einschalten ist, dass wir
die \textbf{absolute Phasenverschiebung} als Abfallprodukt mit dazu bekommen.
%
Wenn wir mit komplexer Wechselstromtechnik \textbf{nur den eingeschwungenen Zustand}
berechnen, kennen wir \textbf{nur die relative} Phasenverschiebung
zwischen Aus- und Eingang, wissen aber nicht, wie dieses tatsächlich zustande kommt.
Es kann ja durchaus sein, dass das System eine zusätzliche Verzögerung
um mehrere Periodendauern beinhaltet,
die im gleichen relativen Phasenwinkel $-\frac{\pi}{4}$ endet.
Oder aber (in der Praxis sehr
unwahrscheinlich, weil es keine akausalen Systeme gibt, aber um den Punkt zu machen
ein hilfreiches Bild): der Ausgang kommt früher als der Eingang.
Das sehen wir in den komplexen Zeigern nicht.

In \fig{fig:4408E33353} ist das Systemverhalten für die drei betrachteten Szenarien
übersichtlich dargestellt.
%
Die Grafik ist absichtlich nicht mit irgendeiner gewählten Dimensionierung
für $R$ und $C$ gemacht, sondern es ist alles allgemein gültig schematisch gezeichnet.
Das System wird oft als \textbf{Tiefpass 1. Ordnung} (Nachrichtentechnik) und
\textbf{PT$_1$-Glied} (Regelungstechnik) bezeichnet.

Anhand der Grafiken, sollten wir uns klarmachen, wie der Pol wandert und
wie sich damit die Signale über die Zeit ändern
wenn
\begin{itemize}
  \item $T_\mathrm{RC}$ vergrößert, also $\omega_\mathrm{RC}$ verkleinert wird
  \item $T_\mathrm{RC}$ verkleinert, also $\omega_\mathrm{RC}$ vergrößert wird
  \item was im Grenzfall passiert, wenn $T_\mathrm{RC}\to\infty$, also
  $\omega_\mathrm{RC}\to 0$ (Hinweis: idealer Integrator)
\end{itemize}

% \begin{align}
% &\frac{1}{2} \left(\frac{e^{\im x}}{2\im} - \frac{e^{-\im x}}{2\im} + \frac{e^{\im x}}{2} + \frac{e^{-\im x}}{2}\right)\\
% &\frac{1}{2} \left(\frac{e^{\im (x+\frac{\pi}{4})}}{2\im} - \frac{e^{-\im (x-\frac{\pi}{4})}}{2\im} + \frac{e^{\im (x+\frac{\pi}{4})}}{2} + \frac{e^{-\im (x-\frac{\pi}{4})}}{2}\right)
% \e^{-\im\frac{\pi}{4}}\\
% &\frac{1}{2} \left(\frac{e^{\im (x+\frac{\pi}{4}-\frac{\pi}{2})}}{2} + \frac{e^{-\im (x-\frac{\pi}{4}-\frac{\pi}{2})}}{2} + \frac{e^{\im (x+\frac{\pi}{4})}}{2} + \frac{e^{-\im (x-\frac{\pi}{4})}}{2}\right)
% \e^{-\im\frac{\pi}{4}}\\
% &\frac{1}{2} \left(\frac{e^{\im (x-\frac{\pi}{4})}}{2} + \frac{e^{-\im (x-\frac{3\pi}{4})}}{2} + \frac{e^{\im (x+\frac{\pi}{4})}}{2} + \frac{e^{-\im (x-\frac{\pi}{4})}}{2}\right)
% \e^{-\im\frac{\pi}{4}}
% \end{align}

\end{itemize}
\end{Loesung}


\begin{figure}[h]
\begin{center}
%sigma < 0
\begin{tikzpicture}
%
\def \axisLength {4}
\def \tic {0.05}
\def \sigmaz {1}
\def \omegaz {1}
\def \convAbsz {-\sigmaz}
\fill[C2!50] (\convAbsz,-\axisLength/2)--(\convAbsz,\axisLength/2)
decorate [decoration={snake,segment length=15pt,amplitude=1pt}]
{(\convAbsz,\axisLength/2)--
(\axisLength/2,\axisLength/2)--
(\axisLength/2,-\axisLength/2)--
(\convAbsz,-\axisLength/2)};
\draw[->] (-\axisLength/2,0)--(\axisLength/2,0) node[right]{\small$\Re\{s\}$};
\draw[->] (0,-\axisLength/2)--(0,\axisLength/2) node[above]{\small$\Im\{s\}$};
\draw[C0, ultra thick] (-\sigmaz,0) node{\Huge $\times$};
\draw (-\sigmaz,\tic)--(-\sigmaz,-\tic) node[below]{$-\frac{1}{T_\mathrm{RC}}$};
\draw (-\tic,\omegaz) -- (\tic,\omegaz) node[right]{$+\omega_\mathrm{RC}=+\frac{1}{T_\mathrm{RC}}$};
\draw (-\tic,-\omegaz) -- (\tic,-\omegaz) node[right]{$-\omega_\mathrm{RC}=-\frac{1}{T_\mathrm{RC}}$};
\draw (1.25,+2.25) node[C2!75]{KB};
\draw (1.25,1.75) node[]{$g=\frac{1}{T_\mathrm{RC}}$};
%
\begin{scope}[shift={(5,-1.5)}]
\begin{axis}[
width=0.45\textwidth,
height=0.3\textwidth,
domain=0:7,
samples=64,
legend pos=outer north east,
xlabel = {$t\rightarrow$},
ylabel = {$h(t)$},
title = {Impulse Response / Impulsantwort},
xmin=-0.1, xmax=7,
ymin=-0.1, ymax=1.1,
xtick={0,1,3,5},
ytick={0, 0.3678, 1},
xticklabels={$0$, $T_\mathrm{RC}$, $3 T_\mathrm{RC}$, $5 T_\mathrm{RC}$},
yticklabels={$0$, $\frac{0.37}{T_\mathrm{RC}}$, $\frac{1}{T_\mathrm{RC}}$},
ymajorgrids=true,
xmajorgrids=true
]
\addplot[mark=None, color=C0, ultra thick]
coordinates {(-4,0)(0,0)(0,1)};
\addplot[mark=None, color=C0, ultra thick]
{exp(-\sigmaz*x)};
\end{axis}
\end{scope}
%
%
%
%
\begin{scope}[shift={(-3,-6.5)}]
\begin{axis}[
width=0.45\textwidth,
height=0.3\textwidth,
domain=0:7,
samples=64,
legend pos=outer north east,
xlabel = {$t\rightarrow$},
ylabel = {$h_\epsilon(t)$},
title = {Step Response / Schrittantwort},
xmin=-0.1, xmax=7,
ymin=-0.1, ymax=1.1,
xtick={0, 1, 3, 5},
ytick={0,0.63,0.95, 1},
xticklabels={$0$, $T_\mathrm{RC}$, $3 T_\mathrm{RC}$, $5 T_\mathrm{RC}$},
yticklabels={$0$, $0.63$, $ $, $1$},
ymajorgrids=true,
xmajorgrids=true
]
\addplot[mark=None, color=C0, ultra thick]
coordinates {(-4,0)(0,0)};
\addplot[mark=None, color=C0, ultra thick]
{1-exp(-\sigmaz*x)};
\end{axis}
\end{scope}
%
%
%
%
\begin{scope}[shift={(5,-6.5)}]
\begin{axis}[
width=0.45\textwidth,
height=0.3\textwidth,
domain=0:7,
samples=50,
legend pos=outer north east,
xlabel = {$t\rightarrow$ für $t_1 \gg T_\mathrm{RC}$},
ylabel = {$\textcolor{C0}{x(t)},\,\,\,\textcolor{C3}{y(t)}$},
title = {Steady State / Eingeschwungen},
xmin=0, xmax=7,
ymin=-1.1, ymax=1.1,
xtick={0,pi/2,pi/2+pi/4,pi,pi+1,2*pi},
ytick={-1,-0.7071,0,0.7071,1},
xticklabels={$t_1$, $ $, $ $, $ $, $ $, $t_1+2\pi T_\mathrm{RC}$},
yticklabels={$-1$, $-\frac{1}{\sqrt{2}}$, $0$, $\frac{1}{\sqrt{2}}$, $1$},
ymajorgrids=true,
xmajorgrids=true
]
\addplot[mark=None, color=C0, ultra thick]
{cos(deg(x))};
\addplot[mark=None, color=C3, ultra thick]
{0.7071*cos(deg(x-pi/4))};
\addplot[mark=None, color=C7, ultra thick]
coordinates {(pi,0)(pi+1,0)};
\draw (370,130) node[]{$T_\mathrm{RC}$};
%
\addplot[mark=None, color=C7, ultra thick]
coordinates {(pi/2,0)(pi/2+pi/4,0)};
\draw (180,90) node[]{$\frac{\pi}{4} T_\mathrm{RC}$};
\end{axis}
\end{scope}
%
%
%
\end{tikzpicture}
\end{center}
%
%
%
\caption{Beschreibung des Ein-Pol LTI-Systems gemäß DGL
$T_\mathrm{RC}\frac{\fsd}{\fsd t} y(t) + y(t) = x(t)$ mittels
\textbf{Polstellen-Nullstellen Diagramm} der Laplace Transformation
$H(s)$ von der Impulsantwort $h(t)$,
Impulsantwort $h(t)$ und Sprungantwort $h_\epsilon(t)$.
Speziell: eingeschwungener Zustand für die Kreisfrequenz
$\omega_\mathrm{RC}=\frac{1}{T_\mathrm{RC}}$
mit charakteristischer Amplitudenabnahme um Faktor $\frac{1}{\sqrt{2}}$ und
Phasenverschiebung um $-\frac{\pi}{4}$. Grafik für Aufgabe \ref{sec:4408E33353}}
\label{fig:4408E33353}
\end{figure}


\clearpage
\begin{mdframed}
\textbf{Ausblick auf UE 4 und 5}:
Wir werden in den nächsten Einheiten weiter lernen, wie sich das System
für $\omega\ll\omega_\mathrm{RC}$ und $\omega\gg\omega_\mathrm{RC}$
verhält (das wissen wir eigentlich schon aus ET) und wie sich das elegant
interpretieren lässt (SigSys).
Wagen wir einen kleinen Vorgriff: die Laplace Variable $s=\sigma + \im\omega$
kann ja den Fall $\sigma=0$ beinhalten, wir sind im Konvergenzbereich
für das betrachtete System. Wir werten die Übertragungsfunktion
\begin{align}
H(s) = \frac{Y(s)}{X(s)} = \frac{1}{T_\mathrm{RC} \cdot s + 1}
\end{align}
dann aus für den Fall $s=\im\omega$
\begin{align}
H(\im\omega) = \frac{Y(\im\omega)}{X(\im\omega)} =
\frac{1}{T_\mathrm{RC} \cdot \im\omega + 1}
\end{align}
Das könnte uns bekannt vorkommen aus der Wechselstromtechnik, vielleicht eher
notiert mit komplexen Spannungsvariablen am RC-Glied
\begin{align}
\frac{\underline{U}_A}{\underline{U}_E} =
\frac{1}{R\,C \cdot \im\omega + 1}.
\end{align}
Dies ist der \textbf{komplexe Spannungsteiler} und stellt
ausschließlich den \textbf{eingeschwungenen Zustand} dar.

In SigSys ist die Funktion $H(\im\omega)$ (also der Spezialfall
für $\sigma=0$!) so wichtig, dass sie einen eigenen Namen erhält: dies ist
der komplexwertige \textbf{Frequenzgang des Systems} der als Betrag und Phase
über die Kreisfrequenz dargestellt werden kann.
Wir haben mutmaßlich schon mal etwas vom Bodediagramm gehört, dahin geht die Reise...

\end{mdframed}

\textbf{Hinweise für weitere Vertiefung}:
\begin{itemize}
  \item Zur vertiefenden Übung für Laplace Hintransformation seien die Übungen SS2019 3.5 und 3.6 empfohlen
  \item Rücktransformation mittels Partialbruchzerlegung kann mit Übungen SS2019 3.7 und 3.8 geübt werden.
  Sie kommt typisch auch bei allen Klausuraufgaben zur Laplace Transformation vor.
  \item Die Lösung für ein LTI-System 2. Ordnung ist in
  \verb|SigSys_UE3_Lowpass2nd.pdf| geschildert. Es lohnt sich sehr das durchzuarbeiten.
 Es zeigt kompakt die zu Fuß Lösung der DGL mittels Fundamentalsystem und mittels Laplace Transformation.
 Der Teil Laplace Transformation ist sehr hilfreich für die Klausurvorbereitung.
\end{itemize}

\newpage
\section{UE 11: Diskrete Fourier Transformation (DFT)}
%
Wir betrachten eine vierten Typ der für SigSys wichtigen Fourier Transformationen,
die Diskrete Fourier Transformation (DFT).
Sie ist die einzige, die exakt mit Computern berechenbar ist (wenn wir
von numerischen Rundungsfehlern absehen).
%
Die anderen uns bisher bekannten drei Fourier Transformation (FT, FR, DTFT)
setzen ja entweder
unendliche lange Signale, unendlich viele Spektralkoeffizienten bzw.
kontinuierliche Signale/Spektren
voraus. Das sind alles Daten-Eigenschaften mit denen Digital-Computer wegen endlicher
Rechen- und Speicherressourcen nicht nativ umgehen können.
%
Daher werden wir sehr oft konfrontiert sein, die DFT als Approximation
der anderen drei Transformationen benutzen zu müssen, weswegen wir eigentlich
besonders fit in DFT sein müssen.
%

Eine der Grundeigenschaften der DFT, nämliche die inhärente
Periodizität sowohl der Signalfolgen als auch der DFT-Spektren jeweils mit
der Länge $N$, ist nun eigentlich
eher praxisfern. Die meisten Signale sind ja eher nicht periodisch, und wenn
doch dann meist überlagert mit nichtperiodischem (Mess)-Rauschen.
%
Wir haben mit der DFT also ein Werkzeug, was gut ist für digitale Rechentechnik,
aber schlecht für die meisten praktisch vorkommenden Signale.
%
Das macht die DFT nun nicht automatisch zu einem schlechten Tool. Wir müssen
es aus Mangel an Alternativen eh wertschätzen und uns erarbeiten, was die
DFT unter verschiedenen Annahmen macht und wie die Ergebnisse zu interpretieren
sind.
%
Dafür müssen wir die DFT-Synthese und DFT-Analyse
und die Eigenschaften der DFT sicher beherrschen.
%
%
Es ist deswegen sinnvoll, wenn wir uns das tiefgründig erarbeiten. Eine einzige
VL/UE-Einheit ist dafür im Grunde zu wenig, daher sollten wir Zeit für
Selbststudium einplanen. Wir vertiefen die DFT im Wintersemester
Mastermodul Digital Signal Processing.
Hier zunächst der Versuch die Essenz in übersichtlicher Form weiterzugeben.

Wir benutzen die Operator Symbole
\begin{align}
\text{DFT: }x[k] &\quad\mydft\quad X[\mu]\\
\text{IDFT: }X[\mu] &\quad\myDFT\quad x[k].
\end{align}
%
Die Analyse Formel ist mit unserer Konvention
(bzgl. der Normierung mit $\frac{1}{N}$ und Vorzeichenwahl im komplexen Dreher)
(Matlab \texttt{fft()}, Python: \texttt{scipy.fft.fft()} bzw. \texttt{numpy.fft.fft()})
\begin{align}
X[\mu] = \mathrm{DFT}_N\{x[k]\} = \sum_{k=0}^{N-1} x[k] \, \e^{-\im\frac{2\pi}{N} k \cdot \mu},
\end{align}
dann lautet die Synthese Formel (Matlab, Python: \texttt{ifft()})
\begin{align}
x[k] = \mathrm{IDFT}_N\{X[\mu]\} = \frac{1}{N} \sum_{\mu=0}^{N-1} X[\mu] \, \e^{+\im\frac{2\pi}{N} k \cdot \mu}.
\end{align}
%
Für eine komplette Transformation brauchen wir alle auftretenden Multiplikationen $k \cdot \mu$,
die wir als $k \times \mu$ Matrix $\bm{K}$ aufschreiben können (wir wollen Summen vermeiden,
die sind unübersichtlich und verschleiern meist den tieferen Zusammenhang)
\begin{align}
\bm{K} =
k \downarrow
\substack{\rightarrow \mu\\
\begin{bmatrix}
0 \cdot 0 & 0 \cdot 1 & 0 \cdot 2 & 0 \cdot 3 & \dots & 0 \cdot (N-1)\\
1 \cdot 0 & 1 \cdot 1 & 1 \cdot 2 & 1 \cdot 3 & \dots & 1 \cdot (N-1)\\
2 \cdot 0 & 2 \cdot 1 & 2 \cdot 2 & 2 \cdot 3 & \dots & 2 \cdot (N-1)\\
3 \cdot 0 & 3 \cdot 1 & 3 \cdot 2 & 3 \cdot 3 & \dots & 3 \cdot (N-1)\\
\vdots & \vdots & \vdots & \vdots & \ddots & \vdots\\
(N-1) \cdot 0 & (N-1) \cdot 1 & (N-1) \cdot 2 & (N-1) \cdot 3 & \dots &  (N-1) \cdot (N-1)
\end{bmatrix}
}
\end{align}
Wir führen die komplexe Zahl $z_1 = W_N = \e^{+\im\frac{2\pi}{N}}$ ein, die
in der $z$-Ebene auf dem Einheitskreis liegt, und die spezielle DTFT
Frequenz $\Omega = \frac{2\pi}{N}$ abbildet. In Analogie zur Fourierreihe ist
das die Frequenz der ersten DFT-Harmonischen, sozusagen die DFT-Grundfrequenz.
Mit der zweiten Matrix-Zeile (wo $k=1$) können wir erkennen,
dass es insgesamt $N$ DFT Frequenzen
$W_N^\mu=\left(\e^{+\im\frac{2\pi}{N}}\right)^\mu$ für $0\leq\mu\leq N-1$
auf dem Einheitskreis gibt.
%
Dies ist in \fig{fig:DFT_UnitCircle} für $N=4$ links und für $N=5$ rechts
veranschaulicht.
Für $N\to\infty$ nähern wir uns der DTFT an, weil wir eine unendliche Anzahl
von Frequenzstützstellen äquiangular auf dem Einheitskreis verteilen.
%
\begin{figure}[t]
\center
\begin{tikzpicture}[scale=1.5]
\def \tic {0.05}
%
\begin{scope}
\draw[C3, thick] (0,0) circle(1);  % unit circle, i.e. DTFT domain
\draw[C3] (0.78,-0.78) node[right]{DTFT $z=\e^{\im\Omega}$};
\draw[C0] (0.78,-1.15) node[right]{DFT $z_\mu = \e^{\im\frac{2\pi}{4}\mu}$};
%
\draw (1+2*\tic,-3*\tic) node{$1$}; % indicate that this is the unit circle
\draw[->] (-1.25,0)--(1.75,0) node[below]{$\Re\{z\}$}; % axis label
\draw[->] (0,-1.25)--(0,1.5) node[above]{$\Im\{z\}$}; % axis label
%
\draw[C0, ultra thick] (0,+1) node{\Huge $\circ$};
\draw[C0, ultra thick] (0,-1) node{\Huge $\circ$};
\draw[C0, ultra thick] (+1,0) node{\Huge $\circ$};
\draw[C0, ultra thick] (-1,0) node{\Huge $\circ$};
%
\draw[] (+1+1*\tic,4*\tic) node[right]{$z_0=(W_4)^0=1$};
\draw[] (2*\tic,+1+4*\tic) node[right]{$z_1=(W_4)^1=\e^{\im\frac{2\pi}{4}}$};
\draw[] (-1+4*\tic,4*\tic) node{$z_2=(W_4)^2$};
\draw[] (4*\tic,-1+4*\tic) node{$z_3$};
%
\draw[->] (0.25,0) node[above]{$\frac{2\pi}{4}$} arc (0:90:0.25) ;
%
\draw[] (0,2.25) node{Lösungen für $z^4 = 1$ für $N=4$ DFT};
\end{scope}
%
\begin{scope}[xshift=5cm]
\draw[C3, thick] (0,0) circle(1);  % unit circle, i.e. DTFT domain
\draw[C3] (0.78,-0.78) node[right]{DTFT $z=\e^{\im\Omega}$};
\draw[C0] (0.78,-1.15) node[right]{DFT $z_\mu = \e^{\im\frac{2\pi}{5}\mu}$};
%
\draw (1+2*\tic,-3*\tic) node{$1$}; % indicate that this is the unit circle
\draw[->] (-1.25,0)--(1.75,0) node[below]{$\Re\{z\}$}; % axis label
\draw[->] (0,-1.25)--(0,1.5) node[above]{$\Im\{z\}$}; % axis label
%
\draw[white] (0,-1)--(0,+1);
\draw[white] (-1,0)--(+1,0);
%
\draw[C0, ultra thick] (1,0) node{\Huge $\circ$};
\draw[C0, ultra thick] (0.30902, 0.95106) node{\Huge $\circ$};
\draw[C0, ultra thick] (-0.80902, 0.58779) node{\Huge $\circ$};
\draw[C0, ultra thick] (-0.80902,- 0.58779) node{\Huge $\circ$};
\draw[C0, ultra thick] (0.30902,- 0.95106) node{\Huge $\circ$};
%
\draw[] (+1+1*\tic,4*\tic) node[right]{$z_0=(W_5)^0=1$};
\draw[] (0.30902+1*\tic, 0.95106+3*\tic) node[right]{$z_1=(W_5)^1=\e^{\im\frac{2\pi}{5}}$};
\draw[] (-0.80902-4*\tic, 0.58779+4*\tic) node{$z_2=(W_5)^2$};
\draw[] (-0.80902-4*\tic,- 0.58779-4*\tic) node{$z_3$};
\draw[] (0.30902+4*\tic,- 0.95106-4*\tic) node{$z_4$};
%
\draw[-] (0,0) -- (1,0);
\draw[-] (0,0) -- (0.30902, 0.95106);
\draw[-] (0,0) -- (-0.80902, 0.58779);
\draw[-] (0,0) -- (-0.80902,- 0.58779);
\draw[-] (0,0) -- (0.30902,- 0.95106);
\draw[->] (0.25,0) node[above]{$\frac{2\pi}{5}$} arc (0:72:0.25) ;
%
\draw[] (0,2.25) node{Lösungen für $z^5 = 1$ für $N=5$ DFT};
\end{scope}
%
\end{tikzpicture}
\caption{DFT-Frequenzen auf dem Einheitskreis.}
\label{fig:DFT_UnitCircle}
\end{figure}


Wir haben es bei der DFT also wieder einmal mit komplexen Zahlen auf dem Einheitskreis zu
tun, was in SigSys oft vorkommt.
%
Die IDFT Formel verlangt den komplexen Zeiger $\e^{+\im\frac{2\pi}{N} (k\cdot \mu)}$.
Wir können die Matrix $\bm{K}$ um diesen Zeiger erweitern zu der sogenannten
\textbf{Fourier Matrix}
\begin{align}
\label{eq:FourierMatrix}
\bm{F} =
k \downarrow
\substack{\rightarrow \mu\\
\begin{bmatrix}
\e^{+\im\frac{2\pi}{N} (0 \cdot 0)} & \e^{+\im\frac{2\pi}{N} (0 \cdot 1)} & \e^{+\im\frac{2\pi}{N} (0 \cdot 2)} & \e^{+\im\frac{2\pi}{N} (0 \cdot 3)} & \dots & \e^{+\im\frac{2\pi}{N} (0 \cdot (N-1))}\\
\e^{+\im\frac{2\pi}{N} (1 \cdot 0)} & \e^{+\im\frac{2\pi}{N} (1 \cdot 1)} & \e^{+\im\frac{2\pi}{N} (1 \cdot 2)} & \e^{+\im\frac{2\pi}{N} (1 \cdot 3)} & \dots & \e^{+\im\frac{2\pi}{N} (1 \cdot (N-1))} \\
\e^{+\im\frac{2\pi}{N} (2 \cdot 0)} & \e^{+\im\frac{2\pi}{N} (2 \cdot 1)} & \e^{+\im\frac{2\pi}{N} (2 \cdot 2)} & \e^{+\im\frac{2\pi}{N} (2 \cdot 3)} & \dots & \e^{+\im\frac{2\pi}{N} (2 \cdot (N-1))}\\
\e^{+\im\frac{2\pi}{N} (3 \cdot 0)} & \e^{+\im\frac{2\pi}{N} (3 \cdot 1)} & \e^{+\im\frac{2\pi}{N} (3 \cdot 2)} & \e^{+\im\frac{2\pi}{N} (3 \cdot 3)} & \dots & \e^{+\im\frac{2\pi}{N} (3 \cdot (N-1))}\\
\vdots & \vdots & \vdots & \vdots & \ddots & \vdots\\
\e^{+\im\frac{2\pi}{N} ((N-1) \cdot 0)} & \e^{+\im\frac{2\pi}{N} ((N-1) \cdot 1)} & \e^{+\im\frac{2\pi}{N} ((N-1) \cdot 2)} & \e^{+\im\frac{2\pi}{N} ((N-1) \cdot 3)} & \dots &  \e^{+\im\frac{2\pi}{N} ((N-1) \cdot (N-1))}
\end{bmatrix}
}
\end{align}
In Büchern findet sich oft die kürzere Variante mit $(W_N)^{k \mu} = W_N^{k \mu}$
\begin{align}
\bm{F} =
k \downarrow
\substack{\rightarrow \mu\\
\begin{bmatrix}
1 & 1 & 1 & 1 & \dots & 1\\[1em]
1 & W_N^1 & W_N^2 & W_N^3 & \dots & W_N^{(N-1)}\\[1em]
1 & W_N^2 & W_N^4 & W_N^6 & \dots & W_N^{2(N-1)}\\[1em]
1 & W_N^3 & W_N^6 & W_N^9 & \dots & W_N^{3(N-1)}\\[1em]
\vdots & \vdots & \vdots &\vdots &\ddots & \vdots\\[1em]
1 & W_N^{(N-1)} & W_N^{2(N-1)} & W_N^{3(N-1)} & \dots & W_N^{(N-1)(N-1)}
\end{bmatrix}
}
\end{align}
Wenn wir das Zeitsignal $x[k]$ der Länge $N$ (also eigentlich periodisch in $N$,
wir nehmen nach Konvention die Periode $0 \leq k \leq N-1$)
und das dazugehörige DFT-Spektrum $X[\mu]$ (auch $N$ periodisch,
Konvention $0\leq \mu \leq N-1$)
als Spalten-Vektoren mit $N$ Einträgen definieren
\begin{align}
\text{Zeitfolge: }
\bm{x}_k =
\begin{bmatrix}
x[k=0]\\
x[k=1]\\
x[k=2]\\
x[k=3]\\
\vdots\\
x[k=N-1]\\
\end{bmatrix}\qquad
\text{DFT-Spektrum: }
\bm{x}_\mu =
\begin{bmatrix}
X[\mu=0]\\
X[\mu=1]\\
X[\mu=2]\\
X[\mu=3]\\
\vdots\\
X[\mu=N-1]\\
\end{bmatrix}
\end{align}
können wir die \textbf{IDFT als Matrixmultiplikation}
\begin{align}
\bm{x}_k = \frac{1}{N} \bm{F} \cdot \bm{x}_\mu
\end{align}
schreiben.
Die zugrunde liegende Linearkombination der Spaltenvektoren von $\bm{F}$
offenbart sehr übersichtlich was die IDFT bei der Synthese einer Zeitfolge macht
\label{LinComb_for_IDFT}
%
\begin{align*}
\begin{bmatrix}
x[0]\\[1em]
x[1]\\[1em]
x[2]\\[1em]
x[3]\\[1em]
\dots\\[1em]
x[N-1]
\end{bmatrix}
=
\frac{X[0]}{N}
\begin{bmatrix}
1\\[1em]
1\\[1em]
1\\[1em]
1\\[1em]
\dots\\[1em]
1
\end{bmatrix}
+\frac{X[1]}{N}
\begin{bmatrix}
1\\[1em]
W_N\\[1em]
W_N^2\\[1em]
W_N^3\\[1em]
\dots\\[1em]
W_N^{(N-1)}
\end{bmatrix}
+\frac{X[2]}{N}
\begin{bmatrix}
1\\[1em]
W_N^2\\[1em]
W_N^4\\[1em]
W_N^6\\[1em]
\dots\\[1em]
W_N^{2(N-1)}
\end{bmatrix}
+\frac{X[3]}{N}
\begin{bmatrix}
1\\[1em]
W_N^3\\[1em]
W_N^6\\[1em]
W_N^9\\[1em]
\dots\\[1em]
W_N^{3(N-1)}
\end{bmatrix}
+
\dots
+\frac{X[N-1]}{N}
\begin{bmatrix}
1\\[1em]
W_N^{1(N-1)}\\[1em]
W_N^{2(N-1)}\\[1em]
W_N^{3(N-1)}\\[1em]
\dots\\[1em]
W_N^{(N-1)(N-1)}
\end{bmatrix}
\end{align*}
Die Spaltenvektoren von $\bm{F}$ stellen die komplexen Zeit-Signale
$\e^{+\im\frac{2\pi}{N}k \cdot 0}$ (nur Einsen),
$\e^{+\im\frac{2\pi}{N}k \cdot 1}$ bis $\e^{+\im\frac{2\pi}{N}k \cdot (N-1)}$
dar, diese sind alle $N$ periodisch, vgl.~\fig{fig:fourier_matrix}
%
\begin{figure}
\centering
\includegraphics[width=0.9\textwidth]{../dft/fourier_matrix}
\caption{$N=8$ Fourier Matrix $\bm{F}$ \eqref{eq:FourierMatrix} so dargestellt, dass
die 8 komplexwertigen DFT Eigensignale getrennt nach Spalten dargestellt sind.
Nach unten läuft die Zeit, nach rechts erhöht sich die DFT-Frequenz.
Spalte 1 ($\mu=0$) beinhaltet als Zeitsignal den Gleichanteil mit Amplitude 1,
Spalte 2 ($\mu=1$) exakt eine Periode einer cos (Realteil, blau) und sin (Imaginärteil, orange)
Schwingung. Spalte 5 ($\mu=4$) enthält 4 Perioden einer Cos-Schwinung.}
\label{fig:fourier_matrix}
\end{figure}


Diese Signale werden mit den zugehörigen Spektralwerten
$X[0]$ (das ist der Gleichanteil), $X[1]$ bis $X[N-1]$ gewichtet.
Für unser Transformationspaar erfolgt bei der IDFT noch die Normierung mit
$\frac{1}{N}$. Alle gewichteten Signale werden dann aufaddiert zum Ergebnissignal
$x[k]$.

Es ist also sehr ähnlich zur Fourierreihensynthese: Überlagerung von
Grundschwingung und Oberschwingungen, nur, dass hier die DFT ein $N$-periodisches
Spektrum an Fourierkoeffizienten aufweist, was zeitdiskrete \textbf{und}
periodische Signale bedingt.
%

Für die Spaltenvektoren $\bm{f}_i$ und $\bm{f}_j$ aus der Matrix $\bm{F}$
folgt die Orthogonalitätsbedingung aus dem Skalarprodukt für komplexe Vektoren
\begin{align}
\bm{f}_i^\mathrm{H} \bm{f}_j =
\begin{cases}
N & i=j\\
0 & i \neq j
\end{cases}
\end{align}
mit dem adjungierten Operator $()^\mathrm{H}$, also konjugiert-komplex und transponiert.
Auch dies kennen wir schon von der Fourierreihe, dort für die kontinuierlichen
Funktionen
$\sin(\mu \omega_0 t)$, $\cos(\mu \omega_0 t)$ und $\e^{\im \mu \omega_0 t}$ die
orthogonal zueinander sind für verschiedene $\mu\in\mathbb{Z}$.
%
Es ist genau die Orthogonalität der Fourier-Matrix $\bm{F}$ die uns die DFT / IDFT
schenkt als Linearkombination $N$ orthogonaler (komplexer) Vektoren.
%

Mit konjugiert-komplexer Transponierung von Matrix $\bm{F}$, also mit $(\bm{F}^*)^\mathrm{T}$, bzw. kürzer mit $\bm{F}^\mathrm{H}$
\begin{align}
\label{eq:FourierMatrixH}
\bm{F}^\mathrm{H} =
\mu \downarrow
\substack{\rightarrow k\\
\begin{bmatrix}
1 & 1 & 1 & 1 & \dots & 1\\[1em]
1 & W_N^{-1} & W_N^{-2} & W_N^{-3} & \dots & W_N^{-(N-1)}\\[1em]
1 & W_N^{-2} & W_N^{-4} & W_N^{-6} & \dots & W_N^{-2(N-1)}\\[1em]
1 & W_N^{-3} & W_N^{-6} & W_N^{-9} & \dots & W_N^{-3(N-1)}\\[1em]
\vdots & \vdots & \vdots &\vdots &\ddots & \vdots\\[1em]
1 & W_N^{-(N-1)} & W_N^{-2(N-1)} & W_N^{-3(N-1)} & \dots & W_N^{-(N-1)(N-1)}
\end{bmatrix}
}
\end{align}
folgt die \textbf{DFT als Matrixmultiplikation}
\begin{align}
\label{eq:ex11_dft_hin}
\bm{x}_\mu = \bm{F}^\mathrm{H} \cdot \bm{x}_k.
\end{align}
Wenn wir also eine Zeile aus der Matrix $\bm{F}^\mathrm{H}$ mit dem Spaltenvektor $\bm{x}_k$
als normales inneres Produkt / Skalarprodukt
verrechnen, ist der konjugiert-komplexe Teil des komplexen Skalarprodukts schon
berücksichtigt.

%
\noindent Hinweis \textbf{Unitäre Fourier Matrix}
Wir könnten das IDFT/DFT-Paar auch mit $\frac{1}{\sqrt{N}}$-Normierung
aufschreiben (das wird sehr gerne in Physik und Mathe benutzt)
\begin{align}
x[k] = \frac{1}{\sqrt{N}} \sum_{\mu=0}^{N-1} X[\mu] \, \e^{+\im\frac{2\pi}{N} k \cdot \mu}
\qquad
X[\mu] = \frac{1}{\sqrt{N} }\sum_{k=0}^{N-1} x[k] \, \e^{-\im\frac{2\pi}{N} k \cdot \mu}
\end{align}
Dann benutzen wir die normierte Fourier-Matrix
\begin{align}
\mathring{\bm{F}} = \frac{\bm{F}}{\sqrt{N}}
\end{align}
für die IDFT / DFT-Matrixoperationen
\begin{align}
\bm{x}_k = \mathring{\bm{F}} \, \bm{x}_\mu
\qquad
\bm{x}_\mu = (\mathring{\bm{F}})^* \, \bm{x}_k,
\end{align}
also bei der DFT die konjugiert-komplexe von $\mathring{\mathbf{F}}$.
%
Dies funktioniert deswegen, weil $\mathring{\mathbf{F}}$ eine unitäre Matrix ist,
also eine orthonormale, komplexe und zudem symmetrische Matrix, für die
\begin{align}
\mathring{\bm{F}} = \mathring{\bm{F}}^\mathrm{T}
\qquad
(\mathring{\bm{F}})^* =
\mathring{\bm{F}}^\mathrm{H} =
\mathring{\bm{F}}^{-1}
\qquad\text{und}\qquad
\mathring{\bm{F}}^{-1} \, \mathring{\bm{F}} =
\mathring{\bm{F}} \, \mathring{\bm{F}}^{-1} =
\bm{I}.
\end{align}
gilt.
%
Damit sehen wir sehr schön, dass die DFT/IDFT eindeutig lösbare
Gleichungssysteme repräsentieren, wobei die Vektoren (bei uns in SigSys also Zeitsignal bzw.
DFT-Spektrum) in eine (spezielle und fundamentale wichtige)
orthonormale Vektorbasis projiziert werden, die wir lernen sollten, sinnvoll
für SigSys Probleme zu interpretieren.


\clearpage
\subsection{Inverse DFT}
\label{sec:D394560597}
\begin{Ziel}
Wir wollen uns als Einstieg in die DFT / IDFT Welt mit der Signalsynthese, also
in SigSys-Anschauung mit der inversen DFT beschäftigen. Dazu diskutieren wir zwei ganz einfache
Fälle, in denen das $N=8$ DFT-Spektrum nur einen bzw. zwei Einträge hat, die nicht Null sind.
Wenn wir die Signalsynthese anhand der zugrundeliegenden Linearkombination
verstanden haben, können wir die Spektren beliebig verkomplizieren und damit
$N$ periodische Signalfolgen synthetisieren.
Die Vorgehensweise ist sehr ähnlich zur Signalsynthese mit der komplexen
Fourierreihe.
\end{Ziel}
\textbf{Aufgabe} {\tiny D394560597}: Gegeben sind die $N=8$ DFT Spektren
\begin{align}
&X_c[\mu] = 8\e^{-\im\frac{\pi}{4}} \sum_{m=-\infty}^{+\infty} \delta[\mu-1 + 8 \cdot m]\\
&X_r[\mu] =
4\e^{-\im\frac{\pi}{4}} \sum_{m=-\infty}^{+\infty} \delta[\mu-1 + 8 \cdot m] +
4\e^{+\im\frac{\pi}{4}} \sum_{m=-\infty}^{+\infty} \delta[\mu-7 + 8 \cdot m]
\end{align}
Berechnen Sie davon die inversen DFTs, also $x_c[k] = \text{IDFT}_8\{X_c[\mu]\}$
und $x_r[k] = \text{IDFT}_8\{X_r[\mu]\}$.

\begin{Werkzeug}
IDFT als Linearkombination $\bm{x}_k = \frac{1}{N} \bm{F} \cdot \bm{x}_\mu$
oder wenn wir es zunächst lieber als Summe mögen
\begin{align}
x[k] = \mathrm{IDFT}_N\{X[\mu]\} = \frac{1}{N} \sum_{\mu=0}^{N-1} X[\mu] \, \e^{+\im\frac{2\pi}{N} k \cdot \mu}
\end{align}
\end{Werkzeug}
\begin{Ansatz}
Es ist sinnvoll die Lage der DFT Frequenzen auf dem Einheitskreis zu
visualisieren, damit wir die Winkel der komplexen Dreher schnell überschauen.
\begin{center}
\begin{tikzpicture}[scale=1.5]
\def \tic {0.05}
\begin{scope}
\draw[C3, thick] (0,0) circle(1);  % unit circle, i.e. DTFT domain
\draw[C3] (0.78,-0.78) node[right]{DTFT $z=\e^{\im\Omega}$};
\draw[C0] (0.78,-1.15) node[right]{DFT $z_\mu = \e^{\im\frac{2\pi}{8}\mu}$};
%
\draw (1+2*\tic,-3*\tic) node{$1$}; % indicate that this is the unit circle
\draw[->] (-1.25,0)--(1.75,0) node[right]{$\Re\{z\}$}; % axis label
\draw[->] (0,-1.25)--(0,1.5) node[above]{$\Im\{z\}$}; % axis label
%
\draw[white] (0,-1)--(0,+1);
\draw[white] (-1,0)--(+1,0);
%
\draw[C0, ultra thick] (1, 0) node{\Huge $\circ$};
\draw[C0, ultra thick] (0.7071, 0.7071) node{\Huge $\circ$};
\draw[C0, ultra thick] (0, 1) node{\Huge $\circ$};
\draw[C0, ultra thick] (-0.7071, 0.7071) node{\Huge $\circ$};
\draw[C0, ultra thick] (-1, 0) node{\Huge $\circ$};
\draw[C0, ultra thick] (-0.7071, -0.7071) node{\Huge $\circ$};
\draw[C0, ultra thick] (0, -1) node{\Huge $\circ$};
\draw[C0, ultra thick] (0.7071, -0.7071) node{\Huge $\circ$};
%
\draw[] (0.7071+1*\tic, 0.7071+3*\tic) node[right]{$z_1=W_8=\e^{\im\frac{2\pi}{8}}$};
%
\draw[] (0,2.25) node{Lösungen für $z^8 = 1$ für $N=8$ DFT};
\end{scope}
%
\end{tikzpicture}
\end{center}
Zudem sollten wir die Fourier Matrix $\bm{F}_8$ einmal zu Papier bringen,
das ist ein wenig Schreibarbeit, aber erkenntnisstiftend.
Zunächst, so wie in der Einleitung dargestellt, die Matrix füllen
\begin{align}
\bm{F}_8 =
\begin{bmatrix}
1 & 1 & 1 & 1 & 1 & 1 & 1 & 1\\
1 & \e^{\im\frac{1}{4}\pi} & \e^{\im\frac{2}{4}\pi} & \e^{\im\frac{3}{4}\pi} & \e^{\im\frac{4}{4}\pi} & \e^{\im\frac{5}{4}\pi} & \e^{\im\frac{6}{4}\pi} & \e^{\im\frac{7}{4}\pi}\\
1 & \e^{\im\frac{2}{4}\pi} & \e^{\im\frac{4}{4}\pi} & \e^{\im\frac{6}{4}\pi} & \e^{\im\frac{8}{4}\pi} & \e^{\im\frac{10}{4}\pi} & \e^{\im\frac{12}{4}\pi} & \e^{\im\frac{14}{4}\pi}\\
1 & \e^{\im\frac{3}{4}\pi} & \e^{\im\frac{6}{4}\pi} & \e^{\im\frac{9}{4}\pi} & \e^{\im\frac{12}{4}\pi} & \e^{\im\frac{15}{4}\pi} & \e^{\im\frac{18}{4}\pi} & \e^{\im\frac{21}{4}\pi}\\
1 & \e^{\im\frac{4}{4}\pi} & \e^{\im\frac{8}{4}\pi} & \e^{\im\frac{12}{4}\pi} & \e^{\im\frac{16}{4}\pi} & \e^{\im\frac{20}{4}\pi} & \e^{\im\frac{24}{4}\pi} & \e^{\im\frac{28}{4}\pi}\\
1 & \e^{\im\frac{5}{4}\pi} & \e^{\im\frac{10}{4}\pi} & \e^{\im\frac{15}{4}\pi} & \e^{\im\frac{20}{4}\pi} & \e^{\im\frac{25}{4}\pi} & \e^{\im\frac{30}{4}\pi} & \e^{\im\frac{35}{4}\pi}\\
1 & \e^{\im\frac{6}{4}\pi} & \e^{\im\frac{12}{4}\pi} & \e^{\im\frac{18}{4}\pi} & \e^{\im\frac{24}{4}\pi} & \e^{\im\frac{30}{4}\pi} & \e^{\im\frac{36}{4}\pi} & \e^{\im\frac{42}{4}\pi}\\
1 & \e^{\im\frac{7}{4}\pi} & \e^{\im\frac{14}{4}\pi} & \e^{\im\frac{21}{4}\pi} & \e^{\im\frac{28}{4}\pi} & \e^{\im\frac{35}{4}\pi} & \e^{\im\frac{42}{4}\pi} & \e^{\im\frac{49}{4}\pi}\\
\end{bmatrix}
\end{align}
und dann die komplexen Dreher vereinfachen
\begin{align}
\bm{F}_8 =
k \downarrow
\substack{\rightarrow \mu\\
\begin{bmatrix}
1 & 1 & 1 & 1 & 1 & 1 & 1 & 1\\
1 & \e^{\im\frac{1}{4}\pi} & +\im & \e^{\im\frac{3}{4}\pi} & -1 & \e^{\im\frac{5}{4}\pi} & -\im & \e^{\im\frac{7}{4}\pi}\\
1 & +\im & -1 & -\im & +1 & +\im & -1 & -\im\\
1 & \e^{\im\frac{3}{4}\pi} & -\im & \e^{\im\frac{1}{4}\pi} & -1 & \e^{\im\frac{7}{4}\pi} & +\im & \e^{\im\frac{5}{4}\pi}\\
1 & -1 & +1 & -1 & +1 & -1 & +1 & -1\\
1 & \e^{\im\frac{5}{4}\pi} & +\im & \e^{\im\frac{7}{4}\pi} & -1 & \e^{\im\frac{1}{4}\pi} & -\im & \e^{\im\frac{3}{4}\pi}\\
1 & -\im & -1 & +\im & +1 & -\im & -1 & +\im\\
1 & \e^{\im\frac{7}{4}\pi} & -\im & \e^{\im\frac{5}{4}\pi} & -1 & \e^{\im\frac{3}{4}\pi} & +\im & \e^{\im\frac{1}{4}\pi}\\
\end{bmatrix}
}
\end{align}
%
Ein wenig anschaulicher als der reine Formelalgorithmus, könnte die folgende
Überlegung sein:
%
Die Werte in den Spaltenvektoren variieren ja über $k$,
das sind also genau die 8 Samples der in $N=8$ periodischen Signale.
Die DFT stellt 8 Frequenzen dar, diese verschiedenen Frequenzen sind in den
Signalen der 8 Spaltenvektoren codiert.
Spalte eins codiert den Gleichanteil, es sind nur Einsen.
%
Spalte zwei, also für $\mu=1$ codiert das Signal $e^{\im\frac{2\pi}{8} k}$.
Im Einheitskreis können wir uns das Zeitsignal anschaulich zusammenbauen:
Mit Inkrement $\mu=1$ laufen wir beginnend bei
$\e^{\im 0}$ und mathematisch positiv alle DFT Frequenzpunkte
ab, diese Werte sind genau die Werte des Zeitsignals.

Spalte drei, also für $\mu=2$ codiert das Signal $e^{\im\frac{2\pi}{8} 2 k}$.
Auf dem Einheitskreis bedeutet dies die Berücksichtigung nur jedes $\mu=2$-ten
DFT Frequenzwerts, wie laufen also zweimal um den Kreis um alle 8 Samples zu
bekommen, weil wir acht (genau genommen sieben, acht sind es um den vollen
Kreis zu schließen) 90 Grad-Sprünge machen.
Für Spalte 8 müssen wir demzufolge jeden $\mu=7$-ten Wert benutzen, es sind
also ingesamt 7 Umdrehungen. Wir könnten auch negativ umlaufen, dann mit
Sprüngen um $8-7=1$ Frequenz,
das ginge ein wenig schneller und stellt einen schönen Zusammenhang zu den
negativen Frequenzen her, die wir in der komplexen Rechnerei ja immer nutzen,
damit es ein wenig einfacher wird.

$\mu=8$ ist nun identisch mit $\mu=0$: ob wir gar nicht springen oder um 8
Punkte resultiert ja im selben statischen DFT-Punkt $\e^{\im 0}$. Daran sehen
wir sehr schön die $N$ Periodizität. Es gibt für diese $N$ Punkte auf dem
Einheitskreis genau 8 verschiedene Möglichkeiten eine komplexe Zeitfolge
zusammenzubauen, diese sind orthogonal zueinander.
Das ist der Sinn dieser (wunderschönen) Vektorbasis.
%
\end{Ansatz}
\begin{ExCalc}
%
Wir müssen zunächst die etwas kryptische Schreibweise der gegebenen DFT
Spektren entziffern.
Wir haben eine Summe von Diracs, die jeweils um $8 m$ verschoben sind. Dies
sorgt dafür, dass der analytische Ausdruck periodisch in 8 ist. Wir hätten auch
(das findet sich oft in Büchern) nur das sogenannte Basisband $0\leq \mu \leq N-1$ angeben
können und müssten dann erwähnen, dass wir es $N$ periodisch auffassen.
%
Damit wir den $N$ \textbf{periodischen Charakter} nie vergessen, hier mal mit Dirac
Impuls Kamm Schreibweise. Das ist auch noch mal eine gute Übung zur
Austasteigenschaft des Diracs.

Alle Diracs werden gleich gewichtet mit einem komplexen Faktor
\begin{align}
8 \e^{-\im\frac{\pi}{4}}  = 8 \cos(\frac{2\pi}{8}) - \im 8 \sin(\frac{2\pi}{8})=
\frac{8}{\sqrt{2}} - \im \frac{8}{\sqrt{2}}
\end{align}
im Spektrum $X_c[\mu]$.
%
Für das Spektrum $X_r[\mu]$ erfolgt die Gewichtung
\begin{align}
4 \e^{\pm\im\frac{\pi}{4}}  = 4 \cos(\frac{2\pi}{8}) \pm \im 4 \sin(\frac{2\pi}{8})=
\frac{4}{\sqrt{2}} \pm \im \frac{4}{\sqrt{2}}
\end{align}

Damit können wir uns nun die Spektren, also genauer Realteil und Imaginärteil
(wahlweise ginge auch Betrag und Phase), visualisieren. Das ist unten dargestellt,
in orange das Basisband $0\leq \mu \leq N-1$, in \textbf{blau} die $N=8$
\textbf{Periodizität} angedeutet.

\begin{center}
%
\begin{tikzpicture}[scale=0.25]
\def\tic{0.1};
\begin{scope}
\draw[->] (-8,0) -- (16,0) node[right]{$\mu$};
\draw[->] (0,-.25) -- (0,6) node[above]{$\Re\{X_c[\mu]\}$};
\draw[stem, C0] plot coordinates{(-8,0) (-7,5.6569) (-6,0) (-5,0) (-4,0) (-3,0) (-2,0) (-1,0)};
\draw[stem, C1] plot coordinates{(0,0) (1,5.6569) (2,0) (3,0) (4,0) (5,0) (6,0) (7,0)};
\draw[stem, C0] plot coordinates{(8,0) (9,5.6569) (10,0) (11,0) (12,0) (13,0) (14,0) (15,0)};
\draw (-7,-\tic) node[below]{$-7$};
\draw (0,-\tic) node[below]{$0$};
\draw (1,-\tic) node[below]{$1$};
\draw (7,-\tic) node[below]{$7$};
\draw (9,-\tic) node[below]{$9$};
%
\draw (16,2) node[below]{$\dots$};
\draw (-9,2) node[below]{$\dots$};
%
\draw (9,5.5) node[above]{$8\cos(\frac{2\pi}{8})$};
\end{scope}
\begin{scope}[yshift=-7cm]
\draw[->] (-8,0) -- (16,0) node[right]{$\mu$};
\draw[->] (0,-6) -- (0,3) node[above]{$\Im\{X_c[\mu]\}$};
\draw[stem, C0] plot coordinates{(-8,0) (-7,-5.6569) (-6,0) (-5,0) (-4,0) (-3,0) (-2,0) (-1,0)};
\draw[stem, C1] plot coordinates{(0,0) (1,-5.6569) (2,0) (3,0) (4,0) (5,0) (6,0) (7,0)};
\draw[stem, C0] plot coordinates{(8,0) (9,-5.6569) (10,0) (11,0) (12,0) (13,0) (14,0) (15,0)};
\draw (-7,\tic) node[above]{$-7$};
\draw (1,\tic) node[above]{$1$};
\draw (7,\tic) node[above]{$7$};
\draw (9,\tic) node[above]{$9$};
%
\draw (16,2) node[below]{$\dots$};
\draw (-9,2) node[below]{$\dots$};
\end{scope}
\end{tikzpicture}
%
\end{center}
%
%
%
%
%
\begin{center}
%
\begin{tikzpicture}[scale=0.25]
\def\tic{0.1};
\begin{scope}
\draw[->] (-8,0) -- (16,0) node[right]{$\mu$};
\draw[->] (0,-.25) -- (0,6) node[above]{$\Re\{X_r[\mu]\}$};
\draw[stem, C0] plot coordinates{(-8,0) (-7,2.8284) (-6,0) (-5,0) (-4,0) (-3,0) (-2,0) (-1,2.8284)};
\draw[stem, C1] plot coordinates{(0,0) (1,2.8284) (2,0) (3,0) (4,0) (5,0) (6,0) (7,2.8284)};
\draw[stem, C0] plot coordinates{(8,0) (9,2.8284) (10,0) (11,0) (12,0) (13,0) (14,0) (15,2.8284)};
\draw (-7,-\tic) node[below]{$-7$};
\draw (0,-\tic) node[below]{$0$};
\draw (1,-\tic) node[below]{$1$};
\draw (7,-\tic) node[below]{$7$};
\draw (9,-\tic) node[below]{$9$};
%
\draw (16,2) node[below]{$\dots$};
\draw (-9,2) node[below]{$\dots$};
%
\draw (9,2.8284) node[above]{$4\cos(\frac{2\pi}{8})$};
%
\draw[C7, thin] (4,-3) -- (4,3); % axial sym
%
\end{scope}
\begin{scope}[yshift=-7cm]
\draw[->] (-8,0) -- (16,0) node[right]{$\mu$};
\draw[->] (0,-6) -- (0,3) node[above]{$\Im\{X_r[\mu]\}$};
\draw[stem, C0] plot coordinates{(-8,0) (-7,-2.8284) (-6,0) (-5,0) (-4,0) (-3,0) (-2,0) (-1,2.8284)};
\draw[stem, C1] plot coordinates{(0,0) (1,-2.8284) (2,0) (3,0) (4,0) (5,0) (6,0) (7,2.8284)};
\draw[stem, C0] plot coordinates{(8,0) (9,-2.8284) (10,0) (11,0) (12,0) (13,0) (14,0) (15,2.8284)};
\draw (-7,\tic) node[above]{$-7$};
\draw (1,\tic) node[above]{$1$};
\draw (7,-\tic) node[below]{$7$};
\draw (9,\tic) node[above]{$9$};
%
\draw (16,2) node[below]{$\dots$};
\draw (-9,2) node[below]{$\dots$};
%
\draw[stem2, C7] plot coordinates{(4,0)};
%
\end{scope}
\end{tikzpicture}
%
\end{center}
%
Wir müssen nun die Linearkombination
\begin{align}
\bm{x}_k  = \frac{1}{8} \bm{F}_8 \cdot \bm{x}_\mu
\end{align}
berechnen, dies ist die IDFT in Matrixschreibweise. Es resultiert die gesuchte
Zeitfolge.
%
Einen Computer würden mit Zahlen versorgen, z.B. mit Matlab

\texttt{ifft([0 8*exp(-1j*pi/4) 0 0 0 0 0 0].')}

(Achtung in Matlab: \texttt{[]'} ist adjungiert, \texttt{[].'} ist transponiert. Hier soll
nur transponiert, also ein Spaltenvektor erzeugt werden, damit wir ein Spaltenvektor
als Ergebnis bekommen.)

Die auf Seite \pageref{LinComb_for_IDFT} aufgezeigte Linearkombination reduziert
sich für das Spektrum $X_c[\mu]$ auf die Benutzung des zweiten Spaltenvektors (die
anderen Gewichte/DFT-Koeffizienten sind ja Null)
\begin{align*}
\bm{x}_{k,c}=
\begin{bmatrix}
x[0]\\[1em]
x[1]\\[1em]
x[2]\\[1em]
x[3]\\[1em]
\dots\\[1em]
x[N-1]
\end{bmatrix}
=
\frac{X_c[\mu=1]}{N}
\begin{bmatrix}
1\\[1em]
W_N\\[1em]
W_N^2\\[1em]
W_N^3\\[1em]
\dots\\[1em]
W_N^{(N-1)}
\end{bmatrix}
\end{align*}
Mit den konkreten Zahlen
\begin{align}
\bm{x}_{k,c} = \frac{1}{8} \cdot 8 \e^{-\im\frac{\pi}{4}}
\begin{bmatrix}
1 \\  \e^{\im\frac{1}{4}\pi} \\ \im \\ \e^{\im\frac{3}{4}\pi}  \\ -1 \\ \e^{\im\frac{5}{4}\pi} \\ -\im \\ \e^{\im\frac{7}{4}\pi}
\end{bmatrix}=
\begin{bmatrix}
\e^{-\im\frac{\pi}{4}}\\
1\\
\e^{+\im\frac{\pi}{4}}\\
\im\\
\e^{\im\frac{3\pi}{4}}\\
-1\\
\e^{-\im\frac{3\pi}{4}}\\
-\im
\end{bmatrix}
\end{align}
Passend dazu der numerische Matlab Output:
\begin{verbatim}
 0.7071 - 0.7071i
 1.0000 + 0.0000i
 0.7071 + 0.7071i
-0.0000 + 1.0000i
-0.7071 + 0.7071i
-1.0000 - 0.0000i
-0.7071 - 0.7071i
 0.0000 - 1.0000i
\end{verbatim}
%
Nachdem wir wissen, wie der zweite Spaltenvektor von $\bm{F}$ zusammengebaut
ist, weil wir die Matrix ja mit Vorwissen intentional erzeugt haben, können wir
auch die analytische Formel für $x_c[k]$ angeben
\begin{align}
x_c[k] = \frac{1}{8} \cdot 8 \e^{-\im\frac{\pi}{4}} \e^{+\im\frac{2\pi}{8} k}=
\e^{\im \frac{2\pi}{8} [k -1] } =
\cos(\frac{2\pi}{8} [k-1]) + \im \sin(\frac{2\pi}{8} [k-1]).
\end{align}
Es ist das um ein Sample verzögerte DFT-'Eigensignal' der Frequenz $\mu=1$, also
die DFT Grundfrequenz, vgl.~\fig{fig:fourier_matrix}.
Amplitude und Phase des DFT Koeffizienten waren
natürlich genauso ausgewählt, dass dieses anschauliche Ergebnis mit
Signalamplitude 1 und Verzögerung um ein Sample rauskommt. Wir könnten
herausfinden, dass die Wahl von $X_c[\mu] = 8\e^{\im\pi}$ das jeweils $\mu$-te DFT Eigensignal
invertiert.

Für das Spektrum $X_r[\mu]$ beinhaltet die Linearkombination $\bm{x}_{k,r}  =
\frac{1}{8} \bm{F}_8 \cdot \bm{x}_{\mu,r}$ zwei von Null unterschiedliche
Einträge, die der zweiten ($\mu=1$) und die der achten ($\mu=7$)
Spalte der Matrix $\bm{F}$
\begin{align}
\bm{x}_{k,r} = \frac{1}{8} \cdot 4 \e^{-\im\frac{\pi}{4}}
\begin{bmatrix}
1 \\  \e^{\im\frac{1}{4}\pi} \\ \im \\ \e^{\im\frac{3}{4}\pi}  \\ -1 \\ \e^{\im\frac{5}{4}\pi} \\ -\im \\ \e^{\im\frac{7}{4}\pi}
\end{bmatrix}
+
\frac{1}{8} \cdot 4 \e^{+\im\frac{\pi}{4}}
\begin{bmatrix}
1 \\ e^{\im\frac{7}{4}\pi} \\ -\im \\ e^{\im\frac{5}{4}\pi} \\ -1 \\ e^{\im\frac{3}{4}\pi} \\ \im \\ e^{\im\frac{1}{4}\pi}
\end{bmatrix}
=
\begin{bmatrix}
\nicefrac{1}{\sqrt{2}}\\
1\\
\nicefrac{1}{\sqrt{2}}\\
0\\
-\nicefrac{1}{\sqrt{2}}\\
-1\\
-\nicefrac{1}{\sqrt{2}}\\
0
\end{bmatrix}
\end{align}
Unser Ergebnis suggeriert, dass die resultierende Zeitfolge rein reellwertig ist.
Das ist in der Tat so, vgl. Symmetrieeigenschaften der FT!
Wir prüfen es kurz analytisch und fragen uns dann, warum
das so sein muss.

Wir kennen die analytischen Ausdrücke der DFT 'Eigensignale' aus der zweiten
und achten Spalte, vgl.~\fig{fig:fourier_matrix}.
Diese mit dem jeweiligen Vorfaktor und $\frac{1}{N}$ für IDFT gewichtet ergibt
\begin{align}
x_r[k]
=
\frac{1}{8} \cdot 4 \e^{-\im\frac{\pi}{4}} \cdot  \e^{\im \frac{2\pi}{8}\cdot 1 k}+
\frac{1}{8} \cdot 4 \e^{+\im\frac{\pi}{4}} \cdot  \e^{\im \frac{2\pi}{8}\cdot 7 k}
\end{align}
%
und weiter ausformuliert und damit vereinfacht (i.e. SigSys must have skills)
\begin{align}
x_r[k]
=
&\frac{1}{2} \e^{\im \frac{2\pi}{8}\cdot 1 k - \im\frac{2\pi}{8}}+
\frac{1}{2} \e^{\im \frac{2\pi}{8}\cdot 7 k} \e^{\im\frac{2\pi}{8}}\\
=
&\frac{1}{2} \e^{\im \frac{2\pi}{8} (k-1)}+
\frac{1}{2} \underbrace{\e^{\im \frac{2\pi}{8}\cdot 8 k}}_{=1} \e^{- \im \frac{2\pi}{8}\cdot k} \e^{\im\frac{2\pi}{8}}\\
=
&\frac{1}{2} \e^{+\im \frac{2\pi}{8} (k-1)}+
\frac{1}{2} \e^{-\im \frac{2\pi}{8} (k-1)}\\
=
&\cos(\frac{2\pi}{8} [k-1]).
\end{align}
Wir bekommen mit dem cos() tatsächlich ein rein reelles Signal, auch wieder
um ein Sample verzögert.

Der Grund ist in den \textbf{Symmetrieeigenschaften} der Transformation zu
suchen. Für \textbf{reellwertige Signale} $x[k]\in\mathbb{R}$ gilt
\begin{itemize}
  \item Axialsymmetrie für Realteil $\Re\{X[\mu]\}$
  \item Axialsymmetrie für Betrag $|X[\mu]|$
  \item Punktsymmetrie für Imaginärteil $\Im\{X[\mu]\}$
  \item Punktsymmetrie für Phase $\angle X[\mu]$
\end{itemize}
bezüglich der Punkte bzw. der Achsen $\mu = m \frac{N}{2}$ mit $m\in\mathbb{Z}$.

Schauen wir unsere Skizze des Spektrums für $X_r[\mu]$ oben genauer an.
Wir sehen in der Grafik die Axialsymmetrie des Realteils
bzgl. $\mu=-4,0,4,8,12$. Die graue
Linie zeigt die für das DFT-Basisband typische Symmetrieachse bei $\frac{N}{2}=4$.
Des weiteren sehen wir die Punktsymmetrie des Imaginärteils bzgl. $m \frac{N}{2}$,
im Bild also bzgl. $\mu=-4,0,4,8,12$. Der graue kleine Punkt bei $\mu=4$ indiziert
den Symmetriepunkt des Basisbandes.
Das Spektrum erfüllt also intentional
die Symmetrieeigenschaften von reellwertigen Signalen.

\textbf{Hinweis 1}: Wir machen uns klar, dass die Symmetrie bei ungeradem $N$
genau mittig zwischen zwei DFT Frequenzen liegt!

\textbf{Hinweis 2}:
Falls $N$ gerade ist und $x[k]$ reellwertig ist, also die $m \frac{N}{2}$
Symmetrien vorliegen, müssen die Spektralwerte $X[\mu = m \frac{N}{2}]$
bei den Symmetriestellen reellwertig sein.
Dies ist im Basisband der Gleichanteil und die DFT-Frequenz
$\e^{\im\pi}$ (halbe Abtastfrequenz).
\end{ExCalc}





\begin{Loesung}
\begin{align}
&x_c[k] = \e^{\im \frac{2\pi}{8} [k -1] } \in\mathbb{C}\\
&x_r[k] = \cos(\frac{2\pi}{8} [k-1]) \in\mathbb{R}
\end{align}
\end{Loesung}




















\newpage
\subsection{DFT einer komplexen Exponentialschwingung}
\label{sec:0C30EB5E76}
\begin{Ziel}
Wir wollen uns anschauen, wie sich die DFT berechnet für eine komplexe
Exponentialschwingung. Wir müssen unbedingt verstanden haben, wie die
Matrixmultiplikation und die analytische Lösung mittels der psinc Funktion
zu dem Ergebnis für nur eine einzige Signalfrequenz kommt. Erst wenn das
klar ist, verstehen wir die DFT komplizierterer Signale. Auch hier gilt
wieder, je mehr wieder verstanden haben, was analytisch passiert, desto einfacher
wird es uns fallen, mit dem Rechner erzeugte DFT-Ergebnisse zu interpretieren.
Ansonsten ist es wieder nur Spieltrieb mit \texttt{fft() / ifft()}.
%
Oder anderer Gedankengang: sich hier einmal die Analytik sauber draufzuschaffen,
wird mutmaßlich weniger Zeitaufwand sein, als sich über ein ganzes Berufsleben
gemittelt immer wieder neu über numerische Ergebnisse zu wundern ;-) Wir werden
die DFT sehr sehr sehr oft benutzen.
\end{Ziel}
\textbf{Aufgabe} {\tiny 0C30EB5E76}: Berechnen Sie für das periodische
Signal $x[k]=x[k+m \cdot N]$, $m\in\mathbb{Z}$
mit der Berechnungsvorschrift für $0\leq k \leq 7$
\begin{align}
x[k] = \e^{+\im \frac{2\pi}{8} \cdot \frac{5}{2} \cdot k}
\end{align}
die $N=8$ DFT, also das Spektrum $X[\mu] = \mathrm{DFT}_8\{x[k]\}$.
%
Skizzieren Sie Betrag und Phase des DFT Spektrums.


\begin{Werkzeug}
Aus Aufgabe \ref{sec:D394560597} kennen wir die Fourier Matrix $\bm{F}_8$.
Wir brauchen für die DFT Hintransformation hier nun die Adjungierte $\bm{F}^\mathrm{H}_8$,
vgl. \eqref{eq:ex11_dft_hin}
\end{Werkzeug}
\begin{Ansatz}
DFT als Matrixmultiplikation
\begin{align}
\bm{x}_\mu = \bm{F}^\mathrm{H} \cdot \bm{x}_k
\end{align}
mit
\begin{align}
%
\bm{x}_k =
\begin{bmatrix}
x[k=0]\\x[k=1]\\x[k=2]\\x[k=3]\\x[k=4]\\x[k=5]\\x[k=6]\\x[k=7]
\end{bmatrix}
%
\qquad
%
\bm{x}_\mu =
\begin{bmatrix}
X[\mu=0]\\X[\mu=1]\\X[\mu=2]\\X[\mu=3]\\X[\mu=4]\\X[\mu=5]\\X[\mu=6]\\X[\mu=7]
\end{bmatrix}
%
\end{align}
und der adjungierten Fourier Matrix
\begin{align}
\bm{F}^\mathrm{H}_8 =
\mu \downarrow
\substack{\rightarrow k\\
\begin{bmatrix}
1 & 1 & 1 & 1 & 1 & 1 & 1 & 1\\
1 & \e^{-\im\frac{1}{4}\pi} & -\im & \e^{-\im\frac{3}{4}\pi} & -1 & \e^{-\im\frac{5}{4}\pi} & +\im & \e^{-\im\frac{7}{4}\pi}\\
1 & -\im & -1 & +\im & +1 & -\im & -1 & +\im\\
1 & \e^{-\im\frac{3}{4}\pi} & +\im & \e^{-\im\frac{1}{4}\pi} & -1 & \e^{-\im\frac{7}{4}\pi} & -\im & \e^{-\im\frac{5}{4}\pi}\\
1 & -1 & +1 & -1 & +1 & -1 & +1 & -1\\
1 & \e^{-\im\frac{5}{4}\pi} & -\im & \e^{-\im\frac{7}{4}\pi} & -1 & \e^{-\im\frac{1}{4}\pi} & +\im & \e^{-\im\frac{3}{4}\pi}\\
1 & +\im & -1 & -\im & +1 & +\im & -1 & -\im\\
1 & \e^{-\im\frac{7}{4}\pi} & +\im & \e^{-\im\frac{5}{4}\pi} & -1 & \e^{-\im\frac{3}{4}\pi} & -\im & \e^{-\im\frac{1}{4}\pi}\\
\end{bmatrix}
}
%
\end{align}
Es ist hilfreich, wenn wir uns nochmal klarmachen, dass durch die Transposition
jetzt $k$ entlang der Spalten variiert und $\mu$ der Zeilenindex ist,
vgl.~\fig{fig:fourier_matrixH}.
\end{Ansatz}
%

\begin{figure}
\centering
\includegraphics[width=0.9\textwidth]{../dft/fourier_matrix_H}
\caption{Konjugiert-komplexe Transponierung der $N=8$ Fourier Matrix,
also $\bm{F}^\mathrm{H}$ \eqref{eq:FourierMatrixH}, so dargestellt, dass
die 8 komplexwertigen DFT Eigensignale getrennt nach Zeilen dargestellt sind.
Da die Zeilen bereits konjugiert-komplexe Vektoren repräsentieren, erfolgt die
DFT-Analyse als einfaches Skalarprodukt zwischen einer dieser Zeilen und einem
zu analysierenden Signal als Spaltenvektor.
Konjugiert-komplex invertiert die Polarität des Imaginäranteils, also nur der Sinussignale,
vgl.~\fig{fig:fourier_matrix}.
}
\label{fig:fourier_matrixH}
\end{figure}


\begin{ExCalc}
Bauen wir uns zunächst den Eingangssignalvektor zusammen
\begin{align}
\bm{x}_k =
\begin{bmatrix}
\e^{+\im \frac{2\pi}{8} \cdot \frac{5}{2} \cdot 0}\\
\e^{+\im \frac{2\pi}{8} \cdot \frac{5}{2} \cdot 1}\\
\e^{+\im \frac{2\pi}{8} \cdot \frac{5}{2} \cdot 2}\\
\e^{+\im \frac{2\pi}{8} \cdot \frac{5}{2} \cdot 3}\\
\e^{+\im \frac{2\pi}{8} \cdot \frac{5}{2} \cdot 4}\\
\e^{+\im \frac{2\pi}{8} \cdot \frac{5}{2} \cdot 5}\\
\e^{+\im \frac{2\pi}{8} \cdot \frac{5}{2} \cdot 6}\\
\e^{+\im \frac{2\pi}{8} \cdot \frac{5}{2} \cdot 7}
\end{bmatrix}
=
\begin{bmatrix}
1\\
\e^{+\im \frac{5\pi}{8}}\\
\e^{+\im \frac{5\pi}{4}}\\
\e^{+\im \frac{15\pi}{8}}\\
\e^{+\im \frac{5\pi}{2}}\\
\e^{+\im \frac{25\pi}{8}}\\
\e^{+\im \frac{15\pi}{4}}\\
\e^{+\im \frac{35\pi}{8}}
\end{bmatrix}
\end{align}


Das innere Produkt (Skalarprodukt) der \underline{ersten} Zeile von $\bm{F}^\mathrm{H}$ und des
Spaltenvektors $\bm{x}_k$ liefert den \underline{ersten} Eintrag des Vektors $\bm{x}_\mu$, also
$X[\mu=0]$ (Gleichanteil)
\begin{align}
&X[\mu=0] =
\begin{bmatrix}
1 & 1 & 1 & 1 & 1 & 1 & 1 & 1
\end{bmatrix}
\begin{bmatrix}
1\\
\e^{+\im \frac{5\pi}{8}}\\
\e^{+\im \frac{5\pi}{4}}\\
\e^{+\im \frac{15\pi}{8}}\\
\e^{+\im \frac{5\pi}{2}}\\
\e^{+\im \frac{25\pi}{8}}\\
\e^{+\im \frac{15\pi}{4}}\\
\e^{+\im \frac{35\pi}{8}}
\end{bmatrix}
=\\
&1 + \im \left(1 - \sqrt{2} - 2\sin(\frac{\pi}{8}) + 2\cos(\frac{\pi}{8})\right)
= \frac{1}{\sin(\frac{5}{16}\pi)} \e^{\im\frac{3}{16}\pi}
\end{align}
%
Für die \underline{zweite} Zeile von $\bm{F}^\mathrm{H}$ folgt das Skalarprodukt für
$X[\mu=1]$ (DFT Grundfrequenz). Damit erhalten wir den \underline{zweiten} Eintrag des
Spaltenvektors $\bm{x}_\mu$.
\begin{align}
&X[\mu=1] =
\begin{bmatrix}
1 & \e^{-\im\frac{1}{4}\pi} & -\im & \e^{-\im\frac{3}{4}\pi} & -1 & \e^{-\im\frac{5}{4}\pi} & +\im & \e^{-\im\frac{7}{4}\pi}
\end{bmatrix}
\begin{bmatrix}
1\\
\e^{+\im \frac{5\pi}{8}}\\
\e^{+\im \frac{5\pi}{4}}\\
\e^{+\im \frac{15\pi}{8}}\\
\e^{+\im \frac{5\pi}{2}}\\
\e^{+\im \frac{25\pi}{8}}\\
\e^{+\im \frac{15\pi}{4}}\\
\e^{+\im \frac{35\pi}{8}}
\end{bmatrix}
=\\
&1 + \im \left(-1 + \sqrt{2} - 2\sin(\frac{\pi}{8}) + 2\cos(\frac{\pi}{8})\right)=
\frac{1}{\sin(\frac{3}{16}\pi)}\,\e^{\im\frac{5}{16}\pi}
\end{align}
%
Wir sehen, dass für dieses vergleichsweise überschaubare Beispiel
Summen zusammenkommen, die relativ viele Umformungen mit Euleridentitäten
benötigen, um sie in geschlossen analytischer Form darstellen zu können
(der Erkenntnisgewinn hält sich in Grenzen, daher können wir das guten Gewissens
links liegen lassen, außer wir wollen ganz viel Euleridentität üben).
%
Das komplette Ergebnis lautet
\begin{align}
\bm{x}_\mu =
\begin{bmatrix}
X[\mu=0]\\X[\mu=1]\\X[\mu=2]\\X[\mu=3]\\X[\mu=4]\\X[\mu=5]\\X[\mu=6]\\X[\mu=7]
\end{bmatrix}
=
\begin{bmatrix}
(\sin(\frac{5}{16}\pi))^{-1} \cdot \e^{+\im \frac{3}{16}\pi}\\
(\sin(\frac{3}{16}\pi))^{-1} \cdot \e^{+\im \frac{5}{16}\pi}\\
(\sin(\frac{1}{16}\pi))^{-1} \cdot \e^{+\im \frac{7}{16}\pi}\\
(\sin(\frac{1}{16}\pi))^{-1} \cdot \e^{-\im \frac{7}{16}\pi}\\
(\sin(\frac{3}{16}\pi))^{-1} \cdot \e^{-\im \frac{5}{16}\pi}\\
(\sin(\frac{5}{16}\pi))^{-1} \cdot \e^{-\im \frac{3}{16}\pi}\\
(\sin(\frac{7}{16}\pi))^{-1} \cdot \e^{-\im \frac{1}{16}\pi}\\
(\sin(\frac{7}{16}\pi))^{-1} \cdot \e^{+\im \frac{1}{16}\pi}\\
\end{bmatrix}
\approx
\begin{bmatrix}
1.2027\e^{+\im \frac{3}{16}\pi}\\
1.8\e^{+\im \frac{5}{16}\pi}\\
5.1258\e^{+\im \frac{7}{16}\pi}\\
5.1258\e^{-\im \frac{7}{16}\pi}\\
1.8\e^{-\im \frac{5}{16}\pi}\\
1.2027\e^{-\im \frac{3}{16}\pi}\\
1.0196\e^{-\im \frac{1}{16}\pi}\\
1.0196\e^{+\im \frac{1}{16}\pi}\\
\end{bmatrix}.
\end{align}
%$\Omega_{0,\mu} = -\frac{2\pi}{8} \cdot \frac{5}{2} + \frac{2\pi}{8} \cdot 2$

%$\Omega_{0,\mu} \frac{8}{2}= (-\frac{2\pi}{8} \cdot \frac{5}{2} + \frac{2\pi}{8} \cdot 2) \frac{8}{2}
%=(- 5\pi + 4\pi) \frac{1}{2} =(- \pi) \frac{1}{2} \rightarrow \sin(-\frac{\pi}{2})=-1
%$

%$\Omega_{0,\mu} \frac{1}{2}=(-\frac{2\pi}{8} \cdot \frac{5}{2} + \frac{2\pi}{8} \cdot 2) \frac{1}{2}=
%(-\frac{\pi}{8} \cdot 5 + \frac{\pi}{2}) \frac{1}{2}=(-\frac{5\pi}{16}  + \frac{4\pi}{16}) = -\frac{\pi}{16}\rightarrow \sin(-\frac{\pi}{16})$
%\begin{align}
%- (-\frac{2\pi}{8} \cdot \frac{5}{2} + \frac{2\pi}{8} \cdot 2 ) \frac{7}{2}\\
%- (-\frac{35\pi}{16} + \frac{28\pi}{16})\\
%\end{align}
%
Wir könnten das z.B. mit Matlab
\begin{verbatim}
X = fft(exp(+1j*2*pi/8*2.5*[0:7].'))
abs(X)
angle(X)/pi*16
\end{verbatim}
und mit Python mit importiertem \texttt{numpy} Paket
\begin{verbatim}
import numpy as np
X = np.fft.fft(np.exp(+1j*2*np.pi/8 * 2.5 * np.arange(8)))
print(np.abs(X))
print(np.angle(X)/np.pi*16)
\end{verbatim}
gegenchecken.
%

Nun ist wichtig, das komplexe Skalarprodukt zu verstehen: ein komplexer
Zeilenvektor aus $\bm{F}^\mathrm{H}$ beinhaltet ein DFT Eigensignal,
konjugiert-komplex ist durch $^\mathrm{H}$ schon beinhaltet,
deshalb haben wir es ganz sauber mit einem komplexen Skalarprodukt zu tun.
Das zu analysierende Signal ist der (bei Bedarf komplexwertige) Spaltenvektor $\bm{x}_k$.
Wenn diese beiden Vektoren sehr kongruent verlaufen ($N$-dimensionaler
komplexer Vektorraum ist schwierig in der Vorstellung, aber die Idee funktioniert
ja genau so in reell 2D), ist viel Signalenergie
bei der betrachteten DFT Frequenz enthalten. Wenn die beiden Vektoren orthogonal
verlaufen ist keine Energie bei der betrachteten DFT Frequenz enthalten.
%
Weil die DFT-Eigensignale an sich orthogonal sind (wir hatten ja intentional
eine wunderschöne orthogonale
Vektorbasis erschaffen) ist die DFT eine Analyse wie viel Amplitude / Energie
bei exakt den $N$ Frequenzen enthalten ist. Weil wir komplex-wertig arbeiten, können
wir auch den Phasenunterschied (bzw. Verzögerung)
zwischen Analysesignal (eine Spalte in $\bm{F}$)
und zu analysierendem Signal $\bm{x}_k$ bestimmen. Nichts anderes macht vom Wesen
die Analyse bei der komplexe Fourierreihe mit den kontinuierlichen, orthogonalen
Funktionen $\e^{\im\omega_0 \mu t}$.
%

Wir haben nun mit dem Skalarprodukt das Wesen der DFT erklärt,
die analytische Darstellung der Ergebnisse aus der Skalarprodukt-Summe
ist viel Gefrickel.
%
Das geht eleganter und bietet einen zweiten wichtigen Blickpunkt. Dazu
machen wir eine Rechnung, die wir so ähnlich schon mal bei der DTFT gesehen haben.
Unter bestimmten Umständen sind Summen geschlossen analytisch darstellbar.
Das haben wir hier zum Glück wieder einmal vorliegen.
Die DFT für unser komplexwertiges Signal lautet allgemein
\begin{align}
X[\mu] =& \sum_{\mu=0}^{N-1} \e^{+\im\,\Omega_0 k} \e^{-\im\,\frac{2\pi}{N} k \mu}
=  \sum_{\mu=0}^{N-1} \e^{+\im\,(\Omega_0 - \frac{2\pi}{N} \mu) \, k}
=  \sum_{\mu=0}^{N-1} \e^{-\im\,(-\Omega_0 + \frac{2\pi}{N} \mu) \, k}
=  \sum_{\mu=0}^{N-1} \e^{-\im\, \Omega_{0,\mu} \,k}
\end{align}
mit $\Omega_{0,\mu} = -\Omega_0 + \frac{2\pi}{N} \mu$.
Dies ist wieder die endliche geometrische Reihe mit der geschlossenen Lösung
(vgl.~Glg. \eqref{eq:ue10_sum_to_psinc}ff (10.71)ff)
\begin{align}
X[\mu] = \e^{-\im \,\Omega_{0,\mu}\,\frac{N-1}{2}} \cdot
\frac{\sin\left(\Omega_{0,\mu}\cdot\frac{N}{2}\right)}{\sin\left(\Omega_{0,\mu}\cdot\frac{1}{2}\right)}
=\e^{-\im \,\Omega_{0,\mu}\,\frac{N-1}{2}} \cdot N \cdot \mathrm{psinc}_N(\Omega_{0,\mu})
.
\end{align}
Wir begegnen wieder einmal der periodischen Sinc Funktion (vgl. Glg.~\eqref{eq:ue10_psinc} (10.74))
\begin{align*}
\mathrm{psinc}_N(\Omega) = \frac{1}{N}\frac{\sin({\Omega\frac{N}{2}})}{\sin({\Omega\frac{1}{2}})}.
\end{align*}
%
Mit diesen Formeln finden wir die analytischen Ausdrücke für Betrag und Phase
vergleichsweise schöner und einfacher, z.B. für $\mu=2$
\begin{align}
&|X[\mu=2]| = \frac{1}{\sin(\frac{\pi}{16})} \approx 5.1258\\
&\angle X[\mu=2] = +\frac{7}{16}\pi
\end{align}
Es ist hilfreich, wenn wir uns klarmachen, dass wir für $\Omega_0=0$ die DFT der
Rechteckfolge $\mathrm{rect}_N[k]$
in dieser Formel abbilden, hier ist wieder einmal ein wichtiger Speziallfall
nebenbei mit abgefallen.

Die beiden Denkkonzepte
a) die Betrachtungsweise des
Skalarprodukts
und
b) das Herauspicken von DFT-Betrag und -Phase aus der psinc Funktion
für eine komplexe Exponentialfolge
sollten wir verinnerlichen. Dann haben für uns die Zahlen,
die ein Rechner bei numerischer DFT beliebiger Folgen liefert, einen tieferen
Sinn.

Quizfrage: Wenn wir das Eingangssignal zu
\begin{align}
x[k] = \e^{+\im \frac{2\pi}{8} \cdot 2 \cdot k}
%\end{align}
\qquad\text{oder}\qquad
%\begin{align}
x[k] = \e^{+\im \frac{2\pi}{8} \cdot 3 \cdot k}
\end{align}
abändern, wie schaut dann das DFT Spektrum aus? Wenn wir das durch reine
Überlegung(!) lösen können, haben wir viel vom Wesen der DFT
verstanden.
%
Falls ein Aha Erlebnis ausbleibt, ruhig nochmal die vorherige Aufgabe~\ref{sec:D394560597} (11.1)
intensiver anschauen.

\end{ExCalc}

\begin{Loesung}
%
%
Das Spektrum ist in \fig{fig:dft_complex_signal_Spectrum_0C30EB5E76} veranschaulicht. Wir sehen
3 Perioden, davon das DFT Basisband für $0\leq \mu \leq 7$ in orange.
Es gibt ein Betragsmaximum bei $\mu=2$ und $\mu=3$, diese DFT Frequenzen sind
also prominenter im Signal vertreten als die anderen Frequenzen.
%
Wir wissen, dass unser Signal eigentlich nur eine Frequenz hat, nämlich
$\Omega=\frac{2\pi}{8}\cdot\frac{5}{2}$.
Warum ist diese Signalenergie in der DFT nun auf die $\mu$, vor allem auf $\mu=2$ und
$\mu=3$  verschmiert und suggeriert sogar einen Gleichanteil bei $\mu=0$.
%
Schuld ist \textbf{nie} die DFT, die macht alles richtig, immer!!, sie bildet
einfach 'nur' einen $N$-dimensionalen Vektor auf einen anderen ab.
%
\textbf{Unsere} obige Aussage---unser Signal bestünde eigentlich nur aus der Frequenz
$\Omega=\frac{2\pi}{8}\cdot\frac{5}{2}$---ist \textbf{nicht korrekt}.
Dazu schauen wir uns das Zeitsignal in \fig{fig:dft_complex_signal_Signal_0C30EB5E76} an, oben
Real- unten Imaginärteil. In orange wieder die angenommene Periode $N=8$, also
die Samples für $0 \leq k \leq 7$.
In grau als durchgemalte Linien (was bei Folgen ja eigentlich strenggenommen
nicht erlaubt ist, aber dadurch wird es besser deutlich) sehen wir das Signal
\begin{align}
\e^{+\im \frac{2\pi}{8} \cdot \frac{5}{2} \cdot k} =
\exp\left(+\im \frac{2\pi}{\nicefrac{16}{5}} \cdot k\right)
\end{align}
also wenn es frei vor sich hinschwingen dürfte.
Wir sehen in der Formel (vgl. Übung~\ref{sec:ue8_dt_systems_intro_conv} (8))
und in der Grafik, dass dieses Signal periodisch in 16 Samples ist.
%
Diese inhärente Signalperiode beschneiden wir durch Wahl unserer DFT Länge,
hier $N=8$, also exakt in der Mitte.
%
Wir können uns bei der Fourierreihe hoffentlich bildlich vorstellen, wie das
Spektrum hochfrequent Energie gewinnt, wenn wir einen Sinusbogen zeitlich
hart anschneiden.
%
Nicht anderes passiert hier, es ist in der Literatur als Spectral leakage,
leakage effect bzw. deutsch Leckeffekt bekannt.
%
Falls das zu analysierende Signal nicht genau mit den Signalperiode(n)
der DFT-Frequenzen übereinstimmt, suggeriert die DFT-inhärente Periodisierung
gewisse Frequenzanteile, die im Signal eigentlich nicht da sind.
%
Um das aber nochmal ganz deutlich zu machen: das $N$-DFT Spektrum ist das exakte
Spektrum des $N$-periodisierten Signals.
Die Fouriermatrix kann nix für unser Unvermögen einen nicht repräsentativen
Signalausschnitt ausgewählt zu haben.
%
Es ist daher unsere Aufgabe sicherzustellen (das braucht viel SigSys Erfahrung),
dass das $N$-periodisierte Signal genau den repräsentativen Signalausschnitt
enthält, für den wir mit der DFT das Spektrum ermitteln wollen.
%
In unserem einfachen Beispiel reicht es schon die DFT Länge auf $N=16$ zu erhöhen.
%
Überlegen wir zuerst warum das so ist, wie das DFT Spektrum ausschauen könnte
und checken dann mit dem Rechner.
%

Die sogenannte Fensterung ist ein weiteres Konzept um Leakage zu unterdrücken,
auf Kosten der Frequenzauflösung. Dies schauen wir uns im Master Modul
Digital Signal Processing genauer an.
%
\end{Loesung}

\begin{figure}
\centering
\includegraphics[width=0.75\textwidth]{../dft/dft_complex_signal_Spectrum_0C30EB5E76}
\caption{Aufgabe \ref{sec:0C30EB5E76}. DFT Spektrum $X[\mu]=\mathrm{DFT}_8\{x[k]\}$ ,
Betrag oben, Phase unten. DFT Basisband in orange.
\texttt{dft\_complex\_signal\_0C30EB5E76.ipynb}}
\label{fig:dft_complex_signal_Spectrum_0C30EB5E76}
\end{figure}
%
\begin{figure}
\centering
\includegraphics[width=0.75\textwidth]{../dft/dft_complex_signal_Signal_0C30EB5E76.pdf}
\caption{Aufgabe \ref{sec:0C30EB5E76}. Komplexe Folge $x[k]$, oben Realteil,
unten Imaginärteil. Orange: Signalausschnitt für die $N=8$ DFT. Das Signal
hätte ohne unsere künstlich vorgenommene Periodisierung mit $N=8$
eine Periode von 16 Samples (siehe graue Linie). Durch die Beschneidung der
Signalperiode kommt es im DFT-Spektrum
\fig{fig:dft_complex_signal_Spectrum_0C30EB5E76} zum sogennanten Leakage Effect.
\texttt{dft\_complex\_signal\_0C30EB5E76.ipynb}}
\label{fig:dft_complex_signal_Signal_0C30EB5E76}
\end{figure}







\clearpage
\subsection{Zyklische / Lineare / Schnelle Faltung}
\label{sec:C8864C8D9F}
\begin{Ziel}
Wir wollen die Konzepte der linearen und zyklischen Faltung verinnerlichen.
Die zyklische Faltung ist wichtig, weil wir mit ihr in der Praxis sehr
oft lineare Faltungsergebnisse ausrechnen. Dies erreichen wir
durch geeignetes Auffüllen von Nullen in die zu faltenden Folgen.
Die zyklische Faltung $y[k] = x[k]\circledast_N h[k]$
entspricht im DFT-Bildbereich der Multiplikation von Spektren, d.h. wir könnten
den Umweg
\begin{align}
y[k] = \mathrm{IDFT}_N\{\qquad\mathrm{DFT}_N\{x[k]\} \cdot \mathrm{DFT}_N\{h[k]\}\qquad\}
\end{align}
nehmen. Es gibt extrem effiziente Algorithmen für die DFT,
unter dem Sammelbegriff Fast Fourier Transform (FFT), die diese Operation für
große $N$ mit Umweg über Hin- und Rücktransformation deutlich schneller machen,
als die zyklische Faltung im Zeitbereich zu rechnen. Dies wird als schnelle
Faltung bezeichnet und ist ein Kernbaustein der digitalen Signalverarbeitung.
Die Grundidee aller FFTs ist die Fourier Matrix geeignet zu Faktorisieren, um
die benötigten (viele redundante) Rechenschritte zu reduzieren. Das schauen
wir uns im Mastermodul Digital Signal Processing genauer an.

Hier wollen wir uns anhand eines einfachen Beispiels mit einfachen Zahlen die
genannten Konzepte verdeutlichen. Weil die Folgen sehr kurz sind, sollten wir
nicht erwarten, dass die schnelle Faltung deutlich performanter ist.
Für größere $N$ fällt dies aber zunehmend deutlicher auf, das sind dann je nach $N$,
Größenordnungen wo wir von vielen Minuten vs. wenige Sekunden reden, also bedenkenswert.
\end{Ziel}
\textbf{Aufgabe} {\tiny C8864C8D9F}:
Berechnen Sie für die beiden Folgen
\begin{align}
&x[k] = -\delta[k] +2 \delta[k-1] + 4 \delta[k-2]\\
&h[k] = +3 \delta[k] + \delta[k-1] + 5 \delta[k-2]
\end{align}
die lineare Faltung $y_l[k] = x[k] \ast h[k]$ und die zyklische Faltung
$y_c[k] = x[k] \circledast_3 h[k]$
mit Periode $3$.
\begin{Werkzeug}
lineare Faltung aus Übung~\ref{sec:ue8_dt_systems_intro_conv} (8),
Matrixmultiplikation, Dualität zyklische Faltung
vs. Multiplikation
\end{Werkzeug}
\begin{Ansatz}
Wir könnten die grafische Methode aus Übung~\ref{sec:ue8_dt_systems_intro_conv} (8) benutzen.
Das ist unten
für die zyklische Faltung ausführlich skizziert.
Dies bildet die Rechenvorschrift
\begin{align}
x[k] \circledast_3 h[k] = \sum_{\kappa=0}^{2} x[\kappa] h[-\kappa + k]
\end{align}
grafisch ab.
Die lineare Faltung nimmt periodisierte Folgen an, hier mit $N=3$, also
$x[k] = x[k+m \cdot N]$ und $h[k] = h[k+m \cdot N]$ mit $m\in\mathbb{Z}$.
Wir müssen die Faltungssumme also nur für $k=0,1,2$ auswerten und können
den Rest periodisch auffüllen.
%
Die grafische Lösung funktioniert nun völlig gleich wie eine lineare Faltung,
wir müssen nur beachten das Signal immer \textbf{zyklisch 'nachzuschieben'} und nur den
\textbf{Bereich} $0 \leq \kappa \leq 2$ \textbf{für die Summe} zu verwenden.
%
Im gewählten Rechenweg ist die Folge $h[-\kappa+k]$ das zeitlich gespiegelte und zu
verschiebende Signal.
%
Die grau rechteckige Markierung zeigt eine Periode bzw. den $\kappa$ Bereich,
der für die Faltungssumme relevant ist.
%
Wir bekommen als Ergebnis
\begin{align}
&y_{cyc}[k] = x[k] \circledast_3 h[k] = 11\delta[k] + 25\delta[k-1] + 9\delta[k-2]\nonumber\\
&y_{cyc}[k] = y_{cyc}[k + 3 m], m\in\mathbb{Z}
\end{align}
%
Die lineare Faltung ergibt
\begin{align}
y_{lin}[k] = x[k] \ast h[k] = -3\delta[k] + 5\delta[k-1] + 9\delta[k-2] + 14\delta[k-3] + 20\delta[k-4]
\end{align}
%
In Matlab könnten wir kurz checken:
\begin{verbatim}
x = [-1; 2; 4];
h = [3; 1; 5];
cconv(h,x,3) % zyklische Faltung mit Periode 3 ==
ifft(fft(x).*fft(h))  % Umweg über DFT/IDFT, elementweise Multiplikation
conv(h,x)  % lineare Faltung ==
ifft(fft([x;0;0]).*fft([h;0;0]))  % zeropadding auf Ny=5, Umweg über DFT/IDFT
\end{verbatim}


\end{Ansatz}

\clearpage
%% k = 0
\begin{center}
\begin{tikzpicture}[scale=0.5]
\def\tic{0.15};
\def\k{0}
\begin{scope}
\fill[C7!30] (-0.5,0) rectangle (2.5,6);
\draw[->] (-8,0) -- (8,0) node[below]{$\kappa$};
\draw[->] (0,-1.5) -- (0,6) node[above]{\shortstack{\textcolor{C0}{$x[\kappa]\,,$}\\\textcolor{C1}{$h[-\kappa+\k]$}}};
\foreach \y in {-1,...,5}{\draw (\tic,\y) -- (-\tic,\y);};
\draw[stem, C1, xshift=2pt, yshift=1pt] plot coordinates{ (-2+\k, 5) (-1+\k,1) (0+\k,3)};
\draw[stem, C1, xshift=2pt, yshift=1pt] plot coordinates{ (-2+\k-3, 5) (-1+\k-3,1) (0+\k-3,3)};
\draw[stem, C1, xshift=2pt, yshift=1pt] plot coordinates{ (-2+\k+3, 5) (-1+\k+3,1) (0+\k+3,3)};
\draw[stem] plot coordinates{(0,-1) (1,2) (2,4)};
\draw[stem] plot coordinates{(0-3,-1) (1-3,2) (2-3,4)};
\draw[stem] plot coordinates{(0+3,-1) (1+3,2) (2+3,4)};
\node at (-6,1){$k=\k$};
\draw[C0] (-6,3) node {$\dots$};
\draw[C1] (-6,4) node {$\dots$};
\draw[C0] (7,3) node {$\dots$};
\draw[C1] (7,4) node {$\dots$};
%
\draw (0,1) node[left]{$1$};
\draw (0,2) node[left]{$2$};
\draw (0,3) node[left]{$3$};
\draw (0,4) node[left]{$4$};
\draw (0,5) node[left]{$5$};
%
\draw (1,0) node[below]{$1$};
\draw (2,0) node[below]{$2$};
%
\end{scope}
\begin{scope}[yshift=-8cm]
\fill[C7!30] (-0.5,-1.5) rectangle (2.5,5);
\draw[->] (-0.5,0) -- (3.5,0) node[right]{$\kappa$};
\draw[->] (0,-2.5) -- (0,5.5) node[above]{$x[\kappa]\cdot h[-\kappa+\k]$};
\foreach \y in {-1,...,5}{\draw (\tic,\y) -- (-\tic,\y);};
\draw[stem, C3] plot coordinates{(0,-3/4) (1,10/4) (2,4/4)};
%
\draw (0,-1) node[left]{$-4$};
\draw (0,1) node[left]{$4$};
\draw (0,2) node[left]{$8$};
\draw (0,3) node[left]{$12$};
\draw (0,4) node[left]{$16$};
\draw (0,5) node[left]{$20$};
%
\draw (1,0) node[below]{$1$};
\draw (2,0) node[below]{$2$};
%
\end{scope}
\begin{scope}[yshift=-8cm, xshift=14cm]
\fill[C7!30] (-0.5,0) rectangle (2.5,7);
\draw[->] (-4,0) -- (8,0) node[right]{$k$};
\draw[->] (0,-0.5) -- (0,8) node[above]{$y_c[k] = x[k] \circledast_3 h[k]$};
\foreach \y in {0,...,6}{\draw (\tic,\y) -- (-\tic,\y);};
\draw[stem, C2!10] plot coordinates{(0,11/4) (1,25/4) (2,9/4)};
\draw[stem, C2!10] plot coordinates{(0-3,11/4) (1-3,25/4) (2-3,9/4)};
\draw[stem, C2!10] plot coordinates{(0+3,11/4) (1+3,25/4) (2+3,9/4)};
\draw[stem, C2, dashed] plot coordinates{(0,11/4)};
\draw[stem, C2, dashed] plot coordinates{(0-3,11/4)};
\draw[stem, C2, dashed] plot coordinates{(0+3,11/4)};
%
\draw (0,1) node[left]{$4$};
\draw (0,2) node[left]{$8$};
\draw (0,3) node[left]{$12$};
\draw (0,4) node[left]{$16$};
\draw (0,5) node[left]{$20$};
\draw (0,6) node[left]{$24$};
%
\draw (1,0) node[below]{$1$};
\draw (2,0) node[below]{$2$};
%
\draw[C2] (-4,4) node {$\dots$};
\draw[C2] (6,4) node {$\dots$};
%
\end{scope}
\end{tikzpicture}
\end{center}

























\clearpage
%% k = 1
\begin{center}
\begin{tikzpicture}[scale=0.5]
\def\tic{0.15};
\def\k{1}
\begin{scope}
\fill[C7!30] (-0.5,0) rectangle (2.5,6);
\draw[->] (-8,0) -- (8,0) node[below]{$\kappa$};
\draw[->] (0,-1.5) -- (0,6) node[above]{\shortstack{\textcolor{C0}{$x[\kappa]\,,$}\\\textcolor{C1}{$h[-\kappa+\k]$}}};
\foreach \y in {-1,...,5}{\draw (\tic,\y) -- (-\tic,\y);};
\draw[stem, C1, xshift=2pt, yshift=1pt] plot coordinates{ (-2+\k, 5) (-1+\k,1) (0+\k,3)};
\draw[stem, C1, xshift=2pt, yshift=1pt] plot coordinates{ (-2+\k-3, 5) (-1+\k-3,1) (0+\k-3,3)};
\draw[stem, C1, xshift=2pt, yshift=1pt] plot coordinates{ (-2+\k+3, 5) (-1+\k+3,1) (0+\k+3,3)};
\draw[stem] plot coordinates{(0,-1) (1,2) (2,4)};
\draw[stem] plot coordinates{(0-3,-1) (1-3,2) (2-3,4)};
\draw[stem] plot coordinates{(0+3,-1) (1+3,2) (2+3,4)};
\node at (-6,1){$k=\k$};
\draw[C0] (-6,3) node {$\dots$};
\draw[C1] (-6,4) node {$\dots$};
\draw[C0] (7,3) node {$\dots$};
\draw[C1] (7,4) node {$\dots$};
%
\draw (0,1) node[left]{$1$};
\draw (0,2) node[left]{$2$};
\draw (0,3) node[left]{$3$};
\draw (0,4) node[left]{$4$};
\draw (0,5) node[left]{$5$};
%
\draw (1,0) node[below]{$1$};
\draw (2,0) node[below]{$2$};
%
\end{scope}
\begin{scope}[yshift=-8cm]
\fill[C7!30] (-0.5,-1.5) rectangle (2.5,5);
\draw[->] (-0.5,0) -- (3.5,0) node[right]{$\kappa$};
\draw[->] (0,-2.5) -- (0,5.5) node[above]{$x[\kappa]\cdot h[-\kappa+\k]$};
\foreach \y in {-1,...,5}{\draw (\tic,\y) -- (-\tic,\y);};
\draw[stem, C3] plot coordinates{(0,-1/4) (1,6/4) (2,20/4)};
%
\draw (0,-1) node[left]{$-4$};
\draw (0,1) node[left]{$4$};
\draw (0,2) node[left]{$8$};
\draw (0,3) node[left]{$12$};
\draw (0,4) node[left]{$16$};
\draw (0,5) node[left]{$20$};
%
\draw (1,0) node[below]{$1$};
\draw (2,0) node[below]{$2$};
%
\end{scope}
\begin{scope}[yshift=-8cm, xshift=14cm]
\fill[C7!30] (-0.5,0) rectangle (2.5,7);
\draw[->] (-4,0) -- (8,0) node[right]{$k$};
\draw[->] (0,-0.5) -- (0,8) node[above]{$y_c[k] = x[k] \circledast_3 h[k]$};
\foreach \y in {0,...,6}{\draw (\tic,\y) -- (-\tic,\y);};
\draw[stem, C2!10] plot coordinates{(0,11/4) (1,25/4) (2,9/4)};
\draw[stem, C2!10] plot coordinates{(0-3,11/4) (1-3,25/4) (2-3,9/4)};
\draw[stem, C2!10] plot coordinates{(0+3,11/4) (1+3,25/4) (2+3,9/4)};
\draw[stem, C2] plot coordinates{(0,11/4)};
\draw[stem, C2] plot coordinates{(0-3,11/4)};
\draw[stem, C2] plot coordinates{(0+3,11/4)};
\draw[stem, C2, dashed] plot coordinates{(1,25/4)};
\draw[stem, C2, dashed] plot coordinates{(1-3,25/4)};
\draw[stem, C2, dashed] plot coordinates{(1+3,25/4)};
%
\draw (0,1) node[left]{$4$};
\draw (0,2) node[left]{$8$};
\draw (0,3) node[left]{$12$};
\draw (0,4) node[left]{$16$};
\draw (0,5) node[left]{$20$};
\draw (0,6) node[left]{$24$};
%
\draw (1,0) node[below]{$1$};
\draw (2,0) node[below]{$2$};
%
\draw[C2] (-4,4) node {$\dots$};
\draw[C2] (6,4) node {$\dots$};
%
\end{scope}
\end{tikzpicture}
\end{center}





















\clearpage
%% k = 2
\begin{center}
\begin{tikzpicture}[scale=0.5]
\def\tic{0.15};
\def\k{2}
\begin{scope}
\fill[C7!30] (-0.5,0) rectangle (2.5,6);
\draw[->] (-8,0) -- (8,0) node[below]{$\kappa$};
\draw[->] (0,-1.5) -- (0,6) node[above]{\shortstack{\textcolor{C0}{$x[\kappa]\,,$}\\\textcolor{C1}{$h[-\kappa+\k]$}}};
\foreach \y in {-1,...,5}{\draw (\tic,\y) -- (-\tic,\y);};
\draw[stem, C1, xshift=2pt, yshift=1pt] plot coordinates{ (-2+\k, 5) (-1+\k,1) (0+\k,3)};
\draw[stem, C1, xshift=2pt, yshift=1pt] plot coordinates{ (-2+\k-3, 5) (-1+\k-3,1) (0+\k-3,3)};
\draw[stem, C1, xshift=2pt, yshift=1pt] plot coordinates{ (-2+\k+3, 5) (-1+\k+3,1) (0+\k+3,3)};
\draw[stem] plot coordinates{(0,-1) (1,2) (2,4)};
\draw[stem] plot coordinates{(0-3,-1) (1-3,2) (2-3,4)};
\draw[stem] plot coordinates{(0+3,-1) (1+3,2) (2+3,4)};
\node at (-6,1){$k=\k$};
\draw[C0] (-6,3) node {$\dots$};
\draw[C1] (-6,4) node {$\dots$};
\draw[C0] (7,3) node {$\dots$};
\draw[C1] (7,4) node {$\dots$};
%
\draw (0,1) node[left]{$1$};
\draw (0,2) node[left]{$2$};
\draw (0,3) node[left]{$3$};
\draw (0,4) node[left]{$4$};
\draw (0,5) node[left]{$5$};
%
\draw (1,0) node[below]{$1$};
\draw (2,0) node[below]{$2$};
%
\end{scope}
\begin{scope}[yshift=-8cm]
\fill[C7!30] (-0.5,-1.5) rectangle (2.5,5);
\draw[->] (-0.5,0) -- (3.5,0) node[right]{$\kappa$};
\draw[->] (0,-2.5) -- (0,5.5) node[above]{$x[\kappa]\cdot h[-\kappa+\k]$};
\foreach \y in {-1,...,5}{\draw (\tic,\y) -- (-\tic,\y);};
\draw[stem, C3] plot coordinates{(0,-5/4) (1,2/4) (2,12/4)};
%
\draw (0,-1) node[left]{$-4$};
\draw (0,1) node[left]{$4$};
\draw (0,2) node[left]{$8$};
\draw (0,3) node[left]{$12$};
\draw (0,4) node[left]{$16$};
\draw (0,5) node[left]{$20$};
%
\draw (1,0) node[below]{$1$};
\draw (2,0) node[below]{$2$};
%
\end{scope}
\begin{scope}[yshift=-8cm, xshift=14cm]
\fill[C7!30] (-0.5,0) rectangle (2.5,7);
\draw[->] (-4,0) -- (8,0) node[right]{$k$};
\draw[->] (0,-0.5) -- (0,8) node[above]{$y_c[k] = x[k] \circledast_3 h[k]$};
\foreach \y in {0,...,6}{\draw (\tic,\y) -- (-\tic,\y);};
\draw[stem, C2] plot coordinates{(0,11/4) (1,25/4) (2,9/4)};
\draw[stem, C2] plot coordinates{(0-3,11/4) (1-3,25/4) (2-3,9/4)};
\draw[stem, C2] plot coordinates{(0+3,11/4) (1+3,25/4) (2+3,9/4)};
%
\draw (0,1) node[left]{$4$};
\draw (0,2) node[left]{$8$};
\draw (0,3) node[left]{$12$};
\draw (0,4) node[left]{$16$};
\draw (0,5) node[left]{$20$};
\draw (0,6) node[left]{$24$};
%
\draw (1,0) node[below]{$1$};
\draw (2,0) node[below]{$2$};
%
\draw[C2] (-4,4) node {$\dots$};
\draw[C2] (6,4) node {$\dots$};
%
\end{scope}
\end{tikzpicture}
\end{center}


\newpage
\begin{ExCalc}
Matrixmultiplikationen speziell zusammengesetzter Matrizen sind Faltungen, siehe
Aufgabe~\ref{sec:FD58EEB1EC} (8.2) für die lineare Faltung.
Die zyklische Faltung wird mit zyklischen Matrizen abgebildet.

\textbf{Zyklische Faltung} (3er) der Spaltenvektoren
\begin{align}
\begin{bmatrix}
-1\\2\\4
\end{bmatrix}
\circledast_3
\begin{bmatrix}
3\\1\\5
\end{bmatrix}
=
\begin{bmatrix}
11\\25\\9
\end{bmatrix}
\end{align}
als Matrix Multiplikation mit \textbf{zyklischen Matrizen} auch \textbf{zirkulante Matrizen}
(diese sind komplett definiert durch die erste Spalte oder erste Zeile).
Bei der Multiplikation mit zyklischen Matrizen gilt Kommutativität, das ist erfreulicherweise
konsistent mit der Kommutativitätseigenschaft der Faltung.
%
Die obigen Spaltenvektoren sind die jeweils \textbf{erste Spalte} der zyklischen Matrizen
\begin{align}
\begin{bmatrix}
-1 &  4 &  2\\
 2 & -1 &  4\\
 4 &  2 & -1
\end{bmatrix}
\cdot
\begin{bmatrix}
3 & 5 & 1\\
1 & 3 & 5\\
5 & 1 & 3
\end{bmatrix}
=
\begin{bmatrix}
11  &   9  &  25\\
25  &  11  &  9\\
 9  &  25  &  11
\end{bmatrix}.
\end{align}
%
Die Multiplikation zweier zyklischer Matrizen erzeugt immer eine zyklische Ergebnismatrix.
Daher machen wir uns klar, dass wir von der Ergebnismatrix nur die erste Spalte
oder erste Zeile ausrechnen müssen, um die gesamte Matrix zu kennen.
%
Deswegen: Die Linearkombination von Spalten der ersten Matrix liefert direkt das gewünschte
zyklische Faltungsergebnis als Spaltenvektor
\begin{align}
3 \cdot
\begin{bmatrix}
-1 \\
 2 \\
 4
\end{bmatrix}
+
1 \cdot
\begin{bmatrix}
4\\
-1\\
2
\end{bmatrix}
+ 5 \cdot
\begin{bmatrix}
2\\
4\\
-1
\end{bmatrix}
=
\begin{bmatrix}
11\\
25\\
 9
\end{bmatrix}.
\end{align}
%


Die \textbf{Lineare Faltung}
\begin{align}
\begin{bmatrix}
-1 \\
 2 \\
 4
\end{bmatrix}
\ast
\begin{bmatrix}
3\\
1\\
5
\end{bmatrix}
=
\begin{bmatrix}
-3\\5\\9\\14\\20
\end{bmatrix}
\end{align}
kann \textbf{über} die \textbf{zyklische Faltung} (5er) dargestellt werden:
Die Länge des Ergebnisses der linearen Faltung ist  $N_y = N_x+N_h-1=5$.
Deswegen füllen wir die beiden Folgen mit Nullen auf Länge $N_y=5$ auf und führen
dann die zyklische Faltung so wie oben mit zyklischen Matrizen aus
\begin{align}
\begin{bmatrix}
-1 \\
 2 \\
 4 \\
 0 \\
 0
\end{bmatrix}
\circledast_5
\begin{bmatrix}
3\\
1\\
5\\
0\\
0
\end{bmatrix}
=
\begin{bmatrix}
-3\\5\\9\\14\\20
\end{bmatrix}
\end{align}
%
Es resultiert das lineare Faltungsergebnis.
%
Die zyklische Matrix-Operation in aller Ausführlichkeit:
die obigen Spaltenvektoren sind wieder die \textbf{ersten Spalten} der zyklischen
Matrizen
\begin{align}
\label{eq:C8864C8D9F_LinConv_with_CycConv}
\begin{bmatrix}
-1  &   0  &   0  &   4  &   2\\
 2  &  -1  &   0  &   0  &   4\\
 4  &   2  &  -1  &   0  &   0\\
 0  &   4  &   2  &  -1  &   0\\
 0  &   0  &   4  &   2  &  -1
\end{bmatrix}
\cdot
\begin{bmatrix}
3  &   0  &   0  &   5  &   1\\
1  &   3  &   0  &   0  &   5\\
5  &   1  &   3  &   0  &   0\\
0  &   5  &   1  &   3  &   0\\
0  &   0  &   5  &   1  &   3
\end{bmatrix}
=
\begin{bmatrix}
-3  &  20  &  14  &   9  &  5\\
 5  &  -3  &  20  &  14  &  9\\
 9  &   5  &  -3  &  20  &  14\\
14  &   9  &   5  &  -3  &  20\\
20  &  14  &   9  &   5  &  -3
\end{bmatrix}
\end{align}
Und auch hier reicht es wieder, nur die erste Spalte der Ergebnismatrix aus der
Linearkombination
\begin{align}
3 \cdot
\begin{bmatrix}
-1\\
 2\\
 4\\
 0\\
 0
\end{bmatrix}
+ 1 \cdot
\begin{bmatrix}
0\\
-1\\
2\\
4\\
0
\end{bmatrix}
+ 5 \cdot
\begin{bmatrix}
0\\
0\\
-1\\
2\\
4
\end{bmatrix}
+ 0 \cdot
\begin{bmatrix}
4\\
0\\
0\\
-1\\
2
\end{bmatrix}
+ 0 \cdot
\begin{bmatrix}
2\\
4\\
0\\
0\\
-1
\end{bmatrix}
=
\begin{bmatrix}
-3\\
 5\\
 9\\
14\\
20
\end{bmatrix}
\end{align}
zu berechnen, weil erstens das Faltungsergebnis damit bekannt und zweitens die
zyklische Matrix vollständig beschreibbar.
%
Die letzen beiden Spalten werden jeweils mit Null gewichtet, sind also
für das Ergebnis nicht relevant. Wenn wir die also weglassen
\begin{align}
3 \cdot
\begin{bmatrix}
-1\\
 2\\
 4\\
 0\\
 0
\end{bmatrix}
+ 1 \cdot
\begin{bmatrix}
0\\
-1\\
2\\
4\\
0
\end{bmatrix}
+ 5 \cdot
\begin{bmatrix}
0\\
0\\
-1\\
2\\
4
\end{bmatrix}
=
\begin{bmatrix}
-1  &   0  &   0\\
 2  &  -1  &   0\\
 4  &   2  &  -1\\
 0  &   4  &   2\\
 0  &   0  &   4
\end{bmatrix}
\cdot
\begin{bmatrix}
3\\1\\5
\end{bmatrix}
=
\begin{bmatrix}
-3\\
 5\\
 9\\
14\\
20
\end{bmatrix}
\end{align}
finden wir den Lösungsweg II mittels \textbf{Toeplitz Matrix} aus Aufgabe~\ref{sec:FD58EEB1EC} (8.2) wieder,
wir sind also erfreulicherweise konsistent mit unserer Betrachtung.


Wir können zyklische Faltung sehr recheneffizient über den DFT-Bildbereich
abwickeln, und damit (wenn wir ordentlich auf die Länge $N_y$ mit Nullen aufgefüllt haben)
auch die lineare Faltung.
%
Dafür benutzen wir die DFT Korrespondenz für die zyklische Faltung im
Zeitbereich vs. Multiplikation im Frequenzbereich
\begin{align}
y[k] = x[k] \circledast_N h[k] \quad\mydft\quad Y[\mu] = X[\mu] \cdot H[\mu],
\end{align}
umgeschrieben
\begin{align}
y[k] = \mathrm{IDFT}_N\{\qquad\mathrm{DFT}_N\{x[k]\} \cdot \mathrm{DFT}_N\{h[k]\}\qquad\}
\end{align}
Für die Faltung sehr langer Folgen ist dieser Umweg über den Bildbereich deutlich
recheneffizienter, als $y[k] = x[k] \circledast_N h[k]$, unter der Voraussetzung,
dass die DFT / IDFT recheneffizient implementiert ist. Dafür gibt es viele Algorithmen,
die als \textbf{Fast Fourier Transform Algorithmen (FFT)} bekannt sind.
%
Dies wird dann als \textbf{schnelle Faltung} bezeichnet. In den Tabellen
ist die direkte und die \textbf{schnelle zyklische Faltung}, sowie
die direkte und \textbf{schnelle
lineare Faltung (aufgrund von Zeropadding)} aufgezeigt mit Matlab
typischen Befehlen (\texttt{.*} ist die elementweise Multiplikation).
%
\end{ExCalc}

%\begin{table}[h]
%\centering
%\begin{tabular}{| l | l | l |}
%\hline
%& zyklische Faltung $N=3$ mit Zeilenvektoren\\\hline
%slow  & \texttt{cconv([-1, 2, 4], [3, 1, 5], 3)} \\\hline
%fast  & \texttt{ifft(fft([-1, 2, 4]) .* fft([3, 1, 5]))}\\\hline
%\end{tabular}
%\end{table}

\begin{table}[h]
\centering
\begin{tabular}{| l | l | l |}
\hline
& zyklische Faltung $N=3$ mit Spaltenvektoren\\\hline
slow  & \texttt{cconv([-1; 2; 4], [3; 1; 5], 3)} \\\hline
fast  & \texttt{ifft(fft([-1; 2; 4]) .* fft([3; 1; 5]))}\\\hline
\end{tabular}
\end{table}



\begin{table}[h]
\centering
\begin{tabular}{| l | l | l |}
\hline
& zyklische Faltung $N=5$ mit Spaltenvektoren (incl. Zeropadding) ==\\\hline
& lineare Faltung $N_x=3$, $N_h=3$, $N_y = N_x+N_h-1=5$ \\\hline
slow & \texttt{cconv([-1; 2; 4; 0; 0], [3; 1; 5; 0; 0], 5)} \\\hline
& \texttt{conv([-1; 2; 4], [3; 1; 5])} \\\hline
fast & \texttt{ifft(fft([-1; 2; 4; 0; 0]) .* fft([3; 1; 5; 0; 0]))} \\\hline
\end{tabular}
\end{table}



\begin{Loesung}
\begin{center}
\begin{tikzpicture}[scale=0.5]
\def\tic{0.15};
\begin{scope}
\fill[C7!30] (-0.5,0) rectangle (2.5,7);
\draw[->] (-4,0) -- (8,0) node[right]{$k$};
\draw[->] (0,-0.5) -- (0,8) node[above]{$y_{cyc}[k] = x[k] \circledast_3 h[k]$};
\foreach \y in {0,...,6}{\draw (\tic,\y) -- (-\tic,\y);};
\draw[stem, C2] plot coordinates{(0,11/4) (1,25/4) (2,9/4)};
\draw[stem, C2] plot coordinates{(0-3,11/4) (1-3,25/4) (2-3,9/4)};
\draw[stem, C2] plot coordinates{(0+3,11/4) (1+3,25/4) (2+3,9/4)};
%
\draw (0,1) node[left]{$4$};
\draw (0,2) node[left]{$8$};
\draw (0,3) node[left]{$12$};
\draw (0,4) node[left]{$16$};
\draw (0,5) node[left]{$20$};
\draw (0,6) node[left]{$24$};
%
\draw (1,0) node[below]{$1$};
\draw (2,0) node[below]{$2$};
%
\draw[C2] (-4,4) node {$\dots$};
\draw[C2] (6,4) node {$\dots$};
%
\end{scope}
%
\begin{scope}[xshift=12cm]
\draw[->] (-2,0) -- (8,0) node[right]{$k$};
\draw[->] (0,-0.5) -- (0,8) node[above]{$y_{lin}[k] = x[k] \ast h[k]$};
\foreach \y in {0,...,6}{\draw (\tic,\y) -- (-\tic,\y);};
\draw[stem, C2] plot coordinates{(-2,0) (-1,0) (0,-3/4) (1,5/4) (2,9/4) (3,14/4) (4,20/4) (5,0) (6,0)};
%
\draw (0,1) node[left]{$4$};
\draw (0,2) node[left]{$8$};
\draw (0,3) node[left]{$12$};
\draw (0,4) node[left]{$16$};
\draw (0,5) node[left]{$20$};
\draw (0,6) node[left]{$24$};
%
\draw (1,0) node[below]{$1$};
\draw (2,0) node[below]{$2$};
%
\end{scope}
\end{tikzpicture}
\end{center}

\begin{align}
&y_{cyc}[k] = 11\delta[k] + 25\delta[k-1] + 9\delta[k-2]\nonumber\\
&y_{cyc}[k] = y_{cyc}[k + 3 m], m\in\mathbb{Z}
\end{align}
%
\begin{align}
y_{lin}[k] = x[k] \ast h[k] = -3\delta[k] + 5\delta[k-1] + 9\delta[k-2] + 14\delta[k-3] + 20\delta[k-4]
\end{align}


\end{Loesung}

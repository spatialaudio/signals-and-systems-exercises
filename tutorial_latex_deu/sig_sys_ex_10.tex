\clearpage
\section{Discrete-Time Fourier Transform (DTFT)}

Die DTFT ist eine von vier Fouriertransformationen, nämlich die FT für
zeitdiskrete Signale, welche nicht notwendigerweise periodisch sind
und für $-\infty \leq k \leq +\infty$ definiert sind.
Diese Signale werden bzgl. der digitalen Kreisfrequenz $\Omega$ analysiert.
%
Die \textbf{Analyseformel} lautet daher
\begin{align}
X(\e^{\im\Omega}) = \sum\limits_{k=-\infty}^{+\infty} x[k] \e^{-\im\Omega k}
\end{align}
%
Damit steht die DTFT in enger Verbindung zur $z$-Transformation
\begin{align}
X(z) = \sum\limits_{k=-\infty}^{+\infty} x[k] z^{-k}
\end{align}
wenn wir für die komplexe Zahl $z$ nur komplexe Zahlen mit Betrag 1 zulassen.
%
Wir erinnern uns an die komplexe Frequenz $z=\e^\Sigma \e^{\im\Omega}$.
%
Die DTFT bildet sich dann als Spezialfall für $\Sigma=0$ ab, also ungedämpfte
komplexe Schwingungen.
%
Damit ist die DTFT als Werkzeug zur Spektralanalyse und Signalsynthese geeignet,
hat aber eher theoretischen Charakter, weil wir unendliche lange Signale
nicht mit einem Computer verarbeiten können.
%
Praktisch relevant sind endliche oder endlich gemachte Folgen.
%
Für diesen Signalausschnitt könnten wir die endliche Summe der DTFT Analyse
ausrechnen, manchmal sogar mit analytischen Lösungen.
%
Wenn wir diesen endlichen Signalausschnitt dann periodisiert auffassen,
gelangen wir zu vierten, uns noch fehlenden Fouriertransformation, der diskreten
Fourier Transformation (DFT).
%
Um jedoch die DFT vollumfänglich zu verstehen, sollten wir zunächst die DTFT
sicher beherrschen.
%
DFT und DTFT sind auch wieder über Abtastung und Interpolation
verknüpft, ähnlich wie wir das für die Fourierreihe und -transformation
kennengelernt hatten.
%
Des weiteren ist die DTFT ein nützliches SigSys-Tool, weil
sie den Frequenzgang von zeitdiskreten Systemen darstellt, also die Auswertung
der Übertragungsfunktion $H(z)$ auf dem Einheitskreis.
%
Wir werden die \textbf{Rücktransformation}
\begin{align}
x[k] = \frac{1}{2\pi} \int\limits_{0}^{2\pi} X(\e^{\im\Omega}) \e^{+\im\Omega k} \fsd \Omega
\end{align}
im Rahmen der Übung mit bekannten Korrespondenzen
\begin{align}
&\delta[k] \dtft 1\\
&1 \dtft \Sha(\frac{\Omega}{2\pi})\\
&\mathrm{rect}_N[k]
\dtft
\e^{-\im\frac{\Omega(N-1)}{2}}\cdot\frac{\sin\left(N\frac{\Omega}{2}\right)}{\sin\left(\frac{\Omega}{2}\right)}\\
&\e^{+\im \Omega_0 k} \dtft \Sha\left(\frac{\Omega-\Omega_0}{2\pi}\right)
\end{align}
meistern, wofür der DTFT-Operator $x[k] \,\dtft\, X(\e^{\im\Omega})$ gilt.
%
Die letzte Korrespondenz bildet Eigenfunktionen für LTI-Systeme in den Bildbereich ab.
Das zugehörige DTFT-Spektrum ist ein Dirac Impuls Kamm und unten beispielhaft skizziert.
%
\begin{center}
\begin{tikzpicture}[scale=0.75]
\begin{scope}
\draw[help lines, C7!50, step=0.5cm] (-4,0) grid (4,1);
\draw[->] (-4 ,0) -- (4.75,0) node[right]{$\Omega$};
\draw[->] (0,-0.1) -- (0,1.75) node[above]{$x[k]=\e^{\im\Omega_0 k} \dtft X(\e^{\im\Omega})=\Sha(\frac{\Omega-\Omega_0}{2\pi})$ für $\Omega_0>0$};
\draw[->, C0, line width=1mm] (0.5,0) node[below]{$\Omega_0$} -- (0.5,1) node[above]{$(2\pi)$};
\draw[->, C0, line width=1mm] (2.5,0) node[below]{$2\pi+\Omega_0$} -- (2.5,1);
\draw[->, C0, line width=1mm] (4.5,0) -- (4.5,1);
\draw[->, C0, line width=1mm] (-1.5,0) -- (-1.5,1);
\draw[->, C0, line width=1mm] (-3.5,0) -- (-3.5,1);
\end{scope}
\end{tikzpicture}
\end{center}
Für $\Omega_0=0$ folgt die Korrespondenz $1 \dtft \Sha(\frac{\Omega}{2\pi})$.

\newpage
\subsection{Faltung komplexe Schwingung mit endlichem Signal}
\label{sec:DF08FCF870}
\begin{Ziel}
Wir wollen anhand eines einfachen LTI Systems mit endlicher Impulsantwort
die Anregung mit einer komplexen Schwingung betrachten und das resultierende
Ausgangssignal im Bildbereich mit der DTFT und im Zeitbereich berechnen und
dabei typische Rechenwege kennenlernen. Für das gewählte
Beispiel ist der Rechenaufwand für Bild- und Zeitbereich
vergleichsweise ähnlich.
%
Wir berechnen wegen der Wahl einer Eigenfunktion als Eingangssignal
am Systemausgang den stationären Zustand. Wir werten also den Frequenzgang
an einer speziellen Frequenz aus, das LTI System ändert nur Amplitude und Phase
des Eingangssignals. Dies finden wir codiert im komplexen Gewichtungsfaktor
im Bildbereich.
\end{Ziel}
\textbf{Aufgabe} {\tiny DF08FCF870}: Falten Sie die beiden Signale
\begin{align}
h[k] =& \frac{1}{2}\delta[k] - \frac{1}{2}\delta[k-2]\\
x[k] =& \e^{+\im \frac{\pi}{4} k}
\end{align}
über den Umweg des DTFT-Bildbereichs und verifizieren Sie das Ergebnis
mit der Faltung im Zeitbereich.

\begin{Werkzeug}
DTFT Hin/Rücktrafo, Faltung vs. Multiplikation, Dirac Impuls Kamm,
Austast- und Multiplikationseigenschaft Dirac Impuls, Dualität:
zeitliche Verschiebung = Phasenverschiebung im Spektrum und anders rum
\end{Werkzeug}
\begin{Ansatz}
Umweg über Bildbereich:
\begin{align}
h[k]\dtft H(\e^{\im\Omega}),\quad x[k]\dtft X(\e^{\im\Omega}),\quad
Y(\e^{\im\Omega}) = H(\e^{\im\Omega}) \cdot X(\e^{\im\Omega}),\quad
y[k] \DTFT Y(\e^{\im\Omega})
\end{align}
Zeitbereich direkt:
\begin{align}
y[k] = \sum_{\kappa=-\infty}^{+\infty} x[-\kappa + k] \cdot h[\kappa]
\end{align}
\end{Ansatz}
\begin{ExCalc}
\textbf{Bildbereich Hin-/Rücktrafo:}
\begin{align}
h[k] = \frac{1}{2}\delta[k] - \frac{1}{2}\delta[k-2]\quad\dtft\quad
H(\e^{\im\Omega}) = \frac{1}{2} - \frac{1}{2} \e^{-\im 2 \Omega}\\
x[k]=\e^{+\im \frac{\pi}{4} k} \quad\dtft\quad
X(\e^{\im\Omega}) = \Sha(\frac{\Omega-\frac{\pi}{4}}{2\pi})
\end{align}
%
\begin{align}
Y(\e^{\im\Omega}) = H(\e^{\im\Omega}) \cdot X(\e^{\im\Omega})\\
Y(\e^{\im\Omega}) = \left(\frac{1}{2} - \frac{1}{2} \e^{-\im 2 \Omega}\right)
\cdot
\Sha(\frac{\Omega-\frac{\pi}{4}}{2\pi})
\end{align}
%
Jetzt wenden wir die Multiplikationseigenschaft des Dirac Impulses
\begin{align}
x[k] \, \delta[k-\kappa] = x[\kappa] \, \delta[k-\kappa]
\end{align}
an (hier allgemein geschrieben für zeitdiskrete Signale, gilt ja aber auch im
Bildbereich).
%
Genau genommen müssen wir diese Multiplikationseigenschaft nun für alle Dirac Impulse
im Dirac Impulskamm anwenden (eine Skizze zur Lage der Impulse bzgl. der Frequenz
hilft vielleicht).
%
Wir finden dann, dass für genau die $\Omega=\frac{\pi}{4}\pm\nu 2\pi$ für $\nu\in\mathbb{N}$
die Multiplikationseigenschaft 'zuschlägt':
\begin{align}
Y(\e^{\im\Omega}) = \left(\frac{1}{2} - \frac{1}{2} \e^{-\im 2 \Omega}\right)
\cdot
\Sha(\frac{\Omega-\frac{\pi}{4}}{2\pi})
=
Y(\e^{\im\Omega}) = \left(\frac{1}{2} - \frac{1}{2} \e^{-\im 2 \Omega}\right)\bigg|_{\Omega=\frac{\pi}{4}\pm\nu 2\pi}
\cdot
\Sha(\frac{\Omega-\frac{\pi}{4}}{2\pi})
\end{align}
Das lässt sich nun vereinfachen, weil die $\e^{\pm \nu \im 4 \pi}=1$ Vieldeutigkeit
im komplexen Dreher nicht mehr weiter berücksichtigt werden muss
\begin{align}
Y(\e^{\im\Omega}) = \left(\frac{1}{2} - \frac{1}{2} \e^{-\im \frac{\pi}{2}}\right)
\cdot
\Sha(\frac{\Omega-\frac{\pi}{4}}{2\pi}) =
\left(\frac{1}{2} + \frac{1}{2} \im \right)
\cdot
\Sha(\frac{\Omega-\frac{\pi}{4}}{2\pi}) =
\frac{1}{\sqrt{2}} \e^{+\im\frac{\pi}{4}}\cdot\Sha(\frac{\Omega-\frac{\pi}{4}}{2\pi})
\end{align}
Die Rücktransformation, also inverse DTFT, ist nun vergleichsweise wenig Aufwand.
%
Der Faktor $\frac{1}{\sqrt{2}} \e^{+\im\frac{\pi}{4}}$ bleibt wegen Linearitätseigenschaft
der DTFT im Zeitbereich erhalten, das Zeitsignal zu $\Sha(\frac{\Omega-\frac{\pi}{4}}{2\pi})$
kennen wir vom Beginn unserer Rechnung
\begin{align}
Y(\e^{\im\Omega}) \quad\DTFT\quad y[k] = \frac{1}{\sqrt{2}} \e^{+\im\frac{\pi}{4}} \cdot \e^{+\im \frac{\pi}{4} k}=
\frac{1}{\sqrt{2}} \e^{+\im \frac{\pi}{4} (k+1)} = \frac{1}{\sqrt{2}} x[k+1].
\end{align}

\textbf{Zeitbereich:}
Nun wollen wir explizit die Faltungssumme
\begin{align}
y[k] = \sum_{\kappa=-\infty}^{+\infty} x[-\kappa + k] \cdot h[\kappa]
\end{align}
ausrechnen, $\kappa$ ist unsere Hilfsvariable bzgl. zeitlicher Spiegelung und
Verschiebung.
%
Die Faltung ist auch in der zeitdiskreten Welt kommutativ, also gilt auch
\begin{align}
y[k] = \sum_{\kappa=-\infty}^{+\infty} x[\kappa] \cdot h[-\kappa + k].
\end{align}
%
Erstere Variante macht sich mutmaßlich eleganter, weil in den Dirac Argumenten
schneller zu sehen ist, was bei der Austast-/Ausblendeigenschaft passiert
(Achtung englisch:
\textit{sifting property}, nicht \textit{shifting property}, das kann bei schlampigem
Lesen zu Verwirrung führen).
%
Faltungssumme also für unsere Aufgabe
%
\begin{align}
y[k] = \sum_{\kappa=-\infty}^{+\infty} \e^{+\im\frac{\pi}{4} (-\kappa+k)} \cdot
\left(\frac{1}{2}\delta[\kappa] - \frac{1}{2}\delta[\kappa-2]\right).
\end{align}
%
Schreiben wir die Terme auseinander
\begin{align}
y[k] =
\sum_{\kappa=-\infty}^{+\infty} \e^{+\im\frac{\pi}{4} (-\kappa+k)} \cdot \frac{1}{2}\delta[\kappa]
-
\sum_{\kappa=-\infty}^{+\infty} \e^{+\im\frac{\pi}{4} (-\kappa+k)} \cdot \frac{1}{2}\delta[\kappa-2]
\end{align}
damit wir die Austasteigenschaft besser überschauen.
%
Es verbleiben nach ihrer Anwendung erwartungsgemäß genau zwei Terme (nämlich die für
$\kappa=2$ und $\kappa=0$)
\begin{align}
y[k] =
\frac{1}{2} \e^{+\im\frac{\pi}{4} (-0+k)}
-
\frac{1}{2} \e^{+\im\frac{\pi}{4} (-(+2)+k)},
\end{align}
welche sich vereinfachen lassen zu
\begin{align}
y[k] =
\frac{1}{2} \e^{+\im\frac{\pi}{4} k}
-
\frac{1}{2} \e^{+\im\frac{\pi}{4} k} \e^{-\im\frac{\pi}{2}} =
(\frac{1}{2} + \frac{1}{2} \im )\e^{+\im\frac{\pi}{4} k}
=
\frac{1}{\sqrt{2}} \e^{+\im \frac{\pi}{4} (k+1)}=
\frac{1}{\sqrt{2}} x[k+1].
\end{align}
Wir bestätigen damit unser Ergebnis aus der Bildbereichsrechnung.

In \fig{fig:DF08FCF870} ist der Realteil der Ein- und Ausgangsfolge vergleichend
in einer Grafik dargestellt. Quizfrage für zu Hause: wie schauen die Imaginärteile
für Ein- und Ausgangssignal aus?
\end{ExCalc}
\begin{Loesung}
Wenn wir $h[k]$ als Impulsantwort eines nicht-rekursiven Systems auffassen
(Detailinfo: das ist ein linearphasiges FIR Filter vom Typ III, weil punktsymmetrisch um $k=1$)
und $x[k]$ als Eigenfunktion für zeitdiskrete LTI-Systeme identifizieren
(das ist eine harmonische und sogar in $N=8$ periodisch schwingende komplexe
Exponentialfolge), wissen wir, dass wir den stationären Zustand des Systems
für genau diese Frequenz ausrechnen.
%
LTI-Systeme ändern nur Betrag und Phase für Eigenfunktionen im stationären Zustand,
alle eventuellen Anfangsbedingungen sind ja abgeklungen.
%
Im DTFT-Bildbereich ist Amplitude und Phase im komplexen Faktor
$\frac{1}{\sqrt{2}} \e^{+\im\frac{\pi}{4}}$
codiert.
%
Im gewählten Beispiel dämpft das System die Amplitude auf $\frac{1}{\sqrt{2}}$,
also -3.01\,dB in Pegeldarstellung.
%
Die Phasenänderung ist für die betrachtete Frequenz $+\frac{\pi}{4}$.
Nachdem dies auch genau der digitalen Kreisfrequenz des Signals entspricht,
ist die Ausgangsfolge
genau um einen Folgenindex voreilend gegenüber der Eingangsfolge, daher der
Zeitshift $k+1$.
%

Wir könnten herausfinden, dass die Faltung mit $h[k]$ für die komplexen
Schwingungen $\e^{\im 0 k}$ (Gleichanteil) und $\e^{\im \pi k}$ ($\pm 1$-Folge,
entspricht halber Abtastfrequenz) Ausgangsfolgen $y[k]=0$ hervorbringt.
%
Wenn wir diesen Fakt verknüpfen können mit der Systemcharakterisierung
im $z$-Bereich (wo liegen die Nullstellen des Systems $H(z) \ztransf h[k]$?,
wie sieht der Betragsfrequenzgang aus?),
dann wissen wir die erlernten Werkzeuge genau richtig einzusetzen.
%
Wir könnten ein Blockschaltbild malen, um die Differenzengleichung die hinter
der Impulsantwort steckt, zu verinnerlichen.
%
Es ist ein nicht-rekursives System mit den Koeffizienten $b_0=+\frac{1}{2}$, $b_1=0$,
$b_2=-\frac{1}{2}$ und $a_0=1$.
%
Mit Matlab könnten wir uns dann mit dem Befehl \texttt{fvtool([1/2,0,-1/2],[1,0,0])}
Computerunterstützung zur Systemanalyse holen.
\end{Loesung}

\begin{figure}
\centering
\includegraphics[width=0.75\textwidth]{../dtft/fir_filter_steady_state_DF08FCF870.pdf}
\caption{Realteil der Ein- und Ausgangsfolgen für Aufgabe \ref{sec:DF08FCF870}.
Hier ausnahmsweise nicht als korrekter stem Plot, um die Voreilung des Ausgangs
um einer Folgenindex deutlicher zu machen. Ausgangsamplitude ist Faktor
$1/\sqrt{2}$ kleiner als Eingangsamplitude.
\texttt{fir\_filter\_steady\_state\_DF08FCF870.ipynb}}
\label{fig:DF08FCF870}
\end{figure}



\clearpage
\subsection{Systemdiskussion für nicht-rekursives System / FIR Filter}
\label{sec:64BE42BAEF}
\begin{Ziel}
Wir wollen die spezielle und für zeitdiskrete SigSys sehr wichtige
Systemklasse, nämlich sogenannte Finite Impulse Response (FIR) Filter, also
Systeme mit endlicher Impulsantwort vertiefen.
Dazu werden wir für ein nicht-rekursives System eine typische Systemdiskussion
im Zeit- und Bildbereich durchspielen.
Um den Frequenzgang auszurechnen, werden wir die DTFT benutzen, im Wesen
entspricht dies der Auswertung der $z$-Übertragungsfunktion auf dem Einheitskreis.
\end{Ziel}
\textbf{Aufgabe} {\tiny 64BE42BAEF}: Gegeben ist das Pol-Nullstellen-Diagramm
eines zeitdiskreten, kausalen Systems.
\begin{center}
\begin{tikzpicture}[scale=1.5]
\def \tic {0.05}
\def \pabs{0} % largest radius for poles to determine roc-> here a FIR
\def \rocmax{1.6} % sketch of outer roc domain
%
% basic diagram features:
\filldraw[even odd rule,C2!50] (0,0) circle(\pabs) decorate
[decoration={snake, segment length=15pt, amplitude=1pt}]
{(0,-3pt) circle(\rocmax)}; % sketch the roc domain
%
\draw[help lines, C7!50, step=0.25cm] (-\rocmax,-\rocmax) grid (\rocmax,\rocmax);
%
\draw[C3, thick] (0,0) circle(1);  % unit circle, i.e. DTFT domain
\draw[C7, thin] (0,0) circle(1.4142);  % unit circle, i.e. DTFT domain
\draw[C7, thin] (0,0) circle(0.7071);  % unit circle, i.e. DTFT domain
%
\draw (1+2*\tic,-2*\tic) node{$1$}; % indicate that this is the unit circle
\draw (0.5,-2*\tic) node{$\nicefrac{1}{2}$};
\draw[->] (-1.75,0)--(1.75,0) node[right]{$\Re\{z\}$}; % axis label
\draw[->] (0,-1.75)--(0,1.75) node[above]{$\Im\{z\}$}; % axis label
%
\draw (1.2*0.86602540378*\rocmax,1.2*0.5*\rocmax) node[C2!75]{KB}; % indicate the roc
%
% the z-transfer function specific stuff:
\draw (0.75,1.25) node[black]{$g=1$}; % indicate gain factor
%
% draw the poles / zeros and if desired ticks:
\draw[C0, ultra thick] (0,0) node{\Huge $\times$};
\draw[C0] (-4*\tic,4*\tic) node{10};
\draw[C0, ultra thick] (0,+1) node{\Huge $\circ$};
\draw[C0, ultra thick] (0,-1) node{\Huge $\circ$};
\draw[C0, ultra thick] (+1,+1) node{\Huge $\circ$};
\draw[C0, ultra thick] (-1,-1) node{\Huge $\circ$};
\draw[C0, ultra thick] (-1,+1) node{\Huge $\circ$};
\draw[C0, ultra thick] (+1,-1) node{\Huge $\circ$};
\draw[C0, ultra thick] (+1/2,+1/2) node{\Huge $\circ$};
\draw[C0, ultra thick] (-1/2,-1/2) node{\Huge $\circ$};
\draw[C0, ultra thick] (-1/2,+1/2) node{\Huge $\circ$};
\draw[C0, ultra thick] (+1/2,-1/2) node{\Huge $\circ$};
\end{tikzpicture}
\end{center}
\begin{itemize}
\setlength\itemsep{-0.25em}
\item[a)] Geben Sie die $z$-Transformierte $H(z)$ an.
\item[b)] Geben Sie die Impulsantwort $h[k] \quad\Ztransf\quad H(z)$ an.
\item[c)] Skizzieren Sie das Signalflussdiagramm $x[k] \rightarrow y[k]$.
\item[d)] Geben Sie die Sprungantwort $h_\epsilon[k]$ an.
\item[e)] Geben Sie an, woran man sieht, dass dieses System stabil ist.
\item[f)] Geben Sie den Betragsfrequenzgang analytisch an und skizzieren diesen über $\Omega$.
\item[g)] Geben Sie den Phasenfrequenzgang analytisch an und skizzieren diesen über $\Omega$.
\item[h)] Geben Sie die Gruppenlaufzeit analytisch an und skizzieren diese über $\Omega$.
\end{itemize}

\begin{Werkzeug}
$z$-Transformation, DTFT, komplexe Zahlen, Pol/Nullstellendarstellung, Blockschaltbild
\end{Werkzeug}
\begin{Ansatz}
Wir müssen zunächst aus dem
Pol-Nullstellen-Diagramm auf die Übertragungsfunktion $H(z)$ schließen. Die
restlichen Aufgaben sind dann lösbar mit bekanntem SigSys Werkzeug.
\end{Ansatz}
%
\begin{ExCalc}
%
\textbf{a) z-Transformierte}
%
Wir haben zehn Nullstellen aufgeteilt in fünf konjugiert-komplexe Paare
\begin{align}
z_{0,1,2} =& \pm \im = 1 \cdot \e^{\pm \im \frac{\pi}{2}}\\
z_{0,3,4} =& +1 \pm \im =  \sqrt{2} \cdot \e^{\pm \im \frac{\pi}{4}}\\
z_{0,5,6} =& -1 \pm \im = \sqrt{2} \cdot \e^{\pm \im \frac{3\pi}{4}}\\
z_{0,7,8} =& +\frac{1}{2} \pm \frac{1}{2}\im = \frac{1}{\sqrt{2}} \cdot \e^{\pm \im \frac{\pi}{4}}\\
z_{0,9,10} =& -\frac{1}{2} \pm \frac{1}{2}\im = \frac{1}{\sqrt{2}} \cdot \e^{\pm \im \frac{3\pi}{4}}
\end{align}
und zehn Pole, alle im Ursprung (die daher für alle Frequenzen $|z|=1$ auf dem
Einheitskreis den gleichen Einfluss haben)
\begin{align}
z_{\infty,1,...,10} = 0
\end{align}
Polynom im Zähler und Nenner aus den NST und Polen aufstellen (Reihenschaltung)
\begin{align}
H(z) = g\cdot
\frac
{\prod\limits_{n=1}^{10} (z-z_{0,n})}
{\prod\limits_{m=1}^{10} (z-z_{\infty,m})} =
\frac{
(z-\im)\cdot
(z+\im)\cdot
(z-(1 + \im))\cdot
(z-(1 - \im))\cdot
(z-(-1 + \im))\cdot
(z-(-1 - \im))\cdot
\dots
}{z^{10}}
\end{align}
und etwas mühsam ausmultiplizieren, bis wir die äquivalenten Darstellungen
\begin{align}
H(z) =& \frac{z^{10} + z^8    + \nicefrac{17}{4}\cdot z^6     + \nicefrac{17}{4}\cdot z^4    + z^2    + 1}{z^{10}}\\
H(z) =& 1            + z^{-2} + \nicefrac{17}{4}\cdot z^{-4}  + \nicefrac{17}{4}\cdot z^{-6} + z^{-8} + z^{-10}
\end{align}
finden.
%
Die Wahl dieser Pole und Nullstelle führt auf schöne Koeffizienten in
$H(z)$. Dies ist ein seltenes Beispiel, in dem sowohl die Winkel der Nullstellen
als auch die Koeffizienten 'schön' sind.
%
Praktische System werden eher 'krumme' Zahlen zurückliefern.
%

\textbf{b) Impulsantwort}
Wir sehen $H(z)$ an, dass das System keine Rückkopplung des Ausgangs zurück in
das System aufweist, also ein nicht-rekursives System ist, also ganz sicher
eine endliche Impulsantwort aufweisen wird.
%
Zeigen wir das in aller Ausführlichkeit: Aus
\begin{align}
H(z) = \frac{Y(z)}{X(z)}
= 1            + z^{-2} + \nicefrac{17}{4}\cdot z^{-4}  + \nicefrac{17}{4}\cdot z^{-6} + z^{-8} + z^{-10}
\end{align}
können wir umstellen zu
\begin{align}
1 \cdot Y(z)
= (1            + z^{-2} + \nicefrac{17}{4}\cdot z^{-4}  + \nicefrac{17}{4}\cdot z^{-6} + z^{-8} + z^{-10})
\cdot X(z),
\end{align}
und sehen, dass für $Y(z)$ nur der Term $z^0=1$ als Gewicht existiert, der gleichzeitig
auch der Koeffizient $a_0=1$ ist. Also keine Rückkopplung, damit nicht-rekursiv.
%
Für ein nicht-rekursives System ist die Impulsantwort und Differenzengleichung
vergleichsweise schnell ermittelt.
%
Wir benutzen den wichtigen Zusammenhang (Zeit vs. Phasenverschiebung)
\begin{align}
x[k-\kappa] \quad\ztransf\quad X(z) \cdot z^{-\kappa}
\end{align}
um zur \textbf{Differenzengleichung} zurückzutransformieren
\begin{align}
y[k] = x[k] + x[k-2] + \nicefrac{17}{4} x[k-4] + \nicefrac{17}{4} x[k-6] + x[k-8] + x[k-10]
\end{align}
Die \textbf{Impulsantwort} erhalten wir durch Rücktransformation von $H(z)$, also
\begin{align}
h[k] = \delta[k] + \delta[k-2] + \nicefrac{17}{4} \delta[k-4] + \nicefrac{17}{4} \delta[k-6] + \delta[k-8] + \delta[k-10],
\end{align}
weil der Verschiebungssatz speziell für den Dirac Impuls
\begin{align}
\delta[k-\kappa] \quad\ztransf\quad 1 \cdot z^{-\kappa}
\end{align}
lautet. Sie ist unten skizziert, es handelt sich um ein sogenanntes
\textbf{linearphasiges FIR-Filter vom Typ I}, vgl. \cite{Oppenheim2010, Oppenheim2004}.
%
\begin{center}
\begin{tikzpicture}[scale=1]
\def\tic{0.1};
\draw[->] (-1.5,0) -- (11.5,0) node[right]{$k$};
\draw[->] (0,0) -- (0,3) node[above]{$h[k]$};
\foreach \x in {-1,0,...,11}{\draw (\x,+\tic) -- (\x,-\tic)  node[below]{$\x$};};
\foreach \y in {1,2}{\draw (\tic,\y) -- (-\tic,\y)  node[left]{$\y$};};
%\foreach \k in {-1,0,...,8}{\draw[stem2,C1] plot coordinates{(\k,{cos(\Om*\k)})};};
\draw[stem2, C0] plot coordinates{(-1,0) (0,1) (1,0) (2,1) (3,0) (4,17/4) (5,0) (6,17/4) (7,0) (8,1) (9,0) (10,1) (11,0)};
\end{tikzpicture}
\end{center}
%
Wenn wir uns $H(z)$ (die Version mit negativen Exponenten in $z$), die DGL und
die Impulsantwort anschauen, sehen wir wie die konstanten Koeffizienten
sich im Bild- und Zeitbereich jeweils abbilden (mit ein wenig Übung geht das
irgendwann im Kopf):
\begin{align}
H(z) =& 1 +& 1 \cdot z^{-2} +&& \nicefrac{17}{4} \cdot z^{-4}  +&& \nicefrac{17}{4} \cdot z^{-6} +&& 1 \cdot z^{-8} +& 1 \cdot z^{-10}\nonumber\\
h[k] =& 1 \cdot \delta[k] +& 1 \cdot \delta[k-2] +&& \nicefrac{17}{4} \cdot \delta[k-4] +&& \nicefrac{17}{4} \cdot \delta[k-6] +&& 1 \cdot \delta[k-8] +& 1 \cdot \delta[k-10]\nonumber\\
y[k] =& 1\cdot x[k] +& 1\cdot x[k-2] +&& \nicefrac{17}{4}\cdot x[k-4] +&& \nicefrac{17}{4}\cdot x[k-6] +&& 1\cdot x[k-8] +& 1\cdot x[k-10]
\end{align}
Wenn wir einen Dirac Impuls in die Differenzengleichung geben, erhalten wir
die Impulsantwort. Das gilt allgemein. Hier, weil wir ein nicht-rekursives
System betrachten, haben wir einen überaus nützlichen Spezialfall:
die Impulsantwort ist endlich und wir können die
DGL direkt nehmen und $y[k]\rightarrow h[k]$ und $x[k]\rightarrow \delta[k]$ usw.
ersetzen. Wenn der Dirac durch alle Zustandsspeicher
$x[k-\kappa]$ durchgelaufen ist, 'verstummt' das System.
%
In der Formel zu $h[k]$ sehen wir das direkt, der letzte Eintrag ist $\delta[k-10]$,
daher ist für $k>10$ die Impulsantwort $h[k>10]=0$.
%
Nicht-rekursive Systeme sind daher immer BIBO-stabil, dem System wird es mit
endlicher Impulsantwort nicht gelingen aus einem bounded input signal ein
unbounded output signal zu erzeugen.
%
Wir verdeutlichen uns das anhand
des Signalflussgraphen der Differenzengleichung im Folgenden.
%

\textbf{c) Signalflussdiagramm}
\red{TBD: Grafik schöner machen}
% make this one nice, code is crap !!!!!s
\begin{center}
\begin{tikzpicture}[thick, scale=0.25]
\tikzstyle{filtBlock} = [draw, rectangle, minimum height=2em, minimum width=2em,anchor=center]
\tikzstyle{filtBranch}=[fill,shape=circle,minimum size=4pt,inner sep=0pt,anchor=center]
\tikzstyle{filtSum} = [draw, circle, inner sep=1pt,node distance=0.8cm,anchor=center]
\newcommand{\filtDelay}{$z^{-1}$}
% FIR
\begin{scope}
\matrix[row sep=2mm, column sep=0.05mm](m1)
{
\node (input){$x[k]$}; &
\node (spliti)[coordinate] {};
& & \node (join1) [filtSum]{$+$};
& & \node (join2) [filtSum]{$+$};
& & \node (join3) [filtSum]{$+$};
& & \node (join4) [filtSum]{$+$};
& & \node (join5) [filtSum]{$+$};
& & \node (join6) [filtSum]{$+$};
& & \node (join7) [filtSum]{$+$};
& & \node (join8) [filtSum]{$+$};
& & \node (join9) [filtSum]{$+$};
& & \node (join10) [filtSum]{$+$};
&\node (output) {$y[k]$}; \\
\node (dc){$ $}; &
\node (splito)[coordinate] {};
& & \node (b1) [filtBlock]{$0$};
& & \node (b2) [filtBlock]{$1$};
& & \node (b3) [filtBlock]{$0$};
& & \node (b4) [filtBlock]{$\nicefrac{17}{4}$};
& & \node (b5) [filtBlock]{$0$};
& & \node (b6) [filtBlock]{$\nicefrac{17}{4}$};
& & \node (b7) [filtBlock]{$0$};
& & \node (b8) [filtBlock]{$1$};
& & \node (b9) [filtBlock]{$0$};
& & \node (b10) [filtBlock]{$1$};
\\
& \node (split0)[coordinate]{}; & \node (delay1)[filtBlock]{\filtDelay};
& \node (split1)[coordinate]{}; & \node (delay2)[filtBlock]{\filtDelay};
& \node (split2)[coordinate]{}; & \node (delay3)[filtBlock]{\filtDelay};
& \node (split3)[coordinate]{}; & \node (delay4)[filtBlock]{\filtDelay};
& \node (split4)[coordinate]{}; & \node (delay5)[filtBlock]{\filtDelay};
& \node (split5)[coordinate]{}; & \node (delay6)[filtBlock]{\filtDelay};
& \node (split6)[coordinate]{}; & \node (delay7)[filtBlock]{\filtDelay};
& \node (split7)[coordinate]{}; & \node (delay8)[filtBlock]{\filtDelay};
& \node (split8)[coordinate]{}; & \node (delay9)[filtBlock]{\filtDelay};
& \node (split9)[coordinate]{}; & \node (delay10)[filtBlock]{\filtDelay};
& \node (split10)[coordinate]{}; &
&&
&&
&&
&&
&;\\
};
\end{scope}
\begin{scope}[start chain=thru]
\chainin (input) [join = by ->];
%\chainin (spliti) [join]; %
\chainin (delay1)[join = by ->];
\chainin (split1)[join];
{[start filtBranch=ff1]
\chainin (b1) [join=by ->];
\chainin (join1) [join=by ->];
};
\chainin (delay2) [join=with split1 by ->];
\chainin (split2)[join];
{[start filtBranch=ff2]
\chainin (b2) [join=by ->];
\chainin (join2) [join=by ->];
};
\chainin (delay3) [join=with split2 by ->];
\chainin (split3)[join];
{[start filtBranch=ff3]
\chainin (b3) [join=by ->];
\chainin (join3) [join=by ->];
};
\chainin (delay4) [join=with split3 by ->];
\chainin (split4)[join];
{[start filtBranch=ff4]
\chainin (b4) [join=by ->];
\chainin (join4) [join=by ->];
};
%
\chainin (delay5) [join=with split4 by ->];
\chainin (split5)[join];
{[start filtBranch=ff5]
\chainin (b5) [join=by ->];
\chainin (join5) [join=by ->];
};
%
\chainin (delay6) [join=with split5 by ->];
\chainin (split6)[join];
{[start filtBranch=ff6]
\chainin (b6) [join=by ->];
\chainin (join6) [join=by ->];
};
%
\chainin (delay7) [join=with split6 by ->];
\chainin (split7)[join];
{[start filtBranch=ff7]
\chainin (b7) [join=by ->];
\chainin (join7) [join=by ->];
};
%
\chainin (delay8) [join=with split7 by ->];
\chainin (split8)[join];
{[start filtBranch=ff8]
\chainin (b8) [join=by ->];
\chainin (join8) [join=by ->];
};
\chainin (delay9) [join=with split8 by ->];
\chainin (split9)[join];
{[start filtBranch=ff9]
\chainin (b9) [join=by ->];
\chainin (join9) [join=by ->];
};
\chainin (delay10) [join=with split9 by ->];
\chainin (split10)[join];
{[start filtBranch=ff10]
\chainin (b10) [join=by ->];
%\chainin (join10) [join=by ->];
};
%};
\chainin (join1) [join=with spliti by ->];
\chainin (join2) [join=by ->];
\chainin (join3) [join=by ->];
\chainin (join4) [join=by ->];
\chainin (join5) [join=by ->];
\chainin (join6) [join=by ->];
\chainin (join7) [join=by ->];
\chainin (join8) [join=by ->];
\chainin (join9) [join=by ->];
\chainin (join10) [join=with join9 by ->, join=with b10 by ->];
\chainin (output) [join=by ->];
\end{scope}
\end{tikzpicture}
\end{center}

Wir können den Dirac durch das Signalflussdiagramm durchschieben.
Zu $k=0$ liegt die Eins vom Dirac am Eingang an, für $k=1$ ist dieses Eins
'gespeichert' im ersten $z^{-1}$ Block (Verzögerung um 1 Sample), für $k=2$ im zweiten $z^{-1}$ Block
usw. bis wir bei $k=10$ nur noch das Signal $x[k-10]$ im System haben, weil
$z^-10$. Ab $k=11$ ist das System in Ruhe.

\textbf{d) Sprungantwort}
Wir könnten im Bildbereich rechnen und dann zurücktransformieren, also
\begin{align}
h_\epsilon[k] = h[k]\ast_k \epsilon[k] \quad\ztransf\quad
H(z) \cdot \frac{z}{z-1} \quad\Ztransf\quad h_\epsilon[k].
\end{align}
Die für die Rücktransformation notwendige Polynomdivision ist aber
vergleichsweise umständlich, bei FIR Systemen müssen wir das eigentlich nie so
machen.
%
Die Lösung findet sich im Zeitbereich einfacher, weil wir
\begin{align}
h_\epsilon[k] = \sum_{\kappa=-\infty}^{k} h[\kappa],
\end{align}
als kumulierte Summe der Impulsantwort kennen. Nachdem letztere endlich ist,
ist auch die kumulierte Summe überschaubar und wegen schöner Zahlen in $h[k]$
bekommen wir hier auch schöne Zahlen in $h_\epsilon[k]$, also
\begin{align}
h_\epsilon[k] =
&\delta[k]
+\delta[k-1]
+2\cdot\delta[k-2]
+2\cdot\delta[k-3]\nonumber\\
&+6\nicefrac{1}{4}\cdot\delta[k-4]
+6\nicefrac{1}{4}\cdot\delta[k-5]\nonumber\\
&+10\nicefrac{1}{2}\cdot\delta[k-6]
+10\nicefrac{1}{2}\cdot\delta[k-7]\nonumber\\
&+11\nicefrac{1}{2}\cdot\delta[k-8]
+11\nicefrac{1}{2}\cdot\delta[k-9]\nonumber\\
&+12\nicefrac{1}{2}\sum_{\kappa=10}^{\infty}\delta[k-\kappa]
\end{align}
Bei nicht-rekursiven Systemen nähert sich die Sprungantwort nicht
asymptotisch einem Endwert für $\kappa\to\infty$, sondern der Wert ist exakt,
eben weil die kumulierte Summe endlich ist. In unserem Fall haben wir für
$h[k\geq 10] = 12\nicefrac{1}{2}$.

\textbf{Stabilität}
Merksatz: Nicht-rekursive System sind immer stabil!

Das erkennen wir, weil
\begin{itemize}
  \item Konvergenzbereich ist \textbf{immer} $|z|>0$ (wird ja durch die Pole im Ursprung bestimmt), umschließt also immer den Einheitskreis
  \item Nicht-rekursive Systeme haben \textbf{immer} eine endliche Impulsantwort, d.h. es kann keinen zeitlich unbeschränkt aufklingenden Verlauf geben
  \item salopp: keine Rückkopplung bedeutet kein Stress (warum macht man dann nicht alles mit FIR?!?...wegen Vor- und Nachteilen...Buch bzw. Mastervorlesung Digitale Signalverarbeitung...)
\end{itemize}



\end{ExCalc}
\begin{Loesung}
\end{Loesung}

\textbf{f) / g) Betrags- und Phasenfrequenzgang}
Es ist sinnvoll, wenn wir wieder einen 'Trick' benutzen, um die Rechnung wesentlich
entspannter zu machen. Ohne diesen Trick, wäre die Rechnung zwar ähnlich durchführbar,
aber sie artet deutlich mehr aus und ohne eine Idee wohin wir rechnen sollten,
ist das fast aussichtslos bzw. bedarf sehr viel SigSys Erfahrung (bitte nicht nach
der Musterlösung fragen, der Autor hat es selbst nicht bis ins letzte Detail
durchgerechnet bzw. schön aufgeschrieben ;-), es geht aber).
%
Der Trick ist das System $H(z)$ in eine Reihenschaltung zu zerlegen in a)
ein Teilsystem mit verschobenem Dirac welches \textbf{linearphasig} ist
und b) ein Teilsystem was \textbf{nullphasig} ist. Wir werden später sehen,
was diese Begriff meinen, aber genau die Eigenschaft der induzierten
Nullphasigkeit erleichtert die folgende Rechnerei enorm.
%
Wir machen den Ansatz
\begin{align}
H(z) = \underbrace{H(z) \cdot z^5}_{H_0(z)} \cdot \underbrace{z^{-5}}_{H_d(z)}= &
\underbrace{\left(1 + z^{-2} + \nicefrac{17}{4}\cdot z^{-4} + \nicefrac{17}{4}\cdot z^{-6} + z^{-8} + z^{-10}\right)
\cdot z^5}_{H_0(z)} \cdot \underbrace{z^{-5}}_{H_d(z)}
\end{align}
wobei $H_0(z)$ das Nullphasensystem ist (deswegen Index $0$) und $H_d(z)$ die
Dirac-Verschiebung (i.e. Delay, deswegen der Index $d$).
Wir machen uns klar, dass zu dieser Reihenschaltung die Faltung
\begin{align}
h[k] = h_0[k] \ast_k h_d[k]
\end{align}
gehört.
%
Die Impulsantwort und Übertragungsfunktion der Teilsysteme lautet also
\begin{align}
H_d(z) =& z^{-5} \quad\Ztransf\quad h_d(k) = \delta[k-5]
\end{align}
\begin{align}
&H_0(z) = z^5 + z^3 + \nicefrac{17}{4}\cdot z^{+1} + \nicefrac{17}{4}\cdot z^{-1} + z^{-3} + z^{-5}
\quad\Ztransf\quad\nonumber\\
&h_0[k] = \delta[k+5] + \delta[k+3] + \nicefrac{17}{4}\cdot\delta[k+1] + \nicefrac{17}{4}\cdot\delta[k-1] + \delta[k-3] + \delta[k-5]
\end{align}
%
Das System $h_0[k]$ ist nun \textbf{axialsymmetrisch} bzgl. $k=0$, das war intentional, weil
eben genau diese Axialsymmetrie die Charaktereigenschaft \textbf{Nullphasigkeit}
nach sich zieht. Es ist damit aber auch ein nicht kausales System, weil die Impulsantwort
links- und rechtsseitig, also beidseitig ist. Das stört uns hier aber nicht, weil
wir ja eher rechnen und das System nicht unbedingt realisieren wollen.
%
\begin{center}
\begin{tikzpicture}[scale=1]
\def\tic{0.1};
\draw[->] (-6.5,0) -- (6.5,0) node[right]{$k$};
\draw[->] (0,0) -- (0,17/4) node[above]{$h_0[k]$};
\foreach \x in {-5,-3,...,6}{\draw (\x,+\tic) -- (\x,-\tic)  node[below]{$\x$};};
\foreach \y in {1,2,...,4}{\draw (\tic,\y) -- (-\tic,\y)  node[left]{$\y$};};
%\foreach \k in {-1,0,...,8}{\draw[stem2,C1] plot coordinates{(\k,{cos(\Om*\k)})};};
\draw[stem2, C0] plot coordinates{(-1,0) (-5,1) (-4,0) (-3,1) (-2,0) (-1,17/4) (0,0) (1,17/4) (2,0) (3,1) (4,0) (5,1) (6,0)};
\end{tikzpicture}
\end{center}
%
Wir kennen das Konzept eigentlich schon sehr gut aus Übung 1.2, da hatten
wir für das axialsymmetrische $\mathrm{rect}(t)$-Signal ein reelles
Sinc-Spektrum ermittelt, was nur Phasensprünge wegen Polaritätswechsel aufweist.
Genau das wird uns hier wieder begegnen.

Der \textbf{Frequenzgang ist Auswerten der z-Transformierten auf dem Einheitskreis}, also
\begin{align}
H_0(z)\bigg|_{z=\e^{\im\Omega}}
=
\e^{\im 5 \Omega} + \e^{\im 3 \Omega} + \nicefrac{17}{4}\cdot \e^{\im \Omega}
 + \nicefrac{17}{4}\cdot \e^{-\im\Omega} + \e^{-\im 3 \Omega} + \e^{-\im 5 \Omega}
\end{align}
%
Die DTFT Analyse Formel (das ist ja der Speziallfall
z-Transformierte auf dem Einheitskreis, weil $z=\e^{\Sigma+\im\Omega}$ mit
$\Sigma=0$) führt daher auf das gleiche Ergebnis
\begin{align}
H_0(\e^{\im\Omega}) = \sum_{k=-\infty}^{+\infty} h_0[k] \e^{-\im\Omega k}=
\end{align}
Diese unendliche Summe hat aber nur Einträge für
\begin{align}
&h_0[k=-5] \e^{-\im\Omega (-5)} =  1\cdot\delta[k+5] \e^{-\im\Omega (-5)}\nonumber\\
&h_0[k=-4] \e^{-\im\Omega (-4)} =  0\cdot\delta[k+4] \e^{-\im\Omega (-4)}\nonumber\\
&h_0[k=-3] \e^{-\im\Omega (-3)} =  1\cdot\delta[k+3] \e^{-\im\Omega (-3)}\nonumber\\
&h_0[k=-2] \e^{-\im\Omega (-2)} =  0\cdot\delta[k+2] \e^{-\im\Omega (-2)}\nonumber\\
&h_0[k=-1] \e^{-\im\Omega (-1)} =  \nicefrac{17}{4}\cdot\delta[k+1] \e^{-\im\Omega (-1)}\nonumber\\
&h_0[k=0] \e^{-\im\Omega (0)} =  0\cdot\delta[k] \e^{-\im\Omega 0}\nonumber\\
&h_0[k=+1] \e^{-\im\Omega (+1)} =  \nicefrac{17}{4}\cdot\delta[k-1] \e^{-\im\Omega (+1)}\nonumber\\
&h_0[k=+2] \e^{-\im\Omega (+2)} =  0\cdot\delta[k-2] \e^{-\im\Omega (+2)}\nonumber\\
&h_0[k=+3] \e^{-\im\Omega (+3)} =  1\cdot\delta[k-3] \e^{-\im\Omega (+3)}\nonumber\\
&h_0[k=+4] \e^{-\im\Omega (+4)} =  0\cdot\delta[k-4] \e^{-\im\Omega (+4)}\nonumber\\
&h_0[k=+5] \e^{-\im\Omega (+5)} =  1\cdot\delta[k-5] \e^{-\im\Omega (+5)}
\end{align}
und führt daher zum gleichen Ergebnis wie $H_0(z)\bigg|_{z=\e^{\im\Omega}}$, also
\begin{align}
H_0(\e^{\im\Omega})
=
\e^{\im 5 \Omega} + \e^{\im 3 \Omega} + \nicefrac{17}{4}\cdot \e^{\im \Omega}
 + \nicefrac{17}{4}\cdot \e^{-\im\Omega} + \e^{-\im 3 \Omega} + \e^{-\im 5 \Omega}.
\end{align}
%
Jetzt macht sich unser 'Trick' bezahlt. Wir sehen, dass wir drei Euler Identitäten
zu jeweils cos machen können. Ohne diesen 'Trick' hätten wir 'krumme' komplexe Dreher
stehen, die wir mühsam umformen hätten müssen um in Betrag und Phase aufteilen zu können.
So aber bekommen wir vergleichsweise elegant
\begin{align}
\label{eq:64BE42BAEF_H0}
H_0(\e^{\im\Omega}) = 2 \cos(5\Omega) + 2 \cos(3\Omega) + \nicefrac{17}{2} \cos(\Omega),
\end{align}
was in \fig{fig:fir_filter_64BE42BAEF_zerophase} skizziert ist.
%
Das Spektrum ist zusammengesetzt aus cos-Schwingungen (über die lineare Frequenzachse!)
mit 1 (grün), 3 (orange)
bzw. 5 (grün ) Schwingungen pro $2\pi$ über $\Omega$.
%
Wir müssen uns hier klar sein, dass ein DTFT-Spektrum $2\pi$-periodisch ist, d.h.
$X(\Omega) = X(\Omega \pm \nu 2\pi)$ mit $\nu\in\mathbb{Z}$. Wir sehen in den
Grafiken nur den
Ausschnitt $0\leq \Omega < 2\pi$. Diese Darstellung oder die Wahl von
$-\pi\leq \Omega < \pi$ sind die gebräuchlichsten, wenn wir
Spektren von zeitdiskreten Signalen und Spektren skizzieren, dort ist ja
alle Information enthalten.
%
Wir sehen, dass $H_0(\e^{\im \Omega})$ zwischen $\Omega=\frac{\pi}{2}$ und $\Omega=\frac{3\pi}{2}$
negative Amplitude hat, dieses System also einen Polaritätswechsel für diese
Frequenzen erzeugt.
Dies spiegelt sich im Phasendiagramm mittels Unstetigkeiten wider.

Das linearphasige System hat den Frequenzgang
\begin{align}
H_d(z)\bigg|_{z=\e^{\im\Omega}} = H_d(\e^{\im \Omega}) = \e^{-\im 5 \Omega}
\end{align}
und damit frequenzunabhängig Betrag 1, Pegel 0 dB
(das ist der Gag an einer reinen digitalen Verzögerung)
\begin{align}
|H_d(z)\bigg|_{z=\e^{\im\Omega}}| = |H_d(\e^{\im \Omega})| = 1
\end{align}
und die Phase
\begin{align}
\angle H_d(\e^{\im \Omega}) = -5 \Omega.
\end{align}
Dies ist bzgl. der digitalen Frequenzvariable $\Omega$ eine \textbf{lineare
Geradengleichung} mit Anstieg $-5$ (abfallend) ohne $y$-Achsen Offset.
Genau dieser Verlauf begründet die Benennung \textbf{linearphasiges System}.
Wir prüfen schnell nach, dass bei $\Omega=\pi$ (das ist die halbe Abtastfrequenz)
die Phase $\angle H_d(\Omega=\pi) = - 5\pi$, also $-5 \cdot 180=-900$ Grad entsprechen muss,
was wir mit \fig{fig:fir_filter_64BE42BAEF_bode_delay} bestätigen können.
%
\begin{figure}
\centering
\includegraphics[width=0.75\textwidth]{../dtft/fir_filter_64BE42BAEF_zerophase.pdf}
\caption{Nullphasensystem $H_0(\e^{\im \Omega})$.
Oben: Zerlegung in die cos-Schwingungen \eq{eq:64BE42BAEF_H0}, Summe und Betrag (rot).
unten: Nullphase von $H_0(\e^{\im \Omega})$ mit
Phasensprung durch Polaritätswechsel bei $\Omega=\frac{\pi}{2}$ und $\Omega=\frac{3\pi}{2}$.
Hier Konvention, dass positive Sprünge um 180 Grad. Der zweite Sprung auf 360 Grad
(und nicht auf Null zurück), damit Phasenspektrum punktsymmetrisch bzgl.
$\Omega=0$ bzw. $\Omega=\pi$ wird. \texttt{fir\_filter\_64BE42BAEF.ipynb}
}
\label{fig:fir_filter_64BE42BAEF_zerophase}
\end{figure}
%
\begin{figure}
\centering
\includegraphics[width=0.75\textwidth]{../dtft/fir_filter_64BE42BAEF_bode_delay.pdf}
\caption{Frequenzgang $H_d(\e^{\im \Omega})$ des Linearphasensystems
$h_d[k]=\delta[k-5] \ztransf H_d(z)=z^{-5}$. Oben: Pegel, unten: Phase.
\texttt{fir\_filter\_64BE42BAEF.ipynb}}
\label{fig:fir_filter_64BE42BAEF_bode_delay}
\end{figure}
%

Den Frequenzgang, also genauer den Pegel und die Phase, des
Gesamtsystems $H(z)\bigg|_{z=\e^{\im\Omega}} = H(\e^{\im \Omega})$
erhalten wir nun wieder mit den Regeln des Bodediagramms:
wir haben zwei Teilsysteme in Reihe. Daher können wir dB und Grad addieren.
Da das linearphasige System $H_d(\e^{\im \Omega})$ Pegel 0 dB hat, gibt es zwischen
dem Pegel von $H_0(\e^{\im \Omega})$ und $H(\e^{\im \Omega})$ keinen Unterschied.
Wir erkennen eine sogenannte Bandstop Filter Charakteristik, d.h. tiefe Frequenzen
(nahe $\Omega=0$) und hohe Frequenzen (nahe $\Omega=\pi$) werden durch das System
gelassen und sogar verstärkt um über 20 dB. Frequenzen nahe $\frac{\pi}{2}$ (ein Viertel
der Abtastfrequenz) werden sehr stark gedämpft. Die Frequenz $\Omega=\frac{\pi}{2}$
wird sogar komplett herausgelöscht, weil die Nullstelle da exakt \textbf{auf}
dem Einheitskreis liegt (siehe PN-Diagramm!),
also dort wo wir den Frequenzgang auswerten, kann sie sich voll entfalten.

Die Phasendiagramme addiert ergibt die untere Grafik
\fig{fig:fir_filter_64BE42BAEF_bode}. Bis auf die beiden Unstetigkeiten
(das ist immer ein wenig Gefrickel :-( und wollen wir an dieser
Stelle nicht vertiefen; siehe dazu z.B. das Referenzwerk
\cite{Oppenheim2010, Oppenheim2004})
handelt es sich immer noch um einen linearen Phasenabfall, also immer
noch um ein \textbf{linearphasiges} System.
%
Merksatz (wir vertiefen das im Mastermodul DSP): Axialsymmetrische und
punktsymmetrische Impulsantworten weisen Linearphasigkeit auf.

\textbf{h) Gruppenlaufzeit}
Das Konzept der Gruppenlaufzeit und der Phasenlaufzeit
gibt es auch für zeitdiskrete Systeme.
Hier lauten die Berechnungsvorschriften
\begin{align}
\tau_\text{GD}(\Omega) = -\frac{\fsd \angle H(\e^{\im \Omega})}{\fsd \Omega}\\
\tau_\text{PD}(\Omega) = -\frac{\angle H(\e^{\im \Omega})}{\Omega}
\end{align}
beides in Samples. Mit der Abtastfrequenz $f_s$ wäre
$\frac{\tau_\text{GD}}{f_s}$ in s.
%
In unserem Beispiel (wenn wir die Unstetigkeiten bei der Ableitung mal umschiffen)
\begin{align}
\tau_\text{GD} = -\frac{\fsd (-5 \Omega)}{\fsd \Omega} = 5\text{ Samples}\\
\end{align}
also eine Konstante, wie in der unteren Grafik für $0\leq \Omega \leq 2 \pi$
veranschaulicht ist.
\begin{center}
\begin{tikzpicture}[scale=1]
\def\tic{0.1};
\draw[->] (0,0) -- (4.5,0) node[right]{$\Omega$};
\draw[->] (0,0) -- (0,1.5) node[above]{$\tau_\text{GD}(\Omega)$ in Samples};
\foreach \x in {0,1,...,4}{\draw (\x,+\tic) -- (\x,-\tic)  node[below]{$\x \frac{\pi}{2}$};};
\foreach \y in {0,5}{\draw (\tic,\y/5) -- (-\tic,\y/5)  node[left]{$\y$};};
\draw[C0, ultra thick] (0,1) -- (4,1);
\draw[C0, ultra thick] (1,1) node{$\circ$};
\draw[C0, ultra thick] (3,1) node{$\circ$};
\end{tikzpicture}
\end{center}
%
Im Gegensatz zu Aufgabe 6.6, wo die frequenzabhängige Gruppenlaufzeit dazu führt,
dass verschiedene Frequenzgruppen unterschiedlich 'schnell' durch das System gelangen,
haben wir hier die vorteilhafte Eigenschaft, dass alle Frequenzgruppen
gleich schnell behandelt werden, also alle mit 5 Samples Verzögerung das
System durchlaufen.
Wenn uns daran gelegen ist, die Phasenlage von Frequenzgruppen nicht kaputt
zu machen, sollten wir also diese Charakteristik wählen.
%

Praxisbeispiel:
Bei Kopfhöreren und Lautsprechern gehobener Preisklasse versuchen Hersteller
dieses Verhalten zusammen mit glattem 0 dB Pegelverlauf bestmöglich anzunähern,
um das Musiksignal so unverfälscht wie möglich zu verstärken.
%
Auf besonders guten Lautsprechern ist dann besonders schlecht produzierter
Content oder mp3s mit niedriger Codierungsqualität eher nicht zu ertragen, weil
die Artefakte dann nicht mehr verschliffen werden bzw. sehr 'ehrlich'
wiedergegeben werden.
%


\begin{figure}
\centering
\includegraphics[width=0.75\textwidth]{../dtft/fir_filter_64BE42BAEF_bode.pdf}
\caption{Bode Diagramm für $H(z)$. Linearphasiges Bandstop Filter.
\texttt{fir\_filter\_64BE42BAEF.ipynb}}
\label{fig:fir_filter_64BE42BAEF_bode}
\end{figure}










\clearpage
\subsection{DTFT-Spektrum der Rechteckfolge}
\label{sec:767B4C89FE}
\begin{Ziel}
Wir wollen uns intensiver mit dem DTFT-Spektrum der Rechteckfolge beschäftigen.
Das steht in Analogie zur Fouriertransformation (FT) der Rechteckfunktion in
der zeitkontinuierlichen SigSys, siehe Übung 1.
Das DTFT-Spektrum einer Rechteckfolge beinhaltet die sogenannte \texttt{psinc}-Funktion,
in Analogie zur 'normalen' sinc-Funktion im Zeitkontinuierlichen.
%
Wir werden die Dualitäten Modulation vs. Verschiebung wieder antreffen, die ähnlich
funktionieren, wir müssen bei der DTFT zusätzlich berücksichtigen, dass
die Spektren $2\pi$-periodisch bzgl. $\Omega$ sind, was impliziert, dass
wir es unter anderem mit einer zyklischen Faltung im Bildbereich zu tun bekommen.
%
\end{Ziel}
\textbf{Aufgabe} {\tiny 767B4C89FE}: Berechnen Sie die DTFT Spektren
\begin{align}
X_4(\e^{\im\Omega}) \quad\DTFT\quad & x_4[k] = \mathrm{rect}_4[k]\\
X_5(\e^{\im\Omega}) \quad\DTFT\quad & x_5[k] = \mathrm{rect}_5[k+2]
\end{align}
Skizzieren Sie die Zeitfolgen, Betrags- und Phasenspektrum, sowie
Real- und Imaginärteil über $\Omega$ und diskutieren Sie daran die auftretenden
Symmetrien der DTFT.
%
Machen Sie sich anhand der axialsymmetrischen Folge
\begin{align}
X_{31}(\e^{\im\Omega}) \quad\DTFT\quad & x_{31}[k] = \mathrm{rect}_{31}[k+15]
\end{align}
mit der Skizze $\Re\{X_{31}\}$ über $\Omega$ klar, warum die periodische
Sinc-Funktion ihren Namen trägt.


\begin{Werkzeug}
Wir brauchen die fundamentale Korrespondenz
\begin{equation}
x[k] = \mathrm{rect}_N[k]
\dtft
X(\e^{\im\Omega})=\e^{-\im\frac{\Omega(N-1)}{2}}\cdot\frac{\sin\left(N\frac{\Omega}{2}\right)}{\sin\left(\frac{\Omega}{2}\right)}
\end{equation}
und den Verschiebungssatz
\begin{equation}
x[k-\kappa]
\dtft
X(\e^{\im\Omega}) \cdot \e^{-\im\Omega\kappa}
\end{equation}
\end{Werkzeug}
\begin{Ansatz}
DTFT Hintransformation für $N$ allgemein
\begin{align}
X(\e^{\im\Omega})
=& \sum_{k=-\infty}^{+\infty} \mathrm{rect}_N[k] \, \e^{-\im\Omega k}\\
=& \sum_{k=0}^{N-1} \e^{-\im\Omega k}
\end{align}
Für eine Verschiebung im Zeitbereich benutzen wir den Verschiebungssatz.
\end{Ansatz}
\begin{ExCalc}
Es ist nun nicht überraschend, dass wir es wieder mit einer geometrischen
Folge zu tun haben werden.
%
Die DTFT ist ja ein Spezialfall der z-Transformation, und dort begegnete uns
die geometrische Folge beim 1-Pol Signal, aus dem die Einheitsfolge ableitbar ist,
aus der wiederum eine endlich Rechteckfolge zusammensetzbar ist (siehe Übung 2.2).

Die endliche Summe aus unserem Ansatz
\begin{align}
X(\e^{\im\Omega})
= \sum_{k=-\infty}^{+\infty} \mathrm{rect}_N[k] \, \e^{-\im\Omega k} = \sum_{k=0}^{N-1} \e^{-\im\Omega k}
\end{align}
ist eine geometrische Reihe mit bekannter geschlossener Lösung
\begin{equation}
X(\e^{\im\Omega})=\frac{1-\e^{-\im\Omega N}}{1-\e^{-\im\Omega}}.
\end{equation}
%
Wir können das umformen
%
\begin{equation}
X(\e^{\im\Omega})=\frac{\e^{-\im\frac{\Omega N}{2}}}{\e^{-\im\frac{\Omega}{2}}}\cdot\frac{\e^{\im\frac{\Omega N}{2}}-\e^{-\im\frac{\Omega N}{2}}}{\e^{\im\frac{\Omega}{2}}-\e^{-\im\frac{\Omega}{2}}}
=\e^{-\im\frac{\Omega(N-1)}{2}}\cdot\frac{\e^{\im\frac{\Omega N}{2}}-\e^{-\im\frac{\Omega N}{2}}}{\e^{\im\frac{\Omega}{2}}-\e^{-\im\frac{\Omega}{2}}}
=\e^{-\im\frac{\Omega(N-1)}{2}}\cdot\frac{\sin\left(N\frac{\Omega}{2}\right)}{\sin\left(\frac{\Omega}{2}\right)},
\end{equation}
%
letzteres mit Euler Identität $2\im\cdot\sin(x)=\e^{\im x}-\e^{-\im x}$.
%
Der sin/sin-Bruch ist so wichtig, dass wir ihm einen eigenen Funktionsnamen
\begin{equation}
\mathrm{psinc}_N(\Omega) := \frac{1}{N}\frac{\sin\left(N\frac{\Omega}{2}\right)}{\sin\left(\frac{\Omega}{2}\right)}
\end{equation}
verpassen, diese Funktion ist bekannt als periodische Sinc-Funktion, in Englisch
oft auch aliased sinc.
%
Sie kommt flächendeckend in allen Natur- und Ingenieurwissenschaften vor, daher
sollten wir uns mit ihr vertraut machen.
%
In Python und Matlab ist sie als sogenannte Dirichlet Funktion \texttt{diric(x,N)}
implementiert, siehe

\url{https://docs.scipy.org/doc/scipy/reference/generated/scipy.special.diric.html}

\url{https://mathworks.com/help/signal/ref/diric.html}

Daher können wir das DTFT Spektrum der Rechteckfolge mit Länge $N$ als
\begin{equation}
X(\e^{\im\Omega})=
\mathrm{psinc}_N(\Omega) \cdot N \cdot \e^{-\im\frac{\Omega(N-1)}{2}}
\end{equation}
schreiben.
%
Es ist wichtig sich klarzumachen, dass $\mathrm{psinc}_N(\Omega)$
\begin{itemize}
\item $4\pi$ periodisch für gerade $N$ (wegen Hauptmaxima mit alternierendem Vorzeichen)
\item $2\pi$ periodisch für ungerade $N$ (alle Hauptmaxima mit positiver Amplitude)
\end{itemize}
Der Betrag ist immer $2\pi$-periodisch.
\end{ExCalc}


\begin{Loesung}
Für $x_4[k]$ müssen wir einfach nur $N$ in das Spektrum einsetzen
\begin{equation}
X_4(\e^{\im\Omega})=\e^{-\im\frac{3\Omega}{2}}\cdot\frac{\sin\left(4\frac{\Omega}{2}\right)}{\sin\left(\frac{\Omega}{2}\right)}
\end{equation}
%
Für $x_5[k]$ verwenden wir den Verschiebungssatz
\begin{align}
x_5[k] = \mathrm{rect}_5[k-(-2)] \quad\dtft\quad X_5(\e^{\im\Omega}) = X(\e^{\im\Omega}) \cdot \e^{-\im\Omega\cdot (-2)}
\end{align}
Wir bekommen
\begin{equation}
X_5(\e^{\im\Omega})
=\e^{-\im\frac{4\Omega}{2}}\cdot\frac{\sin\left(5\frac{\Omega}{2}\right)}{\sin\left(\frac{\Omega}{2}\right)}  \cdot \e^{-\im\Omega\cdot (-2)}=
\frac{\sin\left(5\frac{\Omega}{2}\right)}{\sin\left(\frac{\Omega}{2}\right)}
\end{equation}
Der gleiche Ansatz führt für $x_{31}[k]$ zu
\begin{equation}
X_{31}(\e^{\im\Omega}) = \frac{\sin\left(31\frac{\Omega}{2}\right)}{\sin\left(\frac{\Omega}{2}\right)}
\end{equation}
%
Diese speziell gewählten Zeitverschiebungen um $\frac{N-1}{2}$ hat bei ungeraden
$N$ zur Folge, dass die Rechteckfolge axialsymmetrisch ist und daraus
ein rein reelles DTFT Spektrum resultiert. Das kommt uns bekannt vor aus
Aufgabe \ref{sec:64BE42BAEF}, dort hatten wir
eine reelle, axialsymmetrische Folge und ein nullphasiges Spektrum miteinander
verknüpft. Wir haben hier den gleichen 'Trick' der Phasenverschiebung genutzt,
um die kausale Rechteckfolge axialsymmetrisch zu machen, das geht natürlich
nur für ungerade $N$.
%

In \fig{fig:psinc_767B4C89FE_N4_linphase} sind Signal $x_4[k]$ und DTFT
Spektrum grafisch aufbereitet. Die psinc-Funktion hat zwischen zwei Hauptmaxima
genau $N-1=3$ Nullstellen, also z.B. zwischen $\Omega=0$ und $\Omega=2\pi$,
hier aller $\frac{\pi}{2}$. Es wichtig sich klarzumachen, dass die Nebenmaxima
nicht gleichabständig verteilt sind.
%
An den Nullstellen entstehen dann auch Phasensprünge um 180 deg, weil
atan in einem anderen Quadrant ausgewertet werden muss.

In \fig{fig:psinc_767B4C89FE_N5_linphase} sehen wir zunächst den Fall
$x[k] = \mathrm{rect}_5[k]$, mit $N-1=4$ Nullstellen zwischen zwei Hauptmaxima,
verteilt aller $\frac{2\pi}{5}$.

In \fig{fig:psinc_767B4C89FE_N5_zerophase} ist dann das nullphasige, weil axialsymmetrische
Signal $x_5[k]$ mit DTFT-Spektrum dargestellt.
Das Spektrum ist rein reell, und hat Phasensprünge um 180 Grad nur
noch aufgrund des Polaritätswechsels. Das Phasendiagramm müsste eigentlich
punktsymmetrisch so wie in Abb. 1.7 aufgemalt werden, die Programmierung in Python
mittels \texttt{np.unwrap(np.angle(X))} liefert uns jedoch Sprünge zwischen 0 und 180 Grad,
womit wir in der Praxis umzugehen lernen müssen.

Ganz ähnlich verhält es sich mit $x_{31}[k]$ und dessen DTFT Spektrum, skizziert
in \fig{fig:psinc_767B4C89FE_N31_zerophase}. Die 30 Nullstellen zwischen
zwei Hauptmaxima können wir gerade noch abzählen, wenn wir stark reinzoomen.
An diesem Beispiel erkennen wir, warum die psinc-Funktion ihren Namen trägt.
Je größer $N$, desto mehr nähert sich die psinc-Funktion der 'normalen'
Funktion im Bereich $-\pi\leq\Omega<\pi$ an (sie wird aber nie exakt identisch sein!).
Dieses sogenannte Basisband (wir könnten auch den Bereich $0\leq\Omega<2\pi$ nehmen)
ist dann $2\pi$-periodisch. Das muss so sein, weil die DTFT als Transformation
ja $2\pi$ periodisch ist und psinc ein spezielles, aben eben sehr fundamentales DTFT
Spektrum ist.

Hinweis: wir plotten überall $|\mathrm{psinc}_N(\Omega)\cdot N|$, daher bekommen wir
eine $N$-abhängigen Betrag. psinc selber ist amplituden-normiert 1, wie sinc auch.

\end{Loesung}
%
\begin{figure}
\centering
\includegraphics[width=0.75\textwidth]{../dtft/psinc_767B4C89FE_N4_linphase.pdf}
\caption{$x_4[k]\dtft X_4(\Omega)$.
\texttt{psinc\_767B4C89FE.ipynb}}
\label{fig:psinc_767B4C89FE_N4_linphase}
\end{figure}
%
\begin{figure}
\centering
\includegraphics[width=0.75\textwidth]{../dtft/psinc_767B4C89FE_N5_linphase.pdf}
\caption{$\mathrm{rect}_5[k]$ und zugehöriges DTFT Spektrum.
\texttt{psinc\_767B4C89FE.ipynb}}
\label{fig:psinc_767B4C89FE_N5_linphase}
\end{figure}
%
\begin{figure}
\centering
\includegraphics[width=0.75\textwidth]{../dtft/psinc_767B4C89FE_N5_zerophase.pdf}
\caption{$x_5 \dtft X_5(\Omega)$.
\texttt{psinc\_767B4C89FE.ipynb}}
\label{fig:psinc_767B4C89FE_N5_zerophase}
\end{figure}
%
\begin{figure}
\centering
\includegraphics[width=0.75\textwidth]{../dtft/psinc_767B4C89FE_N31_zerophase.pdf}
\caption{$x_{31}[k]\dtft X_{31}(\Omega)$.
\texttt{psinc\_767B4C89FE.ipynb}}
\label{fig:psinc_767B4C89FE_N31_zerophase}
\end{figure}




\clearpage
\subsection{DTFT-Spektrum der rechteckbegrenzten komplexen Schwingung}
\label{sec:96D7F1EE47}
\begin{Ziel}
In dieser Aufgabe wollen wir uns ähnlich Aufgabe 4.1
die DTFT der rechteckbegrenzten komplexen Exponentialschwingung anschauen.
Wir werden die Aufgabe hier dahingehend modifizieren und erweitern, dass
wir komplexe Signalw DTFT-transformieren und eine Überlagerung von zwei Schwingungen
betrachten. Nachdem die DTFT auch Linearitätseigenschaft besitzt, überlagern
sich die komplexen Einzelspektren. Es stellt sich die Frage, unter welchen
Umständen wir im Gesamtspektrum sehen, dass das transformierte Signal aus
zwei Schwingungen besteht. Dies führt uns zu den Grundlagen der Spekralanalyse.
Nachdem die vorhergehenden Aufgaben sehr phasen-lastig waren, wollen wir
uns hier mehr mit dem Betragsspektrum beschäftigen.
\end{Ziel}
\textbf{Aufgabe} {\tiny 96D7F1EE47}:
Berechnen Sie die DTFT des Signals
\begin{align}
x[k] = w[k] \cdot
\left(
x_1[k] + x_2[k]
\right)
\end{align}
mit $x_1[k] = \e^{+\im\frac{\pi}{2} k}$, $x_2[k] = \e^{+\im\frac{3\pi}{4} k}$
und $w[k] = \mathrm{rect}_8[k]$.
%
Skizzieren Sie die DTFT-Betragsspektren von $x_1[k]$, $x_2[k]$,
der zeitlich begrenzten Folgen $w[k] \cdot x_1[k]$, $w[k] \cdot x_2[k]$, sowie
des zeitlich begrenzten Gesamtsignals $w[k] \cdot x[k]$.


\begin{Werkzeug}
DTFT-Korrespondenzen:
\begin{align}
\e^{+\im\Omega_0 k} \quad\dtft\quad \Sha(\frac{\Omega-\Omega_0}{2\pi})
= 2\pi \sum\limits_{\mu=-\infty}^{+\infty}
\delta(\Omega-\Omega_0 + 2\pi\mu)
\end{align}
%
\begin{equation}
w[k] = \mathrm{rect}_N[k]
\quad\dtft\quad
W(\e^{\im\Omega})=\e^{-\im\frac{\Omega(N-1)}{2}}\cdot\frac{\sin\left(N\frac{\Omega}{2}\right)}{\sin\left(\frac{\Omega}{2}\right)}
\end{equation}
%
Multiplikation / Faltung Dualität (hier als $2\pi$ zyklische Faltung im Spektralbereich):
\begin{align}
x[k] \cdot w[k] \quad\dtft\quad \frac{1}{2\pi} X(\e^{\im \Omega}) \circledast_{2\pi} W(\e^{\im \Omega})
\end{align}
%
Funktionsargument $\Omega$ statt $\e^{\im \Omega}$ erscheint ab jetzt eleganter:
\begin{align}
X(\Omega) \circledast_{2\pi} W(\Omega) =
\int\limits_{\nu=0}^{2\pi} X(\nu) \cdot W(-\nu + \Omega) \fsd \nu =
\int\limits_{\nu=0}^{2\pi} X(-\nu+\Omega) \cdot W(\nu) \fsd \nu
\end{align}
Hilfsvariable für Frequenzspiegelung und -verschiebung ist $\nu$.
\end{Werkzeug}
\begin{Ansatz}
In Aufgabe 4.1 hatten wir zwei Sichtweisen auf das Wesen kennengelernt, dies
lässt sich hier analog anwenden.

\textbf{Sichtweise I}: unendliche Exponentialschwingung zeitlich
begrenzen heißt ideales Dirac-Kamm Spektrum verschleifen,
i.e. aus Sicht der Signalanalyse, Fensterung

\textbf{Sichtweise II}: das periodische, psinc-artige Spektrum der Rechteckfunktion
im Frequenzbereich verschieben i.e. aus Sicht der Nachrichtentechnik, Modulation

Wir haben diesmal eine komplexe Exponentialschwingung.

Wir brauchen zunächst die Spektren der beiden Exponentialschwingungen $x_1[k]$
und $x_2[k]$.
Die Korrespondenz ist gut bekannt und direkt aus der Formelsammlung entnehmbar.
Die Grafik unten skizziert die beiden Fälle, also $\Omega_1=\frac{2\pi}{4} = \frac{\pi}{2}$
und $\Omega_2=\frac{3\pi}{4}$
\begin{center}
\begin{tikzpicture}[scale=0.75]
\begin{scope}
\draw[help lines, C7!50, step=0.5cm] (-4,0) grid (4,1);
\draw[->] (-4 ,0) -- (4.75,0) node[right]{$\Omega$};
\draw[->] (0,-0.1) -- (0,1.75) node[above]{$X_1(\Omega)=\Sha(\frac{\Omega-2\cdot\frac{2\pi}{8}}{2\pi})$};
\draw[->, C0, line width=1mm] (0.5,0) node[below]{$\frac{\pi}{2}$} -- (0.5,1) node[above]{$(2\pi)$};
\draw[->, C0, line width=1mm] (2.5,0) node[below]{$\frac{5\pi}{2}$} -- (2.5,1);
\draw[->, C0, line width=1mm] (4.5,0) -- (4.5,1);
\draw[->, C0, line width=1mm] (-1.5,0) -- (-1.5,1);
\draw[->, C0, line width=1mm] (-3.5,0) -- (-3.5,1);
\end{scope}
\begin{scope}[shift={(10,0)}]
\draw[help lines, C7!50, step=0.5cm] (-4,0) grid (4,1);
\draw[->] (-4 ,0) -- (4.75,0) node[right]{$\Omega$};
\draw[->] (0,-0.1) -- (0,1.75) node[above]{$X_2(\Omega)=\Sha(\frac{\Omega-3\cdot\frac{2\pi}{8}}{2\pi})$};
\draw[->, C0, line width=1mm] (0.75,0) node[below]{$\frac{3\pi}{4}$} -- (0.75,1) node[above]{$(2\pi)$};
\draw[->, C0, line width=1mm] (2.75,0) node[below]{$\frac{11\pi}{4}$} -- (2.75,1);
\draw[->, C0, line width=1mm] (4.75,0) -- (4.75,1);
\draw[->, C0, line width=1mm] (-1.25,0) -- (-1.25,1);
\draw[->, C0, line width=1mm] (-3.25,0) -- (-3.25,1);
\end{scope}
\end{tikzpicture}
\end{center}
Da für die DTFT auch die Linearitätseigenschaft gilt, können wir die beiden Signale
bzw. Spektren zunächst einzeln betrachten und ganz am Ende addieren, also
\begin{align}
x_1[k] + x_2[k] \quad\dtft\quad X_1(\Omega)+X_2(\Omega)
\end{align}
Deswegen gilt auch für die zeitliche Begrenzung mit der Rechteckfolge (auch Rechteckfensterung genannt)
\begin{align}
&w[k] \cdot (x_1[k] + x_2[k]) = w[k] \cdot x_1[k] + w[k] \cdot x_2[k]
\quad\dtft\quad\\
&\frac{1}{2\pi} W(\Omega) \circledast_{2\pi} ( X_1(\Omega)+X_2(\Omega)) = \frac{1}{2\pi} W(\Omega) \circledast_{2\pi} X_1(\Omega) + \frac{1}{2\pi} W(\Omega) \circledast_{2\pi} X_2(\Omega))
\end{align}
$\frac{1}{2\pi}$ bei der Faltung nicht vergessen!

\end{Ansatz}
\begin{ExCalc}
Mit Übung können wir solche Dirac Kamm Faltungen irgendwann in Gedanken durchführen.
Wir sollten die Rechnung aber in aller Ausführlichkeit gesehen haben, weil hier
die zyklische Faltung im Frequenzbereich eine wichtige Rolle spielt,
und im Laufe der Rechnung ein paar Klassiker beim Umformen von komplexen Drehern
und Stolpersteine bzgl. der psinc auf uns warten. Daher:
%
Faltungsintegral mit Hilfsvariable $\nu$, Summe des Dirac Impulskamms mittels
Hilfsvariable $\mu$.
\begin{align}
X(\Omega) \circledast_{2\pi} W(\Omega) =
\int\limits_{\nu=0}^{2\pi} X(\nu) \cdot W(-\nu + \Omega) \fsd \nu =
\int\limits_{\nu=0}^{2\pi} X(-\nu+\Omega) \cdot W(\nu) \fsd \nu
\end{align}
%
Wählen wir, weil es eleganter erscheint die Form mit $W(\nu)$
\begin{align}
w[k] \cdot x_1[k]\dtft
\frac{1}{2\pi}
X_1(\Omega) \circledast_{2\pi} W(\Omega)
=
\frac{1}{2\pi}
\int\limits_{\nu=0}^{2\pi}
2\pi \sum\limits_{\mu=-\infty}^{+\infty}
\delta(-\nu+\Omega-\Omega_1 + 2\pi\mu)
\cdot
\e^{-\im\frac{\nu(N-1)}{2}}\,
\frac{\sin\left(N\frac{\nu}{2}\right)}{\sin\left(\frac{\nu}{2}\right)}\fsd \nu
\end{align}
Die Summe ist der Dirac Impuls Kamm ausgeschrieben. Nun kürzen sich die beiden
$2\pi$-Terme, das ist gut, weil die Diracs mit Gewicht 1 agieren müssen, das
sollte uns unsere Erwartungshaltung sagen.
Daher den $\frac{1}{2\pi}$-Term bei Faltung nicht vergessen.
%
\begin{align}
w[k] \cdot x_1[k]\dtft=
\int\limits_{\nu=0}^{2\pi}
\sum\limits_{\mu=-\infty}^{+\infty}
\delta(-\nu+\Omega-\Omega_1 + 2\pi\mu)
\cdot
\e^{-\im\frac{\nu(N-1)}{2}}\,
\frac{\sin\left(N\frac{\nu}{2}\right)}{\sin\left(\frac{\nu}{2}\right)}\fsd \nu
\end{align}
Das Argument des Dirac wird Null (schlampig formuliert:
an dieser Stelle 'explodiert' der Impuls) wenn
\begin{align}
\nu=+\Omega-\Omega_1 + 2\pi\mu.
\end{align}
Wenn uns die Summe über $\mu$ nervös macht (die kommt vom Dirac Impuls Kamm
der ja auch $2\pi$-periodisch ist), dann spielen wir das für ausgewählte
$\mu$ durch, bis wir ein Muster finden, was sich analytisch aufschreiben lässt.
%
Für $\mu=0$ ohne Mühe
\begin{align}
\nu=+\Omega-\Omega_1
\end{align}
Multiplikationseigenschaft des Diracs anwenden
\begin{align}
\text{für   }\mu=0\rightarrow
\int\limits_{\nu=0}^{2\pi}
\delta(-\nu+\Omega-\Omega_1)
\cdot
\e^{-\im\frac{(\Omega-\Omega_1)(N-1)}{2}}\,
\frac{\sin\left(N\frac{(\Omega-\Omega_1)}{2}\right)}{\sin\left(\frac{(\Omega-\Omega_1)}{2}\right)}\fsd \nu
\end{align}
Und weil wir die Integraloperation stehen haben, können wir sogar die
Austasteigenschaft bemühen (und den Dirac Impuls damit loswerden, der hat dann seine Arbeit verrichtet)
\begin{align}
\mu=0\rightarrow
\e^{-\im\frac{(\Omega-\Omega_1)(N-1)}{2}}\,
\frac{\sin\left(N\frac{(\Omega-\Omega_1)}{2}\right)}{\sin\left(\frac{(\Omega-\Omega_1)}{2}\right)}
\end{align}
%
Schauen wir nun, was für gerade $\mu$, z.B. $\mu=2$, passiert, also zunächst wieder Multiplikationseigenschaft
ansetzen
\begin{align}
\text{für   }\mu=2\rightarrow
\int\limits_{\nu=0}^{2\pi}
\delta(-\nu+\Omega-\Omega_1+2 \cdot 2\pi)
\cdot
\e^{-\im\frac{(\Omega-\Omega_1 +2 \cdot  2\pi)(N-1)}{2}}\,
\frac{\sin\left(N\frac{(\Omega-\Omega_1 + 2 \cdot 2\pi)}{2}\right)}{\sin\left(\frac{(\Omega-\Omega_1 + 2 \cdot 2\pi)}{2}\right)}\fsd \nu
\end{align}
und dann Austasteigenschaft
\begin{align}
\mu=2\rightarrow
\e^{-\im\frac{(\Omega-\Omega_1 +2 \cdot  2\pi)(N-1)}{2}}\,
\frac{\sin\left(N\frac{(\Omega-\Omega_1 + 2 \cdot 2\pi)}{2}\right)}{\sin\left(\frac{(\Omega-\Omega_1 + 2 \cdot 2\pi)}{2}\right)}
\end{align}
%
Den komplexen Dreher können wir vereinfachen (funktioniert für alle geraden $\mu$)
\begin{align}
\e^{-\im\frac{(\Omega-\Omega_1 + 2 \cdot 2\pi)(N-1)}{2}} =
\e^{-\im\frac{(\Omega-\Omega_1)(N-1)}{2}}
\underbrace{\e^{-\im\frac{(4\pi)(N-1)}{2}}}_{1}
\end{align}
und da psinc periodisch in $4\pi$ für gerade $N$ wissen wir, dass
$\frac{\sin\left(N\frac{(\Omega-\Omega_1 + 4\pi)}{2}\right)}{\sin\left(\frac{(\Omega-\Omega_1 + 4\pi)}{2}\right)} = \frac{\sin\left(N\frac{(\Omega-\Omega_1)}{2}\right)}{\sin\left(\frac{(\Omega-\Omega_1)}{2}\right)}$.
Wir bekommen also das gleiche Spektrum, wie für $\mu=0$
\begin{align}
\e^{-\im\frac{(\Omega-\Omega_1)(N-1)}{2}}\,
\frac{\sin\left(N\frac{(\Omega-\Omega_1)}{2}\right)}{\sin\left(\frac{(\Omega-\Omega_1)}{2}\right)}
\end{align}
%
Schauen wir uns das gleiche für ungerade $\mu$, z.B. $\mu=1$ an, also auch wieder
Multiplikationseigenschaft
\begin{align}
\text{für   }\mu=1\rightarrow
\int\limits_{\nu=0}^{2\pi}
\delta(-\nu+\Omega-\Omega_1+1\cdot2\pi)
\cdot
\e^{-\im\frac{(\Omega-\Omega_1 + 1\cdot2\pi)(N-1)}{2}}\,
\frac{\sin\left(N\frac{(\Omega-\Omega_1 + 1\cdot2\pi)}{2}\right)}{\sin\left(\frac{(\Omega-\Omega_1 + 1\cdot2\pi)}{2}\right)}\fsd \nu
\end{align}
und Austasteigenschaft
\begin{align}
\mu=1\rightarrow
\e^{-\im\frac{(\Omega-\Omega_1 + 1\cdot2\pi)(N-1)}{2}}\,
\frac{\sin\left(N\frac{(\Omega-\Omega_1 + 1\cdot2\pi)}{2}\right)}{\sin\left(\frac{(\Omega-\Omega_1 + 1\cdot2\pi)}{2}\right)}
\end{align}
Für gerade $N$ gilt hier nun sowohl Drehung der Polarität beim psinc als auch beim komplexen Dreher
\begin{align}
\frac{\sin\left(N\frac{(\Omega-\Omega_1 + 1\cdot2\pi)}{2}\right)}{\sin\left(\frac{(\Omega-\Omega_1 + 1\cdot2\pi)}{2}\right)}
= -
\frac{\sin\left(N\frac{(\Omega-\Omega_1)}{2}\right)}{\sin\left(\frac{(\Omega-\Omega_1)}{2}\right)}
\end{align}
%
\begin{align}
\e^{-\im\frac{(\Omega-\Omega_1 + 2\pi)(N-1)}{2}}=
\e^{-\im\frac{(\Omega-\Omega_1)(N-1)}{2}}
\e^{-\im\frac{(2\pi)(N-1)}{2}}=
\e^{-\im\frac{(\Omega-\Omega_1)(N-1)}{2}}
\e^{-\im\frac{2\pi N -2\pi}{2}}=
\e^{-\im\frac{(\Omega-\Omega_1)(N-1)}{2}}
\underbrace{\e^{-\im\frac{2\pi N}{2}}}_{1}
\underbrace{\e^{-\im\frac{-2\pi}{2}}}_{-1}
\end{align}
Das Minuszeichen, was wir uns beim psinc 'einhandeln' wird erfreulicherweise
beim komplexen Dreher wieder kompensiert.  Deswegen bekommen wir für ungerade $\mu$
auch wieder das gleiche Spektrum wie für $\mu=0$ heraus.
%
Dies ist nun auch unser Endergebnis der Faltung
\begin{align}
w[k] \cdot x_1[k]\dtft
\e^{-\im\frac{(\Omega-\Omega_1)(N-1)}{2}}\,
\frac{\sin\left(N\frac{(\Omega-\Omega_1)}{2}\right)}{\sin\left(\frac{(\Omega-\Omega_1)}{2}\right)}
\end{align}
Für das zweite Signal ist die Rechnung identisch
\begin{align}
w[k] \cdot x_2[k]\dtft
\e^{-\im\frac{(\Omega-\Omega_2)(N-1)}{2}}\,
\frac{\sin\left(N\frac{(\Omega-\Omega_2)}{2}\right)}{\sin\left(\frac{(\Omega-\Omega_2)}{2}\right)}
\end{align}
%
\end{ExCalc}
%
\begin{Loesung}
Mit den gegebenen Zahlen können wir die Spektren angeben
\begin{align}
w[k] \cdot x_1[k]\dtft
X_{w,1}(\Omega) = \e^{-\im\frac{7(\Omega-\frac{\pi}{2})}{2}}\,
\frac{\sin\left(8\frac{(\Omega-\frac{\pi}{2})}{2}\right)}{\sin\left(\frac{(\Omega-\frac{\pi}{2})}{2}\right)}
%
\\
w[k] \cdot x_2[k]\dtft
X_{w,2}(\Omega) = \e^{-\im\frac{7(\Omega-\frac{3\pi}{4})}{2}}\,
\frac{\sin\left(8\frac{(\Omega-\frac{3\pi}{4})}{2}\right)}{\sin\left(\frac{(\Omega-\frac{3\pi}{4})}{2}\right)}
\end{align}
%
Das ideale (schöne) Dirac-Kamm Spektrum $X_1(\Omega)=\Sha(\frac{\Omega-\Omega_1}{2\pi})$
der komplexen Exponentialschwingung $x_1[k]$, wird also durch
Zeitbegrenzung mit der Rechteckfolge $x_1[k]\cdot w[k]$ verschmiert zu einer periodischen Sinc-Funktion,
hier noch mit Phasenterm, daher einem komplexen Spektrum $X_{w,1}(\Omega)$.
%
Das Betragsspektrum ist in \fig{fig:windowing_96D7F1EE47} rechts oben dargestellt,
wir bemerken die Frequenzachsennormalisierung mit $\pi/4$. Dann können wir
$\Omega_1$ bequem bei $2 \pm 8$ ablesen.

Das genauso verschliffene Betragsspektrum $|X_{w,2}(\Omega)|$ ist in der mittleren Zeile, rechts
dargestellt, hier finden wir das Basisbandmaximum bei $3$ für die gewählte Frequenzachsennormierung.

Die Überlagerung
\begin{align}
w[k](x_1[k]+x_2[k]) \dtft   X_{w,1}(\Omega) + X_{w,2}(\Omega)
\end{align}
ist in \fig{fig:windowing_96D7F1EE47} unten dargestellt.

Durch die Wahl der Frequenzen und der Länge der Rechteckfolge
sind bei der Überlagerung die beiden Hauptmaxima der Einzelfrequenzen klar erkennbar.
Diese liegen aber nicht exakt bei 2 und 3 (das ist ein sehr beliebter Denkfehler),
weil wir komplex addieren!!! Wir können
deswegen so deutlich unterscheiden, weil das Betragsspektrum
$X_{w,1}(\Omega)$ eine Nullstelle bei $\Omega_2$ hat und umgekehrt.
Würden $\Omega_1$ und $\Omega_2$ noch näher zusammenrücken, z.B. $\Omega_2=\frac{5}{2}\cdot\frac{2\pi}{8}$,
wachsen die beiden Einzelhauptmaxima zu einem einzigen breiten Hauptmaximum zusammen.
Wir sehen dann im Spektrum nicht mehr, dass unser Signal aus zwei unterschiedlichen Frequenzen besteht.
%
\end{Loesung}

Die Wahl des optimalen zeitlichen Begrenzungssignals $w[k]$ (im Englischen \textit{window}, daher
die Folgenvariable $w$) und dessen Länge $N$ ist eine eigene 'Wissenschaft' innerhalb
der SigSys, nämlich Spektralanalyse, im speziellen hier für harmonische Signale.
Wir können das an dieser Stelle leider nur anreißen.
%
Wenn als Message hängenbleibt, dass Dirac Kamm Spektren durch Zeitbegrenzung (Multiplikation/Faltung)
verschliffen werden, speziell bei rechteckförmiger Begrenzung psinc-artige Spektren
entstehen, dann haben wir hier unseren Job gut erledigt.
%
Wenn wir dann mit der psinc-Funktion noch vertraut sind, haben wir den SigSys Koffer
bestens gefüllt.
%
Im Master Modul Digital Signal Processing werden wir uns dann eingehender damit
beschäftigen, wir könnten einen ersten Blick riskieren wollen:

\url{https://github.com/spatialaudio/digital-signal-processing-exercises/blob/master/dft/dft_windowing_tutorial/dft_windowing_tutorial.pdf}


\begin{figure}
\centering
\includegraphics[width=\textwidth]{../dtft/windowing_96D7F1EE47.pdf}
\caption{Aufgabe \ref{sec:96D7F1EE47}, Links: Signal im Zeitbereich,
rechts: DTFT-Betragsspektrum.
\texttt{windowing\_96D7F1EE47.ipynb}}
\label{fig:windowing_96D7F1EE47}
\end{figure}

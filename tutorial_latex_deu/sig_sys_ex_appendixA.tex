\clearpage
\section{Appendix A: Dirac Impuls: Austast- / Multiplikationseigenschaft}
%
\subsection{Zeitkontinuierlich}
Definition, kein eigentliches Riemannintegral:
\begin{mdframed}
Austasteigenschaft, Ausblendeigenschaft, englisch: sifting property (nicht: shifting)
\begin{align}
\int\limits_{t=-\infty}^{+\infty} x(t) \, \delta(t-\tau) \, \fsd t \stackrel{\mathrm{def}}= x(t=\tau)
\label{eq:AppA_SifitingCT}
\end{align}
\end{mdframed}
Speziell für $\tau=0$ folgt
\begin{align}
\int\limits_{t=-\infty}^{+\infty} x(t) \, \delta(t) \, \fsd t \stackrel{\mathrm{def}}= x(t=0),
\end{align}
und für $x(t)=1$ und $\tau=0$ folgt
\begin{align}
\int\limits_{t=-\infty}^{+\infty} \delta(t) \, \fsd t \stackrel{\mathrm{def}}= 1.
\end{align}
Aus der obigen Definition finden wir, dass
\begin{align}
\int\limits_{t=-\infty}^{+\infty} x(t) \, \delta(t-\tau) \, \fsd t \stackrel{\mathrm{def}}= x(\tau)\qquad
\int\limits_{t=-\infty}^{+\infty} x(\tau) \, \delta(t-\tau) \, \fsd t \stackrel{\mathrm{def}}= x(\tau)
\end{align}
zum gleichen Ergebnis führen, die Ausdrücke innerhalb des 'Integrals' bzgl.
der Definition also das gleiche machen, damit finden wir durch Vergleich der Integranden beider Integrale die
\begin{mdframed}
Multiplikationseigenschaft
\begin{align}
x(t) \, \delta(t-\tau) = x(\tau) \, \delta(t-\tau)
\end{align}
\end{mdframed}
Speziell für $\tau=0$ folgt
\begin{align}
x(t) \, \delta(t) = x(0) \, \delta(t).
\end{align}
%
Beispiel:
Austasteigenschaft
\begin{align}
\int\limits_{t=-\infty}^{+\infty} \cos(\frac{\pi}{4} t) \, \delta(t-\tau) \, \fsd t \stackrel{\mathrm{def}}=\cos(\frac{\pi}{4} \tau)
\end{align}
Multiplikationseigenschaft
\begin{align}
\cos(\frac{\pi}{4} t) \, \delta(t-\tau) = \cos(\frac{\pi}{4} \tau) \, \delta(t-\tau)
\end{align}

für spezielles $\tau=4$, Austasteigenschaft
\begin{align}
\int\limits_{t=-\infty}^{+\infty} \cos(\frac{\pi}{4} t) \, \delta(t-4) \, \fsd t \stackrel{\mathrm{def}}=\cos(\pi) = -1
\end{align}
%
Multiplikationseigenschaft
\begin{align}
\cos(\frac{\pi}{4} t) \, \delta(t-4) = \cos(\pi) \, \delta(t-4) = -\delta(t-4)
\end{align}

Von der Austasteigenschaft zur Faltung:
\begin{align}
&\int\limits_{\tau=-\infty}^{+\infty} x(\tau) \, \delta(\tau-t) \, \fsd \tau \stackrel{\mathrm{def}}= x(\tau=t)\text{ , ausgehend von \eq{eq:AppA_SifitingCT} mit } t \leftrightarrow \tau\nonumber\\
&\int\limits_{\tau=-\infty}^{+\infty} x(\tau) \, \delta(-(\tau-t)) \, \fsd \tau \stackrel{\mathrm{def}}= x(\tau=t)\text{ , weil }\delta(t)=\delta(-t)\nonumber\\
&\int\limits_{\tau=-\infty}^{+\infty} x(\tau) \, \delta(-\tau+t) \, \fsd \tau \stackrel{\mathrm{def}}= x(t) \text{ , Variante Austasteigenschaft, das ist Faltung mit Neutralelement}\nonumber\\
&\int\limits_{\tau=-\infty}^{+\infty} x(\tau) \, h(-\tau+t) \, \fsd \tau = y(t) \text{ , Faltungsintegral mit LTI-System Impulsantwort}
\end{align}




%
\newpage
\subsection{Zeitdiskret}
Dirac Impuls (Folge) diesmal als exaktes Signal definiert als
\begin{align}
\delta[k] =
\begin{cases}
1 & k=0\\
0 & \text{ sonst}
\end{cases}
\end{align}
und damit sind 'tatsächliche' Berechnungen möglich.
\begin{mdframed}
Austasteigenschaft, Ausblendeigenschaft, englisch: sifting property (nicht: shifting)
\begin{align}
\sum\limits_{k=-\infty}^{+\infty} x[k] \, \delta[k-\kappa] = x[k=\kappa]
\label{eq:AppA_SifitingDT}
\end{align}
\end{mdframed}
Speziell für $\kappa=0$ folgt
\begin{align}
\sum\limits_{k=-\infty}^{+\infty} x[k] \, \delta[k] = x[k=0],
\end{align}
und für $x[k]=1$ und $\kappa=0$ folgt
\begin{align}
\sum\limits_{\kappa=-\infty}^{+\infty} \delta[k] = 1.
\end{align}
Aus der obigen Definition finden wir, dass
\begin{align}
\sum\limits_{k=-\infty}^{+\infty} x[k] \, \delta[k-\kappa] = x[\kappa]\qquad
\sum\limits_{k=-\infty}^{+\infty} x[\kappa] \, \delta[k-\kappa] = x[\kappa]
%\int\limits_{-\infty}^{+\infty} f(t) \, \delta(t-\tau) \, \fsd t \stackrel{\mathrm{def}}= f(\tau)\qquad
%\int\limits_{-\infty}^{+\infty} f(\tau) \, \delta(t-\tau) \, \fsd t \stackrel{\mathrm{def}}= f(\tau)
\end{align}
zum gleichen Ergebnis führen, die Ausdrücke innerhalb der Summe also das gleiche
machen, damit finden wir die
\begin{mdframed}
Multiplikationseigenschaft
\begin{align}
x[k] \, \delta[k-\kappa] = x[\kappa] \, \delta[k-\kappa]
\end{align}
\end{mdframed}
Speziell für $\kappa=0$ folgt
\begin{align}
x[k] \, \delta[k] = x[0] \, \delta[k].
\end{align}
%
Beispiel:
\begin{align}
\sum\limits_{k=-\infty}^{+\infty} \cos(\frac{\pi}{4} k) \, \delta[k-\kappa] =\cos(\frac{\pi}{4} \kappa)\\
\cos(\frac{\pi}{4} k) \, \delta[k-\kappa] = \cos(\frac{\pi}{4} \kappa) \, \delta[k-\kappa]
\end{align}
für spezielles $\kappa=4$
\begin{align}
\sum\limits_{k=-\infty}^{+\infty} \cos(\frac{\pi}{4} k) \, \delta[k-4] =\cos(\pi) = -1\\
\cos(\frac{\pi}{4} k) \, \delta[k-4] = \cos(\pi) \, \delta[k-4] = -\delta[k-4]
\end{align}

Von der Austasteigenschaft zur zeitdiskreten Faltung:
\begin{align}
&\sum\limits_{\kappa=-\infty}^{+\infty} x[\kappa] \, \delta[\kappa-k] = x[\kappa=k]\text{ , ausgehend von \eq{eq:AppA_SifitingDT} mit } k \leftrightarrow \kappa\nonumber\\
&\sum\limits_{\kappa=-\infty}^{+\infty} x[\kappa] \, \delta[-(\kappa-k)] = x[\kappa=k]\text{ , weil }\delta[k]=\delta[-k]\nonumber\\
&\sum\limits_{\kappa=-\infty}^{+\infty} x[\kappa] \, \delta[-\kappa+k] = x[k] \text{ , Variante Austasteigenschaft, das ist Faltung mit Neutralelement}\nonumber\\
&\sum\limits_{\kappa=-\infty}^{+\infty} x[\kappa] \, h[-\kappa+k] = y[k] \text{ , Faltungsintegral mit LTI-System Impulsantwort}
\end{align}

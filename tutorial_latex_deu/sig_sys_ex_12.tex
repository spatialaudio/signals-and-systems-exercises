\newpage
\section{UE 12: Beschreibung diskreter Systeme im Spektralbereich, Eigenschaften diskreter LTI-Systeme}

Wir haben den SigSys Werkzeug Koffer bestens aufgefüllt mit den fundamentalen
Werkzeugen, die wir als Naturwissenschaftler*innen in modifizierter Form
immer wieder antreffen werden. Es ist eh immer die gleiche Message: je mehr wir
die Basics vom Wesen verstanden haben, desto einfacher fällt uns das Adaptieren auf
kompliziertere und/oder praktische Probleme.

Die Idee dieser letzten Übung ist, zwei Betrachtungen, die wir aus der
zeitkontinuierlichen Welt schon kennen, hier für zeitdiskrete Signale
durchzuspielen. Einerseits um zu wiederholen und Gemeinsamkeiten und Unterschiede
klar zu machen. Andererseits, weil es sinnvoll ist,
die prototypische Implementierung der Werkzeuge in Programmcode
ins Licht zu rücken.
Hin und wieder waren ja in den letzten Übungen schon Code-Schnipsel
enthalten, aber nicht zu viele, damit wir nicht dem Spieltrieb verfallen und
die wichtigen analytischen Rechnungen nicht aus den Augen verlieren.

Hier nun zwei Aufgaben, die wir im Detail nicht mehr analytisch ausrechnen,
sie sollten aber noch auf dem Papier
handhabbar sein (der Autor mutmaßt, weil er es selbst zu faul dafür war).
Darum soll es uns hier nicht gehen, sondern vielmehr nochmal zwei wichtige
Aspekte zur Beschreibung von zeitdiskreten Systemen herauszuarbeiten.





\clearpage
\subsection{DFT Interpolation zur DTFT}
\label{sec:6337B75DF2}
\begin{Ziel}
Wir wollen die DFT zu DTFT Rekonstruktion vertiefen. Wir hatten kennengelernt,
dass die vier Fouriertransformierten durch Abtastung und Rekonstruktion
miteinander verknüpfbar sind. Hier wollen wir nun ein DFT-Linienspektrum
interpolieren zu einem kontinuierlichen DTFT-Spektrum. Anders herum, könnten
wir uns dies als ideale Abtastung des DTFT-Spektrums hin zum DFT-Spektrum erklären.
\end{Ziel}
\textbf{Aufgabe} {\tiny 6337B75DF2}: Stellen Sie für das FIR-Filter
aus Aufgabe 10.2 die Betragsfrequenzgänge der
\begin{itemize}
  \setlength\itemsep{-0.5em}
  \item DTFT
  \item DFT
  \item DFT des mit Nullen aufgefüllten Signals auf Länge 16
  \item DFT des mit Nullen aufgefüllten Signals auf Länge 32
\end{itemize}
mit ihren jeweiligen Zeitfolgen grafisch dar. Wir können zur numerischen Lösung
die unten verlinkten Skripte nutzen.
Interpretieren Sie die Grafiken.
\begin{Werkzeug}
Abtastungs-/Rekonstruktionsmodell, Faltung im Spektralbereich ist Multiplikation
im Zeitbereich, Periodische Sinc Funktion.
%
Source Code passend zur Aufgabe unter

\url{https://github.com/spatialaudio/signals-and-systems-exercises/blob/master/dft/}

entweder als Matlab Skript oder Jupyter Notebook
\texttt{dtf2dtft\_6337B75DF2.m /.ipynb}

\end{Werkzeug}
\begin{Ansatz}
Wir setzen die DFT Synthese ($\tilde{x}, \tilde{X}$ als Hinweis, dass $N$-periodisch)
%
\begin{equation}
\tilde{x}[k]=\frac{1}{N} \sum_{\mu=0}^{N-1}\tilde{X}[\mu]\,\e^{+\im\frac{2\pi}{N}k\mu}
\end{equation}
%
in die DTFT Analyse ein und summieren aber nur $0 \leq k \leq N-1$
\begin{align}
X(\Omega)&=\sum_{k=-\infty}^\infty \tilde{x}[k]\cdot\e^{-\im\Omega k}
=\sum_{k=0}^{N-1}\frac{1}{N}\sum_{\mu=0}^{N-1}\tilde{X}[\mu]\,\e^{+\im\frac{2\pi}{N}k\mu}\cdot\e^{-\im\Omega k}
=\sum_{\mu=0}^{N-1}\tilde{X}[\mu]\,\frac{1}{N}\,\sum_{k=0}^{N-1}\e^{-\im k\left(\Omega-\frac{2\pi}{N}\mu\right)}.
\end{align}
%
Das machen wir, weil wir wissen wollen, wie die Doppelsumme
aus dem eigentlich $N$-periodischen $\tilde{x}[k]$,
ein nicht periodisches, endliches Signal $x[k]$ der Länge $N$ erzeugt,
für das ja eine DTFT mit kontinuierlicher Frequenzvariable $\Omega$ existiert.
%
Im Zeitbereich müsste dies einer Multiplikation
\begin{align}
x[k] = \mathrm{rect}_N[k] \cdot \tilde{x}[k]
\end{align}
entsprechen.

\end{Ansatz}
\begin{ExCalc}
Die hintere Summe über $k$ ist wieder die geometrische Reihe, vgl. (10.68), (10.69)
und \cite[(3-39)]{Lyons2011}.
Wir schreiben das noch einmal in aller Ausführlichkeit um, das ist gute Übung,
weil ähnliche Rechenschritte immer wieder auftauchen in SigSys.
%
\begin{align}
X(\Omega)&=\sum_{\mu=0}^{N-1}\tilde{X}[\mu]\cdot\frac{1}{N}\cdot\frac{1-\e^{-\im\left(\Omega-\frac{2\pi}{N}\mu\right)N}}{1-\e^{-\im\left(\Omega-\frac{2\pi}{N}\mu\right)}}\\
&=\sum_{\mu=0}^{N-1}\tilde{X}[\mu]\cdot\frac{1}{N}\cdot\frac{\e^{-\im\frac{\left(\Omega-\frac{2\pi}{N}\mu\right)N}{2}}}{\e^{-\im\frac{\Omega-\frac{2\pi}{N}\mu}{2}}}\cdot\frac{\e^{\im\frac{\left(\Omega-\frac{2\pi}{N}\mu\right)N}{2}}-\e^{-\im\frac{\left(\Omega-\frac{2\pi}{N}\mu\right)N}{2}}}{\e^{\im\frac{\Omega-\frac{2\pi}{N}\mu}{2}}-\e^{-\im\frac{\Omega-\frac{2\pi}{N}\mu}{2}}}\\
&=\sum_{\mu=0}^{N-1}\tilde{X}[\mu]\cdot\frac{1}{N}\cdot\e^{-\im\frac{\left(\Omega-\frac{2\pi}{N}\mu\right)(N-1)}{2}}\cdot\frac{\e^{\im\frac{\left(\Omega-\frac{2\pi}{N}\mu\right)N}{2}}-\e^{-\im\frac{\left(\Omega-\frac{2\pi}{N}\mu\right)N}{2}}}{\e^{\im\frac{\Omega-\frac{2\pi}{N}\mu}{2}}-\e^{-\im\frac{\Omega-\frac{2\pi}{N}\mu}{2}}}.
\end{align}
%
Mit $2\im\cdot\sin(x)=\e^{\im x}-\e^{-\im x}$ können wir vereinfachen
\cite[(2.41)]{Moeser2011}, vgl. \cite[(2.142)]{Rabiner1975}
%
\begin{equation}
X(\Omega)=\sum_{\mu=0}^{N-1}\tilde{X}[\mu]\cdot\e^{-\im\frac{\left(\Omega-\frac{2\pi}{N}\mu\right)(N-1)}{2}}\cdot\frac{1}{N}\cdot\frac{\sin\left(N\frac{\Omega-\frac{2\pi}{N}\mu}{2}\right)}{\sin\left(\frac{\Omega-\frac{2\pi}{N}\mu}{2}\right)}.
\end{equation}
%
Wir stoßen wieder auf die  periodische Sinc Funktion
%
\begin{align}
\text{psinc}_N(\Omega)=\begin{cases}\frac{1}{N}\cdot\frac{\sin\left(\frac{N}{2}\Omega\right)}{\sin\left(\frac{1}{2}\Omega\right)}&\text{für }\Omega\neq2\pi m\\
(-1)^{m(N-1)}&\text{für }\Omega=2\pi m\end{cases},\,\,m\in\mathbb{Z},
\end{align}
%
(Matlab und Python's scipy \texttt{diric(Omega,N)}) und einen Phasenterm (dieser
taucht auf, weil $x[k]$ bei $k=0$ beginnt.
In Übung 7 auf Seite 17 hatten wir es hingegen mit axialsymmetrischen Signale zu
tun, daher kein Phasenterm.)
%
Wir können daher umschreiben zu
%
\begin{equation}
\label{eq:6337B75DF2_DFT2DTFT}
X(\Omega)=\sum_{\mu=0}^{N-1}\tilde{X}[\mu]\cdot\e^{-\im\frac{\left(\Omega-\frac{2\pi}{N}\mu\right)(N-1)}{2}}\cdot\text{psinc}_N\left(\Omega-\frac{2\pi}{N}\mu\right)
\end{equation}
und nochmal die Argumente anders angeordnet
\begin{equation}
X(\Omega)=\frac{1}{N}\sum_{\mu=0}^{N-1}\tilde{X}[\mu\rightarrow \Omega_\mu = \frac{2\pi}{N} \mu]\cdot
\underbrace{\e^{-\im\frac{\left(-\frac{2\pi}{N}\mu+\Omega\right)(N-1)}{2}}\cdot\text{psinc}_N\left(-\frac{2\pi}{N}\mu + \Omega\right)\cdot N}_{H_\mathrm{r}\left(-\Omega_\mu+\Omega\right)}.
\end{equation}

Erinnern wir uns an Abtastung von Spektren. Die
jetzige Formelstruktur hat doch sehr starke Ähnlichkeit mit (7.60):
Ein Linienspektrum (DFT) wird mit einer Sinc bzw. hier jetzt mit der periodischen
Sinc Funktion interpoliert zu einem kontinuierlichen Spektrum (DTFT).
Wenn wir genau hinschauen, sehen wir dass es eine spezielle, diskrete Faltung zweier
Spektren ist (gedrehte und geschobene Hilfsvariable ist $\Omega_\mu$).
%
Eine diskrete (Zeit)-Folge oder eine Linienspektrum jeweils zu kontinuierlich
machen heisst in unserem
SigSys Abtastungsmodell Rekonstruktion, die als Interpolation
aufgefasst werden kann, vgl. Abb. 7.1 und 7.2.
Das in obiger Faltung vorkommende Rekonstruktionsspektrum
\begin{align}
H_\mathrm{r}(\Omega) =
\e^{-\im\frac{\Omega(N-1)}{2}}\cdot\text{psinc}_N\left(\Omega\right)\cdot N =
\e^{-\im\frac{\Omega(N-1)}{2}}
\frac{\sin\left(\frac{N}{2}\Omega\right)}{\sin\left(\frac{1}{2}\Omega\right)}
\DTFT
h_\mathrm{r}[k] = \mathrm{rect}_N[k]
\end{align}
ist die uns bestens bekannte DTFT Korrespondenz der Rechteckfolge der Länge $N$.
Das erfüllt doch prima unsere obige Erwartungshaltung, dass die obige Faltung
im Spektralbereich im Zeitbereich die Signalmultiplikation
\begin{align}
x[k] = h_\mathrm{r}[k]  \cdot \tilde{x}[k] = \mathrm{rect}_N[k] \cdot \tilde{x}[k]
\end{align}
abbilden muss.

\end{ExCalc}
\begin{Loesung}
Die Rekonstruktion \eq{eq:6337B75DF2_DFT2DTFT}
vom DFT-Linienspektrum / periodisches Zeitsignal zum
DTFT-Spektrum / einmaliges Zeitsignal steht daher in enger Analogie zu der
Rekonstruktion aus Übung 7.2 Fourierreihe (Linienspektrum) zu
Fouriertransformation (kontinuierliches Spektrum). Es ist im Grunde das
direkte Pendant für Folgen, also zeitdiskrete Signale.

Diese Interpolation spielt in der Praxis eine wichtige Rolle. Wir werden sie
sehr, sehr oft anwenden, und es ist wichtig, dass uns das bewusst ist.
Ein mögliches Beispiel:
In der Praxis messen wir
die Impulsantwort von BIBO-stabilen nichtrekursiven aber auch rekursiven Systemen,
und zwar so dass sie mit endlicher Länge in einen Rechner abgelegt werden kann.
%
Das können wir auch so machen, solange wir sicherstellen, dass
in unserem Messfenster die Impulsantwort
bis ins Messrauschen abgeklungen ist.
In einer großen Kirche müssten wir dann schon mal 20 s Zeitfenster einplanen,
damit wir den kompletten Nachhall als Antwort auf einen lauten Knall erfassen
können (diesen Impuls zur Anregung des Raums würden wir heutzutage aus technischen,
aber viel wichtiger: sozialverträglichen Gründen nicht mehr mit einer Handfeuerwaffe
erzeugen ;-) ).
%
Wenn wir uns für die Systemeigenschaften, wie den Frequenzgang der gemessenen,
endlichen Impulsantwort der Länge $N$ interessieren, müssten wir erst die
DFT und dann mühsam die DTFT-Interpolationsformel berechnen. Das geht deutlich
eleganter mit sogenanntem Zeropadding, also Auffüllen der Impulsantwort mit
Nullen und nachfolgender DFT mit resultierender Länge.

Je mehr Nullen wir auffüllen, desto genauer tasten wir das DTFT Spektrum ab.
Es ist unsere Aufgabe eine ausreichende Anzahl von Punkten auszuwählen,
so dass wir in der grafischen Auflösung diese Abtastung nicht mehr
von der tatsächlichen DTFT unterscheiden können.

Dies ist beispielhaft in \fig{fig:6337B75DF2} anschaulich dargestellt.
Wir nehmen unser bekanntes FIR-Filter aus Aufgabe 10.2.
In \fig{fig:DTFT_6337B75DF2} ist das Zeitsignal und das exakte DTFT-Betragsspektrum
skizziert (wir hatten es analytisch ausgerechnet, kennen es also sehr gut).
%
Wir könnten diese 11 Samples des FIRs einer 11-Punkte DFT unterziehen und erhalten
das DFT-Spektrum. Linienspektrum heißt in der anderen Domäne immer Periodisierung.
Das ist in \fig{fig:DFT_6337B75DF2} dargestellt. 11 Werte im Zeitbereich
können auch nur maximal 11 Werte exakt im Frequenzbereich erklären. Dies
entspricht kritischer Abtastung, d.h. die orange Zeitfolge wird nicht durch
zeitliches Aliasing kaputt gemacht (dies ist ja nicht mehr reparabel).
In den 11 DFT-Koeffizienten ist daher die komplette Information zum DTFT
Spektrum enthalten. Die obige Interpolationsformel  \eq{eq:6337B75DF2_DFT2DTFT}
verlinkt DFT und DTFT Spektrum.

In den \fig{fig:DFT_zeropad16_6337B75DF2} und \fig{fig:DFT_zeropad32_6337B75DF2}
sind nun zwei Fälle von Zeropadding und längerer DFT dargestellt.
Für den Fall einer DFT-Periodizität von 32 ist der DFT-Betragsfrequenzgang
schon sehr aufschlussreich. Wenn wir die Punkte in diesem Stem-Plot (strenggenommen
nicht erlaubt) verbinden würden, wären wir schon sehr nah am tatsächlichen
DTFT-Spektrum. Das wird in der Praxis sehr oft so gemacht, spricht nichts dagegen,
wir müssen nur klar haben, was wir da treiben.

Ein beliebter Denkfehler ist, dass Zeropadding die DFT-Auflösung erhöht.
Weil wir uns sauber die Abtast-/Rekonstruktionsmodelle und Links zwischen den
Fouriertransformationen erarbeitet haben, wissen wir, dass das eine
falsche Behauptung ist.
Wir erfinden mit Zeropadding keine neue Information! Mehr als 11 Samples FIR
haben wir ja nicht.
Wir rekonstruieren/interpolieren
nur anders, im Falle von Zeropadding eben nicht ideal sondern so,
dass periodische Wiederholungen im Zeitsignal erhalten bleiben.

\end{Loesung}




\begin{figure*}[h!]
\centering
\begin{subfigure}{0.495\textwidth}
\includegraphics[width=\textwidth]{../dft/DTFT_6337B75DF2.pdf}
\caption{FIR-Filter mit Länge 11 oben, DTFT-Betragsfrequenzgang unten,
Aufgabe 10.2}
\label{fig:DTFT_6337B75DF2}
\end{subfigure}
\begin{subfigure}{0.495\textwidth}
\includegraphics[width=\textwidth]{../dft/DFT_zeropad32_6337B75DF2.pdf}
\caption{Periodisiertes Signal mit Länge 32 oben. DFT-Betragsfrequenzgang unten.}
\label{fig:DFT_zeropad32_6337B75DF2}
\end{subfigure}

\begin{subfigure}{0.495\textwidth}
\includegraphics[width=\textwidth]{../dft/DFT_zeropad16_6337B75DF2.pdf}
\caption{Periodisiertes Signal mit Länge 16 oben. DFT-Betragsfrequenzgang unten.}
\label{fig:DFT_zeropad16_6337B75DF2}
\end{subfigure}
\begin{subfigure}{0.495\textwidth}
\includegraphics[width=\textwidth]{../dft/DFT_6337B75DF2.pdf}
\caption{FIR-Filter, also endliches Signal mit Länge 11 oben.
DFT-Betragsfrequenzgang unten.}
\label{fig:DFT_6337B75DF2}
\end{subfigure}
\caption{Signalverläufe und Betragsspektren für Aufgabe \ref{sec:6337B75DF2}.
Aus der DFT \fig{fig:DFT_6337B75DF2} kann die DTFT \fig{fig:DTFT_6337B75DF2}
exakt interpoliert werden. Dies entspricht im Abtast-/Rekonstruktionsmodell
der idealen Rekonstruktion nach kritischer Abtastung.
%
Wir könnten auch anders herum argumentieren: ideale
Abtastung der DTFT führt zur DFT und damit im Zeitbereich zur Periodisierung.
Die DFTs in \fig{fig:DFT_zeropad32_6337B75DF2} und
\fig{fig:DFT_zeropad16_6337B75DF2} sind Beispiele dafür mit Überabtastung, d.h.
mehr Abtastpunkten pro $2\pi$ als erforderlich. Kritische Abtastung erfolgt mit
11 Abtastwerten in \fig{fig:DFT_6337B75DF2}, weniger würde eine Überlappung
von Zeitfolgenwerten und damit zeitliches Aliasing hervorrufen.
}
\label{fig:6337B75DF2}
\end{figure*}









\clearpage
\subsection{Zerlegung in Reihenschaltung aus Minimalphasensystem und Allpass}
\label{sec:55A8105469}
\begin{Ziel}
Ähnlich wie Aufgabe 6.2 wollen wir uns für zeitdiskrete Systeme---deren
Kausalität und Stabilität wir voraussetzen wollen bzw. sicher stellen müssen---eine
Zerlegung in Minimalphasensystem und Allpass vornehmen. Dazu benutzen
wir Systeme dritter Ordnung, d.h. Systeme
mit 3 Polen und 3 Nullstellen. Die Vorgehensweise
ist analog zur zeitkontinuierlichen Welt. Spiegelung von Nullstellen in
der $s$-Ebene überträgt sich im $z$-Bereich auf
Inversion des Betrags der Nullstelle unter Beibehaltung des
Winkels.
%

Die benutzte \texttt{plot\_dtlti\_analysis()} Funktion zur Erzeugung der
typischen System-Kenngrafiken (Impulsantwort, Sprungantwort, Frequenzgänge,
Pol-Nullstellendiagramm) ist eine prototypische
Realisierung, die wir nach Bedarf weiter modifizieren können. Die Python-Variante
enthält ein paar mehr Formatierungsaspekte, während die Matlab-Variante
minimal gehalten und vielleicht übersichtlicher als ihr On-board
\texttt{fvtool} ist.
%
\end{Ziel}
\textbf{Aufgabe} {\tiny 55A8105469}:
Gegeben sind ein sogenanntes Mixed-Phase System $H_\mathrm{mix}(z)$
(Nullstellen sowohl innerhalb als auch außerhalb des
Einheitskreises) und ein maximalphasiges System $H_\mathrm{max}(z)$
(alle Nullstellen außerhalb des Einheitskreises). Alle Pole liegen im inneren
des Einheitskreises, die Systeme sind damit stabil sind, weil wir Kausalität
fordern.

Mixed-Phase System $H_\mathrm{mix}(z)$
\begin{align}
z_{1,2} = 2 \e^{\pm\im\frac{3}{4}\pi},\quad z_3 = \frac{1}{2},\quad
p_{1,2} = \pm\frac{3}{4}\im,\quad p_3 = -\frac{1}{3},\quad
g = + \frac{1}{2}
\end{align}
%z = [2*exp(+1j*3*pi/4), 2*exp(-1j*3*pi/4), 1/2];
%p = [3/4*1j, -3/4*1j, -1/3];
%k = 1/2;

Maximum-Phase System $H_\mathrm{max}(z)$
\begin{align}
z_{1,2} = 2 \e^{\pm\im\frac{3}{4}\pi},\quad z_3 = 2,\quad
p_{1,2} = \pm\frac{3}{4}\im,\quad p_3 = -\frac{1}{3},\quad
g = - \frac{1}{4}
\end{align}
%z = [2*exp(+1j*3*pi/4), 2*exp(-1j*3*pi/4), 2];
%p = [3/4*1j, -3/4*1j, -1/3];
%k = -1/4;

Zerlegen Sie diese Systeme in eine Reihenschaltung aus Minimalphasensystem
$H_\mathrm{min}(z)$
und ein jeweils spezifisches Allpasssystem mit frequenzunabhängiger
Verstärkung 1 (also Pegel 0dB).
Es gilt $|H_\mathrm{max}(z)| = |H_\mathrm{mix}(z)| = |H_\mathrm{min}(z)|$.
%
Stellen Sie alle wichtigen Systemgrafiken des Zeit- und Frequenzbereichs dar,
also Impuls- und Sprungantwort, Frequenzgänge (Pegel, Phase, Gruppenlaufzeit)
und Pol / Nullstellendiagramm.
%
%Wir können zur numerischen Lösung die unten verlinkten Skripte nutzen.
Interpretieren Sie die Grafiken.



\begin{Werkzeug}
Reihenschaltung, Minimal/Maximalphasensystem, Allpass-Filter

Source Code passend zur Aufgabe unter

\url{https://github.com/spatialaudio/signals-and-systems-exercises/tree/master/z_system_analysis/}

entweder als Matlab Skript oder Jupyter Notebook

\texttt{minphase\_allpass\_55A8105469.m / .ipynb}

Jupyter Notebook Plot-Routine \texttt{plot\_dtlti\_analysis()} findet sich
\url{sig_sys_tools.py}


\end{Werkzeug}
\begin{Ansatz}
Um aus gemischtphasigen Systemen ein minimalphasiges zu machen, müssen wir
alle Nullstellen in den Einheitskreis bringen. Dies gelingt mit der Idee, dass
die neue Nullstelle bzgl. des Betragsfrequenzgangs
das gleiche machen muss wie die alte.
Wir schauen uns das hier nicht im Detail an (dafür siehe empfohlene SigSys-Bücher):
die Inversion des Betrags leistet genau das. Aus einer maximalphasigen Nullstelle
wird also eine minimalphasige Nullstellen mittels
\begin{align}
z_\mathrm{maxphase} = |z_\mathrm{maxphase}| e^{\im\angle z_\mathrm{maxphase}}\\
 z_\mathrm{minphase} = \frac{1}{|z_\mathrm{maxphase}|} e^{\im\angle z_\mathrm{maxphase}}
\end{align}
%
Für das gesuchte Minimalphasensystem $H_\mathrm{min}(z)$ finden wir also
\begin{align}
&z_{1,2} = \frac{1}{2} \e^{\pm\im\frac{3}{4}\pi},\quad z_3 = \frac{1}{2}\\
&p_{1,2} = \pm\frac{3}{4}\im,\quad p_3 = -\frac{1}{3}\\
&g = +2
\end{align}
%z = [1/2*exp(+1j*3*pi/4), 1/2*exp(-1j*3*pi/4), 1/2];
%p = [3/4*1j, -3/4*1j, -1/3];
%k = 2;
Den Verstärkungsfaktor finden wir aus der Forderung
$|H_\mathrm{max}(z)| = |H_\mathrm{mix}(z)| = |H_\mathrm{min}(z)|$, was
sich halbwegs elegant für $z=\e^{\im \Omega}$ bei $\Omega=0$ also Gleichanteil
(DC, $\frac{15+6\sqrt(2)}{25} \approx 0.9394$) finden lässt.
%
\end{Ansatz}
\begin{ExCalc}
Nachdem das Minimalphasensystem $H_\mathrm{min}(z)$ vollständig beschrieben ist,
finden wir nun die
jeweiligen Allpässe $H_\mathrm{all,\cdot}(z)$, damit die Reihenschaltungen gelten
\begin{align}
H_\mathrm{mix}(z) = H_\mathrm{min}(z) \cdot H_\mathrm{all,mix}(z)\\
H_\mathrm{max}(z) = H_\mathrm{min}(z) \cdot H_\mathrm{all,max}(z)
\end{align}
%
Für den Allpass benutzen wir alle 'maximalphasigen' Nullstellen (also außerhalb des
Einheitskreises) eines gemischtphasigen Systems und platzieren Pole mit invertiertem
Betrag dieser jeweiligen Nullstellen. Dies führt dann zu einem frequenzunabhängigen
Betragsspektrum. Der Gain-Faktor wird dann so gefunden, dass Betrag Eins ist.
%
%In unserem Beispiel:

Allpass $H_\mathrm{all,mix}(z)$ (System 2. Ordnung, weil Berücksichtigung zweier
maximalphasige Nullstellen)
\begin{align}
&z_{1,2} = \frac{2}{1} \e^{\pm\im\frac{3}{4}\pi}\\
&p_{1,2} = \frac{1}{2} \e^{\pm\im\frac{3}{4}\pi}\\
&g=\frac{1}{4}
\end{align}
%z = [2*exp(+1j*3*pi/4), 2*exp(-1j*3*pi/4), 0];
%p = [1/2*exp(+1j*3*pi/4), 1/2*exp(-1j*3*pi/4), 0];
%k = 1/4;

Allpass $H_\mathrm{all,max}(z)$ (System 3. Ordnung, weil drei maximalphasige
Nullstellen)
\begin{align}
&z_{1,2} = \frac{2}{1} \e^{\pm\im\frac{3}{4}\pi},\quad z_3 = \frac{2}{1}\\
&p_{1,2} = \frac{1}{2} \e^{\pm\im\frac{3}{4}\pi},\quad p_3 = \frac{1}{2}\\
&g=-\frac{1}{8}
\end{align}
%z = [2*exp(+1j*3*pi/4), 2*exp(-1j*3*pi/4), 2]; %all zeros from max phase
%p = [1/2*exp(+1j*3*pi/4), 1/2*exp(-1j*3*pi/4), 1/2];
%k = -1/8;
\end{ExCalc}
\begin{Loesung}
Damit haben wir alle Systembeschreibungen zusammen,
bei Bedarf könnten wir dies noch in die gebrochen rationale Form
mit Koeffizienten $b,a$ bringen. Dann vielleicht sogar so, dass Systeme 2. Ordnung
(second order structures, sos) kaskadiert sind. Das Matlab Skript liefert
genau diese Koeffizienten im Kommandofenster zurück.

Uns geht es hier eher darum mit der Pol/Nullstellen/Verstärkung Darstellung
die gewünschten Grafiken zu erstellen.

Die Zerlegung können wir, ähnlich wie in Aufgabe 6.2, auch mit
Pol / Nullstellendiagrammen grafisch erarbeiten, siehe Bilder unten.
Wir beginnen wieder ganz links mit dem jeweiligen gemischtphasigen System.
Dort tragen wir alle Pole und Nullstellen ein, die dieses System an sich
hat. Allen Nullstellen außerhalb des Einheitskreis wird nun jeweils
eine minimalphasige Nullstelle (durch oben beschriebene Betragsinversion)
zugeordnet, die wir in das Diagramm eintragen, und auch gleich
mit Polstellen an der gleichen Stelle kompensieren.
%
In der oberen Grafik ist das mit den orangen
Nullstellen und Polen angedeutet (in der unteren Grafik wurde drauf verzichtet,
damit es nicht zu unübersichtlich wird). Die zusätzlich eingezeichneten Pole
und genau die Nullstellen außerhalb des Einheitskreises werden nun dem Allpass
im ganz rechten Bild zugeordnet.
Die verbleibenden Pole und Nullstellen (alles im Einheitskreis, damit
Minimalphasigkeit erfüllt ist) werden dem mittleren Bild, also $H_\mathrm{min}(z)$
zugeordnet.

\begin{center}
\begin{tikzpicture}[scale=1.25]
\def \tic {0.05}
%
\begin{scope}  % mixed phase
\def \rocmax{1.5} % sketch of outer roc domain
%
% basic diagram features:
\filldraw[even odd rule,C2!50] (0,0) circle(0.75) decorate
[decoration={snake, segment length=15pt, amplitude=1pt}]
{(0,-3pt) circle(\rocmax)}; % sketch the roc domain
%
\draw[help lines, C7!50, step=0.25cm] (-\rocmax,-\rocmax) grid (\rocmax,\rocmax);
%
\draw[C3, thick] (0,0) circle(1);  % unit circle, i.e. DTFT domain
\draw (1+2*\tic,-2*\tic) node{$1$}; % indicate that this is the unit circle
%\draw[->] (-1.75,0)--(1.75,0) node[right]{$\Re\{z\}$}; % axis label
\draw[->] (0,-1.75)--(0,1.75) node[above]{$\Im\{z\}$}; % axis label
%
\draw[->] (-1.75,0)--(1.75,0) node[right]{$=$}; % axis label
%
\draw (1.2*0.86602540378*\rocmax,1.2*0.5*\rocmax) node[C2!75]{KB}; % indicate the roc
%
% the z-transfer function specific stuff:
\draw (1,1) node[black]{$g=\frac{1}{2}$}; % indicate gain factor
%
% draw the poles / zeros and if desired ticks:
\draw[C0, ultra thick] (0,+0.75) node{\Huge $\times$};
\draw[C0, ultra thick] (0,-0.75) node{\Huge $\times$};
\draw[C0, ultra thick] (-0.333,0) node{\Huge $\times$};
\draw[C0, ultra thick] (-1.4142,+1.4142) node{\Huge $\circ$};
\draw[C0, ultra thick] (-1.4142,-1.4142) node{\Huge $\circ$};
\draw[C0, ultra thick] (0.5,0) node{\Huge $\circ$};
%
\draw[C1, ultra thick] (-0.3536,-0.3536) node{\Huge $\times$};
\draw[C1, ultra thick] (-0.3536,+0.3536) node{\Huge $\times$};
\draw[C1, ultra thick] (-0.3536,-0.3536) node{\Huge $\circ$};
\draw[C1, ultra thick] (-0.3536,+0.3536) node{\Huge $\circ$};
%
\draw (0,-2) node{$H_\mathrm{mix}(z)$};
%
\end{scope}
%
%
%
\begin{scope}[xshift=4cm]  % min phase
\def \rocmax{1.5} % sketch of outer roc domain
%
% basic diagram features:
\filldraw[even odd rule,C2!50] (0,0) circle(0.75) decorate
[decoration={snake, segment length=15pt, amplitude=1pt}]
{(0,-3pt) circle(\rocmax)}; % sketch the roc domain
%
\draw[help lines, C7!50, step=0.25cm] (-\rocmax,-\rocmax) grid (\rocmax,\rocmax);
%
\draw[C3, thick] (0,0) circle(1);  % unit circle, i.e. DTFT domain
\draw (1+2*\tic,-2*\tic) node{$1$}; % indicate that this is the unit circle
%\draw[->] (-1.75,0)--(1.75,0) node[right]{$\Re\{z\}$}; % axis label
\draw[->] (0,-1.75)--(0,1.75) node[above]{$\Im\{z\}$}; % axis label
%
\draw[->] (-1.75,0)--(1.75,0) node[right]{$\cdot$}; % axis label
%
\draw (1.2*0.86602540378*\rocmax,1.2*0.5*\rocmax) node[C2!75]{KB}; % indicate the roc
%
% the z-transfer function specific stuff:
\draw (1,1) node[black]{$g=2$}; % indicate gain factor
%
% draw the poles / zeros and if desired ticks:
\draw[C0, ultra thick] (0,+0.75) node{\Huge $\times$};
\draw[C0, ultra thick] (0,-0.75) node{\Huge $\times$};
\draw[C0, ultra thick] (-0.333,0) node{\Huge $\times$};
\draw[C0, ultra thick] (-0.3536,+0.3536) node{\Huge $\circ$};
\draw[C0, ultra thick] (-0.3536,-0.3536) node{\Huge $\circ$};
\draw[C0, ultra thick] (0.5,0) node{\Huge $\circ$};
%
\draw (0,-2) node{$H_\mathrm{min}(z)$};
%
\end{scope}
%
%
%
\begin{scope}[xshift=8cm]  % allpass for mixed phase
\def \rocmax{1.5} % sketch of outer roc domain
%
% basic diagram features:
\filldraw[even odd rule,C2!50] (0,0) circle(0.5) decorate
[decoration={snake, segment length=15pt, amplitude=1pt}]
{(0,-3pt) circle(\rocmax)}; % sketch the roc domain
%
\draw[help lines, C7!50, step=0.25cm] (-\rocmax,-\rocmax) grid (\rocmax,\rocmax);
%
\draw[C3, thick] (0,0) circle(1);  % unit circle, i.e. DTFT domain
\draw (1+2*\tic,-2*\tic) node{$1$}; % indicate that this is the unit circle
\draw[->] (-1.75,0)--(1.75,0) node[right]{$\Re\{z\}$}; % axis label
\draw[->] (0,-1.75)--(0,1.75) node[above]{$\Im\{z\}$}; % axis label
%
\draw (1.2*0.86602540378*\rocmax,1.2*0.5*\rocmax) node[C2!75]{KB}; % indicate the roc
%
% the z-transfer function specific stuff:
\draw (1,1) node[black]{$g=\frac{1}{4}$}; % indicate gain factor
%
% draw the poles / zeros and if desired ticks:
\draw[C0, ultra thick] (-0.3536,-0.3536) node{\Huge $\times$};
\draw[C0, ultra thick] (-0.3536,+0.3536) node{\Huge $\times$};
\draw[C0, ultra thick] (-1.4142,+1.4142) node{\Huge $\circ$};
\draw[C0, ultra thick] (-1.4142,-1.4142) node{\Huge $\circ$};
%
\draw (0,-2) node{$H_\mathrm{all,mix}(z)$};
%
\end{scope}
%
\end{tikzpicture}
\end{center}
%
%
%
\begin{center}
\begin{tikzpicture}[scale=1]
\def \tic {0.05}
%
\begin{scope}  % max phase
\def \rocmax{2} % sketch of outer roc domain
%
% basic diagram features:
\filldraw[even odd rule,C2!50] (0,0) circle(0.75) decorate
[decoration={snake, segment length=15pt, amplitude=1pt}]
{(0,-3pt) circle(\rocmax)}; % sketch the roc domain
%
\draw[help lines, C7!50, step=0.25cm] (-\rocmax,-\rocmax) grid (\rocmax,\rocmax);
%
\draw[C3, thick] (0,0) circle(1);  % unit circle, i.e. DTFT domain
\draw (1+2*\tic,-2*\tic) node{$1$}; % indicate that this is the unit circle
%\draw[->] (-1.75,0)--(1.75,0) node[right]{$\Re\{z\}$}; % axis label
\draw[->] (0,-1.75)--(0,1.75) node[above]{$\Im\{z\}$}; % axis label
%
\draw[->] (-1.75,0)--(2.25,0) node[right]{$=$}; % axis label
%
\draw (1.2*0.86602540378*\rocmax,1.2*0.5*\rocmax) node[C2!75]{KB}; % indicate the roc
%
% the z-transfer function specific stuff:
\draw (1,1) node[black]{$g=-\frac{1}{4}$}; % indicate gain factor
%
% draw the poles / zeros and if desired ticks:
\draw[C0, ultra thick] (0,+0.75) node{\Huge $\times$};
\draw[C0, ultra thick] (0,-0.75) node{\Huge $\times$};
\draw[C0, ultra thick] (-0.333,0) node{\Huge $\times$};
\draw[C0, ultra thick] (-1.4142,+1.4142) node{\Huge $\circ$};
\draw[C0, ultra thick] (-1.4142,-1.4142) node{\Huge $\circ$};
\draw[C0, ultra thick] (2,0) node{\Huge $\circ$};
%
\draw (0,-2) node{$H_\mathrm{max}(z)$};
%
\end{scope}
%
%
%
\begin{scope}[xshift=5cm]  % min phase
\def \rocmax{2} % sketch of outer roc domain
%
% basic diagram features:
\filldraw[even odd rule,C2!50] (0,0) circle(0.75) decorate
[decoration={snake, segment length=15pt, amplitude=1pt}]
{(0,-3pt) circle(\rocmax)}; % sketch the roc domain
%
\draw[help lines, C7!50, step=0.25cm] (-\rocmax,-\rocmax) grid (\rocmax,\rocmax);
%
\draw[C3, thick] (0,0) circle(1);  % unit circle, i.e. DTFT domain
\draw (1+2*\tic,-2*\tic) node{$1$}; % indicate that this is the unit circle
%\draw[->] (-1.75,0)--(1.75,0) node[right]{$\Re\{z\}$}; % axis label
\draw[->] (0,-1.75)--(0,1.75) node[above]{$\Im\{z\}$}; % axis label
%
\draw[->] (-1.75,0)--(2.25,0) node[right]{$\cdot$}; % axis label
%
\draw (1.2*0.86602540378*\rocmax,1.2*0.5*\rocmax) node[C2!75]{KB}; % indicate the roc
%
% the z-transfer function specific stuff:
\draw (1,1) node[black]{$g=2$}; % indicate gain factor
%
% draw the poles / zeros and if desired ticks:
\draw[C0, ultra thick] (0,+0.75) node{\Huge $\times$};
\draw[C0, ultra thick] (0,-0.75) node{\Huge $\times$};
\draw[C0, ultra thick] (-0.333,0) node{\Huge $\times$};
\draw[C0, ultra thick] (-0.3536,+0.3536) node{\Huge $\circ$};
\draw[C0, ultra thick] (-0.3536,-0.3536) node{\Huge $\circ$};
\draw[C0, ultra thick] (0.5,0) node{\Huge $\circ$};
%
\draw (0,-2) node{$H_\mathrm{min}(z)$};
%
\end{scope}
%
%
%
\begin{scope}[xshift=10cm]  % allpass for max phase
\def \rocmax{2} % sketch of outer roc domain
%
% basic diagram features:
\filldraw[even odd rule,C2!50] (0,0) circle(0.5) decorate
[decoration={snake, segment length=15pt, amplitude=1pt}]
{(0,-3pt) circle(\rocmax)}; % sketch the roc domain
%
\draw[help lines, C7!50, step=0.25cm] (-\rocmax,-\rocmax) grid (\rocmax,\rocmax);
%
\draw[C3, thick] (0,0) circle(1);  % unit circle, i.e. DTFT domain
\draw (1+2*\tic,-2*\tic) node{$1$}; % indicate that this is the unit circle
\draw[->] (-1.75,0)--(2.25,0) node[right]{$\Re\{z\}$}; % axis label
\draw[->] (0,-1.75)--(0,1.75) node[above]{$\Im\{z\}$}; % axis label
%
\draw (1.2*0.86602540378*\rocmax,1.2*0.5*\rocmax) node[C2!75]{KB}; % indicate the roc
%
% the z-transfer function specific stuff:
\draw (1,1) node[black]{$g=-\frac{1}{8}$}; % indicate gain factor
%
% draw the poles / zeros and if desired ticks:
\draw[C0, ultra thick] (-0.3536,-0.3536) node{\Huge $\times$};
\draw[C0, ultra thick] (-0.3536,+0.3536) node{\Huge $\times$};
\draw[C0, ultra thick] (0.5,0) node{\Huge $\times$};
\draw[C0, ultra thick] (-1.4142,+1.4142) node{\Huge $\circ$};
\draw[C0, ultra thick] (-1.4142,-1.4142) node{\Huge $\circ$};
\draw[C0, ultra thick] (2,0) node{\Huge $\circ$};
%
\draw (0,-2) node{$H_\mathrm{all,max}(z)$};
%
\end{scope}
%
\end{tikzpicture}
\end{center}

In den folgenden Grafiken sind alle Systeme und die Reihenschaltungen
grafisch aufbereitet.
In der linken Spalte finden wir die Frequenzgänge Pegel, Phase und Gruppenlaufzeit
jeweils über eine andere Normierung der digitalen Frequenz $\Omega$.
Diese Normierungen sind sehr gängig und wir sollten mit ihnen gut vertraut sein.
Im Pegeldiagramm ist in rot zudem noch der Verlauf bzgl. der logarithmischen
Frequenzachse (im Diagramm oben angedeutet) dargestellt.
Die Gruppenlaufzeit ist im Diagramm in Sekunden angegeben. Zur Erstellung
wurde die Samplingfrequenz $f_s=1$ Hz benutzt, weswegen
die Gruppenlaufzeit auch als Samples interpretiert werden kann, wegen
$\tau_\mathrm{GD,samples} \,/\, f_s = \tau_\mathrm{GD,seconds}$.

In der rechten Spalte sind Impuls- und Sprungantwort über Samples (also unabhängig
von $f_s$) dargestellt. Ganz unten rechts das Pol / Nullstellendiagramm.
Der Konvergenzbereich ist hier als weiße Fläche angedeutet, also alles außerhalb
des grauen Kreises mit Radius der betragsmäßig größten Polstelle.

Gemäß Bodediagramm-Darstellung können wir bei Reihenschaltung von Systemen
\begin{itemize}
  \item Pegel (also die dB) der Einzelsysteme addieren
  \item Phase der Einzelsysteme addieren
  \item Gruppenlaufzeit der Einzelsysteme addieren
\end{itemize}
um Gesamtpegel, -phase und -gruppenlaufzeit zu erhalten.

Dies lässt sich mit den Einzelsystemen \fig{fig:Hmin_55A8105469} und
\fig{fig:Hallmix_55A8105469} zum Gesamtsystem \fig{fig:Hmixcheck_55A8105469}
bzw. \fig{fig:Hmix_55A8105469} checken, z.B. bei $\Omega=\frac{\pi}{2}$ oder
$\Omega=\frac{3\pi}{4}$.

Im Vergleich von $H_\mathrm{min}(z)$, $H_\mathrm{mix}(z)$ und $H_\mathrm{max}(z)$
sehen wir, dass

$H_\mathrm{min}(z)$ die geringste Phasenverschiebung und Gruppenlaufzeit
für die Frequenzen bzw. Frequenzgruppen aufweist und $H_\mathrm{max}(z)$
die höchste Phasenverschiebung und Gruppenlaufzeit hat.

Für große $k$ gehen alle Sprungantworten asymptotisch zum gleichen DC-Anteil
$\frac{15+6\sqrt(2)}{25} \approx 0.9394$ über.

Für Systeme mit gleichem Betragsfrequenzgang (so wie wir das ja hier vorliegen
haben) können wir (nur als Faustformel!) im frühen Verlauf der Impuls- und
Sprungantworten das Phasenverhalten grob erkennen.
Das minimalphasige System hat das Maximum in $h[k]$ und $h_\epsilon[k]$ bei $k=0$,
reagiert also 'sehr schnell', schneller geht es nicht. Die $h[k]$ der mixed und
maximum phase Systeme 'reagieren so richtig' erst bei $k=3$ mit ihrem Maximum.
%
Für $h_\epsilon[k]$ sehen wir, dass das maximalphasige System bei $k=3$ das
Maximum erreicht, während das gemischtphasige System diesen Peak bei $k=2$
hat.
%
Das maximalphasige System ist also das trägste von allen, die durchlaufenden
Signale erhalten die größte Phasenverschiebung und Gruppenlaufzeit. Bis auf
wenige Ausnahmen, ist das in der Praxis eher unerwünschtes Systemverhalten.

\end{Loesung}



% plot_zplane xlim, ylim!!!
\begin{figure}
\includegraphics[width=\textwidth]{../z_system_analysis/Hmix_55A8105469.pdf}
\caption{Mixed Phase System $H_\mathrm{mix}(z)$. Zwei Nullstellen außerhalb
des Einheitskreises.}
\label{fig:Hmix_55A8105469}
\end{figure}
\begin{figure}
\includegraphics[width=\textwidth]{../z_system_analysis/Hmin_55A8105469.pdf}
\caption{Minimum Phase System $H_\mathrm{min}(z)$. Alle Pole und Nullstellen
im Einheitskreis.}
\label{fig:Hmin_55A8105469}
\end{figure}
\begin{figure}
\includegraphics[width=\textwidth]{../z_system_analysis/Hallmix_55A8105469.pdf}
\caption{Allpass $H_\mathrm{all,mix}(z)$ damit
$H_\mathrm{mix}(z) = H_\mathrm{min}(z) \cdot H_\mathrm{all,mix}(z)$ gilt.
Pegeldiagramm links oben zeigt erwartungsgemäß numerisches Rauschen um 0 dB.}
\label{fig:Hallmix_55A8105469}
\end{figure}
\begin{figure}
\includegraphics[width=\textwidth]{../z_system_analysis/Hmixcheck_55A8105469.pdf}
\caption{Check der Reihenschaltung
$H_\mathrm{mix}(z) = H_\mathrm{min}(z) \cdot H_\mathrm{all,mix}(z)$ .}
\label{fig:Hmixcheck_55A8105469}
\end{figure}
%
\begin{figure}
\includegraphics[width=\textwidth]{../z_system_analysis/Hmax_55A8105469.pdf}
\caption{Maximum Phase System $H_\mathrm{max}(z)$. Drei (alle) Nullstellen
außerhalb des Einheitskreises.}
\label{fig:Hmax_55A8105469}
\end{figure}
\begin{figure}
\includegraphics[width=\textwidth]{../z_system_analysis/Hmin_55A8105469.pdf}
\caption{Minimum Phase System $H_\mathrm{min}(z)$. Alle Pole und Nullstellen
im Einheitskreis.}
\label{fig:Hmin_55A8105469_2}
\end{figure}
\begin{figure}
\includegraphics[width=\textwidth]{../z_system_analysis/Hallmax_55A8105469.pdf}
\caption{Allpass $H_\mathrm{all,max}(z)$ damit
$H_\mathrm{max}(z) = H_\mathrm{min}(z) \cdot H_\mathrm{all,max}(z)$ gilt.
Pegeldiagramm links oben zeigt erwartungsgemäß numerisches Rauschen um 0 dB.}
\label{fig:Hallmax_55A8105469}
\end{figure}
\begin{figure}
\includegraphics[width=\textwidth]{../z_system_analysis/Hmaxcheck_55A8105469.pdf}
\caption{Check der Reihenschaltung
$H_\mathrm{max}(z) = H_\mathrm{min}(z) \cdot H_\mathrm{all,max}(z)$ .}
\label{fig:Hmaxcheck_55A8105469}
\end{figure}

\clearpage
\section{Appendix B: Basics Elektrotechnik}
%
Dieser Appendix ersetzt auf keinen Fall ein mehrsemestriges intensives Studieren
der ET, vielleicht hilft er aber sich schneller zurechtzufinden und die Links
zur SigSys zu erkennen.
%
Sehr hilfreiches Buch für Einsteiger*innen in ET ist das Buch
\cite{Marinescu2020}, im Uni Netz als freies E-Book unter

\url{https://link.springer.com/book/10.1007/978-3-658-28884-6}

Wenn wir bisher wirklich gar nichts mit Elektrotechnik zu tun hatten und uns
der Link zur SigSys interessiert, helfen vielleicht die Basics
in den \cite[Kapiteln 1-3]{Marinescu2020}
\begin{itemize}
  \item Ohmsches Gesetz
  \item Kirchhoffsche Gleichungen/Regeln, also Maschenumlauf ergibt Spannung Null,
  Summe alle Ströme rein/raus per Knoten muss Null sein
  \item für Widerstand: Reihenschaltung (R1+R2) mit gleichem Strom,
  Parallelschaltung (1/R1+1/R2) mit gleicher Spannung
\end{itemize}
Dann in \cite[Kapitel 9]{Marinescu2020}
\begin{itemize}
\item Konzept Induktivität u = L di/dt, ideales Bauelement Spule
\item Konzept Kapazität i = C du/dt, ideales Bauelement Kondensator
\item Sinusschwingung, Effektivwert
\item Phasenunterschied zwischen Strom/Spannung bei Widerstand, Spule, Kondensator
\item Kapitel 9.7 / 9.8 beinhaltet Rechnerei an
einfachen R,L,C Schaltungen mit DGL-Ansätzen.
\end{itemize}
Danach vielleicht hilfreich \cite[Kap. 10.2 und 10.3]{Marinescu2020}, d.h.
die komplexe Wechselstromtechnik an R,L,C-Netzwerken, hier steht
der Zusammenhang zwischen Strom und Spannung im Vordergrund.
In \cite[Kap. 13]{Marinescu2020} schaut man sich dann den Zusammenhang zwischen
wechselförmigen Ausgangs- und Eingangsspannung an, während
\cite[Kap. 15]{Marinescu2020} auf Schaltvorgänge eingeht.

Im Wesen schauen wir uns in ET an, wie sich Gleich- oder Wechselgrößen
an elektrischen Netzwerken verhalten entweder im stationären Zustand
oder speziell den Ein-/Ausschaltvorgang von Gleich-/Wechselgrößen.
%
Dies sind in der Welt der SigSys Spezialfälle für die Analyse des LTI-Systems
'elektrisches Netzwerk mit idealen, konzentrierten Bauelementen'.

Im Folgenden ein Versuch die Essenz kompakt zusammenzufassen.

\subsection{Strom i(t) / Spannung u(t) an passiven Bauelementen}

vgl. \cite[Kap. 10.2.2]{Marinescu2020}, Integrationskonstanten Null, d.h. ohne AB

\begin{align}
\text{Widerstand} \qquad & u(t) = R i(t) \qquad& i(t) = \frac{1}{R} u(t)\\
\text{Spule} \qquad & u(t) = L \frac{\fsd i(t)}{\fsd t} \qquad& i(t) = \frac{1}{L} \int u(t) \fsd t\\
\text{Kondensator} \qquad & u(t) = \frac{1}{C} \int i(t) \fsd t \qquad& i(t) = C \frac{\fsd u(t)}{\fsd t}
\end{align}

\subsection{Symbolische Rechnung mit komplexen Zeigern für harmonische Wechselspannung}

vgl. \cite[Kap. 10.3]{Marinescu2020}

Wir machen den Ansatz für komplexwertige, harmonische Signale
\begin{align}
u(t) = \sqrt{2} \, U_\text{eff} \cdot \e^{\im(\omega t + \phi_u)}\\
i(t) = \sqrt{2} \, I_\text{eff} \cdot \e^{\im(\omega t + \phi_i)},
\end{align}
mit den Effektivwerten $U_\text{eff}$, $I_\text{eff}$ und Phasenoffsetwinkeln
$\phi_u$, $\phi_i$.
%
Die obigen Bauteile können in der Praxis nur reellwertige Signale
verarbeiten, daher müssen wir uns entscheiden, ob wir mit dem Realteil oder dem
Imaginärteil von $u(t)$ und $i(t)$ Rückschlüsse zur tatsächlich stattfindenden
Physik ziehen wollen.
Im Grunde ist das Geschmacksfrage, wir entscheiden uns hier 'ja nur', ob wir
physikalisch mit dem Cosinus oder dem Sinus operieren, aber diese Entscheidung
muss dann konsequent beibehalten werden.

Falls wir nun nur eine einzige Kreisfrequenz $\omega$, also z.B. die
EU-typische Stromnetzfrequenz $f=50$ Hz $\rightarrow$ $\omega = 2\pi \cdot 50$ rad/s
in einem elektrischen Netzwerk betrachten, lässt sich die Schreibweise vereinfachen,
wenn wir die Terme $\sqrt{2}$ und $\e^{\im \omega t}$ im Kopf zwar berücksichtigen,
aber eben nicht mehr explizit notieren, weil wir wissen, dass es mit $\omega$
schwingt und wir mit Effektivwerten operieren.
Wir schreiben also kurz die symbolischen Zeiger (gängig ist ein Unterstrich
und Großbuchstabe um das klar zu machen) für Strom und Spannung
\begin{align}
\underline{U} = U_\text{eff} \cdot \e^{\im \phi_u}\\
\underline{I} = I_\text{eff} \cdot \e^{\im \phi_i},
\end{align}
meinen damit aber den vollständigen obigen Ansatz und beachten, ob wir der
$\Re$-Teil oder $\Im$-Teil Konvention folgen wollen.
%
Diese Zeiger werden oft auch als ruhende Effektivwertzeiger bezeichnet.
%
Die sogenannte komplexe Impedanz $\underline{Z}$ ist dann definiert als (es ist konsistent mit dem
obigen vollständigen Ansatz, weil sich $\sqrt{2}$ und $\e^{\im\omega t}$ kürzen)
\begin{align}
\underline{Z} =
\frac{\underline{U}}{\underline{I}} =
\frac{U_\text{eff} \cdot \e^{\im \phi_u}}{I_\text{eff} \cdot \e^{\im \phi_i}} =
\frac{U_\text{eff}}{I_\text{eff}} \e^{\im (\phi_u-\phi_i)}
\end{align}

\subsection{Komplexe Impedanzen für passive Bauelemente}

vgl. \cite[Kap. 10.3.5]{Marinescu2020}

Wenn wir für den Zusammenhang Strom/Spannung am Kondensator
$u(t) = \frac{1}{C} \int i(t)$
die Ansätze
\begin{align}
u(t) = \sqrt{2} \, U_\text{eff} \cdot \e^{\im(\omega t + \phi_u)}\\
i(t) = \sqrt{2} \, I_\text{eff} \cdot \e^{\im(\omega t + \phi_i)}
\end{align}
einsetzen, erhalten wir zunächst
\begin{align}
\sqrt{2} \, U_\text{eff} \cdot \e^{\im(\omega t + \phi_u)} = \frac{1}{C} \int \sqrt{2} \, I_\text{eff} \cdot \e^{\im(\omega t + \phi_i)} \fsd t
\end{align}
und integriert
\begin{align}
\sqrt{2} \, U_\text{eff} \cdot \e^{\im(\omega t + \phi_u)} = \frac{1}{\im \omega C} \sqrt{2} \, I_\text{eff} \cdot \e^{\im(\omega t + \phi_i)}
\end{align}
und zur komplexen Impedanz für den Kondensator umgestellt
\begin{align}
\label{eq:appb_ZC}
\underline{Z} = \frac{\underline{U}}{\underline{I}} = \frac{U_\text{eff}}{I_\text{eff}} \e^{\im (\phi_u-\phi_i)} = \frac{1}{\im \omega C} = \frac{1}{\omega C} \cdot \e^{-\im \frac{\pi}{2}}
\end{align}
%
Für die Spule führt der gleiche Ansatz (diesmal ist eine Ableitung nach der Zeit zu rechnen)
zur komplexen Impedanz
\begin{align}
\underline{Z} = \frac{\underline{U}}{\underline{I}} = \frac{U_\text{eff}}{I_\text{eff}} \e^{\im (\phi_u-\phi_i)} = \im \omega L = \omega L \cdot \e^{+\im \frac{\pi}{2}}
\end{align}
%
Für den Widerstand erhält man die komplexe Impedanz
\begin{align}
\underline{Z} = \frac{\underline{U}}{\underline{I}} = \frac{U_\text{eff}}{I_\text{eff}} \e^{\im (\phi_u-\phi_i)} = R.
\end{align}
%
Wir sehen, dass die komplexe Impedanz des Widerstands eigentlich gar nicht komplexwertig ist.
Zudem ist sie offensichtlich nicht von $\omega$ abhängig, d.h. ein (idealer!) Widerstand
verhält sich bzgl. direkter Proportionalität zwischen Strom und Spannung frequenz\underline{un}abhängig!
%
Für die Spule erkennen wir ein frequenzabhängiges Verhalten zwischen Strom und Spannung, zudem
ist der Phasenwinkel zwischen Spannung und der Referenz Strom $\frac{\pi}{2} \rightarrow 90^\circ$, man sagt,
dass die Spannung dem Strom $90^\circ$ voraus ist.
%
Für den Konsensator erkennen wir auch frequenzabhängiges Verhalten zwischen Strom und Spannung.
Der Phasenwinkel ist diesmal $-\frac{\pi}{2}$, die Spannung ist dem Strom um $90^\circ$ hinterher.



\subsection{Beispiel Wechselspannung am einfachen RC-Glied}

Für das abbgebildete RC-Glied sollen zwei Fälle diskutiert werden
\begin{itemize}
  \item 1. Zusammenhang zwischen Eingangsspannung $u_e(t)$ und Strom $i(t)$
  \item 2. Zusammenhang zwischen Einganngsspannung $u_e(t)$ und Ausgangsspannung $u_a(t)$
\end{itemize}
%
\begin{center}
\begin{circuitikz}[european, scale=0.75]
\node (in) at (1,0){};
\node (in_ground) at (1,-3){};
\node (out) at (4,0){};
\node (out_ground) at (4,-3){};
\draw (in) to [R,l_=$R$,o-] (3,0);
\draw (3,0) to [short,-o,] (out);
\draw (3,0) to [C,l_=$C$,*-*] (3,-3);
\draw (in_ground) to [short,o-o] (out_ground);
\path[draw, bend right, ->, >=latex] (in) edge node[left]{Eingangsspannung $u_e(t)$} (in_ground);
\path[draw, bend left, ->, >=latex] (out) edge node[right]{Ausgangsspannung $u_a(t)$} (out_ground);
\draw (3,-3) to [short,i=${i(t)}$] (1.5,-3);
\node (inp) at (0.5,0){$+$};
\node (inm) at (0.5,-3){$\bot$};
\end{circuitikz}
\end{center}

\subsubsection*{Zusammenhang zwischen Eingangsspannung $u_e(t)$ und Strom $i(t)$}

\cite[Kap. 9.8.2]{Marinescu2020}

Wir wählen die $\Re$-Konvention (Cosinussignal).
%
Es gibt nur einen Strom $i(t)$ im Netzwerk, der durch $R$ und $C$ in der angezeigten
Richtung fließt. Das ist eine ET-Konvention mit dem sperrigen Namen
Verbraucherzählpfeilsystem \cite[S. 7]{Marinescu2020}. Für das gewählte Potentialgefälle
der Quelle $u_e(t)$ von Pluspol zu Masse führt diese Konvention dazu, dass
Spannungsabfälle über Verbrauchern die gleiche Richtung wie der technische
Stromfluss haben. Das sieht man z.B. am Spannungspfeil über dem Kondensator,
diese Spannung ist gleichzeitig auch die uns interessierende Ausgangsspannung.
%
Der tatsächliche Elektronenstrom im elektrischen Netzwerk ist entgegengesetzt
zur getroffenen Konvention.
%
Aus dem Maschenumlauf \cite[Kap. 2.5]{Marinescu2020} für Spannungen folgt
\begin{align}
\label{eq:appb:mascheRC}
u_e(t) = R i(t) +  \frac{1}{C} \int i(t) \fsd t
\end{align}
und mit der symbolischen Methode und komplexen Impedanzen kann elegant (deswegen hat man sich symbolische Methode 'ausgedacht')
\begin{align}
\underline{U}_e = R \underline{I} +  \frac{1}{\im \omega C} \underline{I}
= (R +  \frac{1}{\im \omega C}) \underline{I}
\end{align}
aufgeschrieben werden.
%
Die resultierende komplexe Impedanz zwischen $\underline{U}_e$ und $\underline{I}$
lässt sich in Betrag/Phase darstellen
\begin{align}
\underline{U}_e = \sqrt{R^2 +  \left(\frac{1}{\omega C}\right)^2} \e^{\im \phi_Z} \cdot \underline{I}
\end{align}
mit dem Phasenwinkel
\begin{align}
\phi_Z = \text{atan2}({-\frac{1}{\omega R C}}).
\end{align}
Falls nun die cosinusförmige, stationär einwirkende Eingangsspannung als die
Ursache und der resultierende Stromfluss durch R und C als
Wirkung betrachtet wird, fügen wir den eingestellten (gegebenen = bekannte Ursache)
Effektivwert und Phasenoffsetwinkel der Eingangsspannung ein
\begin{align}
U_{e,\text{eff}} \cdot \e^{\im \phi_u} = \sqrt{R^2 +  \left(\frac{1}{\omega C}\right)^2} \e^{\im \phi_Z} \cdot \underline{I}
\end{align}
und lösen nach dem ruhendem Effektivwertzeiger des Stroms auf
\begin{align}
\underline{I} =
I_\text{eff} \cdot \e^{\im \phi_i} =
\frac{U_{e,\text{eff}}}{\sqrt{R^2 +  \left(\frac{1}{\omega C}\right)^2}}
\cdot \e^{\im (\phi_u-\phi_Z)}
\end{align}
aus dem wir sehr elegant erkennen, wie groß der Effektivwert und zugehörige
Phasenoffsetwinkel des Stroms ist.
%
Wenn wir uns nun noch für den tatsächlichen, harmonischen Stromverlauf
interessieren, müssen wir
die gewonnenen Informationen in den vollständigen Ansatz übertragen
\begin{align}
\sqrt{2} \underline{I} \e^{\im \omega t}=
\sqrt{2}  I_\text{eff} \cdot \e^{\im \phi_i} \e^{\im \omega t} =
\sqrt{2} \frac{U_{e,\text{eff}}}{\sqrt{R^2 +  \left(\frac{1}{\omega C}\right)^2}}
\cdot \e^{\im (\phi_u-\phi_Z)} \e^{\im \omega t}
\end{align}
und den Zeitverlauf durch 'Rücktransformation', also Benutzung der
gewählten Cosinus-Konvention ($\Re$-Teil) bilden
\begin{align}
i(t) =
\Re\{\sqrt{2} \frac{U_{e,\text{eff}}}{\sqrt{R^2 +  \left(\frac{1}{\omega C}\right)^2}}
\cdot \e^{\im (\phi_u-\phi_Z)} \e^{\im \omega t}\}=
\frac{\sqrt{2} \, U_{e,\text{eff}}}{\sqrt{R^2 +  \left(\frac{1}{\omega C}\right)^2}}
\cdot \cos(\omega t + \phi_u-\phi_Z)
\end{align}
%
In \cite[Kap. 9.8.2]{Marinescu2020} ist dann der Funktionsverlauf von $i(t)$
und der Zusammenhang zu $u_e(t)$ zu Ende diskutiert.

\subsubsection*{Zusammenhang zwischen Eingangsspannung $u_e(t)$ und Ausgangsspannung $u_a(t)$}

\cite[Kap. 13.1]{Marinescu2020}

Sehr oft interessieren wir uns für das Verhältnis zwischen Eingangs- und Ausgangsgröße
des Netzwerks (i.e. Annahme: LTI-Systems), hier also zwischen der Eingangsspannung
(i.e. eine Quellenspannung) und der Ausgangsspannung, die hier über den Kondensator
abfällt.
Ein wichtiges Werkzeug dafür ist der sogenannte Spannungsteiler
\cite[Kap. 3.1.2, 11.2.1]{Marinescu2020}, den wir immer dann anwenden können,
wenn wir uns für Spannungsverhältnisse interessieren in Schaltungsbereichen mit
gleichem Stromfluss.
Hier im sehr einfachen Beispiel gibt es nur einen Strom und wir können
zwei komplexe Spannungsteiler aufstellen, einmal Kondensatorspannung bezogen auf
Gesamtspannung (vgl. \eqref{eq:appb_ZC}, \eqref{eq:appb:mascheRC})
\begin{align}
\frac{\underline{U}_a}{\underline{U}_e} = \frac{\frac{1}{\im \omega C}\underline{I}}{R \underline{I}+\frac{1}{\im \omega C} \underline{I}}
\end{align}
und einmal Widerstandsspannung bezogen auf Gesamtspannung
\begin{align}
\frac{\underline{U}_a}{\underline{U}_e} = \frac{R\underline{I}}{R \underline{I}+\frac{1}{\im \omega C} \underline{I}}.
\end{align}
Wir sehen, dass sich in beiden Spannungsteilern der Stromzeiger kürzen lässt,
das ist die Idee beim Spannungsteiler.
Wir sind am ersten Spannungsteiler interessiert und schreiben den ein wenig
um
\begin{align}
\frac{\underline{U}_a}{\underline{U}_e} = \frac{\frac{1}{\im \omega C}}{R +\frac{1}{\im \omega C}}=
\frac{1}{\im \omega R C + 1}
\end{align}
Es ist sinnvoll für $RC$ eine eigene Variable einzuführen, weil sie ganz entscheidend
das Verhalten des RC-Glieds bestimmt.
Gemäß der Einheit Sekunde des Produkts $R C$ führen wie die sogenannte Zeitkonstante
$T_\text{RC} = R C$ ein.
%
Sehr hilfreich ist auch der reziproke Wert, den wir als Grenzkreisfrequenz
$\omega_\text{RC} = \frac{1}{R C}$ in rad/s kennenlernen werden.
%
Mit beiden Variablen lässt sich der Spannungsteiler schreiben zu
\begin{align}
\frac{\underline{U}_a}{\underline{U}_e} = \frac{\frac{1}{\im \omega C}}{R +\frac{1}{\im \omega C}}=
\frac{1}{\im \omega R C + 1} =
\frac{1}{\im \omega T_\text{RC} + 1} =
\frac{1}{\im \frac{\omega}{\omega_\text{RC}} + 1}.
\end{align}
In Übung~\ref{sec:4408E33353} (3.4) sehen wir genau dieses System wieder, dort aus Sicht der SigSys, aber
wir begegnen dem Spannungsteiler sozusagen als Spezialfall wieder.
%
In SigSys Sprech haben wir Ausgang $\underline{U}_a \rightarrow y$ und Eingang
$\underline{U}_e \rightarrow x$ und wir könnten uns z.B. anschauen, was
stationäre harmonische Signalen (bzw. eine Superposition solcher, weil wir
LTI-Eigenschaften annehmen) passiert, wenn sie durch das System RC-Glied geschickt
werden. Wir können dann finden, dass die Fälle $\omega\ll\omega_\text{RC}$ und
$\omega\gg\omega_\text{RC}$ asymptotisch wichtige Grenzfälle sind, der Speziallfall
$\omega=\omega_\text{RC}$ sehr wichtig ist und zur Erkenntnis kommen, dass
das System nur die Amplitude (Effektivwert) und Phasenlage zwischen Aus- und Eingang
ändern kann. All das machen wir auch in einer Elektrotechnik VL/UE, in SigSys
ist es aber eingebettet in den größeren Kontext. Vielleicht spannender Vorgriff:
In Abb.~\ref{fig:bodeplot_examples_pt1_element_AppB}
sieht man, wie das Systemverhalten SigSys-typisch grafisch aufbereitet wird,
das kommt in Übung~\ref{sec:ue5_levelresponse} (5) ausführlich.

\begin{figure}
  \includegraphics[width=\textwidth]{../laplace_system_analysis/bodeplot_examples_pt1_element.pdf}
  \caption{Full SigSys picture of the \textbf{1st order lowpass system from
  exercise~\ref{sec:4408E33353} (3.4)} with $T_\mathrm{RC} = \frac{1}{\omega_\mathrm{RC}} = 1$ s.
  \texttt{bodeplot\_examples.ipynb}}
  \label{fig:bodeplot_examples_pt1_element_AppB}
\end{figure}



\subsection{Beispiel Auf- und Entladevorgang am einfachen RC-Glied}

Für das abbgebildete RC-Glied sollen zwei Fälle diskutiert werden
\begin{itemize}
  \item 1. Aufladevorgang Spannung des Kondensators
  \item 2. Entladevorgang Spannung des Kondensators
\end{itemize}

\begin{center}
\begin{circuitikz}[european, scale=0.75]
\node (in) at (1,0){};
\node (in_ground) at (1,-3){};
\node (out) at (4,0){};
\node (out_ground) at (4,-3){};
\draw (in) to [R,l_=$R$,o-] (3,0);
\draw (3,0) to [short,-o,] (out);
\draw (3,0) to [C,l_=$C$,*-*] (3,-3);
\draw (in_ground) to [short,o-o] (out_ground);
\path[draw, bend right, ->, >=latex] (in) edge node[left]{Eingangsspannung $u_e(t)$} (in_ground);
\path[draw, bend left, ->, >=latex] (out) edge node[right]{Ausgangsspannung $u_a(t)$} (out_ground);
\draw (3,-3) to [short,i=${i(t)}$] (1.5,-3);
\node (inp) at (0.5,0){$+$};
\node (inm) at (0.5,-3){$\bot$};
\end{circuitikz}
\end{center}


\subsubsection*{Aufladevorgang am einfachen RC-Glied}

Zum Zeitpunkt $t=0$ soll die Gleichspannung $U$ als Eingang auf das vorher in
Ruhe befindliche RC Glieds geschaltet werden.
%
Wie verhält sich die Spannung über dem Kondensator $u_a(t)$ über die Zeit?

Wir kennen den Maschenumlauf ab $t\geq 0$
\begin{align}
U = R i(t) + u_a(t)
\end{align}
und wissen, dass der Stromfluss vom Kondensator gemäß
\begin{align}
i(t) = C \frac{\fsd u_a(t)}{\fsd t}
\end{align}
beeinflusst wird.
Die zweite Gleichung in die erste eingesetzt mit Benutzung der Zeitkonstante
$T_\text{RC} = R C$ ergibt eine lineare Differentialgleichung erster Ordnung
\begin{align}
U = R C \frac{\fsd u_a(t)}{\fsd t} + u_a(t) =
T_\text{RC} \frac{\fsd u_a(t)}{\fsd t} + u_a(t)
\end{align}
Wir könnten das noch etwas sauberer aufschreiben mit Anfangsbedingungen
und Inhomogenität
\begin{align}
u_e(t) = T_\text{RC} \frac{\fsd u_a(t)}{\fsd t} + u_a(t),
\qquad u_a(0_-)=0,\quad u_e(t\geq 0) = U.
\end{align}
Diese DGL lässt sich mit Trennung der Variablen lösen und die Anfangsbedingungen
können direkt mit Wahl des Integrationsbereichs berücksichtigt werden.
%
Die Lösung führt auf
\begin{align}
u_a(t \geq 0) = U(1-\e^{-\frac{t}{T_\text{RC}}}) = U - U\e^{-\frac{t}{T_\text{RC}}}
\end{align}
und ist in Abb.~\ref{fig:4408E33353_AppB} links schematisch dargestellt.
%
Die Lösung wird in \cite[Lap. 15.4.4]{Marinescu2020} ausführlich diskutiert.

\subsubsection*{Entladevorgang am einfachen RC-Glied}

Zum Zeitpunkt $t=0$ sei der Kondensator auf die Spannung $U$ aufgeladen und
der Eingang der Schaltung wird bei $t=0$ mit einem Kabel kurzgeschlossen.
Der daraufhin einsetzende Stromfluss entlädt den Kondensator.
Wie verhält sich die Spannung $u_a(t)$ über dem Kondensator?

Diese Lösung wird auch in \cite[Lap. 15.4.4, S. 371]{Marinescu2020} diskutiert.

Maschenumlauf für $t\geq 0$ ergibt
\begin{align}
  u_a(t) + R i(t) = 0
\end{align}
und Strom wie schon vorher
\begin{align}
i(t) = C \frac{\fsd u_a(t)}{\fsd t}.
\end{align}
Dies ergibt
\begin{align}
  u_a(t) = - R C \frac{\fsd u_a(t)}{\fsd t}
\end{align}
und die Anfangsbedingung berücksichtigt
\begin{align}
  u_a(t) = - R C \frac{\fsd u_a(t)}{\fsd t},
  \quad u_a(0_-) = U.
\end{align}
Auch diese lineare DGL erster Ordnung lässt sich mit Trennung der Variablen lösen
zu
\begin{align}
u_a(t) = U \e^{-\frac{t}{T_\text{RC}}}
\end{align}
%
Die Funktion ist in Abb.~\ref{fig:4408E33353_AppB} rechts schematisch dargestellt.


\begin{figure}[h]
\begin{center}
%sigma < 0
\begin{tikzpicture}
%
\def \axisLength {4}
\def \tic {0.05}
\def \sigmaz {1}
\def \omegaz {1}
\def \convAbsz {-\sigmaz}
%
\begin{scope}[]
\begin{axis}[
width=0.45\textwidth,
height=0.3\textwidth,
domain=0:7,
samples=64,
legend pos=outer north east,
xlabel = {$t\rightarrow$},
ylabel = {$u_a(t)$},
title = {Aufladevorgang Kondensatorspannung},
xmin=0, xmax=7,
ymin=-0.1, ymax=1.1,
xtick={0, 1, 3, 5},
ytick={0,0.63,0.95, 1},
xticklabels={$0$, $T_\mathrm{RC}$, $3 T_\mathrm{RC}$, $5 T_\mathrm{RC}$},
yticklabels={$0$, $0.63 U$, $ $, $U$},
ymajorgrids=true,
xmajorgrids=true
]
\addplot[mark=None, color=C0, ultra thick]
coordinates {(-4,0)(0,0)};
\addplot[mark=None, color=C0, ultra thick]
{1-exp(-\sigmaz*x)};
\end{axis}
\end{scope}
%
\begin{scope}[shift={(7,0)}]
\begin{axis}[
width=0.45\textwidth,
height=0.3\textwidth,
domain=0:7,
samples=64,
legend pos=outer north east,
xlabel = {$t\rightarrow$},
ylabel = {$u_a(t)$},
title = {Entladevorgang Kondensatorspannung},
xmin=0, xmax=7,
ymin=-0.1, ymax=1.1,
xtick={0,1,3,5},
ytick={0, 0.3678, 1},
xticklabels={$0$, $T_\mathrm{RC}$, $3 T_\mathrm{RC}$, $5 T_\mathrm{RC}$},
yticklabels={$0$, $0.37 U$, $U$},
ymajorgrids=true,
xmajorgrids=true
]
\addplot[mark=None, color=C0, ultra thick]
coordinates {(-4,0)(0,0)(0,1)};
\addplot[mark=None, color=C0, ultra thick]
{exp(-\sigmaz*x)};
\end{axis}
\end{scope}
%
\end{tikzpicture}
\end{center}

\caption{Auf- und Entladekurven für Gleichspannung am Kondensator}
\label{fig:4408E33353_AppB}
\end{figure}

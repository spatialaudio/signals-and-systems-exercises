\documentclass[11pt,a4paper,DIV=12]{scrartcl}
\usepackage{scrlayer-scrpage}
\usepackage[utf8]{inputenc}
\usepackage{fouriernc}
\usepackage[T1]{fontenc}
\usepackage[german]{babel}
\usepackage[hidelinks]{hyperref}
\usepackage{natbib}
\usepackage{url}
\usepackage{amsmath}
\usepackage{amsfonts}
\usepackage{amssymb}
\usepackage{trfsigns}
\usepackage{bm}
\usepackage{marvosym}
\usepackage{nicefrac}
%\usepackage{graphicx}
%\usepackage{subcaption}
%\usepackage{xcolor}
%\usepackage{comment}
%\usepackage{mdframed}
%\usepackage{tikz}
%\usepackage{circuitikz}
%\usepackage{pgfplots}
%\usepackage{cancel}
\bibliographystyle{dinat}

%\numberwithin{equation}{section}
%\numberwithin{figure}{section}

% \usetikzlibrary{calc}
% \usetikzlibrary{positioning}
% \usetikzlibrary{matrix}
% \usetikzlibrary{chains}
% \usetikzlibrary{shapes.misc}
% \tikzset{cross/.style={cross out, draw,minimum size=2*(#1-\pgflinewidth),inner sep=0pt, outer sep=0pt}}
% \pgfkeys{/pgfplots/axis style/.code={\pgfkeysalso{/pgfplots/every axis/.append style={#1}}}}
% \pgfplotsset{
% mathaxis/.style={
% axis lines=center,
% xtick=\empty,
% ytick=\empty,
% xlabel style=right,
% ylabel style=above,
% % Make sure the origin is shown (http://tex.stackexchange.com/a/91253)
% before end axis/.code={
% \addplot [draw=none, forget plot] coordinates {(0,0)};
% },
% anchor=origin,
% },
% stemaxis/.style={
% mathaxis,
% x=.8em,
% y=6ex,
% enlarge x limits={abs=1.2em},
% enlarge y limits={abs=1.2em},
% }
% }
% \tikzstyle{stem}=[ycomb,mark=*,mark size=10\pgflinewidth,color=C0,ultra thick]
% \tikzstyle{stem2}=[ycomb,mark=*,mark size=5\pgflinewidth,color=C0,ultra thick]

%\newcommand\fsd{\mathrm{d}} %der/int operator
%\newcommand{\sysH}[1]{\mathcal{H}{\{#1\}}}  % system operator
%\renewcommand{\vec}[1]{\mathbf{#1}} %vector
%\newcommand{\eq}[1]{Glg. (\ref{#1})} %ref equation
%\newcommand{\fig}[1]{Abb. \ref{#1}} %ref figure
%\newcommand{\red}{\textcolor{red}}
%\newcommand{\blue}{\textcolor{blue}}

% %Sha symbol:
% \DeclareFontFamily{U}{wncy}{}
% \DeclareFontShape{U}{wncy}{m}{n}{<->wncyr10}{}
% \DeclareSymbolFont{mcy}{U}{wncy}{m}{n}
% \DeclareMathSymbol{\Sha}{\mathord}{mcy}{"58}
% %\newcommand{\Sha}{$\bot \!\! \bot \!\! \bot$}

%\DeclareMathOperator{\rectOP}{\text{rect}}
%\newcommand{\rectN}[2]{\ensuremath{\rectOP_{#1} \left[ #2 \right]}}

%matplotlib colors:
% \definecolor{C0}{HTML}{1f77b4}
% \definecolor{C1}{HTML}{ff7f0e}
% \definecolor{C2}{HTML}{2ca02c}
% \definecolor{C3}{HTML}{d62728}
% \definecolor{C7}{HTML}{7f7f7f}

% \specialcomment{Ziel}{\begin{mdframed}[backgroundcolor=C2!10] \textit{Lernziel}\\\noindent}{\end{mdframed}\noindent}
% \specialcomment{Werkzeug}{\begin{mdframed}[backgroundcolor=C7!10] \textit{Werkzeug}\\\noindent}{\end{mdframed}\noindent}
% \specialcomment{Ansatz}{\begin{mdframed}[backgroundcolor=C3!10] \textit{Ansatz}\\\noindent}{\end{mdframed}\noindent}
% \specialcomment{ExCalc}{\begin{mdframed}[backgroundcolor=C1!10]\noindent \textit{Ausführliche Rechnung}\\\noindent}{\end{mdframed}\noindent}
% \specialcomment{Loesung}{\begin{mdframed}[backgroundcolor=C0!10] \textit{Lösung}\\\noindent}{\end{mdframed}\noindent}

%https://ctan.org/tex-archive/macros/latex/contrib/trfsigns?lang=en
%needs \rmfamily instead of \rm in trfsigns package
% \renewcommand{\ztransf}{\mbox{\setlength{\unitlength}{0.1em}%
%                             \begin{picture}(20,10)%
%                               \put(2,3){\circle{4}}%
%                               \put(4,3){\line(1,0){4.75}}%
%                               \multiput(8.625,3.15)(0.25,0.25){11}{%
%                                 \makebox(0,0){\rmfamily\tiny .}}%
%                               \put(17,3){\line(-1,0){5.75}}%
%                               \put(18,3){\circle*{4}}%
%                             \end{picture}%
%                            }
%                       }
% \renewcommand{\Ztransf}{\mbox{\setlength{\unitlength}{0.1em}%
%                             \begin{picture}(20,10)%
%                               \put(2,3){\circle*{4}}%
%                               \put(3,3){\line(1,0){5.75}}%
%                               \multiput(11.375,3.15)(-0.25,0.25){11}{%
%                                 \makebox(0,0){\rmfamily\tiny .}}%
%                               \put(16,3){\line(-1,0){4.75}}%
%                               \put(18,3){\circle{4}}%
%                             \end{picture}%
%                            }
%                       }

% \newcommand{\dtft}{\mbox{\setlength{\unitlength}{0.1em}%
%                             \begin{picture}(20,10)%
%                               \put(2,3){\circle{4}}%
%                               \put(4,3){\line(1,0){4.75}}%
%                               \multiput(8.625,3.15)(0.25,0.25){11}{%
%                                 \makebox(0,0){\rmfamily\tiny .}}%
%                               \put(17,3){\line(-1,0){5.75}}%
%                               %\put(18,3){\circle*{4}}%
%                             \end{picture}%
%                            }
%                       }
% \newcommand{\DTFT}{\mbox{\setlength{\unitlength}{0.1em}%
%                             \begin{picture}(20,10)%
%                               %\put(2,3){\circle*{4}}%
%                               \put(3,3){\line(1,0){5.75}}%
%                               \multiput(11.375,3.15)(-0.25,0.25){11}{%
%                                 \makebox(0,0){\rmfamily\tiny .}}%
%                               \put(16,3){\line(-1,0){4.75}}%
%                               \put(18,3){\circle{4}}%
%                             \end{picture}%
%                            }
%                       }


% \newcommand{\mydft}{\mbox{\setlength{\unitlength}{0.1em}%
%                             \begin{picture}(20,10)%
%                               \put(2,3){\circle{4}}%
%                               \put(4,3){\line(1,0){4.75}}%
%                               \multiput(8.625,3.15)(0.25,0.25){11}{%
%                                 \makebox(0,0){\rmfamily\tiny .}}%
%                               \put(17,3){\line(-1,0){5.75}}%
%                               %\put(18,3){\circle*{4}}%
%                               \put(6,-4){\scriptsize $N$}
%                             \end{picture}%
%                            }
%                       }
%
% \newcommand{\myDFT}{\mbox{\setlength{\unitlength}{0.1em}%
%                             \begin{picture}(20,10)%
%                               %\put(2,3){\circle*{4}}%
%                               \put(3,3){\line(1,0){5.75}}%
%                               \multiput(11.375,3.15)(-0.25,0.25){11}{%
%                                 \makebox(0,0){\rmfamily\tiny .}}%
%                               \put(16,3){\line(-1,0){4.75}}%
%                               \put(18,3){\circle{4}}%
%                               \put(6,-4){\scriptsize $N$}
%                             \end{picture}%
%                            }
%                       }

%------------------------------------------------------------------------------
% \title{Übung Signal- und Systemtheorie\thanks{
% This tutorial is provided as Open Educational Resource (OER), to be found at
% \url{https://github.com/spatialaudio/signals-and-systems-exercises}
% accompanying the OER lecture
% \url{https://github.com/spatialaudio/signals-and-systems-lecture}.
% %
% Both are licensed under a) the Creative Commons Attribution 4.0 International
% License for text and graphics and b) the MIT License for source code.
% %
% Please attribute material from the tutorial as \textit{Frank Schultz,
% Continuous- and Discrete-Time Signals and Systems - A Tutorial Featuring
% Computational Examples, University of Rostock} with
% \texttt{main file, github URL, commit SHA number and/or version tag, year}.
% }
% \\
% \small Vst.-Nr. 24015}
% %
% \author{Frank Schultz, Sascha Spors\\
% \small Institut für Nachrichtentechnik (INT)\\
% \small Fakultät für Informatik und Elektrotechnik (IEF)\\
% \small Universität Rostock
% }
% %
% \date{\today}
%------------------------------------------------------------------------------

%\ihead{Aufgabe \thesubsection}
\ohead{Signal- und Systemtheorie Übung}
\cfoot{\pagemark}
\ofoot{\tiny\url{https://github.com/spatialaudio/signals-and-systems-exercises}}

\begin{document}
\noindent Signal- und Systemtheorie Übung\footnote{This tutorial is provided as
Open Educational Resource (OER), to be found at
\url{https://github.com/spatialaudio/signals-and-systems-exercises}
accompanying the OER lecture
\url{https://github.com/spatialaudio/signals-and-systems-lecture}.
%
Both are licensed under a) the Creative Commons Attribution 4.0 International
License for text and graphics and b) the MIT License for source code.
%
Please attribute material from the tutorial as \textit{Frank Schultz,
Continuous- and Discrete-Time Signals and Systems - A Tutorial Featuring
Computational Examples, University of Rostock} with
\texttt{main file, github URL, commit SHA number and/or version tag, year}
.}---Frank Schultz, Sascha Spors,
Institut für Nachrichtentechnik (INT),
Fakultät für Informatik und Elektrotechnik (IEF),
Universität Rostock \&
Robert Hauser, Universität Rostock---Version: \today


\section*{Formelsammlung Fourierreihe}

Komplexe Schreibweise der Fourierreihe (vgl. \cite[S. 488]{Bronstein}):

\begin{align}
	x(t)=\sum_{k=-\infty}^{+\infty}X_k\e^{+\im k \omega t}\text{, wobei }X_k=\frac{1}{T}\int_{0}^{T}x(t)\e^{-\im k \omega t}\mathrm{d}t
\end{align}




Differentiation( falls $x(t)$ diffbar)
\begin{align}
	\frac{\mathrm{d}}{\mathrm{d}t}x(t)=\frac{\mathrm{d}}{\mathrm{d}t}\sum_{k=-\infty}^{+\infty}X_k\e^{+\im k \omega t}=\sum_{k=-\infty}^{+\infty}X_k\frac{\mathrm{d}}{\mathrm{d}t}\e^{+\im k \omega t}=\sum_{k=-\infty}^{+\infty}X_k\im\omega k\e^{+\im k \omega t}=_k
\end{align}
unbestimmte Integration:
\begin{align}
	\int x(t)\mathrm{d}t&=\int\sum_{k=-\infty}^{+\infty}X_k\e^{+\im k \omega t}\mathrm{d}t \\
	&= \sum_{k=-\infty}^{-1}\bigg [X_k\int\e^{+\im k \omega t}\mathrm{d}t\bigg ]+\sum_{k=1}^{+\infty}\bigg [X_k\int\e^{+\im k \omega t}\mathrm{d}t\bigg ]+\int X_0\mathrm{d}t\\
	&=\sum_{k=-\infty}^{-1}\frac{1}{\im\omega k}X_k\e^{+\im k \omega t}+\sum_{k=1}^{+\infty}\frac{1}{\im\omega k}X_k\e^{+\im k \omega t}+t\cdot X_0+C
\end{align}
Modulation
\begin{align}
	y(t)=\e^{\im n\omega t}x(t) \quad n\in\mathbb{Z}\\
	Y_k=\frac{1}{T}\int_0^Tx(t)\e^{-\im(k-n)\omega t}\mathrm{d}t\quad\Bigg | \quad m = k-n \\
	Y_{m+n}=\frac{1}{T}\int_0^Tx(t)\e^{-\im  m \omega t}\mathrm{d}t=X_m \\
	Y_{k}=X_{k-n}
\end{align}
Verschiebung
\begin{align}
	y(t)=x(t-\tau)\quad\tau\in\mathbb{R}\\
	Y_k=\frac{1}{T}\int_0^{T}y(t)\e^{-\im\omega k t}\mathrm{d}t=\frac{1}{T}\int_0^{T}x(t-\tau)\e^{-\im k \omega t}\mathrm{d}t \quad \Bigg | \quad l = t-\tau\\
	Y_k=\frac{1}{T}\int_{-\tau}^{T-\tau}x(l)\e^{-\im\omega k (l+\tau)}\mathrm{d}l=\e^{-\im \omega k \tau}\int_{0}^{T}x(l)\e^{-\im\omega k l}\mathrm{d}l \\
	Y_k=\e^{-\im\omega k \tau}X_k
\end{align}
Zeitumkehr
\begin{align}
	&y(t)=x(-t) \\
	&Y_k=\frac{1}{T}\int_0^{T}y(t)\e^{-\im\omega k t}\mathrm{d}t=\frac{1}{T}\int_0^{T}x(-t)\e^{-\im\omega k t}\mathrm{d}t\quad\Bigg | \quad \tau = -t \\
	&Y_k=-\frac{1}{T}\int_0^{-T}x(\tau)\e^{\im\omega k t}\mathrm{d}\tau=\frac{1}{T}\int_{-T}^{0}x(\tau)\e^{\im\omega k t}\mathrm{d}\tau \quad \Bigg |\quad m = -k \\
	&Y_{-m}=\frac{1}{T}\int_{-T}^{0}x(\tau)\e^{-\im\omega m t}\mathrm{d}\tau \\
	&Y_{-m}=X_m \\
	&Y_{k}=X_{-k}
\end{align}
Komplex konjugiert
\begin{align}
	y(t)=x^*(t) \\
	Y_k=\frac{1}{T}\int_0^{T}y(t)\e^{+\im\omega k t}\mathrm{d}t=\frac{1}{T}\int_0^Tx^*(t)\e^{-\im\omega k t}\mathrm{d}t\underset{Beleg benötigt}{=}\frac{1}{T}\Bigg[\int_0^Tx(t)\e^{+\im\omega k t}\mathrm{d}t\Bigg ]^*\quad \Bigg | \quad k = -m \\
	Y_k=\Bigg[\frac{1}{T}\int_0^Tx(t)\e^{-\im\omega m \tau}\mathrm{d}t\Bigg]^*\\
	Y_k=-X_m^*\\
	Y_k=X_{-k}^*
\end{align}
Faltung
\begin{align}
	y(t)&=h(t)\ast x(t)=\int_0^Tx(\tau)h(t-\tau)\mathrm{d}\tau\\
	Y_k&=\frac{1}{T}\int_0^Ty(t)\e^{-\im\omega k t}\mathrm{d}t\\&=\frac{1}{T}\int_0^T\bigg (\int_0^Tx(\tau)h(t-\tau)\mathrm{d}\tau\bigg )\e^{-\omega k t}\mathrm{d}t\\
	Y_k&=\frac{1}{T}\int_0^Tx(\tau)\bigg (\frac{T}{T}\int_0^Th(t-\tau)\e^{-\im \omega k t}\mathrm{d}t\bigg)\mathrm{d}\tau\quad\Bigg | \quad l=t-\tau \\
	Y_k&=\frac{1}{T}\int_0^Tx(\tau)\bigg (\frac{T}{T} \int_{-\tau}^{T-\tau}(h(l)\e^{-\im \omega k (l+\tau)}\mathrm{d}l\bigg)\mathrm{d}\tau=\frac{T}{T}\int_0^Tx(\tau)H_k\e^{-\im\omega k \tau}\mathrm{d}\tau=T\cdot X_k\cdot H_k \\
	Y_k&=T\cdot X_k\cdot Y_k
\end{align}
Multiplikation
\begin{align}
	y(t)&=x(t)\cdot h(t) = \bigg (\sum_{k=-\infty}^{+\infty}X_k\e^{+\im\omega k t}\bigg )\cdot\bigg (\sum_{k=-\infty}{+\infty}H_k\e^{+\im\omega k t}\bigg )\\
	&=(\cdot\cdot\cdot+X_{-3}\e^{-\im\omega 3 t}+X_{-2}\e^{-\im\omega 2 t}+X_{-1}\e^{-\im\omega t}+X_0+X_{1}\e^{+\im\omega t}+X_{2}\e^{+\im\omega 2 t}+X_{3}\e^{+\im\omega 3 t}+\cdot\cdot\cdot)\cdot\\&(\cdot\cdot\cdot +H_{-3}\e^{-\im\omega 3 t}+H_{-2}\e^{-\im\omega 2 t}+H_{-1}\e^{-\im\omega  t}+H_0+H_{1}\e^{+\im\omega  t}+H_{2}\e^{\im\omega 2 t}+H_{3}\e^{\im\omega 3 t}+\cdot\cdot\cdot)\\
	&=\cdot\cdot\cdot+(\cdot\dot\cdot+H_{-3}\cdot X_{-3}+\cdot\cdot\cdot)\e^{-\im\omega 6 t}+(\cdot\cdot\cdot+H_{-3}\cdot X_{-2}+H_{-2}\cdot X_{-3}+\cdot\cdot\cdot)\e^{-\im\omega 5 t}+\\
	&(\cdot\cdot\cdot+X_{-3}\cdot H_{-1}+X_{-2}\cdot H_{-2}+X_{-1}\cdot H_{-3}+\cdot\cdot\cdot)\e^{-\im\omega 4 t}+\\
	&(\cdot\cdot\cdot+X_{-3}\cdot H_0+X_{-2}\cdot H_{-1}+X_{-1}\cdot H_{-2}+X_0\cdot H_{-3}+\cdot\cdot\cdot)\e^{-\im 3 t}+\\
	&(\cdot\cdot\cdot X_{-3}\cdot H_1+X_{-2}\cdot H_0+X_{-1}\cdot H_{-1}+X_0\cdot H_{-1}+X_{1}\cdot H_{-3}+\cdot\cdot\cdot)\e^{-\im\omega 2 t}+\\
	&\sum_{\kappa=-\infty}^{+\infty}X_{\kappa}\cdot H_{-1-\kappa}\e^{-\im\omega t}+\\
	&\sum_{\kappa=-\infty}^{+\infty}X_{\kappa}\cdot H_{-\kappa}+\\
	&\sum_{\kappa=-\infty}^{+\infty}X_{\kappa}\cdot H_{1-\kappa}\e^{\im\omega t}+\\
	&\sum_{\kappa=-\infty}^{+\infty}X_{\kappa}\cdot H_{2-\kappa}\e^{\im\omega 2t}+\\
	&\sum_{\kappa=-\infty}^{+\infty}X_{\kappa}\cdot H_{3-\kappa}\e^{\im\omega 3t}+\cdot\cdot\cdot\\
	&=\sum_{k=-\infty}^{k=+\infty}\bigg (\sum_{\kappa=-\infty}^{+\infty}X_{\kappa}\cdot H_{k-\kappa}\bigg)\e^{+\im\omega k t}\\
	&=\sum_{k=-\infty}^{+\infty}(X_k\ast H_k) \e^{+\im\omega k t}
\end{align}
Skalierung (vgl. \cite{Butz2012} S. 22)\\
$y(t) = x(a\cdot t)$	\\
$a>0$: $C_k$ ändert sich nicht, $\omega_y = \omega_x \cdot a$.\\
$a<0$: $C_{k_y} = C_{-k_x}$, $\omega_y = \omega_x \cdot |a|$.\\
Parseval`sches Theorem (vgl. \cite{KressKaufhold2010} S.45 ):
\begin{align}
	\frac{1}{T}\int_{0}^{T}x(t)\mathrm{d}t=\sum_{k=-\infty}^{+\infty}|X_k|^2
\end{align}
Alle Funktionen in folgender Tabelle sind $T$-periodisch.\\
\begin{tabular}{|lcc|}
	\hline
	&&\\
	&\textbf{Fourierreihe}& \\
	\hline
	&&\\
	& $x(t)=\sum_{k=-\infty}^{+\infty}X_k\e^{+\im\omega k t} $ & $X_k=\frac{1}{T}\int_0^Tx(t)\e^{-\im\omega k t}\mathrm{d}t $ \\&&\\
	\hline
	&&\\
	\textbf{Eigenschaften} & & \\
	Linearität & $Af(t)+Bg(t)$ & $AF_k+BG_K$\\&& \\
	Symmetrien & $x(-t)$ & $X_{-k}$ \\ &&\\
	& $x^*(t)$ & $X^*_{-k}$ \\ &&\\
	\hline
	\textbf{Sätze} & & \\&& \\
	Faltung & $x(t)\ast h(t)$ & $T\cdot X_k\cdot H_k$ \\&& \\
	Multiplikation & $x(t)\cdot h(t)$ & $X_k \ast H_k $ \\&& \\
	Zeitskalierung $(a\in\mathbb{R}\setminus\{0\})$& $x(a\cdot t)$ & $\begin{cases}
		X_k,a\cdot\omega, &a>0\\
		X_{-k},|a|\cdot\omega, &a<0
	\end{cases}$\\
	Verschiebung $(\tau \in \mathbb{R})$ & $x(t-\tau)$& $\e^{-\im\omega k \tau}X_k$\\&& \\
	Modulation $(n \in \mathbb{Z})$& $\e^{-\im n \omega t}x(t)$ & $X_{k-\omega_0}$ \\&& \\
	Differentiation & $\frac{\mathrm{d}}{\mathrm{d}t}x(t)$ & $\im\omega k X_k$\\&& \\
	unbestimmtes Integral &$\int x(t)\mathrm{d}t $ & $\frac{1}{\im\omega k}X_k$\\&& \\
	Parselval`sches Theorem & $\frac{1}{T}\int_0^T|x(t)|^2\mathrm{d}t$ & $\sum_{k=-\infty}^{+\infty}|X_k|^2$\\&& \\
	\hline
\end{tabular}







\renewcommand{\refname}{Buchzitate}
%\cite{*}
\clearpage
\bibliography{literatur}
\end{document}

\clearpage
\section{UE 7: Abtastung von Signalen und Spektren}

Wir lernen die sogenannte äquidistante Abtastung bezüglich der Zeit
(Aufgabe 1) und der (Kreis)-Frequenz (Aufgabe 2) kennen.
%
Dies ist einerseits wichtig, um den Übergang zur zeitdiskreten Signalverarbeitung
herstellen zu können: Sehr viele Signale, die in der Natur vorkommen, sind
ja trotz Digital Age, nach wie vor (zum Glück)
zeit- und wertekontinuierlich. Um sie mit einem
Rechner verarbeiten zu können, müssen wir diese bzgl. der Zeit diskretisieren
(abtasten) und bzgl. der Werte diskretisieren (quantisieren), beide Prozesse
zusammen erzeugen ein digitales Signal.
Daher müssen wir uns mit der Abtastung und Quantisierung von Zeitsignalen
beschäftigen.
Hier in SigSys werden wir zunächst nur die Abtastung kennenlernen, im Mastermodul
Digital Signal Processing folgt dann die Quantisierung.
%
Ein anderer wichtiger Punkt ideale Abtastung und die Rekonstruktion zurück
zu kontinuierlichen Signalen als SigSys Werkzeug einzuführen, ist
die Verknüpfung von verschiedenen Fourier Transformationen. Es gibt im Grunde
ja keine verschiedenen Fourier Transformationen, sondern eigentlich vielmehr
4 verschiedene Typen, die alle für eine bestimmte Signal/Spektrum-Charakteristik
gelten. Wir kennen bisher die Fourier Reihe und die Fourier Transformation.
Wir werden lernen sie elegant zu verknüpfen.

\section*{Werkzeuge zur Abtastung / Rekonstrukion}

Wir brauchen drei wesentliche Zutaten: i) ein Modell als Signalflussgraph und
zwei fundamentale Signale/Spektren: ii) den Dirac Impuls Kamm um die Abtastung
zu modellieren und iii) Impulsantwort/Fenster bzw. deren Spektren für die Rekonstruktion.

\subsection*{i) Modell zur Abtastung und Rekonstruktion}
Die Modelle sind in \fig{fig:sampling_model_time_domain_signals}
und \fig{fig:sampling_model_frequency_domain_signals} dargestellt. Sie leben
von der Dualität Faltung vs. Multiplikation.
Die Multiplikation des abzutastenden Signals/Spektrums mit einem Dirac Impuls Kamm
bestimmt, in welchem Bereich wir abtasten.
Daher sehen wir in \fig{fig:sampling_model_time_domain_signals} die Abtastung
von Zeitsignalen. In \fig{fig:sampling_model_frequency_domain_signals}
sehen wir die Abtastung von Spektren, also Abtastung der Fouriertransformation.
%
\begin{figure}[h]
\centering
\begin{tikzpicture}[align=center,node distance=0.5cm and 0.5cm]
\tikzstyle{block} = [draw, rectangle, minimum height=3em, minimum width=3em]
\begin{scope}
\node (input) {$\displaystyle x(t)$};
\node [draw, circle, right=of input, minimum size=7.5mm] (op1) {$\cdot$};
\node [above=of op1] (sop1) {$s(t)$};
\node [block, right=of op1] (mult) {$1$};
\node [draw, circle, right=of mult, minimum size=7.5mm, xshift=0.5cm] (op2) {$\ast_t$};
\node [above=of op2] (sop2) {$h_\mathrm{r}(t)$};
\node [right=of op2] (output) {$x_\mathrm{r}(t)$};
\draw [->] (input) -- (op1);
\draw [->] (op1) -- (mult);
\draw [->] (mult) node[below, xshift=1.1cm, yshift=0cm] {$x_\mathrm{s}(t)$} -- (op2);
\draw [->] (op2) -- (output);
\draw [->] (sop1) -- (op1);
\draw [->] (sop2) -- (op2);
\node [right=of output, xshift=-0.4cm, yshift=0.5cm] (fourier) {$\fourier$};
\end{scope}
%
\begin{scope}[shift={(8,0)}]
\node (input) {$\displaystyle X(\im\omega)$};
\node [draw, circle, right=of input, minimum size=7.5mm] (op1) {$\ast_\omega$};
\node [above=of op1] (sop1) {$S(\im\omega)$};
\node [block, right=of op1] (mult) {$\frac{1}{2\pi}$};
\node [draw, circle, right=of mult, minimum size=7.5mm, xshift=0.5cm] (op2) {$\cdot$};
\node [above=of op2] (sop2) {$H_\mathrm{r}(\im\omega)$};
\node [right=of op2] (output) {$X_\mathrm{r}(\im\omega)$};
\draw [->] (input) -- (op1);
\draw [->] (op1) -- (mult);
\draw [->] (mult) node[below, xshift=+1.1cm, yshift=0cm] {$X_\mathrm{s}(\im\omega)$} -- (op2);
\draw [->] (op2) -- (output);
\draw [->] (sop1) -- (op1);
\draw [->] (sop2) -- (op2);
\end{scope}
\end{tikzpicture}
\caption{Abtastung und Rekonstruktion für \textbf{zeitliche Signale} mit Dirac Impuls Kamm
Dualität $s(t)\fourier  S(\im\omega)$ und Rekonstruktionsfilter $h_\mathrm{r}(t)$.}
\label{fig:sampling_model_time_domain_signals}
\end{figure}
%
\begin{figure}[h]
\centering
\begin{tikzpicture}[align=center,node distance=0.5cm and 0.5cm]
\tikzstyle{block} = [draw, rectangle, minimum height=3em, minimum width=3em]
\begin{scope}
\node (input) {$\displaystyle x(t)$};
\node [draw, circle, right=of input, minimum size=7.5mm] (op1) {$\ast_t$};
\node [above=of op1] (sop1) {$s(t)$};
\node [draw, circle, right=of op1, minimum size=7.5mm, xshift=0.5cm] (op2) {$\cdot$};
\node [above=of op2] (sop2) {$h_\mathrm{r}(t)$};
\node [block, right=of op2] (mult) {$1$};
\node [right=of mult] (output) {$x_\mathrm{r}(t)$};
\draw [->] (input) -- (op1);
\draw [->] (op1) node[below, xshift=0.9cm, yshift=0cm] {$x_\mathrm{s}(t)$} -- (op2);
\draw [->] (op2) -- (mult);
\draw [->] (mult) -- (output);
\draw [->] (sop1) -- (op1);
\draw [->] (sop2) -- (op2);
\node [right=of output, xshift=-0.4cm, yshift=0.5cm] (fourier) {$\fourier$};
\end{scope}
%
\begin{scope}[shift={(8,0)}]
\node (input) {$\displaystyle X(\im\omega)$};
\node [draw, circle, right=of input, minimum size=7.5mm] (op1) {$\cdot$};
\node [above=of op1] (sop1) {$S(\im\omega)$};
\node [draw, circle, right=of op1, minimum size=7.5mm, xshift=0.5cm] (op2) {$\ast_\omega$};
\node [above=of op2] (sop2) {$H_\mathrm{r}(\im\omega)$};
\node [block, right=of op2] (mult) {$\frac{1}{2\pi}$};
\node [right=of mult] (output) {$X_\mathrm{r}(\im\omega)$};
\draw [->] (input) -- (op1);
\draw [->] (op1) node[below, xshift=0.9cm, yshift=0cm] {$X_\mathrm{s}(\im\omega)$} -- (op2);
\draw [->] (op2) -- (mult);
\draw [->] (mult) -- (output);
\draw [->] (sop1) -- (op1);
\draw [->] (sop2) -- (op2);
\end{scope}
\end{tikzpicture}
\caption{Abtastung und Rekonstruktion für \textbf{Frequenz-Spektren} mit Dirac Impuls Kamm
Dualität $S(\im\omega) \Fourier s(t)$ und Rekonstruktionsspektrum $H_\mathrm{r}(\im\omega)$.}
\label{fig:sampling_model_frequency_domain_signals}
\end{figure}












\subsection*{ii) Dirac Impuls Kamm}
%
In der Übung 4 zur Fouriertransformation hatten wir zunächst ein wichtiges
Signalpaar und gleichzeitig eine wichtige Dualität absichtlich unterschlagen.
%
Diese soll nun hier im Mittelpunkt stehen, weil wir sie für unser Abtastmodell
brauchen: der sogenannte Dirac Impuls Kamm.
%
Genauso wie die Gaussglocke bildet sich der Dirac Impuls Kamm bei der
Fouriertransformation auf sich selbst ab, halt in skalierter Form.
%
Genau dieser Eigenschaft verdanken wir die elegante und kompakte Veranschaulichung
der Signal- oder Spektren-abtastung.

Für das Folgende benutzen wir den kyrillischen Buchstaben $\Sha$ gesprochen
'Sha' als Formelzeichen.
%
Die Korrespondenz der Fourier Transformation des Dirac Impuls Kamms lautet
\begin{mdframed}
\begin{equation}
\frac{1}{T_\mathrm{s}} {\Sha}(\frac{t}{T_\mathrm{s}}) \quad\fourier\quad
{\Sha}(\frac{\omega T_\mathrm{s}}{2 \pi}) =
\Sha(\frac{\omega}{\omega_\mathrm{s}})
%{\Sha}(\frac{\Omega}{2 \pi})
\end{equation}
\end{mdframed}
mit dem zeitlichen Abtastintervall $T_\mathrm{s} = \frac{1}{f_\mathrm{s}}>0$
und der (Kreis)-Abtastfrequenz
\begin{align}
\omega_\mathrm{s} = \frac{2 \pi}{T_\mathrm{s}}  = 2 \pi f_\mathrm{s}.
\end{align}
%
Diese Korrespondenz ist sehr wichtig und wir müssen in der Lage sein, diese
aus der allgemeinen Definition des Dirac Impuls Kamms (in der Formelsammlung gegeben)
\begin{equation}
{\Sha}(t) := \sum_{m=-\infty}^{+\infty} \delta(t-m)% \quad\mathrm{für}\quad m\in\mathbb{Z}.
\end{equation}
fehlerfrei und schnell ableiten zu können.
%
Dazu brauchen wir die Skalierungseigenschaft des Dirac Impulses ($a\in\mathbb{R}$)
\begin{equation}
\delta(a t) = \frac{1}{|a|} \delta(t) \quad\longrightarrow\quad
\delta(t) = |a| \delta(a t),
\end{equation}
wobei die zweitere Darstellung übersichtlicher angewendet werden kann.
%
Für ein beliebiges, zeitliches Abtastintervall $T_\mathrm{s}>0$ können wir
für einen verschobenen Dirac Impuls
\begin{equation}
\delta(t - m T_\mathrm{s}) =
\frac{1}{T_\mathrm{s}}\cdot \delta(\frac{t - m T}{T_\mathrm{s}}) =
\frac{1}{T_\mathrm{s}} \cdot \delta(\frac{t}{T_\mathrm{s}}-m)
\end{equation}
schreiben, wenn wir für die Skalierung $a=\frac{1}{T_\mathrm{s}}$ verwenden.
%
Daraus folgen die äquivalenten Darstellungen, wenn wir eine Summe
verschobener Dirac Impulse einführen
\begin{equation}
\sum_{m=-\infty}^{+\infty} \delta(t-m T_\mathrm{s}) =
\frac{1}{T_\mathrm{s}}\sum_{m=-\infty}^{+\infty} \delta(\frac{t}{T_\mathrm{s}}-m) =
\frac{1}{T_\mathrm{s}} {\Sha}(\frac{t}{T_\mathrm{s}}).
\end{equation}
Die erste Darstellung ist sinnvoll um sich anhand von Skizzen die zeitliche
Lage und das Gewicht $1$ der Dirac Impulse klarzumachen. Die letzte
Darstellung können wir für kompakte Schreibweise bei Rechnungen benutzen. Das
mag zu Beginn unnötig erscheinen, aber erleichtert den Rechen- und Lesefluss
erheblich, sobald wir in dieser Formel im Kopf sofort sehen, wie der Dirac Kamm
ausschaut. Das gelingt mit ein wenig Übung.
%Vielleicht hilft es, wenn wir
%uns nochmal klarmachen, dass das Signal
%$\frac{1}{T_\mathrm{s}}\mathrm{rect}(\frac{t}{T_\mathrm{s}})$
%eine Rechteckfunktion der Fläche 1 mit Breite $T_\mathrm{s}$ und daher
%Höhe $\frac{1}{T_\mathrm{s}}$ darstellt. Dies ist eine ähnliche Zeit- und
%Amplitudenskalierung.

Wir können die allgemeine Definition des Dirac Impuls Kamms
auch für die Frequenzvariable benutzen, also
\begin{equation}
{\Sha}(\omega) := \sum_{\nu=-\infty}^{+\infty} \delta(\omega-\nu)
%\quad\mathrm{für}\quad \nu\in\mathbb{Z}
\end{equation}
einführen.
Führen wir nun die Kreis-Abtastfrequenz $\omega_\mathrm{s}>0$ ein mit der Motivation
diesbezüglich ganzzahlig verschobene Dirac Impulse $\delta(\omega - \nu \omega_s)$
zu realisieren.
Erneut die Skalierungseigenschaft angewandt ergibt dann diesmal für
$a=\frac{1}{\omega_\mathrm{s}}$
\begin{equation}
\delta(\omega - \nu \omega_s) =
\frac{1}{\omega_s}\delta(\frac{\omega - \nu \omega_s}{\omega_s}) =
\frac{1}{\omega_s} \cdot \delta(\frac{\omega}{\omega_s}-\nu).
\end{equation}
%
Damit lässt sich jetzt der Link zum Sha-Operator herstellen
\begin{equation}
\omega_s \sum_{\nu=-\infty}^{+\infty} \delta(\omega-\nu \omega_s) =
\sum_{\nu=-\infty}^{+\infty} \delta(\frac{\omega}{\omega_s}-\nu) =
{\Sha}(\frac{\omega}{\omega_s}).
\end{equation}
%
In der Formelsammlung ist 'nur' die Korrespondenz
\begin{align}
  \Sha(t) \quad\fourier\quad \Sha(\frac{\omega}{2\pi})
\end{align}
gegeben, die wir nun allgemein auch ohne genaue Kenntnis des Sha Operators zeitlich
skalieren können, also genauso wie wir das z.B bei der rect-Funktion machen würden,
\begin{align}
  \Sha(\frac{t}{T_\mathrm{s}}) \quad\fourier\quad
  T_\mathrm{s} \Sha(\frac{\omega T_\mathrm{s}}{2\pi}).
\end{align}
Wir sehen schnell, dass das exakt das Ergebnis von oben ist, unsere Werkzeuge
sind erfreulicherweise konsistent.
%
Damit erhalten wir zusammenfassend die äquivalenten Korrespondenzen
(aus der Formelsammlung ableitbar)
\begin{mdframed}
\begin{equation}
\frac{1}{T_\mathrm{s}} {\Sha}(\frac{t}{T_\mathrm{s}}) \quad\fourier\quad
{\Sha}(\frac{\omega T_\mathrm{s}}{2 \pi}) =
\Sha(\frac{\omega}{\omega_\mathrm{s}})
%{\Sha}(\frac{\Omega}{2 \pi})
\end{equation}
\end{mdframed}
(für Skizzen und tatsächliche Rechnungen)
\begin{mdframed}
\begin{equation}
\sum_{m=-\infty}^{+\infty} \delta(t-m T_\mathrm{s}) \quad\fourier\quad
\omega_s \sum_{\nu=-\infty}^{+\infty} \delta(\omega-\nu \omega_s)
\end{equation}
\end{mdframed}
%
In \fig{fig:DiracImpulsKammSkizze} ist die Korrespondenz der Dirac Impuls Kämme
skizziert, dies werden wir sehr oft brauchen bzw. zur Veranschaulichung benutzen
müssen.

\begin{figure}[h!]
\centering
%
\begin{tikzpicture}[scale=0.75]
\begin{scope}
\draw[->] (-2.1 ,0) -- (2.4,0) node[right]{$t$};
\draw[->] (0,-1.1) -- (0,1.75) node[above]{$x(t)=\frac{1}{T_\mathrm{s}}\Sha(\frac{t}{T_\mathrm{s}})$};
\draw[->, C0, line width=1mm] (+0,0) -- (+0,1);
\draw[->, C0, line width=1mm] (-1,0) -- (-1,1);
\draw[->, C0, line width=1mm] (-2,0) -- (-2,1);
\draw[->, C0, line width=1mm] (-3,0) -- (-3,1);
\draw[->, C0, line width=1mm] (1,0) -- (1,1) node[above] {$(1)$};
\draw[->, C0, line width=1mm] (2,0) -- (2,1);
\draw[->, C0, line width=1mm] (3,0) -- (3,1);
\node at (4,0) {$\fourier$};
\node at (1,-0.4) {$T_\mathrm{s}$};
\node at (-1,-0.4) {$-T_\mathrm{s}$};
\node at (2,-0.4) {$2 T_\mathrm{s}$};
\end{scope}
\begin{scope}[shift={(8,0)}]
\draw[->] (-2.1 ,0) -- (2.4, 0) node[right]{$\omega$};
\draw[->] (0,-1.1) -- (0,1.75) node[above]{$X(\im\omega)=\Sha(\frac{\omega T_\mathrm{s}}{2\pi})$};
\draw[->, C3, line width=1mm] (+0,0) -- (+0,1);
\draw[->, C3, line width=1mm] (-1,0) -- (-1,1);
\draw[->, C3, line width=1mm] (-2,0) -- (-2,1);
\draw[->, C3, line width=1mm] (-3,0) -- (-3,1);
\draw[->, C3, line width=1mm] (1,0) -- (1,1) node[above] {$(\omega_\mathrm{s})$};
\draw[->, C3, line width=1mm] (2,0) -- (2,1);
\draw[->, C3, line width=1mm] (3,0) -- (3,1);
\node at (1,-0.4) {$\omega_\mathrm{s}$};
\node at (-1,-0.4) {$-\omega_\mathrm{s}$};
\node at (2,-0.4) {$2 \omega_\mathrm{s}$};
\end{scope}
\end{tikzpicture}
%
%
%
\caption{Korrespondenz der Fourier Transformation für den Dirac Impuls Kamm.
Dirac Impulse sind äquidistant bzgl. der Zeitachse (links) und
der Frequenzachse (rechts). Zeit-Austast-Intervall $T_\mathrm{s}$.
Kreis-Frequenz-Austast-Intervall $\omega_\textrm{s}=\frac{2\pi}{T_\mathrm{s}}$.}
\label{fig:DiracImpulsKammSkizze}
\end{figure}


\textbf{Hinweis I:} In der Praxis arbeiten und denken wir wahrscheinlich
lieber mit der physikalischen
Frequenz $f$ als mit der Kreisfrequenz $\omega$. Es bietet jedoch hier an, bei der Denke
mit Kreisfrequenzen zu bleiben, weil wir dann konsistent in unserer SigSys-Denke
weitermachen können. Anders herum: jetzt bei der Abtastung von Zeitsignalen
die physikalische Frequenz $f$ zu benutzen, erscheint vielleicht elegant,
weil zunächst einfacher vorstellbar und viel praxisbezogener,
aber macht wegen vielen Umnormierungen mehr Arbeit und bietet Raum für
Schusselfehler und Verwirrung.
Daher die Empfehlung, SigSys einfach weiter mit $\omega$ durchzuziehen,
alle Aufgaben sind auch so gestellt. Wenn die Tools dann sitzen, ist die
Transferleistung zu physikalischen Frequenzen $f$ vergleichsweise einfach, z.B.
ganz am Schluss von Rechnungen bei der grafischen Darstellung.


\subsection*{iii) Rekonstruktion}
Aus der Formelsammlung und den bisher behandelten Themen kennen wir vier
Korrespondenzen der Fourier Transformation die eine Rekonstruktion
ermöglichen (vgl. \fig{fig:ReconstructionSplines})
\begin{align}
h_\mathrm{r}(t) \quad\fourier\quad  & H_\mathrm{r}(\im\omega)\\
\text{rect}(t) \quad\fourier\quad  &\text{sinc}\left( \frac{\omega}{2} \right)\text{ für Abtastung im Zeit-/Frequenzbereich}\\
\text{sinc}(t)  \quad\fourier\quad  &\pi \,\text{rect}\left(\frac{\omega}{2}\right)\text{ für Abtastung im Zeit-/Frequenzbereich}\\
\Lambda(t) \quad\fourier\quad  &\text{sinc}^2\left( \frac{\omega}{2} \right)  \text{ für Abtastung im Zeitbereich}\\
\text{sinc}^2\left( t \right) \quad\fourier\quad  &\pi\Lambda(\frac{\omega}{2})\text{ für Abtastung im Frequenzbereich}
\end{align}
Für die ideale Abtastung und Rekonstruktion im Zeitbereich benutzen
wir in dieser Übung
\begin{align}
h_\mathrm{r}(t) = \frac{1}{\pi}\text{sinc}(t)
\quad\fourier\quad
H_\mathrm{r}(\im\omega) = \text{rect}\left(\frac{\omega}{2}\right).
\end{align}
Wenn wir $h_\mathrm{r}(t)$ als Impulsantwort eines LTI-Systems auffassen (laut
Modell falten wir mit diesem Signal), können wir
dessen Spektrum als ideales Tiefpassfilter mit Grenzfrequenz $\omega_c=1$ rad/s auffassen.
%
Für die ideale Abtastung und Rekonstruktion im Frequenzbereich benutzen
wir in dieser Übung
\begin{align}
H_\mathrm{r}(\im\omega) = \text{sinc}\left( \frac{\omega}{2} \right)
\quad\Fourier\quad
h_\mathrm{r}(t)  = \text{rect}(t)
\end{align}
Wir können $h_\mathrm{r}(t)$ als Zeitfenster um $t=0$ mit Breite 1 auffassen,
weil wir laut Modell damit multiplizieren.
Mit diesen Rekonstruktionen wurde das Abtasttheorem initial auch eingeführt und
bewiesen. Eine historische Abhandlung und wie diese Rekonstruktion
aus einer verallgemeinernden Sichtweise einzuordnen ist, findet sich in
\url{https://doi.org/10.1109/5.843002}.

\textbf{Hinweis II}: Wir können uns das nicht oft genug klarmachen: die ganzen
Formeln der Transformationen interessieren sich nicht welchen Bereich wir wie
interpretieren,
also was wir als Zeitbereich und was als Frequenzbereich deklarieren. Daher werden
wir alle Dinge die wir für einen Bereich lernen, also Signaloperationen, Zusammenhänge usw.
auch im anderen Bereich wieder entdecken, denken wir an die Modulations-/Verschiebungsdualität,
die uns schon oft nützlich war.
Hier ist es nun die Abtastung, die wir in beiden Domänen anwenden können und
im Grunde nicht zwei völlig verschiedene Dinge erlernen müssen. Bis auf ein
paar Normierungen ist das Gedankenkonstrukt völlig identisch. Kopfweh macht
erfahrungsgemäß das sichere Hin- und Herspringen zwischen Zeit- und Bildbereich,
das braucht Übung und auch Erfahrung, zugegeben! Es wird uns aber allgemein
einfacher fallen, wenn wir Abtastung als fundamentale Operation
nicht nur als Spezialfall im Zeitbereich einführen, sondern als fundamentaleres
Konzept. Daher unser didaktischer Faden in der Vorlesung und hier in der
Übung.



\begin{figure}
\includegraphics[width=\textwidth]{../sampling/ReconstructionSplines.pdf}
  \caption{Für die Abtastung im Zeitbereich: Impulsantworten und Spektren von
  Rekonstrukionsfiltern. In blau die Sinc-Impulsantwort des idealen Tiefpassfilters,
  womit das Abtasttheorem 'erfunden' bzw. erstmals bewiesen wurde.}
  \label{fig:ReconstructionSplines}
\end{figure}



\clearpage
\subsection{Abtastung und Rekonstruktion eines Zeitsignals: Sinus}
\label{sec:EF235EE3D8}
\begin{Ziel}
Wir wollen das ideale Abtast- und Rekonstruktionsmodell einmal
anhand speziell gewählter Parameter durchspielen, sowohl im Zeitbereich, als
in der nächsten Aufgabe dann auch im Bildbereich. Wir werden wieder sehen,
dass einfach zu interpretierende
Ergebnisse in einem Bereich, schwer zugängliche Darstellungen im anderen Bereich
bedingen. Hier wird uns der Frequenzbereich eine deutlich anschaulichere Lösung
liefern. Quasi während unserer Rechnerei werden wir auf die
Sinc-Interpolationsformel im Zeitbereich stoßen, die als Grundlage des
Abtasttheorems erst 1933 streng bewiesen wurde.
\end{Ziel}
\textbf{Aufgabe} {\tiny EF235EE3D8}:
Für das dargestellte zeitliche Abtast- und Rekonstruktionsmodell
\begin{center}
\begin{tikzpicture}[align=center,node distance=0.5cm and 0.5cm]
\tikzstyle{block} = [draw, rectangle, minimum height=3em, minimum width=3em]
\begin{scope}
\node (input) {$\displaystyle x(t)$};
\node [draw, circle, right=of input, minimum size=7.5mm] (op1) {$\cdot$};
\node [above=of op1] (sop1) {$s(t)$};
\node [block, right=of op1] (mult) {$1$};
\node [draw, circle, right=of mult, minimum size=7.5mm, xshift=0.5cm] (op2) {$\ast_t$};
\node [above=of op2] (sop2) {$h_\mathrm{r}(t)$};
\node [right=of op2] (output) {$x_\mathrm{r}(t)$};
\draw [->] (input) -- (op1);
\draw [->] (op1) -- (mult);
\draw [->] (mult) node[below, xshift=1.1cm, yshift=0cm] {$x_\mathrm{s}(t)$} -- (op2);
\draw [->] (op2) -- (output);
\draw [->] (sop1) -- (op1);
\draw [->] (sop2) -- (op2);
\node [right=of output, xshift=-0.4cm, yshift=0.5cm] (fourier) {$\fourier$};
\end{scope}
%
\begin{scope}[shift={(8,0)}]
\node (input) {$\displaystyle X(\im\omega)$};
\node [draw, circle, right=of input, minimum size=7.5mm] (op1) {$\ast_\omega$};
\node [above=of op1] (sop1) {$S(\im\omega)$};
\node [block, right=of op1] (mult) {$\frac{1}{2\pi}$};
\node [draw, circle, right=of mult, minimum size=7.5mm, xshift=0.5cm] (op2) {$\cdot$};
\node [above=of op2] (sop2) {$H_\mathrm{r}(\im\omega)$};
\node [right=of op2] (output) {$X_\mathrm{r}(\im\omega)$};
\draw [->] (input) -- (op1);
\draw [->] (op1) -- (mult);
\draw [->] (mult) node[below, xshift=+1.1cm, yshift=0cm] {$X_\mathrm{s}(\im\omega)$} -- (op2);
\draw [->] (op2) -- (output);
\draw [->] (sop1) -- (op1);
\draw [->] (sop2) -- (op2);
\end{scope}
\end{tikzpicture}
\end{center}
soll für
\begin{align}
x(t)= \sin(3 t)\qquad
%\end{align}
%
%\begin{align}
s(t) = \frac{2}{\pi} {\Sha}(\frac{2 t}{\pi})\qquad
%\end{align}
%
%\begin{align}
H_r(\mathrm{j}\omega) = \frac{\pi}{2}\mathrm{rect}(\frac{\omega}{4})
\end{align}
die Abtastung und die Rekonstruktion berechnet und skizziert werden.

\begin{itemize}
  \item Berechnen Sie die Spektren $X(\im\omega)$, $S(\im\omega)$, $X_\mathrm{s}(\im\omega)$, $X_\mathrm{r}(\im\omega)$.
  \item Berechnen Sie die Signale $x_\mathrm{s}(t)$, $h_\mathrm{r}(t)$ und $x_\mathrm{r}(t)$.
  \item Skizzieren Sie alle Signale, d.h. $x(t)$, $s(t)$, $x_\mathrm{s}(t)$, $h_\mathrm{r}(t)$ und $x_\mathrm{r}(t)$
  \item Skizzieren Sie alle Spektren, d.h.  $X(\im\omega)$, $S(\im\omega)$, $X_\mathrm{s}(\im\omega)$, $H_\mathrm{r}(\im\omega)$ und $X_\mathrm{r}(\im\omega)$
  \item Ist die Rekonstruktion des Zeitsignals erfolgreich?
\end{itemize}






\begin{Werkzeug}
Alles was wir bisher im SigSys Koffer sortiert vorfinden :-)
\end{Werkzeug}
\begin{Ansatz}
Wir finden zunächst für alle gegebenen Signale und Spektren die direkten
Fouriertransformierten.

\textbf{Eingang, Signal und Spektrum}:
Wir haben
\begin{equation}
x(t) = \sin(\omega_0 t) = \sin(3 t),
\end{equation}
also  $\omega_0=3$ rad/s gegeben. Das sind 3 Schwingungen pro $2\pi$ oder eben
als Periodendauer $T_0 = \frac{2\pi}{3}$ s.
Die blaue Kurve in \fig{fig:EF235EE3D8_TimeDomain} zeigt das Sinussignal.
Daraus folgt die Fourier Transformierte
\begin{equation}
X(\mathrm{j} \omega) =
\mathrm{j} \pi \left( \delta(\omega+\omega_0) - \delta(\omega-\omega_0) \right) =
\mathrm{j} \pi \left( \delta(\omega+3) - \delta(\omega-3) \right),
\end{equation}
also ein rein reelles, ungerades Spektrum. Es ist in \fig{fig:Sampling_02_Sine_EF235EE3D8}
dargestellt.

\textbf{Abtastung mit Dirac Impuls Kamm, Signal und Spektrum}:
Wir können $T_\mathrm{s} = \frac{\pi}{2}$ und
$\omega_\mathrm{s} = \frac{2 \pi}{T_\mathrm{s}} = \frac{2 \pi}{\frac{\pi}{2}} = 4$
herauslesen und damit die Korrespondenz
\begin{equation}
s(t) = \frac{2}{\pi} {\Sha}(\frac{2 t}{\pi}) =
\sum_{m=-\infty}^{+\infty} \delta(t-m T_\mathrm{s})
\quad\fourier\quad
S(\mathrm{j}\omega) = {\Sha}(\frac{\omega}{4}) =
4 \sum_{\nu=-\infty}^{+\infty} \delta(\omega-4 \nu)
\end{equation}
angeben. Das Spektrum, ein Dirac Impuls Kamm ist in \fig{fig:Sampling_01_DiracComb_EF235EE3D8}
dargestellt. Hier erfolgt die Darstellung mit Pfeil und Gewicht so wie wir es
für den Dirac Impuls eingeführt haben.

In \fig{fig:EF235EE3D8_TimeDomain} ist der Dirac Impuls Kamm $s(t)$ des Zeitbereichs
in rot dargestellt. Hier erfolgt die Darstellung der Dirac Impulse leicht abgewandelt
mit eine Raute. Es ist leichter in dem Diagramm alle beteiligten Signale einheitlich
zu programmieren und wir stellen so die Verknüpfung zur sogenannten
\texttt{stem}-Darstellung von diskreten Folgen her, auch wenn $s(t)$ streng genommen
noch keine zeitdiskrete Folge ist.

\textbf{Rekonstruktionsfilter mit idealem Tiefpass, Signal und Spektrum}:
Wir sind mittlerweile sehr sicher bei den Korrespondenzen zur Rechteck- und
Spaltfunktion und finden
\begin{equation}
h_\mathrm{r}(t) = \mathrm{sinc}(2 t) \quad\fourier\quad
H_\mathrm{r}(\mathrm{j}\omega) = \frac{\pi}{2}\mathrm{rect}(\frac{\omega}{4}).
\end{equation}

Die Impulsantwort ist in \fig{fig:Sampling_05_LowpassIR_EF235EE3D8} aufgetragen,
das Spektrum, also der Frequenzgang in \fig{fig:Sampling_03_DiscreteSine_EF235EE3D8}.
Es handelt sich um das ideale Tiefpassfilter (das Filter als Neutrum ist die in der
Signalverarbeitung üblichere Bezeichnung) mit einer Grenzfrequenz von $\omega_c=2$ rad/s.
Das Filter lässt Signalanteile der Frequenzen $0\leq |\omega_c|<2$ perfekt passieren
(nur hier in der Amplitude $\nicefrac{\pi}{2}$ gewichtet), höherfrequente
Signalanteile, also $|\omega_c|>2$, werden perfekt unterdrückt.
Das Filter ist nicht realisierbar, da es eine unendliche Impulsantwort hat.
Es ist aber herausragend gut geeignet, die Theorie der Abtastung und Rekonstruktion
zu beschreiben, in fact: damit wurde das Abtasttheorem erfunden. Dies ist verglichen
mit anderen Tools aus SigSys noch gar nicht so lange her. Die potentielle
Rekonstruktion mit der Sinc-Funktion datiert auf 1910er Jahre (1915 Whittaker),
die Abtastung und fehlerfreie Rekonstruktion auf die 1930er (1933 Kotelnikov).
Ab 1948 ist es auch in der westlichen Literatur durch 'Neuerfindung' von Shannon
flächendeckend bekannt geworden und dann als Shannon'sches Abtasttheorem in die
Technikhistorie eingegangen.
\end{Ansatz}
\begin{ExCalc}
Wir müssen nun 'einfach' dem Signaflussgraphen des idealen Abtast-Rekonstruktionsmodells
folgen und nach und nach alle Signale und deren Spektren zusammentragen und dabei
verstehen, was im Detail passiert.

\textbf{Abtastung}:
Die Abtastung in Formeln aufgeschrieben lautet für den Zeitbereich
\begin{equation}
x_\mathrm{s}(t) = \sin(3 t) \cdot \frac{2}{\pi} {\Sha}(\frac{2 t}{\pi}) =
\sin(3 t) \cdot \sum_{m=-\infty}^{+\infty} \delta(t-m \frac{\pi}{2})=
\sum_{m=-\infty}^{+\infty} \sin(3 m \frac{\pi}{2}) \cdot \delta(t-m \frac{\pi}{2}),
\end{equation}
letzteres unter Ausnutzung der Multiplikationseigenschaft des Dirac Impulses.

In \fig{fig:TimeSignals_EF235EE3D8} ist das der mit $\sin(3 m \frac{\pi}{2})$
gewichtete  Dirac Impuls Kamm in schwarz, der Abstand der Dirac Impulse ist jeweils
$T_\mathrm{s}=\nicefrac{\pi}{2}$ s,
die Gewichte können durch das Sinusargument nur $0$ oder $\pm 1$ sein.
Machen wir uns hier klar: Diese Dirac Impulse enger zusammenschieben und daher
den Sinus noch genauer über noch mehr unterschiedliche Gewichte erfassen, erfordert
Abtastintervall $T_\mathrm{s}$ verringern, also (Kreis)-Abtastfrequenz $\omega_\mathrm{s}$
erhöhen.

Im Frequenzbereich korrespondiert dieses abgetastete Signal mit der Faltung der Spektren
\begin{equation}
X_\mathrm{s}(\im\omega) = \frac{1}{2\pi}
\left[
\mathrm{j} \pi \left( \delta(\omega+3) - \delta(\omega-3) \right)
\ast_\omega 4 \sum_{\nu=-\infty}^{+\infty} \delta(\omega-4 \nu)
\right].
\end{equation}
%
Hier im gewählten Beispiel haben wir es wieder einmal mit der Faltung zweier
Dirac Impulse zu tun. Es hilft uns nun wieder die Beziehung (vgl. Glg. (4.59))
\begin{align}
\delta(\omega-\omega_1) \ast_\omega \delta(\omega-\omega_2)=
\delta(\omega-[\omega_1+\omega_2]).
\end{align}
Nehmen wir das einmal in aller Ausführlichkeit auseinander, wir müssen es einmal
gesehen haben, später können wir uns das anhand von Skizzen veranschaulichen.
Zunächst erst einmal aufteilen und die Vorfaktoren ein wenig vereinfachen
%
% \begin{equation}
% X_\mathrm{s}(\im\omega) = \frac{4\pi\im}{2\pi}
% \left[
% \delta(\omega+3)
% \ast_\omega \sum_{\nu=-\infty}^{+\infty} \delta(\omega-4 \nu)
% \right]-
% \frac{4\pi\im}{2\pi}
% \left[
% \delta(\omega-3)
% \ast_\omega \sum_{\nu=-\infty}^{+\infty} \delta(\omega-4 \nu)
% \right].
% \end{equation}
%
\begin{equation}
X_\mathrm{s}(\im\omega) = 2\im
\delta(\omega+3)
\ast_\omega \sum_{\nu=-\infty}^{+\infty} \delta(\omega-4 \nu)
\quad-
2\im
\delta(\omega-3)
\ast_\omega \sum_{\nu=-\infty}^{+\infty} \delta(\omega-4 \nu).
\end{equation}
Sodann, damit wir es schneller sehen, umschreiben mit Vorzeichen
\begin{equation}
X_\mathrm{s}(\im\omega) = 2\im
\delta(\omega-(-3))
\ast_\omega \sum_{\nu=-\infty}^{+\infty} \delta(\omega-4 \nu)
\quad-
2\im
\delta(\omega-(+3))
\ast_\omega \sum_{\nu=-\infty}^{+\infty} \delta(\omega-4 \nu).
\end{equation}
Jetzt können wir die Faltung zweier Dirac Impulse für jedes einzelne $\nu$
der Summe durchspielen
\begin{align}
\delta(\omega-(-3)) \ast_\omega \delta(\omega-4 \nu) = \delta(\omega - [(-3)+4\nu])\\
\delta(\omega-(+3)) \ast_\omega \delta(\omega-4 \nu) = \delta(\omega - [(+3)+4\nu])
\end{align}
und können irgendwann einsehen, dass wir für die Berücksichtigung aller $\nu$
wieder Summen einführen können in der Form
\begin{equation}
X_\mathrm{s}(\im\omega) = 2\im
\sum_{\nu=-\infty}^{+\infty} \delta(\omega - [(-3)+4\nu])
\quad-
2\im
\sum_{\nu=-\infty}^{+\infty} \delta(\omega - [(+3)+4\nu]).
\end{equation}
Die erste Summe ist ein Dirac Impulskamm mit positiven Gewichten $2\im$, die
zweite ein Dirac Impulskamm mit negativen Gewichten $-2\im$. Der Abstand zwischen
den Dirac Impulsen beträgt $4$ rad/s, wobei wir hier die Einheit der
Übersichtlichkeit nicht mitschreiben.
%
Wir können uns das nochmal anders sortieren bzgl. der Variation mit $\nu$,
nämlich in Linksverschiebung (die beiden ersten Terme) und in Rechtsverschiebung
(die beiden letzten Terme) auftrennen, also
\begin{align}
X_\mathrm{s}(\im\omega) =
&+2\im\sum_{\nu=-\infty}^{-1} \delta(\omega - [4\nu-3])
-2\im\sum_{\nu=-\infty}^{-1} \delta(\omega - [4\nu+3])\\
&+2\im\sum_{\nu=0}^{+\infty} \delta(\omega - [4\nu-3])
-2\im\sum_{\nu=0}^{+\infty} \delta(\omega - [4\nu+3]).
\end{align}
Zu Erinnerung: die Zahl $4$ ist unsere Abtast-Kreisfrequenz und die Zahl $3$
die Kreisfrequenz des Sinus, also Parameter aus der Aufgabenstellung.

Damit können wir uns nun klarmachen, wie
\fig{fig:Sampling_03_DiscreteSine_EF235EE3D8} entsteht. Wir haben das Gesamtspektrum
in 4 Dirac Impuls Kämme zerlegt, die immer mit dem gleichen Frequenzabstand
(der Abtast-Kreisfrequenz agieren, das ist ja der Sinn der Sache) und wegen
des abgetasteten Sinussignals einen Offset von $\pm 3$ rad/s und $\pm 2\im$ Gewicht
aufweisen.
Wenn wir das in Skizzen später aufmalen, müssen wir 'nur' das Eingangsspektrum
sukzessive um Vielfache der Abtast-Kreisfrequenz verschieben und für die Amplitude
den bei der Faltung entstehenden Faktor $\frac{1}{2\pi}$ nicht vergessen, vgl.
Vorlesungsfolie 9-5.
\end{ExCalc}
\begin{Loesung}
\textbf{Rekonstruktion}:
Die Rekonstruktion ist im \textbf{Frequenzbereich} vergleichsweise einfach einzusehen,
da wir die Spektren $X_\mathrm{s}(\im\omega)$ und $H_\mathrm{r}(\im\omega)$
multiplizieren, was anschaulich gelingt, siehe
\fig{fig:Sampling_06_DiscreteSineLowpassSpectrum_EF235EE3D8}.
In Formeln ist das ein wenig umständlicher, aber im Vergleich zum Zeitbereich immer
noch überschaubar. Wir tragen zusammen
\begin{align}
&X_\mathrm{r}(\im\omega) =
H_\mathrm{r}(\mathrm{j}\omega) \cdot X_\mathrm{s}(\im\omega) =\\
&2\im
\frac{\pi}{2}\mathrm{rect}(\frac{\omega}{4}) \cdot \sum_{\nu=-\infty}^{+\infty} \delta(\omega - [(-3)+4\nu])
\quad-
2\im
\frac{\pi}{2}\mathrm{rect}(\frac{\omega}{4}) \cdot \sum_{\nu=-\infty}^{+\infty} \delta(\omega - [(+3)+4\nu]).
\end{align}
%
Jetzt können wir wieder einzelne $\nu$ anschauen, also überlegen, für welche $\nu$
der Ausdruck
\begin{equation}
\pm 2\im \frac{\pi}{2}\mathrm{rect}(\frac{\omega}{4}) \cdot \delta(\omega - [(\mp 3)+4\nu])
\end{equation}
nicht Null ist. Das Tiefpassfilter unterdrückt alle Frequenzen $|(\mp 3)+4\nu|>2$,
und weil $\nu\in\mathbb{Z}$ bleiben nur die beiden Frequenzanteile bei $\nu=\pm 1$,
die sowohl innerhalb des Durchlassbereichs des Rekonstruktionsfilters liegen
und gleichzeitig durch die Austasteigenschaft des Diracs eine Rolle im Spektrum
spielen. Es verbleiben also die Anteile
\begin{equation}
\pm \im \pi  \delta(\omega - [(\mp 3)+4 \cdot (\pm 1)]) =
+ \im \pi  \delta(\omega - [-3+4]) - \im \pi  \delta(\omega - [3-4])=
\im \pi  \delta(\omega - 1) - \im \pi  \delta(\omega + 1),
\end{equation}
und somit das rekonstruierte Spektrum
\begin{align}
X_\mathrm{r}(\im\omega) = \im \pi  \delta(\omega - 1) - \im \pi  \delta(\omega + 1),
\end{align}
welches in \fig{fig:Sampling_07_ReconstructedSine_EF235EE3D8} dargestellt ist.
Wir finden das korrespondierende rekonstruierte Zeitsignal ohne viel Mühe, weil
es eine gut bekannte Korrespondenz (nur mit mit invertierter Polarität) ist, also
\begin{equation}
x_r(t) = -\quad\sin(1\cdot t) \quad\fourier\quad X_r(\mathrm{j} \omega) =
 -\quad\mathrm{j} \pi [\delta(\omega+1) - \delta(\omega-1)].
\end{equation}
In \fig{fig:TimeSignals_EF235EE3D8} ist es das orangfarbene Sinussignal.
Aus $x(t)=\sin(3\,t)$ ist nach dieser Abtastung und Rekonstruktion mit idealem
Tiefpassfilter $x_r(t) = -\sin(1\cdot t)$ geworden. Falls wir uns nicht verrechnet
haben (sollten wir nicht, es könnten nur irgendwo unbeabsichtigte Typos am Start sein),
müssen wir also konstatieren, dass die Rekonstruktion nicht fehlerfrei gelingt.

Im \textbf{Zeitbereich} ist die Rechnerei ein wenig komplizierter (es ist im Grunde
nichts Neues, sondern bekannte SigSys Werkzeuge), aber die Interpretation ist deutlich
komplexer.
Machen wir zunächst den Ansatz gemäß Signalflussgraph
\begin{align}
&x_\mathrm{s}(t) =
\sum_{m=-\infty}^{+\infty} \sin(3 m \frac{\pi}{2}) \cdot \delta(t-m \frac{\pi}{2})\\
&h_\mathrm{r}(t) = \mathrm{sinc}(2 t)\\
&x_\mathrm{r}(t) = h_\mathrm{r}(t) \ast_t x_\mathrm{s}(t)
\end{align}
Wir erinnern uns, dass wir die Faltung eigentlich umgehen wollen, indem
wir die Multiplikationsdualität verwenden (haben wir im Frequenzbereich
ja bereits gemacht), hier aber didaktisch explizit gewollt.
Also schreiben wir das Faltungsintegral tatsächlich hin
\begin{align}
x_\mathrm{r}(t) =&
\mathrm{sinc}(2 t)
\ast_t
\sum_{m=-\infty}^{+\infty} \sin(3 m \frac{\pi}{2}) \cdot \delta(t-m \frac{\pi}{2})\\
=&
\int\limits_{-\infty}^{\infty}
\sum_{m=-\infty}^{+\infty} \sin(3 m \frac{\pi}{2}) \cdot \delta(\tau-m \frac{\pi}{2})
\cdot \mathrm{sinc}(2 [-\tau + t]) \fsd \tau\\
=&
\sum_{m=-\infty}^{+\infty} \sin(3 m \frac{\pi}{2}) \cdot
\int\limits_{-\infty}^{\infty} \delta(\tau-m \frac{\pi}{2})
\cdot \mathrm{sinc}(2 [-\tau + t]) \fsd \tau\\
=&
\sum_{m=-\infty}^{+\infty} \sin(3 m \frac{\pi}{2}) \cdot \mathrm{sinc}(2 [-m \frac{\pi}{2} + t]),
\label{eq:EF235EE3D8_SincSpecialResult}
\end{align}
wobei letzteres der Austasteigenschaft zu verdanken ist. Das Faltungsintegral selber
war also gar nicht so schlimm.
%
Wir erinnern uns, dass wir die Parameter Abtastintervall, $T_\mathrm{s}=\frac{\pi}{2}$ s,
Abtast-Kreisfrequenz $\omega_\mathrm{s}=4$  rad/s, Grenzfrequenz ideales
Tiefpassfilter $\omega_\mathrm{c} = 2$ rad/s und Sinusfrequenz $\omega_0=3$ rad/s
speziell gewählt hatten.
\end{Loesung}
%
%

\begin{mdframed}
Jetzt steht als letztes Ergebnis eine Summendarstellung, das allgemeingültig
\begin{align}
x_\mathrm{r}(t)
= \sum_{m=-\infty}^{+\infty} x(m T_\mathrm{s}) \cdot \mathrm{sinc}(\omega_\mathrm{c} [-m T_\mathrm{s} + t]),
\end{align}
lautet.
Nehmen wir nun für die Grenzfrequenz des idealen Tiefpassfilters
\begin{align}
\textbf{1.}:\quad\omega_\mathrm{c} = \frac{\omega_\mathrm{s}}{2}
\end{align}
an, dann lässt sich mit dem Zusammenhang
$\omega_\mathrm{s} = \frac{2\pi}{T_\mathrm{s}}$
\begin{align}
\label{eq:EF235EE3D8:ZeitSincInterpolation}
x_\mathrm{r}(t)=
\sum_{m=-\infty}^{+\infty} x(m T_\mathrm{s}) \cdot \mathrm{sinc}(\frac{\omega_\mathrm{s}}{2} [-m T_\mathrm{s} + t])=
\sum_{m=-\infty}^{+\infty} x(m T_\mathrm{s}) \cdot \mathrm{sinc}\left(\pi \cdot \frac{t-m T_\mathrm{s}}{T_\mathrm{s}}\right)
\end{align}
schreiben. Diese Formel gilt zunächst immer noch allgemein, 'nur' mit der
fixen Verknüpfung der Grenzfrequenz des idealen Tiefpasses mit der halben Abtastfrequenz.
Nehmen wir nun weiter an, dass das abzutastende Signal $x(t)$ nur Frequenzen
\begin{align}
\textbf{2.}:\quad<\omega_\mathrm{c}\quad\text{also}\quad<\frac{\omega_\mathrm{s}}{2}
\end{align}
enthält, dann (und nur dann, also bei Einhaltung von 1. und 2.) gilt
\begin{align}
x_\mathrm{r}(t) \equiv x(t).
\end{align}
Der mathematisch saubere Beweis dieser \textbf{perfekten Rekonstruktion}
mittels der Summenformel (sogenannte \textbf{Sinc-Interpolation}, 1915, Whittaker)
unter den beiden Bedingungen \textbf{1./2.} (1933, Kotelnikov) bilden die theoretische
Grundlage des \textbf{Abtasttheorem}s für tiefpassbegrenzte Signale.
%
Hinweis: in anderer Literatur ist $\mathrm{sinc}(x)=\frac{\sin(\pi x)}{\pi x}$ definiert,
dann schreiben wir den Interpolator in der Form
$\mathrm{sinc}\left(\frac{t-m T_\mathrm{s}}{T_\mathrm{s}}\right)$.
\end{mdframed}


\begin{Loesung}
Wir sehen relativ schnell ein, dass wir in der gestellten Aufgabe Bedingung 1.
erfüllen, aber nicht Bedingung 2., daher müssen wir erwarten, dass die perfekte
Rekonstruktion $x_\mathrm{r}(t) \equiv x(t)$ nicht gelingt,
was wir ja auch schon wissen.
%
Anders als im Frequenzbereich, werden wir unserer Summendarstellung
\eq{eq:EF235EE3D8_SincSpecialResult} im Zeitbereich das analytische Ergebnis
$x_\mathrm{r}(t)=-\sin(t)$ nicht so einfach analytisch entlocken können.
Das wollen wir uns hier auch nicht zur Aufgabe machen, weil wir ja elegantere
SigSys Tools haben, die uns das einfacher zeigen.
%

Sehr viel wichtiger müssen wir uns klar machen, was diese Summierung über
verschobene, gewichtetet Spaltfunktionen macht, das numerische Auswerten
überlassen wir dem Computer. Dazu mag die
\fig{fig:SincInterpolation_Aliasing_EF235EE3D8}
und
\fig{fig:SincInterpolation_NoAliasing_EF235EE3D8}
helfen.


\fig{fig:SincInterpolation_Aliasing_EF235EE3D8} zeigt exakt den Fall
\eq{eq:EF235EE3D8_SincSpecialResult}.
%
Für \fig{fig:SincInterpolation_NoAliasing_EF235EE3D8} wurde die Abtastfrequenz
auf $\omega_\mathrm{s}=8$ rad/s und gemäß Bedingung 1. die Grenzfrequenz
auf $\omega_\mathrm{c}=4$ rad/s erhöht. Wir sehen, dass wir nun auch Bedingung
2., also $\omega_0 < \frac{\omega_\mathrm{s}}{2}$ einhalten und damit
$x_\mathrm{r}(t) \equiv x(t)$ gilt, also fehlerfreie Abtastung und Rekonstruktion.

Wir müssen zu einer wichtigen Erkenntnis erlangen, um die Sinc-Interpolation
vom Wesen zu verstehen.
Es kommen zeitlich verschobene Sinc-Funktionen vor
$\mathrm{sinc}\left(\pi \cdot \frac{t-m T_\mathrm{s}}{T_\mathrm{s}}\right)$,
deren absolutes Maximum jeweils exakt bei $m T_\mathrm{s}$ liegt. Das ist
genau an der Stelle des $m$-ten Abtastwerts, also wo das Gewicht
$x(m T_\mathrm{s})$ mittels des Dirac Impuls Kamms ausgetastet wurde.
Die Nullstellen der Sinc-Funktionen liegen
nun durch 'geschickte' zeitliche Skalierung genau in allen anderen
$m T_\mathrm{s}$. D.h.
eine einzige Sinc-Funktion berücksichtigt nur einen einzigen Abtastwert.
Die Überlagerung aller Sinc-Funktionen (was wir schwer per Auge überblicken können)
liefert dann die rekonstruierte Funktion.

In \fig{fig:SincInterpolation_Aliasing_EF235EE3D8} sind die gewichteten Sinc-Funktionen
beispielhaft
für $m=1,2,3,4,5$ bunt dargestellt. Die Gewichte (also die Abtastwerte) für $m=2,4$
sind Null, wegen $\sin(3 m \frac{\pi}{2})$, daher ist auch die Sinc Funktion für diese
$m$ Null. Für $m=1$ resultiert die grüne Sinc-Funktion mit Gewicht -1, für $m=3$
die lila Kurve mit Gewicht 1, und magenta für $m=5$ mit Gewicht -1.

Viele weitere (eine endliche Zahl Sinc-Funktionen) sind in in grauen dünnen Linien
mit eingemalt. Die Überlagerung liefert die Sinc-Interpolation als rekonstruiertes
Signal, im Computer wurde $-200\leq m\leq 200$ berücksichtigt,
was ausreichend hoch gewählt ist,
dass das simulierte Ergebnis auch mit dem analytischen Ergebnis übereinstimmt.

\end{Loesung}










\begin{figure*}[h!]
\centering
\begin{subfigure}{0.49\textwidth}
\includegraphics[width=\textwidth]{../sampling/Sampling_02_Sine_EF235EE3D8.pdf}
\caption{Eingang $X(\im\omega)$.}
\label{fig:Sampling_02_Sine_EF235EE3D8}
\end{subfigure}
\begin{subfigure}{0.49\textwidth}
\includegraphics[width=\textwidth]{../sampling/Sampling_01_DiracComb_EF235EE3D8.pdf}
\caption{Dirac Impuls Kamm $S(\im\omega)$.}
\label{fig:Sampling_01_DiracComb_EF235EE3D8}
\end{subfigure}
\begin{subfigure}{0.49\textwidth}
\includegraphics[width=\textwidth]{../sampling/Sampling_03_DiscreteSine_EF235EE3D8.pdf}
\caption{Abtastversion $X_\mathrm{s}(\im\omega)$.}
\label{fig:Sampling_03_DiscreteSine_EF235EE3D8}
\end{subfigure}
\begin{subfigure}{0.49\textwidth}
\includegraphics[width=\textwidth]{../sampling/Sampling_04_LowpassSpectrum_EF235EE3D8.pdf}
\caption{Rekonstruktionsfilter Übertragungsfunktion $H_\mathrm{r}(\im\omega)$.}
\label{fig:Sampling_04_LowpassSpectrum_EF235EE3D8}
\end{subfigure}
\begin{subfigure}{0.49\textwidth}
\includegraphics[width=\textwidth]{../sampling/Sampling_06_DiscreteSineLowpassSpectrum_EF235EE3D8.pdf}
\caption{
\fig{fig:Sampling_03_DiscreteSine_EF235EE3D8}
und
\fig{fig:Sampling_04_LowpassSpectrum_EF235EE3D8}
zusammen.
}
\label{fig:Sampling_06_DiscreteSineLowpassSpectrum_EF235EE3D8}
\end{subfigure}
\begin{subfigure}{0.49\textwidth}
\includegraphics[width=\textwidth]{../sampling/Sampling_07_ReconstructedSine_EF235EE3D8.pdf}
\caption{Rekonstruierter Ausgang $X_\mathrm{r}(\im\omega)$.}
\label{fig:Sampling_07_ReconstructedSine_EF235EE3D8}
\end{subfigure}
\caption{Aufgabe \ref{sec:EF235EE3D8} Frequenzbereich.}
\label{fig:EF235EE3D8_FrequencyDomain}
\end{figure*}






\begin{figure*}[h!]
\centering
\begin{subfigure}{0.75\textwidth}
\includegraphics[width=\textwidth]{../sampling/Sampling_05_LowpassIR_EF235EE3D8.pdf}
\caption{Rekonstruktionsfilter Impulsantwort $h_\mathrm{r}(t)$}
\label{fig:Sampling_05_LowpassIR_EF235EE3D8}
\end{subfigure}

\begin{subfigure}{0.75\textwidth}
\includegraphics[width=\textwidth]{../sampling/TimeSignals_EF235EE3D8.pdf}
\caption{blau: Eingang $x(t)$, rot: Dirac Impuls Kamm $s(t)$,
schwarz: Abtastversion $x_\mathrm{s}(t)$, orange: rekonstruierter Ausgang $x_\mathrm{r}(t)$}
\label{fig:TimeSignals_EF235EE3D8}
\end{subfigure}
\caption{Aufgabe \ref{sec:EF235EE3D8} Zeitbereich.}
\label{fig:EF235EE3D8_TimeDomain}
\end{figure*}





\begin{figure*}[h!]
\centering
\begin{subfigure}{0.75\textwidth}
\includegraphics[width=\textwidth]{../sampling/SincInterpolation_Aliasing_EF235EE3D8.pdf}
\caption{Abtastung im Zeitbereich und \textbf{nicht fehlerfreie Rekonstruktion}
durch Unterabtastung (Bedingung 2. verletzt).}
\label{fig:SincInterpolation_Aliasing_EF235EE3D8}
\end{subfigure}

\begin{subfigure}{0.75\textwidth}
\includegraphics[width=\textwidth]{../sampling/SincInterpolation_NoAliasing_EF235EE3D8.pdf}
\caption{Abtastung im Zeitbereich und \textbf{fehlerfreie Rekonstruktion},
Bedingung 1. und 2. eingehalten.}
\label{fig:SincInterpolation_NoAliasing_EF235EE3D8}
\end{subfigure}
\caption{Aufgabe \ref{sec:EF235EE3D8} Zeitbereich.
Darstellung der Sinc-Interpolation.}
\label{fig:SincInterpolation_EF235EE3D8}
\end{figure*}






\clearpage
\section*{Vorbetrachtung: Zusammenhang zwischen Fourier Reihe und Fourier
Transformation,
oder:
Abtastung \& Rekonstruktion eines Fourier-Transformation-Spektrums}
%
Betrachten wir die komplexe Fourierreihe für die Periodendauer
$T_s$ und die Grundkreisfrequenz $\omega_s=\frac{2\pi}{T_s}$.
Dann ist die Signalsynthese für ein $T_s$-periodisches Signal mittels Fourierreihe
gegeben
\begin{align}
  \tilde{x}(t) = \sum_{\nu=-\infty}^{+\infty} X[\nu\,\omega_s]
  \e^{+\im\,(\omega_s \nu)\cdot t}.
\end{align}
Wir bezeichnen hier die Fourierkoeffizienten mit der Schreibweise
$X[\nu\,\omega_s]$,
dann sind wir mit dem Großbuchstaben $X$ mehr in unserer SigSys-Welt und mit
dem Term $\nu\,\omega_s$ können wir sofort sehen,
welche tatsächliche Frequenz $\omega$ welcher Harmonischen $\nu$ zuzuordnen ist.
Die eckigen Klammern sollen andeuten, dass
es abzählbare $\nu$ gibt, also dass die 'Funktion' $X[\nu\omega_s]$
als Zahlenfolge aufzufassen ist. Das ist dann auch für das Kapitel zeitdiskrete
Signale- und Systeme eine Schreibweise die wir oft benutzen.
%
Nehmen wir ein weiteres Signal
\begin{align}
  x(t) =
  \begin{cases}
  \tilde{x}(t)&\text{für}\quad-\frac{T_s}{2} \leq t < \frac{T_s}{2}\\
  0&\text{sonst}
  \end{cases}
\end{align}
an, welches mit dem periodischen Signal $\tilde{x}(t)$
in einer einzelnen Periode um $t=0$ herum übereinstimmt
und außerhalb dieses Zeitbereichs Null ist, also ein einmaliger Vorgang.
Wenn wir weiter annehmen, dass die Amplitude in dieser Periode beschränkt ist,
ist Fouriertransformierte zu
\begin{align}
  X(\im\omega) = \int\limits_{-\frac{T_s}{2}}^{+\frac{T_s}{2}} x(t) \e^{-\im\omega t} \fsd t
\end{align}
gegeben.
Nun können wir für $x(t)$ die Syntheseformel der komplexen Fourierreihe einsetzen,
also
\begin{align}
X(\im\omega) = \int\limits_{-\frac{T_s}{2}}^{+\frac{T_s}{2}}
\left(\sum_{\nu=-\infty}^{+\infty} X[\nu\,\omega_s] \e^{\im\,(\omega_s \nu)\cdot t}\right)
\e^{-\im\omega t} \fsd t.
\end{align}
Das lässt sich umstellen und ausrechnen zu
\begin{align}
&X(\im\omega) = \sum_{\nu=-\infty}^{+\infty} X[\nu\,\omega_s]
\int\limits_{-\frac{T_s}{2}}^{+\frac{T_s}{2}}
\e^{-\im (\omega-\nu\omega_s) t} \fsd t=
\sum_{\nu=-\infty}^{+\infty} X[\nu\,\omega_s]
\frac{\e^{-\im (\omega-\nu\omega_s) t}}{-\im (\omega-\nu\omega_s)}
\bigg|_{-\frac{T_s}{2}}^{+\frac{T_s}{2}}\\
%\end{align}
%Grenzen einsetzen und wie üblich (siehe Übung 1) vereinfachen
%\begin{align}
&X(\im\omega) = \sum_{\nu=-\infty}^{+\infty} X[\nu\,\omega_s]\left(
-\frac{\e^{-\im (\omega-\nu\omega_s)\frac{T_s}{2}}}{\im (\omega-\nu\omega_s)}+
\frac{\e^{\im (\omega-\nu\omega_s)\frac{T_s}{2}}}{\im (\omega-\nu\omega_s)}\right)
\cdot \frac{2}{2} \cdot \frac{\frac{T_s}{2}}{\frac{T_s}{2}} =
%\sum_{\nu=-\infty}^{+\infty} X[\nu\,\omega_s] \cdot T_s \cdot
%\frac{\sin\left((\omega-\nu\omega_s)\frac{T_s}{2}\right)}{(\omega-\nu\omega_s)\frac{T_s}{2}}=
\sum_{\nu=-\infty}^{+\infty} X[\nu\,\omega_s] \cdot T_s \cdot
\mathrm{sinc}\left((\omega-\nu\omega_s)\frac{T_s}{2}\right).
\end{align}
Wenn wir mit $T_s=\frac{2\pi}{\omega_s}$ noch umschreiben
\begin{align}
\label{eq:SamplingFreqSincInterp}
X(\im\omega) = \sum_{\nu=-\infty}^{+\infty} \left(T_s X[\nu\,\omega_s]\right) \cdot
\mathrm{sinc}\left(\pi\cdot\frac{\omega-\nu\omega_s}{\omega_s}\right),
\end{align}
dann erkennen wir eine \textbf{verblüffende Ähnlichkeit mit der Sinc-Interpolation
im Zeitbereich} \eq{eq:EF235EE3D8:ZeitSincInterpolation}.
%
Wir können mutmaßen (es ist tatsächlich so), dass wir es hier mit einem
Abtast- und Rekonstruktionsproblem zu tun haben und gerade die Rekonstruktionsformel
neu erfunden haben, über die sich die Fourierreihe $X[\nu\,\omega_s]$
und die Fouriertransformation $X(\im\omega)$
verknüpfen lassen: Wir tasten Spektren ab und rekonstruieren diese.

Es ist nun sinnvoll, wenn wir uns nochmal Übung 1 hervorholen, speziell
die Aufgaben 1.1 und 1.2. Die beiden Grafiken der Zeitsignale und die
Ergebnis-Formeln der Spektren sind unten abgebildet, angepasst an unsere
jetzige Notation mit $\omega_s=\frac{2\pi}{T_s}$.
%
\begin{center}
\begin{tikzpicture}[domain=0:2]
\def\T{0.4}
%
\begin{scope}[shift={(0,0)}]
\draw[->] (-3,0) -- (3,0) node[below right] {$t$};
\draw[->] (0,0) -- (0,1.5) node[above] {$\tilde{x}(t)$};
\foreach \pos in {-2,...,2} {
\draw[-, C1, ultra thick] (\pos-\T,0) -- (\pos-\T,1) -- (\pos+\T,1) -- (\pos+\T,0) -- (\pos+1-\T,0);
};
\draw[-, C1, ultra thick] (-\T,0) node[below] {$\frac{-T_h}{2}$};
\draw[-, C1, ultra thick] (+\T,0) node[below] {$\frac{+T_h}{2}$};
\draw[-, C1, ultra thick] (1,0) node[below] {$T_s$};
\draw[-, C1, ultra thick] (2,0) node[below] {$2 T_s$};
\draw[-, C1, ultra thick] (-2-\T,1) node[left] {$A$};
\draw[-, C7, thin] (0.5,0) -- (0.5,1.1) node[above] {$\frac{T_\mathrm{s}}{2}$};
\draw[-, C7, thin] (-0.5,0) -- (-0.5,1.1) node[above] {$\frac{-T_\mathrm{s}}{2}$};
\draw[-, black, ultra thick] (3.5,-1.1) node[left]
{Glg. (1.10): $X[\nu\,\omega_s] = A \frac{T_h}{T_s} \cdot \mathrm{sinc}(\omega_s \nu \frac{T_h}{2})$}; %hard coded ref!!!
\end{scope}
%
\begin{scope}[shift={(8,0)}]
\draw[->] (-3,0) -- (3,0) node[below right] {$t$};
\draw[->] (0,0) -- (0,1.5) node[above] {${x}(t)$};
\foreach \pos in {-0,...,0} {
\draw[-, C0, ultra thick] (\pos-\T-2,0) -- (\pos-\T,0) -- (\pos-\T,1) -- (\pos+\T,1) -- (\pos+\T,0) -- (\pos+1-\T+2,0);
};
\draw[-, C0, ultra thick] (-\T,0) node[below] {$\frac{-T_h}{2}$};
\draw[-, C0, ultra thick] (+\T,0) node[below] {$\frac{+T_h}{2}$};
\draw[-, C0, ultra thick] (1,0) node[below] {$T_s$};
\draw[-, C0, ultra thick] (2,0) node[below] {$2 T_s$};
\draw[-, C0, ultra thick] (0-\T,1) node[left] {$A$};
\draw[-, C7, thin] (0.5,0) -- (0.5,1.1) node[above] {$\frac{T_\mathrm{s}}{2}$};
\draw[-, C7, thin] (-0.5,0) -- (-0.5,1.1) node[above] {$\frac{-T_\mathrm{s}}{2}$};
\draw[-, black, ultra thick] (3.5,-1.1) node[left]
{Glg. (1.21): $X(\im \omega) = A T_h \cdot \mathrm{sinc}(\omega \frac{T_h}{2})$}; %hard coded ref!!!
\end{scope}
\end{tikzpicture}
\end{center}
%
Spielen wir die Abtastung eines Spektrums nun mal mit den beiden Grafiken
in Gedanken durch (wir rechnen es gleich im Detail aus):
\begin{itemize}
  \item es gibt ein zeitlich begrenztes Signal $x(t)$ (blau)
  und dazu gehört ein nicht bandbegrenztes Spektrum im Bildbereich der
  Fouriertransformation, wir wissen, dass es ein sinc-förmiges Spektrum ist (Abb. 1.6)
  \item dieses Spektrum wird nun im Frequenzbereich mit einem Dirac Impuls Kamm
  abgetastet, es entsteht ein sogenanntes Linienspektrum, vgl. Abb. 1.1 blaue Punkte
  \item Abtastung heißt im anderen Bereich Faltung:
  also für den Zeitbereich Faltung mit dem Dirac Impuls Kamm, dadurch
  entstehen periodische Wiederholungen
  des zeitlich begrenzten Signals, es entsteht ein periodisches Zeitsignal
  $\tilde{x}(t)$ (orange, vgl. Aufgabe 1.1)
  \item wir wissen, dass periodische Zeitsignale mit einer
  komplexen Fourierreihe beschrieben werden können, der Bildbereich der Fourierreihe
  ist ein Linienspektrum und enthält die Spektrums-Information nach der Abtastung
  \item Bei der Rekonstruktion wird mittels der oben gefundenen Sinc-Interpolation
  \eq{eq:SamplingFreqSincInterp}
  aus diesem Linienspektrum wieder eine kontinuierliches Spektrum, im besten
  Fall (d.h. perfekter Rekonstruktion) erhalten wir das originale Spektrum zurück
  \item im Zeitbereich bedeutet dies äquivalent, dass alle periodischen
  Wiederholungen weggeschnitten werden und wieder nur das zeitlich begrenzte,
  originale Signal übrig bleibt.
\end{itemize}
Die Rekonstruktion gelingt dann fehlerfrei, wenn der zeitlich begrenzte
Signalausschnitt nicht über $\pm \frac{T_s}{2}$ hinausragt, also für die Skizze
$\frac{T_h}{2}<\frac{T_s}{2}$. Dann (und nur dann)
beinhaltet das Linienspektrum der Fourierreihe \textbf{alle} Information,
um das Spektrum der Fouriertransformation
des zeitlich begrenzten Signalausschnitts beschreiben zu können. Dafür gibt es
nur einen einzig möglichen Verlauf, der mittels der Sinc-Interpolationsformel
definiert ist. Das ist das Abtasttheorem für zeitlich begrenzte Signal um den
Nullpunkt $t=0$.

Ähnlich wie in der vorhergehenden Aufgabe, kann man in der Sinc-Interpolationsformel
wieder nicht sehen, dass perfekt rekonstruiert werden kann, was in dem anderen
Bereich (hier der Zeitbereich) schnell und übersichtlich offenbart wird.

\newpage
\subsection{Abtastung und Rekonstruktion eines Spektrums: Sinc}
\label{sec:45C76AFB33}
\begin{Ziel}
Mit einem einfachen Beispiel wollen wir die Abtastung/Rekonstruktion von Spektren
mal durchspielen und dabei die gewonnene Erkenntnisse aus der
Vorbetrachtung vertiefen.
Diesmal wird der Dirac Impuls Kamm im Frequenzbereich multipliziert und daraus
folgt eine Faltung mit dem Dirac Impuls Kamm im Zeitbereich.
Im Frequenzbereich wird mit dem Rekonstruktionsspektrum gefaltet, daher
im Zeitbereich mit einem Rekonstruktionszeitfenster multipliziert.
Im Frequenzbereich bekommen wir die Sinc-Interpolationsformel, der wir
mit bloßem Auge nicht ansehen, dass sie in dem gewählten Beispiel perfekt
rekonstruiert. Im Zeitbereich ist das mit einfachen Skizzen sofort ersichtlich.
\end{Ziel}
\textbf{Aufgabe} {\tiny 45C76AFB33}:
Für das dargestellte Abtast- und Rekonstruktionsmodell von Spektren
\begin{center}
\begin{tikzpicture}[align=center,node distance=0.5cm and 0.5cm]
\tikzstyle{block} = [draw, rectangle, minimum height=3em, minimum width=3em]
\begin{scope}
\node (input) {$\displaystyle x(t)$};
\node [draw, circle, right=of input, minimum size=7.5mm] (op1) {$\ast_t$};
\node [above=of op1] (sop1) {$s(t)$};
\node [draw, circle, right=of op1, minimum size=7.5mm, xshift=0.5cm] (op2) {$\cdot$};
\node [above=of op2] (sop2) {$h_\mathrm{r}(t)$};
\node [block, right=of op2] (mult) {$1$};
\node [right=of mult] (output) {$x_\mathrm{r}(t)$};
\draw [->] (input) -- (op1);
\draw [->] (op1) node[below, xshift=0.9cm, yshift=0cm] {$x_\mathrm{s}(t)$} -- (op2);
\draw [->] (op2) -- (mult);
\draw [->] (mult) -- (output);
\draw [->] (sop1) -- (op1);
\draw [->] (sop2) -- (op2);
\node [right=of output, xshift=-0.4cm, yshift=0.5cm] (fourier) {$\fourier$};
\end{scope}
%
\begin{scope}[shift={(8,0)}]
\node (input) {$\displaystyle X(\im\omega)$};
\node [draw, circle, right=of input, minimum size=7.5mm] (op1) {$\cdot$};
\node [above=of op1] (sop1) {$S(\im\omega)$};
\node [draw, circle, right=of op1, minimum size=7.5mm, xshift=0.5cm] (op2) {$\ast_\omega$};
\node [above=of op2] (sop2) {$H_\mathrm{r}(\im\omega)$};
\node [block, right=of op2] (mult) {$\frac{1}{2\pi}$};
\node [right=of mult] (output) {$X_\mathrm{r}(\im\omega)$};
\draw [->] (input) -- (op1);
\draw [->] (op1) node[below, xshift=0.9cm, yshift=0cm] {$X_\mathrm{s}(\im\omega)$} -- (op2);
\draw [->] (op2) -- (mult);
\draw [->] (mult) -- (output);
\draw [->] (sop1) -- (op1);
\draw [->] (sop2) -- (op2);
\end{scope}
\end{tikzpicture}
\end{center}
soll für
\begin{align}
X(\im\omega) = \mathrm{sinc}(\frac{\omega}{4})\qquad
%
S(\im\omega)=\frac{1}{2\pi} \Sha(\frac{\omega}{2\pi})\qquad
%
h_r(t) = 2\pi\mathrm{rect}(t)
\end{align}
die Abtastung von $X(\im\omega)$ und die Rekonstruktion zu $X_\mathrm{r}(\im\omega)$
berechnet und skizziert werden.

\begin{itemize}
  \item Berechnen / Skizzieren Sie alle Signale, d.h. $x(t)$, $s(t)$, $x_\mathrm{s}(t)$, $h_\mathrm{r}(t)$ und $x_\mathrm{r}(t)$
  \item Berechnen / Skizzieren Sie alle Spektren, d.h.  $X(\im\omega)$, $S(\im\omega)$, $X_\mathrm{s}(\im\omega)$, $H_\mathrm{r}(\im\omega)$ und $X_\mathrm{r}(\im\omega)$
  \item Ist die Rekonstruktion des Spektrums erfolgreich?
\end{itemize}

\begin{Werkzeug}
Siehe Aufgabe \ref{sec:EF235EE3D8}.
\end{Werkzeug}
\begin{Ansatz}
Siehe Aufgabe \ref{sec:EF235EE3D8}, in Gedanken müssen wir hier Faltung und
Multiplikation drehen. Das kann am Anfang für Verwirrung sorgen, wenn wir das
aber mal durchgespielt haben, werden wir mit einem
universellen Blick auf die Dinge belohnt.
\end{Ansatz}
\begin{ExCalc}
\textbf{Eingang, Signal und Spektrum}:
\begin{align}
x(t) = 2 \mathrm{rect}(2 t)
\quad\fourier\quad
X(\im\omega) = \mathrm{sinc}(\frac{\omega}{4})
\end{align}
%
\textbf{Abtastung, Signal und Spektrum}:
Fundament für eine sinnvolle Bearbeitung solcher Abtastaufgaben ist zunächst die
Dualität fehlerfrei niederzuschreiben
\begin{align}
\frac{1}{T_\mathrm{s}} {\Sha}(\frac{t}{T_\mathrm{s}}) \quad\fourier\quad
{\Sha}(\frac{\omega T_\mathrm{s}}{2 \pi}) =
\Sha(\frac{\omega}{\omega_\mathrm{s}})\\
\sum_{m=-\infty}^{+\infty} \delta(t-m T_\mathrm{s}) \quad\fourier\quad
\omega_s \sum_{\nu=-\infty}^{+\infty} \delta(\omega-\nu \omega_s).
\end{align}
Aus ${\Sha}(\frac{\omega T_\mathrm{s}}{2 \pi})$ ermitteln wir
$T_\mathrm{s} = 1$ s und damit $\omega_\mathrm{s}=\frac{2\pi}{T_s} = 2\pi$ rad/s.
Die Abtast-(kreis)-frequenz $\omega_\mathrm{s}=\frac{2\pi}{T_s} = 2\pi$ rad/s
hätten wir auch direkt aus der Darstellung $\Sha(\frac{\omega}{\omega_\mathrm{s}})$
ermitteln können.
%
Der Dirac Impuls Kamm im Frequenzbereich aus der Aufgabenstellung ist nun noch
gewichtet mit $\frac{1}{2\pi}$, was wir beidseitig einführen (Linearitätseigenschaft)
\begin{align}
\frac{1}{2\pi}\cdot\frac{1}{T_\mathrm{s}} {\Sha}(\frac{t}{T_\mathrm{s}}) \quad\fourier\quad
\frac{1}{2\pi}\cdot{\Sha}(\frac{\omega T_\mathrm{s}}{2 \pi}) =
\frac{1}{2\pi}\cdot\Sha(\frac{\omega}{\omega_\mathrm{s}})\\
\frac{1}{2\pi}\cdot\sum_{m=-\infty}^{+\infty} \delta(t-m T_\mathrm{s}) \quad\fourier\quad
\frac{1}{2\pi}\cdot\omega_s \sum_{\nu=-\infty}^{+\infty} \delta(\omega-\nu \omega_s)
\end{align}
%
Gegebene Zahlenwerte berücksichtigt
\begin{align}
\frac{1}{2\pi}\cdot\frac{1}{1} {\Sha}(\frac{t}{1}) \quad\fourier\quad
\frac{1}{2\pi}\cdot{\Sha}(\frac{\omega \cdot 1}{2 \pi}) =
\frac{1}{2\pi}\cdot\Sha(\frac{\omega}{2\pi})\\
\frac{1}{2\pi}\cdot\sum_{m=-\infty}^{+\infty} \delta(t-m \cdot 1) \quad\fourier\quad
\frac{1}{2\pi}\cdot 2\pi \sum_{\nu=-\infty}^{+\infty} \delta(\omega-\nu 2\pi)
\end{align}
%
Für das \textbf{Rekonstruktionszeitfenster} können wir direkt aus der Formelsammlung
\begin{align}
h_r(t) = 2\pi\mathrm{rect}(t)
\quad\fourier\quad
H_r(\im\omega) = 2\pi\mathrm{sinc}(\frac{\omega}{2})
\end{align}
aufschreiben, ganz ohne Zeitskalierung, nur mit Linearitätseigenschaft.
%
\end{ExCalc}
\begin{Loesung}
%
Im \textbf{Zeitbereich} ist die Abtastung des Spektrums
\begin{align}
x_\mathrm{s}(t) = x(t) \ast_t s(t) = \frac{1}{2\pi} \sum_{m=-\infty}^{+\infty} 2\mathrm{rect}(2[t-1])
\end{align}
und die Spektren-Rekonstruktion
\begin{align}
x_\mathrm{r}(t) = h_\mathrm{r}(t) \cdot x_\mathrm{s}(t) =
2\pi \mathrm{rect}(t) \cdot \left(
\frac{1}{2\pi} \sum_{m=-\infty}^{+\infty} 2\mathrm{rect}(2[t-1])
\right) = 2\mathrm{rect}(2t) \equiv x(t)
\end{align}
vergleichsweise einfach einzusehen, vor allem mit der zugehörigen Grafik unten.
Wir sehen, dass in diesem Fall perfekte Rekonstruktion möglich ist.
%
\begin{center}
\begin{tikzpicture}[domain=0:2]
\def\T{0.25}
%
\begin{scope}[shift={(0,0)}]
\draw[->] (-3,0) -- (3,0) node[below right] {$t$};
\draw[->] (0,0) -- (0,1.5) node[above] {$x(t)$};
\foreach \pos in {-0,...,0} {
\draw[-, C0, ultra thick] (\pos-\T-2,0) -- (\pos-\T,0) -- (\pos-\T,1) -- (\pos+\T,1) -- (\pos+\T,0) -- (\pos+1-\T+2,0);
};
\draw[-, C0, ultra thick] (-\T,0) node[below] {$\frac{-1}{4}$};
\draw[-, C0, ultra thick] (+\T,0) node[below] {$\frac{1}{4}$};
\draw[-, C0, ultra thick] (1,0) node[below] {$1$};
\draw[-, C0, ultra thick] (2,0) node[below] {$2$};
\draw[-, C0, ultra thick] (-\T,1) node[left] {$2$};
\end{scope}
%
\begin{scope}[shift={(7,0)}]
\draw[->] (-3,0) -- (3,0) node[below right] {$t$};
\draw[->] (0,0) -- (0,1.5) node[above] {$s(t)$};
\foreach \pos in {-2,...,2} {
\draw[->, black, ultra thick] (\pos,0) -- (\pos,1);
};
\draw[-, C0, ultra thick] (-\T,0) node[below] {$\frac{-1}{4}$};
\draw[-, C0, ultra thick] (+\T,0) node[below] {$\frac{1}{4}$};
\draw[-, black, ultra thick] (1,0) node[below] {$1$};
\draw[-, black, ultra thick] (2,0) node[below] {$2$};
\draw[-, black, ultra thick] (-2-\T,1) node[left] {$(\frac{1}{2\pi})$};
\end{scope}
%
\begin{scope}[shift={(0,-3)}]
\draw[->] (-3,0) -- (3,0) node[below right] {$t$};
\draw[->] (0,0) -- (0,1.5) node[above] {$x_\mathrm{s}(t) = x(t)\ast s(t)$};
\foreach \pos in {-2,...,2} {
\draw[-, C1, ultra thick] (\pos-\T,0) -- (\pos-\T,1) -- (\pos+\T,1) -- (\pos+\T,0) -- (\pos+1-\T,0);
};
\draw[-, C1, ultra thick] (-\T,0) node[below] {$\frac{-1}{4}$};
\draw[-, C1, ultra thick] (+\T,0) node[below] {$\frac{1}{4}$};
\draw[-, C1, ultra thick] (1,0) node[below] {$1$};
\draw[-, C1, ultra thick] (2,0) node[below] {$2$};
\draw[-, C1, ultra thick] (-2-\T,1) node[left] {$\frac{2}{2\pi}$};
\end{scope}
%
\begin{scope}[shift={(7,-3)}]
\draw[->] (-3,0) -- (3,0) node[below right] {$t$};
\draw[->] (0,0) -- (0,1.5) node[above] {$h_\mathrm{r}(t)$};
\draw[-, C2, ultra thick] (-3,0) -- (-1/2,0) -- (-1/2,1) -- (+1/2,1) -- (+1/2,0) -- (3,0);
\draw[-, C2, ultra thick] (-1/2,0) node[below] {$\frac{-1}{2}$};
\draw[-, C2, ultra thick] (+1/2,0) node[below] {$\frac{+1}{2}$};
\draw[-, C2, ultra thick] (1,0) node[below] {$1$};
\draw[-, C2, ultra thick] (2,0) node[below] {$2$};
\draw[-, C2, ultra thick] (-1/2,1) node[left] {$2\pi$};
\end{scope}
%
\begin{scope}[shift={(3.5,-6)}]
\draw[->] (-3,0) -- (3,0) node[below right] {$t$};
\draw[->] (0,0) -- (0,1.5) node[above] {$x_\mathrm{r}(t) = x_\mathrm{s}(t) \cdot h_\mathrm{r}(t)$};
\foreach \pos in {-0,...,0} {
\draw[densely dashed, C3, ultra thick] (\pos-\T-2,0) -- (\pos-\T,0) -- (\pos-\T,1) -- (\pos+\T,1) -- (\pos+\T,0) -- (\pos+1-\T+2,0);
};
\draw[-, C3, ultra thick] (-\T,0) node[below] {$\frac{-1}{4}$};
\draw[-, C3, ultra thick] (+\T,0) node[below] {$\frac{1}{4}$};
\draw[-, C3, ultra thick] (1,0) node[below] {$1$};
\draw[-, C3, ultra thick] (2,0) node[below] {$2$};
\draw[-, C3, ultra thick] (-\T,1) node[left] {$2$};
\end{scope}
%
\end{tikzpicture}
\end{center}
%
%
%

Die Abtastung eines Spektrums und dessen Rekonstruktion ist im
\textbf{Frequenzbereich} wieder deutlich frickeliger.
Es ist im Grunde aber die gleiche Rechnung wie in der vorangegangenen Aufgabe,
wir müssen auch wieder das Faltungsintegral explizit aufschreiben (als
Frequenzhilfsvariable $o$ für das Integral) und so umformen, dass
nur die Summe über alle $\nu$ bleibt. Das ist dann wieder die Sinc-Interpolation,
diesmal im Frequenzbereich.
\begin{align}
X_r(\im\omega) =& \frac{1}{2\pi} X_s(\im\omega) \ast_\omega H_r(\im\omega)\\
=&\frac{1}{2\pi}
\left(\sum_{\nu=-\infty}^{+\infty} \mathrm{sinc}(\frac{2\pi\nu}{4}) \delta(\omega-\nu 2 \pi)\right)
\ast_\omega
\left(2\pi\mathrm{sinc}(\frac{\omega}{2})\right)\\
=&\int\limits_{-\infty}^{+\infty}
\frac{1}{2\pi}
\left(\sum_{\nu=-\infty}^{+\infty} \mathrm{sinc}(\frac{2\pi\nu}{4}) \delta(o-\nu 2 \pi)\right)
\left(2\pi\mathrm{sinc}(\frac{-o+\omega}{2}\right)
\fsd o\\
=&
\sum_{\nu=-\infty}^{+\infty} \mathrm{sinc}(\frac{2\pi\nu}{4})
\int\limits_{-\infty}^{+\infty}
\delta(o-\nu 2 \pi)\cdot
\mathrm{sinc}(\frac{-o+\omega}{2})
\fsd o\\
=&
\sum_{\nu=-\infty}^{+\infty} \mathrm{sinc}(\frac{2\pi\nu}{4})\cdot
\mathrm{sinc}(\frac{-\nu 2 \pi+\omega}{2})\\
%=&
%\sum_{\nu=-\infty}^{+\infty} \mathrm{sinc}(\frac{2\pi\nu}{4})\cdot
%\mathrm{sinc}(\frac{\omega-\nu 2 \pi}{2})\\
=&
\sum_{\nu=-\infty}^{+\infty} \mathrm{sinc}(\frac{2\pi\nu}{4})\cdot
\mathrm{sinc}(\pi\cdot\frac{\omega-\nu 2 \pi}{2\pi})
\end{align}
Dieses Ergebnis kennen wir in allgemeiner Form aus \eq{eq:SamplingFreqSincInterp}.
%
Wir können aus Übung 1 Glg. (1.10) leicht ermitteln, dass die Koeffizienten
der komplexen Fourierreihe für unser gewähltes Signal $x_\mathrm{s}(t)$ mit
$\frac{T_h}{2} = \frac{T_s}{4}$ und $A=2$
\begin{equation}
X[\nu\,\omega_s] = \mathrm{sinc}(\omega_s \nu \frac{T_s}{4})
\end{equation}
lautet und weil $T_\mathrm{s}=1$ s, $\omega_\mathrm{s}=2\pi$ rad/s folgt
\begin{equation}
T_s \cdot X[\nu\,\omega_s] = 1 \cdot \mathrm{sinc}(2\pi \nu \frac{1}{4}).
\end{equation}
Dies ist genau der Term, der in unserer Lösung als Gewicht für die Sinc-Interpolation
vorkommt. Sprich, wir benutzen die Koeffizienten der Fourierreihe
(periodische Rechteckfunktion im Zeitbereich) um damit
zur Fouriertransformierten (einmaliger Rechteckimpuls) zur interpolieren.
Wir sehen der Sinc-Interpolationsformel die perfekte Rekonstruktion wieder nicht an,
in der mit Computer errechneten Grafik \fig{fig:SpectrumSampling_Th_Ts1_2_45C76AFB33},
können wir das gut herauslesen, auch wenn dort nur über endliche Anzahl der
wichtigsten $\nu$ summiert wurde.

In grau sind die einzelnen $\mathrm{sinc}(\pi\cdot\frac{\omega-\nu 2 \pi}{2\pi})$
gewichtet mit ihrem zugehörigen Abtastwert dagestellt.
Diese Sinc-Funktionen haben jeweils ihr absolutes Maximum bei $\nu 2 \pi$ (durch
Verschiebung entlang der Frequenzachse). Durch ihre spezielle Frequenzskalierung
besitzen sie Nullstellen für alle anderen $\mu$. Deswegen
berücksichtigt $\mathrm{sinc}(\pi\cdot\frac{\omega-\nu 2 \pi}{2\pi})$
exakt nur den Abtastwert bei $\nu 2 \pi$. Das ist die Grundidee der Rekonstruktion
mit unendlich langer Sinc-Funktion. Die Überlagerung (das was wir eben schwer
in den Formeln sehen) aller Sinc-Funktionen liefert dann die kontinuierliche Funktion,
hier die das Spektrum der Fourier Transformation.
%
In \fig{fig:SpectrumSampling_Th_TsXX45C76AFB33} sind zwei weitere Beispiele
dargestellt, für die $T_h$ variiert wurde.
In \fig{fig:SpectrumSampling_Th_Ts1_1_45C76AFB33} sehen wir den Fall der
Unterabtastung des Spektrums, im Zeitbereich überlappen sich jetzt die
Rechteckimpulse. Es gelingt keine fehlerfreie Rekonstruktion, die Sinc-Interpolation
interpoliert zwar fleißig, aber aufgrund der zur Verfügung stehenden (zu wenig)
Abtastwerte eben nicht zum originalen Spektrum.
\end{Loesung}


\begin{figure}
\centering
\includegraphics[width=0.75\textwidth]{../sampling/SpectrumSampling_Th_Ts1_2_45C76AFB33.pdf}
\caption{Aufgabe \ref{sec:45C76AFB33}: Frequenzbereich.}
\label{fig:SpectrumSampling_Th_Ts1_2_45C76AFB33}
\end{figure}



\begin{figure*}[h]
\centering
\begin{subfigure}{0.75\textwidth}
\includegraphics[width=\textwidth]{../sampling/SpectrumSampling_Th_Ts3_4_45C76AFB33.pdf}
\caption{Rekonstruktion perfekt, weil $\frac{T_h}{2}<\frac{T_s}{2}$.}
\label{fig:SpectrumSampling_Th_Ts1_1_45C76AFB33}
\end{subfigure}

\begin{subfigure}{0.75\textwidth}
\includegraphics[width=\textwidth]{../sampling/SpectrumSampling_Th_Ts4_3_45C76AFB33.pdf}
\caption{Rekonstruktion nicht perfekt, Unterabtastung, weil $\frac{T_h}{2}>\frac{T_s}{2}$.}
\label{fig:SpectrumSampling_Th_Ts1_1_45C76AFB33}
\end{subfigure}
% \begin{subfigure}{0.49\textwidth}
% \includegraphics[width=\textwidth]{../sampling/SpectrumSampling_Th_Ts1_1_45C76AFB33.pdf}
% \caption{Kritische Abtastung, d.h. $\frac{T_h}{2}=\frac{T_s}{2}$ erzeugt ein
% Gleichanteil im Zeitbereich, wo nur ein Fourierkoeffizient $\mu=0$ zur Beschreibung
% nötig ist.}
% \label{fig:SpectrumSampling_Th_Ts1_1_45C76AFB33}
% \end{subfigure}
\caption{Spektrenabtastung- und rekonstruktion.}
\label{fig:SpectrumSampling_Th_TsXX45C76AFB33}
\end{figure*}

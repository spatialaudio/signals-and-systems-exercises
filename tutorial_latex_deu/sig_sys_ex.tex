\documentclass[11pt,a4paper,DIV=12]{scrartcl}
\usepackage[utf8]{inputenc}
\usepackage{fouriernc}
\usepackage[T1]{fontenc}
\usepackage[german]{babel}
\usepackage[hidelinks]{hyperref}
\usepackage{natbib}
\usepackage{url}
\usepackage{amsmath}
\usepackage{amsfonts}
\usepackage{amssymb}
\usepackage{trfsigns}
\usepackage{graphicx}
\usepackage{subcaption}
\usepackage{xcolor}
\usepackage{comment}
\usepackage{mdframed}
\usepackage{tikz}

\bibliographystyle{dinat}

\numberwithin{equation}{section}
\numberwithin{figure}{section}

\newcommand\fsd{\mathrm{d}} %Ableitungsoperator
\renewcommand{\vec}[1]{\mathbf{#1}} %Vektor
\newcommand{\eq}[1]{Glg. (\ref{#1})} %Zitat Gleichung
\newcommand{\fig}[1]{Abb. \ref{#1}} %Zitat Abbildung
\newcommand{\red}{\textcolor{red}}

\definecolor{C0}{HTML}{1f77b4}
\definecolor{C1}{HTML}{ff7f0e}
\definecolor{C2}{HTML}{2ca02c}
\definecolor{C3}{HTML}{d62728}
\definecolor{C7}{HTML}{7f7f7f}

\specialcomment{Ziel}{\begin{mdframed}[backgroundcolor=C2!10] \textit{Lernziel}\\\noindent}{\end{mdframed}\noindent}
\specialcomment{Werkzeug}{\begin{mdframed}[backgroundcolor=C7!10] \textit{Werkzeug}\\\noindent}{\end{mdframed}\noindent}
\specialcomment{Ansatz}{\begin{mdframed}[backgroundcolor=C3!10] \textit{Ansatz}\\\noindent}{\end{mdframed}\noindent}
\specialcomment{ExCalc}{\begin{mdframed}[backgroundcolor=C1!10]\noindent \textit{Ausführliche Rechnung}\\\noindent}{\end{mdframed}\noindent}
\specialcomment{Loesung}{\begin{mdframed}[backgroundcolor=C0!10] \textit{Lösung}\\\noindent}{\end{mdframed}\noindent}

%\excludecomment{Ziel}
%\excludecomment{Werkzeug}
%\excludecomment{ExCalc}

% \newpage
% \subsection{Überschrift}
% \label{sec:0123456789}
% \begin{Ziel}
% \end{Ziel}
% \textbf{Aufgabe} {\tiny 0123456789}: Berechnen Sie...
% \begin{Werkzeug}
% \end{Werkzeug}
% \begin{Ansatz}
% \end{Ansatz}
% \begin{ExCalc}
% \end{ExCalc}
% \begin{Loesung}
% \end{Loesung}


% C964DD7400
% 114F06AFAA
% A0F7C530F3
% 0B03A693AD
% 31AEFEF90B
% 4408E33353
% 8ADC89E8AB
% 3EE13FE7F4
% 479082ABB1
% 30B8653E1C
% 501B527770
% 81203BB953
% 2CE2BF3BF0
% 9DB89709BC
% 95FC09C101
% AE7A1E23F4
% 8638A34D09
% 11792AAE1A
% CC195E9826
% 4186EBC30B
% E2A3404BAE
% 4E088FAB0E
% 7D194BC8E5
% BD36CBB4D9
% 64E3F03858
% 0139F0A8F3
% 0A97EC5C0F
% F9F42E8F2D
% 5912031D86
% 14E35F9B88
% C3C06D25E1
% CA150937F8
% FBE36B0684
% AF3B15E0D3
% 1D3D68B312
% C97E6AA17A
% 42DE8C69F7
% 31C1621D73
% 25F7F29E2A

%------------------------------------------------------------------------------
\title{Signal- und Systemtheorie\\
Übung\thanks{
This tutorial is provided as Open Educational Resource (OER), to be found at
\url{https://github.com/spatialaudio/signals-and-systems-exercises}
accompanying the OER lecture
\url{https://github.com/spatialaudio/signals-and-systems-lecture}.
%
Both are licensed under a) the Creative Commons Attribution 4.0 International
License for text and graphics and b) the MIT License for source code.
%
Please attribute material from the tutorial as \textit{Frank Schultz,
Continuous- and Discrete-Time Signals and Systems - A Tutorial Featuring
Computational Examples, University of Rostock} with
\texttt{main file, github URL, commit number and/or version tag, year}.
}
\\
\small Universität Rostock Vst.-Nr. 24015}
%
\author{Dr. Frank Schultz, Prof. Sascha Spors\\
\small Institut für Nachrichtentechnik (INT)\\
\small Fakultät für Informatik und Elektrotechnik (IEF)\\
\small Universität Rostock
}
%
\date{Sommersemester 2020, Version: \today}
%------------------------------------------------------------------------------
\begin{document}
\maketitle
\tableofcontents
%------------------------------------------------------------------------------
\section*{Einleitung}
%
Aus aktuellem Anlass einer hoffentlich einmaligen COVID-19 Pandemie, entsteht
dieses ausführliche Skript mit Übungsaufgaben, um die Signal- und Systemtheorie
(SigSys) zu vertiefen.
%
Typischerweise ist die in SigSys benutzte Mathematik einfacher als so mancher Stoff
aus Mathematik Grundlagenvorlesungen.
%
Das was wir aber an Mathematik brauchen, muss sicher beherrscht werden.
%
Dies gelingt durch individuelle Übung.
%
Das Klären des \textbf{Was?, Warum? Wie?} und die vorangestellte \textbf{Frage des Wesens}
bei der Vermittlung und Vertiefung von Stoff liegt in der
Verantwortung der Dozierenden.
%
Dies zu beherzigen und dem streng zu folgen ist viel wichtiger als die neuesten
oder die tradiertesten technischen Errungenschaften für die Didaktik zu benutzen.
%
Es gibt unterirdisch schlechte, genauso wie brillante Tafelvorlesungen,
E-Learning Videos, Lehrbücher usw.
%
Am Ende des Tages ist \textbf{Verstehen} ein hoch individueller Erarbeitungs-
und Selbstreflexionsprozess, den jede(r) mit eigener Strategie bewältigt.
%

Ich hab mich nach langem Überlegen für ein klassisches Skript entschieden,
um den nötigen Stoff (komplett neu) aufzubereiten.
%
Jede Übung werde ich in einem kurzen Screencast-Video des vorliegenden Skripts
ankommentieren und teasen (ich hoffe ich komme zeitlich mit allem hinterher).
%
Unterstützt wird dieses Übungsskript durch kleine Python-Skripte in
Jupyter Notebooks (*.ipynb Dateien),
welche alle hier enthaltenen Diagramme erzeugen, sodass Sie
detailliert nachschauen können und mit diesen Beispielen eigene Versuche machen
können.
%
Wenn Ihnen irgendwo in unserem Material das \textbf{Was?, Warum? Wie?} und das
Lernziel fehlt oder ausbaufähig erscheint, geben Sie bitte Bescheid!

Im StudIP haben wir eine \textbf{Formelsammlung} bereitgestellt. Dort sind die
wichtigsten Zusammenhänge der Signal- und Systemtheorie kompakt auf vier DIN A4
Seiten zusammengetragen. Es ist sehr sinnvoll, sich diese Formeln zu erarbeiten
und zu verstehen. Diese Formelsammlung steht während der Klausur zur Verfügung.

Im StudIP, finden sich zudem \textbf{Altklausuren} der letzten Semester,
teilweise mit Lösungen, sowie die \textbf{Übungsblätter} des letzten
Sommersemesters 2019 für Fans den nichtlinearen Lernens.

Das LaTex-Projekt zu diesem vorliegenden Skript und alle Jupyter Notebooks
gibt es unter
\begin{mdframed}[backgroundcolor=C2!10]
\url{https://github.com/spatialaudio/signals-and-systems-exercises}
\end{mdframed}
%
Regelmäßiges Update des Skripts als fertiges PDF erfolgt im StudIP.
%
Unter
\begin{mdframed}[backgroundcolor=C2!10]
\url{https://github.com/spatialaudio/signals-and-systems-lecture}
\end{mdframed}
finden sich die Unterlagen der Vorlesung.
%
Jeder ist eingeladen bei diesen zwei Projekten Fehler zu finden, Verbesserungen
vorzuschlagen oder sogar selber vorzunehmen (mittels neuem git branch).
%
Zudem sollten wir die StudIP-Mittel \textbf{Wiki, Forum, Blubber} usw. zusammen
intensiv nutzen um das kommende Semester bestmöglich zu bestreiten.
%
Ich werde zur planmäßigen Übungszeit Mi 11:00 bis 13:00 online im StudIP sein
und für Blubber etc. zur Verfügung stehen.
%
Gerne jederzeit Fragen ins Forum (dann haben alle etwas davon).

\newpage
\subsection*{Aufbau Aufgabe}
\begin{Ziel}
Das Wesen vorher klären mit Was?, Warum?, Wie? ist essentiell, damit wir das
Lernziel erfassen können und Lernen nicht willkürliches Puzzleteilsammeln ist.
Aus dem Was?, Warum?, Wie? und der gestellten Aufgabe sollten wir immer eine
Erwartungshaltung entwickeln, bevor wir uns rein handwerklich der Lösung nähern.
\end{Ziel}
\textbf{Aufgabe:} Berechnen Sie..., Zeigen Sie..., Falls...warum..., Erläutern
Sie anhand...
\begin{Werkzeug}
Hier meist Formeln, Zusammenhänge, Modelle. Wir benutzen die Formelsammlung
wo es geht und lesen ideal andere Literatur zum Thema, um die nötigen Werkzeuge
aus anderen Blickwinkeln zu erfassen.
\end{Werkzeug}
\begin{Ansatz}
Problemstellung in eine Formel überführt. Als nächstes können wir rechnen, also
reines Handwerk betreiben.
\end{Ansatz}
\begin{ExCalc}
Ein möglicher Rechenweg, wir sollten mit eigenem Stil
rechnen.
Händisches Rechnen, im Vergleich zu reinem Lesen und computergestützten Lösungen,
steigert die kognitive Kompetenz und erleichtert dem Kopf den Zugang zum Thema.
\end{ExCalc}
\begin{Loesung}
Endergebnisse, Grafiken und ganz wichtig ist --- nicht der fragwürdige Antwortsatz
aus der Schulphysik --- sondern vielmehr eine Interpretation der Lösung
für das gestellte Problem. Was ist unser Erkenntnisgewinn? Was sehen wir in den
Formeln? Was lesen wir aus den Grafiken? Wurde unsere Erwartungshaltung erfüllt
oder nicht.

In der Klausur ist typischerweise so viel Stoff abzufragen und so wenig Zeit,
dass so etwas leicht auf der Strecke bleiben kann. Jedoch eine im Kopf
durchgeführte Interpretation ist ein hervorragender Plausibilitätscheck, ob das was
wir da gerechnet haben, sein kann. Und je mehr Aufgaben wir
vorher geübt haben (eigentlich je mehr wir vom Wesen verstanden haben und in
Echtzeit verwenden können; das wird aber gern mit viel Üben oder gar
Auswendiglernen gleichgesetzt),
desto leichter und schneller wird uns das fallen.
\end{Loesung}


\setcounter{section}{0}  % UE xx in our current sequence

%------------------------------------------------------------------------------
\newpage
\section{UE 1: Wiederholung Mathe Fourier-Reihe/-Transformation, Lösen von DGL}
Zielsetzung / Objectives: Bereits bestehende Kenntnisse aus der Mathematik
wiederholen und in den Kontext der SigSys stellen.
%Ausblick was/wie/warum und Verknüpfung mit bereits
%bekanntem MatheZeug, Begrifflichkeit Signal, System

\begin{itemize}
\item Komplexe Fourierreihe, Signalanalyse und -synthese
\item Fouriertransformation, Signalanalyse und -synthese
\item Dualität Sinc/Spalt-Funktion vs. Rect-Funktion
\item Dualität Verschiebung vs. Modulation
%\item Ausblick zwei weitere Transformationen für Signalanalyse und -synthese: DFT, DTFT
\item DGL lösen mittels Eigenfunktionen, charakteristisches Polynom vs. Eigenwerte/-vektoren einer Matrix, Systemanalyse
%\item Message: SigSys fundamentalisiert, sammelt und vereinfacht AlltagMath für typ. Ing Aufgaben bzgl. System/Signale
%Auch wenn SigSys-Hand-Rechnerei veraltet erscheint im vgl zu den 'neuen' trendy Topics, it's fundamental Need to know
\end{itemize}
%
Wir benutzen im Grunde noch kein Signal- und Systemtheorie (SigSys) spezifisches Wissen,
sondern benutzen das in Mathe erworbene Wissen zur Fourierreihe und -transformation
und Differentialgleichungen. Die vergleichsweise einfach gehaltenen
Übungen dienen aber dazu fundamental wichtige Bausteine der SigSys zu teasen.


\newpage
\subsection{Komplexe Fourier-Reihe Beispiel Rechteckschwingung}
\label{sec:D1483A84E2}
%
\begin{Ziel}
Wir wollen für die Fourierreihe den wichtigen Zusammenhang zwischen der Rechteckfunktion
und der Spaltfunktion anhand eines fundamentalen Beispiels erfassen. Zudem enthält die
ausführliche Rechnung Rechenschritte und Umformungen welche wir so oder ähnlicher Form
immer wieder brauchen werden.
\end{Ziel}
\textbf{Aufgabe} {\tiny D1483A84E2}: Berechnen Sie die komplexe Fourier-Reihe für
\begin{itemize}
\item axialsymmetrische periodische Rechteckschwingung $x(t)$ mit Periode $T$
und Amplitude $A$
\item Tastverhältnis $0<\frac{T_h}{T}\leq 1$
\item Grundkreisfrequenz $\omega_0 = \frac{2\pi}{T}$ mit $\omega_0>0$
\end{itemize}
%
\begin{figure}[h!]
\centering
\begin{tikzpicture}[domain=0:2]
\def\T{0.4}
\draw[->] (-3,0) -- (3,0) node[below right] {$t$};
\draw[->] (0,0) -- (0,1.5) node[above] {$x(t)$};
\foreach \pos in {-2,...,2} {
\draw[-, C0, ultra thick] (\pos-\T,0) -- (\pos-\T,1) -- (\pos+\T,1) -- (\pos+\T,0) -- (\pos+1-\T,0);
};
\draw[-, C0, ultra thick] (-\T,0) node[below] {$\frac{-T_h}{2}$};
\draw[-, C0, ultra thick] (+\T,0) node[below] {$\frac{+T_h}{2}$};
\draw[-, C0, ultra thick] (1,0) node[below] {$T$};
\draw[-, C0, ultra thick] (2,0) node[below] {$2 T$};
\draw[-, C0, ultra thick] (-2-\T,1) node[left] {$A$};
\end{tikzpicture}
\end{figure}
%
\begin{Werkzeug}
Synthese mit komplexer Fourierreihe
%
\begin{equation}
x(t) = \sum\limits_{k=-\infty}^{-\infty} c_k \cdot \e^{+\im \omega_0 k t}
\end{equation}
%
Analyse mit komplexer Fourierreihe
\begin{equation}
c_k = \frac{1}{T} \int\limits_{-T/2}^{+T/2} x(t) \cdot \e^{-\im \omega_0 k t} \mathrm{d}t
\end{equation}
%
Hinweis: Wir verwenden hinfort nur noch selten die Fourierreihe mit
reellen Koeffizienten und die Zerlegung in $\cos$- und $\sin$-Schwingungen.
%
Diese Darstellung mag zunächst zugänglicher erscheinen als der komplexe
Dreher $\e^{\pm\im \omega_0 k t}$, aber die Rechnerei ist am Ende deutlich
umfangreicher.
%
Aus gleichem Grund haben wir in der Elektrotechnik die komplexe
Wechselstromrechnung eingeführt.
%
Zudem gestaltet sich das Aufzeigen der Verknüpfungen zu den anderen
Fourier-Transformationen mit der komplexen Variante deutlich eleganter.
\end{Werkzeug}

\begin{Ansatz}
%
Wir müssen innerhalb einer Periode nur die Teile berücksichtigen, wo
die Funktion nicht Null ist. Am einfachsten ist das, wenn wir den
Zeitraum $-T/2$ bis $+T/2$ berücksichtigen und daher nur von $-T_h/2$ bis $+T_h/2$
integrieren.
\begin{equation}
c_k = \frac{1}{T} \int\limits_{-T_h/2}^{+T_h/2} A \e^{-\im \omega_0 k t} \mathrm{d}t
\end{equation}
%
\end{Ansatz}
%
\begin{ExCalc}
Integrieren
\begin{equation}
c_k = \frac{A}{T} \frac{\e^{-\im \omega_0 k t}}{-\im \omega_0 k }\bigg|_{-T_h/2}^{+T_h/2}
\end{equation}
Grenzen einsetzen
\begin{equation}
c_k = \frac{A}{T} \frac{\e^{-\im \omega_0 k T_h/2} - \e^{+\im \omega_0 k T_h/2}}{-\im \omega_0 k }
\end{equation}
Vorzeichen
\begin{equation}
c_k = \frac{A}{T} \frac{\e^{+\im \omega_0 k T_h/2}-\e^{-\im \omega_0 k T_h/2}}{\im \omega_0 k }
\end{equation}
Erweitern mit dem Ziel $\sin(x) = \frac{\e^{+\im x}-\e^{-\im x}}{2\im}$, $\mathrm{sinc}(x):=\frac{\sin(x)}{x}$
\begin{equation}
c_k = \frac{A}{T} \frac{\e^{+\im \omega_0 k T_h/2}-\e^{-\im \omega_0 k T_h/2}}{\im \omega_0 k } \cdot \frac{T_h/2}{T_h/2} \cdot \frac{2}{2}
\end{equation}
Umsortieren, Kürzen
\begin{equation}
c_k = \frac{A}{T} \frac{\e^{+\im \omega_0 k T_h/2}-\e^{-\im \omega_0 k T_h/2}}{2\im \omega_0 k T_h/2} \cdot \frac{T_h/2}{1} \cdot \frac{2}{1}
\end{equation}
%
\begin{equation}
c_k = \frac{A}{T} \frac{\e^{+\im \omega_0 k T_h/2}-\e^{-\im \omega_0 k T_h/2}}{2\im \omega_0 k T_h/2} \cdot T_h
\end{equation}
%
\begin{equation}
c_k = A \frac{T_h}{T} \cdot \mathrm{sinc}(\omega_0 k \frac{T_h}{2})
\end{equation}
\end{ExCalc}

\begin{Loesung}
%
\begin{equation}
\label{eq:D1483A84E2_Loesung}
T \cdot c_k = A T_h \cdot \mathrm{sinc}(\omega_0 k \frac{T_h}{2}) = A T_h \cdot \mathrm{sinc}(k \pi \frac{T_h}{T})
\end{equation}
%
Wir sollten uns mit der Spaltfunktion
\begin{equation}
\mathrm{sinc}(x) := \frac{\sin(x)}{x}
\end{equation}
vertraut machen, z.B.
hier \url{https://mathworld.wolfram.com/SincFunction.html} oder in einschlägigen
Mathe- und Signalverarbeitungsbüchern. Wichtig ist, die Umhüllende, die Nullstellen
und den Verlauf der lokalen Maxima / Minima grob auf dem Schirm zu haben.
%
Besonders relevant für die aktuelle Aufgabe ist, dass die die Spaltfunktion
breiter und schmaler bzgl. des gleichen Ausschnitts
$-K \leq k \leq +K$ gemacht wird.

Zunächst halten wir fest: a) eine periodische Rechteckfunktion hat eine sinc-artige,
komplexe Fourierreihe, b) die Fourierkoeffizienten sind reell, das ist
hier ein Spezialfall der komplexen Fourierreihe c) die Spalt-Funktion wird
wegen diskretem $k$ nur an bestimmten Stellen ausgewertet.
Die Fourierreihe ist also als Folge
über alle $k$ der komplexen Fourierkoeffizienten $c_k$ aufzufassen. Wir werden
das später Linienspektrum bezeichnen.

Grafische Darstellungen helfen uns das Verhalten der komplexen Fourierreihe und
Fourier-Synthese besser zu verstehen.
%
Dafür wurde für die \fig{fig:D1483A84E2_0} bis \fig{fig:D1483A84E2_4}
jeweils die sinc-förmige Fourierreihe (links) und das rechteck-förmig Signal
(rechts) dargestellt, das Tastverhältnis $T_h$ wurde
variiert, die Amplitude zu $A = 1/T_h$ gewählt, sowie $T=1$ s konstant gehalten.
%
Die Fouriersynthese erfolgt mit $-40 \leq k \leq +40$.
%
Wir erkennen, dass die Breite der Spalt-Funktion für steigendes Tastverhältnis
$\frac{T_h}{T}$ abnimmt.
%
Im Extremfall $\frac{T_h}{T} = 1$ resultiert ein Gleichsignal, und daher
nur $c_0 \neq 0$, alle anderen $c_k=0$.
%
Spannend ist zudem der Fall $\frac{T_h}{T} = 1/2$: hier ist jeder zweite $c_k=0$,
weil wir da eine Nullstelle der Spalt-Funktion auswerten.

Sehr vereinfacht, aber vom Wesen wichtig:
Je mehr Gleichsignal, desto weniger Fourierkoeffizienten werden benötigt um das
Signal zu approximieren. Je mehr impulshafte Änderung
$\frac{T_h}{T} \to  0$, desto mehr Fourierkoeffizienten werden benötigt, also
desto breiter das sinc-förmige Linienspektrum.
%


\end{Loesung}


\begin{figure}
\includegraphics[width=\textwidth]{../fs/D1483A84E2_0.pdf}
\caption{Koeffizienten der komplexen Fourierreihe $c_k$ (links) und Synthese $x(t)$ (rechts) für
$T=1$ s, $T_h=0.1$ s, $A=1/T_h$. \texttt{FourierSeries\_D1483A84E2.ipynb}}
\label{fig:D1483A84E2_0}
\end{figure}

\begin{figure}
\includegraphics[width=\textwidth]{../fs/D1483A84E2_1.pdf}
\caption{Koeffizienten der komplexen Fourierreihe $c_k$ (links) und Synthese $x(t)$ (rechts) für
$T=1$ s, $T_h=0.2$ s, $A=1/T_h$. \texttt{FourierSeries\_D1483A84E2.ipynb}}
\label{fig:D1483A84E2_1}
\end{figure}

\begin{figure}
\includegraphics[width=\textwidth]{../fs/D1483A84E2_2.pdf}
\caption{Koeffizienten der komplexen Fourierreihe $c_k$ (links) und Synthese $x(t)$ (rechts) für
$T=1$ s, $T_h=0.5$ s, $A=1/T_h$. \texttt{FourierSeries\_D1483A84E2.ipynb}}
\label{fig:D1483A84E2_2}
\end{figure}

\begin{figure}
\includegraphics[width=\textwidth]{../fs/D1483A84E2_3.pdf}
\caption{Koeffizienten der komplexen Fourierreihe $c_k$ (links) und Synthese $x(t)$ (rechts) für
$T=1$ s, $T_h=0.8$ s, $A=1/T_h$. \texttt{FourierSeries\_D1483A84E2.ipynb}}
\label{fig:D1483A84E2_3}
\end{figure}

\begin{figure}
\includegraphics[width=\textwidth]{../fs/D1483A84E2_4.pdf}
\caption{Koeffizienten der komplexen Fourierreihe $c_k$ (links) und Synthese $x(t)$ (rechts) für
$T=1$ s, $T_h=1$ s, $A=1/T_h$. \texttt{FourierSeries\_D1483A84E2.ipynb}}
\label{fig:D1483A84E2_4}
\end{figure}




\newpage
\subsection{Fourier-Transformation Beispiel Rechteckfunktion}
\label{sec:8C3958BE4F}
\begin{Ziel}
Wir wollen für die Fouriertransformation die wichtige Dualität zwischen der
Rechteckfunktion und der Spaltfunktion anhand eines fundamentalen Beispiels erfassen.
Zudem enthält auch hier die ausführliche Rechnung Rechenschritte und Umformungen
welche wir so oder ähnlicher Form immer wieder brauchen werden.
\end{Ziel}
\textbf{Aufgabe} {\tiny 8C3958BE4F}: Berechnen Sie die Fourier-Transformation für
\begin{itemize}
\item axialsymmetrische Rechteckfunktion der Breite $T_h$ und Amplitude $A$
\end{itemize}

\begin{figure}[h!]
\centering
\begin{tikzpicture}[domain=0:2]
\draw[->] (-2,0) -- (2,0) node[below right] {$t$};
\draw[->] (0,0) -- (0,1.5) node[above] {$x(t)$};
\draw[-, C0, ultra thick] (-2,0) -- (-0.5,0) node[below] {$\frac{-T_h}{2}$} -- (-0.5,1) node[left] {$A$} -- (0.5,1) -- (0.5,0) node[below] {$\frac{+T_h}{2}$} -- (2,0);
\end{tikzpicture}
\end{figure}

\begin{Werkzeug}
Synthese mit Fouriertransformation
\begin{align}
x(t) = \frac{1}{2\pi} \int\limits_{-\infty}^{+\infty} X(\im\omega) \, \e^{+\im \omega t} \fsd \omega
\end{align}
%
Analyse mit Fouriertransformation
\begin{align}
X(\im \omega) = \int\limits_{-\infty}^{+\infty} x(t) \, \e^{-\im \omega t} \fsd t
\end{align}
\end{Werkzeug}

\begin{Ansatz}
\begin{equation}
X(\im \omega) = \int\limits_{-T_h/2}^{+T_h/2} A \e^{-\im \omega t} \mathrm{d}t
\end{equation}
\end{Ansatz}

\begin{ExCalc}
Integrieren
\begin{equation}
X(\im \omega) = A \frac{\e^{-\im \omega t}}{-\im \omega} \bigg|_{-T_h/2}^{+T_h/2}
\end{equation}
Grenzen einsetzen
\begin{equation}
X(\im \omega) = A \frac{\e^{-\im \omega T_h/2}-\e^{+\im \omega T_h/2}}{-\im \omega}
\end{equation}
Vorzeichen
\begin{equation}
X(\im \omega) = A \frac{\e^{+\im \omega T_h/2}-\e^{-\im \omega T_h/2}}{\im \omega}
\end{equation}
Erweitern mit dem Ziel $\sin(x) = \frac{\e^{+\im x}-\e^{-\im x}}{2\im}$, $\mathrm{sinc}(x):=\frac{\sin(x)}{x}$
\begin{equation}
X(\im \omega) = A \frac{\e^{+\im \omega T_h/2}-\e^{-\im \omega T_h/2}}{\im \omega} \frac{T_h/2}{T_h/2} \cdot \frac{2}{2}
\end{equation}
Umsortieren
\begin{equation}
X(\im \omega) = A \frac{\e^{+\im \omega T_h/2}-\e^{-\im \omega T_h/2}}{2 \im \omega T_h/2} \frac{T_h/2}{1} \cdot \frac{2}{1}
\end{equation}
\end{ExCalc}

\begin{Loesung}
\begin{equation}
\label{eq:8C3958BE4F_Loesung}
X(\im \omega) = A T_h \cdot \mathrm{sinc}(\omega \frac{T_h}{2})
\end{equation}

Wir bekommen wieder eine Spaltfunktion, wie schon bei der Fourierreihe aus Aufgabe
\ref{sec:D1483A84E2}, vgl. \eq{eq:D1483A84E2_Loesung}.
%
Diesmal, weil $\omega$ kontinuierlich wird die Spaltfunktion überall ausgewertet,
die Fouriertransformation ist also eine kontinuierliche Funktion über $\omega$.
%
Auch hier ist $X(\im\omega)\in\mathbb{R}$ ein Spezialfall.
%
Der Faktor $\frac{T_h}{2}$ im sinc-Argument bestimmt wieder die Breite der
Spaltfunktion.

Je \textbf{kleiner} $T_h$, also je \textbf{schmal}er der
\textbf{Rechteck}impuls, desto \textbf{breit}er die \textbf{Spalt}funktion.

Je \textbf{größer} $T_h$, also je \textbf{breit}er der
\textbf{Rechteck}impuls, desto \textbf{schmal}er die \textbf{Spalt}funktion.
%
Dieser Zusammenhang ist fundamental in der Signalverarbeitung und der
Kommunikationstechnik und ist in \fig{fig:8C3958BE4F} grafisch aufbereitet.

Was erwarten für $T_h\to 0$ (idealer Impuls) und was für $T_h\to \infty$
(Gleichspannung)? In beiden Fällen bekommen
wir Probleme beim Integrieren. Aber dies sind in der Signaltheorie überaus wichtige
Grenzfälle, für die auch Fouriertransformierte existieren. Wir werden sie bald
kennenlernen.

Wir werden in der VL und UE sehr oft das Elementarsignal

\begin{equation}
\text{rect}(t) := \begin{cases} 1 & |t| < \frac{1}{2} \\ \frac{1}{2} & |t| = \frac{1}{2} \\ 0 & |t| > \frac{1}{2} \end{cases}\quad,
\end{equation}
also den Rechteckimpuls mit $T_h=1$ und $A=1$, zum Rechnen benutzen.
%
Die Fouriertransformierte von $\text{rect}(t)$ lautet dann abgeleitet aus
\eq{eq:8C3958BE4F_Loesung} einfach
$X(\im \omega) = \mathrm{sinc}(\frac{\omega}{2})$.
%
Eine oft benutzte Schreibweise von Fouriertransformationspaaren (sogenannte
Korrespondenzen, siehe Formelsammlung) benutzt den Operator $\laplace$.
%
Wir schreiben für das soeben gefundene Transformationspaar
\begin{equation}
x(t) = \text{rect}(t) \quad \laplace \quad X(\im \omega) = \mathrm{sinc}(\frac{\omega}{2})
\end{equation}
%
Eine weitere Korrespondenz ist
\begin{equation}
x(t) = \text{sinc}(t) \quad \laplace \quad X(\im \omega) = \pi \, \mathrm{rect}(\frac{\omega}{2}),
\end{equation}
also bis auf den Faktor $\pi$ vertauscht.
%Diese Korrespondenz direkt zu
%beweisen, also die Spaltfunktion fourierzutransformieren ist nicht trivial, aber
%man kann die inverse Fourier Transformation berechnen. Das ist wieder nur der
%Rechteckimpuls unter dem Integralkern.
%
Die Dualität \textbf{Spaltfunktion transformiert ergibt Rechteckfunktion und umgekehrt},
dürfen wir die ganze Zeit nicht aus den Augen verlieren.

Schauen wir noch kurz zurück in die \fig{fig:D1483A84E2_0} bis \fig{fig:D1483A84E2_4}.
Dort wurde in grau immer der Verlauf der hier diskutierten Fouriertransformierten
hinterlegt. Man sieht, dass die Werte der Fourierkoeffizienten $T \cdot c_k$
\eq{eq:D1483A84E2_Loesung} aus Aufgabe \ref{sec:D1483A84E2}
mit den Werten von $X(\im\omega)$ \eq{eq:8C3958BE4F_Loesung} übereinstimmen bei
$\omega_0 k = \omega$. Dies ist kein Zufall und werden wir als Abtastung
kennenlernen.



\end{Loesung}

\begin{figure*}[h!]
\centering
\begin{subfigure}{1\textwidth}
\centering
\includegraphics[width=\textwidth]{../ft/8C3958BE4F_1.pdf}
\caption{Rechteckimpulse für verschiedene $T_h$ und gewähltem $A=1/T_h$.
Um die resultierenden Amplitudenunterschiede sinnvoll darstellen zu können,
ist $\log_{10}x(t)$ über $t$ aufgetragen. Da $\log_{10}(0)=-\infty$
ist nur der Signalteil mit Amplitude $A=1/T_h$ sichtbar.}
\label{fig:8C3958BE4F_1}
\end{subfigure}
\\
\begin{subfigure}{1\textwidth}
\centering
\includegraphics[width=\textwidth]{../ft/8C3958BE4F_0.pdf}
\caption{Fouriertransformation der Rechteck-Impulse aus
\fig{fig:8C3958BE4F_1}. Durch die Wahl $A=1/T_h$ haben alle Rechteckimpulse Fläche
1 und alle Fouriertransformierten
das gleiche globale Maximum (bei $\omega=0$ rad/s) mit Wert 1.}
\label{fig:8C3958BE4F_0}
\end{subfigure}
%
\caption{Rechteckimpulse (oben) und Fouriertransformierte (unten).
Liniendickenvariation zu besseren Unterscheidbarkeit bei Graustufenanzeige.
\texttt{FourierTransformation\_8C3958BE4F.ipynb}}
\label{fig:8C3958BE4F}
\end{figure*}


\begin{figure}[h!]
\includegraphics[width=\textwidth]{../ft/8C3958BE4F_SingleCase_MagPhase.pdf}
  \caption{Betrag (oben) und Phase (unten) der rellwertigen
  Fouriertransformierten \eq{eq:8C3958BE4F_Loesung}. Für eine Variable
  $a\in\mathbb{R}$ gilt $a = -a \e^{\pm\im\pi}$, daher
  der Phasensprung um $\pi$ bei negativer Amplitude der Spaltfunktion.
\texttt{FourierTransformation\_8C3958BE4F.ipynb}}
  \label{fig:8C3958BE4F_SingleCase_MagPhase}
\end{figure}




\clearpage
\subsection{Fourier-Transformation Signal-Verschiebung}
\label{sec:A8A2DEE53A}
\begin{Ziel}
Wir wollen uns erarbeiten, wie sich eine Zeitverschiebung auf die
Fouriertransformierte auswirkt und damit das Verschiebungsgesetz kennenlernen.
Zeitverschiebung bedeutet immer Phasenänderung in der Fouriertransformierten.
\end{Ziel}
\textbf{Aufgabe} {\tiny A8A2DEE53A}: Berechnen Sie die Fourier-Transformation für
\begin{itemize}
\item dargestellte Rechteckfunktion der Breite $T_h$ und Amplitude $A$
\end{itemize}
%
\begin{figure}[h!]
\centering
\begin{tikzpicture}[domain=0:2]
\draw[->] (-2,0) -- (2,0) node[below right] {$t$};
\draw[->] (0,0) -- (0,1.5) node[above] {$x(t)$};
\draw[-, C0, ultra thick] (-2,0) -- (-0.0,0) node[below] {$0$} -- (-0,1) node[left] {$A$} -- (1,1) -- (1,0) node[below] {$T_h$} -- (2,0);
\end{tikzpicture}
\end{figure}
%
Wir erkennen die in Aufgabe \ref{sec:8C3958BE4F} hier um $T_h/2$, also nach rechts,
verschobene Rechteckfunktion.


\begin{Werkzeug}
nothing new here...
Analyse mit Fouriertransformation
\begin{align}
X(\im \omega) = \int\limits_{-\infty}^{+\infty} x(t) \, \e^{-\im \omega t} \fsd t
\end{align}
\end{Werkzeug}
\begin{Ansatz}
\begin{equation}
X(\im \omega) = \int\limits_{0}^{T_h} A \e^{-\im \omega t} \mathrm{d}t
\end{equation}
\end{Ansatz}


\begin{ExCalc}
Integrieren
\begin{equation}
X(\im \omega) = A \frac{\e^{-\im \omega t}}{-\im \omega} \bigg|_{0}^{T_h}
\end{equation}
Grenzen einsetzen
\begin{equation}
X(\im \omega) = A \frac{\e^{-\im \omega T_h}-1}{-\im \omega}
\end{equation}
Vorzeichen
\begin{equation}
X(\im \omega) = A \frac{1 - \e^{-\im \omega T_h}}{\im \omega}
\end{equation}
Erweitern mit dem Ziel $\sin(x) = \frac{\e^{+\im x}-\e^{-\im x}}{2\im}$, $\mathrm{sinc}(x):=\frac{\sin(x)}{x}$
\begin{equation}
X(\im \omega) = A \frac{1 - \e^{-\im \omega T_h}}{\im \omega} \underbrace{\e^{+\im\omega\frac{T_h}{2}} \e^{-\im\omega\frac{T_h}{2}}}_{=1}=
A \frac{\e^{+\im\omega\frac{T_h}{2}} - \e^{-\im\omega\frac{T_h}{2}}}{\im\omega} \e^{-\im\omega\frac{T_h}{2}}
\end{equation}
\begin{equation}
X(\im \omega) = A \frac{\e^{+\im\omega\frac{T_h}{2}} - \e^{-\im\omega\frac{T_h}{2}}}{\im\omega} \e^{-\im\omega\frac{T_h}{2}}
\cdot \frac{T_h/2}{T_h/2} \cdot \frac{2}{2}
\end{equation}
Umsortieren
\begin{equation}
X(\im \omega) = A T_h \frac{\e^{+\im\omega\frac{T_h}{2}} - \e^{-\im\omega\frac{T_h}{2}}}{2 \im \cdot \omega T_h/2} \e^{-\im\omega\frac{T_h}{2}} =
A T_h \frac{\sin(\omega\frac{T_h}{2})}{\omega\frac{T_h}{2}} \e^{-\im\omega\frac{T_h}{2}}
\end{equation}
\end{ExCalc}


\begin{Loesung}
\begin{equation}
\label{eq:A8A2DEE53A_Loesung}
X(\im \omega) = A T_h \cdot \mathrm{sinc}(\omega\frac{T_h}{2}) \cdot  \e^{-\im\omega\frac{T_h}{2}}
\end{equation}
%
Wir erkennen, dass \eq{eq:8C3958BE4F_Loesung} um den komplexen Dreher / Zeiger $\e^{-\im\omega\frac{T_h}{2}}$
erweitert wird und nun $X(\im \omega)\in\mathbb{C}$.
%
Wir müssen bei grafischen Darstellungen daher streng immer zwei Diagramme anfertigen:
\begin{itemize}
  \item Realteil + Imaginärteil oder
  \item Betrag + Phase
\end{itemize}
%
Eine Darstellung in Betrag und Phase ist üblich, weil dies bei praktischen
Problemen, schnell hilfreiche Informationen offenbart. Das haben wir auch schon
bei der Wechselstromrechnung benutzt.


Im Gegensatz zum vorherigen Beispiel und \fig{fig:8C3958BE4F_SingleCase_MagPhase},
wo $X(\im \omega)\in\mathbb{R}<0$ Phasensprünge um $\pi$ verursacht haben, müssen
wir hier eine stückweise stetige, linear Phasenfunktion $\phi(\omega)$ berücksichtigen.
Schauen wir schonmal auf \fig{fig:A8A2DEE53A} unten, um uns ein Bild zu machen.
%
Der Anstieg der Geradenteile lässt sich direkt aus dem komplexen Dreher rauslesen
zu $-\frac{T_h}{2}$.
%
Für die Fouriertransformierte in Betrag und Phase dargestellt, gilt mit $k\in\mathbb{Z}$
\begin{equation}
X(\im \omega) = \bigg|A T_h \cdot \mathrm{sinc}(\omega\frac{T_h}{2})\bigg|
\cdot  \e^{-\im\omega\frac{T_h}{2}}\cdot
\begin{cases}
\e^{\im (0+2\pi k)}\quad \text{für}\quad \mathrm{sinc}(\omega\frac{T_h}{2})\geq0\\
\e^{\im (\pi + 2\pi k)}\quad \text{für}\quad\mathrm{sinc}(\omega\frac{T_h}{2})<0.
\end{cases}
\end{equation}
Überall da wo die Spaltfunktion einen Polaritärswechsel macht, entsteht
eine Phasen-Unstetigkeitstelle, weil die Phase springt um $\pi$ bzw. wegen der
$2\pi$-Vieldeutigkeit um $\pi + 2\pi k$.
%
Die Phasenfunktion ist daher
\begin{equation}
\phi(\omega) = -\omega\frac{T_h}{2} +
\begin{cases}
2\pi k\quad \text{für}\quad \mathrm{sinc}(\omega\frac{T_h}{2})\geq0\\
\pi + 2\pi k\quad \text{für}\quad\mathrm{sinc}(\omega\frac{T_h}{2})<0,
\end{cases}
\end{equation}
Die numpy-Funktion \verb|angle()| wertet die Phase zwischen $-\pi$ und $+\pi$ aus,
deswegen bekommen wir den in \fig{fig:A8A2DEE53A} dargestellten Verlauf der Phase.

Daher nochmal: Jenseits der Phasensprünge, ist zu realisieren, dass
eine zeitliche Verschiebung um $\frac{T_h}{2}$ in der Fouriertransformierten
zu einer Phasenänderung mit Anstieg $-\frac{T_h}{2}$ (negativ, fallend)
führt und der Betrag erhalten bleibt.
%
In \fig{fig:A8A2DEE53A} ist daher das Verhalten von \eq{eq:A8A2DEE53A_Loesung} als
Betrag und Phase über $\omega$ dargestellt.

Der Betrag der Spaltfunktion ist einfach zu erhalten, wir klappen einfach alle
negativen Funktionswerte in Richtung positiver Ordinate.
%
Die Phase ist die stückweise stetig Gerade über $\omega$ durch den Ursprung
mit Anstieg $-T_h/2$, also fallend.
%
Diese \textbf{Phase}nfunktion ist bzgl. des Ursprungs \textbf{punktsymmetrisch}.
Dies gilt immer für die Fouriertransformierten $X(\im\omega)$ von \textbf{reellwertigen}
Signalen $x(t)$.
%
Größere Zeitverschiebung $T_h/2$ bedeutet größere Phasenverschiebung, also steilere
fallende Geradenstücke.

Das was wir anhand eines speziellen (wichtigen) Beispiels gezeigt haben, gilt auch
allgemein. Die Korrespondenz zur \textbf{Zeitverschiebung} um $\tau\in\mathbb{R}$
---bekannt als \textbf{Verschiebungssatz}---lautet
\begin{align}
x(t) & \quad \laplace \quad X(\im\omega)\\
x(t - \tau) & \quad \laplace \quad X(\im\omega) \cdot \e^{-\im\omega \tau}.
\end{align}
%
Man kann natürlich auch eine Fouriertransformierte um die Kreisfrequenz
$\omega_0\in\mathbb{R}$ verschieben, vgl. Aufgabe \ref{sec:9D652BE72B}.
%
Die Korrespondenz zur \textbf{Frequenzverschiebung}
---bekannt als \textbf{Modulationssatz}---lautet
\begin{align}
x(t) & \quad \laplace \quad X(\im\omega)\\
\label{eq:A8A2DEE53A_ModTheoremTime}
x(t) \cdot \e^{+\im\omega_0 t} & \quad \laplace \quad X(\im[\omega-\omega_0]).
\end{align}
%
Bemerkenswert ist der gekreuzte Zusammenhang Zeit-/ Frequenzverschiebung
vs.
Frequenz-/ Zeitmodulation,
jedoch mit anderem Vorzeichen
in im komplexen Dreher.
\end{Loesung}

\begin{figure}[h!]
\includegraphics[width=0.9\textwidth]{../ft/A8A2DEE53A.pdf}
  \caption{Betrag (oben) und Phase (unten) der Fouriertransformierten \eq{eq:A8A2DEE53A_Loesung}.
\textbf{Vorsicht:} Die Grafik suggeriert, dass eine zeitliche Rechts-Verschiebung
sowohl Phase als auch Betrag ändert. Dies passiert aber hier speziell, weil wir in der
Aufgabe nach einer Verschiebung um $T_h/2$ gefragt haben, und $T_h$
im Argument der Spaltfunktion die Breite dieser bestimmt.
\texttt{FourierTransformation\_A8A2DEE53A.ipynb}}
  \label{fig:A8A2DEE53A}
\end{figure}


\clearpage
\subsection{Fourier-Transformation Signal-Zeitumkehr}
\label{sec:1CFE5FE3A1}
\begin{Ziel}
Wir wollen uns erarbeiten, wie sich eine Zeitumkehr in der Fouriertransformierten
auswirkt und dabei eine wichtige Symmetrie der Korrespondenzen kennenlernen.
Zeitumkehr ist Phasenumkehr in der Fouriertransformierten.
\end{Ziel}

\textbf{Aufgabe} {\tiny 1CFE5FE3A1}: Berechnen Sie die Fourier-Transformation für
\begin{itemize}
\item dargestellte Rechteckfunktion der Breite $T_h$ und Amplitude $A$
\end{itemize}
%
\begin{figure}[h!]
\centering
\begin{tikzpicture}[domain=0:2]
\draw[->] (-2,0) -- (2,0) node[below right] {$t$};
\draw[->] (0,0) -- (0,1.5) node[above] {$x(t)$};
\draw[-, C0, ultra thick] (-2,0) -- (-1,0) node[below] {$-T_h$} -- (-1,1) node[left] {$A$} -- (0,1) -- (0,0) -- (2,0);
\end{tikzpicture}
\end{figure}
%
Wir erkennen
\begin{itemize}
\item die in Aufgabe \ref{sec:8C3958BE4F} hier um $-T_h/2$, also nach links,
verschobene Rechteckfunktion
\item die in Aufgabe \ref{sec:A8A2DEE53A} hier zeitgedrehte Rechteckfunktion
\end{itemize}

\begin{Werkzeug}
Korrespondenz Zeitverschiebung um $\tau\in\mathbb{R}$:
\begin{align}
x(t) & \quad \laplace \quad X(\im\omega)\\
x(t - \tau) & \quad \laplace \quad X(\im\omega) \cdot \e^{-\im\omega \tau}.
\end{align}
Analyse mit Fouriertransformation:
\begin{align}
X(\im \omega) = \int\limits_{-\infty}^{+\infty} x(t) \, \e^{-\im \omega t} \fsd t
\end{align}
\end{Werkzeug}

\begin{Ansatz}
Mit Kenntnis der Fouriertransformierten aus Aufgabe \ref{sec:8C3958BE4F}, d.h. der
unverschobenen Zeitfunktion, können wir den Zeitverschiebungssatz anwenden:

\begin{equation}
x(t - (-T_h/2)) \quad \laplace \quad X(\im\omega) \cdot \e^{-\im\omega (-T_h/2)}
\end{equation}
Das Endergebnis für die hier gestellte Aufgabe lautet daher
\begin{equation}
X(\im \omega) = A T_h \cdot \mathrm{sinc}(\omega\frac{T_h}{2}) \cdot  \e^{+\im\omega\frac{T_h}{2}}.
\end{equation}
Wir können aber auch wieder zu Fuß das Integral der Fourier Transformation
\begin{equation}
X(\im \omega) = \int\limits_{-T_h}^{0} A \e^{-\im \omega t} \mathrm{d}t
\end{equation}
mit ähnlichem Vorgehen wie Aufgabe \ref{sec:A8A2DEE53A} lösen und müssen das
gleiche Endergebnis bekommen, wie die folgende ausführliche Rechnung zeigt.
\end{Ansatz}
\begin{ExCalc}
%Integrieren
\begin{equation}
X(\im \omega) = A \frac{\e^{-\im \omega t}}{-\im \omega} \bigg|_{-T_h}^{0}
\end{equation}
%Grenzen einsetzen
\begin{equation}
X(\im \omega) = A \frac{1 - \e^{+\im \omega T_h}}{-\im \omega}
\end{equation}
%Vorzeichen
\begin{equation}
X(\im \omega) = A \frac{\e^{+\im \omega T_h}-1}{\im \omega}
\end{equation}
%Erweitern mit dem Ziel $\sin(x) = \frac{\e^{+\im x}-\e^{-\im x}}{2\im}$, $\mathrm{sinc}(x):=\frac{\sin(x)}{x}$
\begin{equation}
X(\im \omega) = A \frac{\e^{+\im \omega T_h}-1}{\im \omega} \underbrace{\e^{-\im\omega\frac{T_h}{2}} \e^{+\im\omega\frac{T_h}{2}}}_{=1}=
A \frac{\e^{+\im\omega\frac{T_h}{2}} - \e^{-\im\omega\frac{T_h}{2}}}{\im\omega} \e^{+\im\omega\frac{T_h}{2}}
\end{equation}
\begin{equation}
X(\im \omega) = A \frac{\e^{+\im\omega\frac{T_h}{2}} - \e^{-\im\omega\frac{T_h}{2}}}{\im\omega} \e^{+\im\omega\frac{T_h}{2}}
\cdot \frac{T_h/2}{T_h/2} \cdot \frac{2}{2}
\end{equation}
%Umsortieren
\begin{equation}
X(\im \omega) = A T_h \frac{\e^{+\im\omega\frac{T_h}{2}} - \e^{-\im\omega\frac{T_h}{2}}}{2 \im \cdot \omega T_h/2} \e^{+\im\omega\frac{T_h}{2}} =
A T_h \frac{\sin(\omega\frac{T_h}{2})}{\omega\frac{T_h}{2}} \e^{+\im\omega\frac{T_h}{2}}
\end{equation}
\end{ExCalc}
\begin{Loesung}
Das Ergebnis
\begin{equation}
\label{eq:1CFE5FE3A1_Loesung}
X(\im \omega) = A T_h \cdot \mathrm{sinc}(\omega\frac{T_h}{2}) \cdot  \e^{+\im\omega\frac{T_h}{2}}.
\end{equation}
ist in \fig{fig:1CFE5FE3A1} dargestellt. Der Betrag
$\bigg|A T_h \cdot \mathrm{sinc}(\omega\frac{T_h}{2})\bigg|$
ist gleich wie in Aufgabe \ref{sec:A8A2DEE53A}, weil wir gelernt haben, dass
eine Zeitverschiebung nur die Phase in der Fouriertransformierten ändert.
%

Der wichtige neue Punkt ist, dass die \textbf{unstetige Phasenfunktion} $\phi(\omega)$
\textbf{diesmal ansteigend} ist,  da
\begin{equation}
X(\im \omega) = \bigg|A T_h \cdot \mathrm{sinc}(\omega\frac{T_h}{2})\bigg|
\cdot  \e^{+\im\omega\frac{T_h}{2}}\cdot
\begin{cases}
\e^{\im (0+2\pi k)}\quad \text{für}\quad \mathrm{sinc}(\omega\frac{T_h}{2})\geq0\\
\e^{\im (\pi + 2\pi k)}\quad \text{für}\quad\mathrm{sinc}(\omega\frac{T_h}{2})<0.
\end{cases}
\end{equation}

\begin{equation}
\phi(\omega) = +\omega\frac{T_h}{2} +
\begin{cases}
2\pi k\quad \text{für}\quad \mathrm{sinc}(\omega\frac{T_h}{2})\geq0\\
\pi + 2\pi k\quad \text{für}\quad\mathrm{sinc}(\omega\frac{T_h}{2})<0,
\end{cases}
\end{equation}
Die Phase ist eine stückweise stetige Geradengleichung über $\omega$ durch den
Ursprung und mit Anstieg $+T_h/2$, also diesmal steigend!
%
Die \textbf{Zeitumkehr} erzeugt also eine \textbf{Phasenumkehr}, vgl.
Aufgabe \ref{sec:A8A2DEE53A} fallende Phase, Aufgabe \ref{sec:1CFE5FE3A1} hier steigende Phase.
%
Dies gilt auch wieder verallgemeinert und ist eine wichtige \textbf{Symmetrie}eigenschaft
der Fourier Transformation
%
\begin{align}
x(+t) \quad \laplace \quad X(+\im\omega) =& |X(\im\omega)| \, \e^{+\im\angle X(\im\omega)}\\
x(-t) \quad \laplace \quad X(-\im\omega) =& |X(\im\omega)| \, \e^{-\im\angle X(\im\omega)}
\end{align}
%
Zufällig in unserem Beispiel konnten wir das Problem auch über Zeitverschiebung
lösen, weil wir die unverschobene Rechteckfunktion bereits gut kennen.
\end{Loesung}
%
\begin{figure}[h!]
\includegraphics[width=\textwidth]{../ft/1CFE5FE3A1.pdf}
  \caption{Betrag (oben) und Phase (unten) der Fouriertransformierten \eq{eq:1CFE5FE3A1_Loesung}.
\textbf{Vorsicht:} Die Grafik suggeriert, dass eine zeitliche Links-Verschiebung
sowohl Phase als auch Betrag ändert. Dies passiert aber hier speziell, weil wir in der
Aufgabe nach einer Verschiebung um $-T_h/2$ gefragt haben, und $T_h$
im Argument der Spaltfunktion die Breite dieser bestimmt.
\texttt{FourierTransformation\_1CFE5FE3A1.ipynb}}
  \label{fig:1CFE5FE3A1}
\end{figure}





\clearpage
\subsection{Fourier-Transformation Signal-Modulation}
\label{sec:9D652BE72B}
\begin{Ziel}
Anhand eines speziellen Beispiels erfassen wir das Wesen des Modulationstheorems.
\end{Ziel}
\textbf{Aufgabe} {\tiny 9D652BE72B}: Berechnen Sie die Fouriertransformation für
\begin{itemize}
\item axialsymmetrische Rechteckfunktion der Breite $T_h$ und Amplitude $A$
\item die mit der komplexen Schwingung $\e^{+\im \omega_0 t}$ mit Kreisfrequenz $\omega_0>0$
multipliziert wird
\end{itemize}
Wir wollen die Fouriertransformation Korrespondenz \eq{eq:A8A2DEE53A_ModTheoremTime}
\begin{align}
x(t) \cdot \e^{+\im\omega_0 t} & \quad \laplace \quad X(\im[\omega-\omega_0])
\end{align}
nochmal am Beispiel sehen.

\begin{figure}[h!]
\centering
\begin{tikzpicture}[domain=0:2]
\draw[->] (-2,0) -- (2,0) node[below right] {$t$};
\draw[->] (0,0) -- (0,1.5) node[above] {$x(t)$};
\draw[-, C0, ultra thick] (-2,0) -- (-0.5,0) node[below] {$\frac{-T_h}{2}$} -- (-0.5,1) node[left] {$A$} -- (0.5,1) -- (0.5,0) node[below] {$\frac{+T_h}{2}$} -- (2,0);
\end{tikzpicture}
\end{figure}

\begin{Werkzeug}
Analyse mit Fouriertransformation
\begin{align}
X(\im \omega) = \int\limits_{-\infty}^{+\infty} x(t) \, \e^{-\im \omega t} \fsd t
\end{align}
\end{Werkzeug}
\begin{Ansatz}
\begin{align}
X_m(\im \omega) = \int\limits_{-T_h/2}^{+T_h/2} \e^{+\im \omega_0 t} \cdot A \, \e^{-\im \omega t} \mathrm{d}t
\end{align}
\end{Ansatz}
\begin{ExCalc}
\begin{align}
X_m(\im \omega) = A \int\limits_{-T_h/2}^{+T_h/2} \e^{-\im (\omega-\omega_0) t} \mathrm{d}t =
A \frac{\e^{-\im (\omega-\omega_0) t}}{-\im (\omega-\omega_0)}\bigg|_{-T_h/2}^{+T_h/2} =
A \frac{\e^{-\im (\omega-\omega_0) T_h/2} - \e^{+\im (\omega-\omega_0) T_h/2}}{-\im (\omega-\omega_0)}
\end{align}
Vorzeichen drehen, erweitern, $\sin(x) = \frac{\e^{+\im x}-\e^{-\im x}}{2\im}$, $\mathrm{sinc}(x):=\frac{\sin (x)}{x}$
\begin{align}
X_m(\im \omega) = A \frac{\e^{+\im (\omega-\omega_0) T_h/2} - \e^{-\im (\omega-\omega_0) T_h/2}}{\im (\omega-\omega_0)} \cdot \frac{T_h/2}{T_h/2} \cdot \frac{2}{2}
\end{align}
\end{ExCalc}
\begin{Loesung}
\begin{align}
X_m(\im \omega) = A T_h \cdot \mathrm{sinc}\left((\omega-\omega_0) \frac{T_h}{2}\right)
\end{align}
Wir können mit Vergleich dieser und  Aufgabe \ref{sec:8C3958BE4F} beobachten, dass
die Korrespondenz
\begin{align}
\e^{+\im \omega_0 t} \cdot x(t) \quad\laplace\quad X\left(\im [\omega-\omega_0]\right)
\end{align}
gilt.
%
Dieser Zusammenhang gilt generell und ist als Modulationstheorem der
Fourier Transformation bekannt.
%
Das Spektrum (die Fouriertransformierte) $X(\im \omega)$ wird im Frequenzbereich
um $\omega_0$ verschoben, wenn $x(t)$ mit der komplexen Trägerschwingung
$\e^{+\im \omega_0 t}$ multipliziert wird. In der Nachrichtentechnik bezeichnet man
diesen Vorgang als Modulation von $x(t)$ auf den Träger $\e^{+\im \omega_0 t}$.
Dies benutzt man um Spektren in einen anderen (freien) Frequenzbereich zu
verschieben. Das werden wir bald im Detail besser verstehen.
\end{Loesung}


\begin{mdframed}
\textit{Ausblick:}
%
\\\noindent Wir könnten an dieser Stelle durchaus schon in der Lage sein, uns die
Fouriertransformierten von $x(t) = \cos(\omega_0 t)$ und $x(t) = \sin(\omega_0 t)$
abzuleiten und in Betrag und Phase zu skizzieren.
%
Dazu ist es nicht ratsam, dass Transformationsintegral lösen zu wollen,
es konvergiert nicht, sondern die bisher bekannten Korrespondenzen zusammen
mit den Euler Identitäten
\begin{align}
\cos(\omega_0 t) = \frac{\e^{+\im\omega_0 t}+\e^{-\im\omega_0 t}}{2}\qquad
\sin(\omega_0 t) = \frac{\e^{+\im\omega_0 t}-\e^{-\im\omega_0 t}}{2\im}
\end{align}
zu benutzen.
%
Diese Idee durchzieht die SigSys: Suchen nach und Arbeiten mit passenden
Korrespondenzen. Wir werden als Ingenieure sehr wahrscheinlich keine unbekannte
Fouriertransformierte neu finden! Lassen wir dieses Feld den Integralprofis, wie
\cite{Abramowitz1972}, \cite{Gradshteyn2007} und lernen stattdessen was wir mit den
Tools anstellen können und was sie uns im Wesen sagen.
\end{mdframed}

%
In den bisherigen Aufgaben haben wir uns mit einfachen Signalen und der
Fouriertransformierten beschäftigt, sprich wir haben geschaut, welche Frequenzen
stecken mit welcher Amplitude in den untersuchten Signalen.
%
Als Spezialfall bekamen wir für periodische Signale Fourierkoeffizienten.
%
Das ist der \textbf{SigSys-Teil: Signalanalyse und -synthese}. Wir werden insgesamt
\textbf{vier Fouriertransformationen} kennenlernen, alle spezialisiert auf bestimmte
Signaltypen, zwei kennen wir nun schon ein wenig besser: Fourierreihe und
Fouriertransformation.
Die anderen beiden sind

\url{https://de.wikipedia.org/wiki/Diskrete_Fourier-Transformation}

\url{https://de.wikipedia.org/wiki/Fouriertransformation_für_zeitdiskrete_Signale}

Die Idee mit Korrespondenzen zu arbeiten, bleibt für alle Transformationen
erhalten. Auch werden wir die Sinc/Rect-Dualitäten immer wieder finden,
genauso wie Zeit-/Frequenzverschiebung.




\begin{comment}
\newpage
\subsection{Fourier-Transformation Cosinus-Schwingung}
\label{sec:610482EF57}
\begin{Ziel}
Mit dem (eigentlich noch recht bescheidenen) Wissen zur Fourier-Transformation
können wir trotzdem schon eine ganze Menge fundamental wichtiger Aspekte ableiten,
diese sind in Büchern bereits bestens abgedeckt.
Es kann jedoch nicht schaden, hin und wieder Sachen neu zu erfinden.
Wir werden die Qualität je schmaler/breiter Rechteckfunktion desto breiter/schmaler
Spaltfunktion erfassen und uns das zunutze machen, um die auf direktem Wege
für uns noch nicht lösbare Aufgabenstellung trotzdem zu lösen.
\end{Ziel}
\textbf{Aufgabe} {\tiny 610482EF57}: Berechnen Sie die Fouriertransformation für
\begin{itemize}
\item Cosinus-Schwingung $\cos(\omega_1 t)$ mit Kreisfrequenz $\omega_1>0$
\end{itemize}
\begin{Werkzeug}
Analyse mit Fouriertransformation
\begin{align}
X(\im \omega) = \int\limits_{-\infty}^{+\infty} x(t) \, \e^{-\im \omega t} \fsd t
\end{align}
\end{Werkzeug}
\begin{Ansatz}
\begin{align}
X(\im \omega) = \int\limits_{-\infty}^{+\infty} \cos(\omega_1 t) \, \e^{-\im \omega t} \fsd t
\end{align}
Dieses Integral konvergiert nicht und suggeriert, dass es keine Lösung gibt, was
natürlich nicht stimmt, sonst gäbe es die Aufgabenstellung nicht. Wir müssen
anders herangehen.
%
Wir könnten $x(t) = \cos(\omega_1 t)$ definieren und das Modulationstheorem
aus vorheriger Aufgabe mit $\omega_0=0$ zu Hilfe nehmen. Das bringt uns nur keinen
Schritt weiter.
%
Wir könnten $x(t) = 1$ definieren und das Modulationstheorem
aus vorheriger Aufgabe mit $\omega_0=\omega_1$ zu Hilfe nehmen. Dann haben wir
zwei Probleme mehr: a) wir kennen die Transformationsregel für die Modulation
mit dem reellen Cosinussignal noch nicht und b) wir kennen die Lösung der
Fourier Transformation des unmodulierten Signals $x(t)=1$, also
\begin{align}
X(\im \omega) = \int\limits_{-\infty}^{+\infty} 1 \e^{-\im \omega t} \fsd t = ?
\end{align}
noch nicht. Es stresst hier wieder, dass das Integral nicht konvergiert. Mit
unendlichen Grenzen können wir hier noch nicht gut umgehen.
%
Benutzen wir also unser Handwerkszeug aus den vorigen Übungen und legen zunächst
eine endliche Integrationsgrenze fest, nämlich mit Hilfe der
axialsymmetrischen Rechteckfunktion der Breite $T$ und Amplitude $A$. Diesen
Rechteckverlauf lassen wir dann immer breiter werden, bis er sich im Grenzfall
von $-\infty$ bis
$+\infty$ erstreckt.
%
Unser Ansatz lautet also
\begin{align}
X(\im \omega) = \int\limits_{-T_h/2}^{+T_h/2} A \cos(\omega_1 t) \, \e^{-\im \omega t} \fsd t
\end{align}
Wir erkennen, dass der Reckteckimpuls auf den Cosinus-Träger moduliert wird, aber
wir kennen das Modulationsgesetz diesbezüglich noch nicht.
Daher lösen wir das Integral zu Fuß.
\end{Ansatz}
\begin{ExCalc}
Wir benutzen $\cos(x) = \frac{1}{2}(\e^{\im x} + \e^{-\im x})$ um
\begin{align}
X(\im \omega) = \int\limits_{-T_h/2}^{+T_h/2} A \frac{1}{2}(\e^{\im \omega_1} + \e^{-\im \omega_1}) \, \e^{-\im \omega t} \fsd t =
\frac{A}{2} \int\limits_{-T_h/2}^{+T_h/2} (\e^{\im \omega_1} + \e^{-\im \omega_1}) \, \e^{-\im \omega t} \fsd t
\end{align}
%
\begin{align}
X(\im \omega) = \int\limits_{-T_h/2}^{+T_h/2} \left(\e^{-\im (\omega-\omega_1) t} + \e^{-\im (\omega+\omega_1) t} \right) \fsd t
\end{align}
%
\begin{align}
X(\im \omega) =
\frac{A}{2} \frac{\e^{-\im (\omega-\omega_1) t}}{-\im (\omega-\omega_1)}\bigg|_{-T_h/2}^{+T_h/2}
+\frac{A}{2} \frac{\e^{-\im (\omega+\omega_1) t}}{-\im (\omega+\omega_1)}\bigg|_{-T_h/2}^{+T_h/2}
\end{align}
%
\begin{align}
X(\im \omega) =
\frac{A}{2} \frac{\e^{-\im (\omega-\omega_1) T_h/2} - \e^{+\im (\omega-\omega_1) T_h/2}}{-\im (\omega-\omega_1)}
+\frac{A}{2} \frac{\e^{-\im (\omega+\omega_1) T_h/2} - \e^{+\im (\omega+\omega_1) T_h/2}}{-\im (\omega+\omega_1)}
\end{align}
%
Vorzeichen, Erweitern
%
\begin{align}
X(\im \omega) =
\frac{A}{2} \frac{\e^{+\im (\omega-\omega_1) T_h/2} - \e^{-\im (\omega-\omega_1) T_h/2}}{\im (\omega-\omega_1)} \cdot \frac{T_h/2}{T_h/2}
+\frac{A}{2} \frac{\e^{+\im (\omega+\omega_1) T_h/2} - \e^{-\im (\omega+\omega_1) T_h/2}}{\im (\omega+\omega_1)} \cdot \frac{T_h/2}{T_h/2}
\end{align}
%
Mit $\sin(x) = \frac{\e^{+\im x}-\e^{-\im x}}{2\im}$, $\mathrm{sinc}(x):=\frac{\sin (x)}{x}$
\begin{align}
X(\im \omega) =
A \frac{T_h}{2} \mathrm{sinc}\left((\omega-\omega_1) \frac{T_h}{2}\right) +
A \frac{T_h}{2} \mathrm{sinc}\left((\omega+\omega_1) \frac{T_h}{2}\right)
\end{align}
Zur Kontrolle: für $\omega_1 = 0$ bekommen wir das Ergebnis aus Aufgabe
\ref{sec:8C3958BE4F}, ein Indiz, dass wir uns nicht verrechnet haben.
\end{ExCalc}
%
\begin{Loesung}
Wir bekommen zwei mit $A \frac{T_h}{2}$ gewichtete Spaltfunktionen,
verschoben um $\pm \omega_1$ auf der Frequenzachse. Das Maximum der ersten Spaltfunktion
liegt bei $+\omega_1$ (positive Frequenz), das Maximum der zweiten Spaltfunktion
liegt bei $-\omega_1$ (negative Frequenz).



Was ist nun mit der ursprünglichen Aufgabe die Fourier Transformierte des Cosinus
zu finden?
%
Wir können sie ehrlicherweise nur mit einem Vorgriff auf kommenden Stoff erarbeiten, das
ist aber so fundamental, dass es sich lohnt, das hier schonmal anzudeuten.
%
Die Grundidee ist bekanntes mathematisches Handwerkszeug: wir lassen
$T_h\to \infty$ gehen. Dann wird sich für beide gewichtete Spaltfunktionen ein
geschlossener Ausdruck finden lassen.

\end{Loesung}

\begin{mdframed}
\textit{Vorgriff}
%
Zunächst nehmen wir unser vorläufiges Endergebnis und schreiben die Spaltfunction
aus
\begin{align}
X(\im \omega) =
A \frac{T_h}{2} \frac{\sin\left((\omega-\omega_1) \frac{T_h}{2}\right)}{(\omega-\omega_1) \frac{T_h}{2}} +
A \frac{T_h}{2} \frac{\sin\left((\omega+\omega_1) \frac{T_h}{2}\right)}{(\omega+\omega_1) \frac{T_h}{2}}
\end{align}
kürzen ein wenig und erweitern mit $\pi$
\begin{align}
X(\im \omega) =
A \pi \frac{\sin\left((\omega-\omega_1) \frac{T_h}{2}\right)}{\pi (\omega-\omega_1)} +
A \pi \frac{\sin\left((\omega+\omega_1) \frac{T_h}{2}\right)}{\pi (\omega+\omega_1)}
\end{align}
und stellen im sin-Argument noch ein wenig die Brüche um
\begin{align}
X(\im \omega) =
A \pi \frac{\sin\left( \frac{(\omega-\omega_1)}{\frac{2}{T_h}}  \right)}{\pi (\omega-\omega_1)} +
A \pi \frac{\sin\left( \frac{(\omega+\omega_1)}{\frac{2}{T_h}}  \right)}{\pi (\omega+\omega_1)}
\end{align}
weil wir jetzt einen Ausdruck ähnlich $\sin(\omega / \epsilon)$ stehen haben.
%
Die Mathematiker schenken uns nun für folgenden Grenzübergang
\begin{align}
\delta(\omega) = \lim_{\epsilon\to 0} \frac{\sin(\frac{\omega}{\epsilon})}{\pi \omega}
\end{align}
einen geschlossenen Ausdruck $\delta(\omega)$.
%
Wenden wir $\epsilon\to 0$ für unseren spezifischen Term $\frac{2}{T_h}\to 0$ an,
was wie gewünscht $T_h\to \infty$ bedingt,
können wir die obige Definition verwenden und schreiben
\begin{align}
X(\im \omega) = A \pi \delta(\omega-\omega_1) + A \pi \delta(\omega+\omega_1)
\end{align}
%
Für $A=1$ folgt weiter trivial
\begin{align}
X(\im \omega) = \pi \, \delta(\omega-\omega_1) + \pi \, \delta(\omega+\omega_1)
\end{align}
%
Machen wir uns klar, dass durch die Wahl $T_h\to \infty$ (also Verbreiterung
der Rechteckfunktion bis ins Unendliche) die gesuchte Korrespondenz
\begin{align}
\cos(\omega_1 t) \quad\laplace\quad \pi \, \delta(\omega-\omega_1) + \pi \, \delta(\omega+\omega_1)
\end{align}
gefunden wurde.
%
Der Ausdruck $\delta(\omega)$ ist bekannt als Dirac-Impuls, es ist aufgrund seiner
Eigenschaften keine normale Funktion, sondern eine Distribution.
Momentan hilft es uns weiter, wenn wir uns den Dirac-Impuls $\delta(\omega)$
als unendlich hohe, unendlich schmale Rechteckfunktion der Fläche 1 an der Stelle
$\omega=0$ vorstellen, siehe Grafiken unten.

Die Fourier-Transformation des Cosinus besteht daher aus zwei mit $\pi$ gewichteten
Dirac-Impulsen, die verschoben sind. Der erste Dirac-Impuls ist bei $+\omega_1$
(positive Kreisfrequenz) und der zweite Dirac ist bei $-\omega_1$
(negative Kreisfrequenz).
%
Wir erkennen eine Analogie zur komplexen Fourierreihe bei der es auch zweier
komplexer Fourierkoeffizienten (bei $\pm k$) bedarf um eine Cosinus-Schwingung
darzustellen bzw. zu synthetisieren.
%
Dies soll als Ausblich des \textbf{Dirac-Impulses} erstmal genügen.
\end{mdframed}
\end{comment}







\newpage
\subsection{Dirac-Impuls}
\label{sec:D410BDAAE0}
\begin{Ziel}
Wir führen den Dirac-Impuls ein. Ohne den geht in SigSys (fast) nix, bekannt auch als
Dirac-Delta-Impuls, Dirac-Stoß, Einheitsimpuls.
Wir werden den Dirac-Impuls bei der Beschreibung von Signalen und Systemen
in fundamentalen Zusammenhängen benutzen (müssen).
Wir haben in den bisherigen Aufgaben ganz gut vorgearbeitet und können uns
mit der Rechteck- und Spaltfunktion dem Phänomen Dirac-Impuls nähern.
Dem Ingenieur macht das Konzept am Anfang ein wenig Bauchweh, weil
der Dirac-Impuls keine klassische Funktion ist, sondern eine Distribution, d.h.
es gibt bestimmte Funktionen, für die zwar ein gewünschter Grenzwert nicht existiert,
die aber gleiche Grundeigenschaften besitzen, wenn man sich dieser Grenze annähert.
Diese Grundeigenschaften werden wir kennenlernen als sehr nützliche
Rechenregeln des Dirac-Impulses, und damit haben wir eigentlich das nötige
Handwerkszeug für SigSys dann schon parat.
Die Rechteck- und Spaltfunktion sind zwei solcher Funktionen mit gleichen
Grundeigenschaften die einen Dirac-Impuls ausmachen.

Zudem werden wir zwei fundamentale Korrespondenzen der Fouriertransformation
kennenlernen, nämlich die des Dirac-Impulses und die des Gleichsignals. Das
war bisher noch nicht lösbar, weil das Transformationsintegral nicht konvergiert.
\end{Ziel}
\textbf{Aufgabe} {\tiny D410BDAAE0}:
Finden Sie $x(t)=\delta(t)\quad \laplace \quad X(\im\omega)=?$ und
$x(t)= ? \quad \laplace \quad X(\im\omega)=2\pi \delta(\omega)$
\begin{Werkzeug}
Führen wir für $\epsilon\in\mathbb{N}^+$ eine Funktion $\delta_\epsilon(x)$ ein, die entweder
\begin{equation}
\delta_\epsilon(x) := \epsilon \cdot \mathrm{rect}(\epsilon \, x) =
\begin{cases} \epsilon & |x| < \frac{1}{2 \epsilon} \\ \frac{\epsilon}{2} & |x| = \frac{1}{2 \epsilon} \\ 0 & |x| > \frac{1}{2 \epsilon} \end{cases}\quad
\end{equation}
oder
\begin{equation}
\delta_\epsilon(x) := \frac{\epsilon}{\pi} \cdot \mathrm{sinc}(\epsilon \, x) = \frac{\sin(\epsilon \, x)}{\pi x}
\end{equation}
ist.
Wir entdecken die, zwar ein wenig normiert, aber uns mittlerweile vertraute
Rechteck- und Spaltfunktion  wieder.
%
Es gilt für beide Funktion, und zwar unabhängig von $\epsilon$ (für die
Rechteckfunktion ist das leicht einzusehen, für die sinc-Funktion überlassen wir
das den Mathematikern oder müssen uns vertiefter mit dem Integralsinus auseinander setzen)
\begin{equation}
\int\limits_{-\infty}^{+\infty} \delta_\epsilon(x) \fsd x = 1,
\end{equation}
d.h. die \textbf{Fläche ist immer 1} bei der gewählten Normierung der
Rechteck- und Spaltfunktion.
%
Nun sind die Grenzwerte
\begin{equation}
\lim_{\epsilon\to\infty} \delta_\epsilon(x)
\end{equation}
leider nicht definiert.
%
Machen wir uns aber klar, dass sowohl die Rechteck- als auch Spaltfunktion mit
größer werdendem $\epsilon\to \infty$ immer schmaler und höher werden und sich
einem Impuls annähern, siehe \fig{fig:1CFE5FE3A1}.
%
Die Mathematik\footnote{\url{https://dlmf.nist.gov/1.17} bzw. Bücher
\cite[Kapitel 1.15]{Arfken2005}, \cite{Burg2013}}
liefert uns nun eine Lösung im Sinne eines Grenzwerts für eine
Sequenz von Integralen über $\delta_\epsilon(x)$, $\epsilon=1,2,3,...$
\begin{equation}
\lim_{\epsilon\to\infty} \int\limits_{-\infty}^{+\infty}
\delta_\epsilon(x-x_0) \cdot f(x) \, \fsd x = f(x_0).
\end{equation}
Wir sehen, dass dieser Grenzwert auf eine Funktion $f(x)$ einwirkt und als Lösung
$f(x_0)$ hervorbringt, wenn $\delta_\epsilon(x)$ um $x_0$ verschoben wurde, also
$\delta_\epsilon(x-x_0)$ im Integral benutzt wurde.
%
Das ist das was uns primär als Ingenieure interessiert.
%
Der für uns etwas sperrige Ausdruck
\begin{equation}
\lim_{\epsilon\to\infty} \int\limits_{-\infty}^{+\infty}
\delta_\epsilon(x-x_0) \cdot f(x) \, \fsd x
\end{equation}
bekommt jetzt eine Definition mit Einführung des sogenannten Dirac-Impulses $\delta(x)$
\begin{equation}
\label{eq:DiracDistrDef}
\lim_{\epsilon\to\infty} \int\limits_{-\infty}^{+\infty}
\delta_\epsilon(x-x_0) \cdot f(x) \, \fsd x
:= \int\limits_{-\infty}^{+\infty} \delta(x-x_0) \cdot f(x) \, \fsd x \stackrel{\mathrm{def}}= f(x_0)
\end{equation}
%
Die rechte Seite ist \textbf{kein} klassisches Riemann-Integral, sondern eine
\textbf{Definition} (nämlich die des linken Grenzwerts),
für uns aber eine nützliche \text{Rechenregel} und Grundeigenschaft des
Dirac-Impulses. Diese ist als \textbf{Austasteigenschaft} oder
\textbf{Ausblendeigenschaft} bekannt. Je nachdem wie wir denken wollen:
Austasten nur des Funktionswertes $f(x_0)$ oder Ausblenden aller Funktionswerte
außer $f(x_0)$.

Aus dieser können wir sofort ableiten, dass
\begin{equation}
\label{}
\int\limits_{-\infty}^{+\infty} \delta(x) \cdot f(x) \, \fsd x = f(0)
\end{equation}
wenn $x_0=0$ gewählt wurde.
%
Weiter finden wir
\begin{equation}
\int\limits_{-\infty}^{+\infty} \delta(x) \fsd x = 1
\end{equation}
wenn $f(x)=1$ gewählt wurde.
%
Beide Beziehungen konnten wir nur deswegen sehr schnell und einfach ableiten, weil
wir die Definition benutzt haben. Wir müssen uns aber klar machen, dass wir
nicht das Integral gelöst haben, sondern vielmehr die Rechenregel der Austastung
verwendet haben.
%
Die Versuchung das letzte Integral als Fläche des Dirac-Impulses zu interpretieren
ist sehr groß, und wenn wir den Grenzübergang mit der Rechteckfunktion in Gedanken
tatsächlich vollziehen auch nicht falsch, aber wir können das streng genommen
nicht aus dem Definitionsintegral
$\int\limits_{-\infty}^{+\infty} \delta(x-x_0) \cdot f(x) \, \fsd x \stackrel{\mathrm{def}}= f(x_0)$
des Dirac-Impulses auslesen.
%
Statt Fläche, sprechen wir hinfort vom Gewicht 1 des Dirac-Impulses.

\end{Werkzeug}

\begin{figure}[h!]
\includegraphics[width=\textwidth]{../dirac_impulse_ct/D410BDAAE0.pdf}
  \caption{Rechteck- (oben) und Spaltfunktion (unten) unterschiedlicher Breite und Höhe.
  Durch die Normierung verhalten sich Breite und Höhe umgekehrt proportional und
  gehen daher in den Grenzfällen zu einem unendlich kleinem Gleichsignal oder einem unendlich kurzen und hohen
  Impuls über. Diese beiden Funktionen erfüllen im Grenzfall $\epsilon\to\infty$
  die Distributionsforderungen des Dirac-Impulses \eq{eq:DiracDistrDef}.
  \texttt{dirac\_impulse\_CT\_D410BDAAE0.ipynb}}
  \label{fig:D410BDAAE0}
\end{figure}

\begin{figure}[h!]
\centering
%
\begin{tikzpicture}0
\def\l{-2.5}
\def\r{+2.5}
\draw[->] (-2+\l,0) -- (2+\l,0) node[right] {$t$};
\draw[->] (0+\l,0) -- (0+\l,1.5) node[above] {$x(t)=\delta(t)$};
\draw[->, C0, line width=1mm] (0+\l,0) -- (0+\l,1) node[left] {$(1)$};
\node at (0,1) {$\laplace$};
\draw[->] (-2+\r,0) -- (2+\r,0) node[right] {$\omega$};
\draw[->] (0+\r,0) -- (0+\r,1.5) node[above] {$X(\im\omega)=1$};
\draw[-, C0, line width=1mm] (-2+\r,1) -- (2+\r,1) node[right] {$1$};
\end{tikzpicture}
%
\begin{tikzpicture}0
\def\l{-2.5}
\def\r{+2.5}
\draw[->] (-2+\l,0) -- (2+\l,0) node[right] {$t$};
\draw[->] (0+\l,0) -- (0+\l,1.5) node[above] {$x(t)=1$};
\draw[-, C0, line width=1mm] (-2+\l,1) node[left] {$1$} -- (2+\l,1);
\node at (0,1) {$\laplace$};
\draw[->] (-2+\r,0) -- (2+\r,0) node[right] {$\omega$};
\draw[->] (0+\r,0) -- (0+\r,1.5) node[above] {$X(\im\omega)=2\pi\delta(\omega)$};
\draw[->, C0, line width=1mm] (0+\r,0) -- (0+\r,1) node[right] {$(2\pi)$};
\end{tikzpicture}
%
\caption{Korrespondenzen Fouriertransformation: \eq{eq:D410BDAAE0_Loesung1} (oben),
\eq{eq:D410BDAAE0_Loesung2} (unten).}
\label{fig:D410BDAAE0_Korrespondenzen}
\end{figure}



\cleardoublepage
\begin{Ansatz}
Die gesuchte Korrespondenz $x(t)=\delta(t)\quad \laplace \quad X(\im\omega)=?$
können wir ableiten aus der Ausblendeigenschaft
$\int\limits_{-\infty}^{+\infty} \delta(t-t_0) \cdot f(t) \, \fsd t \stackrel{\mathrm{def}}= f(t_0)$
mit $f(t) = \e^{-\im \omega t}$ und $t_0=0$.
%
Dann bekommt die Ausblendeigenschaft das Aussehen der Fouriertransformation
des Dirac-Impulses
\begin{equation}
X(\im\omega) = \int\limits_{-\infty}^{+\infty} \delta(t) \cdot \e^{-\im \omega t} \, \fsd t \stackrel{\mathrm{def}}= \e^{-\im \omega \cdot 0} = 1
\end{equation}

Auch die gesuchte Korrespondenz
$x(t)= ? \quad \laplace \quad X(\im\omega)=2\pi \delta(\omega)$ lässt sich
ableiten aus der Ausblendeigenschaft, diesmal mit der Funktionsvariablen Kreisfrequenz, also mit

$\int\limits_{-\infty}^{+\infty} \delta(\omega-\omega_0) \cdot f(\omega) \, \fsd \omega \stackrel{\mathrm{def}}= f(\omega_0)$
für $f(\omega) = \e^{+\im \omega t}$ (Vorzeichen der inversen Fouriertransformation $+$!) und $\omega_0=0$.
%
Dann bekommen wir
\begin{equation}
2\pi \cdot \int\limits_{-\infty}^{+\infty} \delta(\omega) \cdot \e^{+\im \omega t} \, \fsd \omega \stackrel{\mathrm{def}}= 2\pi \cdot \e^{+\im \omega \cdot 0} = 2\pi
\end{equation}
%
Weil bei der inversen Fouriertransformation
\begin{align}
x(t) = \frac{1}{2\pi} \int\limits_{-\infty}^{+\infty} X(\im\omega) \, \e^{+\im \omega t} \fsd \omega
\end{align}
noch der Faktor $\frac{1}{2\pi}$ berücksichtigt werden muss, erhalten wir
\begin{equation}
x(t) = \int\limits_{-\infty}^{+\infty} \delta(\omega) \cdot \e^{+\im \omega t} \, \fsd \omega \stackrel{\mathrm{def}}= \e^{+\im \omega \cdot 0} = 1
\end{equation}

\end{Ansatz}
%\begin{ExCalc}
%\end{ExCalc}
\begin{Loesung}
Damit haben wir zwei weitere wichtige Korrespondenzen gefunden, in der Formelsammlung
ganz oben auf der Liste:
\begin{align}
\label{eq:D410BDAAE0_Loesung1}
x(t)= \delta(t) &\quad \laplace \quad X(\im\omega)=1\\
\label{eq:D410BDAAE0_Loesung2}
x(t)= 1 &\quad \laplace \quad X(\im\omega)=2\pi \delta(\omega)\qquad \text{Dirac mit Gewicht}\,2\pi
\end{align}
%
Diese sind in \fig{fig:D410BDAAE0_Korrespondenzen} veranschaulicht.
%
Wesentlich ist, dass ein Dirac-Impuls alle Frequenzen gleich gewichtet enthält
(\fig{fig:D410BDAAE0_Korrespondenzen} oben)
und dass ein Gleichsignal, nur einen Dirac-Impuls bei $\omega=0$ enthält
(\fig{fig:D410BDAAE0_Korrespondenzen} unten).
Weil hier nichts schwingt, gibt es nur ein Dirac-Gewicht beim Gleichanteil.

\end{Loesung}




\begin{mdframed}
\textit{Ausblick:}
%
\\\noindent
Wir haben bisher die Korrespondenzen
\begin{align}
x(t)= \delta(t) &\quad \laplace \quad X(\im\omega)=1\\
x(t)= 1 &\quad \laplace \quad X(\im\omega)=2\pi \delta(\omega)\\
x(t - \tau) & \quad \laplace \quad X(\im\omega) \cdot \e^{-\im\omega \tau}\\
x(t) \cdot \e^{+\im\omega_0 t} & \quad \laplace \quad X(\im[\omega-\omega_0])\\
\end{align}
kennengelernt.
%
Diese lassen sich nun geschickt kombinieren, um weitere Korrespondenzen zu finden,
z.B. ein verschobener Dirac-Impuls im Zeitbereich
\begin{align}
\delta(t-\tau)& \quad \laplace \quad 1 \cdot \e^{-\im\omega \tau}.
\end{align}
Wir erinnern uns: Zeitverschiebung für zu Phasenänderung im Frequenzbereich.

Oder ein verschobener Dirac-Impuls im Frequenzbereich
\begin{align}
1 \cdot \e^{+\im\omega_0 t} & \quad \laplace \quad 2\pi\delta(\omega-\omega_0),
\end{align}
bedeutet Phasenänderung im Zeitbereich, speziell resultiert eine Multiplikation
mit einer komplexen Schwingung auf das ursprüngliche Gleichsignal $x(t)=1$.
%
In der nächsten Übung wird diese Korrespondenz nochmal näher mit einem
Rechteckzeitsignal beleuchtet. Aus der zeitlichen Begrenzung des Gleichsignals
zum Rechtecksignal resultiert ein Übergang vom Dirac-Impuls zur Spaltfunktion
für die Fouriertransformierte.
\end{mdframed}














\newpage
\subsection{Lineare Differentialgleichungen}
\label{sec:A7BEE9E24E}
%
Zunächst keine Aufgabe, sondern erst ein Ausblick in den Kontext einer Wiederholung
gestellt.
Wir betrachten hier einleitend nun den \textbf{SigSys-Teil: Systemanalyse und -synthese}.
Dafür werden wir die Laplace-Transformation (für zeitkontinuierliche Signale, Systeme)
und die $z$-Transformation (für zeitdiskrete Signale/Systeme) kennenlernen, um
Differentialgleichungen und Differenzengleichungen elegant lösbar zu machen.
%
Signale und Systeme sind natürlich sehr eng verknüpft, die hier aufgespannte
Trennung sollten wir daher nicht dogmatisch verfolgen, sondern dient als grober
Überblick, was uns in SigSys erwarten wird.

In Mathe haben wir das Konzept der Differentialgleichung (DGL) kennengelernt.
%
Wir haben verschiedene Lösungsmöglichkeiten kennengelernt (Variation
der Konstanten, Variablentrennung, Eigenlösungen und charakteristisches Polynom,
Störgliedansätze, usw. Ich persönlich fand das in meinem Studium damals
ziemlich wirr, Rechnen konnte man damit irgendwann, aber Verstehen hat erst
später eingesetzt. Das muss eigentlich nicht so sein, wie das Buch
\cite{Strang2014} zeigt.
Anyway...).

Lineare DLG mit konstanten (also \textbf{nicht zeitveränderliche}n) Koeffizienten der
ersten und zweiten Ordnung sind die wichtigsten Vertreter, die wir als Ingenieure
in SigSys brauchen, weil man damit beeindruckend viele Dinge bauen und modellieren
kann, Feder-Masse-Dämpfer und RLC-Schwingkreis sind die beiden Klassiker.
%
In den Grundlagen der E-Technik lernten wir alles erst zu Fuß zu lösen, also
die Wechselstromtechnik und die Einschaltvorgänge an RLC-Schaltungen.
%
NachrichtentechnikerInnen bauen sich aus diesen DGLs Filter, früher analog, heute digital.
%
RegelungstechnikerInnen bauen sich im Grunde die gleichen Filter, nennen sie nur
selten Filter, sondern eher Glieder mit z.B. differenzierendem, proportionalem,
integrierendem, usw. Verhalten.
%
Die Brücke zwischen E-Technik und Nachrichten-/Regelungstechnik ist SigSys!
Das sollten wir immer im Hinterkopf behalten.

Wir werden daher in der SigSys eine elegante Form finden,
diesen DGL-Typ für beliebige Störgliedansätze zu lösen (am Ende des Tages
müssen wir ja praktisch vorkommende Signale verarbeiten und können uns nicht
auf schicke Störgliedformeln beschränken)
und das Konzept der Eigenlösungen und des charakteristischen
Polynoms vertiefen bzw. aus anderem Blickwinkel betrachten (am Ende des Tages
wollen wir für eine bestimmte Aufgabe eine Lösung mit gewünschter Genauigkeit
mit dem einfachsten Aufwand erhalten. Praktisch die gesamte SigSys ist daraufhin
ausgelegt, auch wenn sich das momentan noch nicht so anfühlt).

Der folgende Abschnitt soll daher die SigSys Denke ein wenig schmackhaft machen und
greift den didaktisch brillanten Faden aus \cite{Strang2014} auf. Viele Dinge
müssten uns aus der Mathe-VL vertraut vorkommen.
%
Zum Auffrischen sei \cite{Burg2013} sehr empfohlen, das ist SigSys aus Sicht von
Mathematikern.
%
\subsubsection{Lineare DGL 2. Ordnung mit konstanten Koeffizienten}
Betrachten wir beispielhaft die lineare DGL
\begin{align}
\label{eq:A7BEE9E24E_DGL}
A \frac{\fsd^2 y(t)}{\fsd t^2} + B \frac{\fsd y(t)}{\fsd t} + C y(t) = x(t),
\end{align}
mit den konstanten Koeffizienten $A,B,C\in\mathbb{R}$.
%
Die DGL stellt ein System dar mit dem Eingangssignal $x(t)$
(das ist die Störfunktion / Anregungsfunktion) und dem Ausgangssignal
$y(t)$ (das ist die Lösung der DGL).
%
In der SigSys Literatur ist die Notation von Eingang $x(t)$ und Ausgang $y(t)$
sehr weit verbreitet, aber wir sind gut beraten das im Detail vorher zu checken.
%
Die Koeffizienten $A,B,C$ bestimmen im Wesen, wie das System den Eingang $x(t)$
im Sinne der ersten und zweiten Ableitung \textbf{linear ändert} zum Ausgang $y(t)$.

Die Lösung der DGL ist immer die Superposition von homogener und partikulärer
Lösung
%
\begin{align}
y(t) = y_h(t) + y_p(t)
\end{align}
mit Berücksichtigung der Anfangsbedingungen meist für exakt $t=0$, also gegebenem
$y(0)$, $y'(0)$.
%
Die homogene Lösung findet alle $y_h(t)$ für den Fall $x(t)=0$, daher auch oft
\textbf{Nulllösungen} genannt. Für das spezielle Eingangssignal $x(t)\neq 0$ in
\eq{eq:A7BEE9E24E_DGL} erhalten wir die partikuläre Lösung $y_p(t)$.
%
\subsubsection{Homogene Lösung: Charakteristisches Polynom}
%
Wir haben gelernt (und vielleicht sogar didaktisch fein hergeleitet bekommen,
warum das so sein muss), dass man die homogene Lösung dieser DGL mit dem
Lösungsansatz
\begin{equation}
y_h(t) \propto \e^{\lambda t}\qquad \lambda\in\mathbb{C}
\end{equation}
erhält ($\propto$ meint proportional zu, dann können wir hier erst einmal
eventuelle Vorfaktoren weglassen).
%
Setzen wir diesen Ansatz in \eq{eq:A7BEE9E24E_DGL} ein, führen die Ableitungen
aus, erhalten wir das \textbf{charakteristische Polynom} in der Klammer von
\begin{equation}
(A \lambda^2 + B \lambda + C) \cdot \e^{\lambda t} = 0
\end{equation}
mit den zwei Lösungen ($\e^{\lambda t}$ bekommen wir nicht zu Null)
\begin{align}
\label{eq:A7BEE9E24E_lambdas}
\lambda_{1,2} = \frac{-B \pm \sqrt{B^2-4 A C}}{2 A},
\end{align}
die
\begin{itemize}
  \item a) reell und unterschiedlich sind, wenn $B^2>4 A C$
  \item b) reell und identisch sind, wenn $B^2 = 4 A C$
  \item c) konjugiert-komplex sind, wenn $B^2<4 A C$.
\end{itemize}
%
Für die Fälle a) und c) gilt immer der Ansatz
\begin{mdframed}[backgroundcolor=C3!10]
\begin{align}
\label{eq:A7BEE9E24E_Eigenloesungen}
y_h(t) = c_1 \e^{\lambda_1 t} + c_2 \e^{\lambda_2 t},
\end{align}
\end{mdframed}
%
Wichtig zu erkennen ist, dass wir---je nachdem wie $\lambda_1$ und $\lambda_2$
beschaffen ist (also abhängig von den Koeffizienten $A,B,C$)---verschiedene
Funktionsverläufe als Nulllösungen erhalten.
Es ginge:
\begin{itemize}
  \item harmonische komplexe, sin- oder cos-Schwingung
  \item gedämpfte komplexe, sin- oder cos-Schwingung
  \item aufklingende komplexe, sin- oder cos-Schwingung (in der Praxis zu vermeiden)
  \item abfallender exponentieller Verlauf
  \item ansteigender exponentieller Verlauf (siehe COVID-19, in der Praxis zu vermeiden).
\end{itemize}
Dies sind \textbf{Eigenlösungen} der homogenen DGL und können mit
\eq{eq:A7BEE9E24E_Eigenloesungen} beschrieben werden.
%
Die noch fehlende Eigenlösung für Fall b) ist $y_h(t) = t \e^{\lambda_1 t}$, das
ist der sogenannte Resonanzfall und bedarf deswegen einer eigenständigen Lösung,
den wir hier mal vernachlässigen wollen.

\subsubsection{Homogene Lösung: Eigenwerte und Vektoren der Koeffizientenmatrix}
%
Es ist bemerkenswert, dass die Nulllösungen
auch hervorgehen, wenn man Eigenwerte und Eigenvektoren einer ganz bestimmten Matrix
berechnet. Das ist natürlich kein Zufall und mit der Denke können wir viele
nützliche Tools aus der Lineare Algebra mitverwenden!

Um die Matrix aufzustellen, schreiben wir zunächst die homogene DGL um zu
\begin{align}
\frac{\fsd^2 y(t)}{\fsd t^2} + \frac{B}{A} \frac{\fsd y(t)}{\fsd t} + \frac{C}{A} y(t) = 0,
\end{align}
%
Wir können diese DGL 2. Ordnung auch in zwei getrennten DGL schreiben. Das wird
der Grundstein für die Matrix-Sichtweise.
%
Mit der gängigen abgekürzten Schreibweise für Zeitableitung
\begin{align}
\frac{\fsd y(t)}{\fsd t} = y'(t)
\end{align}
können wir schreiben
\begin{align}
\frac{\fsd^2 y(t)}{\fsd t^2} = \frac{\fsd y'(t)}{\fsd t} = - \frac{B}{A}  \frac{\fsd y(t)}{\fsd t} - \frac{C}{A}  y(t).
\end{align}
%
Diese beiden Gleichungen können wir nun als einfaches Matrix-Gleichungssystem formulieren
\begin{align}
\label{eq:A7BEE9E24E_uAU}
\vec{u}' = \frac{\fsd}{\fsd t}
\begin{bmatrix}
y(t) \\ y'(t)
\end{bmatrix}
=
\begin{bmatrix}
0 & 1 \\ - \frac{C}{A} & - \frac{B}{A}
\end{bmatrix}
\begin{bmatrix}
y(t) \\ y'(t)
\end{bmatrix}
=
\vec{A} \vec{u}.
\end{align}
%
Wir finden in der Matrix $\vec{A}$ alle Koeffizienten $A,B,C$ wieder, daher
ist es sinnvoll diese als Koeffizientenmatrix zu bezeichnen und sich klarzumachen,
dass diese Matrix das System vollständig beschreibt.

Behauptung:
Dieses Gleichungssystem lässt sich mittels Eigenwert $\lambda$ und
Eigenvektor $\vec{x}$ lösen, also
\begin{align}
\vec{A} \vec{x} = \lambda \vec{x},
\end{align}
wobei es kein Zufall ist, dass wir auch hier die Variable $\lambda$, wie schon
beim charakteristischen Polynom verwenden: es sind die gleichen Zahlen.
Wir wissen, dass man die unbekannten Eigenwerte mittels Ansatz
$(\vec{A} - \lambda \vec{I})\vec{x} = \vec{0}$, also singulärer Matrix
$(\vec{A} - \lambda \vec{I})$, also $\mathrm{det}(\vec{A} - \lambda \vec{I})=0$
findet.
%
Daher
\begin{align}
\vec{A} - \lambda \vec{I} =
\begin{bmatrix}
0 & 1 \\ -\frac{C}{A} & -\frac{B}{A}
\end{bmatrix}
-\lambda
\begin{bmatrix}
1 & 0 \\ 0 & 1
\end{bmatrix}
=
\begin{bmatrix}
-\lambda & 1 \\ -\frac{C}{A} & -\frac{B}{A} - \lambda
\end{bmatrix}
\end{align}
und derer Determinante
\begin{align}
-\lambda \cdot (-\frac{B}{A} - \lambda) - (-\frac{C}{A})\cdot 1 = \lambda^2 + \frac{B}{A} \lambda + \frac{C}{A}.
\end{align}
%
$\mathrm{det}(\vec{A} - \lambda \vec{I})=0$ wird also zu
\begin{align}
A \lambda^2 + B \lambda + C = 0
\end{align}
mit den \textbf{zwei Eigenwerten} $\lambda_{1,2}$ aus \eq{eq:A7BEE9E24E_lambdas},
weil identisch mit dem charakteristischen Polynom.
%
Für die beiden zugehörigen Eigenvektoren $\vec{x}_{1}$ und $\vec{x}_{2}$ finden sich wegen
dem geforderten
\begin{align}
(\vec{A} - \lambda_{1,2} \vec{I}) \, \vec{x}_{1,2} = \vec{0}
\end{align}
die Lösungen
\begin{align}
\vec{x}_{1,2} =
\begin{bmatrix}
1 \\ \lambda_{1,2}
\end{bmatrix}.
\end{align}
%
Wir schreiben für $\vec{A} \vec{x} = \lambda \vec{x}$
%
\begin{align}
\vec{A}
\begin{bmatrix}
1 \\ \lambda_{1,2}
\end{bmatrix} = \lambda_{1,2}
\begin{bmatrix}
1 \\ \lambda_{1,2}
\end{bmatrix},
\end{align}
%
mit
\begin{align}
\lambda_{1,2} = \frac{-B \pm \sqrt{B^2-4 A C}}{2 A}.
\end{align}
%
Wichtig: das gerade beschriebene Vorgehen funktioniert nur, wenn
$\lambda_1 \neq \lambda_2$, also tatsächlich zwei unabhängige Eigenvektoren vorliegen.
%
$\lambda_1 = \lambda_2$ ist wieder der Spezialfall der Resonanz.

Unser Ausgangspunkt war $\vec{A}  \vec{u} = \vec{u}'$ in \eq{eq:A7BEE9E24E_uAU}.
Wie verknüpfen wir das mit der gefundenen Eigenwert/-vektor-Darstellung?
%
Kein stringenter Beweis, aber fundamentale Beobachtung, nämlich
$\frac{\fsd \e^{a t}}{\fsd t} = a \e^{a t}$.
%
Es ist nicht verboten zu erweitern
\begin{align}
\vec{A}
\begin{bmatrix}
1 \\ \lambda_{1,2}
\end{bmatrix}
\e^{\lambda_{1,2} t}
= \lambda_{1,2}
\begin{bmatrix}
1 \\ \lambda_{1,2}
\end{bmatrix}
\e^{\lambda_{1,2} t}
\end{align}
und sehen, dass wenn wir
\begin{align}
\vec{u}_{1,2} = \vec{x}_{1,2} \cdot \e^{\lambda_{1,2} t}
\end{align}
definieren, die Gleichung \eq{eq:A7BEE9E24E_uAU} dann aufgeht, weil bei der Ableitung zu
$\vec{u}'_{1,2}$ der benötigte Vorfaktor $\lambda_{1,2}$ entsteht.
%
$\vec{x}_{1,2}$ ist also im Grunde zeitbereinigt, enthält aber als Eigenvektor
alle Informationen, die wir brauchen, um das Systemverhalten zu beschreiben.
%
$\vec{u}_{1,2}$ ist dann die Version, wo die zeitliche Änderung mittels
$\e^{\lambda_{1,2} t}$ berücksichtigt ist.
%
Das gewünschte Gleichungssystem $\vec{u}'=\vec{A}  \vec{u}$ kann nun einfach
gelöst werden:
%
Die Linearkombination aller (hier der beiden) Vektoren $\vec{u}_{1,2}$ ergibt den
Lösungsraum
\begin{mdframed}[backgroundcolor=C3!10]
\begin{align}
\begin{bmatrix}
y(t) \\ y'(t)
\end{bmatrix} =
c_1 \e^{\lambda_1 t}
\begin{bmatrix}
1 \\ \lambda_{1}
\end{bmatrix}
+
c_2 \e^{\lambda_2 t}
\begin{bmatrix}
1 \\ \lambda_{2}
\end{bmatrix},
\end{align}
\end{mdframed}
%
wobei man für die erste Zeile die Lösung $y(t)$ erkennt und für die zweite $y'(t)$
(man erkennt das, aber wir hatten eben $\vec{u}$ in \eq{eq:A7BEE9E24E_uAU}
auch so definiert).
%
Wir haben damit die gleiche Lösung der DGL wie beim Rechnen mit dem
charakteristischen Polynom gefunden.
%
Mit dieser homogenen Lösung können wir dann partikuläre Lösungen mit
Variation der Konstanten berechnen.

Soweit der kurze Abriss mit hoffentlich wiederholendem Charakter.
%
Wir machen weiter mit einem fundamentalen Zusammenhang, der die Grundlage der
SigSys-Theorie für linear Systeme (also lineare DGL mit konstanten Koeffizienten)
ist.
%
Keine Sorge, wenn das folgende ein wenig sperrige Kost ist, es ist mutmaßlich
Neuland, aber der Teaser hier ist wichtig, damit wir ein Gefühl für das Wesen von SigSys
bekommen.

\subsubsection{Spezielle Anregung mit Dirac-Impuls}
Wir betrachten die homogene Lösung unter einer bestimmten Anfangsbedingung
(vgl. \cite[S.97]{Strang2014})
\begin{align}
A y''(t) + B y'(t) + C y(t) = 0,\quad y(0)=0,\quad y'(0)=\frac{1}{A}.
\end{align}
%
Immer noch für $\lambda_1 \neq \lambda_2$ !!! wissen wir von oben
\begin{align}
y_h(t) = &c_1 \e^{\lambda_1 t} + c_2 \e^{\lambda_2 t},\qquad
y_h'(t) = &c_1 \lambda_1 \, \e^{\lambda_1 t} + c_2 \lambda_2 \, \e^{\lambda_2 t}
\end{align}
und mit $y_p=0$ wird $y=y_h+y_p=y_h$.
Die Anfangsbedingungen berücksichtigt
\begin{align}
y(0) = &c_1 + c_2 = 0,\qquad
y'(0) = &c_1 \lambda_1 + c_2 \lambda_2 = \frac{1}{A}
\end{align}
Daraus folgt
\begin{equation}
c_1 = -c_2 = \frac{1}{A} \, \frac{1}{\lambda_1 - \lambda_2}
\end{equation}
und diese Koeffizienten eingesetzt
\begin{align}
\label{sec:A7BEE9E24E_hyh}
y(t) =
\frac{1}{A} \, \frac{\e^{\lambda_1 t} - \e^{\lambda_2 t}}{\lambda_1 - \lambda_2}.
\end{align}
%
Soweit, so gut. Schaut aus wie just another Lösung, aber diese Lösung hat es in sich:
%

Erinnern wir uns an den Dirac-Impuls aus Aufgabe \ref{sec:D410BDAAE0},
den unendlich hohen und schmalen Impuls,
wofür wir Rechenregeln finden mussten, weil er keine klassische Funktion ist.
%
Wenn wir diesen Dirac-Impuls als Anregung / Eingang / Störgröße für unsere DGL
wählen und verschwindende Anfangswerte als linksseitigen! Grenzwert
berücksichtigen (System vor der Anregung bei exakt $t=0$ in Ruhe), also

\begin{mdframed}[backgroundcolor=C3!10]
\begin{align}
A y''(t) + B y'(t) + C y(t) = \delta(t),\qquad y'(0-) = 0,\qquad y(0-) = 0
\end{align}
\end{mdframed}
erhalten wir als Lösung \eq{sec:A7BEE9E24E_hyh}!!
%\begin{align}
%\label{sec:A7BEE9E24E_hyp}
%y(t) =
%\frac{1}{A} \, \frac{\e^{\lambda_1 t} - \e^{\lambda_2 t}}{\lambda_1 - \lambda_2}.
%\end{align}
%
Die Rechnung schauen wir uns zu einem späteren Zeitpunkt an. Es gelingt mit Variation
der Konstanten und Benutzung der Austasteigenschaft des Dirac Impulses,
vgl. \cite[S.133ff]{Strang2014}.
%
Auch das warum schauen wir uns später an. Wir stellen hier zunächst fest, dass:
homogene DGL mit speziellen Anfangsbedingungen zu $t=0$ ist gleich wie
Lösung der Anregung mit Dirac-Impuls $\delta(t)$ zum Zeitpunkt $t=0$ mit
linksseitigen, verschwindenden Anfangsbedingungen.

In SigSys bezeichnen wir diese Lösung als \textbf{Impulsantwort} $h(t)$
(das ist die Antwort der DGL auf den Dirac Impuls mit Gewicht 1, Mathematiker
nennen diese Funktion auch \textbf{Green'sche Funktion})
\begin{mdframed}[backgroundcolor=C3!10]
\begin{align}
h(t) =
\frac{1}{A}\,\frac{\e^{\lambda_1 t} - \e^{\lambda_2 t}}{\lambda_1 - \lambda_2}
\qquad \lambda_1 \neq \lambda_2.
\end{align}
\end{mdframed}

\newpage
Mit dieser können wir die \textbf{partikuläre Lösung jeder beliebigen Anregung},
also Störfunktion $x(t)$ angeben, und zwar mit dem sogenannten
\textbf{Faltungsintegral} beginnend ab unserem hier gewähltem
Zeitpunkt $t=0$
%
\begin{mdframed}[backgroundcolor=C3!10]
\begin{equation}
y_p(t) = \int\limits_{0}^{t} h(t-\tau) x(\tau) \fsd \tau
\end{equation}
\end{mdframed}
%
Falls das System vorher in Ruhe war, ist dies auch die komplette Lösung $y(t)$.
%

Wir finden dieses Faltungsintegral (mit allgemeineren Grenzen) in der
Formelsammlung in dem Abschnitt Faltung, da die Operation allgemeingültig als
\textbf{Faltung} bekannt ist.
%
Für einen Zeitpunkt $t$ werden alle Beiträge der Eingangsgröße $x$ mit der Impulsantwort $h$
gewichtet und \textit{addiert}. Bezüglich eines Zeitpunkts $t$, werden \textit{aktuelle}
Eingangswerte $x(t)$ mit dem \textit{Anfang} der Impulsantwort, \textit{lange}
zurückliegende Eingangswerte mit dem \textit{Ende} der Impulsantwort gewichtet.
%
Wir werden das bald im Detail lernen, es ist ein ganz wesentliches Integral und
eine ganz wesentliche Operation und deswegen \textbf{immer klausurrelevant}.
%
Machen wir uns nochmal klar: in $h(t)$ steckt die Systeminformation in Form
der Eigenwerte $\lambda_{1,2}$, also am Ende des Tages in Form der Koeffizienten $A,B,C$.
Die zeitliche Komponente ist durch $\e^{\lambda_{1,2} t}$ abgebildet.

Der hier nicht näher betrachtete Fall der Resonanz $\lambda_1 = \lambda_2$
hat eine eigene Impulsantwort $h(t) = t \e^{\lambda t}$.
%
Das Faltungsintegral gilt konsistent
für alle Fälle a), b), c). Das Integral weiß ja und interessiert sich auch nicht,
dass wir eine Funktion als Impulsantwort auffassen und wir Fallunterscheidungen
vornehmen müssen, je nachdem wie $A$, $B$ und $C$ beschaffen sind.

Wir fragen uns jetzt zu Recht, ob das Faltungsintegral
uns das Leben so viel einfacher macht, als DGLs zu lösen, so wie wir das in Mathe
gelernt haben.
%
Unbedingt ja! Deswegen gehört die SigSys ja zum Grundausbildungskanon für Ingenieure.
Es erhellt erstens, was bei DGLs eigentlich passiert, die Rechnerei
ist nicht notwendigerweise einfacher geworden, aber wir werden zweitens mit der
\textbf{Laplace Transformation} eine
Methode kennenlernen diese ganze \textbf{DGL-Rechnerei und Faltungsoperation auf
einfache algebraische Funktionen runterzubrechen}. Aus der Faltung von zwei
Zeitfunktionen wird dann die Multiplikation zweier Laplace Transformierter.
%
Die Laplace Transformierte der Impulsantwort enthält viele Informationen auf
sehr übersichtliche Weise und ist perfekt für die tägliche Ingenieurarbeit, sei
es in der Regelungstechnik, Kommunikationstechnik oder Maschinenbau, Mechatronik.


\subsubsection{Aufgabe: Homogene Lösung einer DGL 2. Ordnung mit konstanten
Koeffizienten und Anfangsbedingungen}
\label{sec:A7BEE9E24E_Aufgabe}
\begin{Ziel}
Anhand eines speziellen Beispieles, wollen wir die Impulsantwort berechnen. Auch,
wenn wir unsere Lösung jetzt noch nicht anderweitig verifizieren können,
ist es sinnvoll hier nochmal eine DGL zu lösen und dann gleich einen sehr wichtigen
Spezialfall.
\end{Ziel}
\textbf{Aufgabe} {\tiny A7BEE9E24E}: Berechnen Sie für die DGL 2. Ordnung
\begin{align}
\frac{16}{25} \ddot{y}(t) + \frac{24}{25} \dot{y}(t) + y(t) = 0,
\quad y(0)=0,\quad y'(0)=\frac{25}{16}.
\end{align}
die Lösung $y(t)$ für $t\geq 0$.
\begin{Werkzeug}
Entweder wir benutzen die Sachen aus dem vorangegangen Repetitorium Abschnitt
\ref{sec:A7BEE9E24E} oder
wir befragen unsere Mathe-Vorlesungsunterlagen. Ziel soll zunächst sein, diese
DGL korrekt zu lösen.
\end{Werkzeug}
\begin{Ansatz}
Im github Repository

\url{https://github.com/spatialaudio/signals-and-systems-exercises}

findet sich unter \verb|laplace_transform/solving_2nd_order_ode.tex| eine
ausführlichste Lösung.
Dieses Dokument werden wir zu einem späteren Zeitpunkt, ca. in der Mitte der
UE, sowieso ausführlich behandeln. Es steht als PDF im StudIP.
\end{Ansatz}
%\begin{ExCalc}
%\end{ExCalc}
\begin{Loesung}
Wir sollten zur Lösung
\begin{align}
y(t) = \frac{25}{16} \e^{-\frac{3}{4} t} \sin(t), \qquad t\geq 0
\end{align}
gelangen.
%
Dies ist eine Sinusschwingung die exponentiell gedämpft wird und mit
$\frac{25}{16}$ gewichtet ist.
Wegen des gestellten Problems haben wir die Impulsantwort dieser DGL berechnet.
\end{Loesung}
  % Einführung, FS, FT, Dirac, ODE2nd, h(t)




%
%Laplace Trafo macht jetzt etwas sehr elegantes: anstatt jedesmal das Matrix Problem aufzustellen (DGL n-ter Ordnung ergibt nxn Matrix und n Eigenwerte/Eigenvektoren), hat man das Problem
%mit Rechenregeln direkt in ein algebriasches Problem überführt, und nicht nur das es erlaubt das elebante Handling von Eingangsgrößen und Anfangs/Randbedingungen. Dadurch werden DGL erst bequem oder überhauot lösbar. Die Einführung der Laplace vor ca. 100 Jahren und ihre Formalisierung in den 1960/70er Jahren hat maßgeblich zum Weiterentwiklung analoger Signalverabrietung beigetragen und muss auch heute noch als wichtige Essenz des IngHandwerks verstanden werden.

%\begin{comment}
%------------------------------------------------------------------------------
\newpage
\section{UE 2: Basics: Elementarsignale, Lineare Systeme}
Zielsetzung / Objectives:
Basics Signale / Lineare Systeme


Fahrplan
\begin{itemize}
\item Elementarsignale rect, sprung, exp(jwt), Superposition, Zeit/Amplitudenskalierung, Zeitverschiebung/inversion
\item Systeme Test auf Linearität, i.e. Superposition, Zeit Verschiebung, Amplitudenskalierung
\item einfache aber plakative Beispiele, vlt. Corona Zeitreihe
\item Ausblick: Verknüfung Signal -> System -> Signal, vs. DGL, vs. FR, d.h. Rückgriff auf UE1
\item Message System: DGL ist mühsam, viele lineare Systeme lassen sich deutlich einfacher und elegnater lösen
\item Message Signal: manchmal(sehr oft) ist es vorteilhaft das Signal anders darzustellen (Frequenzanalyse), um
an Informationen ranzukommen, BSP: funktechnik, Video
\end{itemize}


%------------------------------------------------------------------------------
\newpage
\section{UE 3: Faltung zeitkontinuierlich}
Zielsetzung / Objectives: Idee der Faltung, Faltungsintegral allgemein, Link zu System: Faltung mit einer Impulsantwort

Fahrplan
\begin{itemize}
\item Tiefpass erster Ordnung, Sprungantowrt analytisch, Impulsantwort analytisch, Faltung für rect analytisch und grafisch
\item Link Sprung / Impuls
\item DGL lösen vs. Falten
\item Beispiel SOS Hochpass analytisch/grafisch, Interpretation/Erwartungshaltung was sehen wir in Immuls/Sprung, vgl. zu Tiefpass
\item Faltung Eigenfunktion mit Impulsantowt, Teaser zu Amplitude und Phase
\item Beispiel: Bildverabreitunt Glättungsfilter = Tiefpass, Kanten
\item Message: Faltungsintegral fundamental zur beschreibung Signale->LTI Systeme->Signale
\item Message: Faltungsintegral bzw. die Impulsantwort analytisch lösen/beschreiben oft kompliziert bis hin zu unlösbar, daher suchen wir elegante adequate beschreibungen, zB im Freqz
vgl. eine einfache 2nd order H(s) vs. komplizierter Ausdruck für Impz
\end{itemize}


%------------------------------------------------------------------------------
\newpage
\section{UE 4: DGL vs. Impulsantwort vs. Laplace Trafo}
Zielsetzung / Objectives: Sinn der Laplace Trafo, HinTrafo, Link Impulsantowrt H(s), Laplace Ebene

Fahrplan
\begin{itemize}
\item von DGL Tiefpass1st zu h(t) zu H(s) mit Def Laplace Trafo
\item Direkter Link DGL konst Koeff zu H(s)
\item Polstellen in Lapalce sind NST des Char Pol, zusätzlich in Laplace: NST
\item Analytisches Rechnen einfache Laplace Hin Trafos, stückweise stetige Signale
\item Korrespondenzen / Eigenschaften motivieren, Sprung / Impulse Zusammenhang in Laplace, Mod, Delay, was passiert jeweils im PZ
\item Faltung vs. Mult
\item Message: bisher vermeintlich noch nicht so viel gewonnen, außer andere Denke, aber in der übernächsten UE wird das alles sehr sinnvoll anzuwenden sein, aus Faltung Mult machen hat  großen Impact
\end{itemize}


%------------------------------------------------------------------------------
\newpage
\section{UE 5: Laplace Rücktrafo, ROC}
Zielsetzung / Objectives: Damit das Tool Laplace vollständig sinnvoll ist, brauchen wir auch die Rücktrafo,
komplexes Integral :-(, meist nicht rechnen, sondern mit Korrespondeze oder der
partialbruchzerlegung (Spezialfall der allg. Rücktrafo Integralsatz), wir wollen aus einer PZ Verteliung eine Impulsanfort/Springantwort/Signal finden

Fahrplan
\begin{itemize}
\item H(s) tiefpass/hochpass rücktransfomrieren, wir wissen aus UE3 was die Lösung sein muss
\item PZB Beispiele
\item KB / ROC Problematik, bisher kausale Systeme/Signale, aber Mehrdeutigekt, daher BSP links/rechts/beidseitg (diese Reihenfolge, wichtig, weil viel weniger verwirrend, also erst Spezialfall, dann allgemeinfall)
\item Message: wir können nun Laplace Hin/Rück trafo und es wurde bahuptet, dass es DGLs einfacher lösbar macht, das schauen wir uns in der nächsten UE im Detail an
\item Message: wir können mit Laplace Dinge machen die mit FT nicht gehen, KOnverhenz des Integrals erzwingen, damit Sprunghafte Signale erlaubt
\end{itemize}


%------------------------------------------------------------------------------
\newpage
\section{UE 6: Beispiel System 2. Ordung Laplace vs. DGL}
Zielsetzung / Objectives: Anhand eines Tief- oder Hochpasssystems 2. Ordnung soll
einmal alles durchgepsielt werden, um den Sinn/Eleganz der Laplace Trafo zu demonstrieren, Falt zu Mult

Fahrplan
\begin{itemize}
\item DGL homo + verschiedene inhom, so dass man Sprung/Impulsantowrt und Aexp(jwt+phi) errechnet
\item verschiedenene Schwingungszustände durch verschiedene NST des char Polynoms, versch Ansätze für part Ansatz mühsam
\item Laplace H(s) + Anfangs/Randbed, Diskussion PZ
\item Inverse Laplace für Sprung/Impuls/exp-Antowrten
\item Message: man sollte hier nun gesehen haben, dass Laplace in der Tat eleganter ist, zumindest für die hier betrachteten LTI 2nd order Systeme
\item Systemstabilität andeuten
\item Message: Statt nur hin/her zu transfomieren, können wir aber im Laplace Bereich noch mehr Info aus der PZ Ebene rausholen, daszu setzen wir sigma=0 und landen bei
einem Spezialfall der Lapace Trafo, nämlich der Fourier Trafo
\end{itemize}


%------------------------------------------------------------------------------
\newpage
\section{UE 7: Fourier Trafo}
Zielsetzung / Objectives: Trafo für eingeschwungene Zustände, Anaylse Frequenzgang Systeme, FrequenzAnaylse spezieller Signaltypen

Fahrplan
\begin{itemize}
\item Begriff Spektrum
\item Anayltische Beisoiele für Mod, Delay, Soektralanalysue
\item Energie
\item Korrespondenzen
\item Gem/Untersch Fouirer / Laplace
\item Link zur FR, periodische Signale vs. Linienspektrum -> wir kennen jetzt 2 von 4 Trafos
\item Message: Rückgriff aus UE1, jetzt macht die Anwendnung vlt mehr Sinn
\end{itemize}


%------------------------------------------------------------------------------
\newpage
\section{UE 8: Bode Plot/Nyquist Plot}
Zielsetzung / Objectives: Ing Tools um aus Laplace
Fourier Abschätzungen zu Systemeigenschaften zu treffen, früher manuell wichtig, heute
Computer, aber Kopfschätzungen auch heute noch sehr relavent, was macht ein Pol, eine NST

Fahrplan
\begin{itemize}
\item Bode Regeln PZ für Magnitude
\item Bode Regeln PZ für Phase
\item Beispiel Bode für das System aus UE 6
\item Typische 'Kurven'diskussion für System ist also: H(s), h(t), he(t), PZ, KB, Mag/Phase
\item Parallel / Reihenschaltund anhand Bode, Ausblick: Feedback Regler
\item Stabilität durch Pole schieben
\item Message: Bode und Nyquist Plots waren in analog Zeiten DIE Tools, Abschätzung
per Hand auch heute noch wichtig, weil Verständnis für die Dinge
\item Message: wir sind damit durch mit zeitkont Signalen, es lohnt sich für alle bisherig behandelten Trafos (FT, FR, Laplace) exolizit nochmal mit den Eigenschaften z ubeschäftigen -> evtl. Sonderrechenblatt wo nur das behandelt wird
\item Schaubild x->h->y vs. X->H->Y
\end{itemize}


%------------------------------------------------------------------------------
\newpage
\section{UE 9: Abtastung / Rekonstruktion}
Zielsetzung / Objectives: Erklärung Abtastung im Spektralbereich mittels Fourier trafo

Fahrplan
\begin{itemize}
\item Rückgriff Fourier Trafo, Dirac Kamm, Exp/Sin/Cos
\item Rückgriff Spektrum / Signal Vershc / Mod-> das brauchen wir jetz wieder
\item Anwendung: Signal Abtastung und Rekonsturktion
\item Sinc Rect Dualität again
\item Ausblick Signal vs. Spektrum Denke (Beweis Abtasttheorem Kotelnikov, FH Lange)
\item Message: WKS-Abtasttheorem ist ein mögliches Szenario mit Signalannahmen und
idealer (d.h. praktisch nicht realisierbarer) Reko, Praxis: andere Interpolatoren, oder aber andere Sampling Schemes, je nach Signaltyp
\end{itemize}


%------------------------------------------------------------------------------
\newpage
\section{UE 10: Basics Elementarsignale, Lineare Systeme}
Zielsetzung / Objectives: Für Folgen kann man ähnliche
Elementarsignale und Eigenschaften für LTI Systeme aufstellen, Eigensignale/lösungen von DiffGl

Fahrplan
\begin{itemize}
\item Rect/Exp/Sin/Cos, Periodizität, Eigensignale, Stem Plots!
\item Denke zu DiffGlg
\item Zeitverschiebung, Modulation, Zeitstreckung/Stauchung, Ampitudenskal, Zeit/Ampl-Inversion
\item Checks auf System Lineariät
\item Message: es gibt fundamentale Gemeinsamkeiten zwischen zeitkont/diskret, aber
auch ein paar sehr wichtige Unterschiede. Das Wesen wie Tools benutzt werden ist aber komplett gleich
\end{itemize}

%------------------------------------------------------------------------------
\newpage
\section{UE 11: The Big Picture FT / FR / DTFT / DFT}
Zielsetzung / Objectives: Erkennen/Erarbeiten der großen Zusammenhänge,
die DFT ist die FR für diskrete Signale, die DTFT ist die FT für diskrete Signale.
Vom Wesen, das was UE1 für zeitkont. Signale war, Zusammenhang für diskrete Siganle aufzeigen
Didaktisch: funktioniert bestens als WDH/Neueinstieg und als Teaser für das Kommende

Fahrplan
\begin{itemize}
\item Gemeinsamkeiten/Unterschiede Signale und deren Spektren (Periodizität, Linienspektrum)
\item Sinc / Rect Dualitäten, Dirac aus Sinc
\item Link als Ausblick: Laplace und z-Trafo als Systembeschreibungs Trafos und Frequenzgang in der FT und DTFT
\item Ausblick: es gibt auch wieder Faltung vs. Mult, Pole/Nullstellen, aus DGL wird DifffGlg
\item Message: jede Signaltyp hat seine Spektraldarstellung in exp()-Eigenfunktionen
für viele Signale ist diese Zerlegung sinnvoll, für andere aber gar nicht (KOnvergenz Fourierreihe)
wichtig ist, die richtige Trafo für das richtige Problem und souveränder Wechsel zwischen den Trafos, ist
Ing Handwerk
\end{itemize}

%------------------------------------------------------------------------------
\newpage
\section{UE 12: Zeitkdiskrete Faltung}
Zielsetzung / Objectives: Fundamentals der Faltung für Zeitdiskrete Signale

Fahrplan
\begin{itemize}
\item Faltungssume
\item Grafisch vs. Analytisch
\item Periodische / Linear Faltung
\item Korr vs. Conv summe (Machine Learning Link)
\item Message: Faltung diskret vtl. sogar einfacher als Integral, weil Summe
zugänglicher, Wesen erfassen: lange zurücliegende Samples werden mit sehr späten Samples der IR verrechnet, Ursache Wirkung plausibilisieren (Dirac Kamm)
\end{itemize}

%------------------------------------------------------------------------------
\newpage
\section{UE 13: z-Trafo}
Zielsetzung / Objectives: Sinn und Eigenschaften der zTrafo erlernen, weil wir ja eigentlich zu Analyse von Systemen nicht immer falten wollen, also Falt vs. Mult auch bei Zeitdiskret erarbeiten, statt Laplace machen wir das mit zT

Fahrplan
\begin{itemize}
\item aus DiffGL wird ein z Polynom, Analogie zu Laplace
\item z-Plane statt s-Plane
\item Korrespondenzen einfache rekursive / nicht rek Systeme, exp(jWt)
\item Beispiele Hin / Rücktrafo / ROC, Pole / NST
\item Signal X(s), Y(s), System H(s)
\item Schaubild x->h->y vs. X->H->Y
\item Message: mit Schaubild udn Vorwissen Laplace sollte wir das Tool zT zu schätzen wissen
\item TBD: DiffGL -> z Vorwärts,Rückwärts, Centered Int, Reihenentwicklung
\end{itemize}


%------------------------------------------------------------------------------
\newpage
\section{UE 14: DTFT / DFT Im Detail}
Zielsetzung / Objectives: vertiefendes Kennenlernen der beiden Neuen Fourier Trafos

Fahrplan
\begin{itemize}
\item einfache Hin/Rück, Korrespondenzen, va. Mod, Mult, Dly
\item DTFT Mag/Phase eines FIR, Was machen Nullstellen im Spektrum, Ausblick: FensterDesign als Spezialfall von Design endlicher Folgen-> Codes mit bestimmtem Spektrum
\item schnelle Faltung über zeropadded DFT
\item Message: DTFT und DFT sind Pendants von FT und FR, spezielles Problem in best Domäne sehr elegant mit richtiger Trafo
\end{itemize}



%------------------------------------------------------------------------------
\newpage
\section{UE 15: z-Trafo großes Beispiel SOS}
Zielsetzung / Objectives: komplette analytische 'Kurvendiskussion' für rek und nichtrek System

Fahrplan
\begin{itemize}
\item rek System 2nd order, zB wieder der Tiefpass, impz, step, H(z), PZ, Mag/Phase usw.
\item nicht rek System, Spezialfall Sym IR und Linearphasigkeit, impz, step, H(z), PZ, Mag/Phase usw.
\item Min/allpass Phase
\item Inversion
\item Ausblick: Bode Approx hier etwas unangenehmer, aber im Prinzip gleiche
\item Wichtig:!!! Code für IIR / FIR Filter, FIR ist Conv, IIR ist DiffGl
\item Message: einmal Systemanalyse z-Trafo/DTFT für IIR/FIR durchgespielt, in der Praxis werden die Systeme nur komplizierte, nicht aber die Tools
\end{itemize}
%\end{comment}

\cite{*}
\newpage
\bibliography{literatur}
\end{document}
